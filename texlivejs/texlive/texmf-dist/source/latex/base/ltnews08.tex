% \iffalse meta-comment
%
% Copyright 1993 1994 1995 1996 1997 1998 1999 2000 2001 2002 2003 2004 2005 2006 2007 2008 2009
% The LaTeX3 Project and any individual authors listed elsewhere
% in this file. 
% 
% This file is part of the LaTeX base system.
% -------------------------------------------
% 
% It may be distributed and/or modified under the
% conditions of the LaTeX Project Public License, either version 1.3c
% of this license or (at your option) any later version.
% The latest version of this license is in
%    http://www.latex-project.org/lppl.txt
% and version 1.3c or later is part of all distributions of LaTeX 
% version 2005/12/01 or later.
% 
% This file has the LPPL maintenance status "maintained".
% 
% The list of all files belonging to the LaTeX base distribution is
% given in the file `manifest.txt'. See also `legal.txt' for additional
% information.
% 
% The list of derived (unpacked) files belonging to the distribution 
% and covered by LPPL is defined by the unpacking scripts (with 
% extension .ins) which are part of the distribution.
% 
% \fi
% Filename: ltnews08.tex

% This is issue 8 of LaTeX News.

\documentclass
%    [lw35fonts]
   {ltnews}

% \usepackage[T1]{fontenc}

\publicationmonth{December}
\publicationyear{1997}
\publicationissue{8}

\begin{document}

\maketitle

\section{New supported font encodings}

Two new font encodings are supported as options to the \textsf{fontenc}
package:
\begin{description}
\item [OT4]
This is a seven-bit encoding designed for Polish. The \LaTeX\ support
was developed by Mariusz Olko.
\item [TS1] This is the `Text Companion Encoding'; it contains symbols
designed to be used in text, as opposed to mathematical formulas, and
some accents designed for uppercase letters.  It is currently
supported by the `tc' fonts, which match the T1 encoded `ec' text
fonts.  A subset of the glyphs in this encoding is supported by
virtual fonts distributed with the PostScript font metrics on the
\textsc{ctan} archives. (This is the `8c' encoding in Karl Berry's
fontname scheme.)  The \textsf{textcomp} package provides access to
this encoding but here is a warning to current users of that package:
some of the internal names for the characters have changed.
\end{description}


\section{New input encodings}

These additions to the \textsf{inputenc} package are
\texttt{decmulti} (the DEC Multinational
Character Set,  contributed by M.~Y.~Chartoire) 
and \texttt{cp1250} (an MS-Windows encoding for Central and Eastern
Europe, contributed by Marcin Woli\'nski).  There is also a
\texttt{cp1252} encoding that is identical to \texttt{ansinew}.


\section{Tools}

The \textsf{calc} package (used in many examples in \emph{The \LaTeX\
Companion}) has been contributed to this distribution by Kresten Krab
Thorup and Frank Jensen. This is essentially the same as the version
that has been available from the \textsc{ctan} archives for some time,
with one minor change: to use \LaTeX-style error messages.  It enables
the use of arithmetic expressions within arguments to standard
\LaTeX{} commands where a length or a counter value is required.  For
example:
\begin{verbatim}
  \setcounter {page} { \value{page} * 2 + 1 }
  \parbox { 3in - ( 2mm + \textwidth / 9 ) }
\end{verbatim}

There have also been some improvements to several other packages in
this collection.  In particular, \textsf{bm} now works correctly with
constructions such as \verb|\bm{f'}| involving \texttt{'} or other
characters which use \TeX's special ``\verb|\mathcode"8000|''
feature.  Also, \textsf{multicol} sets the length \verb|\columnwidth|
to an appropriate value; this enables it to work with classes that
support two-column setting, e.g.,~the AMS classes.


\section{Graphics}

The special \verb|oztex.def| driver file has been removed, and Oz\TeX\
support has been merged with dvips, following advice from
Andrew Trevorrow about Oz\TeX~3.x.

The \textsf{keyval} package has had some internal improvements: to
use \LaTeX\ format error messages; and to avoid `\verb|#| doubling'.
This latter change means that the \verb|command| key for the
\textsf{graphicx} version of \verb|\includegraphics| should now be used
with one \verb|#| rather than two. For example, \verb|command = `gunzip #1|.
Fortunately this key is almost never used in practice, so few if any
documents should be affected by this change.


\section{\LaTeX3 experimental programming conventions}

As announced at the \TeX\ Users Group meeting  (Summer 1997), a group of
highly experimental packages will soon be released to allow experienced\\
\TeX\ programmers to experiment with, and comment on, a proposed set
of syntax conventions and basic data-types that might form
the basis for programming large scale projects in \TeX.
They will be located in\\
this CTAN directory:
\begin{verbatim}  
  CTAN:macros/latex/packages/expl3
\end{verbatim}
The documentation of this material is as follows: individual package
files provide outline, draft documentation; there is an article that
gives an overview of the syntax and related concepts; there is a
\texttt{readme.txt} file containing a brief description of the
collection.

All aspects of these packages are liable, indeed likely, to change.
They should not be used at this stage for anything
that requires a stable system.  However, we do encourage people to
experiment with these packages, and to send comments on them to the
\texttt{LaTeX-L} mailing list.
To subscribe to this list, mail to:
\begin{verbatim}
  listserv@urz.uni-heidelberg.de
\end{verbatim}
the following one line message:
\begin{verbatim}
  subscribe LATEX-L <<first-name>> <<second-name>>
\end{verbatim}

% Revert to this if gets too full.
% See \texttt{modguide.tex} for
% information on how to subscribe to \texttt{LaTeX-L}.

\end{document}

