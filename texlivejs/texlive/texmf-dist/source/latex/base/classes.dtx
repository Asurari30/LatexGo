% \iffalse meta-comment
%
% Copyright 1993-2016
% The LaTeX3 Project and any individual authors listed elsewhere
% in this file.
%
% This file is part of the LaTeX base system.
% -------------------------------------------
%
% It may be distributed and/or modified under the
% conditions of the LaTeX Project Public License, either version 1.3c
% of this license or (at your option) any later version.
% The latest version of this license is in
%    http://www.latex-project.org/lppl.txt
% and version 1.3c or later is part of all distributions of LaTeX
% version 2005/12/01 or later.
%
% This file has the LPPL maintenance status "maintained".
%
% The list of all files belonging to the LaTeX base distribution is
% given in the file `manifest.txt'. See also `legal.txt' for additional
% information.
%
% The list of derived (unpacked) files belonging to the distribution
% and covered by LPPL is defined by the unpacking scripts (with
% extension .ins) which are part of the distribution.
%
% \fi
%
% \iffalse
% \section{Identification}
%
%    These document classes can only be used with \LaTeXe, so we make
%    sure that an appropriate message is displayed when another \TeX{}
%    format is used.
% \changes{v1.3p}{1995/11/30}{Added date of \LaTeX\ format to argument
%    of \cs{NeedsTeXFormat}}
%    \begin{macrocode}
%<article|report|book>\NeedsTeXFormat{LaTeX2e}[1995/12/01]
%    \end{macrocode}
%
%    Announce the Class name and its version:
%    \begin{macrocode}
%<article>\ProvidesClass{article}
%<report>\ProvidesClass{report}
%<book>\ProvidesClass{book}
%<10pt&!bk>\ProvidesFile{size10.clo}
%<11pt&!bk>\ProvidesFile{size11.clo}
%<12pt&!bk>\ProvidesFile{size12.clo}
%<10pt&bk>\ProvidesFile{bk10.clo}
%<11pt&bk>\ProvidesFile{bk11.clo}
%<12pt&bk>\ProvidesFile{bk12.clo}
%<*driver>
\ProvidesFile{classes.drv}
%</driver>
              [2014/09/29 v1.4h
%<article|report|book> Standard LaTeX document class]
%<10pt|11pt|12pt>      Standard LaTeX file (size option)]
%    \end{macrocode}
%
% \section{A driver for this document}
%
% The next bit of code contains the documentation driver file for
% \TeX{}, i.e., the file that will produce the documentation you are
% currently reading. It will be extracted from this file by the
% {\sc docstrip} program.
%
% \changes{1.0f}{1993/12/07}{Use class ltxdoc document class}
% \changes{1.0r}{1994/02/28}{Moved driver code in order not to need a
%    separate driver}
%    \begin{macrocode}
%<*driver>
]
\documentclass{ltxdoc}
%    \end{macrocode}
%
%    We don't want everything to appear in the index
%    \begin{macrocode}
\DoNotIndex{\',\.,\@M,\@@input,\@Alph,\@alph,\@addtoreset,\@arabic}
\DoNotIndex{\@badmath,\@centercr,\@cite}
\DoNotIndex{\@dotsep,\@empty,\@float,\@gobble,\@gobbletwo,\@ignoretrue}
\DoNotIndex{\@input,\@ixpt,\@m,\@minus,\@mkboth}
\DoNotIndex{\@ne,\@nil,\@nomath,\@plus,\roman,\@set@topoint}
\DoNotIndex{\@tempboxa,\@tempcnta,\@tempdima,\@tempdimb}
\DoNotIndex{\@tempswafalse,\@tempswatrue,\@viipt,\@viiipt,\@vipt}
\DoNotIndex{\@vpt,\@warning,\@xiipt,\@xipt,\@xivpt,\@xpt,\@xviipt}
\DoNotIndex{\@xxpt,\@xxvpt,\\,\ ,\addpenalty,\addtolength,\addvspace}
\DoNotIndex{\advance,\ast,\begin,\begingroup,\bfseries,\bgroup,\box}
\DoNotIndex{\bullet}
\DoNotIndex{\cdot,\cite,\CodelineIndex,\cr,\day,\DeclareOption}
\DoNotIndex{\def,\DisableCrossrefs,\divide,\DocInput,\documentclass}
\DoNotIndex{\DoNotIndex,\egroup,\ifdim,\else,\fi,\em,\endtrivlist}
\DoNotIndex{\EnableCrossrefs,\end,\end@dblfloat,\end@float,\endgroup}
\DoNotIndex{\endlist,\everycr,\everypar,\ExecuteOptions,\expandafter}
\DoNotIndex{\fbox}
\DoNotIndex{\filedate,\filename,\fileversion,\fontsize,\framebox,\gdef}
\DoNotIndex{\global,\halign,\hangindent,\hbox,\hfil,\hfill,\hrule}
\DoNotIndex{\hsize,\hskip,\hspace,\hss,\if@tempswa,\ifcase,\or,\fi,\fi}
\DoNotIndex{\ifhmode,\ifvmode,\ifnum,\iftrue,\ifx,\fi,\fi,\fi,\fi,\fi}
\DoNotIndex{\input}
\DoNotIndex{\jobname,\kern,\leavevmode,\let,\leftmark}
\DoNotIndex{\list,\llap,\long,\m@ne,\m@th,\mark,\markboth,\markright}
\DoNotIndex{\month,\newcommand,\newcounter,\newenvironment}
\DoNotIndex{\NeedsTeXFormat,\newdimen}
\DoNotIndex{\newlength,\newpage,\nobreak,\noindent,\null,\number}
\DoNotIndex{\numberline,\OldMakeindex,\OnlyDescription,\p@}
\DoNotIndex{\pagestyle,\par,\paragraph,\paragraphmark,\parfillskip}
\DoNotIndex{\penalty,\PrintChanges,\PrintIndex,\ProcessOptions}
\DoNotIndex{\protect,\ProvidesClass,\raggedbottom,\raggedright}
\DoNotIndex{\refstepcounter,\relax,\renewcommand}
\DoNotIndex{\rightmargin,\rightmark,\rightskip,\rlap,\rmfamily}
\DoNotIndex{\secdef,\selectfont,\setbox,\setcounter,\setlength}
\DoNotIndex{\settowidth,\sfcode,\skip,\sloppy,\slshape,\space}
\DoNotIndex{\symbol,\the,\trivlist,\typeout,\tw@,\undefined,\uppercase}
\DoNotIndex{\usecounter,\usefont,\usepackage,\vfil,\vfill,\viiipt}
\DoNotIndex{\viipt,\vipt,\vskip,\vspace}
\DoNotIndex{\wd,\xiipt,\year,\z@}
%    \end{macrocode}
%    We do want an index, using linenumbers
%    \begin{macrocode}
\EnableCrossrefs
\CodelineIndex
%    \end{macrocode}
%    We use so many \file{docstrip} modules that we set the
%    \texttt{StandardModuleDepth} counter to 1.
%    \begin{macrocode}
\setcounter{StandardModuleDepth}{1}
%    \end{macrocode}
%    The following command retrieves the date and version information
%    from the file.
%    \begin{macrocode}
\GetFileInfo{classes.drv}
%    \end{macrocode}
%    Some commonly used abbreviations
% \changes{v1.2w}{1994/12/01}{Use \cs{newcommand*}}
%    \begin{macrocode}
\newcommand*{\Lopt}[1]{\textsf {#1}}
\newcommand*{\file}[1]{\texttt {#1}}
\newcommand*{\Lcount}[1]{\textsl {\small#1}}
\newcommand*{\pstyle}[1]{\textsl {#1}}
%    \end{macrocode}
%    We also want the full details.
%    \begin{macrocode}
\begin{document}
\DocInput{classes.dtx}
\PrintIndex
% ^^A\PrintChanges
\end{document}
%</driver>
%    \end{macrocode}
%
% \fi
%
% \changes{v1.0d}{1993/11/30}{remove \cs{@in}, made option makeindex
%    a synonym for option makeidx}
% \changes{v1.0d}{1993/11/30}{removed \cs{@minus}, \cs{@plus},
%    \cs{@settopoint}, \cs{@setfontsize}; they are now in the
%    kernel}
% \changes{v1.0d}{1993/11/30}{Added use of \cs{NeedsTeXFormat}}
% \changes{v1.0d}{1993/11/30}{Replaced \cs{bf} with \cs{bfseries};
%    \cs{rm} with \cs{rmfamily}}
% \changes{v1.0d}{1993/11/30}{Made equation and eqnarray environments
%    in the fleqn option up to date with latex.dtx}
% \changes{v1.0f}{1993/12/08}{Made all lines shorter than 72 characters}
% \changes{v1.0g}{1993/12/08}{Made change in eqnarray for the fleqn
%    option, as suggested by Rainer.}
% \changes{v1.0h}{1993/12/18}{Made the definitions of the font- and
%    size-changing commands use \cs{renew} rather than \cs{new}.
%    Defined the float parameters with \cs{renewcommand} rather than
%    \cs{newcommand}.  Corrected some typos in the fleqn option.
%    Replaced two occurrences of -\cs{@secpenalty} by
%    \cs{@secpenalty}.  ASAJ.}
% \changes{v1.0j}{1993/12/20}{Added \cs{ProvidesFile} to size files}
% \changes{v1.0j}{1993/12/10}{Use \cs{cmd} in change entries}
% \changes{v1.0k}{1994/01/09}{Removed some typos/bugs}
% \changes{v1.0l}{1994/01/11}{add the extension to the names of the
%     files}
% \changes{v1.0l}{1994/01/10}{Changed version numbering; moved leqno
%    and fleqn options to an external file.}
% \changes{v1.0n}{1994/01/19}{Removed code for makeidx option and made
%    it a separate package; removed use of \cs{setlength} from list
%    parameters.}
% \changes{v1.0o}{1994/01/31}{Small documentation changes}
% \changes{v1.0q}{1994/02/16}{Small documentation changes}
% \changes{v1.1a}{1994/03/12}{Removed \cs{typeout} messages}
% \changes{v1.1f}{1994/04/15}{Inserted forgotten line break}
% \changes{v1.2a}{1994/03/17}{Added openright option. (LL)}
% \changes{v1.2b}{1994/03/17}{Added the \ldots{}matter commands. (LL)}
% \changes{v1.2c}{1994/03/17}{Fixed page numbering in titlepage
%    env. (LL)}
% \changes{v1.2d}{1994/04/11}{Checked the file for long lines and
%    wrapped them when necessary; made a slight implementation
%    modification to the openright and openany options.}
% \changes{v1.2i}{1994/04/28}{Use LaTeX instead of LaTeX2e in messages}
% \changes{v1.2j}{1994/05/01}{Removed the use of \cs{fileversion}
%    c.s.}
% \changes{v1.2l}{1994/05/11}{changed some \cs{changes} entries}
% \changes{v1.2m}{1994/05/12}{Forgot a few entries}
% \changes{v1.2o}{1994/05/24}{Changed file information}
% \changes{v1.2p}{1994/05/27}{Moved identification and driver to the
%    front of the file}
% \changes{v1.2t}{1994/06/22}{Refrased a few sentences to prevent
%    overfull hboxes}
% \changes{v1.2v}{1994/12/01}{Made the oneside option work for the
%    book class}
% \changes{v1.2w}{1994/12/01}{Use \cs{newcommand*} for commands with
%    arguments}
% \changes{v1.2z}{1995/05/16}{Always use \cs{cs} in \cs{changes}
%    entries}
% \changes{v1.3a}{1995/05/17}{Replaced all \cs{hbox to} by \cs{hb@xt@}}
% \changes{v1.3d}{1995/06/05}{Replaced all \cs{uppercase} by
%    \cs{MakeUppercase}}
% \changes{v1.3l}{1995/10/20}{Disabled in compatibility mode all
%    options that are new in \LaTeXe.}
% \changes{v1.3v}{1997/06/16}{Documentation fixes.}
%
%
% \title{Standard Document Classes for \LaTeX{} version 2e\thanks{This
%    file has version number \fileversion, last revised \filedate.}}
%
% \author{%
% Copyright (C) 1992 by Leslie Lamport \and
% Copyright (C) 1994-97 by Frank Mittelbach \and Johannes Braams
% }
% \date{\filedate}
% \MaintainedByLaTeXTeam{latex}
% \maketitle
% \tableofcontents
%
% \StopEventually{}    ^^A
%
% \section{The {\sc docstrip} modules}
%
% The following modules are used in the implementation to direct
% {\sc docstrip} in generating the external files:
% \begin{center}
% \begin{tabular}{ll}
%   article & produce the documentclass article\\
%   report  & produce the documentclass report\\
%   size10  & produce the class option for 10pt\\
%   size11  & produce the class option for 11pt\\
%   size12  & produce the class option for 12pt\\
%   book    & produce the documentclass book\\
%   bk10    & produce the book class option for 10pt\\
%   bk11    & produce the book class option for 11pt\\
%   bk12    & produce the book class option for 12pt\\
%   driver  & produce a documentation driver file \\
% \end{tabular}
% \end{center}
%
% \section{Initial Code}
%
%    In this part we define a few commands that are used later on.
%
% \begin{macro}{\@ptsize}
%    This control sequence is used to store the second digit of the
%    pointsize we are typesetting in. So, normally, it's value is one
%    of 0, 1 or 2.
%    \begin{macrocode}
%<*article|report|book>
\newcommand\@ptsize{}
%    \end{macrocode}
% \end{macro}
%
% \begin{macro}{\if@restonecol}
%    When the document has to be printed in two columns, we sometimes
%    have to temporarily switch to one column. This switch is used to
%    remember to switch back.
%    \begin{macrocode}
\newif\if@restonecol
%    \end{macrocode}
% \end{macro}
%
% \begin{macro}{\if@titlepage}
%    A switch to indicate if a titlepage has to be produced.  For the
%    article document class the default is not to make a separate
%    titlepage.
%    \begin{macrocode}
\newif\if@titlepage
%<article>\@titlepagefalse
%<!article>\@titlepagetrue
%    \end{macrocode}
% \end{macro}
%
% \begin{macro}{\if@openright}
%    A switch to indicate if chapters must start on a right-hand page.
%    The default for the report class is no; for the book class it's
%    yes.
%    \begin{macrocode}
%<!article>\newif\if@openright
%    \end{macrocode}
% \end{macro}
%
% \changes{v1.3k}{1995/08/27}{Macro \cs{if@openbib} removed}
%
% \begin{macro}{\if@mainmatter}
% \changes{v1.2v}{1994/12/01}{Moved the allocation of
%    \cs{if@mainmatter} here}
%
%    The switch |\if@mainmatter|, only available in the document class
%    book, indicates whether we are processing the main material in
%    the book.
%    \begin{macrocode}
%<book>\newif\if@mainmatter \@mainmattertrue
%    \end{macrocode}
%  \end{macro}
%
% \section{Declaration of Options}
%
%
% \subsection{Setting Paper Sizes}
%
%    The variables |\paperwidth| and |\paperheight| should reflect the
%    physical paper size after trimming. For desk printer output this
%    is usually the real paper size since there is no post-processing.
%    Classes for real book production will probably add other paper
%    sizes and additionally the production of crop marks for trimming.
%    In compatibility mode, these (and some of the subsequent) options
%    are disabled, as they were not present in \LaTeX 2.09.
% \changes{v1.0g}{1993/12/09}{Removed typo, A4 is not 279 mm high}
%    \begin{macrocode}
\if@compatibility\else
\DeclareOption{a4paper}
   {\setlength\paperheight {297mm}%
    \setlength\paperwidth  {210mm}}
\DeclareOption{a5paper}
   {\setlength\paperheight {210mm}%
    \setlength\paperwidth  {148mm}}
\DeclareOption{b5paper}
   {\setlength\paperheight {250mm}%
    \setlength\paperwidth  {176mm}}
\DeclareOption{letterpaper}
   {\setlength\paperheight {11in}%
    \setlength\paperwidth  {8.5in}}
\DeclareOption{legalpaper}
   {\setlength\paperheight {14in}%
    \setlength\paperwidth  {8.5in}}
\DeclareOption{executivepaper}
   {\setlength\paperheight {10.5in}%
    \setlength\paperwidth  {7.25in}}
%    \end{macrocode}
%
%    The option \Lopt{landscape} switches the values of |\paperheight|
%    and |\paperwidth|, assuming the dimensions were given for portrait
%    paper.
%    \begin{macrocode}
\DeclareOption{landscape}
   {\setlength\@tempdima   {\paperheight}%
    \setlength\paperheight {\paperwidth}%
    \setlength\paperwidth  {\@tempdima}}
\fi
%    \end{macrocode}
%
% \subsection{Choosing the type size}
%
%    The type size options are handled by defining |\@ptsize| to contain
%    the last digit of the size in question and branching on |\ifcase|
%    statements. This is done for historical reasons to stay compatible
%    with other packages that use the |\@ptsize| variable to select
%    special actions. It makes the declarations of size options less
%    than 10pt difficult, although one can probably use \texttt{9}
%    and \texttt{8} assuming that a class wont define both
%    \Lopt{8pt} and \Lopt{18pt} options.
%
%    \begin{macrocode}
\if@compatibility
  \renewcommand\@ptsize{0}
\else
\DeclareOption{10pt}{\renewcommand\@ptsize{0}}
\fi
\DeclareOption{11pt}{\renewcommand\@ptsize{1}}
\DeclareOption{12pt}{\renewcommand\@ptsize{2}}
%    \end{macrocode}
%
%
%  \subsection{Two-side or one-side printing}
%
%    For two-sided printing we use the switch |\if@twoside|. In
%    addition we have to set the |\if@mparswitch| to get any margin
%    paragraphs into the outside margin.
%    \begin{macrocode}
\if@compatibility\else
\DeclareOption{oneside}{\@twosidefalse \@mparswitchfalse}
\fi
\DeclareOption{twoside}{\@twosidetrue  \@mparswitchtrue}
%    \end{macrocode}
%
%
%  \subsection{Draft option}
%
%    If the user requests \Lopt{draft} we show any overfull boxes.
%    We could probably add some more interesting stuff to this option.
%    \begin{macrocode}
\DeclareOption{draft}{\setlength\overfullrule{5pt}}
\if@compatibility\else
\DeclareOption{final}{\setlength\overfullrule{0pt}}
\fi
%    \end{macrocode}
%
%  \subsection{Titlepage option}
%    An article usually has no separate titlepage, but the user can
%    request one.
%    \begin{macrocode}
\DeclareOption{titlepage}{\@titlepagetrue}
\if@compatibility\else
\DeclareOption{notitlepage}{\@titlepagefalse}
\fi
%    \end{macrocode}
%
%  \subsection{openright option}
%    This option determines whether or not a chapter must start on
%    a right-hand page
%    request one.
%    \begin{macrocode}
%<!article>\if@compatibility
%<book>\@openrighttrue
%<!article>\else
%<!article>\DeclareOption{openright}{\@openrighttrue}
%<!article>\DeclareOption{openany}{\@openrightfalse}
%<!article>\fi
%    \end{macrocode}
%
%  \subsection{Twocolumn printing}
%
%    Two-column and one-column printing is again realized via a switch.
%    \begin{macrocode}
\if@compatibility\else
\DeclareOption{onecolumn}{\@twocolumnfalse}
\fi
\DeclareOption{twocolumn}{\@twocolumntrue}
%    \end{macrocode}
%
%  \subsection{Equation numbering on the left}
%
%    The option \Lopt{leqno} can be used to get the equation numbers
%    on the left side of the equation. It loads code which is generated
%    automatically from the kernel files when the format is built.
%    If the equation number does get a special formatting then instead
%    of using the kernel file the class would need to provide the code
%    explicitly.
%    \begin{macrocode}
\DeclareOption{leqno}{%%
%% This is file `leqno.sty',
%% generated with the docstrip utility.
%%
%% The original source files were:
%%
%% latex209.dtx  (with options: `leqno')
%% 
%% This is a generated file.
%% 
%% The source is maintained by the LaTeX Project team and bug
%% reports for it can be opened at https://latex-project.org/bugs.html
%% (but please observe conditions on bug reports sent to that address!)
%% 
%% 
%% Copyright 1993-2017
%% The LaTeX3 Project and any individual authors listed elsewhere
%% in this file.
%% 
%% This file was generated from file(s) of the LaTeX base system.
%% --------------------------------------------------------------
%% 
%% It may be distributed and/or modified under the
%% conditions of the LaTeX Project Public License, either version 1.3c
%% of this license or (at your option) any later version.
%% The latest version of this license is in
%%    https://www.latex-project.org/lppl.txt
%% and version 1.3c or later is part of all distributions of LaTeX
%% version 2005/12/01 or later.
%% 
%% This file has the LPPL maintenance status "maintained".
%% 
%% This file may only be distributed together with a copy of the LaTeX
%% base system. You may however distribute the LaTeX base system without
%% such generated files.
%% 
%% The list of all files belonging to the LaTeX base distribution is
%% given in the file `manifest.txt'. See also `legal.txt' for additional
%% information.
%% 
%% The list of derived (unpacked) files belonging to the distribution
%% and covered by LPPL is defined by the unpacking scripts (with
%% extension .ins) which are part of the distribution.
\@obsoletefile{leqno.clo}{leqno.sty}
%%
%% This is file `leqno.sty',
%% generated with the docstrip utility.
%%
%% The original source files were:
%%
%% latex209.dtx  (with options: `leqno')
%% 
%% This is a generated file.
%% 
%% The source is maintained by the LaTeX Project team and bug
%% reports for it can be opened at https://latex-project.org/bugs.html
%% (but please observe conditions on bug reports sent to that address!)
%% 
%% 
%% Copyright 1993-2017
%% The LaTeX3 Project and any individual authors listed elsewhere
%% in this file.
%% 
%% This file was generated from file(s) of the LaTeX base system.
%% --------------------------------------------------------------
%% 
%% It may be distributed and/or modified under the
%% conditions of the LaTeX Project Public License, either version 1.3c
%% of this license or (at your option) any later version.
%% The latest version of this license is in
%%    https://www.latex-project.org/lppl.txt
%% and version 1.3c or later is part of all distributions of LaTeX
%% version 2005/12/01 or later.
%% 
%% This file has the LPPL maintenance status "maintained".
%% 
%% This file may only be distributed together with a copy of the LaTeX
%% base system. You may however distribute the LaTeX base system without
%% such generated files.
%% 
%% The list of all files belonging to the LaTeX base distribution is
%% given in the file `manifest.txt'. See also `legal.txt' for additional
%% information.
%% 
%% The list of derived (unpacked) files belonging to the distribution
%% and covered by LPPL is defined by the unpacking scripts (with
%% extension .ins) which are part of the distribution.
\@obsoletefile{leqno.clo}{leqno.sty}
%%
%% This is file `leqno.sty',
%% generated with the docstrip utility.
%%
%% The original source files were:
%%
%% latex209.dtx  (with options: `leqno')
%% 
%% This is a generated file.
%% 
%% The source is maintained by the LaTeX Project team and bug
%% reports for it can be opened at https://latex-project.org/bugs.html
%% (but please observe conditions on bug reports sent to that address!)
%% 
%% 
%% Copyright 1993-2017
%% The LaTeX3 Project and any individual authors listed elsewhere
%% in this file.
%% 
%% This file was generated from file(s) of the LaTeX base system.
%% --------------------------------------------------------------
%% 
%% It may be distributed and/or modified under the
%% conditions of the LaTeX Project Public License, either version 1.3c
%% of this license or (at your option) any later version.
%% The latest version of this license is in
%%    https://www.latex-project.org/lppl.txt
%% and version 1.3c or later is part of all distributions of LaTeX
%% version 2005/12/01 or later.
%% 
%% This file has the LPPL maintenance status "maintained".
%% 
%% This file may only be distributed together with a copy of the LaTeX
%% base system. You may however distribute the LaTeX base system without
%% such generated files.
%% 
%% The list of all files belonging to the LaTeX base distribution is
%% given in the file `manifest.txt'. See also `legal.txt' for additional
%% information.
%% 
%% The list of derived (unpacked) files belonging to the distribution
%% and covered by LPPL is defined by the unpacking scripts (with
%% extension .ins) which are part of the distribution.
\@obsoletefile{leqno.clo}{leqno.sty}
\input{leqno.clo}
\endinput
%%
%% End of file `leqno.sty'.

\endinput
%%
%% End of file `leqno.sty'.

\endinput
%%
%% End of file `leqno.sty'.
}
%    \end{macrocode}
%
%  \subsection{Flush left displays}
%
%    The option \Lopt{fleqn} redefines the displayed math environments
%    in such a way that they come out flush left, with an indentation
%    of |\mathindent| from the prevailing left margin. It loads
%    code which is generated
%    automatically from the kernel files when the format is built.
% \changes{v1.0h}{1993/12/18}{Corrected some typos.  ASAJ.}
%    \begin{macrocode}
\DeclareOption{fleqn}{%%
%% This is file `fleqn.sty',
%% generated with the docstrip utility.
%%
%% The original source files were:
%%
%% latex209.dtx  (with options: `fleqn')
%% 
%% This is a generated file.
%% 
%% The source is maintained by the LaTeX Project team and bug
%% reports for it can be opened at https://latex-project.org/bugs.html
%% (but please observe conditions on bug reports sent to that address!)
%% 
%% 
%% Copyright 1993-2017
%% The LaTeX3 Project and any individual authors listed elsewhere
%% in this file.
%% 
%% This file was generated from file(s) of the LaTeX base system.
%% --------------------------------------------------------------
%% 
%% It may be distributed and/or modified under the
%% conditions of the LaTeX Project Public License, either version 1.3c
%% of this license or (at your option) any later version.
%% The latest version of this license is in
%%    https://www.latex-project.org/lppl.txt
%% and version 1.3c or later is part of all distributions of LaTeX
%% version 2005/12/01 or later.
%% 
%% This file has the LPPL maintenance status "maintained".
%% 
%% This file may only be distributed together with a copy of the LaTeX
%% base system. You may however distribute the LaTeX base system without
%% such generated files.
%% 
%% The list of all files belonging to the LaTeX base distribution is
%% given in the file `manifest.txt'. See also `legal.txt' for additional
%% information.
%% 
%% The list of derived (unpacked) files belonging to the distribution
%% and covered by LPPL is defined by the unpacking scripts (with
%% extension .ins) which are part of the distribution.
\@obsoletefile{fleqn.clo}{fleqn.sty}
%%
%% This is file `fleqn.sty',
%% generated with the docstrip utility.
%%
%% The original source files were:
%%
%% latex209.dtx  (with options: `fleqn')
%% 
%% This is a generated file.
%% 
%% The source is maintained by the LaTeX Project team and bug
%% reports for it can be opened at https://latex-project.org/bugs.html
%% (but please observe conditions on bug reports sent to that address!)
%% 
%% 
%% Copyright 1993-2017
%% The LaTeX3 Project and any individual authors listed elsewhere
%% in this file.
%% 
%% This file was generated from file(s) of the LaTeX base system.
%% --------------------------------------------------------------
%% 
%% It may be distributed and/or modified under the
%% conditions of the LaTeX Project Public License, either version 1.3c
%% of this license or (at your option) any later version.
%% The latest version of this license is in
%%    https://www.latex-project.org/lppl.txt
%% and version 1.3c or later is part of all distributions of LaTeX
%% version 2005/12/01 or later.
%% 
%% This file has the LPPL maintenance status "maintained".
%% 
%% This file may only be distributed together with a copy of the LaTeX
%% base system. You may however distribute the LaTeX base system without
%% such generated files.
%% 
%% The list of all files belonging to the LaTeX base distribution is
%% given in the file `manifest.txt'. See also `legal.txt' for additional
%% information.
%% 
%% The list of derived (unpacked) files belonging to the distribution
%% and covered by LPPL is defined by the unpacking scripts (with
%% extension .ins) which are part of the distribution.
\@obsoletefile{fleqn.clo}{fleqn.sty}
%%
%% This is file `fleqn.sty',
%% generated with the docstrip utility.
%%
%% The original source files were:
%%
%% latex209.dtx  (with options: `fleqn')
%% 
%% This is a generated file.
%% 
%% The source is maintained by the LaTeX Project team and bug
%% reports for it can be opened at https://latex-project.org/bugs.html
%% (but please observe conditions on bug reports sent to that address!)
%% 
%% 
%% Copyright 1993-2017
%% The LaTeX3 Project and any individual authors listed elsewhere
%% in this file.
%% 
%% This file was generated from file(s) of the LaTeX base system.
%% --------------------------------------------------------------
%% 
%% It may be distributed and/or modified under the
%% conditions of the LaTeX Project Public License, either version 1.3c
%% of this license or (at your option) any later version.
%% The latest version of this license is in
%%    https://www.latex-project.org/lppl.txt
%% and version 1.3c or later is part of all distributions of LaTeX
%% version 2005/12/01 or later.
%% 
%% This file has the LPPL maintenance status "maintained".
%% 
%% This file may only be distributed together with a copy of the LaTeX
%% base system. You may however distribute the LaTeX base system without
%% such generated files.
%% 
%% The list of all files belonging to the LaTeX base distribution is
%% given in the file `manifest.txt'. See also `legal.txt' for additional
%% information.
%% 
%% The list of derived (unpacked) files belonging to the distribution
%% and covered by LPPL is defined by the unpacking scripts (with
%% extension .ins) which are part of the distribution.
\@obsoletefile{fleqn.clo}{fleqn.sty}
\input{fleqn.clo}
\endinput
%%
%% End of file `fleqn.sty'.

\endinput
%%
%% End of file `fleqn.sty'.

\endinput
%%
%% End of file `fleqn.sty'.
}
%    \end{macrocode}
%
% \subsection{Open bibliography}
%
%    The option \Lopt{openbib} produces the ``open'' bibliography
%    style, in which each block starts on a new line, and succeeding
%    lines in a block are indented by |\bibindent|.
% \changes{v1.3k}{1995/08/27}{openbib option reimplemented}
%    \begin{macrocode}
\DeclareOption{openbib}{%
%    \end{macrocode}
%    First some hook into the bibliography environment is filled.
%    \begin{macrocode}
  \AtEndOfPackage{%
   \renewcommand\@openbib@code{%
      \advance\leftmargin\bibindent
      \itemindent -\bibindent
      \listparindent \itemindent
      \parsep \z@
      }%
%    \end{macrocode}
%    In addition the definition of |\newblock| is overwritten.
%    \begin{macrocode}
   \renewcommand\newblock{\par}}%
}
%    \end{macrocode}
%
%
% \section{Executing Options}
%
%    Here we execute the default options to initialize certain
%    variables. Note that the document class `book' always uses two
%    sided printing.
%    \begin{macrocode}
%<*article>
\ExecuteOptions{letterpaper,10pt,oneside,onecolumn,final}
%</article>
%<*report>
\ExecuteOptions{letterpaper,10pt,oneside,onecolumn,final,openany}
%</report>
%<*book>
\ExecuteOptions{letterpaper,10pt,twoside,onecolumn,final,openright}
%</book>
%    \end{macrocode}
%
%    The |\ProcessOptions| command causes the execution of the code
%    for every option \Lopt{FOO}
%    which is declared and for which the user typed
%    the \Lopt{FOO} option in his
%    |\documentclass| command.  For every option \Lopt{BAR} he typed,
%    which is not declared, the option is assumed to be a global option.
%    All options will be passed as document options to any
%    |\usepackage| command in the document preamble.
%    \begin{macrocode}
\ProcessOptions
%    \end{macrocode}
%    Now that all the options have been executed we can load the
%    chosen class option file that contains all size dependent code.
%    \begin{macrocode}
%<!book>\input{size1\@ptsize.clo}
%<book>\input{bk1\@ptsize.clo}
%</article|report|book>
%    \end{macrocode}
%
%  \section{Loading Packages}
%
%  The standard class files do not load additional packages.
%
%
% \section{Document Layout}
% \label{sec:classes:maincode}
%
%  In this section we are finally dealing with the nasty typographical
%  details.
%
% \subsection{Fonts}
%
%    \LaTeX\ offers the user commands to change the size of the font,
%    relative to the `main' size. Each relative size changing command
%    |\size| executes the command
%    |\@setfontsize||\size|\meta{font-size}\meta{baselineskip} where:
%
%    \begin{description}
%    \item[\meta{font-size}] The absolute size of the font to use from
%        now on.
%
%    \item[\meta{baselineskip}] The normal value of |\baselineskip|
%        for the size of the font selected. (The actual value will be
%        |\baselinestretch| * \meta{baselineskip}.)
%    \end{description}
%
%    A number of commands, defined in the \LaTeX{} kernel, shorten the
%    following  definitions and are used throughout. They are:
% \begin{center}
% \begin{tabular}{ll@{\qquad}ll@{\qquad}ll}
%  \verb=\@vpt= & 5 & \verb=\@vipt= & 6 & \verb=\@viipt= & 7 \\
%  \verb=\@viiipt= & 8 & \verb=\@ixpt= & 9 & \verb=\@xpt= & 10 \\
%  \verb=\@xipt= & 10.95 & \verb=\@xiipt= & 12 & \verb=\@xivpt= & 14.4\\
%  ...
%  \end{tabular}
%  \end{center}
%
% \begin{macro}{\normalsize}
% \begin{macro}{\@normalsize}
% \changes{v1.0o}{1994/01/31}{\cs{@normalsize} now defined in the
%    kernel}
%
%    The user level command for the main size is |\normalsize|.
%    Internally \LaTeX{} uses |\@normalsize| when it refers to the
%    main size. |\@normalsize| will be defined to work like
%    |\normalsize| if the latter is redefined from its default
%    definition (that just issues an error message). Otherwise
%    |\@normalsize| simply selects a 10pt/12pt size.
%
%    The |\normalsize| macro also sets new values for\\
%    |\abovedisplayskip|, |\abovedisplayshortskip| and
%    |\belowdisplayshortskip|.
%
% \changes{v1.0e}{1993/12/07}{\cs{normalsize} doesn't exist, so use
%    \cs{newcommand}}
% \changes{v1.0h}{1993/12/18}{\cs{normalsize} is now defined in the
%    kernel, so use \cs{renewcommand}.  ASAJ.}
%    \begin{macrocode}
%<*10pt|11pt|12pt>
\renewcommand\normalsize{%
%<*10pt>
   \@setfontsize\normalsize\@xpt\@xiipt
   \abovedisplayskip 10\p@ \@plus2\p@ \@minus5\p@
   \abovedisplayshortskip \z@ \@plus3\p@
   \belowdisplayshortskip 6\p@ \@plus3\p@ \@minus3\p@
%</10pt>
%<*11pt>
   \@setfontsize\normalsize\@xipt{13.6}%
   \abovedisplayskip 11\p@ \@plus3\p@ \@minus6\p@
   \abovedisplayshortskip \z@ \@plus3\p@
   \belowdisplayshortskip 6.5\p@ \@plus3.5\p@ \@minus3\p@
%</11pt>
%<*12pt>
   \@setfontsize\normalsize\@xiipt{14.5}%
   \abovedisplayskip 12\p@ \@plus3\p@ \@minus7\p@
   \abovedisplayshortskip \z@ \@plus3\p@
   \belowdisplayshortskip 6.5\p@ \@plus3.5\p@ \@minus3\p@
%</12pt>
%    \end{macrocode}
%    The |\belowdisplayskip| is always equal to the
%    |\abovedisplayskip|. The parameters of the first level list are
%    always given by |\@listI|.
%    \begin{macrocode}
   \belowdisplayskip \abovedisplayskip
   \let\@listi\@listI}
%    \end{macrocode}
%
%    We initially choose the normalsize font.
%    \begin{macrocode}
\normalsize
%    \end{macrocode}
% \end{macro}
% \end{macro}
%
% \begin{macro}{\small}
%    This is similar to |\normalsize|.
% \changes{v1.0h}{1993/12/18}{\cs{small} is now defined in the kernel,
%    so use \cs{renewcommand}.  ASAJ.}
% \changes{v1.2e}{1994/04/14}{\cs{small} is no longer defined in the
%    kernel; use \cs{newcommand}}
%    \begin{macrocode}
\newcommand\small{%
%<*10pt>
   \@setfontsize\small\@ixpt{11}%
   \abovedisplayskip 8.5\p@ \@plus3\p@ \@minus4\p@
   \abovedisplayshortskip \z@ \@plus2\p@
   \belowdisplayshortskip 4\p@ \@plus2\p@ \@minus2\p@
   \def\@listi{\leftmargin\leftmargini
               \topsep 4\p@ \@plus2\p@ \@minus2\p@
               \parsep 2\p@ \@plus\p@ \@minus\p@
               \itemsep \parsep}%
%</10pt>
%<*11pt>
   \@setfontsize\small\@xpt\@xiipt
   \abovedisplayskip 10\p@ \@plus2\p@ \@minus5\p@
   \abovedisplayshortskip \z@ \@plus3\p@
   \belowdisplayshortskip 6\p@ \@plus3\p@ \@minus3\p@
   \def\@listi{\leftmargin\leftmargini
               \topsep 6\p@ \@plus2\p@ \@minus2\p@
               \parsep 3\p@ \@plus2\p@ \@minus\p@
               \itemsep \parsep}%
%</11pt>
%<*12pt>
   \@setfontsize\small\@xipt{13.6}%
   \abovedisplayskip 11\p@ \@plus3\p@ \@minus6\p@
   \abovedisplayshortskip \z@ \@plus3\p@
   \belowdisplayshortskip 6.5\p@ \@plus3.5\p@ \@minus3\p@
   \def\@listi{\leftmargin\leftmargini
               \topsep 9\p@ \@plus3\p@ \@minus5\p@
               \parsep 4.5\p@ \@plus2\p@ \@minus\p@
               \itemsep \parsep}%
%</12pt>
   \belowdisplayskip \abovedisplayskip
}
%    \end{macrocode}
% \end{macro}
%
% \begin{macro}{\footnotesize}
%    This is similar to |\normalsize|.
% \changes{v1.0h}{1993/12/18}{\cs{footnotesize} is now defined in the
%    kernel, so use \cs{renewcommand}.  ASAJ.}
% \changes{v1.2e}{1994/04/14}{use \cs{newcommand} again}
%    \begin{macrocode}
\newcommand\footnotesize{%
%<*10pt>
   \@setfontsize\footnotesize\@viiipt{9.5}%
   \abovedisplayskip 6\p@ \@plus2\p@ \@minus4\p@
   \abovedisplayshortskip \z@ \@plus\p@
   \belowdisplayshortskip 3\p@ \@plus\p@ \@minus2\p@
   \def\@listi{\leftmargin\leftmargini
               \topsep 3\p@ \@plus\p@ \@minus\p@
               \parsep 2\p@ \@plus\p@ \@minus\p@
               \itemsep \parsep}%
%</10pt>
%<*11pt>
   \@setfontsize\footnotesize\@ixpt{11}%
   \abovedisplayskip 8\p@ \@plus2\p@ \@minus4\p@
   \abovedisplayshortskip \z@ \@plus\p@
   \belowdisplayshortskip 4\p@ \@plus2\p@ \@minus2\p@
   \def\@listi{\leftmargin\leftmargini
               \topsep 4\p@ \@plus2\p@ \@minus2\p@
               \parsep 2\p@ \@plus\p@ \@minus\p@
               \itemsep \parsep}%
%</11pt>
%<*12pt>
   \@setfontsize\footnotesize\@xpt\@xiipt
   \abovedisplayskip 10\p@ \@plus2\p@ \@minus5\p@
   \abovedisplayshortskip \z@ \@plus3\p@
   \belowdisplayshortskip 6\p@ \@plus3\p@ \@minus3\p@
   \def\@listi{\leftmargin\leftmargini
               \topsep 6\p@ \@plus2\p@ \@minus2\p@
               \parsep 3\p@ \@plus2\p@ \@minus\p@
               \itemsep \parsep}%
%</12pt>
   \belowdisplayskip \abovedisplayskip
}
%</10pt|11pt|12pt>
%    \end{macrocode}
% \end{macro}
%
% \begin{macro}{\scriptsize}
% \begin{macro}{\tiny}
% \begin{macro}{\large}
% \begin{macro}{\Large}
% \begin{macro}{\LARGE}
% \begin{macro}{\huge}
% \begin{macro}{\Huge}
%    These are all much simpler than the previous macros, they just
%    select a new fontsize, but leave the parameters for displays and
%    lists alone.
% \changes{v1.0h}{1993/12/18}{These are now defined in the kernel,
%    so use \cs{renewcommand}.  ASAJ.}
% \changes{v1.2e}{1994/04/14}{use \cs{newcommand} again}
%    \begin{macrocode}
%<*10pt>
\newcommand\scriptsize{\@setfontsize\scriptsize\@viipt\@viiipt}
\newcommand\tiny{\@setfontsize\tiny\@vpt\@vipt}
\newcommand\large{\@setfontsize\large\@xiipt{14}}
\newcommand\Large{\@setfontsize\Large\@xivpt{18}}
\newcommand\LARGE{\@setfontsize\LARGE\@xviipt{22}}
\newcommand\huge{\@setfontsize\huge\@xxpt{25}}
\newcommand\Huge{\@setfontsize\Huge\@xxvpt{30}}
%</10pt>
%<*11pt>
\newcommand\scriptsize{\@setfontsize\scriptsize\@viiipt{9.5}}
\newcommand\tiny{\@setfontsize\tiny\@vipt\@viipt}
\newcommand\large{\@setfontsize\large\@xiipt{14}}
\newcommand\Large{\@setfontsize\Large\@xivpt{18}}
\newcommand\LARGE{\@setfontsize\LARGE\@xviipt{22}}
\newcommand\huge{\@setfontsize\huge\@xxpt{25}}
\newcommand\Huge{\@setfontsize\Huge\@xxvpt{30}}
%</11pt>
%<*12pt>
\newcommand\scriptsize{\@setfontsize\scriptsize\@viiipt{9.5}}
\newcommand\tiny{\@setfontsize\tiny\@vipt\@viipt}
\newcommand\large{\@setfontsize\large\@xivpt{18}}
\newcommand\Large{\@setfontsize\Large\@xviipt{22}}
\newcommand\LARGE{\@setfontsize\LARGE\@xxpt{25}}
\newcommand\huge{\@setfontsize\huge\@xxvpt{30}}
\let\Huge=\huge
%</12pt>
%    \end{macrocode}
% \end{macro}
% \end{macro}
% \end{macro}
% \end{macro}
% \end{macro}
% \end{macro}
% \end{macro}
%
%
% \subsection{Paragraphing}
%
% \begin{macro}{\lineskip}
% \begin{macro}{\normallineskip}
%    These parameters control \TeX's behaviour when two lines tend to
%    come too close together.
%    \begin{macrocode}
%<*article|report|book>
\setlength\lineskip{1\p@}
\setlength\normallineskip{1\p@}
%    \end{macrocode}
% \end{macro}
% \end{macro}
%
% \begin{macro}{\baselinestretch}
%    This is used as a multiplier for |\baselineskip|. The default is
%    to \emph{not} stretch the baselines. Note that if this command
%    doesn't resolve to ``empty'' any \texttt{plus} or \texttt{minus}
%    part in the specification of |\baselineskip| is ignored.
%    \begin{macrocode}
\renewcommand\baselinestretch{}
%    \end{macrocode}
% \end{macro}
%
% \begin{macro}{\parskip}
% \begin{macro}{\parindent}
%    |\parskip| gives extra vertical space between paragraphs and
%    |\parindent| is the width of the paragraph indentation. The value
%    of |\parindent| depends on whether we are in two column mode.
% \changes{v1.0m}{1994/01/12}{\cs{parindent} should be different,
%    depending on the pointsize}
%    \begin{macrocode}
\setlength\parskip{0\p@ \@plus \p@}
%</article|report|book>
%<*10pt|11pt|12pt>
\if@twocolumn
  \setlength\parindent{1em}
\else
%<10pt>  \setlength\parindent{15\p@}
%<11pt>  \setlength\parindent{17\p@}
%<12pt>  \setlength\parindent{1.5em}
\fi
%</10pt|11pt|12pt>
%    \end{macrocode}
% \end{macro}
% \end{macro}
%
%  \begin{macro}{\smallskipamount}
%  \begin{macro}{\medskipamount}
%  \begin{macro}{\bigskipamount}
%    The values for these three parameters are set in the \LaTeX\
%    kernel. They should perhaps vary, according to the size option
%    specified. But as they have always had the same value regardless
%    of the size option we do not change them to stay compatible with
%    both \LaTeX~2.09 and older releases of \LaTeXe.
% \changes{v1.3n}{1995/10/29}{Added setting the values of
%    \cs{...skipamount}}
%    \begin{macrocode}
%<*10pt|11pt|12pt>
\setlength\smallskipamount{3\p@ \@plus 1\p@ \@minus 1\p@}
\setlength\medskipamount{6\p@ \@plus 2\p@ \@minus 2\p@}
\setlength\bigskipamount{12\p@ \@plus 4\p@ \@minus 4\p@}
%</10pt|11pt|12pt>
%    \end{macrocode}
%  \end{macro}
%  \end{macro}
%  \end{macro}
%
% \begin{macro}{\@lowpenalty}
% \begin{macro}{\@medpenalty}
% \begin{macro}{\@highpenalty}%
%    The commands |\nopagebreak| and |\nolinebreak| put in penalties
%    to discourage these breaks at the point they are put in.
%    They use |\@lowpenalty|, |\@medpenalty| or |\@highpenalty|,
%    dependent on their argument.
%    \begin{macrocode}
%<*article|report|book>
\@lowpenalty   51
\@medpenalty  151
\@highpenalty 301
%    \end{macrocode}
% \end{macro}
% \end{macro}
% \end{macro}
%
% \begin{macro}{\clubpenalty}
% \begin{macro}{\widowpenalty}
%    These penalties are use to discourage club and widow lines.
%    Because we use their default values we only show them here,
%    commented out.
%    \begin{macrocode}
% \clubpenalty  150
% \widowpenalty 150
%    \end{macrocode}
% \end{macro}
% \end{macro}
%
% \begin{macro}{\displaywidowpenalty}
% \begin{macro}{\predisplaypenalty}
% \begin{macro}{\postdisplaypenalty}
%    Discourage (but not so much) widows in front of a math display
%    and forbid breaking directly in front of a display. Allow break
%    after a display without a penalty. Again the default values are
%    used, therefore we only show them here.
%    \begin{macrocode}
% \displaywidowpenalty 50
% \predisplaypenalty   10000
% \postdisplaypenalty  0
%    \end{macrocode}
% \end{macro}
% \end{macro}
% \end{macro}
%
% \begin{macro}{\interlinepenalty}
%    Allow the breaking of a page in the middle of a paragraph.
%    \begin{macrocode}
% \interlinepenalty 0
%    \end{macrocode}
% \end{macro}
%
%
% \begin{macro}{\brokenpenalty}
%    We allow the breaking of a page after a hyphenated line.
% \changes{v1.1a}{1994/03/12}{Show correct default which is 100}
%    \begin{macrocode}
% \brokenpenalty 100
%</article|report|book>
%    \end{macrocode}
% \end{macro}
%
%
% \subsection{Page Layout}
%
%    All margin dimensions are measured from a point one inch from the
%    top and lefthand side of the page.
%
% \subsubsection{Vertical spacing}
%
% \begin{macro}{\headheight}
% \begin{macro}{\headsep}
% \begin{macro}{\topskip}
%    The |\headheight| is the height of the box that will contain the
%    running head. The |\headsep| is the distance between the bottom
%    of the running head and the top of the text. The |\topskip| is
%    the |\baselineskip| for the first line on a page; \LaTeX's output
%    routine will not work properly if it has the value 0pt, so do not
%    do that!
%    \begin{macrocode}
%<*10pt|11pt|12pt>
\setlength\headheight{12\p@}
%<!bk>\setlength\headsep   {25\p@}
%<10pt&bk>\setlength\headsep   {.25in}
%<11pt&bk>\setlength\headsep   {.275in}
%<12pt&bk>\setlength\headsep   {.275in}
%<10pt>\setlength\topskip   {10\p@}
%<11pt>\setlength\topskip   {11\p@}
%<12pt>\setlength\topskip   {12\p@}
%    \end{macrocode}
% \end{macro}
% \end{macro}
% \end{macro}
%
% \begin{macro}{\footskip}
%    The distance from the baseline of the box which contains the
%    running footer to the baseline of last line of text is controlled
%    by the |\footskip|.
%    \begin{macrocode}
%<!bk>\setlength\footskip{30\p@}
%<10pt&bk>\setlength\footskip{.35in}
%<11pt&bk>\setlength\footskip{.38in}
%<12pt&bk>\setlength\footskip{30\p@}
%    \end{macrocode}
% \end{macro}
%
% \begin{macro}{\maxdepth}
% \changes{v1.2k}{1994/05/06}{Added setting of \cs{maxdepth} and
%    \cs{@maxdepth}}
% \changes{v1.3j}{1995/08/16}{Take setting of
%    \cs{@maxdepth} out again}
%    The \TeX\ primitive register |\maxdepth| has a function that is
%    similar to that of |\topskip|. The register |\@maxdepth| should
%    always contain a copy of |\maxdepth|. This is achieved by setting
%    it internally at |\begin{document}|. In both plain \TeX\ and
%    \LaTeX~2.09 |\maxdepth| had a fixed value of \texttt{4pt}; in
%    native \LaTeX2e\ mode we let the value depend on the typesize. We
%    set it so that |\maxdepth| $+$ |\topskip| $=$ typesize $\times
%    1.5$. As it happens, in these classes |\topskip| is equal to the
%    typesize, therefore we set |\maxdepth| to half the value of
%    |\topskip|.
%    \begin{macrocode}
\if@compatibility \setlength\maxdepth{4\p@} \else
\setlength\maxdepth{.5\topskip} \fi
%    \end{macrocode}
% \end{macro}
%
% \subsubsection{The dimension of text}
%
% \begin{macro}{\textwidth}
%    When we are in compatibility mode we have to make sure that the
%    dimensions of the printed area are not different from what the
%    user was used to see.
%
%    \begin{macrocode}
\if@compatibility
  \if@twocolumn
    \setlength\textwidth{410\p@}
  \else
%<10pt&!bk>    \setlength\textwidth{345\p@}
%<11pt&!bk>    \setlength\textwidth{360\p@}
%<12pt&!bk>    \setlength\textwidth{390\p@}
%<10pt&bk>    \setlength\textwidth{4.5in}
%<11pt&bk>    \setlength\textwidth{5in}
%<12pt&bk>    \setlength\textwidth{5in}
  \fi
%    \end{macrocode}
%    When we are not in compatibility mode we can set some of the
%    dimensions differently, taking into account the paper size for
%    instance.
%    \begin{macrocode}
\else
%    \end{macrocode}
%    First, we calculate the maximum |\textwidth|, which we will allow
%    on the selected paper and store it in |\@tempdima|. Then we store
%    the length of a line with approximately 60--70 characters in
%    |\@tempdimb|. The values given are more or less suitable when
%    Computer Modern fonts are used.
% \changes{v1.1a}{1994/03/12}{Have old values for width in native mode}
%    \begin{macrocode}
  \setlength\@tempdima{\paperwidth}
  \addtolength\@tempdima{-2in}
%<10pt>  \setlength\@tempdimb{345\p@}
%<11pt>  \setlength\@tempdimb{360\p@}
%<12pt>  \setlength\@tempdimb{390\p@}
%    \end{macrocode}
%
%    Now we can set the |\textwidth|, depending on whether we will be
%    setting one or two columns.
%
%    In two column mode each \emph{column} shouldn't be wider than
%    |\@tempdimb| (which could happen on \textsc{a3} paper for
%    instance).
%    \begin{macrocode}
  \if@twocolumn
    \ifdim\@tempdima>2\@tempdimb\relax
      \setlength\textwidth{2\@tempdimb}
    \else
      \setlength\textwidth{\@tempdima}
    \fi
%    \end{macrocode}
%
%    In one column mode the text should not be wider than the minimum
%    of the paperwidth (minus 2 inches for the margins) and the
%    maximum length of a line as defined by the number of characters.
%    \begin{macrocode}
  \else
    \ifdim\@tempdima>\@tempdimb\relax
      \setlength\textwidth{\@tempdimb}
    \else
      \setlength\textwidth{\@tempdima}
    \fi
  \fi
\fi
%    \end{macrocode}
%
%    Here we modify the width of the text a little to be a whole
%    number of points.
%    \begin{macrocode}
\if@compatibility\else
  \@settopoint\textwidth
\fi
%    \end{macrocode}
% \end{macro}
%
% \begin{macro}{\textheight}
%    Now that we have computed the width of the text, we have to take
%    care of the height. The |\textheight| is the height of text
%    (including footnotes and figures, excluding running head and
%    foot).
%
%    First make sure that the compatibility mode gets the same
%    dimensions as we had with \LaTeX2.09. The number of lines was
%    calculated as the floor of the old |\textheight| minus
%    |\topskip|, divided by |\baselineskip| for |\normalsize|. The
%    old value of |\textheight| was 528pt.
%
%    \begin{macrocode}
\if@compatibility
%<10pt&!bk>  \setlength\textheight{43\baselineskip}
%<10pt&bk>  \setlength\textheight{41\baselineskip}
%<11pt>  \setlength\textheight{38\baselineskip}
%<12pt>  \setlength\textheight{36\baselineskip}
%    \end{macrocode}
%
%    Again we compute this, depending on the papersize and depending
%    on the baselineskip that is used, in order to have a whole number
%    of lines on the page.
%    \begin{macrocode}
\else
  \setlength\@tempdima{\paperheight}
%    \end{macrocode}
%
%    We leave at least a 1 inch margin on the top and the bottom of
%    the page.
%    \begin{macrocode}
  \addtolength\@tempdima{-2in}
%    \end{macrocode}
%
%    We also have to leave room for the running headers and footers.
%    \begin{macrocode}
  \addtolength\@tempdima{-1.5in}
%    \end{macrocode}
%
%    Then we divide the result by the current |\baselineskip| and
%    store this in the count register |\@tempcnta|, which then
%    contains the number of lines that fit on this page.
%    \begin{macrocode}
  \divide\@tempdima\baselineskip
  \@tempcnta=\@tempdima
%    \end{macrocode}
%
%    From this we can calculate the height of the text.
%    \begin{macrocode}
  \setlength\textheight{\@tempcnta\baselineskip}
\fi
%    \end{macrocode}
%
%    The first line on the page has a height of |\topskip|.
%    \begin{macrocode}
\addtolength\textheight{\topskip}
%    \end{macrocode}
% \end{macro}
%
%
%
% \subsubsection{Margins}
%
%    Most of the values of these parameters are now calculated, based
%    on the papersize in use. In the calculations the |\marginparsep|
%    needs to be taken into account so we give it its value first.
%
% \begin{macro}{\marginparsep}
% \begin{macro}{\marginparpush}
%    The horizontal space between the main text and marginal notes is
%    determined by |\marginparsep|, the minimum vertical separation
%    between two marginal notes is controlled by |\marginparpush|.
%    \begin{macrocode}
\if@twocolumn
 \setlength\marginparsep {10\p@}
\else
%<10pt&!bk>  \setlength\marginparsep{11\p@}
%<11pt&!bk>  \setlength\marginparsep{10\p@}
%<12pt&!bk>  \setlength\marginparsep{10\p@}
%<bk>  \setlength\marginparsep{7\p@}
\fi
%<10pt|11pt>\setlength\marginparpush{5\p@}
%<12pt>\setlength\marginparpush{7\p@}
%    \end{macrocode}
% \end{macro}
% \end{macro}
%
%    Now we can give the values for the other margin parameters. For
%    native \LaTeXe, these are calculated.
% \begin{macro}{\oddsidemargin}
% \begin{macro}{\evensidemargin}
% \begin{macro}{\marginparwidth}
%    First we give the values for the compatibility mode.
%
%    Values for two-sided printing:
%    \begin{macrocode}
\if@compatibility
%<*bk>
%<10pt>   \setlength\oddsidemargin   {.5in}
%<11pt>   \setlength\oddsidemargin   {.25in}
%<12pt>   \setlength\oddsidemargin   {.25in}
%<10pt>   \setlength\evensidemargin  {1.5in}
%<11pt>   \setlength\evensidemargin  {1.25in}
%<12pt>   \setlength\evensidemargin  {1.25in}
%<10pt>   \setlength\marginparwidth {.75in}
%<11pt>   \setlength\marginparwidth {1in}
%<12pt>   \setlength\marginparwidth {1in}
%</bk>
%<*!bk>
  \if@twoside
%<10pt>     \setlength\oddsidemargin   {44\p@}
%<11pt>     \setlength\oddsidemargin   {36\p@}
%<12pt>     \setlength\oddsidemargin   {21\p@}
%<10pt>     \setlength\evensidemargin  {82\p@}
%<11pt>     \setlength\evensidemargin  {74\p@}
%<12pt>     \setlength\evensidemargin  {59\p@}
%<10pt>     \setlength\marginparwidth {107\p@}
%<11pt>     \setlength\marginparwidth {100\p@}
%<12pt>     \setlength\marginparwidth {85\p@}
%    \end{macrocode}
%    Values for one-sided printing:
%    \begin{macrocode}
  \else
%<10pt>     \setlength\oddsidemargin   {63\p@}
%<11pt>     \setlength\oddsidemargin   {54\p@}
%<12pt>     \setlength\oddsidemargin   {39.5\p@}
%<10pt>     \setlength\evensidemargin  {63\p@}
%<11pt>     \setlength\evensidemargin  {54\p@}
%<12pt>     \setlength\evensidemargin  {39.5\p@}
%<10pt>     \setlength\marginparwidth  {90\p@}
%<11pt>     \setlength\marginparwidth  {83\p@}
%<12pt>     \setlength\marginparwidth  {68\p@}
  \fi
%</!bk>
%    \end{macrocode}
%    And values for two column mode:
%    \begin{macrocode}
  \if@twocolumn
     \setlength\oddsidemargin  {30\p@}
     \setlength\evensidemargin {30\p@}
     \setlength\marginparwidth {48\p@}
  \fi
%    \end{macrocode}
%
%    When we are not in compatibility mode we can take the dimensions
%    of the selected paper into account.
%
%    The values for |\oddsidemargin| and |\marginparwidth| will be set
%    depending on the status of the |\if@twoside|.
%
%    If |@twoside| is true (which is always the case for book) we make
%    the inner margin smaller than the outer one.
%    \begin{macrocode}
\else
  \if@twoside
    \setlength\@tempdima        {\paperwidth}
    \addtolength\@tempdima      {-\textwidth}
    \setlength\oddsidemargin    {.4\@tempdima}
    \addtolength\oddsidemargin  {-1in}
%    \end{macrocode}
%    The width of the margin for text is set to the remainder of the
%    width except for a `real margin' of white space of width 0.4in.
%    A check should perhaps be built in to ensure that the (text)
%    margin width does not get too small!
%
% \changes{v1.1a}{1994/03/12}{New algorithm for \cs{oddsidemargin}}
% \changes{v1.1a}{1994/03/12}{New algorithm for \cs{marginparwidth}}
% \changes{v1.2z}{1995/04/14}{Also take \cs{marginparsep} into account
%    here}
%    \begin{macrocode}
    \setlength\marginparwidth   {.6\@tempdima}
    \addtolength\marginparwidth {-\marginparsep}
    \addtolength\marginparwidth {-0.4in}
%    \end{macrocode}
%    For one-sided printing we center the text on the page, by
%    calculating the difference between |\textwidth| and
%    |\paperwidth|. Half of that difference is than used for
%    the margin (thus |\oddsidemargin| is |1in| less).
%    \begin{macrocode}
  \else
    \setlength\@tempdima        {\paperwidth}
    \addtolength\@tempdima      {-\textwidth}
    \setlength\oddsidemargin    {.5\@tempdima}
    \addtolength\oddsidemargin  {-1in}
    \setlength\marginparwidth   {.5\@tempdima}
    \addtolength\marginparwidth {-\marginparsep}
    \addtolength\marginparwidth {-0.4in}
    \addtolength\marginparwidth {-.4in}
  \fi
%    \end{macrocode}
%    With the above algorithm the |\marginparwidth| can come out quite
%    large which we may not want.
%    \begin{macrocode}
  \ifdim \marginparwidth >2in
     \setlength\marginparwidth{2in}
  \fi
%    \end{macrocode}
%    Having done these calculations we make them pt values.
%    \begin{macrocode}
  \@settopoint\oddsidemargin
  \@settopoint\marginparwidth
%    \end{macrocode}
%
%    The |\evensidemargin| can now be computed from the values set
%    above.
% \changes{v1.0l}{1994/01/11}{Computing of \cs{evensidemargin}
%    should only occur in compatibility mode}
%    \begin{macrocode}
  \setlength\evensidemargin  {\paperwidth}
  \addtolength\evensidemargin{-2in}
  \addtolength\evensidemargin{-\textwidth}
  \addtolength\evensidemargin{-\oddsidemargin}
%    \end{macrocode}
%    Setting |\evensidemargin| to a full point value may produce a
%    small error. However it will lie within the error range a
%    doublesided printer of today's technology can accurately print.
%    \begin{macrocode}
  \@settopoint\evensidemargin
\fi
%    \end{macrocode}
% \end{macro}
% \end{macro}
% \end{macro}
%
% \begin{macro}{\topmargin}
%    The |\topmargin| is the distance between the top of `the
%    printable area'---which is 1 inch below the top of the
%    paper--and the top of the box which contains the running head.
%
%    It can now be computed from the values set above.
%    \begin{macrocode}
\if@compatibility
%<!bk>  \setlength\topmargin{27pt}
%<10pt&bk>  \setlength\topmargin{.75in}
%<11pt&bk>  \setlength\topmargin{.73in}
%<12pt&bk>  \setlength\topmargin{.73in}
\else
  \setlength\topmargin{\paperheight}
  \addtolength\topmargin{-2in}
  \addtolength\topmargin{-\headheight}
  \addtolength\topmargin{-\headsep}
  \addtolength\topmargin{-\textheight}
  \addtolength\topmargin{-\footskip}     % this might be wrong!
%    \end{macrocode}
%    By changing the factor in the next line the complete page
%    can be shifted vertically.
% \changes{v1.2u}{1994/07/13}{Moved rounding of \cs{topmargin} to
%    native mode}
%    \begin{macrocode}
  \addtolength\topmargin{-.5\topmargin}
  \@settopoint\topmargin
\fi
%    \end{macrocode}
% \end{macro}
%
%
% \subsubsection{Footnotes}
%
% \begin{macro}{\footnotesep}
%    |\footnotesep| is the height of the strut placed at the beginning
%    of every footnote. It equals the  height of a normal
%    |\footnotesize| strut in this
%    class, thus no extra space occurs between footnotes.
%    \begin{macrocode}
%<10pt>\setlength\footnotesep{6.65\p@}
%<11pt>\setlength\footnotesep{7.7\p@}
%<12pt>\setlength\footnotesep{8.4\p@}
%    \end{macrocode}
% \end{macro}
%
% \begin{macro}{\footins}
%    |\skip\footins| is the space between the last line of the main
%    text and the top of the first footnote.
%    \begin{macrocode}
%<10pt>\setlength{\skip\footins}{9\p@ \@plus 4\p@ \@minus 2\p@}
%<11pt>\setlength{\skip\footins}{10\p@ \@plus 4\p@ \@minus 2\p@}
%<12pt>\setlength{\skip\footins}{10.8\p@ \@plus 4\p@ \@minus 2\p@}
%</10pt|11pt|12pt>
%    \end{macrocode}
% \end{macro}
%
% \subsubsection{Float placement parameters}
%
% All float parameters are given default values in the \LaTeXe{}
% kernel. For this reason parameters that are not counters
% need to be set with |\renewcommand|.
%
% \paragraph{Limits for the placement of floating objects}
%
% \begin{macro}{\c@topnumber}
%    The \Lcount{topnumber} counter holds the maximum number of
%    floats that can appear on the top of a text page.
%    \begin{macrocode}
%<*article|report|book>
\setcounter{topnumber}{2}
%    \end{macrocode}
% \end{macro}
%
% \begin{macro}{\topfraction}
%    This indicates the maximum part of a text page that can be
%    occupied by floats at the top.
% \changes{v1.0h}{1993/12/18}{Replaced \cs{newcommand} with
%    \cs{renewcommand}.  ASAJ.}
%    \begin{macrocode}
\renewcommand\topfraction{.7}
%    \end{macrocode}
% \end{macro}
%
% \begin{macro}{\c@bottomnumber}
%    The \Lcount{bottomnumber} counter holds the maximum number of
%    floats that can appear on the bottom of a text page.
%    \begin{macrocode}
\setcounter{bottomnumber}{1}
%    \end{macrocode}
% \end{macro}
%
% \begin{macro}{\bottomfraction}
%    This indicates the maximum part of a text page that can be
%    occupied by floats at the bottom.
% \changes{v1.0h}{1993/12/18}{Replaced \cs{newcommand} with
%    \cs{renewcommand}.  ASAJ.}
%    \begin{macrocode}
\renewcommand\bottomfraction{.3}
%    \end{macrocode}
% \end{macro}
%
% \begin{macro}{\c@totalnumber}
%    This indicates the maximum number of floats that can appear on
%    any text page.
%    \begin{macrocode}
\setcounter{totalnumber}{3}
%    \end{macrocode}
% \end{macro}
%
% \begin{macro}{\textfraction}
%    This indicates the minimum part of a text page that has to be
%    occupied by text.
% \changes{v1.0h}{1993/12/18}{Replaced \cs{newcommand} with
%    \cs{renewcommand}.  ASAJ.}
%    \begin{macrocode}
\renewcommand\textfraction{.2}
%    \end{macrocode}
% \end{macro}
%
% \begin{macro}{\floatpagefraction}
%    This indicates the minimum part of a page that has to be
%    occupied by floating objects before a `float page' is produced.
% \changes{v1.0h}{1993/12/18}{Replaced \cs{newcommand} with
%    \cs{renewcommand}.  ASAJ.}
%    \begin{macrocode}
\renewcommand\floatpagefraction{.5}
%    \end{macrocode}
% \end{macro}
%
% \begin{macro}{\c@dbltopnumber}
%    The \Lcount{dbltopnumber} counter holds the maximum number of
%    two column floats that can appear on the top of a two column text
%    page.
%    \begin{macrocode}
\setcounter{dbltopnumber}{2}
%    \end{macrocode}
% \end{macro}
%
% \begin{macro}{\dbltopfraction}
%    This indicates the maximum part of a two column text page that
%    can be occupied by two column floats at the top.
% \changes{v1.0h}{1993/12/18}{Replaced \cs{newcommand} with
%    \cs{renewcommand}.  ASAJ.}
%    \begin{macrocode}
\renewcommand\dbltopfraction{.7}
%    \end{macrocode}
% \end{macro}
%
% \begin{macro}{\dblfloatpagefraction}
%    This indicates the minimum part of a page that has to be
%    occupied by two column wide floating objects before a `float
%    page' is produced.
% \changes{v1.0h}{1993/12/18}{Replaced \cs{newcommand} with
%    \cs{renewcommand}.  ASAJ.}
%    \begin{macrocode}
\renewcommand\dblfloatpagefraction{.5}
%</article|report|book>
%    \end{macrocode}
% \end{macro}
%
% \paragraph{Floats on a text page}
%
% \begin{macro}{\floatsep}
% \begin{macro}{\textfloatsep}
% \begin{macro}{\intextsep}
%    When a floating object is placed on a page with text, these
%    parameters control the separation between the float and the other
%    objects on the page. These parameters are used for both
%    one-column mode and single-column floats in two-column mode.
%
%    |\floatsep| is the space between adjacent floats that are moved
%    to the top or bottom of the text page.
%
%    |\textfloatsep| is the space between the main text and floats
%    at the top or bottom of the page.
%
%    |\intextsep| is the space between in-text floats and the text.
%    \begin{macrocode}
%<*10pt>
\setlength\floatsep    {12\p@ \@plus 2\p@ \@minus 2\p@}
\setlength\textfloatsep{20\p@ \@plus 2\p@ \@minus 4\p@}
\setlength\intextsep   {12\p@ \@plus 2\p@ \@minus 2\p@}
%</10pt>
%<*11pt>
\setlength\floatsep    {12\p@ \@plus 2\p@ \@minus 2\p@}
\setlength\textfloatsep{20\p@ \@plus 2\p@ \@minus 4\p@}
\setlength\intextsep   {12\p@ \@plus 2\p@ \@minus 2\p@}
%</11pt>
%<*12pt>
\setlength\floatsep    {12\p@ \@plus 2\p@ \@minus 4\p@}
\setlength\textfloatsep{20\p@ \@plus 2\p@ \@minus 4\p@}
\setlength\intextsep   {14\p@ \@plus 4\p@ \@minus 4\p@}
%</12pt>
%    \end{macrocode}
% \end{macro}
% \end{macro}
% \end{macro}
%
% \begin{macro}{\dblfloatsep}
% \begin{macro}{\dbltextfloatsep}
%    When floating objects that span the whole |\textwidth| are placed
%    on a text page when we are in twocolumn mode the separation
%    between the float and the text is controlled by |\dblfloatsep|
%    and |\dbltextfloatsep|.
%
%    |\dblfloatsep| is the space between adjacent floats that are moved
%    to the top or bottom of the text page.
%
%    |\dbltextfloatsep| is the space between the main text and floats
%    at the top or bottom of the page.
%
%    \begin{macrocode}
%<*10pt>
\setlength\dblfloatsep    {12\p@ \@plus 2\p@ \@minus 2\p@}
\setlength\dbltextfloatsep{20\p@ \@plus 2\p@ \@minus 4\p@}
%</10pt>
%<*11pt>
\setlength\dblfloatsep    {12\p@ \@plus 2\p@ \@minus 2\p@}
\setlength\dbltextfloatsep{20\p@ \@plus 2\p@ \@minus 4\p@}
%</11pt>
%<*12pt>
\setlength\dblfloatsep    {14\p@ \@plus 2\p@ \@minus 4\p@}
\setlength\dbltextfloatsep{20\p@ \@plus 2\p@ \@minus 4\p@}
%</12pt>
%    \end{macrocode}
% \end{macro}
% \end{macro}
%
% \paragraph{Floats on their own page or column}
%
% \begin{macro}{\@fptop}
% \begin{macro}{\@fpsep}
% \begin{macro}{\@fpbot}
%    When floating objects are placed on separate pages the layout of
%    such pages is controlled by these parameters. At the top of the
%    page |\@fptop| amount of stretchable whitespace is inserted, at
%    the bottom of the page we get an |\@fpbot| amount of stretchable
%    whitespace. Between adjacent floats the |\@fpsep| is inserted.
%
%    These parameters are used for the placement of floating objects
%    in one column mode, or in single column floats in two column
%    mode.
%
%    Note that at least one of the two parameters |\@fptop| and
%    |\@fpbot| should contain a |plus ...fil| to allow filling the
%    remaining empty space.
%    \begin{macrocode}
%<*10pt>
\setlength\@fptop{0\p@ \@plus 1fil}
\setlength\@fpsep{8\p@ \@plus 2fil}
\setlength\@fpbot{0\p@ \@plus 1fil}
%</10pt>
%<*11pt>
\setlength\@fptop{0\p@ \@plus 1fil}
\setlength\@fpsep{8\p@ \@plus 2fil}
\setlength\@fpbot{0\p@ \@plus 1fil}
%</11pt>
%<*12pt>
\setlength\@fptop{0\p@ \@plus 1fil}
\setlength\@fpsep{10\p@ \@plus 2fil}
\setlength\@fpbot{0\p@ \@plus 1fil}
%</12pt>
%    \end{macrocode}
% \end{macro}
% \end{macro}
% \end{macro}
%
% \begin{macro}{\@dblfptop}
% \begin{macro}{\@dblfpsep}
% \begin{macro}{\@dblfpbot}
%    Double column floats in two column mode are handled with similar
%    parameters.
%    \begin{macrocode}
%<*10pt>
\setlength\@dblfptop{0\p@ \@plus 1fil}
\setlength\@dblfpsep{8\p@ \@plus 2fil}
\setlength\@dblfpbot{0\p@ \@plus 1fil}
%</10pt>
%<*11pt>
\setlength\@dblfptop{0\p@ \@plus 1fil}
\setlength\@dblfpsep{8\p@ \@plus 2fil}
\setlength\@dblfpbot{0\p@ \@plus 1fil}
%</11pt>
%<*12pt>
\setlength\@dblfptop{0\p@ \@plus 1fil}
\setlength\@dblfpsep{10\p@ \@plus 2fil}
\setlength\@dblfpbot{0\p@ \@plus 1fil}
%</12pt>
%<*article|report|book>
%    \end{macrocode}
% \end{macro}
% \end{macro}
% \end{macro}
%
% \subsection{Page Styles}
%
%    The page style \pstyle{foo} is defined by defining the command
%    |\ps@foo|.   This command should make only local definitions.
%    There should be no stray spaces in the definition, since they
%    could lead to mysterious extra spaces in the output (well, that's
%    something that should be always avoided).
%
% \begin{macro}{\@evenhead}
% \begin{macro}{\@oddhead}
% \begin{macro}{\@evenfoot}
% \begin{macro}{\@oddfoot}
%    The |\ps@...| command defines the macros |\@oddhead|,
%    |\@oddfoot|, |\@evenhead|, and |\@evenfoot| to define the running
%    heads and feet---e.g., |\@oddhead| is the macro to produce the
%    contents of the heading box for odd-numbered pages.  It is called
%    inside an |\hbox| of width |\textwidth|.
% \end{macro}
% \end{macro}
% \end{macro}
% \end{macro}
%
% \subsubsection{Marking conventions}
%
%    To make headings determined by the sectioning commands, the page
%    style defines the commands |\chaptermark|, |\sectionmark|,
%    \ldots,\\
%     where |\chaptermark{|\meta{TEXT}|}| is called by
%    |\chapter| to set a mark, and so on.
%
%    The |\...mark| commands and the |\...head| macros are defined
%    with the help of the following macros.  (All the |\...mark|
%    commands should be initialized to no-ops.)
%
%    \LaTeX{} extends \TeX's |\mark| facility by producing two kinds
%    of marks, a `left' and a `right' mark, using the following
%    commands:
%    \begin{flushleft}
%     |\markboth{|\meta{LEFT}|}{|\meta{RIGHT}|}|: Adds both marks.
%
%     |\markright{|\meta{RIGHT}|}|: Adds a `right' mark.
%
%     |\leftmark|: Used in the |\@oddhead|, |\@oddfoot|, |\@evenhead|
%                  or |\@evenfoot| macros, it gets the current `left'
%                  mark.  |\leftmark| works like \TeX's |\botmark|
%                  command.
%
%     |\rightmark|: Used in the |\@oddhead|, |\@oddfoot|, |\@evenhead|
%                   or  |\@evenfoot| macros, it gets the current
%                   `right' mark. |\rightmark| works like \TeX's
%                   |\firstmark| command.
%    \end{flushleft}
%
%    The marking commands work reasonably well for right marks
%    `numbered within' left marks---e.g., the left mark is changed by a
%    |\chapter| command and the right mark is changed by a |\section|
%    command.  However, it does produce somewhat anomalous results if
%    two |\markboth|'s occur on the same page.
%
%
%    Commands like |\tableofcontents| that should set the marks in some
%    page styles use a |\@mkboth| command, which is |\let| by the
%    pagestyle command (|\ps@...|)  to |\markboth| for setting the
%    heading or to |\@gobbletwo| to do nothing.
%
%
% \subsubsection{Defining the page styles}
% \label{sec:classes:pagestyle}
%
%    The pagestyles \pstyle{empty} and \pstyle{plain} are defined in
%    \file{latex.dtx}.
%
% \begin{macro}{\ps@headings}
%    The definition of the page style \pstyle{headings} has to be
%    different for two sided printing than it is for one sided
%    printing.
%
%    \begin{macrocode}
\if@twoside
  \def\ps@headings{%
%    \end{macrocode}
%    The running feet are empty in this page style, the running head
%    contains the page number and one of the marks.
%    \begin{macrocode}
      \let\@oddfoot\@empty\let\@evenfoot\@empty
      \def\@evenhead{\thepage\hfil\slshape\leftmark}%
      \def\@oddhead{{\slshape\rightmark}\hfil\thepage}%
%    \end{macrocode}
%
%    When using this page style, the contents of the running head is
%    determined by the chapter and section titles. So we |\let|
%    |\@mkboth| to |\markboth|.
%    \begin{macrocode}
      \let\@mkboth\markboth
%    \end{macrocode}
%
%    For the article document class we define |\sectionmark| to clear
%    the right mark and put the number of the section (when it is
%    numbered) and its title in the left mark. The rightmark is set by
%    |\subsectionmark| to contain the subsection titles.
%
%    Note the use of |##1| for the parameter of the |\sectionmark|
%    command, which will be defined when |\ps@headings| is executed.
%
% \changes{v1.2z}{1995/04/03}{Removed extra dot after \cs{thesection}
%    (PR 1519)}
% \changes{v1.3c}{1995/05/25}{Replace \cs{hskip}
%    \texttt{1em}\cs{relax} with \cs{quad}}
%    \begin{macrocode}
%<*article>
    \def\sectionmark##1{%
      \markboth {\MakeUppercase{%
        \ifnum \c@secnumdepth >\z@
          \thesection\quad
        \fi
        ##1}}{}}%
    \def\subsectionmark##1{%
      \markright {%
        \ifnum \c@secnumdepth >\@ne
          \thesubsection\quad
        \fi
        ##1}}}
%</article>
%    \end{macrocode}
%
%    In the report and book document classes we use the |\chaptermark|
%    and |\sectionmark| macros to fill the running heads.
%
%    Note the use of |##1| for the parameter of the |\chaptermark|
%    command, which will be defined when |\ps@headings| is executed.
%
%    \begin{macrocode}
%<*report|book>
    \def\chaptermark##1{%
      \markboth {\MakeUppercase{%
        \ifnum \c@secnumdepth >\m@ne
%<book>          \if@mainmatter
            \@chapapp\ \thechapter. \ %
%<book>          \fi
        \fi
        ##1}}{}}%
    \def\sectionmark##1{%
      \markright {\MakeUppercase{%
        \ifnum \c@secnumdepth >\z@
          \thesection. \ %
        \fi
        ##1}}}}
%</report|book>
%    \end{macrocode}
%
%    The definition of |\ps@headings| for one sided printing can be
%    much simpler, because we treat even and odd pages the same.
%    Therefore we don't need to define |\@even...|.
%    \begin{macrocode}
\else
  \def\ps@headings{%
    \let\@oddfoot\@empty
    \def\@oddhead{{\slshape\rightmark}\hfil\thepage}%
    \let\@mkboth\markboth
%    \end{macrocode}
%    We use |\markright| now instead of |\markboth| as we did for two
%    sided printing.
%    \begin{macrocode}
%<*article>
    \def\sectionmark##1{%
      \markright {\MakeUppercase{%
        \ifnum \c@secnumdepth >\m@ne
          \thesection\quad
        \fi
        ##1}}}}
%</article>
%    \end{macrocode}
%
%    \begin{macrocode}
%<*report|book>
    \def\chaptermark##1{%
      \markright {\MakeUppercase{%
        \ifnum \c@secnumdepth >\m@ne
%<book>          \if@mainmatter
            \@chapapp\ \thechapter. \ %
%<book>          \fi
        \fi
        ##1}}}}
%</report|book>
\fi
%    \end{macrocode}
% \end{macro}
%
% \begin{macro}{\ps@myheadings}
%    The definition of the page style \pstyle{myheadings} is fairly
%    simple because the user determines the contents of the running
%    head himself by using the |\markboth| and |\markright| commands.
%
%    \begin{macrocode}
\def\ps@myheadings{%
    \let\@oddfoot\@empty\let\@evenfoot\@empty
    \def\@evenhead{\thepage\hfil\slshape\leftmark}%
    \def\@oddhead{{\slshape\rightmark}\hfil\thepage}%
%    \end{macrocode}
%
%    We have to make sure that the marking commands that are used by
%    the chapter and section headings are disabled. We do this
%    |\let|ting them to a macro that gobbles its argument(s).
%    \begin{macrocode}
    \let\@mkboth\@gobbletwo
%<!article>    \let\chaptermark\@gobble
    \let\sectionmark\@gobble
%<article>    \let\subsectionmark\@gobble
    }
%    \end{macrocode}
% \end{macro}
%
% \section{Document Markup}
%
% \subsection{The title}
%
% \begin{macro}{\title}
% \begin{macro}{\author}
% \begin{macro}{\date}
%    These three macros are provided by \file{latex.dtx} to provide
%    information about the title, author(s) and date of the document.
%    The information is stored away in internal control sequences.
%    It is the task of the |\maketitle| command to use the
%    information provided. The definitions of these macros are shown
%    here for information.
%    \begin{macrocode}
% \newcommand*{\title}[1]{\gdef\@title{#1}}
% \newcommand*{\author}[1]{\gdef\@author{#1}}
% \newcommand*{\date}[1]{\gdef\@date{#1}}
%    \end{macrocode}
%    The |\date| macro gets today's date by default.
%    \begin{macrocode}
% \date{\today}
%    \end{macrocode}
% \end{macro}
% \end{macro}
% \end{macro}
%
% \begin{macro}{\maketitle}
%    The definition of |\maketitle| depends on whether a separate
%    title page is made. This is the default for the report and book
%    document classes, but for the article class it is optional.
%
%    When we are making a title page, we locally redefine
%    |\footnotesize| and |footnoterule| to change the appearance of
%    the footnotes that are produced by the |\thanks| command;
%    these changes affect all footnotes.
% \changes{v1.3o}{1995/11/02}{(CAR) Make \cs{footnote} always work in
%      title, etc}
%    \begin{macrocode}
  \if@titlepage
  \newcommand\maketitle{\begin{titlepage}%
  \let\footnotesize\small
  \let\footnoterule\relax
  \let \footnote \thanks
%    \end{macrocode}
%    We center the entire title vertically; the centering is set off a
%    little by adding a |\vskip|. (In compatibility mode the pagenumber
%    is set to 0 by the titlepage environment to keep the behaviour
%    of \LaTeX\ 2.09 style files.)
% \changes{v1.0g}{1993/12/09}{Removed the setting of the page number,
%    when not in compatibility mode}
% \changes{v1.2c}{1994/03/17}{Removed setting of page number, now done
%    in titlepage environment}
%    \begin{macrocode}
  \null\vfil
  \vskip 60\p@
%    \end{macrocode}
%    Then we set the title, in a |\LARGE| font; leave a little space
%    and set the author(s) in a |\large| font. We do this inside a
%    tabular environment to get them in a single column.
%    Before the date we leave a little whitespace again.
%    \begin{macrocode}
  \begin{center}%
    {\LARGE \@title \par}%
    \vskip 3em%
    {\large
     \lineskip .75em%
      \begin{tabular}[t]{c}%
        \@author
      \end{tabular}\par}%
      \vskip 1.5em%
    {\large \@date \par}%       % Set date in \large size.
  \end{center}\par
%    \end{macrocode}
%    Then we call |\@thanks| to print the information that goes into
%    the footnote and finish the page.
%    \begin{macrocode}
  \@thanks
  \vfil\null
  \end{titlepage}%
%    \end{macrocode}
%    We reset the \Lcount{footnote} counter, disable |\thanks| and
%    |\maketitle| and save some storage space by emptying the internal
%    information macros.
% \changes{v1.3j}{1995/08/16}{use \cs{let} to save space}
% \changes{v1.3n}{1995/10/29}{Empty \cs{@date} as well}
%    \begin{macrocode}
  \setcounter{footnote}{0}%
  \global\let\thanks\relax
  \global\let\maketitle\relax
  \global\let\@thanks\@empty
  \global\let\@author\@empty
  \global\let\@date\@empty
  \global\let\@title\@empty
%    \end{macrocode}
%    After the title is set the declaration commands |\title|, etc.\
%    can vanish.
%    The definition of |\and| makes only sense within the argument of
%    |\author| so this can go as well.
% \changes{v1.3k}{1995/08/27}{Disable \cs{title} and similar decls}
%    \begin{macrocode}
  \global\let\title\relax
  \global\let\author\relax
  \global\let\date\relax
  \global\let\and\relax
}
%    \end{macrocode}
%    When the title is not on a page of its own, the layout of the
%    title is a little different. We use symbols to mark the footnotes
%    and we have to deal with two column documents.
%
%    Therefore we first start a new group to keep changes local. Then
%    we redefine |\thefootnote| to use |\fnsymbol|; and change
%    |\@makefnmark| so that footnotemarks have zero width (to make the
%    centering of the author names look better).
% \changes{v1.2s}{1994/06/02}{Reset \cs{@makefntext}}
% \changes{v1.3a}{1995/05/17}{Use \cs{@makefnmark} in definition of
%    \cs{@makefntext}}
% \changes{v1.3g}{1995/06/26}{Fix definition of \cs{@makefnmark} and
%    \cs{@makefntext} to a) work and b) without using math}
%    \begin{macrocode}
\else
\newcommand\maketitle{\par
  \begingroup
    \renewcommand\thefootnote{\@fnsymbol\c@footnote}%
    \def\@makefnmark{\rlap{\@textsuperscript{\normalfont\@thefnmark}}}%
    \long\def\@makefntext##1{\parindent 1em\noindent
            \hb@xt@1.8em{%
                \hss\@textsuperscript{\normalfont\@thefnmark}}##1}%
%    \end{macrocode}
%    If this is a twocolumn document we start a new page in twocolumn
%    mode, with the title set to the full width of the text. The
%    actual printing of the title information is left to
%    |\@maketitle|.
% \changes{v1.2k}{1994/05/06}{Added check on number of columns in use
%    locally}
%    \begin{macrocode}
    \if@twocolumn
      \ifnum \col@number=\@ne
        \@maketitle
      \else
        \twocolumn[\@maketitle]%
      \fi
    \else
%    \end{macrocode}
%    When this is not a twocolumn document we just start a new page,
%    prevent floating objects from appearing on the top of this page
%    and print the title information.
%    \begin{macrocode}
      \newpage
      \global\@topnum\z@   % Prevents figures from going at top of page.
      \@maketitle
    \fi
%    \end{macrocode}
%    This page gets a \pstyle{plain} layout. We call |\@thanks| to
%    produce the footnotes.
%    \begin{macrocode}
    \thispagestyle{plain}\@thanks
%    \end{macrocode}
%    Now we can close the group, reset the \Lcount{footnote} counter,
%    disable |\thanks|, |\maketitle| and |\@maketitle| and save some
%    storage space by emptying the internal information macros.
% \changes{v1.3j}{1995/08/16}{use \cs{let} to save space}
% \changes{v1.3k}{1995/08/27}{Disable \cs{title} and similar decls}
% \changes{v1.3n}{1995/10/29}{Empty \cs{@date} as well}
%    \begin{macrocode}
  \endgroup
  \setcounter{footnote}{0}%
  \global\let\thanks\relax
  \global\let\maketitle\relax
  \global\let\@maketitle\relax
  \global\let\@thanks\@empty
  \global\let\@author\@empty
  \global\let\@date\@empty
  \global\let\@title\@empty
  \global\let\title\relax
  \global\let\author\relax
  \global\let\date\relax
  \global\let\and\relax
}
%    \end{macrocode}
% \end{macro}
%
% \begin{macro}{\@maketitle}
%    This macro takes care of formatting the title information when we
%    have no separate title page.
%
%    We always start a new page, leave some white space and center the
%    information. The title is set in a |\LARGE| font, the author
%    names and the date in a |\large| font.
% \changes{v1.3o}{1995/11/02}{(CAR) Make \cs{footnote} always work in
%      title, etc}
%    \begin{macrocode}
\def\@maketitle{%
  \newpage
  \null
  \vskip 2em%
  \begin{center}%
  \let \footnote \thanks
    {\LARGE \@title \par}%
    \vskip 1.5em%
    {\large
      \lineskip .5em%
      \begin{tabular}[t]{c}%
        \@author
      \end{tabular}\par}%
    \vskip 1em%
    {\large \@date}%
  \end{center}%
  \par
  \vskip 1.5em}
\fi
%    \end{macrocode}
% \end{macro}
%
% \subsection{Chapters and Sections}
%
% \subsubsection{Building blocks} The definitions in this part of the
%    class file make use of two internal macros, |\@startsection| and
%    |\secdef|. To understand
%    what is going on here, we describe their syntax.
%
%    The macro |\@startsection| has 6 required arguments, optionally
%    followed by  a $*$, an optional argument and a required argument:
%
%    |\@startsection|\meta{name}\meta{level}\meta{indent}^^A
%                    \meta{beforeskip}\meta{afterskip}\meta{style}
%            optional *\\
%    \null\hphantom{\bslash @startsection}^^A
%            |[|\meta{altheading}|]|\meta{heading}
%
%    It is a generic command to start a section, the arguments have
%    the following meaning:
%
%    \begin{description}
%    \item[\meta{name}] The name of the user level command, e.g.,
%          `section'.
%    \item[\meta{level}] A number, denoting the depth of the section
%          -- e.g., chapter=1, section = 2, etc.  A section number
%          will be printed if and only if \meta{level} $<=$  the value
%          of the \Lcount{secnumdepth} counter.
%    \item[\meta{indent}] The indentation of the heading from the left
%          margin
%    \item[\meta{beforeskip}] The absolute value of this argument
%          gives the skip to leave above the heading. If it is
%          negative, then the paragraph indent of the text following
%          the heading is suppressed.
%    \item[\meta{afterskip}] If positive, this gives the skip to leave
%          below the heading, else it gives the skip to leave to the
%          right of a run-in heading.
%    \item[\meta{style}] Commands to set the style of the heading.
%    \item[$*$] When this is missing the heading is numbered and the
%          corresponding counter is incremented.
%    \item[\meta{altheading}] Gives an alternative heading to use in
%          the table of contents and in the running heads. This should
%          not be present when the $*$ form is used.
%    \item[\meta{heading}] The heading of the new section.
%    \end{description}
%  A sectioning command is normally defined to |\@startsection| and
%  its first six arguments.
%
%    The macro |\secdef| can be used when a sectioning command is
%    defined without using |\@startsection|. It has two arguments:
%
%    |\secdef|\meta{unstarcmds}\meta{starcmds}
%
%    \begin{description}
%    \item[\meta{unstarcmds}] Used for the normal form of the
%          sectioning command.
%    \item[\meta{starcmds}] Used for the $*$-form of the
%          sectioning command.
%    \end{description}
%
%    You can use |\secdef| as follows:
% \begin{verbatim}
%       \def\chapter { ... \secdef \CMDA \CMDB }
%       \def\CMDA    [#1]#2{ ... }  % Command to define
%                                   % \chapter[...]{...}
%       \def\CMDB    #1{ ... }      % Command to define
%                                   % \chapter*{...}
% \end{verbatim}
%
% \subsubsection{Mark commands}
%
% \begin{macro}{\chaptermark}
% \begin{macro}{\sectionmark}
% \begin{macro}{\subsectionmark}
% \begin{macro}{\subsubsectionmark}
% \begin{macro}{\paragraphmark}
% \begin{macro}{\subparagraphmark}
%    Default initializations of |\...mark| commands.  These commands
%    are used in the definition of the page styles (see
%    section~\ref{sec:classes:pagestyle}) Most of them are already defined by
%    \file{latex.dtx}, so they are only shown here.
%
%    \begin{macrocode}
%<!article>\newcommand*\chaptermark[1]{}
% \newcommand*\sectionmark[1]{}
% \newcommand*\subsectionmark[1]{}
% \newcommand*\subsubsectionmark[1]{}
% \newcommand*\paragraphmark[1]{}
% \newcommand*\subparagraphmark[1]{}
%    \end{macrocode}
% \end{macro}
% \end{macro}
% \end{macro}
% \end{macro}
% \end{macro}
% \end{macro}
%
% \subsubsection{Define Counters}
%
% \begin{macro}{\c@secnumdepth}
%    The value of the counter \Lcount{secnumdepth} gives the depth of
%    the highest-level sectioning command that is to produce section
%    numbers.
%    \begin{macrocode}
%<article>\setcounter{secnumdepth}{3}
%<!article>\setcounter{secnumdepth}{2}
%    \end{macrocode}
% \end{macro}
%
% \begin{macro}{\c@part}
% \begin{macro}{\c@chapter}
% \begin{macro}{\c@section}
% \begin{macro}{\c@subsection}
% \begin{macro}{\c@subsubsection}
% \begin{macro}{\c@paragraph}
% \begin{macro}{\c@subparagraph}
%    These counters are used for the section numbers. The macro\\
%    |\newcounter{|\meta{newctr}|}[|\meta{oldctr}|]|\\
%     defines\meta{newctr} to be a counter, which is reset to zero when
%    counter \meta{oldctr} is stepped. Counter \meta{oldctr} must
%    already be defined.
%
%    \begin{macrocode}
\newcounter {part}
%<article>\newcounter {section}
%<*report|book>
\newcounter {chapter}
\newcounter {section}[chapter]
%</report|book>
\newcounter {subsection}[section]
\newcounter {subsubsection}[subsection]
\newcounter {paragraph}[subsubsection]
\newcounter {subparagraph}[paragraph]
%    \end{macrocode}
% \end{macro}
% \end{macro}
% \end{macro}
% \end{macro}
% \end{macro}
% \end{macro}
% \end{macro}
%
% \begin{macro}{\thepart}
% \begin{macro}{\thechapter}
% \begin{macro}{\thesection}
% \begin{macro}{\thesubsection}
% \begin{macro}{\thesubsubsection}
% \begin{macro}{\theparagraph}
% \begin{macro}{\thesubparagraph}
%    For any counter \Lcount{CTR}, |\theCTR| is a macro that defines
%    the printed version of counter \Lcount{CTR}.  It is defined in
%    terms of the following macros:
%
%    |\arabic{|\Lcount{COUNTER}|}| prints the value of
%    \Lcount{COUNTER} as an arabic numeral.
%
%    |\roman{|\Lcount{COUNTER}|}| prints the value of
%    \Lcount{COUNTER} as a lowercase roman numberal.
%
%    |\Roman{|\Lcount{COUNTER}|}| prints the value of
%    \Lcount{COUNTER} as an uppercase roman numberal.
%
%    |\alph{|\Lcount{COUNTER}|}| prints the value of \Lcount{COUNTER}
%    as a lowercase letter: $1 =$~a, $2 =$~ b, etc.
%
%    |\Alph{|\Lcount{COUNTER}|}| prints the value of \Lcount{COUNTER}
%    as an uppercase letter: $1 =$~A, $2 =$~B, etc.
%
%    Actually to save space the internal counter repesentations
%    and the commands operating on those are used.
%    \begin{macrocode}
\renewcommand \thepart {\@Roman\c@part}
%<article>\renewcommand \thesection {\@arabic\c@section}
%<*report|book>
\renewcommand \thechapter {\@arabic\c@chapter}
\renewcommand \thesection {\thechapter.\@arabic\c@section}
%</report|book>
\renewcommand\thesubsection   {\thesection.\@arabic\c@subsection}
\renewcommand\thesubsubsection{\thesubsection.\@arabic\c@subsubsection}
\renewcommand\theparagraph    {\thesubsubsection.\@arabic\c@paragraph}
\renewcommand\thesubparagraph {\theparagraph.\@arabic\c@subparagraph}
%    \end{macrocode}
% \end{macro}
% \end{macro}
% \end{macro}
% \end{macro}
% \end{macro}
% \end{macro}
% \end{macro}
%
% \begin{macro}{\@chapapp}
%    |\@chapapp| is initially defined to be `|\chaptername|'. The
%    |\appendix| command redefines it to be `|\appendixname|'.
%
%    \begin{macrocode}
%<report|book>\newcommand\@chapapp{\chaptername}
%    \end{macrocode}
% \end{macro}
%
%  \subsubsection{Front Matter, Main Matter, and Back Matter}
%
%    A book contains these three (logical) sections. The switch
%    |\@mainmatter| is true iff we are processing Main Matter.  When
%    this switch is false, the |\chapter| command does not print
%    chapter numbers.
%
%    Here we define the commands that start these sections.
%  \begin{macro}{\frontmatter}
%    This command starts Roman page numbering and turns off chapter
%    numbering.  Since this restarts the page numbering from 1, it
%    should also ensure that a recto page is used.
% \changes{v1.3r}{1996/05/26}{Make this command react to the option
%    \texttt{openany}}
% \changes{v1.3y}{1998/05/05}{Two years on: Make this command not
%    react to the option \texttt{openany} as this makes the
%    verso/recto numbering wrong: see pr/2754 for discussion}
%    \begin{macrocode}
%<*book>
\newcommand\frontmatter{%
%   \if@openright
    \cleardoublepage
%   \else
%     \clearpage
%   \fi
  \@mainmatterfalse
  \pagenumbering{roman}}
%    \end{macrocode}
%  \end{macro}
%
%  \begin{macro}{\mainmatter}
%    This command clears the page, starts arabic page numbering and
%    turns on chapter numbering.  Since this restarts the page numbering
%    from 1, it should also ensure that a recto page is used.
% \changes{v1.3r}{1996/05/26}{Make this command react to the option
%    \texttt{openany}}
% \changes{v1.3y}{1998/05/05}{Two years on: Make this command not
%    react to the option \texttt{openany} as this makes the
%    verso/recto numbering wrong: see pr/2754 for discussion}
%    \begin{macrocode}
\newcommand\mainmatter{%
%   \if@openright
    \cleardoublepage
%   \else
%     \clearpage
%   \fi
  \@mainmattertrue
  \pagenumbering{arabic}}
%    \end{macrocode}
%  \end{macro}
%
%  \begin{macro}{\backmatter}
%    This clears the page, turns off chapter numbering and leaves page
%    numbering unchanged.
%    \begin{macrocode}
\newcommand\backmatter{%
  \if@openright
    \cleardoublepage
  \else
    \clearpage
  \fi
  \@mainmatterfalse}
%</book>
%    \end{macrocode}
%  \end{macro}
%
% \subsubsection{Parts}
%
% \begin{macro}{\part}
%    The command to start a new part of our document.
%
%    In the article class the definition of |\part| is rather simple;
%    we start a new paragraph, add a little white space, suppress the
%    indentation of the first paragraph and make use of |\secdef|.
%    As in other sectioning commands (cf.\ |\@startsection| in the
%    {\LaTeXe} kernel), we need to check the |@noskipsec| switch and
%    force horizontal mode if it is set.
% \changes{v1.4a}{1999/01/07}{Check \texttt{@noskipsec} switch and
%      possibly force horizontal mode; see PR/2889.}
%    \begin{macrocode}
%<*article>
\newcommand\part{%
   \if@noskipsec \leavevmode \fi
   \par
   \addvspace{4ex}%
   \@afterindentfalse
   \secdef\@part\@spart}
%</article>
%    \end{macrocode}
%
%    For the report and book classes we things a bit different.
%
%    We start a new (righthand) page and use the \pstyle{plain}
%    pagestyle.
% \changes{v1.3r}{1996/05/26}{Make this command react to the option
%    \texttt{openany}}
%    \begin{macrocode}
%<*report|book>
\newcommand\part{%
  \if@openright
    \cleardoublepage
  \else
    \clearpage
  \fi
  \thispagestyle{plain}%
%    \end{macrocode}
%    When we are making a two column document, this will be a one
%    column page. We use |@tempswa| to remember to switch back to two
%    columns.
%    \begin{macrocode}
  \if@twocolumn
    \onecolumn
    \@tempswatrue
  \else
    \@tempswafalse
  \fi
%    \end{macrocode}
%    We need an empty box to prevent the fil glue from disappearing.
% \changes{v1.3j}{1995/08/16}{Replace \cs{hbox} by \cs{null}}
%    \begin{macrocode}
  \null\vfil
%    \end{macrocode}
%    Here we use |\secdef| to indicate which commands to use to make
%    the actual heading.
%    \begin{macrocode}
  \secdef\@part\@spart}
%</report|book>
%    \end{macrocode}
%
% \begin{macro}{\@part}
%    This macro does the actual formatting of the title of the part.
%    Again the macro is differently defined for the article document
%    class than for the document classes report and book.

%    When \Lcount{secnumdepth} is larger than $-1$ for the
%    document class article, we have a numbered
%    part, otherwise it is unnumbered.
%    \begin{macrocode}
%<*article>
\def\@part[#1]#2{%
    \ifnum \c@secnumdepth >\m@ne
      \refstepcounter{part}%
      \addcontentsline{toc}{part}{\thepart\hspace{1em}#1}%
    \else
      \addcontentsline{toc}{part}{#1}%
    \fi
%    \end{macrocode}
%    We  print the title flush left in the article class.
%    Also we prevent breaking between lines and reset the font.
% \changes{v1.3c}{1995/05/25}{replace \cs{reset@font} with
%    \cs{normalfont}}
%    \begin{macrocode}
    {\parindent \z@ \raggedright
     \interlinepenalty \@M
     \normalfont
%    \end{macrocode}
%    When this is a numbered part we have to print the number and the
%    title. The |\nobreak| should prevent a page break here.
% \changes{v1.4e}{2001/05/24}{Replaced tilde with \cs{nobreakspace}
%                             (pr/3310)}
%    \begin{macrocode}
     \ifnum \c@secnumdepth >\m@ne
       \Large\bfseries \partname\nobreakspace\thepart
       \par\nobreak
     \fi
     \huge \bfseries #2%
%    \end{macrocode}
%    Now we empty the mark registers, leave some white space and let
%    |\@afterheading| take care of suppressing the indentation.
%    \begin{macrocode}
     \markboth{}{}\par}%
    \nobreak
    \vskip 3ex
    \@afterheading}
%</article>
%    \end{macrocode}
%
%    When \Lcount{secnumdepth} is larger than $-2$ for the
%    document class report and book, we have a numbered
%    part, otherwise it is unnumbered.
%    \begin{macrocode}
%<*report|book>
\def\@part[#1]#2{%
    \ifnum \c@secnumdepth >-2\relax
      \refstepcounter{part}%
      \addcontentsline{toc}{part}{\thepart\hspace{1em}#1}%
    \else
      \addcontentsline{toc}{part}{#1}%
    \fi
%    \end{macrocode}
%    We empty the mark registers and center the title on the page in the
%    report and book document classes.
%    Also we prevent breaking between lines and reset the font.
% \changes{v1.3c}{1995/05/25}{replace \cs{reset@font} with
%    \cs{normalfont}}
% \changes{v1.3j}{1995/08/16}{add missing percent}
%    \begin{macrocode}
    \markboth{}{}%
    {\centering
     \interlinepenalty \@M
     \normalfont
%    \end{macrocode}
%    When this is a numbered part we have to print the number.
% \changes{v1.4e}{2001/05/24}{Replaced tilde with \cs{nobreakspace}
%                             (pr/3310)}
%    \begin{macrocode}
     \ifnum \c@secnumdepth >-2\relax
       \huge\bfseries \partname\nobreakspace\thepart
       \par
%    \end{macrocode}
%    We leave some space before we print the title and leave the
%    finishing up to |\@endpart|.
%    \begin{macrocode}
       \vskip 20\p@
     \fi
     \Huge \bfseries #2\par}%
    \@endpart}
%</report|book>
%    \end{macrocode}
% \end{macro}
%
% \begin{macro}{\@spart}
%    This macro does the actual formatting of the title of the part
%    when the star form of the user command was used. In this case we
%    \emph{never} print a number. Otherwise the formatting is the
%    same.
%
%    The differences between the definition of this macro in the
%    article document class and in the report and book document
%    classes are similar as they were for |\@part|.
% \changes{v1.3c}{1995/05/25}{replace \cs{reset@font} with
%    \cs{normalfont}}
%    \begin{macrocode}
%<*article>
\def\@spart#1{%
    {\parindent \z@ \raggedright
     \interlinepenalty \@M
     \normalfont
     \huge \bfseries #1\par}%
     \nobreak
     \vskip 3ex
     \@afterheading}
%</article>
%<*report|book>
\def\@spart#1{%
    {\centering
     \interlinepenalty \@M
     \normalfont
     \Huge \bfseries #1\par}%
    \@endpart}
%</report|book>
%    \end{macrocode}
% \end{macro}
%
% \begin{macro}{\@endpart}
% \changes{v1.3j}{1995/08/16}{move docstrip guard to avoid defining
%    \cs{@endpart} in article}
%    This macro finishes the part page, for both |\@part| and
%    |\@spart|.
%
%    First we fill the current page.
%    \begin{macrocode}
%<*report|book>
\def\@endpart{\vfil\newpage
%    \end{macrocode}
%    Then, when we are in twosided mode and chapters are supposed to
%    be on right hand sides, we produce a completely blank page.
% \changes{v1.4b}{2000/05/19}{Only add empty page after part if
%    twoside and openright (pr/3155)}
%    \begin{macrocode}
              \if@twoside
               \if@openright
                \null
                \thispagestyle{empty}%
                \newpage
               \fi
              \fi
%    \end{macrocode}
%    When this was a two column document we have to switch back to two
%    column mode.
%    \begin{macrocode}
              \if@tempswa
                \twocolumn
              \fi}
%</report|book>
%    \end{macrocode}
% \end{macro}
% \end{macro}
%
% \subsubsection{Chapters}
%
% \begin{macro}{\chapter}
%    A chapter should always start on a new page therefore we start by
%    calling |\clearpage| and setting the pagestyle for this page to
%    \pstyle{plain}.
%    \begin{macrocode}
%<*report|book>
\newcommand\chapter{\if@openright\cleardoublepage\else\clearpage\fi
                    \thispagestyle{plain}%
%    \end{macrocode}
%    Then we prevent floats from appearing at the top of this page
%    because it looks weird to see a floating object above a chapter
%    title.
%    \begin{macrocode}
                    \global\@topnum\z@
%    \end{macrocode}
%    Then we suppress the indentation of the first paragraph by
%    setting the switch |\@afterindent| to |false|. We use |\secdef|
%    to specify the macros to use for actually setting the chapter
%    title.
%    \begin{macrocode}
                    \@afterindentfalse
                    \secdef\@chapter\@schapter}
%    \end{macrocode}
%
% \begin{macro}{\@chapter}
%    This macro is called when we have a numbered chapter. When
%    \Lcount{secnumdepth} is larger than $-1$ and, in the book
%    class, |\@mainmatter| is true, we display the chapter
%    number. We also inform the user that a new chapter is about to be
%    typeset by writing a message to the terminal.
%    \begin{macrocode}
\def\@chapter[#1]#2{\ifnum \c@secnumdepth >\m@ne
%<book>                       \if@mainmatter
                         \refstepcounter{chapter}%
                         \typeout{\@chapapp\space\thechapter.}%
                         \addcontentsline{toc}{chapter}%
                                   {\protect\numberline{\thechapter}#1}%
%<*book>
                       \else
                         \addcontentsline{toc}{chapter}{#1}%
                       \fi
%</book>
                    \else
                      \addcontentsline{toc}{chapter}{#1}%
                    \fi
%    \end{macrocode}
%    After having written an entry to the table of contents we store
%    the (alternative) title of this chapter with |\chaptermark| and
%    add some white space to the lists of figures and tables.
%    \begin{macrocode}
                    \chaptermark{#1}%
                    \addtocontents{lof}{\protect\addvspace{10\p@}}%
                    \addtocontents{lot}{\protect\addvspace{10\p@}}%
%    \end{macrocode}
%    Then we call upon |\@makechapterhead| to format the actual
%    chapter title. We have to do this in a special way when we are in
%    twocolumn mode in order to have the chapter title use the entire
%    |\textwidth|. In one column mode we call |\@afterheading| which
%    takes care of suppressing the indentation.
%    \begin{macrocode}
                    \if@twocolumn
                      \@topnewpage[\@makechapterhead{#2}]%
                    \else
                      \@makechapterhead{#2}%
                      \@afterheading
                    \fi}
%    \end{macrocode}
%
% \begin{macro}{\@makechapterhead}
%    The macro above uses |\@makechapterhead|\meta{text} to format the
%    heading of the chapter.
%
%    We begin by leaving some white space. The we open a group in
%    which we have a paragraph indent of 0pt, and in which we have the
%    text set ragged right. We also reset the font.
% \changes{v1.3c}{1995/05/25}{replace \cs{reset@font} with
%    \cs{normalfont}}
%    \begin{macrocode}
\def\@makechapterhead#1{%
  \vspace*{50\p@}%
  {\parindent \z@ \raggedright \normalfont
%    \end{macrocode}
%    Then we check whether the number of the chapter has to be printed.
%    If so we leave some whitespace between the chapternumber and its
%    title.
% \changes{v1.2v}{1994/11/30}{Added a \cs{nobreak} to prevent a
%    pagebreak between the chapternumber and the chaptertitle}
% \changes{v1.3j}{1995/08/16}{replace braces by \cs{space}}
%    \begin{macrocode}
    \ifnum \c@secnumdepth >\m@ne
%<book>      \if@mainmatter
        \huge\bfseries \@chapapp\space \thechapter
        \par\nobreak
        \vskip 20\p@
%<book>      \fi
    \fi
%    \end{macrocode}
%    Now we set the title in a large bold font. We prevent a pagebreak
%    from occurring in the middle of or after the title. Finally we
%    leave some whitespace before the text begins.
% \changes{v1.2v}{1994/11/30}{Added \cs{interlinepenalty}\cs{@M} to
%    prevent a pagebreak in the middle of a title}
%    \begin{macrocode}
    \interlinepenalty\@M
    \Huge \bfseries #1\par\nobreak
    \vskip 40\p@
  }}
%    \end{macrocode}
% \end{macro}
% \end{macro}
%
% \begin{macro}{\@schapter}
%    This macro is called when we have an unnumbered chapter. It is
%    much simpler than |\@chapter| because it only needs to typeset
%    the chapter title.
%    \begin{macrocode}
\def\@schapter#1{\if@twocolumn
                   \@topnewpage[\@makeschapterhead{#1}]%
                 \else
                   \@makeschapterhead{#1}%
                   \@afterheading
                 \fi}
%    \end{macrocode}
%
% \begin{macro}{\@makeschapterhead}
%    The macro above uses |\@makeschapterhead|\meta{text}to format
%    the heading of the chapter. It is similar to |\@makechapterhead|
%    except that it never has to print a chapter number.
%
% \changes{v1.2v}{1994/11/30}{Added \cs{interlinepenalty}\cs{@M} to
%    prevent a pagebreak in the middle of a title}
% \changes{v1.3c}{1995/05/25}{replace \cs{reset@font} with
%    \cs{normalfont}}
%    \begin{macrocode}
\def\@makeschapterhead#1{%
  \vspace*{50\p@}%
  {\parindent \z@ \raggedright
    \normalfont
    \interlinepenalty\@M
    \Huge \bfseries  #1\par\nobreak
    \vskip 40\p@
  }}
%</report|book>
%    \end{macrocode}
% \end{macro}
% \end{macro}
% \end{macro}
%
%
% \subsubsection{Lower level headings}
%
%    These commands all make use of |\@startsection|.
% \begin{macro}{\section}
%    This gives a normal heading with white space above and below the
%    heading, the title set in |\Large\bfseries|, and no indentation
%    on the first paragraph.
% \changes{v1.3c}{1995/05/25}{replace \cs{reset@font} with
%    \cs{normalfont}}
%    \begin{macrocode}
\newcommand\section{\@startsection {section}{1}{\z@}%
                                   {-3.5ex \@plus -1ex \@minus -.2ex}%
                                   {2.3ex \@plus.2ex}%
                                   {\normalfont\Large\bfseries}}
%    \end{macrocode}
% \end{macro}
%
% \begin{macro}{\subsection}
%    This gives a normal heading with white space above and below the
%    heading, the title set in |\large\bfseries|, and no indentation
%    on the first paragraph.
%    \begin{macrocode}
\newcommand\subsection{\@startsection{subsection}{2}{\z@}%
                                     {-3.25ex\@plus -1ex \@minus -.2ex}%
                                     {1.5ex \@plus .2ex}%
                                     {\normalfont\large\bfseries}}
%    \end{macrocode}
% \end{macro}
%
% \begin{macro}{\subsubsection}
%    This gives a normal heading with white space above and below the
%    heading, the title set in |\normalsize\bfseries|, and no
%    indentation on the first paragraph.
%    \begin{macrocode}
\newcommand\subsubsection{\@startsection{subsubsection}{3}{\z@}%
                                     {-3.25ex\@plus -1ex \@minus -.2ex}%
                                     {1.5ex \@plus .2ex}%
                                     {\normalfont\normalsize\bfseries}}
%    \end{macrocode}
% \end{macro}
%
% \begin{macro}{\paragraph}
%    This gives a run-in heading with white space above and to the
%    right of the heading, the title set in |\normalsize\bfseries|.
%    \begin{macrocode}
\newcommand\paragraph{\@startsection{paragraph}{4}{\z@}%
                                    {3.25ex \@plus1ex \@minus.2ex}%
                                    {-1em}%
                                    {\normalfont\normalsize\bfseries}}
%    \end{macrocode}
% \end{macro}
%
% \begin{macro}{\subparagraph}
%    This gives an indented run-in heading with white space above and
%    to the right of the heading, the title set in
%    |\normalsize\bfseries|.
%    \begin{macrocode}
\newcommand\subparagraph{\@startsection{subparagraph}{5}{\parindent}%
                                       {3.25ex \@plus1ex \@minus .2ex}%
                                       {-1em}%
                                      {\normalfont\normalsize\bfseries}}
%    \end{macrocode}
% \end{macro}
%
% \subsection{Lists}
%
% \subsubsection{General List Parameters}
%
% The following commands are used to set the default values for the list
% environment's parameters. See the \LaTeX{} manual for an explanation
% of the meanings of the parameters.  Defaults for the list
% environment are set as follows.  First, |\rightmargin|,
% |\listparindent| and |\itemindent| are set to 0pt.  Then, for a Kth
% level list, the command |\@listK| is called, where `K' denotes `i',
% '`i', ... , `vi'.  (I.e., |\@listiii| is called for a third-level
% list.)  By convention, |\@listK| should set |\leftmargin| to
% |\leftmarginK|.
%
% \begin{macro}{\leftmargin}
% \begin{macro}{\leftmargini}
% \begin{macro}{\leftmarginii}
% \begin{macro}{\leftmarginiii}
% \begin{macro}{\leftmarginiv}
% \begin{macro}{\leftmarginv}
% \begin{macro}{\leftmarginvi}
% \changes{v1.0m}{1994/01/12}{Use em instead of pt to remain
%    compatible with old styles}
% \changes{v1.3q}{1995/12/20}{Temporary(?) fix: revert to setting
%    \cs{leftmargin} at outer level}
%
% When we are in two column mode some of the margins are set somewhat
% smaller.
%    \begin{macrocode}
\if@twocolumn
  \setlength\leftmargini  {2em}
\else
  \setlength\leftmargini  {2.5em}
\fi
%    \end{macrocode}
%    Until the whole of the parameter setting in these files is
%    rationalised, we need to set the value of |\leftmargin| at this
%    outer level.
%    \begin{macrocode}
\leftmargin  \leftmargini
%    \end{macrocode}
%    The following three are calculated so  that they are larger than
%    the sum of |\labelsep| and the width of the default labels (which
%    are `(m)', `vii.' and `M.').
%    \begin{macrocode}
\setlength\leftmarginii  {2.2em}
\setlength\leftmarginiii {1.87em}
\setlength\leftmarginiv  {1.7em}
\if@twocolumn
  \setlength\leftmarginv  {.5em}
  \setlength\leftmarginvi {.5em}
\else
  \setlength\leftmarginv  {1em}
  \setlength\leftmarginvi {1em}
\fi
%    \end{macrocode}
% \end{macro}
% \end{macro}
% \end{macro}
% \end{macro}
% \end{macro}
% \end{macro}
% \end{macro}
%
% \begin{macro}{\labelsep}
% \begin{macro}{\labelwidth}
% \changes{v1.0m}{1994/01/12}{Use em instead of pt to remain
%    compatible with old styles}
%    |\labelsep| is the distance between the label and the text of an
%    item; |\labelwidth| is the width of the label.
%    \begin{macrocode}
\setlength  \labelsep  {.5em}
\setlength  \labelwidth{\leftmargini}
\addtolength\labelwidth{-\labelsep}
%    \end{macrocode}
% \end{macro}
% \end{macro}
%
% \begin{macro}{\partopsep}
%    When the user leaves a blank line before the environment an extra
%    vertical space of |\partopsep| is inserted, in addition to
%    |\parskip| and |\topsep|.
% \changes{v1.0m}{1994/01/12}{\cs{partopsep} should be different,
%    depending on the pointsize}
%    \begin{macrocode}
%</article|report|book>
%<10pt>\setlength\partopsep{2\p@ \@plus 1\p@ \@minus 1\p@}
%<11pt>\setlength\partopsep{3\p@ \@plus 1\p@ \@minus 1\p@}
%<12pt>\setlength\partopsep{3\p@ \@plus 2\p@ \@minus 2\p@}
%    \end{macrocode}
% \end{macro}
%
% \begin{macro}{\@beginparpenalty}
% \begin{macro}{\@endparpenalty}
%    These penalties are inserted before and after a list or paragraph
%    environment. They are set to a bonus value to encourage page
%    breaking at these points.
% \begin{macro}{\@itempenalty}
%    This penalty is inserted between list items.
%    \begin{macrocode}
%<*article|report|book>
\@beginparpenalty -\@lowpenalty
\@endparpenalty   -\@lowpenalty
\@itempenalty     -\@lowpenalty
%</article|report|book>
%    \end{macrocode}
% \end{macro}
% \end{macro}
% \end{macro}
%
% \begin{macro}{\@listi}
% \begin{macro}{\@listI}
% |\@listi| defines the values of
% |\leftmargin|, |\parsep|, |\topsep|, |\itemsep|, etc.\ for the
% lists that appear on top-level. Its definition is modified by the
% font-size commands (eg within |\small| the list parameters get
% ``smaller'' values).
%
% For this reason \@listI is defined to hold a saved copy of \@listi
% so that |\normalsize| can switch all parameters back.
%
%    \begin{macrocode}
%<*10pt|11pt|12pt>
\def\@listi{\leftmargin\leftmargini
%<*10pt>
            \parsep 4\p@ \@plus2\p@ \@minus\p@
            \topsep 8\p@ \@plus2\p@ \@minus4\p@
            \itemsep4\p@ \@plus2\p@ \@minus\p@}
%</10pt>
%<*11pt>
            \parsep 4.5\p@ \@plus2\p@ \@minus\p@
            \topsep 9\p@   \@plus3\p@ \@minus5\p@
            \itemsep4.5\p@ \@plus2\p@ \@minus\p@}
%</11pt>
%<*12pt>
            \parsep 5\p@  \@plus2.5\p@ \@minus\p@
            \topsep 10\p@ \@plus4\p@   \@minus6\p@
            \itemsep5\p@  \@plus2.5\p@ \@minus\p@}
%</12pt>
\let\@listI\@listi
%    \end{macrocode}
%    We initialise the parameters although strictly speaking that
%    is not necessary.
%    \begin{macrocode}
\@listi
%    \end{macrocode}
% \end{macro}
% \end{macro}
%
% \begin{macro}{\@listii}
% \begin{macro}{\@listiii}
% \begin{macro}{\@listiv}
% \begin{macro}{\@listv}
% \begin{macro}{\@listvi}
%    Here are the same macros for the higher level lists. Note that
%    they don't have saved versions and are not modified by the font
%    size commands. In other words this class assumes that nested
%    lists only appear in |\normalsize|, i.e.\ the main document size.
%    \begin{macrocode}
\def\@listii {\leftmargin\leftmarginii
              \labelwidth\leftmarginii
              \advance\labelwidth-\labelsep
%<*10pt>
              \topsep    4\p@ \@plus2\p@ \@minus\p@
              \parsep    2\p@ \@plus\p@  \@minus\p@
%</10pt>
%<*11pt>
              \topsep    4.5\p@ \@plus2\p@ \@minus\p@
              \parsep    2\p@   \@plus\p@  \@minus\p@
%</11pt>
%<*12pt>
              \topsep    5\p@   \@plus2.5\p@ \@minus\p@
              \parsep    2.5\p@ \@plus\p@    \@minus\p@
%</12pt>
              \itemsep   \parsep}
\def\@listiii{\leftmargin\leftmarginiii
              \labelwidth\leftmarginiii
              \advance\labelwidth-\labelsep
%<10pt>              \topsep    2\p@ \@plus\p@\@minus\p@
%<11pt>              \topsep    2\p@ \@plus\p@\@minus\p@
%<12pt>              \topsep    2.5\p@\@plus\p@\@minus\p@
              \parsep    \z@
              \partopsep \p@ \@plus\z@ \@minus\p@
              \itemsep   \topsep}
\def\@listiv {\leftmargin\leftmarginiv
              \labelwidth\leftmarginiv
              \advance\labelwidth-\labelsep}
\def\@listv  {\leftmargin\leftmarginv
              \labelwidth\leftmarginv
              \advance\labelwidth-\labelsep}
\def\@listvi {\leftmargin\leftmarginvi
              \labelwidth\leftmarginvi
              \advance\labelwidth-\labelsep}
%</10pt|11pt|12pt>
%    \end{macrocode}
% \end{macro}
% \end{macro}
% \end{macro}
% \end{macro}
% \end{macro}
%
% \subsubsection{Enumerate}
%
%    The enumerate environment uses  four counters: \Lcount{enumi},
%    \Lcount{enumii}, \Lcount{enumiii} and \Lcount{enumiv}, where
%    \Lcount{enumN} controls the numbering of the Nth level
%    enumeration.
%
% \begin{macro}{\theenumi}
% \begin{macro}{\theenumii}
% \begin{macro}{\theenumiii}
% \begin{macro}{\theenumiv}
%    The counters are already defined in \file{latex.dtx}, but their
%    representation is changed here.
%
%    \begin{macrocode}
%<*article|report|book>
\renewcommand\theenumi{\@arabic\c@enumi}
\renewcommand\theenumii{\@alph\c@enumii}
\renewcommand\theenumiii{\@roman\c@enumiii}
\renewcommand\theenumiv{\@Alph\c@enumiv}
%    \end{macrocode}
% \end{macro}
% \end{macro}
% \end{macro}
% \end{macro}
%
% \begin{macro}{\labelenumi}
% \begin{macro}{\labelenumii}
% \begin{macro}{\labelenumiii}
% \begin{macro}{\labelenumiv}
%    The label for each item is generated by the commands\\
%    |\labelenumi| \ldots\ |\labelenumiv|.
%    \begin{macrocode}
\newcommand\labelenumi{\theenumi.}
\newcommand\labelenumii{(\theenumii)}
\newcommand\labelenumiii{\theenumiii.}
\newcommand\labelenumiv{\theenumiv.}
%    \end{macrocode}
% \end{macro}
% \end{macro}
% \end{macro}
% \end{macro}
%
% \begin{macro}{\p@enumii}
% \begin{macro}{\p@enumiii}
% \begin{macro}{\p@enumiv}
%    The expansion of |\p@enumN||\theenumN| defines the output of a
%    |\ref| command when referencing an item of the Nth level of an
%    enumerated list.
%    \begin{macrocode}
\renewcommand\p@enumii{\theenumi}
\renewcommand\p@enumiii{\theenumi(\theenumii)}
\renewcommand\p@enumiv{\p@enumiii\theenumiii}
%    \end{macrocode}
% \end{macro}
% \end{macro}
% \end{macro}
%
% \subsubsection{Itemize}
%
% \begin{macro}{\labelitemi}
% \begin{macro}{\labelitemii}
% \changes{v1.2k}{1994/05/06}{Inserted \cs{normalfont}}
% \changes{v1.3s}{1996/08/24}{Replaced -{}- with \cs{textendash}}
% \changes{v1.3u}{1996/10/31}{Changed to \cs{textbullet},
%                 \cs{textasteriskcentered} and \cs{textperiodcentered}}
% \begin{macro}{\labelitemiii}
% \begin{macro}{\labelitemiv}
%    Itemization is controlled by four commands: |\labelitemi|,
%    |\labelitemii|, |\labelitemiii|, and |\labelitemiv|, which define
%    the labels of thevarious itemization levels: the symbols used are
%    bullet, bold en-dash, centered asterisk and centred dot.
%
%    \begin{macrocode}
\newcommand\labelitemi{\textbullet}
\newcommand\labelitemii{\normalfont\bfseries \textendash}
\newcommand\labelitemiii{\textasteriskcentered}
\newcommand\labelitemiv{\textperiodcentered}
%    \end{macrocode}
% \end{macro}
% \end{macro}
% \end{macro}
% \end{macro}
%
% \subsubsection{Description}
%
% \begin{environment}{description}
%    The description environment is defined here -- while the itemize
%    and enumerate environments are defined in \file{latex.dtx}.
%
%    \begin{macrocode}
\newenvironment{description}
               {\list{}{\labelwidth\z@ \itemindent-\leftmargin
                        \let\makelabel\descriptionlabel}}
               {\endlist}
%    \end{macrocode}
% \end{environment}
%
% \begin{macro}{\descriptionlabel}
%    To change the formatting of the label, you must redefine
%    |\descriptionlabel|.
%
% \changes{v1.2k}{1994/05/06}{Inserted \cs{normalfont}}
% \changes{v1.2y}{1995/01/31}{made command short}
%    \begin{macrocode}
\newcommand*\descriptionlabel[1]{\hspace\labelsep
                                \normalfont\bfseries #1}
%    \end{macrocode}
% \end{macro}
%
% \subsection{Defining new environments}
%
% \subsubsection{Abstract}
%
% \begin{environment}{abstract}
%    When we are producing a separate titlepage we also put the
%    abstract on a page of its own. It will be centred vertically on
%    the page.
%
%    Note that this environment is not defined for books.
% \changes{v1.3e}{1995/06/19}{Added setting of \cs{@endparpenalty}
%         to avoid page break after abstract heading.}
%    \begin{macrocode}
% \changes{v1.3m}{1995/10/23}{Added setting of \cs{beginparpenalty} to
%    discourage page break before abstract heading.}
%<*article|report>
\if@titlepage
  \newenvironment{abstract}{%
      \titlepage
      \null\vfil
      \@beginparpenalty\@lowpenalty
      \begin{center}%
        \bfseries \abstractname
        \@endparpenalty\@M
      \end{center}}%
     {\par\vfil\null\endtitlepage}
%    \end{macrocode}
%    When we are not making a separate titlepage --the default for the
%    article document class-- we have to check if we are in twocolumn
%    mode. In that case the abstract is as a |\section*|, otherwise
%    the quotation environment is used to typeset the abstract.
%    \begin{macrocode}
\else
  \newenvironment{abstract}{%
      \if@twocolumn
        \section*{\abstractname}%
      \else
        \small
        \begin{center}%
          {\bfseries \abstractname\vspace{-.5em}\vspace{\z@}}%
        \end{center}%
        \quotation
      \fi}
      {\if@twocolumn\else\endquotation\fi}
\fi
%</article|report>
%    \end{macrocode}
% \end{environment}
%
% \subsubsection{Verse}
%
% \begin{environment}{verse}
%   The verse environment is defined by making clever use of the
%   list environment's parameters.  The user types |\\| to end a line.
%   This is implemented by |\let|'ing |\\| equal |\@centercr|.
%
% \changes{v1.3j}{1995/08/16}{stop \cs{item} scanning for [ with
%    \cs{relax}}
%    \begin{macrocode}
\newenvironment{verse}
               {\let\\\@centercr
                \list{}{\itemsep      \z@
                        \itemindent   -1.5em%
                        \listparindent\itemindent
                        \rightmargin  \leftmargin
                        \advance\leftmargin 1.5em}%
                \item\relax}
               {\endlist}
%    \end{macrocode}
% \end{environment}
%
% \subsubsection{Quotation}
%
% \begin{environment}{quotation}
%   The quotation environment is also defined by making clever use of
%   the list environment's parameters. The lines in the environment
%   are set smaller than |\textwidth|. The first line of a paragraph
%   inside this environment is indented.
%
% \changes{v1.3j}{1995/08/16}{stop \cs{item} scanning for [ with
%    \cs{relax}}
%    \begin{macrocode}
\newenvironment{quotation}
               {\list{}{\listparindent 1.5em%
                        \itemindent    \listparindent
                        \rightmargin   \leftmargin
                        \parsep        \z@ \@plus\p@}%
                \item\relax}
               {\endlist}
%    \end{macrocode}
% \end{environment}
%
% \subsubsection{Quote}
%
% \begin{environment}{quote}
%   The quote environment is like the quotation environment except
%   that paragraphs are not indented.
%
% \changes{v1.3j}{1995/08/16}{stop \cs{item} scanning for [ with
%    \cs{relax}}
%    \begin{macrocode}
\newenvironment{quote}
               {\list{}{\rightmargin\leftmargin}%
                \item\relax}
               {\endlist}
%    \end{macrocode}
% \end{environment}
%
% \subsubsection{Theorem}
%
%    This document class does not define it's own theorem environments,
%    the defaults, supplied by \file{latex.dtx} are available.
%
% \subsubsection{Titlepage}
%
% \begin{environment}{titlepage}
%  In the normal environments, the titlepage environment does nothing
%  but start and end a page, and inhibit page numbers.  In the report
%  style, it also resets the page number to one, and then sets it
%  back to one at the end.  In compatibility mode, it sets the
%  page number to zero. This is incorrect since it results in using
%  the page parameters for a right-hand page but it is the way it was.
%  In two-column style, it still makes a
%  one-column page.
%
% \changes{v1.0g}{1993/12/09}{Moved the setting of
%    \cs{@restonecolfalse}}
% \changes{v1.2c}{1994/03/17}{page :!= 0 only in compatibility mode
%    (LL)}
% \changes{v1.2d}{1994/04/11}{Moved \cs{cleardoublepage} inside
%    definition of titlepage environment}
% \changes{v1.3i}{1995/08/08}{New implementation with support for
%      twoside and openright option}
%
%    First we do give the definition for compatibility mode.
%    \begin{macrocode}
\if@compatibility
\newenvironment{titlepage}
    {%
%<book>      \cleardoublepage
      \if@twocolumn
        \@restonecoltrue\onecolumn
      \else
        \@restonecolfalse\newpage
      \fi
      \thispagestyle{empty}%
      \setcounter{page}\z@
    }%
    {\if@restonecol\twocolumn \else \newpage \fi
    }
%    \end{macrocode}
%
%    And here is the one for native \LaTeXe{}.
%    \begin{macrocode}
\else
\newenvironment{titlepage}
    {%
%<book>      \cleardoublepage
      \if@twocolumn
        \@restonecoltrue\onecolumn
      \else
        \@restonecolfalse\newpage
      \fi
      \thispagestyle{empty}%
      \setcounter{page}\@ne
    }%
    {\if@restonecol\twocolumn \else \newpage \fi
%    \end{macrocode}
%    If we are not in two-side mode the first page after the title page
%    should also get page number 1.
%    \begin{macrocode}
     \if@twoside\else
        \setcounter{page}\@ne
     \fi
    }
\fi
%    \end{macrocode}
% \end{environment}
%
% \subsubsection{Appendix}
%
% \begin{macro}{\appendix}
%
%    The |\appendix| command is not really an environment, it is a
%    macro that makes some changes in the way things are done.
%
%    In the article document class the |\appendix| command must do the
%    following:
%    \begin{itemize}
%    \item reset the section and subsection counters to zero,
%    \item redefine |\thesection| to produce alphabetic appendix
%        numbers. This redefinition is done globally to ensure that it
%        survives even if |\appendix| is issued within an environment such
%        as \texttt{multicols}.
%    \end{itemize}
%
% \changes{1.3z}{1998/09/19}{Redefine \cs{thesection} globally (pr/2862)}
%    \begin{macrocode}
%<*article>
\newcommand\appendix{\par
  \setcounter{section}{0}%
  \setcounter{subsection}{0}%
  \gdef\thesection{\@Alph\c@section}}
%</article>
%    \end{macrocode}
%
%    In the report and book document classes the |\appendix| command
%    must do the following:
%    \begin{itemize}
%    \item reset the chapter and section counters to zero,
%    \item set |\@chapapp| to |\appendixname| (for messages),
%    \item redefine the chapter counter to produce appendix numbers,
%    \item possibly redefine the |\chapter| command if appendix titles
%        and headings are to look different from chapter titles and
%        headings. This redefinition is done globally to ensure that it
%        survives even if |\appendix| is issued within an environment such
%        as \texttt{multicols}.
%    \end{itemize}
%
% \changes{1.3z}{1998/09/19}{Redefine \cs{thechapter} and
%                            \cs{@chapapp} globally (pr/2862)}
%    \begin{macrocode}
%<*report|book>
\newcommand\appendix{\par
  \setcounter{chapter}{0}%
  \setcounter{section}{0}%
  \gdef\@chapapp{\appendixname}%
  \gdef\thechapter{\@Alph\c@chapter}}
%</report|book>
%    \end{macrocode}
% \end{macro}
%
% \subsection{Setting parameters for existing environments}
%
% \subsubsection{Array and tabular}
%
% \begin{macro}{\arraycolsep}
%    The columns in an array environment are separated by
%    2|\arraycolsep|.
%    \begin{macrocode}
\setlength\arraycolsep{5\p@}
%    \end{macrocode}
% \end{macro}
%
% \begin{macro}{\tabcolsep}
%    The columns in an tabular environment are separated by
%    2|\tabcolsep|.
%    \begin{macrocode}
\setlength\tabcolsep{6\p@}
%    \end{macrocode}
% \end{macro}
%
% \begin{macro}{\arrayrulewidth}
%    The width of rules in the array and tabular environments is given
%    by\\ |\arrayrulewidth|.
%    \begin{macrocode}
\setlength\arrayrulewidth{.4\p@}
%    \end{macrocode}
% \end{macro}
%
% \begin{macro}{\doublerulesep}
%    The space between adjacent rules in the array and tabular
%    environments is given by |\doublerulesep|.
%    \begin{macrocode}
\setlength\doublerulesep{2\p@}
%    \end{macrocode}
% \end{macro}
%
% \subsubsection{Tabbing}
%
% \begin{macro}{\tabbingsep}
%    This controls the space that the |\'| command puts in. (See
%    \LaTeX{} manual for an explanation.)
%    \begin{macrocode}
\setlength\tabbingsep{\labelsep}
%    \end{macrocode}
% \end{macro}
%
% \subsubsection{Minipage}
%
% \begin{macro}{\@minipagerestore}
%    The macro |\@minipagerestore| is called upon entry to a minipage
%    environment to set up things that are to be handled differently
%    inside a minipage environment. In the current styles, it does
%    nothing.
% \end{macro}
%
% \begin{macro}{\@mpfootins}
%    Minipages have their own footnotes; |\skip||\@mpfootins| plays
%    same r\^ole for footnotes in a minipage as |\skip||\footins| does
%    for ordinary footnotes.
%
%    \begin{macrocode}
\skip\@mpfootins = \skip\footins
%    \end{macrocode}
% \end{macro}
%
% \subsubsection{Framed boxes}
%
% \begin{macro}{\fboxsep}
%    The space left by |\fbox| and |\framebox| between the box and the
%    text in it.
% \begin{macro}{\fboxrule}
%    The width of the rules in the box made by |\fbox| and |\framebox|.
%    \begin{macrocode}
\setlength\fboxsep{3\p@}
\setlength\fboxrule{.4\p@}
%    \end{macrocode}
% \end{macro}
% \end{macro}
%
% \subsubsection{Equation and eqnarray}
%
% \begin{macro}{\theequation}
%    When within chapters, the equation counter will be reset at
%    the beginning of a new chapter and the equation number will
%    be prefixed by the chapter number.
% \changes{v1.3u}{1996/10/31}{Added test for non-zero chapter number}
%
%    This code  must follow the |\chapter| definition or, more exactly,
%    the definition of the chapter counter.
%    \begin{macrocode}
%<article>\renewcommand \theequation {\@arabic\c@equation}
%<*report|book>
\@addtoreset {equation}{chapter}
\renewcommand\theequation
  {\ifnum \c@chapter>\z@ \thechapter.\fi \@arabic\c@equation}
%</report|book>
%    \end{macrocode}
% \end{macro}
%
% \begin{macro}{\jot}
%    |\jot| is the extra space added between lines of an eqnarray
%    environment. The default value is used.
%    \begin{macrocode}
% \setlength\jot{3pt}
%    \end{macrocode}
% \end{macro}
%
% \begin{macro}{\@eqnnum}
%    The macro |\@eqnnum| defines how equation numbers are to appear in
%    equations. Again the default is used.
%
%    \begin{macrocode}
% \def\@eqnnum{(\theequation)}
%    \end{macrocode}
% \end{macro}
%
% \subsection{Floating objects}
%
%    The file \file{latex.dtx} only defines a number of tools with
%    which floating objects can be defined. This is done in the
%    document class. It needs to define the following macros for each
%    floating object of type \texttt{TYPE} (e.g., \texttt{TYPE} =
%    figure).
%
%    \begin{description}
%    \item[\texttt{\bslash fps@TYPE}]
%        The default placement specifier for floats of type
%        \texttt{TYPE}.
%
%    \item[\texttt{\bslash ftype@TYPE}]
%        The type number for floats of type \texttt{TYPE}.  Each
%        \texttt{TYPE} has associated a unique positive \texttt
%        {TYPE} number, which is a power of two.  E.g., figures might
%        have type number 1, tables type number 2, programs type
%        number 4, etc.
%
%    \item[\texttt{\bslash ext@TYPE}]
%        The file extension indicating the file on which the contents
%        list for float type \texttt{TYPE} is stored.  For example,
%        |\ext@figure| = `lof'.
%
%    \item[\texttt{\bslash fnum@TYPE}]
%        A macro to generate the figure number for a caption. For
%        example, |\fnum@TYPE| == `Figure |\thefigure|'.
%
%    \item[\texttt{\bslash @makecaption}{\meta{num}}{\meta{text}}]
%        A macro to make a caption, with \meta{num} the value produced
%        by |\fnum@...| and \meta{text} the text of the caption. It
%        can assume it's in a |\parbox| of the appropriate width.
%        This will be used for \emph{all} floating objects.
%
%    \end{description}
%
%    The actual environment that implements a floating object such as
%    a figure is defined using the macros |\@float| and |\end@float|,
%    which are defined in \file{latex.dtx}.
%
%    An environment that implements a single column floating object is
%    started with |\@float{|\texttt{TYPE}|}[|\meta{placement}|]| of type
%    \texttt{TYPE} with \meta{placement} as the placement specifier.
%    The default value of \meta{PLACEMENT} is defined by |\fps@TYPE|.
%
%    The environment is ended by |\end@float|.  E.g., |\figure| ==
%    |\@float|{figure}, |\endfigure| == |\end@float|.
%
% \subsubsection{Figure}
%
%    Here is the implementation of the figure environment.
%
% \begin{macro}{\c@figure}
%    First we have to allocate a counter to number the figures.
%
%    In the report and book document classes figures within chapters are
%    numbered per chapter.
% \changes{v1.3u}{1996/10/31}{Added test for non-zero chapter number}
%    \begin{macrocode}
%<*article>
\newcounter{figure}
\renewcommand \thefigure {\@arabic\c@figure}
%</article>
%<*report|book>
\newcounter{figure}[chapter]
\renewcommand \thefigure
     {\ifnum \c@chapter>\z@ \thechapter.\fi \@arabic\c@figure}
%</report|book>
%    \end{macrocode}
% \end{macro}
%
% \begin{macro}{\fps@figure}
% \begin{macro}{\ftype@figure}
% \begin{macro}{\ext@figure}
% \begin{macro}{\num@figure}
%    Here are the parameters for the floating objects of type `figure'.
% \changes{v1.4e}{2001/05/24}{Replaced tilde with \cs{nobreakspace}
%                             (pr/3310)}
%    \begin{macrocode}
\def\fps@figure{tbp}
\def\ftype@figure{1}
\def\ext@figure{lof}
\def\fnum@figure{\figurename\nobreakspace\thefigure}
%    \end{macrocode}
% \end{macro}
% \end{macro}
% \end{macro}
% \end{macro}
%
% \begin{environment}{figure}
% \begin{environment}{figure*}
%    And the definition of the actual environment. The form with the
%    |*| is used for double column figures.
%    \begin{macrocode}
\newenvironment{figure}
               {\@float{figure}}
               {\end@float}
\newenvironment{figure*}
               {\@dblfloat{figure}}
               {\end@dblfloat}
%    \end{macrocode}
% \end{environment}
% \end{environment}
%
% \subsubsection{Table}
%
%    Here is the implementation of the table environment. It is very
%    much the same as the figure environment.
%
% \begin{macro}{\c@table}
%    First we have to allocate a counter to number the tables.
%
%    In the report and book document classes tables within chapters are
%    numbered per chapter.
% \changes{v1.3u}{1996/10/31}{Added test for non-zero chapter number}
%    \begin{macrocode}
%<*article>
\newcounter{table}
\renewcommand\thetable{\@arabic\c@table}
%</article>
%<*report|book>
\newcounter{table}[chapter]
\renewcommand \thetable
     {\ifnum \c@chapter>\z@ \thechapter.\fi \@arabic\c@table}
%</report|book>
%    \end{macrocode}
% \end{macro}
%
% \begin{macro}{\fps@table}
% \begin{macro}{\ftype@table}
% \begin{macro}{\ext@table}
% \begin{macro}{\num@table}
%    Here are the parameters for the floating objects of type `table'.
% \changes{v1.4e}{2001/05/24}{Replaced tilde with \cs{nobreakspace}
%                             (pr/3310)}
%    \begin{macrocode}
\def\fps@table{tbp}
\def\ftype@table{2}
\def\ext@table{lot}
\def\fnum@table{\tablename\nobreakspace\thetable}
%    \end{macrocode}
% \end{macro}
% \end{macro}
% \end{macro}
% \end{macro}
%
% \begin{environment}{table}
% \begin{environment}{table*}
%    And the definition of the actual environment. The form with the
%    |*| is used for double column tables.
%    \begin{macrocode}
\newenvironment{table}
               {\@float{table}}
               {\end@float}
\newenvironment{table*}
               {\@dblfloat{table}}
               {\end@dblfloat}
%    \end{macrocode}
% \end{environment}
% \end{environment}
%
% \subsubsection{Captions}
%
% \begin{macro}{\@makecaption}
%    The |\caption| command calls |\@makecaption| to format the
%    caption of floating objects. It gets two arguments,
%    \meta{number}, the number of the floating object and \meta{text},
%    the text of the caption. Usually \meta{number} contains a string
%    such as `Figure 3.2'. The macro can assume it is called inside a
%    |\parbox| of right width, with |\normalsize|.
%
% \begin{macro}{\abovecaptionskip}
% \begin{macro}{\belowcaptionskip}
%    These lengths contain the amount of white space to leave above
%    and below the caption.
%    \begin{macrocode}
\newlength\abovecaptionskip
\newlength\belowcaptionskip
\setlength\abovecaptionskip{10\p@}
\setlength\belowcaptionskip{0\p@}
%    \end{macrocode}
% \end{macro}
% \end{macro}
%
%    The definition of this macro is |\long| in order to allow more
%    then one paragraph in a caption.
%    \begin{macrocode}
\long\def\@makecaption#1#2{%
  \vskip\abovecaptionskip
%    \end{macrocode}
%    We want to see if the caption fits on one line on the page,
%    therefore we first typeset it in a temporary box.
% \changes{v1.2q}{1994/05/29}{Use \cs{sbox}\cs{@tempboxa} instead of
%    \cs{setbox}\cs{@tempboxa}\cs{hbox} to make this colour safe}
%    \begin{macrocode}
  \sbox\@tempboxa{#1: #2}%
%    \end{macrocode}
%    We can the measure its width. It that is larger than the current
%    |\hsize| we typeset the caption as an ordinary paragraph.
%    \begin{macrocode}
  \ifdim \wd\@tempboxa >\hsize
    #1: #2\par
%    \end{macrocode}
%    If the caption fits, we center it. Because this uses an |\hbox|
%    directly in vertical mode, it does not execute the |\everypar|
%    tokens; the only thing that could be needed here is resetting the
%    `minipage flag' so we do this explicitly.
% \changes{v1.2x}{1994/12/09}{Due to a change in the way floats are
%    handled we need to set the \cs{if@minipage} switch to false}
%    \begin{macrocode}
  \else
    \global \@minipagefalse
    \hb@xt@\hsize{\hfil\box\@tempboxa\hfil}%
  \fi
  \vskip\belowcaptionskip}
%    \end{macrocode}
% \end{macro}
%
% \subsection{Font changing}
%
%    Here we supply the declarative font changing commands that were
%    common in \LaTeX\ version 2.09 and earlier. These commands work
%    in text mode \emph{and} in math mode. They are provided for
%    compatibility, but one should start using the |\text...| and
%    |\math...| commands instead. These commands are defined using
%    |\DeclareTextFontCommand|, a command with three arguments: the
%    user command to be defined; \LaTeX\ commands to execute in text
%    mode and \LaTeX\ commands to execute in math mode.
%
% \changes{v1.0g}{1993/12/12}{Distinguished between compatibility and
%    `normal' mode for the font changing commands.}
% \changes{v1.0h}{1993/12/18}{These are now defined in the kernel, so
%    use \cs{@renewfontswitch}.  Compatibility mode defines
%    \cs{@renewfontswitch} to do nothing, so we don't need to check
%    for compatibility mode any more.}
% \changes{v1.0j}{1993/12/20}{Added \cs{normalfont} back in the
%    definitions of \cs{rm} etc. as this should be the default
%    behaviour}
% \changes{v1.2e}{1994/04/14}{\cs{@renewfontswitch} has become
%    \cs{DeclareOldFontCommand}}
%
%  \begin{macro}{\rm}
% \changes{v1.0f}{1993/12/08}{Macro added}
%  \begin{macro}{\tt}
% \changes{v1.0f}{1993/12/08}{Macro added}
%  \begin{macro}{\sf}
% \changes{v1.0f}{1993/12/08}{Macro added}
%
%    The commands to change the family. When in compatibility mode we
%    select the `default' font first, to get \LaTeX2.09 behaviour.
%    \begin{macrocode}
\DeclareOldFontCommand{\rm}{\normalfont\rmfamily}{\mathrm}
\DeclareOldFontCommand{\sf}{\normalfont\sffamily}{\mathsf}
\DeclareOldFontCommand{\tt}{\normalfont\ttfamily}{\mathtt}
%    \end{macrocode}
%  \end{macro}
%  \end{macro}
%  \end{macro}
%
%  \begin{macro}{\bf}
% \changes{v1.0f}{1993/12/08}{Macro added}
%    The command to change to the bold series. One should use
%    |\mdseries| to explicitly switch back to medium series.
%    \begin{macrocode}
\DeclareOldFontCommand{\bf}{\normalfont\bfseries}{\mathbf}
%    \end{macrocode}
%  \end{macro}
%
%  \begin{macro}{\sl}
% \changes{v1.0f}{1993/12/08}{Macro added}
% \changes{v1.2g}{1994/04/24}{Added warning if used in math mode}
%  \begin{macro}{\it}
% \changes{v1.0f}{1993/12/08}{Macro added}
%  \begin{macro}{\sc}
% \changes{v1.0f}{1993/12/08}{Macro added}
% \changes{v1.2g}{1994/04/24}{Added warning if used in math mode}
%
%    And the commands to change the shape of the font. The slanted and
%    small caps shapes are not available by default as math alphabets,
%    so those changes do nothing in math mode. However, we do warn the
%    user that the selection will not have any effect.One should use
%    |\upshape| to explicitly change back to the upright shape.
%    \begin{macrocode}
\DeclareOldFontCommand{\it}{\normalfont\itshape}{\mathit}
\DeclareOldFontCommand{\sl}{\normalfont\slshape}{\@nomath\sl}
\DeclareOldFontCommand{\sc}{\normalfont\scshape}{\@nomath\sc}
%    \end{macrocode}
%  \end{macro}
%  \end{macro}
%  \end{macro}
%
% \begin{macro}{\cal}
% \changes{v1.0g}{1993/12/12}{Macro added}
% \begin{macro}{\mit}
% \changes{v1.0g}{1993/12/12}{Macro added}
%
%    The commands |\cal| and |\mit| should only be used in math mode,
%    outside math mode they have no effect. Currently the New Font
%    Selection Scheme defines these commands to generate warning
%    messages. Therefore we have to define them `by hand'.
% \changes{v1.2w}{1994/12/01}{Now define \cs{cal} and \cs{mit} using
%    \cs{DeclareRobustCommand*}}
% \changes{v1.3j}{1995/08/16}{Remove surplus braces}
%    \begin{macrocode}
\DeclareRobustCommand*\cal{\@fontswitch\relax\mathcal}
\DeclareRobustCommand*\mit{\@fontswitch\relax\mathnormal}
%    \end{macrocode}
%  \end{macro}
%  \end{macro}
%
% \section{Cross Referencing}
% \subsection{Table of Contents, etc.}
%
%     A |\section| command writes a
%     |\contentsline{section}{|\meta{title}|}{|\meta{page}|}| command
%     on the \file{.toc} file, where \meta{title} contains the
%     contents of the entry and \meta{page} is the page number. If
%     sections are being numbered, then \meta{title} will be of the
%     form |\numberline{|\meta{num}|}{|\meta{heading}|}| where
%     \meta{num} is the number produced by |\thesection|.  Other
%     sectioning commands work similarly.
%
%     A |\caption| command in a `figure' environment writes
%
%     |\contentsline{figure}{\numberline{|\meta{num}|}{|%
%                    \meta{caption}|}}{|\meta{page}|}|
%
%     on the .\file{lof} file, where \meta{num} is the number produced
%     by |\thefigure| and \meta{caption} is the figure caption.  It
%     works similarly for a `table' environment.
%
%    The command |\contentsline{|\meta{name}|}| expands to
%    |\l@|\meta{name}.  So, to specify the table of contents, we must
%    define |\l@chapter|, |\l@section|, |\l@subsection|, ... ; to
%    specify the list of figures, we must define |\l@figure|; and so
%    on.  Most of these can be defined with the |\@dottedtocline|
%    command, which works as follows.
%
%    |\@dottedtocline{|\meta{level}|}{|\meta{indent}|}{|^^A
%                      \meta{numwidth}|}{|^^A
%                      \meta{title}|}{|\meta{page}|}|
%
%    \begin{description}
%    \item[\meta{level}] An entry is produced only if\meta{ level}
%        $<=$ value of the \Lcount{tocdepth} counter.  Note,
%        |\chapter| is level 0, |\section| is level 1, etc.
%    \item[\meta{indent}] The indentation from the outer left margin
%        of the start   of the contents line.
%    \item[\meta{numwidth}] The width of a box in which the section
%        number is to go, if \meta{title} includes a |\numberline|
%        command.
%    \end{description}
%
% \begin{macro}{\@pnumwidth}
% \begin{macro}{\@tocrmarg}
% \begin{macro}{\@dotsep}
%    This command uses the following three parameters, which are set
%    with a |\newcommand| (so em's can be used to make them depend upon
%    the font).
%    \begin{description}
%    \item[\texttt{\bslash @pnumwidth}] The width of a box in which the
%        page number is put.
% \changes{v1.2v}{1994/10/29}{Changed documentation from $!>$ or $!=$ to
%    $\ge$}
%    \item[\texttt{\bslash @tocrmarg}] The right margin for multiple
%        line entries.  One wants |\@tocrmarg| $\ge$ |\@pnumwidth|
%    \item[\texttt{\bslash @dotsep}] Separation between dots, in mu
%        units. Should be defined as a number like 2 or 1.7
%    \end{description}
%
%    \begin{macrocode}
\newcommand\@pnumwidth{1.55em}
\newcommand\@tocrmarg{2.55em}
\newcommand\@dotsep{4.5}
%<article>\setcounter{tocdepth}{3}
%<!article>\setcounter{tocdepth}{2}
%    \end{macrocode}
% \end{macro}
% \end{macro}
% \end{macro}
%
% \subsubsection{Table of Contents}
%
% \begin{macro}{\tableofcontents}
%    This macro is used to request that \LaTeX{} produces a table of
%    contents. In the report and book document classes the tables of
%    contents, figures etc. are always set in single-column style.
%
% \changes{v1.0g}{1993/12/09}{Moved the setting of
%    \cs{@restonecolfalse}}
% \changes{v1.4h}{2007/10/19}{Explain why \cs{@mkboth} is inside the heading
%                         arg for \cs{tableofcontents} (pr/3285 and pr/3984)}
%    \begin{macrocode}
\newcommand\tableofcontents{%
%<*report|book>
    \if@twocolumn
      \@restonecoltrue\onecolumn
    \else
      \@restonecolfalse
    \fi
%    \end{macrocode}
%    The title is set using the |\chapter*| command, making sure that
%    the running head --if one is required-- contains the right
%    information.
%    \begin{macrocode}
    \chapter*{\contentsname
%</report|book>
%<article>    \section*{\contentsname
%    \end{macrocode}
%    The code for |\@mkboth| is placed inside the heading to avoid any
%    influence on vertical spacing after the heading (in some cases). For
%    other commands, such as |\listoffigures| below this has been changed from
%    the \LaTeX{}2.09 version as it will produce a serious bug if used in
%    two-column mode (see, pr/3285). However |\tableofcontents| is always
%    typeset in one-column mode in these classes, therefore the somewhat
%    inconsistent setting has been retained for compatibility reasons.
%    \begin{macrocode}
        \@mkboth{%
           \MakeUppercase\contentsname}{\MakeUppercase\contentsname}}%
%    \end{macrocode}
%    The the actual table of contents is made by calling
%    |\@starttoc{toc}|. After that we restore twocolumn mode if
%    necessary.
%    \begin{macrocode}
    \@starttoc{toc}%
%<!article>    \if@restonecol\twocolumn\fi
    }
%    \end{macrocode}
% \end{macro}
%
% \begin{macro}{\l@part}
%    Each sectioning command needs an additional macro  to format its
%    entry in the table of contents, as described above. The macro for
%    the entry for parts is defined in a special way.
%
%    First we make sure that if a pagebreak should occur, it occurs
%    \emph{before} this entry. Also a little whitespace is added and a
%    group begun to keep changes local.
% \changes{v1.0h}{1993/12/18}{Replaced -\cs{@secpenalty} by
%    \cs{@secpenalty}.  ASAJ.}
% \changes{v1.2i}{1994/04/28}{Don't print a toc line when the tocdepth
%    counter is less then -1}
% \changes{v1.3b}{1995/05/23}{Added missing braces around argument
%           to \cs{addpenalty}.}
% \changes{v1.3x}{1997/10/10}{Removed setting of \cs{@tempdima} as
%    this macro does not use \cs{numberline} to set the toc line.}
% \changes{v1.4a}{1998/10/12}{we should use \cs{@tocrmarg}; see PR/2881.}
%    \begin{macrocode}
\newcommand*\l@part[2]{%
  \ifnum \c@tocdepth >-2\relax
%<article>    \addpenalty\@secpenalty
%<!article>    \addpenalty{-\@highpenalty}%
    \addvspace{2.25em \@plus\p@}%
%    \end{macrocode}
%    The macro |\numberline| requires that the width of the box that
%    holds the part number is stored in \LaTeX's scratch register
%    |\@tempdima|. Therefore we initialize it there even though we do
%    not use |\numberline| internally---the value used is quite large
%    so that something like |\numberline{VIII}| would still work.
% \changes{v1.4d}{2001/04/21}{Initialize \cs{@tempdima} to some
%    sensible value (pr/3327)}
%    \begin{macrocode}
    \setlength\@tempdima{3em}%
    \begingroup
%    \end{macrocode}
%    We set |\parindent| to 0pt and use |\rightskip| to leave
%    enough room for the pagenumbers.\footnote{^^A
%        We should really set \cs{rightskip} to \cs{@tocrmarg} instead
%        of \cs{@pnumwidth} (no version of {\LaTeX} ever did this),
%        otherwise the \cs{rightskip} is too small.
%        Unfortunately this can't be changed in {\LaTeXe} as we don't
%        want to create different versions of {\LaTeXe} which produce
%        different typset output unless this is absolutely necessary;
%        instead we suspend it for \LaTeX3.}
%    To prevent overfull box messages the |\parfillskip| is set to a
%    negative value.
%    \begin{macrocode}
      \parindent \z@ \rightskip \@pnumwidth
      \parfillskip -\@pnumwidth
%    \end{macrocode}
%    Now we can set the entry, in a large bold font. We make sure to
%    leave vertical mode, set the part title and add the pagenumber,
%    set flush right.
%    \begin{macrocode}
      {\leavevmode
       \large \bfseries #1\hfil \hb@xt@\@pnumwidth{\hss #2}}\par
%    \end{macrocode}
%    Prevent a pagebreak immediately after this entry, but use
%    |\everypar| to reset the |\if@nobreak| switch. Finally we close
%    the group.
% \changes{v1.3j}{1995/08/16}{Add missing percent}
%    \begin{macrocode}
       \nobreak
%<article>       \if@compatibility
         \global\@nobreaktrue
         \everypar{\global\@nobreakfalse\everypar{}}%
%<article>      \fi
    \endgroup
  \fi}
%    \end{macrocode}
% \end{macro}
%
% \begin{macro}{\l@chapter}
%    This macro formats the entries in the table of contents for
%    chapters. It is very similar to |\l@part|
%
%    First we make sure that if a pagebreak should occur, it occurs
%    \emph{before} this entry. Also a little whitespace is added and a
%    group begun to keep changes local.
% \changes{v1.2i}{1994/04/28}{Don't print a toc line when the tocdepth
%    counter is less than 0}
% \changes{v1.3b}{1995/05/23}{Added missing braces around argument
%           to \cs{addpenalty}.}
% \changes{v1.4a}{1998/10/12}{we should use \cs{@tocrmarg}; see PR/2881.}
%    \begin{macrocode}
%<*report|book>
\newcommand*\l@chapter[2]{%
  \ifnum \c@tocdepth >\m@ne
    \addpenalty{-\@highpenalty}%
    \vskip 1.0em \@plus\p@
%    \end{macrocode}
%
%    The macro |\numberline| requires that the width of the box that
%    holds the part number is stored in \LaTeX's scratch register
%    |\@tempdima|. Therefore we initialize it there even though we do
%    not use |\numberline| internally (the position as well as the
%    values seems questionable but can't be changed without producing
%    compatibility problems). We begin a group, and change
%    some of the paragraph parameters (see also the remark at
%    \cs{l@part} regarding \cs{rightskip}).
%    \begin{macrocode}
    \setlength\@tempdima{1.5em}%
    \begingroup
      \parindent \z@ \rightskip \@pnumwidth
      \parfillskip -\@pnumwidth
%    \end{macrocode}
%    Then we leave vertical mode and switch to a bold font.
%    \begin{macrocode}
      \leavevmode \bfseries
%    \end{macrocode}
%    Because we do not use |\numberline| here, we have do some fine
%    tuning `by hand', before we can set the entry. We discourage but
%    not disallow a pagebreak immediately after a chapter entry.
%    \begin{macrocode}
      \advance\leftskip\@tempdima
      \hskip -\leftskip
      #1\nobreak\hfil \nobreak\hb@xt@\@pnumwidth{\hss #2}\par
      \penalty\@highpenalty
    \endgroup
  \fi}
%</report|book>
%    \end{macrocode}
% \end{macro}
%
% \begin{macro}{\l@section}
%    In the article document class the entry in the table of contents
%    for sections looks much like the chapter entries for the report
%    and book document classes.
%
%    First we make sure that if a pagebreak should occur, it occurs
%    \emph{before} this entry. Also a little whitespace is added and a
%    group begun to keep changes local.
% \changes{v1.0h}{1993/12/18}{Replaced -\cs{@secpenalty} by
%    \cs{@secpenalty}.  ASAJ.}
% \changes{v1.2i}{1994/04/28}{Don't print a toc line when the tocdepth
%    counter is less than 1.}
% \changes{v1.4a}{1998/10/12}{we should use \cs{@tocrmarg}; see PR/2881.}
%    \begin{macrocode}
%<*article>
\newcommand*\l@section[2]{%
  \ifnum \c@tocdepth >\z@
    \addpenalty\@secpenalty
    \addvspace{1.0em \@plus\p@}%
%    \end{macrocode}
%
%    The macro |\numberline| requires that the width of the box that
%    holds the part number is stored in \LaTeX's scratch register
%    |\@tempdima|. Therefore we put it there. We begin a group, and
%    change some of the paragraph parameters (see also the remark at
%    \cs{l@part} regarding \cs{rightskip}).
%    \begin{macrocode}
    \setlength\@tempdima{1.5em}%
    \begingroup
      \parindent \z@ \rightskip \@pnumwidth
      \parfillskip -\@pnumwidth
%    \end{macrocode}
%    Then we leave vertical mode and switch to a bold font.
%    \begin{macrocode}
      \leavevmode \bfseries
%    \end{macrocode}
%    Because we do not use |\numberline| here, we have do some fine
%    tuning `by hand', before we can set the entry. We discourage but
%    not disallow a pagebreak immediately after a chapter entry.
%    \begin{macrocode}
      \advance\leftskip\@tempdima
      \hskip -\leftskip
      #1\nobreak\hfil \nobreak\hb@xt@\@pnumwidth{\hss #2}\par
    \endgroup
  \fi}
%</article>
%    \end{macrocode}
%    In the report and book document classes the definition for
%    |\l@section| is much simpler.
%    \begin{macrocode}
%<*report|book>
\newcommand*\l@section{\@dottedtocline{1}{1.5em}{2.3em}}
%</report|book>
%    \end{macrocode}
% \end{macro}
%
% \begin{macro}{\l@subsection}
% \begin{macro}{\l@subsubsection}
% \begin{macro}{\l@paragraph}
% \begin{macro}{\l@subparagraph}
%    All lower level entries are defined using the macro
%    |\@dottedtocline| (see above).
%    \begin{macrocode}
%<*article>
\newcommand*\l@subsection{\@dottedtocline{2}{1.5em}{2.3em}}
\newcommand*\l@subsubsection{\@dottedtocline{3}{3.8em}{3.2em}}
\newcommand*\l@paragraph{\@dottedtocline{4}{7.0em}{4.1em}}
\newcommand*\l@subparagraph{\@dottedtocline{5}{10em}{5em}}
%</article>
%<*report|book>
\newcommand*\l@subsection{\@dottedtocline{2}{3.8em}{3.2em}}
\newcommand*\l@subsubsection{\@dottedtocline{3}{7.0em}{4.1em}}
\newcommand*\l@paragraph{\@dottedtocline{4}{10em}{5em}}
\newcommand*\l@subparagraph{\@dottedtocline{5}{12em}{6em}}
%</report|book>
%    \end{macrocode}
% \end{macro}
% \end{macro}
% \end{macro}
% \end{macro}
%
% \subsubsection{List of figures}
%
% \begin{macro}{\listoffigures}
%    This macro is used to request that \LaTeX{} produces a list of
%    figures. It is very similar to |\tableofcontents|.
%
% \changes{v1.0g}{1993/12/09}{Moved the setting of
%    \cs{@restonecolfalse}}
% \changes{v1.4c}{2001/01/06}{Moved \cs{@mkboth} out of heading
%                             arg (pr/3285)}
%    \begin{macrocode}
\newcommand\listoffigures{%
%<*report|book>
    \if@twocolumn
      \@restonecoltrue\onecolumn
    \else
      \@restonecolfalse
    \fi
    \chapter*{\listfigurename}%
%</report|book>
%<article>    \section*{\listfigurename}%
      \@mkboth{\MakeUppercase\listfigurename}%
              {\MakeUppercase\listfigurename}%
    \@starttoc{lof}%
%<report|book>    \if@restonecol\twocolumn\fi
    }
%    \end{macrocode}
% \end{macro}
%
% \begin{macro}{\l@figure}
%    This macro produces an entry in the list of figures.
%    \begin{macrocode}
\newcommand*\l@figure{\@dottedtocline{1}{1.5em}{2.3em}}
%    \end{macrocode}
% \end{macro}
%
% \subsubsection{List of tables}
%
% \begin{macro}{\listoftables}
%    This macro is used to request that \LaTeX{} produces a list of
%    tables. It is very similar to |\tableofcontents|.
%
% \changes{v1.0g}{1993/12/09}{Moved the setting of
%    \cs{@restonecolfalse}}
% \changes{v1.4c}{2001/01/06}{Moved \cs{@mkboth} out of heading
%                             arg (pr/3285)}
%    \begin{macrocode}
\newcommand\listoftables{%
%<*report|book>
    \if@twocolumn
      \@restonecoltrue\onecolumn
    \else
      \@restonecolfalse
    \fi
    \chapter*{\listtablename}%
%</report|book>
%<article>    \section*{\listtablename}%
      \@mkboth{%
          \MakeUppercase\listtablename}%
         {\MakeUppercase\listtablename}%
    \@starttoc{lot}%
%<report|book>    \if@restonecol\twocolumn\fi
    }
%    \end{macrocode}
% \end{macro}
%
% \begin{macro}{\l@table}
%    This macro produces an entry in the list of tables.
%    \begin{macrocode}
\let\l@table\l@figure
%    \end{macrocode}
% \end{macro}
%
% \subsection{Bibliography}
%
% \begin{macro}{\bibindent}
%    The ``open'' bibliography format uses an indentation of
%    |\bibindent|.
%    \begin{macrocode}
\newdimen\bibindent
\setlength\bibindent{1.5em}
%    \end{macrocode}
% \end{macro}
%
% \begin{environment}{thebibliography}
%    The `thebibliography' environment executes the following
%    commands:
%
%    |\renewcommand{\newblock}{\hskip.11em \@plus.33em \@minus.07em}|\\
%      --- Defines the ``closed'' format, where the blocks (major units
%      of information) of an entry run together.
%
%    |\sloppy|  --- Used because it's rather hard to do line breaks in
%      bibliographies,
%
%    |\sfcode`\.=1000\relax| ---
%      Causes a `.' (period) not to produce an end-of-sentence space.
%
%    The implementation of this environment is based on the generic
%    list environment. It uses the \Lcount{enumiv} counter internally
%    to generate the labels of the list.
%
%    When an empty `thebibliography' environment is found, a warning
%    is issued.
%
% \changes{v1.0i}{1993/12/19}{Corrected definition of thebibliography
%    for article}
% \changes{v1.2z}{1995/05/09}{added a missing percent character}
% \changes{v1.3b}{1995/05/23}{Added missing braces in definition
%    of thebibliography environment.}
% \changes{v1.3j}{1995/08/16}{remove surplus spaces}
% \changes{v1.3k}{1995/08/27}{Code for openbib changed}
% \changes{v1.3t}{1996/10/05}{Added setting value of \cs{@clubpenalty}}
%    \begin{macrocode}
\newenvironment{thebibliography}[1]
%<*article>
     {\section*{\refname}%
%    \end{macrocode}
%    The |\@mkboth| was moved out of the heading argument since at
%    least in report and book (twocolumn option) there are definitions
%    for |\chapter| which would swallow it otherwise.
% \changes{v1.4c}{2001/01/06}{Moved \cs{@mkboth} out of heading
%                             arg (pr/3285)}
%    \begin{macrocode}
      \@mkboth{\MakeUppercase\refname}{\MakeUppercase\refname}%
%</article>
%<*!article>
     {\chapter*{\bibname}%
      \@mkboth{\MakeUppercase\bibname}{\MakeUppercase\bibname}%
%</!article>
      \list{\@biblabel{\@arabic\c@enumiv}}%
           {\settowidth\labelwidth{\@biblabel{#1}}%
            \leftmargin\labelwidth
            \advance\leftmargin\labelsep
            \@openbib@code
            \usecounter{enumiv}%
            \let\p@enumiv\@empty
            \renewcommand\theenumiv{\@arabic\c@enumiv}}%
      \sloppy
%    \end{macrocode}
%    This is setting the normal (non-infinite) value of
%    |\clubpenalty| for the whole of this environment,
%    so we must reset its stored value also.  (Why is there a |%| after
%    the second 4000 below?)
%    \begin{macrocode}
      \clubpenalty4000
      \@clubpenalty \clubpenalty
      \widowpenalty4000%
      \sfcode`\.\@m}
     {\def\@noitemerr
       {\@latex@warning{Empty `thebibliography' environment}}%
      \endlist}
%    \end{macrocode}
% \end{environment}
%
% \begin{macro}{\newblock}
%    The default definition for |\newblock| is to produce a small space.
% \changes{v1.3k}{1995/08/27}{Default changed.}
%    \begin{macrocode}
\newcommand\newblock{\hskip .11em\@plus.33em\@minus.07em}
%    \end{macrocode}
% \end{macro}
%
% \begin{macro}{\@openbib@code}
%    The default definition for |\@openbib@code| is to do nothing.
%    It will be changed by the \Lopt{openbib} option.
% \changes{v1.3k}{1995/08/27}{Macro added}
%    \begin{macrocode}
\let\@openbib@code\@empty
%    \end{macrocode}
% \end{macro}
%
% \begin{macro}{\@biblabel}
%    The label for a |\bibitem[...]| command is produced by this
%    macro. The default from \file{latex.dtx} is used.
%    \begin{macrocode}
% \renewcommand*{\@biblabel}[1]{[#1]\hfill}
%    \end{macrocode}
% \end{macro}
%
% \begin{macro}{\@cite}
%    The output of the |\cite| command is produced by this macro. The
%    default from \file{latex.dtx} is used.
%    \begin{macrocode}
% \renewcommand*{\@cite}[1]{[#1]}
%    \end{macrocode}
% \end{macro}
%
%  \subsection{The index}
%
% \begin{environment}{theindex}
%    The environment `theindex' can be used for indices. It makes an
%    index with two columns, with each entry a separate paragraph. At
%    the user level the commands |\item|, |\subitem| and |\subsubitem|
%    are used to produce index entries of various levels. When a new
%    letter of the alphabet is encountered an amount of |\indexspace|
%    white space can be added.
%
%
% \changes{v1.0g}{1993/12/09}{Moved the setting of
%    \cs{@restonecoltrue}}
%    \begin{macrocode}
\newenvironment{theindex}
               {\if@twocolumn
                  \@restonecolfalse
                \else
                  \@restonecoltrue
                \fi
%<article>                \twocolumn[\section*{\indexname}]%
%<!article>                \twocolumn[\@makeschapterhead{\indexname}]%
                \@mkboth{\MakeUppercase\indexname}%
                        {\MakeUppercase\indexname}%
                \thispagestyle{plain}\parindent\z@
%    \end{macrocode}
%    Parameter changes to |\columnseprule| and |\columnsep| have to be
%    done after |\twocolumn| has acted. Otherwise they can affect the
%    last page before the index.
% \changes{ v1.4f}{2004/02/16}{Moved setting of \cs{columnsep} and
%    \cs{columnseprule} later to avoid affecting the wrong page (pr/3616)}
%    \begin{macrocode}
                \parskip\z@ \@plus .3\p@\relax
                \columnseprule \z@
                \columnsep 35\p@
                \let\item\@idxitem}
%    \end{macrocode}
%    When the document continues after the index and it was a one
%    column document we have to switch back to one column after the
%    index.
%    \begin{macrocode}
               {\if@restonecol\onecolumn\else\clearpage\fi}
%    \end{macrocode}
% \end{environment}
%
% \begin{macro}{\@idxitem}
% \begin{macro}{\subitem}
% \begin{macro}{\subsubitem}
%    These macros are used to format the entries in the index. ^^AA ???
% \changes{v1.3f}{1995/06/23}{Corrected error in definition of
%                         \cs{@idxitem}.}
% \changes{v1.3j}{1995/08/16}{use \cs{@idxitem} to save space}
%    \begin{macrocode}
\newcommand\@idxitem{\par\hangindent 40\p@}
\newcommand\subitem{\@idxitem \hspace*{20\p@}}
\newcommand\subsubitem{\@idxitem \hspace*{30\p@}}
%    \end{macrocode}
% \end{macro}
% \end{macro}
% \end{macro}
%
% \begin{macro}{\indexspace}
%    The amount of white space that is inserted between `letter
%    blocks' in the index.
%    \begin{macrocode}
\newcommand\indexspace{\par \vskip 10\p@ \@plus5\p@ \@minus3\p@\relax}
%    \end{macrocode}
% \end{macro}
%
% \subsection{Footnotes}
%
% \begin{macro}{\footnoterule}
%    Usually, footnotes are separated from the main body of the text
%    by a small rule. This rule is drawn by the macro |\footnoterule|.
%    We have to make sure that the rule takes no vertical space (see
%    \file{plain.tex}) so we compensate for the natural height of the
%    rule of 0.4pt by adding the right amount of vertical skip.
%
%    To prevent the rule from colliding with the footnote we first add
%    a little negative vertical skip, then we put the rule and make
%    sure we end up at the same point where we begun this operation.
% \changes{v1.3a}{1995/05/17}{use \cs{@width}}
%    \begin{macrocode}
\renewcommand\footnoterule{%
  \kern-3\p@
  \hrule\@width.4\columnwidth
  \kern2.6\p@}
%    \end{macrocode}
% \end{macro}
%
% \begin{macro}{\c@footnote}
%    Footnotes are numbered within chapters in the report and book
%    document styles.
%    \begin{macrocode}
%<!article>\@addtoreset{footnote}{chapter}
%    \end{macrocode}
% \end{macro}
%
% \begin{macro}{\@makefntext}
%    The footnote mechanism of \LaTeX{} calls the macro |\@makefntext|
%    to produce the actual footnote. The macro gets the text of the
%    footnote as its argument and should use |\@thefnmark| as the mark
%    of the footnote. The macro |\@makefntext|is called when
%    effectively inside a |\parbox| of width |\columnwidth| (i.e.,
%    with |\hsize| = |\columnwidth|).
%
%   An example of what can be achieved is given by the following piece
%   of \TeX\ code.
% \begin{verbatim}
%          \newcommand\@makefntext[1]{%
%             \@setpar{\@@par
%                      \@tempdima = \hsize
%                      \advance\@tempdima-10pt
%                      \parshape \@ne 10pt \@tempdima}%
%             \par
%             \parindent 1em\noindent
%             \hbox to \z@{\hss\@makefnmark}#1}
% \end{verbatim}
%    The effect of this definition is that all lines of the footnote
%    are indented by 10pt, while the first line of a new paragraph is
%    indented by 1em. To change these dimensions, just substitute the
%    desired value for `10pt' (in both places) or `1em'.  The mark is
%    flushright against the footnote.
%
%    In these document classes we use a simpler macro, in which the
%    footnote text is set like an ordinary text paragraph, with no
%    indentation except on the first line of a paragraph, and the
%    first line of the footnote. Thus, all the macro must do is set
%    |\parindent| to the appropriate value for succeeding paragraphs
%    and put the proper indentation before the mark.
%
% \changes{v1.1a}{1994/03/13}{Use \cs{@makefnmark} to generate
%    footnote marker}
%    \begin{macrocode}
\newcommand\@makefntext[1]{%
    \parindent 1em%
    \noindent
    \hb@xt@1.8em{\hss\@makefnmark}#1}
%    \end{macrocode}
% \end{macro}
%
% \begin{macro}{\@makefnmark}
%    The footnote markers that are printed in the text to point to the
%    footnotes should be produced by the macro |\@makefnmark|. We use
%    the default definition for it.
%    \begin{macrocode}
%\renewcommand\@makefnmark{\hbox{\@textsuperscript
%                                  {\normalfont\@thefnmark}}}
%    \end{macrocode}
% \end{macro}
%
% \section{Initialization}
%
% \subsection{Words}
%
% This document class is for documents prepared in the English language.
% To prepare a version for another language, various English words must
% be replaced.  All the English words that require replacement are
% defined below in command names. These commands may be redefined in
% any class or package that is customising \LaTeX\ for use with
% non-English languages.
% \changes{v1.3h}{1995/07/20}{Split up to save save stack /1742}
%
% \begin{macro}{\contentsname}
% \begin{macro}{\listfigurename}
% \begin{macro}{\listtablename}
%    \begin{macrocode}
\newcommand\contentsname{Contents}
\newcommand\listfigurename{List of Figures}
\newcommand\listtablename{List of Tables}
%    \end{macrocode}
% \end{macro}
% \end{macro}
% \end{macro}
%
% \begin{macro}{\refname}
% \begin{macro}{\bibname}
% \begin{macro}{\indexname}
%    \begin{macrocode}
%<article>\newcommand\refname{References}
%<report|book>\newcommand\bibname{Bibliography}
\newcommand\indexname{Index}
%    \end{macrocode}
% \end{macro}
% \end{macro}
% \end{macro}
%
% \begin{macro}{\figurename}
% \begin{macro}{\tablename}
%    \begin{macrocode}
\newcommand\figurename{Figure}
\newcommand\tablename{Table}
%    \end{macrocode}
% \end{macro}
% \end{macro}
%
% \begin{macro}{\partname}
% \begin{macro}{\chaptername}
% \begin{macro}{\appendixname}
% \begin{macro}{\abstractname}
%    \begin{macrocode}
\newcommand\partname{Part}
%<report|book>\newcommand\chaptername{Chapter}
\newcommand\appendixname{Appendix}
%<!book>\newcommand\abstractname{Abstract}
%    \end{macrocode}
% \end{macro}
% \end{macro}
% \end{macro}
% \end{macro}
%
% \subsection{Date}
%
% \begin{macro}{\today}
%    This macro uses the \TeX\ primitives |\month|, |\day| and |\year|
%    to provide the date of the \LaTeX-run.
%
%    At |\begin{document}| this definition will be optimised
%    so that the names of all the `wrong' months are not stored.
%    This optimisation is not done here as that would `freeze'
%    |\today| in any special purpose format made by loading the class
%    file into the format file.
% \changes{v1.3j}{1995/08/16}{use \cs{edef} to save a lot of space}
% \changes{v1.3w}{1997/10/06}{use \cs{def} again, latex/2620}
%    \begin{macrocode}
\def\today{\ifcase\month\or
  January\or February\or March\or April\or May\or June\or
  July\or August\or September\or October\or November\or December\fi
  \space\number\day, \number\year}
%    \end{macrocode}
% \end{macro}
%
% \subsection{Two column mode}
%
% \begin{macro}{\columnsep}
%    This gives the distance between two columns in two column mode.
%    \begin{macrocode}
\setlength\columnsep{10\p@}
%    \end{macrocode}
% \end{macro}
%
% \begin{macro}{\columnseprule}
%    This gives the width of the rule between two columns in two
%    column mode. We have no visible rule.
%    \begin{macrocode}
\setlength\columnseprule{0\p@}
%    \end{macrocode}
% \end{macro}
%
% \subsection{The page style}
%    We have \pstyle{plain} pages in the document classes article and
%    report unless the user specified otherwise. In the `book'
%    document class we use the page style \pstyle{headings} by
%    default. We use arabic pagenumbers.
%    \begin{macrocode}
%<!book>\pagestyle{plain}
%<book>\pagestyle{headings}
\pagenumbering{arabic}
%    \end{macrocode}
%
% \subsection{Single or double sided printing}
%
%
% \changes{v1.2v}{1994/11/10}{removed typo}
%    When the \Lopt{twoside} option wasn't specified, we don't try to
%    make each page as long as all the others.
%    \begin{macrocode}
\if@twoside
\else
  \raggedbottom
\fi
%    \end{macrocode}
%    When the \Lopt{twocolumn} option was specified we call
%    |\twocolumn| to activate this mode. We try to make each column as
%    long as the others, but call |sloppy| to make our life easier.
%    \begin{macrocode}
\if@twocolumn
  \twocolumn
  \sloppy
  \flushbottom
%    \end{macrocode}
%    Normally we call |\onecolumn| to initiate typesetting in one
%    column.
%    \begin{macrocode}
\else
  \onecolumn
\fi
%</article|report|book>
%    \end{macrocode}
%
% \changes{v1.3i}{1995/08/09}{Moved code for generic class options
% leqno and fleqn to kernel file}
%
% \Finale
%
\endinput

