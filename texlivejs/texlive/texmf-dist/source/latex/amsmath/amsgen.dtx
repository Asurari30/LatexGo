%
% \iffalse meta-comment
%
% Copyright 1995, 1999 American Mathematical Society.
% Copyright 2016 LaTeX3 Project and American Mathematical Society. 
%
% This work may be distributed and/or modified under the
% conditions of the LaTeX Project Public License, either version 1.3c
% of this license or (at your option) any later version.
% The latest version of this license is in
%   https://www.latex-project.org/lppl.txt
% and version 1.3c or later is part of all distributions of LaTeX
% version 2005/12/01 or later.
% 
% This work has the LPPL maintenance status `maintained'.
% 
% The Current Maintainer of this work is the LaTeX3 Project.
%
% \fi
%
% \iffalse
%<*driver>
\documentclass{amsdtx}
\def\MaintainedByLaTeXTeam#1{%
\begin{center}%
\fbox{\fbox{\begin{tabular}{@{}l@{}}%
This file is maintained by the \LaTeX{} Project team.\\%
Bug reports can be opened (category \texttt{#1}) at\\%
\url{https://latex-project.org/bugs/}.\end{tabular}}}\end{center}}
\usepackage{amsgen}
\GetFileInfo{amsgen.sty}
\begin{document}
\title{The \pkg{amsgen} package}
\author{American Mathematical Society\\Michael Downes}
\date{Version \fileversion, \filedate}
\DocInput{amsgen.dtx}
\end{document}
%</driver>
% \fi
%
% \maketitle
% \MaintainedByLaTeXTeam{amslatex}
%
% \MakeShortVerb\|
%
% \section{Introduction}
%    This is an internal package for storing common functions
%    that are shared by more than one package in the \amslatex/
%    distribution. Some of these might eventually make it into the \latex/
%    kernel.
%
% \StopEventually{}
%
%    Standard package info.
%    Using \cs{ProvidesFile} rather than \cs{ProvidesPackage} because
%    the latter, when input by, e.g, \cls{amsbook}, results in
%    \texttt{LaTeX warning: You have requested document class `amsbook',
%    but the document class provides `amsgen'.}
%    \begin{macrocode}
\NeedsTeXFormat{LaTeX2e}% LaTeX 2.09 can't be used (nor non-LaTeX)
[1994/12/01]% LaTeX date must December 1994 or later
\ProvidesFile{amsgen.sty}[1999/11/30 v2.0 generic functions]
%    \end{macrocode}
%
% \section{Implementation}
%    Some general macros shared by \fn{amsart.dtx}, \fn{amsmath.dtx},
%    \fn{amsfonts.dtx}, \dots
%
% \begin{macro}{\@saveprimitive}
%    The \pkg{amsmath} package redefines a number of \tex/ primitives.
%    In case some preceding package also decided to redefine one of
%    those same primitives, we had better do some checking to make
%    sure that we are able to save the primitive meaning for internal
%    use. This is handled by the \cs{@saveprimitive} function. We
%    follow the example of \cs{@@input} where the primitive meaning is
%    stored in an internal control sequence with a |@@| prefix.
%    Primitive control sequences can be distinguished by the fact that
%    \cs{string} and \cs{meaning} return the same information.
%    \begin{macrocode}
\providecommand{\@saveprimitive}[2]{\begingroup\escapechar`\\\relax
  \edef\@tempa{\string#1}\edef\@tempb{\meaning#1}%
  \ifx\@tempa\@tempb \global\let#2#1%
  \else
%    \end{macrocode}
%    Check to see if \arg{2} was already given the desired primitive
%    meaning somewhere else.
%    \begin{macrocode}
    \edef\@tempb{\meaning#2}%
    \ifx\@tempa\@tempb
    \else
      \@latex@error{Unable to properly define \string#2; primitive
      \noexpand#1no longer primitive}\@eha
    \fi
  \fi
  \endgroup}
%    \end{macrocode}
% \end{macro}
%
% \begin{macro}{\@xp}
% \begin{macro}{\@nx}
%    Shorthands for long command names.
%    \begin{macrocode}
\let\@xp=\expandafter
\let\@nx=\noexpand
%    \end{macrocode}
%    \end{macro}
%    \end{macro}
%
%  \begin{macro}{\@emptytoks}
%    A token register companion for \cs{@empty}. Saves a little main mem and
%    probably makes initializations such as |\toks@{}| run faster too.
%    \begin{macrocode}
\newtoks\@emptytoks
%    \end{macrocode}
%  \end{macro}
%
% \begin{macro}{\@oparg}
%    Use of \cs{@oparg} simplifies some constructions where a macro
%    takes an optional argument in square brackets. We can't use
%    \cn{newcommand} here because this function might be previously
%    defined by the \pkg{amsmath} package in a loading sequence such as
% \begin{verbatim}
% \usepackage{amsmath,amsthm}
% \end{verbatim}
%    \begin{macrocode}
\def\@oparg#1[#2]{\@ifnextchar[{#1}{#1[#2]}}
%    \end{macrocode}
% \end{macro}
%
% \begin{macro}{\@ifempty}
% \begin{macro}{\@ifnotempty}
%    |\@ifnotempty| and |\@ifempty| use category 11 |@| characters to
%    test whether the argument is empty or not, since these are highly
%    unlikely to occur in the argument. As with \cn{@oparg}, there is a
%    possibility that these commands were defined previously in
%    \fn{amsmath.sty}.
%    \begin{macrocode}
\long\def\@ifempty#1{\@xifempty#1@@..\@nil}
\long\def\@xifempty#1#2@#3#4#5\@nil{%
  \ifx#3#4\@xp\@firstoftwo\else\@xp\@secondoftwo\fi}
%    \end{macrocode}
%    \cs{@ifnotempty} is a shorthand that makes code read better when
%    no action is needed in the empty case. At a cost of double
%    argument-reading---so for often-executed code, avoiding
%    \cs{@ifnotempty} might be wise.
%    \begin{macrocode}
\long\def\@ifnotempty#1{\@ifempty{#1}{}}
%    \end{macrocode}
% \end{macro}
% \end{macro}
%
%    Some abbreviations to conserve token mem.
%    \begin{macrocode}
\def\FN@{\futurelet\@let@token}
\def\DN@{\def\next@}
\def\RIfM@{\relax\ifmmode}
\def\setboxz@h{\setbox\z@\hbox}
\def\wdz@{\wd\z@}
\def\boxz@{\box\z@}
\def\relaxnext@{\let\@let@token\relax}
%    \end{macrocode}
%
% \begin{macro}{\new@ifnextchar}
%    This macro is a new version of \latex/'s \verb+\@ifnextchar+,
%    macro that does not skip over spaces.
%    \begin{macrocode}
\long\def\new@ifnextchar#1#2#3{%
%    \end{macrocode}
%    By including the space after the equals sign, we make it possible
%    for \cs{new@ifnextchar} to do look-ahead for any token, including a
%    space!
%    \begin{macrocode}
  \let\reserved@d= #1%
  \def\reserved@a{#2}\def\reserved@b{#3}%
  \futurelet\@let@token\new@ifnch
}
%
\def\new@ifnch{%
  \ifx\@let@token\reserved@d \let\reserved@b\reserved@a \fi
  \reserved@b
}
%    \end{macrocode}
% \end{macro}
%
% \begin{macro}{\@ifstar}
%    There will essentially never be a space before the \qc{\*}, so
%    using \cs{@ifnextchar} is unnecessarily slow.
%    \begin{macrocode}
\def\@ifstar#1#2{\new@ifnextchar *{\def\reserved@a*{#1}\reserved@a}{#2}}
%    \end{macrocode}
% \end{macro}
%
%    The hook \cs{every@size} was changed to \cs{every@math@size} in the
%    December 1994 release of \latex/ and its calling procedures changed.
%    If \cs{every@math@size} is undefined it means the user has an older
%    version of \latex/ so we had better define it and patch a couple of
%    functions (\cs{glb@settings} and \cs{set@fontsize}).
%    \begin{macrocode}
\@ifundefined{every@math@size}{%
%    \end{macrocode}
%    Reuse the same token register; since it was never used except for
%    the purposes that are affected below, this is OK.
%    \begin{macrocode}
\let\every@math@size=\every@size
\def\glb@settings{%
     \expandafter\ifx\csname S@\f@size\endcsname\relax
       \calculate@math@sizes
     \fi
     \csname S@\f@size\endcsname
      \ifmath@fonts
%       \ifnum \tracingfonts>\tw@
%          \@font@info{Setting up math fonts for
%                  \f@size/\f@baselineskip}\fi
        \begingroup
          \escapechar\m@ne
          \csname mv@\math@version \endcsname
          \globaldefs\@ne
          \let \glb@currsize \f@size
          \math@fonts
        \endgroup
        \the\every@math@size
      \else
%       \ifnum \tracingfonts>\tw@
%        \@font@info{No math setup for \f@size/\f@baselineskip}%
%       \fi
      \fi
}
%    \end{macrocode}
%    Remove |\the\every@size| from \cs{size@update}.
%    \begin{macrocode}
\def\set@fontsize#1#2#3{%
    \@defaultunits\@tempdimb#2pt\relax\@nnil
    \edef\f@size{\strip@pt\@tempdimb}%
    \@defaultunits\@tempskipa#3pt\relax\@nnil
    \edef\f@baselineskip{\the\@tempskipa}%
    \edef\f@linespread{#1}%
    \let\baselinestretch\f@linespread
      \def\size@update{%
        \baselineskip\f@baselineskip\relax
        \baselineskip\f@linespread\baselineskip
        \normalbaselineskip\baselineskip
        \setbox\strutbox\hbox{%
          \vrule\@height.7\baselineskip
                \@depth.3\baselineskip
                \@width\z@}%
%%%     \the\every@size
        \let\size@update\relax}%
  }
}{}% end \@ifundefined test
%    \end{macrocode}
%
% \begin{macro}{\ex@}
%    The \cs{ex@} variable provides a small unit of space for use in
%    math-mode constructions, that varies according to the current type
%    size. For example, the \cn{pmb} command uses \cs{ex@} units.
%    Since a macro or mu unit solution for the \meta{dimen} \cs{ex@} won't
%    work without changing a lot of current code in the \pkg{amsmath}
%    package, we set \cs{ex@} through the \cs{every@math@size} hook.
%    The value of \cs{ex@} is scaled nonlinearly in a range of roughly
%    0.5pt to 1.5pt, by the function \cs{compute@ex@}.
%    \begin{macrocode}
\newdimen\ex@
\addto@hook\every@math@size{\compute@ex@}
%    \end{macrocode}
% \end{macro}
%
%    \cs{compute@ex@} computes \cs{ex@} as a nonlinear scaling from 10pt
%    to current font size (\cs{f@size}). Using .97 as the multiplier makes 1
%    |ex@| $\approx$ .9pt when the current type size is 8pt and 1 |ex@|
%    $\approx$ 1.1pt when the current type size is 12pt.
%
%    The formula is essentially
% \begin{displaymath}
% \newcommand{\points}{\mbox{pt}}
% \newcommand{\floor}[1]{\lfloor#1\rfloor}
%      1\points \pm (1\points - (.97)^{\floor{\vert 10 - n\vert}})
% \end{displaymath}
%    where $n$ = current type size, but adjusted to differentiate
%    half-point sizes as well as whole point sizes, and there is a
%    cutoff for extraordinarily large values of \cs{f@size} ($>$ 20pt)
%    so that the value of \cs{ex@} never exceeds 1.5pt.
%
%    \begin{macrocode}
\def\compute@ex@{%
  \begingroup
  \dimen@-\f@size\p@
  \ifdim\dimen@<-20\p@
%    \end{macrocode}
%    Never make \cs{ex@} larger than 1.5pt.
%    \begin{macrocode}
    \global\ex@ 1.5\p@
  \else
%    \end{macrocode}
%    Adjust by the reference size and multiply by 2 to allow for
%    half-point sizes.
%    \begin{macrocode}
    \advance\dimen@10\p@ \multiply\dimen@\tw@
%    \end{macrocode}
%    Save information about the current sign of \cs{dimen@}.
%    \begin{macrocode}
    \edef\@tempa{\ifdim\dimen@>\z@ -\fi}%
%    \end{macrocode}
%    Get the absolute value of \cs{dimen@}.
%    \begin{macrocode}
    \dimen@ \ifdim\dimen@<\z@ -\fi \dimen@
    \advance\dimen@-\@m sp % fudge factor
%    \end{macrocode}
%    Here we use \cs{vfuzz} merely as a convenient scratch register
%    \begin{macrocode}
    \vfuzz\p@
%    \end{macrocode}
%    Multiply in a loop.
%    \begin{macrocode}
    \def\do{\ifdim\dimen@>\z@
      \vfuzz=.97\vfuzz
      \advance\dimen@ -\p@
%\message{\vfuzz: \the\vfuzz, \dimen@: \the\dimen@}%
      \@xp\do \fi}%
    \do
    \dimen@\p@ \advance\dimen@-\vfuzz
    \global\ex@\p@
    \global\advance\ex@ \@tempa\dimen@
  \fi
  \endgroup
%\typeout{\string\f@size: \f@size}\showthe\ex@
}
%    \end{macrocode}
%    Tests of the \cs{compute@ex@} function yield the following results:
%
% \begin{center}\begin{tabular}{rlrl}
%    \cs{f@size}& \cs{ex@}&   \cs{f@size}& \cs{ex@}\\
%    10&          1.0pt&      9&           0.94089pt\\
%    11&          1.05911pt&  8.7&         0.91266pt\\
%    12&          1.11473pt&  8.5&         0.91266pt\\
%    14.4&        1.23982pt&  8.4&         0.88527pt\\
%    17.28&       1.36684pt&  8&           0.88527pt\\
%    20.74&       1.5pt&      7&           0.83293pt\\
%    19.5&        1.4395pt&   6&           0.78369pt\\
%                       &&    5&           0.73737pt\\
%                       &&    1&           0.57785pt
% \end{tabular}\end{center}
%
%  \begin{macro}{\@addpunct}
%    Use of the \cs{@addpunct} function allows ending punctuation in
%    section headings and elsewhere to be intelligently omitted
%    when punctuation is already present.
%    \begin{macrocode}
\def\@addpunct#1{\ifnum\spacefactor>\@m \else#1\fi}
%    \end{macrocode}
%  \end{macro}
%
%  \begin{macro}{\frenchspacing}
%    Change \cn{frenchspacing} to ensure that \cs{@addpunct} will
%    continue to work properly even when `french' spacing is in effect.
%    \begin{macrocode}
\def\frenchspacing{\sfcode`\.1006\sfcode`\?1005\sfcode`\!1004%
  \sfcode`\:1003\sfcode`\;1002\sfcode`\,1001 }
%    \end{macrocode}
%  \end{macro}
%
% \subsection{Miscellaneous}
%    \begin{macrocode}
\def\nomath@env{\@amsmath@err{%
  \string\begin{\@currenvir} allowed only in paragraph mode%
}\@ehb% "You've lost some text"
}
%    \end{macrocode}
%     A trade-off between main memory space and hash size; using
%     \verb+\Invalid@@+ saves 14 bytes of main memory for each use of
%     \verb+\Invalid@+, at the cost of one control sequence name.
%     \verb+\Invalid@+ is currently used about five times and
%     \verb+\Invalid@@+ is used by itself in some other instances,
%     which means that it saves us more memory than \verb+\FN@+,
%     \verb+\RIfM@+, and some of the other abbreviations above.
%    \begin{macrocode}
\def\Invalid@@{Invalid use of \string}
%    \end{macrocode}
%
%    The usual \cs{endinput} to ensure that random garbage at the end of
%    the file doesn't get copied by \fn{docstrip}.
%    \begin{macrocode}
\endinput
%    \end{macrocode}
%
% \CheckSum{337}
% \Finale
