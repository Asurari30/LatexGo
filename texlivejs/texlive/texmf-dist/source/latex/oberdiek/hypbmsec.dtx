% \iffalse meta-comment
%
% File: hypbmsec.dtx
% Version: 2016/05/16 v2.5
% Info: Bookmarks in sectioning commands
%
% Copyright (C) 1998-2000, 2006, 2007 by
%    Heiko Oberdiek <heiko.oberdiek at googlemail.com>
%    2016
%    https://github.com/ho-tex/oberdiek/issues
%
% This work may be distributed and/or modified under the
% conditions of the LaTeX Project Public License, either
% version 1.3c of this license or (at your option) any later
% version. This version of this license is in
%    http://www.latex-project.org/lppl/lppl-1-3c.txt
% and the latest version of this license is in
%    http://www.latex-project.org/lppl.txt
% and version 1.3 or later is part of all distributions of
% LaTeX version 2005/12/01 or later.
%
% This work has the LPPL maintenance status "maintained".
%
% This Current Maintainer of this work is Heiko Oberdiek.
%
% This work consists of the main source file hypbmsec.dtx
% and the derived files
%    hypbmsec.sty, hypbmsec.pdf, hypbmsec.ins, hypbmsec.drv.
%
% Distribution:
%    CTAN:macros/latex/contrib/oberdiek/hypbmsec.dtx
%    CTAN:macros/latex/contrib/oberdiek/hypbmsec.pdf
%
% Unpacking:
%    (a) If hypbmsec.ins is present:
%           tex hypbmsec.ins
%    (b) Without hypbmsec.ins:
%           tex hypbmsec.dtx
%    (c) If you insist on using LaTeX
%           latex \let\install=y% \iffalse meta-comment
%
% File: hypbmsec.dtx
% Version: 2016/05/16 v2.5
% Info: Bookmarks in sectioning commands
%
% Copyright (C) 1998-2000, 2006, 2007 by
%    Heiko Oberdiek <heiko.oberdiek at googlemail.com>
%    2016
%    https://github.com/ho-tex/oberdiek/issues
%
% This work may be distributed and/or modified under the
% conditions of the LaTeX Project Public License, either
% version 1.3c of this license or (at your option) any later
% version. This version of this license is in
%    http://www.latex-project.org/lppl/lppl-1-3c.txt
% and the latest version of this license is in
%    http://www.latex-project.org/lppl.txt
% and version 1.3 or later is part of all distributions of
% LaTeX version 2005/12/01 or later.
%
% This work has the LPPL maintenance status "maintained".
%
% This Current Maintainer of this work is Heiko Oberdiek.
%
% This work consists of the main source file hypbmsec.dtx
% and the derived files
%    hypbmsec.sty, hypbmsec.pdf, hypbmsec.ins, hypbmsec.drv.
%
% Distribution:
%    CTAN:macros/latex/contrib/oberdiek/hypbmsec.dtx
%    CTAN:macros/latex/contrib/oberdiek/hypbmsec.pdf
%
% Unpacking:
%    (a) If hypbmsec.ins is present:
%           tex hypbmsec.ins
%    (b) Without hypbmsec.ins:
%           tex hypbmsec.dtx
%    (c) If you insist on using LaTeX
%           latex \let\install=y% \iffalse meta-comment
%
% File: hypbmsec.dtx
% Version: 2016/05/16 v2.5
% Info: Bookmarks in sectioning commands
%
% Copyright (C) 1998-2000, 2006, 2007 by
%    Heiko Oberdiek <heiko.oberdiek at googlemail.com>
%    2016
%    https://github.com/ho-tex/oberdiek/issues
%
% This work may be distributed and/or modified under the
% conditions of the LaTeX Project Public License, either
% version 1.3c of this license or (at your option) any later
% version. This version of this license is in
%    http://www.latex-project.org/lppl/lppl-1-3c.txt
% and the latest version of this license is in
%    http://www.latex-project.org/lppl.txt
% and version 1.3 or later is part of all distributions of
% LaTeX version 2005/12/01 or later.
%
% This work has the LPPL maintenance status "maintained".
%
% This Current Maintainer of this work is Heiko Oberdiek.
%
% This work consists of the main source file hypbmsec.dtx
% and the derived files
%    hypbmsec.sty, hypbmsec.pdf, hypbmsec.ins, hypbmsec.drv.
%
% Distribution:
%    CTAN:macros/latex/contrib/oberdiek/hypbmsec.dtx
%    CTAN:macros/latex/contrib/oberdiek/hypbmsec.pdf
%
% Unpacking:
%    (a) If hypbmsec.ins is present:
%           tex hypbmsec.ins
%    (b) Without hypbmsec.ins:
%           tex hypbmsec.dtx
%    (c) If you insist on using LaTeX
%           latex \let\install=y% \iffalse meta-comment
%
% File: hypbmsec.dtx
% Version: 2016/05/16 v2.5
% Info: Bookmarks in sectioning commands
%
% Copyright (C) 1998-2000, 2006, 2007 by
%    Heiko Oberdiek <heiko.oberdiek at googlemail.com>
%    2016
%    https://github.com/ho-tex/oberdiek/issues
%
% This work may be distributed and/or modified under the
% conditions of the LaTeX Project Public License, either
% version 1.3c of this license or (at your option) any later
% version. This version of this license is in
%    http://www.latex-project.org/lppl/lppl-1-3c.txt
% and the latest version of this license is in
%    http://www.latex-project.org/lppl.txt
% and version 1.3 or later is part of all distributions of
% LaTeX version 2005/12/01 or later.
%
% This work has the LPPL maintenance status "maintained".
%
% This Current Maintainer of this work is Heiko Oberdiek.
%
% This work consists of the main source file hypbmsec.dtx
% and the derived files
%    hypbmsec.sty, hypbmsec.pdf, hypbmsec.ins, hypbmsec.drv.
%
% Distribution:
%    CTAN:macros/latex/contrib/oberdiek/hypbmsec.dtx
%    CTAN:macros/latex/contrib/oberdiek/hypbmsec.pdf
%
% Unpacking:
%    (a) If hypbmsec.ins is present:
%           tex hypbmsec.ins
%    (b) Without hypbmsec.ins:
%           tex hypbmsec.dtx
%    (c) If you insist on using LaTeX
%           latex \let\install=y\input{hypbmsec.dtx}
%        (quote the arguments according to the demands of your shell)
%
% Documentation:
%    (a) If hypbmsec.drv is present:
%           latex hypbmsec.drv
%    (b) Without hypbmsec.drv:
%           latex hypbmsec.dtx; ...
%    The class ltxdoc loads the configuration file ltxdoc.cfg
%    if available. Here you can specify further options, e.g.
%    use A4 as paper format:
%       \PassOptionsToClass{a4paper}{article}
%
%    Programm calls to get the documentation (example):
%       pdflatex hypbmsec.dtx
%       makeindex -s gind.ist hypbmsec.idx
%       pdflatex hypbmsec.dtx
%       makeindex -s gind.ist hypbmsec.idx
%       pdflatex hypbmsec.dtx
%
% Installation:
%    TDS:tex/latex/oberdiek/hypbmsec.sty
%    TDS:doc/latex/oberdiek/hypbmsec.pdf
%    TDS:source/latex/oberdiek/hypbmsec.dtx
%
%<*ignore>
\begingroup
  \catcode123=1 %
  \catcode125=2 %
  \def\x{LaTeX2e}%
\expandafter\endgroup
\ifcase 0\ifx\install y1\fi\expandafter
         \ifx\csname processbatchFile\endcsname\relax\else1\fi
         \ifx\fmtname\x\else 1\fi\relax
\else\csname fi\endcsname
%</ignore>
%<*install>
\input docstrip.tex
\Msg{************************************************************************}
\Msg{* Installation}
\Msg{* Package: hypbmsec 2016/05/16 v2.5 Bookmarks in sectioning commands (HO)}
\Msg{************************************************************************}

\keepsilent
\askforoverwritefalse

\let\MetaPrefix\relax
\preamble

This is a generated file.

Project: hypbmsec
Version: 2016/05/16 v2.5

Copyright (C) 1998-2000, 2006, 2007 by
   Heiko Oberdiek <heiko.oberdiek at googlemail.com>

This work may be distributed and/or modified under the
conditions of the LaTeX Project Public License, either
version 1.3c of this license or (at your option) any later
version. This version of this license is in
   http://www.latex-project.org/lppl/lppl-1-3c.txt
and the latest version of this license is in
   http://www.latex-project.org/lppl.txt
and version 1.3 or later is part of all distributions of
LaTeX version 2005/12/01 or later.

This work has the LPPL maintenance status "maintained".

This Current Maintainer of this work is Heiko Oberdiek.

This work consists of the main source file hypbmsec.dtx
and the derived files
   hypbmsec.sty, hypbmsec.pdf, hypbmsec.ins, hypbmsec.drv.

\endpreamble
\let\MetaPrefix\DoubleperCent

\generate{%
  \file{hypbmsec.ins}{\from{hypbmsec.dtx}{install}}%
  \file{hypbmsec.drv}{\from{hypbmsec.dtx}{driver}}%
  \usedir{tex/latex/oberdiek}%
  \file{hypbmsec.sty}{\from{hypbmsec.dtx}{package}}%
  \nopreamble
  \nopostamble
  \usedir{source/latex/oberdiek/catalogue}%
  \file{hypbmsec.xml}{\from{hypbmsec.dtx}{catalogue}}%
}

\catcode32=13\relax% active space
\let =\space%
\Msg{************************************************************************}
\Msg{*}
\Msg{* To finish the installation you have to move the following}
\Msg{* file into a directory searched by TeX:}
\Msg{*}
\Msg{*     hypbmsec.sty}
\Msg{*}
\Msg{* To produce the documentation run the file `hypbmsec.drv'}
\Msg{* through LaTeX.}
\Msg{*}
\Msg{* Happy TeXing!}
\Msg{*}
\Msg{************************************************************************}

\endbatchfile
%</install>
%<*ignore>
\fi
%</ignore>
%<*driver>
\NeedsTeXFormat{LaTeX2e}
\ProvidesFile{hypbmsec.drv}%
  [2016/05/16 v2.5 Bookmarks in sectioning commands (HO)]%
\documentclass{ltxdoc}
\usepackage{holtxdoc}[2011/11/22]
\begin{document}
  \DocInput{hypbmsec.dtx}%
\end{document}
%</driver>
% \fi
%
%
% \CharacterTable
%  {Upper-case    \A\B\C\D\E\F\G\H\I\J\K\L\M\N\O\P\Q\R\S\T\U\V\W\X\Y\Z
%   Lower-case    \a\b\c\d\e\f\g\h\i\j\k\l\m\n\o\p\q\r\s\t\u\v\w\x\y\z
%   Digits        \0\1\2\3\4\5\6\7\8\9
%   Exclamation   \!     Double quote  \"     Hash (number) \#
%   Dollar        \$     Percent       \%     Ampersand     \&
%   Acute accent  \'     Left paren    \(     Right paren   \)
%   Asterisk      \*     Plus          \+     Comma         \,
%   Minus         \-     Point         \.     Solidus       \/
%   Colon         \:     Semicolon     \;     Less than     \<
%   Equals        \=     Greater than  \>     Question mark \?
%   Commercial at \@     Left bracket  \[     Backslash     \\
%   Right bracket \]     Circumflex    \^     Underscore    \_
%   Grave accent  \`     Left brace    \{     Vertical bar  \|
%   Right brace   \}     Tilde         \~}
%
% \GetFileInfo{hypbmsec.drv}
%
% \title{The \xpackage{hypbmsec} package}
% \date{2016/05/16 v2.5}
% \author{Heiko Oberdiek\thanks
% {Please report any issues at https://github.com/ho-tex/oberdiek/issues}\\
% \xemail{heiko.oberdiek at googlemail.com}}
%
% \maketitle
%
% \begin{abstract}
% This package expands the syntax of the sectioning commands. If the
% argument of the sectioning commands isn't usable as outline entry,
% a replacement for the bookmarks can be given.
% \end{abstract}
%
% \tableofcontents
%
% \newcommand{\type}[1]{\textsf{#1}}
%
% ^^A No thread support.
% \newenvironment{article}[1]{}{}
%
% \section{Usage}
%
% \subsection{Bookmarks restrictions}\label{sec:restrictions}
%    Outline entries (bookmarks) are written to a file and have
%    to obey the pdf specification.
%    Therefore they have several restrictions:
%    \begin{itemize}
%    \item Bookmarks have to be encoded in PDFDocEncoding^^A
%          \footnote{\Package{hyperref} doesn't support
%            Unicode.}.
%    \item They should only expand to letters and spaces.
%    \item The result of expansion have to be a valid pdf string.
%    \item Stomach commands like \cmd{\relax}, box commands, math,
%          assignments, or definitions aren't allowed.
%    \item Short entries are recommended, which allow a clear view.
%    \end{itemize}
%
% \subsection{\texorpdfstring{\cmd{\texorpdfstring}}{^^A
%    \textbackslash texorpdfstring}}
%    The generic way in package \Package{hyperref} is the use
%    of \cmd{\texorpdfstring}^^A
%    \footnote{In versions of \Package{hyperref} below 6.54 see
%      \cmd{\ifbookmark}.}:
%    \begin{quote}
%\begin{verbatim}
%\section{Pythagoras:
%  \texorpdfstring{$a^2+b^2=c^2}{%
%    a\texttwosuperior\ + b\texttwosuperior\ =
%    c\texttwosuperior}%
%}
%\end{verbatim}
%    \end{quote}
%
% \subsection{Sectioning commands}
%    The package \Package{hyperref} automatically generates
%    bookmarks from the sectioning commands,
%    unless it is suppressed by an option.
%    Commands that structure the text are here called
%    ``sectioning commands'':
%    \begin{quote}
%    \cmd{\part}, \cmd{\chapter},\\
%    \cmd{\section}, \cmd{\subsection}, \cmd{\subsubsection},\\
%    \cmd{\paragraph}, \cmd{\subparagraph}
%    \end{quote}
%
% \subsection{Places\texorpdfstring{ for sectioning strings}{}}
%    \label{sec:places}
%    The argument(s) of these commands are used on several places:
%    \begin{description}
%    \item[\type{text}]
%      The current text without restrictions.
%    \item[\type{toc}]
%      The headlines and the table of contents with the
%      restrictions of ``moving arguments''.
%    \item[\type{out}]
%      The outlines with many restrictions: The outline
%      have to expand to a valid pdf string with PDFDocEncoding
%      (see \ref{sec:restrictions}).
%    \end{description}
%
% \subsection{\texorpdfstring{Solution with o}{O}ptional arguments}
%    If the user wants to use a footnote within a sectioning command,
%    the \LaTeX{} solution is an optional argument:
%    \begin{quote}
%      |\section[Title]{Title\footnote{Footnote text}}|
%    \end{quote}
%    Now |Title| without the footnote is used in the headlines and
%    the table of contents. Also \Package{hyperref} uses it for the
%    bookmarks.
%
%    This package \Package{\filename} offers two possibilities to
%    specify a separate outline entry:
%    \begin{itemize}
%    \item An additional second optional argument in square brackets.
%    \item An additional optional argument in parentheses (in
%          assoziation with a pdf string that is internally surrounded
%          by parentheses, too).
%    \end{itemize}
%    Because \Package{\filename} stores the original meaning of the
%    sectioning commands and uses them again, there should be no
%    problems with packages that redefine the sectioning commands, if
%    these packages doesn't change the syntax.
%
% \subsection{Syntax}
%    The following examples show the syntax of the sectioning
%    commands. For the places the strings appear the abbreviations
%    are used, that are introduced in \ref{sec:places}.
%
% \subsubsection{\texorpdfstring{Star form}{^^A
%    \textbackslash section*\{\}}}
%    The behaviour of the star form isn't changed. The string
%    appears only in the current text:
%    \begin{article}{Syntax}
%    \begin{quote}
%      |\section*{text}|
%    \end{quote}
%    \end{article}
%
% \subsubsection{\texorpdfstring{Without optional arguments}{^^A
%    \textbackslash section\{\}}}
%    The normal case, the string in the mandatory argument is
%    used for all places:
%    \begin{article}{Syntax}
%    \begin{quote}
%      |\section{text, toc, out}|
%    \end{quote}
%    \end{article}
%
% \subsubsection{\texorpdfstring{One optional argument}{^^A
%    \textbackslash section[]\{\}}}
%    Also the form with one optional parameter in square brackets isn't
%    new; for the bookmarks the optional parameter is used:
%    \begin{article}{Syntax}
%    \begin{quote}
%      |\section[toc, out]{text}|
%    \end{quote}
%    \end{article}
%
% \subsubsection{\texorpdfstring{Two optional arguments}{^^A
%    \textbackslash section[][out]\{\}}}\label{sec:two}
%    The second optional parameter in square brackets is introduced
%    by this package to specify an outline entry:
%    \begin{article}{Syntax}
%    \begin{quote}
%      |\section[toc][out]{text}|
%    \end{quote}
%    \end{article}
%
% \subsubsection{\texorpdfstring{Optional argument in parentheses}{^^A
%    \textbackslash section(out)\{\}}}
%    Often the \type{toc} and the \type{text} string would be the same.
%    With the method of the two optional arguments in square brackets
%    (see \ref{sec:two}) this string must be given twice,
%    if the user only wants to specify a different outline entry.
%    Therefore this package offers another possibility:
%    In association with the internal representation in the pdf file
%    an outline entry can be given in parentheses.
%    So the package can easily distinguish between
%    the two forms of optional arguments and the order does not matter:
%    \begin{article}{Syntax}
%    \begin{quote}
%      |\section(out){toc, text}|\\
%      |\section[toc](out){text}|\\
%      |\section(out)[toc]{text}|
%    \end{quote}
%    \end{article}
%
% \subsection{Without \Package{hyperref}}
%    Package \Package{\filename} uses \Package{hyperref} for support of
%    the bookmarks, but this package is not required.
%    If \Package{hyperref} isn't loaded, or
%    is called with a driver that doesn't support bookmarks,
%    package \Package{\filename} shouldn't be removed,
%    because this would lead to
%    a wrong syntax of the sectioning commands.
%    In any cases package \Package{\filename}
%    supports its syntax and ignores the outline entries,
%    if there are no code for bookmarks.
%    So it is possible to write texts,
%    that are processed with several drivers to get different output
%    formats.
%
% \subsection{Protecting parentheses}
%    If the string itself contains parentheses, they have to be hidden
%    from \TeX's argument parsing mechanism.
%    The argument should be surrounded
%    by curly braces:
%    \begin{quote}
%      |\section({outlines(bookmarks)}){text, toc}|
%    \end{quote}
%    With version 6.54 of \Package{hyperref} the other standard method
%    works, too: The closing parenthesis is protected:
%    \begin{quote}
%      |\section(outlines(bookmarks{)}){text, toc}|
%    \end{quote}
%
% \StopEventually{
% }
%
% \section{Implementation}
%    \begin{macrocode}
%<*package>
%    \end{macrocode}
%    Package identification.
%    \begin{macrocode}
\NeedsTeXFormat{LaTeX2e}
\ProvidesPackage{hypbmsec}%
  [2016/05/16 v2.5 Bookmarks in sectioning commands (HO)]
%    \end{macrocode}
%
%    Because of redifining the sectioning commands, it is dangerous
%    to reload the package several times.
%    \begin{macrocode}
\@ifundefined{hbs@do}{}{%
  \PackageInfo{hypbmsec}{Package 'hypbmsec' is already loaded}%
  \endinput
}
%    \end{macrocode}
%
%    \begin{macro}{\hbs@do}
%    The redefined sectioning commands calls \cmd{\hbs@do}. It does
%    \begin{itemize}
%    \item handle the star case.
%    \item resets the macros that store the entries for the outlines
%          (\cmd{\hbs@bmstring}) and table of contents (\cmd{\hbs@tocstring}).
%    \item store the sectioning command |#1| in \cmd{\hbs@seccmd}
%          for later reuse.
%    \item at last call \cmd{\hbs@checkarg} that scans and interprets the
%          parameters of the redefined sectioning command.
%    \end{itemize}
%    \begin{macrocode}
\def\hbs@do#1{%
  \@ifstar{#1*}{%
    \let\hbs@tocstring\relax
    \let\hbs@bmstring\relax
    \let\hbs@seccmd#1%
    \hbs@checkarg
  }%
}
%    \end{macrocode}
%    \end{macro}
%
%    \begin{macro}{\hbs@checkarg}
%    \cmd{\hbs@checkarg} determines the type of the next argument:
%    \begin{itemize}
%    \item An optional argument in square brackets can be an entry
%          for the table of contents or the bookmarks. It will be
%          read by \cmd{\hbs@getsquare}
%    \item An optional argument in parentheses is an outline entry.
%          This is worked off by \cmd{\hbs@getbookmark}.
%    \item If there are no more optional arguments, \cmd{\hbs@process}
%          reads the mandatory argument and calls the original
%          sectioning commands.
%    \end{itemize}
%    \begin{macrocode}
\def\hbs@checkarg{%
  \@ifnextchar[\hbs@getsquare{%
    \@ifnextchar(\hbs@getbookmark\hbs@process
  }%
}
%    \end{macrocode}
%    \end{macro}
%
%    \begin{macro}{\hbs@getsquare}
%    \cmd{\hbs@getsquare} reads an optional argument in square
%    brackets and determines, if this is an entry for the table
%    of contents or the bookmarks.
%    \begin{macrocode}
\long\def\hbs@getsquare[#1]{%
  \ifx\hbs@tocstring\relax
    \def\hbs@tocstring{#1}%
  \else
    \hbs@bmdef{#1}%
  \fi
  \hbs@checkarg
}
%    \end{macrocode}
%    \end{macro}
%
%    \begin{macro}{\hbs@getbookmark}
%    \cmd{\hbs@getbookmark} reads an outline entry in parentheses.
%    \begin{macrocode}
\def\hbs@getbookmark(#1){%
  \hbs@bmdef{#1}%
  \hbs@checkarg
}
%    \end{macrocode}
%    \end{macro}
%
%    \begin{macro}{\hbs@bmdef}
%    The command \cmd{\hbs@bmdef} save the bookmark entry in
%    parameter |#1| in the macro \cmd{\hbs@bmstring} and catches
%    the case, if the user has given several outline strings.
%    \begin{macrocode}
\def\hbs@bmdef#1{%
  \ifx\hbs@bmstring\relax
    \def\hbs@bmstring{#1}%
  \else
    \PackageError{hypbmsec}{%
      Sectioning command with too many parameters%
    }{%
      You can only give one outline entry.%
    }%
  \fi
}
%    \end{macrocode}
%    \end{macro}
%
%    \begin{macro}{\hbs@process}
%    The parameter |#1| is the mandatory argument of the sectioning
%    commands. \cmd{\hbs@process} calls the original sectioning command
%    stored in \cmd{\hbs@seccmd} with arguments that depend of which
%    optional argument are used previously.
%    \begin{macrocode}
\long\def\hbs@process#1{%
  \ifx\hbs@tocstring\relax
    \ifx\hbs@bmstring\relax
      \hbs@seccmd{#1}%
    \else
      \begingroup
        \def\x##1{\endgroup
          \hbs@seccmd{\texorpdfstring{#1}{##1}}%
        }%
      \expandafter\x\expandafter{\hbs@bmstring}%
    \fi
  \else
    \ifx\hbs@bmstring\relax
      \expandafter\hbs@seccmd\expandafter[%
        \expandafter{\hbs@tocstring}%
      ]{#1}%
    \else
      \expandafter\expandafter\expandafter
      \hbs@seccmd\expandafter\expandafter\expandafter[%
        \expandafter\expandafter\expandafter
        \texorpdfstring
        \expandafter\expandafter\expandafter{%
          \expandafter\hbs@tocstring\expandafter
        }\expandafter{%
          \hbs@bmstring
        }%
      ]{#1}%
    \fi
  \fi
}
%    \end{macrocode}
%    \end{macro}
%
%    We have to check, whether package \Package{hyperref} is loaded
%    and have to provide a definition for \cmd{\texorpdfstring}.
%    Because \Package{hyperref} can be loaded after this package,
%    we do the work later (\cmd{\AtBeginDocument}).
%
%    This code only checks versions of \Package{hyperref} that
%    define \cmd{\ifbookmark} (v6.4x until v6.53) or
%    \cmd{\texorpdfstring} (v6.54 and above). Older versions aren't
%    supported.
%    \begin{macrocode}
\AtBeginDocument{%
  \@ifundefined{texorpdfstring}{%
    \@ifundefined{ifbookmark}{%
      \let\texorpdfstring\@firstoftwo
      \@ifpackageloaded{hyperref}{%
        \PackageInfo{hypbmsec}{%
          \ifx\hy@driver\@empty
            Default driver %
          \else
            '\hy@driver' %
          \fi
          of hyperref not supported,\MessageBreak
          bookmark parameters will be ignored%
        }%
      }{%
        \PackageInfo{hypbmsec}{%
          Package hyperref not loaded,\MessageBreak
          bookmark parameters will be ignored%
        }%
      }%
    }%
    {%
      \newcommand\texorpdfstring[2]{\ifbookmark{#2}{#1}}%
      \PackageWarningNoLine{hypbmsec}{%
        Old hyperref version found,\MessageBreak
        update of hyperref recommended%
      }%
    }%
  }{}%
%    \end{macrocode}
%
%    Other packages are allowed to redefine the sectioning commands,
%    if they does not change the syntax. Therefore the redefinitons
%    of this package should be done after the other packages.
%    \begin{macrocode}
  \let\hbs@part         \part
  \let\hbs@section      \section
  \let\hbs@subsection   \subsection
  \let\hbs@subsubsection\subsubsection
  \let\hbs@paragraph    \paragraph
  \let\hbs@subparagraph \subparagraph
  \renewcommand\part         {\hbs@do\hbs@part}%
  \renewcommand\section      {\hbs@do\hbs@section}%
  \renewcommand\subsection   {\hbs@do\hbs@subsection}%
  \renewcommand\subsubsection{\hbs@do\hbs@subsubsection}%
  \renewcommand\paragraph    {\hbs@do\hbs@paragraph}%
  \renewcommand\subparagraph {\hbs@do\hbs@subparagraph}%
  \begingroup\expandafter\expandafter\expandafter\endgroup
  \expandafter\ifx\csname chapter\endcsname\relax\else
    \let\hbs@chapter    \chapter
    \renewcommand\chapter    {\hbs@do\hbs@chapter}%
  \fi
}
%    \end{macrocode}
%
%    \begin{macrocode}
%</package>
%    \end{macrocode}
%
% \section{Installation}
%
% \subsection{Download}
%
% \paragraph{Package.} This package is available on
% CTAN\footnote{\url{http://ctan.org/pkg/hypbmsec}}:
% \begin{description}
% \item[\CTAN{macros/latex/contrib/oberdiek/hypbmsec.dtx}] The source file.
% \item[\CTAN{macros/latex/contrib/oberdiek/hypbmsec.pdf}] Documentation.
% \end{description}
%
%
% \paragraph{Bundle.} All the packages of the bundle `oberdiek'
% are also available in a TDS compliant ZIP archive. There
% the packages are already unpacked and the documentation files
% are generated. The files and directories obey the TDS standard.
% \begin{description}
% \item[\CTAN{install/macros/latex/contrib/oberdiek.tds.zip}]
% \end{description}
% \emph{TDS} refers to the standard ``A Directory Structure
% for \TeX\ Files'' (\CTAN{tds/tds.pdf}). Directories
% with \xfile{texmf} in their name are usually organized this way.
%
% \subsection{Bundle installation}
%
% \paragraph{Unpacking.} Unpack the \xfile{oberdiek.tds.zip} in the
% TDS tree (also known as \xfile{texmf} tree) of your choice.
% Example (linux):
% \begin{quote}
%   |unzip oberdiek.tds.zip -d ~/texmf|
% \end{quote}
%
% \paragraph{Script installation.}
% Check the directory \xfile{TDS:scripts/oberdiek/} for
% scripts that need further installation steps.
% Package \xpackage{attachfile2} comes with the Perl script
% \xfile{pdfatfi.pl} that should be installed in such a way
% that it can be called as \texttt{pdfatfi}.
% Example (linux):
% \begin{quote}
%   |chmod +x scripts/oberdiek/pdfatfi.pl|\\
%   |cp scripts/oberdiek/pdfatfi.pl /usr/local/bin/|
% \end{quote}
%
% \subsection{Package installation}
%
% \paragraph{Unpacking.} The \xfile{.dtx} file is a self-extracting
% \docstrip\ archive. The files are extracted by running the
% \xfile{.dtx} through \plainTeX:
% \begin{quote}
%   \verb|tex hypbmsec.dtx|
% \end{quote}
%
% \paragraph{TDS.} Now the different files must be moved into
% the different directories in your installation TDS tree
% (also known as \xfile{texmf} tree):
% \begin{quote}
% \def\t{^^A
% \begin{tabular}{@{}>{\ttfamily}l@{ $\rightarrow$ }>{\ttfamily}l@{}}
%   hypbmsec.sty & tex/latex/oberdiek/hypbmsec.sty\\
%   hypbmsec.pdf & doc/latex/oberdiek/hypbmsec.pdf\\
%   hypbmsec.dtx & source/latex/oberdiek/hypbmsec.dtx\\
% \end{tabular}^^A
% }^^A
% \sbox0{\t}^^A
% \ifdim\wd0>\linewidth
%   \begingroup
%     \advance\linewidth by\leftmargin
%     \advance\linewidth by\rightmargin
%   \edef\x{\endgroup
%     \def\noexpand\lw{\the\linewidth}^^A
%   }\x
%   \def\lwbox{^^A
%     \leavevmode
%     \hbox to \linewidth{^^A
%       \kern-\leftmargin\relax
%       \hss
%       \usebox0
%       \hss
%       \kern-\rightmargin\relax
%     }^^A
%   }^^A
%   \ifdim\wd0>\lw
%     \sbox0{\small\t}^^A
%     \ifdim\wd0>\linewidth
%       \ifdim\wd0>\lw
%         \sbox0{\footnotesize\t}^^A
%         \ifdim\wd0>\linewidth
%           \ifdim\wd0>\lw
%             \sbox0{\scriptsize\t}^^A
%             \ifdim\wd0>\linewidth
%               \ifdim\wd0>\lw
%                 \sbox0{\tiny\t}^^A
%                 \ifdim\wd0>\linewidth
%                   \lwbox
%                 \else
%                   \usebox0
%                 \fi
%               \else
%                 \lwbox
%               \fi
%             \else
%               \usebox0
%             \fi
%           \else
%             \lwbox
%           \fi
%         \else
%           \usebox0
%         \fi
%       \else
%         \lwbox
%       \fi
%     \else
%       \usebox0
%     \fi
%   \else
%     \lwbox
%   \fi
% \else
%   \usebox0
% \fi
% \end{quote}
% If you have a \xfile{docstrip.cfg} that configures and enables \docstrip's
% TDS installing feature, then some files can already be in the right
% place, see the documentation of \docstrip.
%
% \subsection{Refresh file name databases}
%
% If your \TeX~distribution
% (\teTeX, \mikTeX, \dots) relies on file name databases, you must refresh
% these. For example, \teTeX\ users run \verb|texhash| or
% \verb|mktexlsr|.
%
% \subsection{Some details for the interested}
%
% \paragraph{Attached source.}
%
% The PDF documentation on CTAN also includes the
% \xfile{.dtx} source file. It can be extracted by
% AcrobatReader 6 or higher. Another option is \textsf{pdftk},
% e.g. unpack the file into the current directory:
% \begin{quote}
%   \verb|pdftk hypbmsec.pdf unpack_files output .|
% \end{quote}
%
% \paragraph{Unpacking with \LaTeX.}
% The \xfile{.dtx} chooses its action depending on the format:
% \begin{description}
% \item[\plainTeX:] Run \docstrip\ and extract the files.
% \item[\LaTeX:] Generate the documentation.
% \end{description}
% If you insist on using \LaTeX\ for \docstrip\ (really,
% \docstrip\ does not need \LaTeX), then inform the autodetect routine
% about your intention:
% \begin{quote}
%   \verb|latex \let\install=y\input{hypbmsec.dtx}|
% \end{quote}
% Do not forget to quote the argument according to the demands
% of your shell.
%
% \paragraph{Generating the documentation.}
% You can use both the \xfile{.dtx} or the \xfile{.drv} to generate
% the documentation. The process can be configured by the
% configuration file \xfile{ltxdoc.cfg}. For instance, put this
% line into this file, if you want to have A4 as paper format:
% \begin{quote}
%   \verb|\PassOptionsToClass{a4paper}{article}|
% \end{quote}
% An example follows how to generate the
% documentation with pdf\LaTeX:
% \begin{quote}
%\begin{verbatim}
%pdflatex hypbmsec.dtx
%makeindex -s gind.ist hypbmsec.idx
%pdflatex hypbmsec.dtx
%makeindex -s gind.ist hypbmsec.idx
%pdflatex hypbmsec.dtx
%\end{verbatim}
% \end{quote}
%
% \section{Catalogue}
%
% The following XML file can be used as source for the
% \href{http://mirror.ctan.org/help/Catalogue/catalogue.html}{\TeX\ Catalogue}.
% The elements \texttt{caption} and \texttt{description} are imported
% from the original XML file from the Catalogue.
% The name of the XML file in the Catalogue is \xfile{hypbmsec.xml}.
%    \begin{macrocode}
%<*catalogue>
<?xml version='1.0' encoding='us-ascii'?>
<!DOCTYPE entry SYSTEM 'catalogue.dtd'>
<entry datestamp='$Date$' modifier='$Author$' id='hypbmsec'>
  <name>hypbmsec</name>
  <caption>Hypertext bookmarks in sectioning commands.</caption>
  <authorref id='auth:oberdiek'/>
  <copyright owner='Heiko Oberdiek' year='1998-2000,2006,2007'/>
  <license type='lppl1.3'/>
  <version number='2.5'/>
  <description>
    Bookmark entries can be given as another argument to the LaTeX
    sectioning commands. The <xref refid='hyperref'>hyperref</xref>
    package is required to get the bookmarks, but the syntax
    works without it.
    <p/>
    This package is part of the <xref refid='oberdiek'>oberdiek</xref>
    bundle.
  </description>
  <documentation details='Package documentation'
      href='ctan:/macros/latex/contrib/oberdiek/hypbmsec.pdf'/>
  <ctan file='true' path='/macros/latex/contrib/oberdiek/hypbmsec.dtx'/>
  <miktex location='oberdiek'/>
  <texlive location='oberdiek'/>
  <install path='/macros/latex/contrib/oberdiek/oberdiek.tds.zip'/>
</entry>
%</catalogue>
%    \end{macrocode}
%
% \begin{History}
%   \begin{Version}{1998/11/20 v1.0}
%   \item
%     First version.
%   \item
%     It merges package \xpackage{hysecopt} and
%   \item
%     package \xpackage{hypbmpar}.
%   \item
%     Published for the DANTE'99 meeting^^A
%     \URL{}{http://dante99.cs.uni-dortmund.de/handouts/oberdiek/hypbmsec.sty}.
%   \end{Version}
%   \begin{Version}{1999/04/12 v2.0}
%   \item
%     Adaptation to \Package{hyperref} version 6.54.
%   \item
%     Documentation in dtx format.
%   \item
%     Copyright: LPPL (\CTAN{macros/latex/base/lppl.txt})
%   \item
%     First CTAN release.
%   \end{Version}
%   \begin{Version}{2000/03/22 v2.1}
%   \item
%     Bug fix in redefinition of \cmd{\chapter}.
%   \item
%     Copyright: LPPL 1.2
%   \end{Version}
%   \begin{Version}{2006/02/20 v2.2}
%   \item
%     Code is not changed.
%   \item
%     New DTX framework.
%   \item
%     LPPL 1.3
%   \end{Version}
%   \begin{Version}{2007/03/05 v2.3}
%   \item
%     Bug fix: Expand \cs{hbs@tocstring} and \cs{hbs@bmstring} before
%     calling \cs{hbs@seccmd}.
%   \end{Version}
%   \begin{Version}{2007/04/11 v2.4}
%   \item
%     Line ends sanitized.
%   \end{Version}
%   \begin{Version}{2016/05/16 v2.5}
%   \item
%     Documentation updates.
%   \end{Version}
% \end{History}
%
% \PrintIndex
%
% \Finale
\endinput

%        (quote the arguments according to the demands of your shell)
%
% Documentation:
%    (a) If hypbmsec.drv is present:
%           latex hypbmsec.drv
%    (b) Without hypbmsec.drv:
%           latex hypbmsec.dtx; ...
%    The class ltxdoc loads the configuration file ltxdoc.cfg
%    if available. Here you can specify further options, e.g.
%    use A4 as paper format:
%       \PassOptionsToClass{a4paper}{article}
%
%    Programm calls to get the documentation (example):
%       pdflatex hypbmsec.dtx
%       makeindex -s gind.ist hypbmsec.idx
%       pdflatex hypbmsec.dtx
%       makeindex -s gind.ist hypbmsec.idx
%       pdflatex hypbmsec.dtx
%
% Installation:
%    TDS:tex/latex/oberdiek/hypbmsec.sty
%    TDS:doc/latex/oberdiek/hypbmsec.pdf
%    TDS:source/latex/oberdiek/hypbmsec.dtx
%
%<*ignore>
\begingroup
  \catcode123=1 %
  \catcode125=2 %
  \def\x{LaTeX2e}%
\expandafter\endgroup
\ifcase 0\ifx\install y1\fi\expandafter
         \ifx\csname processbatchFile\endcsname\relax\else1\fi
         \ifx\fmtname\x\else 1\fi\relax
\else\csname fi\endcsname
%</ignore>
%<*install>
\input docstrip.tex
\Msg{************************************************************************}
\Msg{* Installation}
\Msg{* Package: hypbmsec 2016/05/16 v2.5 Bookmarks in sectioning commands (HO)}
\Msg{************************************************************************}

\keepsilent
\askforoverwritefalse

\let\MetaPrefix\relax
\preamble

This is a generated file.

Project: hypbmsec
Version: 2016/05/16 v2.5

Copyright (C) 1998-2000, 2006, 2007 by
   Heiko Oberdiek <heiko.oberdiek at googlemail.com>

This work may be distributed and/or modified under the
conditions of the LaTeX Project Public License, either
version 1.3c of this license or (at your option) any later
version. This version of this license is in
   http://www.latex-project.org/lppl/lppl-1-3c.txt
and the latest version of this license is in
   http://www.latex-project.org/lppl.txt
and version 1.3 or later is part of all distributions of
LaTeX version 2005/12/01 or later.

This work has the LPPL maintenance status "maintained".

This Current Maintainer of this work is Heiko Oberdiek.

This work consists of the main source file hypbmsec.dtx
and the derived files
   hypbmsec.sty, hypbmsec.pdf, hypbmsec.ins, hypbmsec.drv.

\endpreamble
\let\MetaPrefix\DoubleperCent

\generate{%
  \file{hypbmsec.ins}{\from{hypbmsec.dtx}{install}}%
  \file{hypbmsec.drv}{\from{hypbmsec.dtx}{driver}}%
  \usedir{tex/latex/oberdiek}%
  \file{hypbmsec.sty}{\from{hypbmsec.dtx}{package}}%
  \nopreamble
  \nopostamble
  \usedir{source/latex/oberdiek/catalogue}%
  \file{hypbmsec.xml}{\from{hypbmsec.dtx}{catalogue}}%
}

\catcode32=13\relax% active space
\let =\space%
\Msg{************************************************************************}
\Msg{*}
\Msg{* To finish the installation you have to move the following}
\Msg{* file into a directory searched by TeX:}
\Msg{*}
\Msg{*     hypbmsec.sty}
\Msg{*}
\Msg{* To produce the documentation run the file `hypbmsec.drv'}
\Msg{* through LaTeX.}
\Msg{*}
\Msg{* Happy TeXing!}
\Msg{*}
\Msg{************************************************************************}

\endbatchfile
%</install>
%<*ignore>
\fi
%</ignore>
%<*driver>
\NeedsTeXFormat{LaTeX2e}
\ProvidesFile{hypbmsec.drv}%
  [2016/05/16 v2.5 Bookmarks in sectioning commands (HO)]%
\documentclass{ltxdoc}
\usepackage{holtxdoc}[2011/11/22]
\begin{document}
  \DocInput{hypbmsec.dtx}%
\end{document}
%</driver>
% \fi
%
%
% \CharacterTable
%  {Upper-case    \A\B\C\D\E\F\G\H\I\J\K\L\M\N\O\P\Q\R\S\T\U\V\W\X\Y\Z
%   Lower-case    \a\b\c\d\e\f\g\h\i\j\k\l\m\n\o\p\q\r\s\t\u\v\w\x\y\z
%   Digits        \0\1\2\3\4\5\6\7\8\9
%   Exclamation   \!     Double quote  \"     Hash (number) \#
%   Dollar        \$     Percent       \%     Ampersand     \&
%   Acute accent  \'     Left paren    \(     Right paren   \)
%   Asterisk      \*     Plus          \+     Comma         \,
%   Minus         \-     Point         \.     Solidus       \/
%   Colon         \:     Semicolon     \;     Less than     \<
%   Equals        \=     Greater than  \>     Question mark \?
%   Commercial at \@     Left bracket  \[     Backslash     \\
%   Right bracket \]     Circumflex    \^     Underscore    \_
%   Grave accent  \`     Left brace    \{     Vertical bar  \|
%   Right brace   \}     Tilde         \~}
%
% \GetFileInfo{hypbmsec.drv}
%
% \title{The \xpackage{hypbmsec} package}
% \date{2016/05/16 v2.5}
% \author{Heiko Oberdiek\thanks
% {Please report any issues at https://github.com/ho-tex/oberdiek/issues}\\
% \xemail{heiko.oberdiek at googlemail.com}}
%
% \maketitle
%
% \begin{abstract}
% This package expands the syntax of the sectioning commands. If the
% argument of the sectioning commands isn't usable as outline entry,
% a replacement for the bookmarks can be given.
% \end{abstract}
%
% \tableofcontents
%
% \newcommand{\type}[1]{\textsf{#1}}
%
% ^^A No thread support.
% \newenvironment{article}[1]{}{}
%
% \section{Usage}
%
% \subsection{Bookmarks restrictions}\label{sec:restrictions}
%    Outline entries (bookmarks) are written to a file and have
%    to obey the pdf specification.
%    Therefore they have several restrictions:
%    \begin{itemize}
%    \item Bookmarks have to be encoded in PDFDocEncoding^^A
%          \footnote{\Package{hyperref} doesn't support
%            Unicode.}.
%    \item They should only expand to letters and spaces.
%    \item The result of expansion have to be a valid pdf string.
%    \item Stomach commands like \cmd{\relax}, box commands, math,
%          assignments, or definitions aren't allowed.
%    \item Short entries are recommended, which allow a clear view.
%    \end{itemize}
%
% \subsection{\texorpdfstring{\cmd{\texorpdfstring}}{^^A
%    \textbackslash texorpdfstring}}
%    The generic way in package \Package{hyperref} is the use
%    of \cmd{\texorpdfstring}^^A
%    \footnote{In versions of \Package{hyperref} below 6.54 see
%      \cmd{\ifbookmark}.}:
%    \begin{quote}
%\begin{verbatim}
%\section{Pythagoras:
%  \texorpdfstring{$a^2+b^2=c^2}{%
%    a\texttwosuperior\ + b\texttwosuperior\ =
%    c\texttwosuperior}%
%}
%\end{verbatim}
%    \end{quote}
%
% \subsection{Sectioning commands}
%    The package \Package{hyperref} automatically generates
%    bookmarks from the sectioning commands,
%    unless it is suppressed by an option.
%    Commands that structure the text are here called
%    ``sectioning commands'':
%    \begin{quote}
%    \cmd{\part}, \cmd{\chapter},\\
%    \cmd{\section}, \cmd{\subsection}, \cmd{\subsubsection},\\
%    \cmd{\paragraph}, \cmd{\subparagraph}
%    \end{quote}
%
% \subsection{Places\texorpdfstring{ for sectioning strings}{}}
%    \label{sec:places}
%    The argument(s) of these commands are used on several places:
%    \begin{description}
%    \item[\type{text}]
%      The current text without restrictions.
%    \item[\type{toc}]
%      The headlines and the table of contents with the
%      restrictions of ``moving arguments''.
%    \item[\type{out}]
%      The outlines with many restrictions: The outline
%      have to expand to a valid pdf string with PDFDocEncoding
%      (see \ref{sec:restrictions}).
%    \end{description}
%
% \subsection{\texorpdfstring{Solution with o}{O}ptional arguments}
%    If the user wants to use a footnote within a sectioning command,
%    the \LaTeX{} solution is an optional argument:
%    \begin{quote}
%      |\section[Title]{Title\footnote{Footnote text}}|
%    \end{quote}
%    Now |Title| without the footnote is used in the headlines and
%    the table of contents. Also \Package{hyperref} uses it for the
%    bookmarks.
%
%    This package \Package{\filename} offers two possibilities to
%    specify a separate outline entry:
%    \begin{itemize}
%    \item An additional second optional argument in square brackets.
%    \item An additional optional argument in parentheses (in
%          assoziation with a pdf string that is internally surrounded
%          by parentheses, too).
%    \end{itemize}
%    Because \Package{\filename} stores the original meaning of the
%    sectioning commands and uses them again, there should be no
%    problems with packages that redefine the sectioning commands, if
%    these packages doesn't change the syntax.
%
% \subsection{Syntax}
%    The following examples show the syntax of the sectioning
%    commands. For the places the strings appear the abbreviations
%    are used, that are introduced in \ref{sec:places}.
%
% \subsubsection{\texorpdfstring{Star form}{^^A
%    \textbackslash section*\{\}}}
%    The behaviour of the star form isn't changed. The string
%    appears only in the current text:
%    \begin{article}{Syntax}
%    \begin{quote}
%      |\section*{text}|
%    \end{quote}
%    \end{article}
%
% \subsubsection{\texorpdfstring{Without optional arguments}{^^A
%    \textbackslash section\{\}}}
%    The normal case, the string in the mandatory argument is
%    used for all places:
%    \begin{article}{Syntax}
%    \begin{quote}
%      |\section{text, toc, out}|
%    \end{quote}
%    \end{article}
%
% \subsubsection{\texorpdfstring{One optional argument}{^^A
%    \textbackslash section[]\{\}}}
%    Also the form with one optional parameter in square brackets isn't
%    new; for the bookmarks the optional parameter is used:
%    \begin{article}{Syntax}
%    \begin{quote}
%      |\section[toc, out]{text}|
%    \end{quote}
%    \end{article}
%
% \subsubsection{\texorpdfstring{Two optional arguments}{^^A
%    \textbackslash section[][out]\{\}}}\label{sec:two}
%    The second optional parameter in square brackets is introduced
%    by this package to specify an outline entry:
%    \begin{article}{Syntax}
%    \begin{quote}
%      |\section[toc][out]{text}|
%    \end{quote}
%    \end{article}
%
% \subsubsection{\texorpdfstring{Optional argument in parentheses}{^^A
%    \textbackslash section(out)\{\}}}
%    Often the \type{toc} and the \type{text} string would be the same.
%    With the method of the two optional arguments in square brackets
%    (see \ref{sec:two}) this string must be given twice,
%    if the user only wants to specify a different outline entry.
%    Therefore this package offers another possibility:
%    In association with the internal representation in the pdf file
%    an outline entry can be given in parentheses.
%    So the package can easily distinguish between
%    the two forms of optional arguments and the order does not matter:
%    \begin{article}{Syntax}
%    \begin{quote}
%      |\section(out){toc, text}|\\
%      |\section[toc](out){text}|\\
%      |\section(out)[toc]{text}|
%    \end{quote}
%    \end{article}
%
% \subsection{Without \Package{hyperref}}
%    Package \Package{\filename} uses \Package{hyperref} for support of
%    the bookmarks, but this package is not required.
%    If \Package{hyperref} isn't loaded, or
%    is called with a driver that doesn't support bookmarks,
%    package \Package{\filename} shouldn't be removed,
%    because this would lead to
%    a wrong syntax of the sectioning commands.
%    In any cases package \Package{\filename}
%    supports its syntax and ignores the outline entries,
%    if there are no code for bookmarks.
%    So it is possible to write texts,
%    that are processed with several drivers to get different output
%    formats.
%
% \subsection{Protecting parentheses}
%    If the string itself contains parentheses, they have to be hidden
%    from \TeX's argument parsing mechanism.
%    The argument should be surrounded
%    by curly braces:
%    \begin{quote}
%      |\section({outlines(bookmarks)}){text, toc}|
%    \end{quote}
%    With version 6.54 of \Package{hyperref} the other standard method
%    works, too: The closing parenthesis is protected:
%    \begin{quote}
%      |\section(outlines(bookmarks{)}){text, toc}|
%    \end{quote}
%
% \StopEventually{
% }
%
% \section{Implementation}
%    \begin{macrocode}
%<*package>
%    \end{macrocode}
%    Package identification.
%    \begin{macrocode}
\NeedsTeXFormat{LaTeX2e}
\ProvidesPackage{hypbmsec}%
  [2016/05/16 v2.5 Bookmarks in sectioning commands (HO)]
%    \end{macrocode}
%
%    Because of redifining the sectioning commands, it is dangerous
%    to reload the package several times.
%    \begin{macrocode}
\@ifundefined{hbs@do}{}{%
  \PackageInfo{hypbmsec}{Package 'hypbmsec' is already loaded}%
  \endinput
}
%    \end{macrocode}
%
%    \begin{macro}{\hbs@do}
%    The redefined sectioning commands calls \cmd{\hbs@do}. It does
%    \begin{itemize}
%    \item handle the star case.
%    \item resets the macros that store the entries for the outlines
%          (\cmd{\hbs@bmstring}) and table of contents (\cmd{\hbs@tocstring}).
%    \item store the sectioning command |#1| in \cmd{\hbs@seccmd}
%          for later reuse.
%    \item at last call \cmd{\hbs@checkarg} that scans and interprets the
%          parameters of the redefined sectioning command.
%    \end{itemize}
%    \begin{macrocode}
\def\hbs@do#1{%
  \@ifstar{#1*}{%
    \let\hbs@tocstring\relax
    \let\hbs@bmstring\relax
    \let\hbs@seccmd#1%
    \hbs@checkarg
  }%
}
%    \end{macrocode}
%    \end{macro}
%
%    \begin{macro}{\hbs@checkarg}
%    \cmd{\hbs@checkarg} determines the type of the next argument:
%    \begin{itemize}
%    \item An optional argument in square brackets can be an entry
%          for the table of contents or the bookmarks. It will be
%          read by \cmd{\hbs@getsquare}
%    \item An optional argument in parentheses is an outline entry.
%          This is worked off by \cmd{\hbs@getbookmark}.
%    \item If there are no more optional arguments, \cmd{\hbs@process}
%          reads the mandatory argument and calls the original
%          sectioning commands.
%    \end{itemize}
%    \begin{macrocode}
\def\hbs@checkarg{%
  \@ifnextchar[\hbs@getsquare{%
    \@ifnextchar(\hbs@getbookmark\hbs@process
  }%
}
%    \end{macrocode}
%    \end{macro}
%
%    \begin{macro}{\hbs@getsquare}
%    \cmd{\hbs@getsquare} reads an optional argument in square
%    brackets and determines, if this is an entry for the table
%    of contents or the bookmarks.
%    \begin{macrocode}
\long\def\hbs@getsquare[#1]{%
  \ifx\hbs@tocstring\relax
    \def\hbs@tocstring{#1}%
  \else
    \hbs@bmdef{#1}%
  \fi
  \hbs@checkarg
}
%    \end{macrocode}
%    \end{macro}
%
%    \begin{macro}{\hbs@getbookmark}
%    \cmd{\hbs@getbookmark} reads an outline entry in parentheses.
%    \begin{macrocode}
\def\hbs@getbookmark(#1){%
  \hbs@bmdef{#1}%
  \hbs@checkarg
}
%    \end{macrocode}
%    \end{macro}
%
%    \begin{macro}{\hbs@bmdef}
%    The command \cmd{\hbs@bmdef} save the bookmark entry in
%    parameter |#1| in the macro \cmd{\hbs@bmstring} and catches
%    the case, if the user has given several outline strings.
%    \begin{macrocode}
\def\hbs@bmdef#1{%
  \ifx\hbs@bmstring\relax
    \def\hbs@bmstring{#1}%
  \else
    \PackageError{hypbmsec}{%
      Sectioning command with too many parameters%
    }{%
      You can only give one outline entry.%
    }%
  \fi
}
%    \end{macrocode}
%    \end{macro}
%
%    \begin{macro}{\hbs@process}
%    The parameter |#1| is the mandatory argument of the sectioning
%    commands. \cmd{\hbs@process} calls the original sectioning command
%    stored in \cmd{\hbs@seccmd} with arguments that depend of which
%    optional argument are used previously.
%    \begin{macrocode}
\long\def\hbs@process#1{%
  \ifx\hbs@tocstring\relax
    \ifx\hbs@bmstring\relax
      \hbs@seccmd{#1}%
    \else
      \begingroup
        \def\x##1{\endgroup
          \hbs@seccmd{\texorpdfstring{#1}{##1}}%
        }%
      \expandafter\x\expandafter{\hbs@bmstring}%
    \fi
  \else
    \ifx\hbs@bmstring\relax
      \expandafter\hbs@seccmd\expandafter[%
        \expandafter{\hbs@tocstring}%
      ]{#1}%
    \else
      \expandafter\expandafter\expandafter
      \hbs@seccmd\expandafter\expandafter\expandafter[%
        \expandafter\expandafter\expandafter
        \texorpdfstring
        \expandafter\expandafter\expandafter{%
          \expandafter\hbs@tocstring\expandafter
        }\expandafter{%
          \hbs@bmstring
        }%
      ]{#1}%
    \fi
  \fi
}
%    \end{macrocode}
%    \end{macro}
%
%    We have to check, whether package \Package{hyperref} is loaded
%    and have to provide a definition for \cmd{\texorpdfstring}.
%    Because \Package{hyperref} can be loaded after this package,
%    we do the work later (\cmd{\AtBeginDocument}).
%
%    This code only checks versions of \Package{hyperref} that
%    define \cmd{\ifbookmark} (v6.4x until v6.53) or
%    \cmd{\texorpdfstring} (v6.54 and above). Older versions aren't
%    supported.
%    \begin{macrocode}
\AtBeginDocument{%
  \@ifundefined{texorpdfstring}{%
    \@ifundefined{ifbookmark}{%
      \let\texorpdfstring\@firstoftwo
      \@ifpackageloaded{hyperref}{%
        \PackageInfo{hypbmsec}{%
          \ifx\hy@driver\@empty
            Default driver %
          \else
            '\hy@driver' %
          \fi
          of hyperref not supported,\MessageBreak
          bookmark parameters will be ignored%
        }%
      }{%
        \PackageInfo{hypbmsec}{%
          Package hyperref not loaded,\MessageBreak
          bookmark parameters will be ignored%
        }%
      }%
    }%
    {%
      \newcommand\texorpdfstring[2]{\ifbookmark{#2}{#1}}%
      \PackageWarningNoLine{hypbmsec}{%
        Old hyperref version found,\MessageBreak
        update of hyperref recommended%
      }%
    }%
  }{}%
%    \end{macrocode}
%
%    Other packages are allowed to redefine the sectioning commands,
%    if they does not change the syntax. Therefore the redefinitons
%    of this package should be done after the other packages.
%    \begin{macrocode}
  \let\hbs@part         \part
  \let\hbs@section      \section
  \let\hbs@subsection   \subsection
  \let\hbs@subsubsection\subsubsection
  \let\hbs@paragraph    \paragraph
  \let\hbs@subparagraph \subparagraph
  \renewcommand\part         {\hbs@do\hbs@part}%
  \renewcommand\section      {\hbs@do\hbs@section}%
  \renewcommand\subsection   {\hbs@do\hbs@subsection}%
  \renewcommand\subsubsection{\hbs@do\hbs@subsubsection}%
  \renewcommand\paragraph    {\hbs@do\hbs@paragraph}%
  \renewcommand\subparagraph {\hbs@do\hbs@subparagraph}%
  \begingroup\expandafter\expandafter\expandafter\endgroup
  \expandafter\ifx\csname chapter\endcsname\relax\else
    \let\hbs@chapter    \chapter
    \renewcommand\chapter    {\hbs@do\hbs@chapter}%
  \fi
}
%    \end{macrocode}
%
%    \begin{macrocode}
%</package>
%    \end{macrocode}
%
% \section{Installation}
%
% \subsection{Download}
%
% \paragraph{Package.} This package is available on
% CTAN\footnote{\url{http://ctan.org/pkg/hypbmsec}}:
% \begin{description}
% \item[\CTAN{macros/latex/contrib/oberdiek/hypbmsec.dtx}] The source file.
% \item[\CTAN{macros/latex/contrib/oberdiek/hypbmsec.pdf}] Documentation.
% \end{description}
%
%
% \paragraph{Bundle.} All the packages of the bundle `oberdiek'
% are also available in a TDS compliant ZIP archive. There
% the packages are already unpacked and the documentation files
% are generated. The files and directories obey the TDS standard.
% \begin{description}
% \item[\CTAN{install/macros/latex/contrib/oberdiek.tds.zip}]
% \end{description}
% \emph{TDS} refers to the standard ``A Directory Structure
% for \TeX\ Files'' (\CTAN{tds/tds.pdf}). Directories
% with \xfile{texmf} in their name are usually organized this way.
%
% \subsection{Bundle installation}
%
% \paragraph{Unpacking.} Unpack the \xfile{oberdiek.tds.zip} in the
% TDS tree (also known as \xfile{texmf} tree) of your choice.
% Example (linux):
% \begin{quote}
%   |unzip oberdiek.tds.zip -d ~/texmf|
% \end{quote}
%
% \paragraph{Script installation.}
% Check the directory \xfile{TDS:scripts/oberdiek/} for
% scripts that need further installation steps.
% Package \xpackage{attachfile2} comes with the Perl script
% \xfile{pdfatfi.pl} that should be installed in such a way
% that it can be called as \texttt{pdfatfi}.
% Example (linux):
% \begin{quote}
%   |chmod +x scripts/oberdiek/pdfatfi.pl|\\
%   |cp scripts/oberdiek/pdfatfi.pl /usr/local/bin/|
% \end{quote}
%
% \subsection{Package installation}
%
% \paragraph{Unpacking.} The \xfile{.dtx} file is a self-extracting
% \docstrip\ archive. The files are extracted by running the
% \xfile{.dtx} through \plainTeX:
% \begin{quote}
%   \verb|tex hypbmsec.dtx|
% \end{quote}
%
% \paragraph{TDS.} Now the different files must be moved into
% the different directories in your installation TDS tree
% (also known as \xfile{texmf} tree):
% \begin{quote}
% \def\t{^^A
% \begin{tabular}{@{}>{\ttfamily}l@{ $\rightarrow$ }>{\ttfamily}l@{}}
%   hypbmsec.sty & tex/latex/oberdiek/hypbmsec.sty\\
%   hypbmsec.pdf & doc/latex/oberdiek/hypbmsec.pdf\\
%   hypbmsec.dtx & source/latex/oberdiek/hypbmsec.dtx\\
% \end{tabular}^^A
% }^^A
% \sbox0{\t}^^A
% \ifdim\wd0>\linewidth
%   \begingroup
%     \advance\linewidth by\leftmargin
%     \advance\linewidth by\rightmargin
%   \edef\x{\endgroup
%     \def\noexpand\lw{\the\linewidth}^^A
%   }\x
%   \def\lwbox{^^A
%     \leavevmode
%     \hbox to \linewidth{^^A
%       \kern-\leftmargin\relax
%       \hss
%       \usebox0
%       \hss
%       \kern-\rightmargin\relax
%     }^^A
%   }^^A
%   \ifdim\wd0>\lw
%     \sbox0{\small\t}^^A
%     \ifdim\wd0>\linewidth
%       \ifdim\wd0>\lw
%         \sbox0{\footnotesize\t}^^A
%         \ifdim\wd0>\linewidth
%           \ifdim\wd0>\lw
%             \sbox0{\scriptsize\t}^^A
%             \ifdim\wd0>\linewidth
%               \ifdim\wd0>\lw
%                 \sbox0{\tiny\t}^^A
%                 \ifdim\wd0>\linewidth
%                   \lwbox
%                 \else
%                   \usebox0
%                 \fi
%               \else
%                 \lwbox
%               \fi
%             \else
%               \usebox0
%             \fi
%           \else
%             \lwbox
%           \fi
%         \else
%           \usebox0
%         \fi
%       \else
%         \lwbox
%       \fi
%     \else
%       \usebox0
%     \fi
%   \else
%     \lwbox
%   \fi
% \else
%   \usebox0
% \fi
% \end{quote}
% If you have a \xfile{docstrip.cfg} that configures and enables \docstrip's
% TDS installing feature, then some files can already be in the right
% place, see the documentation of \docstrip.
%
% \subsection{Refresh file name databases}
%
% If your \TeX~distribution
% (\teTeX, \mikTeX, \dots) relies on file name databases, you must refresh
% these. For example, \teTeX\ users run \verb|texhash| or
% \verb|mktexlsr|.
%
% \subsection{Some details for the interested}
%
% \paragraph{Attached source.}
%
% The PDF documentation on CTAN also includes the
% \xfile{.dtx} source file. It can be extracted by
% AcrobatReader 6 or higher. Another option is \textsf{pdftk},
% e.g. unpack the file into the current directory:
% \begin{quote}
%   \verb|pdftk hypbmsec.pdf unpack_files output .|
% \end{quote}
%
% \paragraph{Unpacking with \LaTeX.}
% The \xfile{.dtx} chooses its action depending on the format:
% \begin{description}
% \item[\plainTeX:] Run \docstrip\ and extract the files.
% \item[\LaTeX:] Generate the documentation.
% \end{description}
% If you insist on using \LaTeX\ for \docstrip\ (really,
% \docstrip\ does not need \LaTeX), then inform the autodetect routine
% about your intention:
% \begin{quote}
%   \verb|latex \let\install=y% \iffalse meta-comment
%
% File: hypbmsec.dtx
% Version: 2016/05/16 v2.5
% Info: Bookmarks in sectioning commands
%
% Copyright (C) 1998-2000, 2006, 2007 by
%    Heiko Oberdiek <heiko.oberdiek at googlemail.com>
%    2016
%    https://github.com/ho-tex/oberdiek/issues
%
% This work may be distributed and/or modified under the
% conditions of the LaTeX Project Public License, either
% version 1.3c of this license or (at your option) any later
% version. This version of this license is in
%    http://www.latex-project.org/lppl/lppl-1-3c.txt
% and the latest version of this license is in
%    http://www.latex-project.org/lppl.txt
% and version 1.3 or later is part of all distributions of
% LaTeX version 2005/12/01 or later.
%
% This work has the LPPL maintenance status "maintained".
%
% This Current Maintainer of this work is Heiko Oberdiek.
%
% This work consists of the main source file hypbmsec.dtx
% and the derived files
%    hypbmsec.sty, hypbmsec.pdf, hypbmsec.ins, hypbmsec.drv.
%
% Distribution:
%    CTAN:macros/latex/contrib/oberdiek/hypbmsec.dtx
%    CTAN:macros/latex/contrib/oberdiek/hypbmsec.pdf
%
% Unpacking:
%    (a) If hypbmsec.ins is present:
%           tex hypbmsec.ins
%    (b) Without hypbmsec.ins:
%           tex hypbmsec.dtx
%    (c) If you insist on using LaTeX
%           latex \let\install=y\input{hypbmsec.dtx}
%        (quote the arguments according to the demands of your shell)
%
% Documentation:
%    (a) If hypbmsec.drv is present:
%           latex hypbmsec.drv
%    (b) Without hypbmsec.drv:
%           latex hypbmsec.dtx; ...
%    The class ltxdoc loads the configuration file ltxdoc.cfg
%    if available. Here you can specify further options, e.g.
%    use A4 as paper format:
%       \PassOptionsToClass{a4paper}{article}
%
%    Programm calls to get the documentation (example):
%       pdflatex hypbmsec.dtx
%       makeindex -s gind.ist hypbmsec.idx
%       pdflatex hypbmsec.dtx
%       makeindex -s gind.ist hypbmsec.idx
%       pdflatex hypbmsec.dtx
%
% Installation:
%    TDS:tex/latex/oberdiek/hypbmsec.sty
%    TDS:doc/latex/oberdiek/hypbmsec.pdf
%    TDS:source/latex/oberdiek/hypbmsec.dtx
%
%<*ignore>
\begingroup
  \catcode123=1 %
  \catcode125=2 %
  \def\x{LaTeX2e}%
\expandafter\endgroup
\ifcase 0\ifx\install y1\fi\expandafter
         \ifx\csname processbatchFile\endcsname\relax\else1\fi
         \ifx\fmtname\x\else 1\fi\relax
\else\csname fi\endcsname
%</ignore>
%<*install>
\input docstrip.tex
\Msg{************************************************************************}
\Msg{* Installation}
\Msg{* Package: hypbmsec 2016/05/16 v2.5 Bookmarks in sectioning commands (HO)}
\Msg{************************************************************************}

\keepsilent
\askforoverwritefalse

\let\MetaPrefix\relax
\preamble

This is a generated file.

Project: hypbmsec
Version: 2016/05/16 v2.5

Copyright (C) 1998-2000, 2006, 2007 by
   Heiko Oberdiek <heiko.oberdiek at googlemail.com>

This work may be distributed and/or modified under the
conditions of the LaTeX Project Public License, either
version 1.3c of this license or (at your option) any later
version. This version of this license is in
   http://www.latex-project.org/lppl/lppl-1-3c.txt
and the latest version of this license is in
   http://www.latex-project.org/lppl.txt
and version 1.3 or later is part of all distributions of
LaTeX version 2005/12/01 or later.

This work has the LPPL maintenance status "maintained".

This Current Maintainer of this work is Heiko Oberdiek.

This work consists of the main source file hypbmsec.dtx
and the derived files
   hypbmsec.sty, hypbmsec.pdf, hypbmsec.ins, hypbmsec.drv.

\endpreamble
\let\MetaPrefix\DoubleperCent

\generate{%
  \file{hypbmsec.ins}{\from{hypbmsec.dtx}{install}}%
  \file{hypbmsec.drv}{\from{hypbmsec.dtx}{driver}}%
  \usedir{tex/latex/oberdiek}%
  \file{hypbmsec.sty}{\from{hypbmsec.dtx}{package}}%
  \nopreamble
  \nopostamble
  \usedir{source/latex/oberdiek/catalogue}%
  \file{hypbmsec.xml}{\from{hypbmsec.dtx}{catalogue}}%
}

\catcode32=13\relax% active space
\let =\space%
\Msg{************************************************************************}
\Msg{*}
\Msg{* To finish the installation you have to move the following}
\Msg{* file into a directory searched by TeX:}
\Msg{*}
\Msg{*     hypbmsec.sty}
\Msg{*}
\Msg{* To produce the documentation run the file `hypbmsec.drv'}
\Msg{* through LaTeX.}
\Msg{*}
\Msg{* Happy TeXing!}
\Msg{*}
\Msg{************************************************************************}

\endbatchfile
%</install>
%<*ignore>
\fi
%</ignore>
%<*driver>
\NeedsTeXFormat{LaTeX2e}
\ProvidesFile{hypbmsec.drv}%
  [2016/05/16 v2.5 Bookmarks in sectioning commands (HO)]%
\documentclass{ltxdoc}
\usepackage{holtxdoc}[2011/11/22]
\begin{document}
  \DocInput{hypbmsec.dtx}%
\end{document}
%</driver>
% \fi
%
%
% \CharacterTable
%  {Upper-case    \A\B\C\D\E\F\G\H\I\J\K\L\M\N\O\P\Q\R\S\T\U\V\W\X\Y\Z
%   Lower-case    \a\b\c\d\e\f\g\h\i\j\k\l\m\n\o\p\q\r\s\t\u\v\w\x\y\z
%   Digits        \0\1\2\3\4\5\6\7\8\9
%   Exclamation   \!     Double quote  \"     Hash (number) \#
%   Dollar        \$     Percent       \%     Ampersand     \&
%   Acute accent  \'     Left paren    \(     Right paren   \)
%   Asterisk      \*     Plus          \+     Comma         \,
%   Minus         \-     Point         \.     Solidus       \/
%   Colon         \:     Semicolon     \;     Less than     \<
%   Equals        \=     Greater than  \>     Question mark \?
%   Commercial at \@     Left bracket  \[     Backslash     \\
%   Right bracket \]     Circumflex    \^     Underscore    \_
%   Grave accent  \`     Left brace    \{     Vertical bar  \|
%   Right brace   \}     Tilde         \~}
%
% \GetFileInfo{hypbmsec.drv}
%
% \title{The \xpackage{hypbmsec} package}
% \date{2016/05/16 v2.5}
% \author{Heiko Oberdiek\thanks
% {Please report any issues at https://github.com/ho-tex/oberdiek/issues}\\
% \xemail{heiko.oberdiek at googlemail.com}}
%
% \maketitle
%
% \begin{abstract}
% This package expands the syntax of the sectioning commands. If the
% argument of the sectioning commands isn't usable as outline entry,
% a replacement for the bookmarks can be given.
% \end{abstract}
%
% \tableofcontents
%
% \newcommand{\type}[1]{\textsf{#1}}
%
% ^^A No thread support.
% \newenvironment{article}[1]{}{}
%
% \section{Usage}
%
% \subsection{Bookmarks restrictions}\label{sec:restrictions}
%    Outline entries (bookmarks) are written to a file and have
%    to obey the pdf specification.
%    Therefore they have several restrictions:
%    \begin{itemize}
%    \item Bookmarks have to be encoded in PDFDocEncoding^^A
%          \footnote{\Package{hyperref} doesn't support
%            Unicode.}.
%    \item They should only expand to letters and spaces.
%    \item The result of expansion have to be a valid pdf string.
%    \item Stomach commands like \cmd{\relax}, box commands, math,
%          assignments, or definitions aren't allowed.
%    \item Short entries are recommended, which allow a clear view.
%    \end{itemize}
%
% \subsection{\texorpdfstring{\cmd{\texorpdfstring}}{^^A
%    \textbackslash texorpdfstring}}
%    The generic way in package \Package{hyperref} is the use
%    of \cmd{\texorpdfstring}^^A
%    \footnote{In versions of \Package{hyperref} below 6.54 see
%      \cmd{\ifbookmark}.}:
%    \begin{quote}
%\begin{verbatim}
%\section{Pythagoras:
%  \texorpdfstring{$a^2+b^2=c^2}{%
%    a\texttwosuperior\ + b\texttwosuperior\ =
%    c\texttwosuperior}%
%}
%\end{verbatim}
%    \end{quote}
%
% \subsection{Sectioning commands}
%    The package \Package{hyperref} automatically generates
%    bookmarks from the sectioning commands,
%    unless it is suppressed by an option.
%    Commands that structure the text are here called
%    ``sectioning commands'':
%    \begin{quote}
%    \cmd{\part}, \cmd{\chapter},\\
%    \cmd{\section}, \cmd{\subsection}, \cmd{\subsubsection},\\
%    \cmd{\paragraph}, \cmd{\subparagraph}
%    \end{quote}
%
% \subsection{Places\texorpdfstring{ for sectioning strings}{}}
%    \label{sec:places}
%    The argument(s) of these commands are used on several places:
%    \begin{description}
%    \item[\type{text}]
%      The current text without restrictions.
%    \item[\type{toc}]
%      The headlines and the table of contents with the
%      restrictions of ``moving arguments''.
%    \item[\type{out}]
%      The outlines with many restrictions: The outline
%      have to expand to a valid pdf string with PDFDocEncoding
%      (see \ref{sec:restrictions}).
%    \end{description}
%
% \subsection{\texorpdfstring{Solution with o}{O}ptional arguments}
%    If the user wants to use a footnote within a sectioning command,
%    the \LaTeX{} solution is an optional argument:
%    \begin{quote}
%      |\section[Title]{Title\footnote{Footnote text}}|
%    \end{quote}
%    Now |Title| without the footnote is used in the headlines and
%    the table of contents. Also \Package{hyperref} uses it for the
%    bookmarks.
%
%    This package \Package{\filename} offers two possibilities to
%    specify a separate outline entry:
%    \begin{itemize}
%    \item An additional second optional argument in square brackets.
%    \item An additional optional argument in parentheses (in
%          assoziation with a pdf string that is internally surrounded
%          by parentheses, too).
%    \end{itemize}
%    Because \Package{\filename} stores the original meaning of the
%    sectioning commands and uses them again, there should be no
%    problems with packages that redefine the sectioning commands, if
%    these packages doesn't change the syntax.
%
% \subsection{Syntax}
%    The following examples show the syntax of the sectioning
%    commands. For the places the strings appear the abbreviations
%    are used, that are introduced in \ref{sec:places}.
%
% \subsubsection{\texorpdfstring{Star form}{^^A
%    \textbackslash section*\{\}}}
%    The behaviour of the star form isn't changed. The string
%    appears only in the current text:
%    \begin{article}{Syntax}
%    \begin{quote}
%      |\section*{text}|
%    \end{quote}
%    \end{article}
%
% \subsubsection{\texorpdfstring{Without optional arguments}{^^A
%    \textbackslash section\{\}}}
%    The normal case, the string in the mandatory argument is
%    used for all places:
%    \begin{article}{Syntax}
%    \begin{quote}
%      |\section{text, toc, out}|
%    \end{quote}
%    \end{article}
%
% \subsubsection{\texorpdfstring{One optional argument}{^^A
%    \textbackslash section[]\{\}}}
%    Also the form with one optional parameter in square brackets isn't
%    new; for the bookmarks the optional parameter is used:
%    \begin{article}{Syntax}
%    \begin{quote}
%      |\section[toc, out]{text}|
%    \end{quote}
%    \end{article}
%
% \subsubsection{\texorpdfstring{Two optional arguments}{^^A
%    \textbackslash section[][out]\{\}}}\label{sec:two}
%    The second optional parameter in square brackets is introduced
%    by this package to specify an outline entry:
%    \begin{article}{Syntax}
%    \begin{quote}
%      |\section[toc][out]{text}|
%    \end{quote}
%    \end{article}
%
% \subsubsection{\texorpdfstring{Optional argument in parentheses}{^^A
%    \textbackslash section(out)\{\}}}
%    Often the \type{toc} and the \type{text} string would be the same.
%    With the method of the two optional arguments in square brackets
%    (see \ref{sec:two}) this string must be given twice,
%    if the user only wants to specify a different outline entry.
%    Therefore this package offers another possibility:
%    In association with the internal representation in the pdf file
%    an outline entry can be given in parentheses.
%    So the package can easily distinguish between
%    the two forms of optional arguments and the order does not matter:
%    \begin{article}{Syntax}
%    \begin{quote}
%      |\section(out){toc, text}|\\
%      |\section[toc](out){text}|\\
%      |\section(out)[toc]{text}|
%    \end{quote}
%    \end{article}
%
% \subsection{Without \Package{hyperref}}
%    Package \Package{\filename} uses \Package{hyperref} for support of
%    the bookmarks, but this package is not required.
%    If \Package{hyperref} isn't loaded, or
%    is called with a driver that doesn't support bookmarks,
%    package \Package{\filename} shouldn't be removed,
%    because this would lead to
%    a wrong syntax of the sectioning commands.
%    In any cases package \Package{\filename}
%    supports its syntax and ignores the outline entries,
%    if there are no code for bookmarks.
%    So it is possible to write texts,
%    that are processed with several drivers to get different output
%    formats.
%
% \subsection{Protecting parentheses}
%    If the string itself contains parentheses, they have to be hidden
%    from \TeX's argument parsing mechanism.
%    The argument should be surrounded
%    by curly braces:
%    \begin{quote}
%      |\section({outlines(bookmarks)}){text, toc}|
%    \end{quote}
%    With version 6.54 of \Package{hyperref} the other standard method
%    works, too: The closing parenthesis is protected:
%    \begin{quote}
%      |\section(outlines(bookmarks{)}){text, toc}|
%    \end{quote}
%
% \StopEventually{
% }
%
% \section{Implementation}
%    \begin{macrocode}
%<*package>
%    \end{macrocode}
%    Package identification.
%    \begin{macrocode}
\NeedsTeXFormat{LaTeX2e}
\ProvidesPackage{hypbmsec}%
  [2016/05/16 v2.5 Bookmarks in sectioning commands (HO)]
%    \end{macrocode}
%
%    Because of redifining the sectioning commands, it is dangerous
%    to reload the package several times.
%    \begin{macrocode}
\@ifundefined{hbs@do}{}{%
  \PackageInfo{hypbmsec}{Package 'hypbmsec' is already loaded}%
  \endinput
}
%    \end{macrocode}
%
%    \begin{macro}{\hbs@do}
%    The redefined sectioning commands calls \cmd{\hbs@do}. It does
%    \begin{itemize}
%    \item handle the star case.
%    \item resets the macros that store the entries for the outlines
%          (\cmd{\hbs@bmstring}) and table of contents (\cmd{\hbs@tocstring}).
%    \item store the sectioning command |#1| in \cmd{\hbs@seccmd}
%          for later reuse.
%    \item at last call \cmd{\hbs@checkarg} that scans and interprets the
%          parameters of the redefined sectioning command.
%    \end{itemize}
%    \begin{macrocode}
\def\hbs@do#1{%
  \@ifstar{#1*}{%
    \let\hbs@tocstring\relax
    \let\hbs@bmstring\relax
    \let\hbs@seccmd#1%
    \hbs@checkarg
  }%
}
%    \end{macrocode}
%    \end{macro}
%
%    \begin{macro}{\hbs@checkarg}
%    \cmd{\hbs@checkarg} determines the type of the next argument:
%    \begin{itemize}
%    \item An optional argument in square brackets can be an entry
%          for the table of contents or the bookmarks. It will be
%          read by \cmd{\hbs@getsquare}
%    \item An optional argument in parentheses is an outline entry.
%          This is worked off by \cmd{\hbs@getbookmark}.
%    \item If there are no more optional arguments, \cmd{\hbs@process}
%          reads the mandatory argument and calls the original
%          sectioning commands.
%    \end{itemize}
%    \begin{macrocode}
\def\hbs@checkarg{%
  \@ifnextchar[\hbs@getsquare{%
    \@ifnextchar(\hbs@getbookmark\hbs@process
  }%
}
%    \end{macrocode}
%    \end{macro}
%
%    \begin{macro}{\hbs@getsquare}
%    \cmd{\hbs@getsquare} reads an optional argument in square
%    brackets and determines, if this is an entry for the table
%    of contents or the bookmarks.
%    \begin{macrocode}
\long\def\hbs@getsquare[#1]{%
  \ifx\hbs@tocstring\relax
    \def\hbs@tocstring{#1}%
  \else
    \hbs@bmdef{#1}%
  \fi
  \hbs@checkarg
}
%    \end{macrocode}
%    \end{macro}
%
%    \begin{macro}{\hbs@getbookmark}
%    \cmd{\hbs@getbookmark} reads an outline entry in parentheses.
%    \begin{macrocode}
\def\hbs@getbookmark(#1){%
  \hbs@bmdef{#1}%
  \hbs@checkarg
}
%    \end{macrocode}
%    \end{macro}
%
%    \begin{macro}{\hbs@bmdef}
%    The command \cmd{\hbs@bmdef} save the bookmark entry in
%    parameter |#1| in the macro \cmd{\hbs@bmstring} and catches
%    the case, if the user has given several outline strings.
%    \begin{macrocode}
\def\hbs@bmdef#1{%
  \ifx\hbs@bmstring\relax
    \def\hbs@bmstring{#1}%
  \else
    \PackageError{hypbmsec}{%
      Sectioning command with too many parameters%
    }{%
      You can only give one outline entry.%
    }%
  \fi
}
%    \end{macrocode}
%    \end{macro}
%
%    \begin{macro}{\hbs@process}
%    The parameter |#1| is the mandatory argument of the sectioning
%    commands. \cmd{\hbs@process} calls the original sectioning command
%    stored in \cmd{\hbs@seccmd} with arguments that depend of which
%    optional argument are used previously.
%    \begin{macrocode}
\long\def\hbs@process#1{%
  \ifx\hbs@tocstring\relax
    \ifx\hbs@bmstring\relax
      \hbs@seccmd{#1}%
    \else
      \begingroup
        \def\x##1{\endgroup
          \hbs@seccmd{\texorpdfstring{#1}{##1}}%
        }%
      \expandafter\x\expandafter{\hbs@bmstring}%
    \fi
  \else
    \ifx\hbs@bmstring\relax
      \expandafter\hbs@seccmd\expandafter[%
        \expandafter{\hbs@tocstring}%
      ]{#1}%
    \else
      \expandafter\expandafter\expandafter
      \hbs@seccmd\expandafter\expandafter\expandafter[%
        \expandafter\expandafter\expandafter
        \texorpdfstring
        \expandafter\expandafter\expandafter{%
          \expandafter\hbs@tocstring\expandafter
        }\expandafter{%
          \hbs@bmstring
        }%
      ]{#1}%
    \fi
  \fi
}
%    \end{macrocode}
%    \end{macro}
%
%    We have to check, whether package \Package{hyperref} is loaded
%    and have to provide a definition for \cmd{\texorpdfstring}.
%    Because \Package{hyperref} can be loaded after this package,
%    we do the work later (\cmd{\AtBeginDocument}).
%
%    This code only checks versions of \Package{hyperref} that
%    define \cmd{\ifbookmark} (v6.4x until v6.53) or
%    \cmd{\texorpdfstring} (v6.54 and above). Older versions aren't
%    supported.
%    \begin{macrocode}
\AtBeginDocument{%
  \@ifundefined{texorpdfstring}{%
    \@ifundefined{ifbookmark}{%
      \let\texorpdfstring\@firstoftwo
      \@ifpackageloaded{hyperref}{%
        \PackageInfo{hypbmsec}{%
          \ifx\hy@driver\@empty
            Default driver %
          \else
            '\hy@driver' %
          \fi
          of hyperref not supported,\MessageBreak
          bookmark parameters will be ignored%
        }%
      }{%
        \PackageInfo{hypbmsec}{%
          Package hyperref not loaded,\MessageBreak
          bookmark parameters will be ignored%
        }%
      }%
    }%
    {%
      \newcommand\texorpdfstring[2]{\ifbookmark{#2}{#1}}%
      \PackageWarningNoLine{hypbmsec}{%
        Old hyperref version found,\MessageBreak
        update of hyperref recommended%
      }%
    }%
  }{}%
%    \end{macrocode}
%
%    Other packages are allowed to redefine the sectioning commands,
%    if they does not change the syntax. Therefore the redefinitons
%    of this package should be done after the other packages.
%    \begin{macrocode}
  \let\hbs@part         \part
  \let\hbs@section      \section
  \let\hbs@subsection   \subsection
  \let\hbs@subsubsection\subsubsection
  \let\hbs@paragraph    \paragraph
  \let\hbs@subparagraph \subparagraph
  \renewcommand\part         {\hbs@do\hbs@part}%
  \renewcommand\section      {\hbs@do\hbs@section}%
  \renewcommand\subsection   {\hbs@do\hbs@subsection}%
  \renewcommand\subsubsection{\hbs@do\hbs@subsubsection}%
  \renewcommand\paragraph    {\hbs@do\hbs@paragraph}%
  \renewcommand\subparagraph {\hbs@do\hbs@subparagraph}%
  \begingroup\expandafter\expandafter\expandafter\endgroup
  \expandafter\ifx\csname chapter\endcsname\relax\else
    \let\hbs@chapter    \chapter
    \renewcommand\chapter    {\hbs@do\hbs@chapter}%
  \fi
}
%    \end{macrocode}
%
%    \begin{macrocode}
%</package>
%    \end{macrocode}
%
% \section{Installation}
%
% \subsection{Download}
%
% \paragraph{Package.} This package is available on
% CTAN\footnote{\url{http://ctan.org/pkg/hypbmsec}}:
% \begin{description}
% \item[\CTAN{macros/latex/contrib/oberdiek/hypbmsec.dtx}] The source file.
% \item[\CTAN{macros/latex/contrib/oberdiek/hypbmsec.pdf}] Documentation.
% \end{description}
%
%
% \paragraph{Bundle.} All the packages of the bundle `oberdiek'
% are also available in a TDS compliant ZIP archive. There
% the packages are already unpacked and the documentation files
% are generated. The files and directories obey the TDS standard.
% \begin{description}
% \item[\CTAN{install/macros/latex/contrib/oberdiek.tds.zip}]
% \end{description}
% \emph{TDS} refers to the standard ``A Directory Structure
% for \TeX\ Files'' (\CTAN{tds/tds.pdf}). Directories
% with \xfile{texmf} in their name are usually organized this way.
%
% \subsection{Bundle installation}
%
% \paragraph{Unpacking.} Unpack the \xfile{oberdiek.tds.zip} in the
% TDS tree (also known as \xfile{texmf} tree) of your choice.
% Example (linux):
% \begin{quote}
%   |unzip oberdiek.tds.zip -d ~/texmf|
% \end{quote}
%
% \paragraph{Script installation.}
% Check the directory \xfile{TDS:scripts/oberdiek/} for
% scripts that need further installation steps.
% Package \xpackage{attachfile2} comes with the Perl script
% \xfile{pdfatfi.pl} that should be installed in such a way
% that it can be called as \texttt{pdfatfi}.
% Example (linux):
% \begin{quote}
%   |chmod +x scripts/oberdiek/pdfatfi.pl|\\
%   |cp scripts/oberdiek/pdfatfi.pl /usr/local/bin/|
% \end{quote}
%
% \subsection{Package installation}
%
% \paragraph{Unpacking.} The \xfile{.dtx} file is a self-extracting
% \docstrip\ archive. The files are extracted by running the
% \xfile{.dtx} through \plainTeX:
% \begin{quote}
%   \verb|tex hypbmsec.dtx|
% \end{quote}
%
% \paragraph{TDS.} Now the different files must be moved into
% the different directories in your installation TDS tree
% (also known as \xfile{texmf} tree):
% \begin{quote}
% \def\t{^^A
% \begin{tabular}{@{}>{\ttfamily}l@{ $\rightarrow$ }>{\ttfamily}l@{}}
%   hypbmsec.sty & tex/latex/oberdiek/hypbmsec.sty\\
%   hypbmsec.pdf & doc/latex/oberdiek/hypbmsec.pdf\\
%   hypbmsec.dtx & source/latex/oberdiek/hypbmsec.dtx\\
% \end{tabular}^^A
% }^^A
% \sbox0{\t}^^A
% \ifdim\wd0>\linewidth
%   \begingroup
%     \advance\linewidth by\leftmargin
%     \advance\linewidth by\rightmargin
%   \edef\x{\endgroup
%     \def\noexpand\lw{\the\linewidth}^^A
%   }\x
%   \def\lwbox{^^A
%     \leavevmode
%     \hbox to \linewidth{^^A
%       \kern-\leftmargin\relax
%       \hss
%       \usebox0
%       \hss
%       \kern-\rightmargin\relax
%     }^^A
%   }^^A
%   \ifdim\wd0>\lw
%     \sbox0{\small\t}^^A
%     \ifdim\wd0>\linewidth
%       \ifdim\wd0>\lw
%         \sbox0{\footnotesize\t}^^A
%         \ifdim\wd0>\linewidth
%           \ifdim\wd0>\lw
%             \sbox0{\scriptsize\t}^^A
%             \ifdim\wd0>\linewidth
%               \ifdim\wd0>\lw
%                 \sbox0{\tiny\t}^^A
%                 \ifdim\wd0>\linewidth
%                   \lwbox
%                 \else
%                   \usebox0
%                 \fi
%               \else
%                 \lwbox
%               \fi
%             \else
%               \usebox0
%             \fi
%           \else
%             \lwbox
%           \fi
%         \else
%           \usebox0
%         \fi
%       \else
%         \lwbox
%       \fi
%     \else
%       \usebox0
%     \fi
%   \else
%     \lwbox
%   \fi
% \else
%   \usebox0
% \fi
% \end{quote}
% If you have a \xfile{docstrip.cfg} that configures and enables \docstrip's
% TDS installing feature, then some files can already be in the right
% place, see the documentation of \docstrip.
%
% \subsection{Refresh file name databases}
%
% If your \TeX~distribution
% (\teTeX, \mikTeX, \dots) relies on file name databases, you must refresh
% these. For example, \teTeX\ users run \verb|texhash| or
% \verb|mktexlsr|.
%
% \subsection{Some details for the interested}
%
% \paragraph{Attached source.}
%
% The PDF documentation on CTAN also includes the
% \xfile{.dtx} source file. It can be extracted by
% AcrobatReader 6 or higher. Another option is \textsf{pdftk},
% e.g. unpack the file into the current directory:
% \begin{quote}
%   \verb|pdftk hypbmsec.pdf unpack_files output .|
% \end{quote}
%
% \paragraph{Unpacking with \LaTeX.}
% The \xfile{.dtx} chooses its action depending on the format:
% \begin{description}
% \item[\plainTeX:] Run \docstrip\ and extract the files.
% \item[\LaTeX:] Generate the documentation.
% \end{description}
% If you insist on using \LaTeX\ for \docstrip\ (really,
% \docstrip\ does not need \LaTeX), then inform the autodetect routine
% about your intention:
% \begin{quote}
%   \verb|latex \let\install=y\input{hypbmsec.dtx}|
% \end{quote}
% Do not forget to quote the argument according to the demands
% of your shell.
%
% \paragraph{Generating the documentation.}
% You can use both the \xfile{.dtx} or the \xfile{.drv} to generate
% the documentation. The process can be configured by the
% configuration file \xfile{ltxdoc.cfg}. For instance, put this
% line into this file, if you want to have A4 as paper format:
% \begin{quote}
%   \verb|\PassOptionsToClass{a4paper}{article}|
% \end{quote}
% An example follows how to generate the
% documentation with pdf\LaTeX:
% \begin{quote}
%\begin{verbatim}
%pdflatex hypbmsec.dtx
%makeindex -s gind.ist hypbmsec.idx
%pdflatex hypbmsec.dtx
%makeindex -s gind.ist hypbmsec.idx
%pdflatex hypbmsec.dtx
%\end{verbatim}
% \end{quote}
%
% \section{Catalogue}
%
% The following XML file can be used as source for the
% \href{http://mirror.ctan.org/help/Catalogue/catalogue.html}{\TeX\ Catalogue}.
% The elements \texttt{caption} and \texttt{description} are imported
% from the original XML file from the Catalogue.
% The name of the XML file in the Catalogue is \xfile{hypbmsec.xml}.
%    \begin{macrocode}
%<*catalogue>
<?xml version='1.0' encoding='us-ascii'?>
<!DOCTYPE entry SYSTEM 'catalogue.dtd'>
<entry datestamp='$Date$' modifier='$Author$' id='hypbmsec'>
  <name>hypbmsec</name>
  <caption>Hypertext bookmarks in sectioning commands.</caption>
  <authorref id='auth:oberdiek'/>
  <copyright owner='Heiko Oberdiek' year='1998-2000,2006,2007'/>
  <license type='lppl1.3'/>
  <version number='2.5'/>
  <description>
    Bookmark entries can be given as another argument to the LaTeX
    sectioning commands. The <xref refid='hyperref'>hyperref</xref>
    package is required to get the bookmarks, but the syntax
    works without it.
    <p/>
    This package is part of the <xref refid='oberdiek'>oberdiek</xref>
    bundle.
  </description>
  <documentation details='Package documentation'
      href='ctan:/macros/latex/contrib/oberdiek/hypbmsec.pdf'/>
  <ctan file='true' path='/macros/latex/contrib/oberdiek/hypbmsec.dtx'/>
  <miktex location='oberdiek'/>
  <texlive location='oberdiek'/>
  <install path='/macros/latex/contrib/oberdiek/oberdiek.tds.zip'/>
</entry>
%</catalogue>
%    \end{macrocode}
%
% \begin{History}
%   \begin{Version}{1998/11/20 v1.0}
%   \item
%     First version.
%   \item
%     It merges package \xpackage{hysecopt} and
%   \item
%     package \xpackage{hypbmpar}.
%   \item
%     Published for the DANTE'99 meeting^^A
%     \URL{}{http://dante99.cs.uni-dortmund.de/handouts/oberdiek/hypbmsec.sty}.
%   \end{Version}
%   \begin{Version}{1999/04/12 v2.0}
%   \item
%     Adaptation to \Package{hyperref} version 6.54.
%   \item
%     Documentation in dtx format.
%   \item
%     Copyright: LPPL (\CTAN{macros/latex/base/lppl.txt})
%   \item
%     First CTAN release.
%   \end{Version}
%   \begin{Version}{2000/03/22 v2.1}
%   \item
%     Bug fix in redefinition of \cmd{\chapter}.
%   \item
%     Copyright: LPPL 1.2
%   \end{Version}
%   \begin{Version}{2006/02/20 v2.2}
%   \item
%     Code is not changed.
%   \item
%     New DTX framework.
%   \item
%     LPPL 1.3
%   \end{Version}
%   \begin{Version}{2007/03/05 v2.3}
%   \item
%     Bug fix: Expand \cs{hbs@tocstring} and \cs{hbs@bmstring} before
%     calling \cs{hbs@seccmd}.
%   \end{Version}
%   \begin{Version}{2007/04/11 v2.4}
%   \item
%     Line ends sanitized.
%   \end{Version}
%   \begin{Version}{2016/05/16 v2.5}
%   \item
%     Documentation updates.
%   \end{Version}
% \end{History}
%
% \PrintIndex
%
% \Finale
\endinput
|
% \end{quote}
% Do not forget to quote the argument according to the demands
% of your shell.
%
% \paragraph{Generating the documentation.}
% You can use both the \xfile{.dtx} or the \xfile{.drv} to generate
% the documentation. The process can be configured by the
% configuration file \xfile{ltxdoc.cfg}. For instance, put this
% line into this file, if you want to have A4 as paper format:
% \begin{quote}
%   \verb|\PassOptionsToClass{a4paper}{article}|
% \end{quote}
% An example follows how to generate the
% documentation with pdf\LaTeX:
% \begin{quote}
%\begin{verbatim}
%pdflatex hypbmsec.dtx
%makeindex -s gind.ist hypbmsec.idx
%pdflatex hypbmsec.dtx
%makeindex -s gind.ist hypbmsec.idx
%pdflatex hypbmsec.dtx
%\end{verbatim}
% \end{quote}
%
% \section{Catalogue}
%
% The following XML file can be used as source for the
% \href{http://mirror.ctan.org/help/Catalogue/catalogue.html}{\TeX\ Catalogue}.
% The elements \texttt{caption} and \texttt{description} are imported
% from the original XML file from the Catalogue.
% The name of the XML file in the Catalogue is \xfile{hypbmsec.xml}.
%    \begin{macrocode}
%<*catalogue>
<?xml version='1.0' encoding='us-ascii'?>
<!DOCTYPE entry SYSTEM 'catalogue.dtd'>
<entry datestamp='$Date$' modifier='$Author$' id='hypbmsec'>
  <name>hypbmsec</name>
  <caption>Hypertext bookmarks in sectioning commands.</caption>
  <authorref id='auth:oberdiek'/>
  <copyright owner='Heiko Oberdiek' year='1998-2000,2006,2007'/>
  <license type='lppl1.3'/>
  <version number='2.5'/>
  <description>
    Bookmark entries can be given as another argument to the LaTeX
    sectioning commands. The <xref refid='hyperref'>hyperref</xref>
    package is required to get the bookmarks, but the syntax
    works without it.
    <p/>
    This package is part of the <xref refid='oberdiek'>oberdiek</xref>
    bundle.
  </description>
  <documentation details='Package documentation'
      href='ctan:/macros/latex/contrib/oberdiek/hypbmsec.pdf'/>
  <ctan file='true' path='/macros/latex/contrib/oberdiek/hypbmsec.dtx'/>
  <miktex location='oberdiek'/>
  <texlive location='oberdiek'/>
  <install path='/macros/latex/contrib/oberdiek/oberdiek.tds.zip'/>
</entry>
%</catalogue>
%    \end{macrocode}
%
% \begin{History}
%   \begin{Version}{1998/11/20 v1.0}
%   \item
%     First version.
%   \item
%     It merges package \xpackage{hysecopt} and
%   \item
%     package \xpackage{hypbmpar}.
%   \item
%     Published for the DANTE'99 meeting^^A
%     \URL{}{http://dante99.cs.uni-dortmund.de/handouts/oberdiek/hypbmsec.sty}.
%   \end{Version}
%   \begin{Version}{1999/04/12 v2.0}
%   \item
%     Adaptation to \Package{hyperref} version 6.54.
%   \item
%     Documentation in dtx format.
%   \item
%     Copyright: LPPL (\CTAN{macros/latex/base/lppl.txt})
%   \item
%     First CTAN release.
%   \end{Version}
%   \begin{Version}{2000/03/22 v2.1}
%   \item
%     Bug fix in redefinition of \cmd{\chapter}.
%   \item
%     Copyright: LPPL 1.2
%   \end{Version}
%   \begin{Version}{2006/02/20 v2.2}
%   \item
%     Code is not changed.
%   \item
%     New DTX framework.
%   \item
%     LPPL 1.3
%   \end{Version}
%   \begin{Version}{2007/03/05 v2.3}
%   \item
%     Bug fix: Expand \cs{hbs@tocstring} and \cs{hbs@bmstring} before
%     calling \cs{hbs@seccmd}.
%   \end{Version}
%   \begin{Version}{2007/04/11 v2.4}
%   \item
%     Line ends sanitized.
%   \end{Version}
%   \begin{Version}{2016/05/16 v2.5}
%   \item
%     Documentation updates.
%   \end{Version}
% \end{History}
%
% \PrintIndex
%
% \Finale
\endinput

%        (quote the arguments according to the demands of your shell)
%
% Documentation:
%    (a) If hypbmsec.drv is present:
%           latex hypbmsec.drv
%    (b) Without hypbmsec.drv:
%           latex hypbmsec.dtx; ...
%    The class ltxdoc loads the configuration file ltxdoc.cfg
%    if available. Here you can specify further options, e.g.
%    use A4 as paper format:
%       \PassOptionsToClass{a4paper}{article}
%
%    Programm calls to get the documentation (example):
%       pdflatex hypbmsec.dtx
%       makeindex -s gind.ist hypbmsec.idx
%       pdflatex hypbmsec.dtx
%       makeindex -s gind.ist hypbmsec.idx
%       pdflatex hypbmsec.dtx
%
% Installation:
%    TDS:tex/latex/oberdiek/hypbmsec.sty
%    TDS:doc/latex/oberdiek/hypbmsec.pdf
%    TDS:source/latex/oberdiek/hypbmsec.dtx
%
%<*ignore>
\begingroup
  \catcode123=1 %
  \catcode125=2 %
  \def\x{LaTeX2e}%
\expandafter\endgroup
\ifcase 0\ifx\install y1\fi\expandafter
         \ifx\csname processbatchFile\endcsname\relax\else1\fi
         \ifx\fmtname\x\else 1\fi\relax
\else\csname fi\endcsname
%</ignore>
%<*install>
\input docstrip.tex
\Msg{************************************************************************}
\Msg{* Installation}
\Msg{* Package: hypbmsec 2016/05/16 v2.5 Bookmarks in sectioning commands (HO)}
\Msg{************************************************************************}

\keepsilent
\askforoverwritefalse

\let\MetaPrefix\relax
\preamble

This is a generated file.

Project: hypbmsec
Version: 2016/05/16 v2.5

Copyright (C) 1998-2000, 2006, 2007 by
   Heiko Oberdiek <heiko.oberdiek at googlemail.com>

This work may be distributed and/or modified under the
conditions of the LaTeX Project Public License, either
version 1.3c of this license or (at your option) any later
version. This version of this license is in
   http://www.latex-project.org/lppl/lppl-1-3c.txt
and the latest version of this license is in
   http://www.latex-project.org/lppl.txt
and version 1.3 or later is part of all distributions of
LaTeX version 2005/12/01 or later.

This work has the LPPL maintenance status "maintained".

This Current Maintainer of this work is Heiko Oberdiek.

This work consists of the main source file hypbmsec.dtx
and the derived files
   hypbmsec.sty, hypbmsec.pdf, hypbmsec.ins, hypbmsec.drv.

\endpreamble
\let\MetaPrefix\DoubleperCent

\generate{%
  \file{hypbmsec.ins}{\from{hypbmsec.dtx}{install}}%
  \file{hypbmsec.drv}{\from{hypbmsec.dtx}{driver}}%
  \usedir{tex/latex/oberdiek}%
  \file{hypbmsec.sty}{\from{hypbmsec.dtx}{package}}%
  \nopreamble
  \nopostamble
  \usedir{source/latex/oberdiek/catalogue}%
  \file{hypbmsec.xml}{\from{hypbmsec.dtx}{catalogue}}%
}

\catcode32=13\relax% active space
\let =\space%
\Msg{************************************************************************}
\Msg{*}
\Msg{* To finish the installation you have to move the following}
\Msg{* file into a directory searched by TeX:}
\Msg{*}
\Msg{*     hypbmsec.sty}
\Msg{*}
\Msg{* To produce the documentation run the file `hypbmsec.drv'}
\Msg{* through LaTeX.}
\Msg{*}
\Msg{* Happy TeXing!}
\Msg{*}
\Msg{************************************************************************}

\endbatchfile
%</install>
%<*ignore>
\fi
%</ignore>
%<*driver>
\NeedsTeXFormat{LaTeX2e}
\ProvidesFile{hypbmsec.drv}%
  [2016/05/16 v2.5 Bookmarks in sectioning commands (HO)]%
\documentclass{ltxdoc}
\usepackage{holtxdoc}[2011/11/22]
\begin{document}
  \DocInput{hypbmsec.dtx}%
\end{document}
%</driver>
% \fi
%
%
% \CharacterTable
%  {Upper-case    \A\B\C\D\E\F\G\H\I\J\K\L\M\N\O\P\Q\R\S\T\U\V\W\X\Y\Z
%   Lower-case    \a\b\c\d\e\f\g\h\i\j\k\l\m\n\o\p\q\r\s\t\u\v\w\x\y\z
%   Digits        \0\1\2\3\4\5\6\7\8\9
%   Exclamation   \!     Double quote  \"     Hash (number) \#
%   Dollar        \$     Percent       \%     Ampersand     \&
%   Acute accent  \'     Left paren    \(     Right paren   \)
%   Asterisk      \*     Plus          \+     Comma         \,
%   Minus         \-     Point         \.     Solidus       \/
%   Colon         \:     Semicolon     \;     Less than     \<
%   Equals        \=     Greater than  \>     Question mark \?
%   Commercial at \@     Left bracket  \[     Backslash     \\
%   Right bracket \]     Circumflex    \^     Underscore    \_
%   Grave accent  \`     Left brace    \{     Vertical bar  \|
%   Right brace   \}     Tilde         \~}
%
% \GetFileInfo{hypbmsec.drv}
%
% \title{The \xpackage{hypbmsec} package}
% \date{2016/05/16 v2.5}
% \author{Heiko Oberdiek\thanks
% {Please report any issues at https://github.com/ho-tex/oberdiek/issues}\\
% \xemail{heiko.oberdiek at googlemail.com}}
%
% \maketitle
%
% \begin{abstract}
% This package expands the syntax of the sectioning commands. If the
% argument of the sectioning commands isn't usable as outline entry,
% a replacement for the bookmarks can be given.
% \end{abstract}
%
% \tableofcontents
%
% \newcommand{\type}[1]{\textsf{#1}}
%
% ^^A No thread support.
% \newenvironment{article}[1]{}{}
%
% \section{Usage}
%
% \subsection{Bookmarks restrictions}\label{sec:restrictions}
%    Outline entries (bookmarks) are written to a file and have
%    to obey the pdf specification.
%    Therefore they have several restrictions:
%    \begin{itemize}
%    \item Bookmarks have to be encoded in PDFDocEncoding^^A
%          \footnote{\Package{hyperref} doesn't support
%            Unicode.}.
%    \item They should only expand to letters and spaces.
%    \item The result of expansion have to be a valid pdf string.
%    \item Stomach commands like \cmd{\relax}, box commands, math,
%          assignments, or definitions aren't allowed.
%    \item Short entries are recommended, which allow a clear view.
%    \end{itemize}
%
% \subsection{\texorpdfstring{\cmd{\texorpdfstring}}{^^A
%    \textbackslash texorpdfstring}}
%    The generic way in package \Package{hyperref} is the use
%    of \cmd{\texorpdfstring}^^A
%    \footnote{In versions of \Package{hyperref} below 6.54 see
%      \cmd{\ifbookmark}.}:
%    \begin{quote}
%\begin{verbatim}
%\section{Pythagoras:
%  \texorpdfstring{$a^2+b^2=c^2}{%
%    a\texttwosuperior\ + b\texttwosuperior\ =
%    c\texttwosuperior}%
%}
%\end{verbatim}
%    \end{quote}
%
% \subsection{Sectioning commands}
%    The package \Package{hyperref} automatically generates
%    bookmarks from the sectioning commands,
%    unless it is suppressed by an option.
%    Commands that structure the text are here called
%    ``sectioning commands'':
%    \begin{quote}
%    \cmd{\part}, \cmd{\chapter},\\
%    \cmd{\section}, \cmd{\subsection}, \cmd{\subsubsection},\\
%    \cmd{\paragraph}, \cmd{\subparagraph}
%    \end{quote}
%
% \subsection{Places\texorpdfstring{ for sectioning strings}{}}
%    \label{sec:places}
%    The argument(s) of these commands are used on several places:
%    \begin{description}
%    \item[\type{text}]
%      The current text without restrictions.
%    \item[\type{toc}]
%      The headlines and the table of contents with the
%      restrictions of ``moving arguments''.
%    \item[\type{out}]
%      The outlines with many restrictions: The outline
%      have to expand to a valid pdf string with PDFDocEncoding
%      (see \ref{sec:restrictions}).
%    \end{description}
%
% \subsection{\texorpdfstring{Solution with o}{O}ptional arguments}
%    If the user wants to use a footnote within a sectioning command,
%    the \LaTeX{} solution is an optional argument:
%    \begin{quote}
%      |\section[Title]{Title\footnote{Footnote text}}|
%    \end{quote}
%    Now |Title| without the footnote is used in the headlines and
%    the table of contents. Also \Package{hyperref} uses it for the
%    bookmarks.
%
%    This package \Package{\filename} offers two possibilities to
%    specify a separate outline entry:
%    \begin{itemize}
%    \item An additional second optional argument in square brackets.
%    \item An additional optional argument in parentheses (in
%          assoziation with a pdf string that is internally surrounded
%          by parentheses, too).
%    \end{itemize}
%    Because \Package{\filename} stores the original meaning of the
%    sectioning commands and uses them again, there should be no
%    problems with packages that redefine the sectioning commands, if
%    these packages doesn't change the syntax.
%
% \subsection{Syntax}
%    The following examples show the syntax of the sectioning
%    commands. For the places the strings appear the abbreviations
%    are used, that are introduced in \ref{sec:places}.
%
% \subsubsection{\texorpdfstring{Star form}{^^A
%    \textbackslash section*\{\}}}
%    The behaviour of the star form isn't changed. The string
%    appears only in the current text:
%    \begin{article}{Syntax}
%    \begin{quote}
%      |\section*{text}|
%    \end{quote}
%    \end{article}
%
% \subsubsection{\texorpdfstring{Without optional arguments}{^^A
%    \textbackslash section\{\}}}
%    The normal case, the string in the mandatory argument is
%    used for all places:
%    \begin{article}{Syntax}
%    \begin{quote}
%      |\section{text, toc, out}|
%    \end{quote}
%    \end{article}
%
% \subsubsection{\texorpdfstring{One optional argument}{^^A
%    \textbackslash section[]\{\}}}
%    Also the form with one optional parameter in square brackets isn't
%    new; for the bookmarks the optional parameter is used:
%    \begin{article}{Syntax}
%    \begin{quote}
%      |\section[toc, out]{text}|
%    \end{quote}
%    \end{article}
%
% \subsubsection{\texorpdfstring{Two optional arguments}{^^A
%    \textbackslash section[][out]\{\}}}\label{sec:two}
%    The second optional parameter in square brackets is introduced
%    by this package to specify an outline entry:
%    \begin{article}{Syntax}
%    \begin{quote}
%      |\section[toc][out]{text}|
%    \end{quote}
%    \end{article}
%
% \subsubsection{\texorpdfstring{Optional argument in parentheses}{^^A
%    \textbackslash section(out)\{\}}}
%    Often the \type{toc} and the \type{text} string would be the same.
%    With the method of the two optional arguments in square brackets
%    (see \ref{sec:two}) this string must be given twice,
%    if the user only wants to specify a different outline entry.
%    Therefore this package offers another possibility:
%    In association with the internal representation in the pdf file
%    an outline entry can be given in parentheses.
%    So the package can easily distinguish between
%    the two forms of optional arguments and the order does not matter:
%    \begin{article}{Syntax}
%    \begin{quote}
%      |\section(out){toc, text}|\\
%      |\section[toc](out){text}|\\
%      |\section(out)[toc]{text}|
%    \end{quote}
%    \end{article}
%
% \subsection{Without \Package{hyperref}}
%    Package \Package{\filename} uses \Package{hyperref} for support of
%    the bookmarks, but this package is not required.
%    If \Package{hyperref} isn't loaded, or
%    is called with a driver that doesn't support bookmarks,
%    package \Package{\filename} shouldn't be removed,
%    because this would lead to
%    a wrong syntax of the sectioning commands.
%    In any cases package \Package{\filename}
%    supports its syntax and ignores the outline entries,
%    if there are no code for bookmarks.
%    So it is possible to write texts,
%    that are processed with several drivers to get different output
%    formats.
%
% \subsection{Protecting parentheses}
%    If the string itself contains parentheses, they have to be hidden
%    from \TeX's argument parsing mechanism.
%    The argument should be surrounded
%    by curly braces:
%    \begin{quote}
%      |\section({outlines(bookmarks)}){text, toc}|
%    \end{quote}
%    With version 6.54 of \Package{hyperref} the other standard method
%    works, too: The closing parenthesis is protected:
%    \begin{quote}
%      |\section(outlines(bookmarks{)}){text, toc}|
%    \end{quote}
%
% \StopEventually{
% }
%
% \section{Implementation}
%    \begin{macrocode}
%<*package>
%    \end{macrocode}
%    Package identification.
%    \begin{macrocode}
\NeedsTeXFormat{LaTeX2e}
\ProvidesPackage{hypbmsec}%
  [2016/05/16 v2.5 Bookmarks in sectioning commands (HO)]
%    \end{macrocode}
%
%    Because of redifining the sectioning commands, it is dangerous
%    to reload the package several times.
%    \begin{macrocode}
\@ifundefined{hbs@do}{}{%
  \PackageInfo{hypbmsec}{Package 'hypbmsec' is already loaded}%
  \endinput
}
%    \end{macrocode}
%
%    \begin{macro}{\hbs@do}
%    The redefined sectioning commands calls \cmd{\hbs@do}. It does
%    \begin{itemize}
%    \item handle the star case.
%    \item resets the macros that store the entries for the outlines
%          (\cmd{\hbs@bmstring}) and table of contents (\cmd{\hbs@tocstring}).
%    \item store the sectioning command |#1| in \cmd{\hbs@seccmd}
%          for later reuse.
%    \item at last call \cmd{\hbs@checkarg} that scans and interprets the
%          parameters of the redefined sectioning command.
%    \end{itemize}
%    \begin{macrocode}
\def\hbs@do#1{%
  \@ifstar{#1*}{%
    \let\hbs@tocstring\relax
    \let\hbs@bmstring\relax
    \let\hbs@seccmd#1%
    \hbs@checkarg
  }%
}
%    \end{macrocode}
%    \end{macro}
%
%    \begin{macro}{\hbs@checkarg}
%    \cmd{\hbs@checkarg} determines the type of the next argument:
%    \begin{itemize}
%    \item An optional argument in square brackets can be an entry
%          for the table of contents or the bookmarks. It will be
%          read by \cmd{\hbs@getsquare}
%    \item An optional argument in parentheses is an outline entry.
%          This is worked off by \cmd{\hbs@getbookmark}.
%    \item If there are no more optional arguments, \cmd{\hbs@process}
%          reads the mandatory argument and calls the original
%          sectioning commands.
%    \end{itemize}
%    \begin{macrocode}
\def\hbs@checkarg{%
  \@ifnextchar[\hbs@getsquare{%
    \@ifnextchar(\hbs@getbookmark\hbs@process
  }%
}
%    \end{macrocode}
%    \end{macro}
%
%    \begin{macro}{\hbs@getsquare}
%    \cmd{\hbs@getsquare} reads an optional argument in square
%    brackets and determines, if this is an entry for the table
%    of contents or the bookmarks.
%    \begin{macrocode}
\long\def\hbs@getsquare[#1]{%
  \ifx\hbs@tocstring\relax
    \def\hbs@tocstring{#1}%
  \else
    \hbs@bmdef{#1}%
  \fi
  \hbs@checkarg
}
%    \end{macrocode}
%    \end{macro}
%
%    \begin{macro}{\hbs@getbookmark}
%    \cmd{\hbs@getbookmark} reads an outline entry in parentheses.
%    \begin{macrocode}
\def\hbs@getbookmark(#1){%
  \hbs@bmdef{#1}%
  \hbs@checkarg
}
%    \end{macrocode}
%    \end{macro}
%
%    \begin{macro}{\hbs@bmdef}
%    The command \cmd{\hbs@bmdef} save the bookmark entry in
%    parameter |#1| in the macro \cmd{\hbs@bmstring} and catches
%    the case, if the user has given several outline strings.
%    \begin{macrocode}
\def\hbs@bmdef#1{%
  \ifx\hbs@bmstring\relax
    \def\hbs@bmstring{#1}%
  \else
    \PackageError{hypbmsec}{%
      Sectioning command with too many parameters%
    }{%
      You can only give one outline entry.%
    }%
  \fi
}
%    \end{macrocode}
%    \end{macro}
%
%    \begin{macro}{\hbs@process}
%    The parameter |#1| is the mandatory argument of the sectioning
%    commands. \cmd{\hbs@process} calls the original sectioning command
%    stored in \cmd{\hbs@seccmd} with arguments that depend of which
%    optional argument are used previously.
%    \begin{macrocode}
\long\def\hbs@process#1{%
  \ifx\hbs@tocstring\relax
    \ifx\hbs@bmstring\relax
      \hbs@seccmd{#1}%
    \else
      \begingroup
        \def\x##1{\endgroup
          \hbs@seccmd{\texorpdfstring{#1}{##1}}%
        }%
      \expandafter\x\expandafter{\hbs@bmstring}%
    \fi
  \else
    \ifx\hbs@bmstring\relax
      \expandafter\hbs@seccmd\expandafter[%
        \expandafter{\hbs@tocstring}%
      ]{#1}%
    \else
      \expandafter\expandafter\expandafter
      \hbs@seccmd\expandafter\expandafter\expandafter[%
        \expandafter\expandafter\expandafter
        \texorpdfstring
        \expandafter\expandafter\expandafter{%
          \expandafter\hbs@tocstring\expandafter
        }\expandafter{%
          \hbs@bmstring
        }%
      ]{#1}%
    \fi
  \fi
}
%    \end{macrocode}
%    \end{macro}
%
%    We have to check, whether package \Package{hyperref} is loaded
%    and have to provide a definition for \cmd{\texorpdfstring}.
%    Because \Package{hyperref} can be loaded after this package,
%    we do the work later (\cmd{\AtBeginDocument}).
%
%    This code only checks versions of \Package{hyperref} that
%    define \cmd{\ifbookmark} (v6.4x until v6.53) or
%    \cmd{\texorpdfstring} (v6.54 and above). Older versions aren't
%    supported.
%    \begin{macrocode}
\AtBeginDocument{%
  \@ifundefined{texorpdfstring}{%
    \@ifundefined{ifbookmark}{%
      \let\texorpdfstring\@firstoftwo
      \@ifpackageloaded{hyperref}{%
        \PackageInfo{hypbmsec}{%
          \ifx\hy@driver\@empty
            Default driver %
          \else
            '\hy@driver' %
          \fi
          of hyperref not supported,\MessageBreak
          bookmark parameters will be ignored%
        }%
      }{%
        \PackageInfo{hypbmsec}{%
          Package hyperref not loaded,\MessageBreak
          bookmark parameters will be ignored%
        }%
      }%
    }%
    {%
      \newcommand\texorpdfstring[2]{\ifbookmark{#2}{#1}}%
      \PackageWarningNoLine{hypbmsec}{%
        Old hyperref version found,\MessageBreak
        update of hyperref recommended%
      }%
    }%
  }{}%
%    \end{macrocode}
%
%    Other packages are allowed to redefine the sectioning commands,
%    if they does not change the syntax. Therefore the redefinitons
%    of this package should be done after the other packages.
%    \begin{macrocode}
  \let\hbs@part         \part
  \let\hbs@section      \section
  \let\hbs@subsection   \subsection
  \let\hbs@subsubsection\subsubsection
  \let\hbs@paragraph    \paragraph
  \let\hbs@subparagraph \subparagraph
  \renewcommand\part         {\hbs@do\hbs@part}%
  \renewcommand\section      {\hbs@do\hbs@section}%
  \renewcommand\subsection   {\hbs@do\hbs@subsection}%
  \renewcommand\subsubsection{\hbs@do\hbs@subsubsection}%
  \renewcommand\paragraph    {\hbs@do\hbs@paragraph}%
  \renewcommand\subparagraph {\hbs@do\hbs@subparagraph}%
  \begingroup\expandafter\expandafter\expandafter\endgroup
  \expandafter\ifx\csname chapter\endcsname\relax\else
    \let\hbs@chapter    \chapter
    \renewcommand\chapter    {\hbs@do\hbs@chapter}%
  \fi
}
%    \end{macrocode}
%
%    \begin{macrocode}
%</package>
%    \end{macrocode}
%
% \section{Installation}
%
% \subsection{Download}
%
% \paragraph{Package.} This package is available on
% CTAN\footnote{\url{http://ctan.org/pkg/hypbmsec}}:
% \begin{description}
% \item[\CTAN{macros/latex/contrib/oberdiek/hypbmsec.dtx}] The source file.
% \item[\CTAN{macros/latex/contrib/oberdiek/hypbmsec.pdf}] Documentation.
% \end{description}
%
%
% \paragraph{Bundle.} All the packages of the bundle `oberdiek'
% are also available in a TDS compliant ZIP archive. There
% the packages are already unpacked and the documentation files
% are generated. The files and directories obey the TDS standard.
% \begin{description}
% \item[\CTAN{install/macros/latex/contrib/oberdiek.tds.zip}]
% \end{description}
% \emph{TDS} refers to the standard ``A Directory Structure
% for \TeX\ Files'' (\CTAN{tds/tds.pdf}). Directories
% with \xfile{texmf} in their name are usually organized this way.
%
% \subsection{Bundle installation}
%
% \paragraph{Unpacking.} Unpack the \xfile{oberdiek.tds.zip} in the
% TDS tree (also known as \xfile{texmf} tree) of your choice.
% Example (linux):
% \begin{quote}
%   |unzip oberdiek.tds.zip -d ~/texmf|
% \end{quote}
%
% \paragraph{Script installation.}
% Check the directory \xfile{TDS:scripts/oberdiek/} for
% scripts that need further installation steps.
% Package \xpackage{attachfile2} comes with the Perl script
% \xfile{pdfatfi.pl} that should be installed in such a way
% that it can be called as \texttt{pdfatfi}.
% Example (linux):
% \begin{quote}
%   |chmod +x scripts/oberdiek/pdfatfi.pl|\\
%   |cp scripts/oberdiek/pdfatfi.pl /usr/local/bin/|
% \end{quote}
%
% \subsection{Package installation}
%
% \paragraph{Unpacking.} The \xfile{.dtx} file is a self-extracting
% \docstrip\ archive. The files are extracted by running the
% \xfile{.dtx} through \plainTeX:
% \begin{quote}
%   \verb|tex hypbmsec.dtx|
% \end{quote}
%
% \paragraph{TDS.} Now the different files must be moved into
% the different directories in your installation TDS tree
% (also known as \xfile{texmf} tree):
% \begin{quote}
% \def\t{^^A
% \begin{tabular}{@{}>{\ttfamily}l@{ $\rightarrow$ }>{\ttfamily}l@{}}
%   hypbmsec.sty & tex/latex/oberdiek/hypbmsec.sty\\
%   hypbmsec.pdf & doc/latex/oberdiek/hypbmsec.pdf\\
%   hypbmsec.dtx & source/latex/oberdiek/hypbmsec.dtx\\
% \end{tabular}^^A
% }^^A
% \sbox0{\t}^^A
% \ifdim\wd0>\linewidth
%   \begingroup
%     \advance\linewidth by\leftmargin
%     \advance\linewidth by\rightmargin
%   \edef\x{\endgroup
%     \def\noexpand\lw{\the\linewidth}^^A
%   }\x
%   \def\lwbox{^^A
%     \leavevmode
%     \hbox to \linewidth{^^A
%       \kern-\leftmargin\relax
%       \hss
%       \usebox0
%       \hss
%       \kern-\rightmargin\relax
%     }^^A
%   }^^A
%   \ifdim\wd0>\lw
%     \sbox0{\small\t}^^A
%     \ifdim\wd0>\linewidth
%       \ifdim\wd0>\lw
%         \sbox0{\footnotesize\t}^^A
%         \ifdim\wd0>\linewidth
%           \ifdim\wd0>\lw
%             \sbox0{\scriptsize\t}^^A
%             \ifdim\wd0>\linewidth
%               \ifdim\wd0>\lw
%                 \sbox0{\tiny\t}^^A
%                 \ifdim\wd0>\linewidth
%                   \lwbox
%                 \else
%                   \usebox0
%                 \fi
%               \else
%                 \lwbox
%               \fi
%             \else
%               \usebox0
%             \fi
%           \else
%             \lwbox
%           \fi
%         \else
%           \usebox0
%         \fi
%       \else
%         \lwbox
%       \fi
%     \else
%       \usebox0
%     \fi
%   \else
%     \lwbox
%   \fi
% \else
%   \usebox0
% \fi
% \end{quote}
% If you have a \xfile{docstrip.cfg} that configures and enables \docstrip's
% TDS installing feature, then some files can already be in the right
% place, see the documentation of \docstrip.
%
% \subsection{Refresh file name databases}
%
% If your \TeX~distribution
% (\teTeX, \mikTeX, \dots) relies on file name databases, you must refresh
% these. For example, \teTeX\ users run \verb|texhash| or
% \verb|mktexlsr|.
%
% \subsection{Some details for the interested}
%
% \paragraph{Attached source.}
%
% The PDF documentation on CTAN also includes the
% \xfile{.dtx} source file. It can be extracted by
% AcrobatReader 6 or higher. Another option is \textsf{pdftk},
% e.g. unpack the file into the current directory:
% \begin{quote}
%   \verb|pdftk hypbmsec.pdf unpack_files output .|
% \end{quote}
%
% \paragraph{Unpacking with \LaTeX.}
% The \xfile{.dtx} chooses its action depending on the format:
% \begin{description}
% \item[\plainTeX:] Run \docstrip\ and extract the files.
% \item[\LaTeX:] Generate the documentation.
% \end{description}
% If you insist on using \LaTeX\ for \docstrip\ (really,
% \docstrip\ does not need \LaTeX), then inform the autodetect routine
% about your intention:
% \begin{quote}
%   \verb|latex \let\install=y% \iffalse meta-comment
%
% File: hypbmsec.dtx
% Version: 2016/05/16 v2.5
% Info: Bookmarks in sectioning commands
%
% Copyright (C) 1998-2000, 2006, 2007 by
%    Heiko Oberdiek <heiko.oberdiek at googlemail.com>
%    2016
%    https://github.com/ho-tex/oberdiek/issues
%
% This work may be distributed and/or modified under the
% conditions of the LaTeX Project Public License, either
% version 1.3c of this license or (at your option) any later
% version. This version of this license is in
%    http://www.latex-project.org/lppl/lppl-1-3c.txt
% and the latest version of this license is in
%    http://www.latex-project.org/lppl.txt
% and version 1.3 or later is part of all distributions of
% LaTeX version 2005/12/01 or later.
%
% This work has the LPPL maintenance status "maintained".
%
% This Current Maintainer of this work is Heiko Oberdiek.
%
% This work consists of the main source file hypbmsec.dtx
% and the derived files
%    hypbmsec.sty, hypbmsec.pdf, hypbmsec.ins, hypbmsec.drv.
%
% Distribution:
%    CTAN:macros/latex/contrib/oberdiek/hypbmsec.dtx
%    CTAN:macros/latex/contrib/oberdiek/hypbmsec.pdf
%
% Unpacking:
%    (a) If hypbmsec.ins is present:
%           tex hypbmsec.ins
%    (b) Without hypbmsec.ins:
%           tex hypbmsec.dtx
%    (c) If you insist on using LaTeX
%           latex \let\install=y% \iffalse meta-comment
%
% File: hypbmsec.dtx
% Version: 2016/05/16 v2.5
% Info: Bookmarks in sectioning commands
%
% Copyright (C) 1998-2000, 2006, 2007 by
%    Heiko Oberdiek <heiko.oberdiek at googlemail.com>
%    2016
%    https://github.com/ho-tex/oberdiek/issues
%
% This work may be distributed and/or modified under the
% conditions of the LaTeX Project Public License, either
% version 1.3c of this license or (at your option) any later
% version. This version of this license is in
%    http://www.latex-project.org/lppl/lppl-1-3c.txt
% and the latest version of this license is in
%    http://www.latex-project.org/lppl.txt
% and version 1.3 or later is part of all distributions of
% LaTeX version 2005/12/01 or later.
%
% This work has the LPPL maintenance status "maintained".
%
% This Current Maintainer of this work is Heiko Oberdiek.
%
% This work consists of the main source file hypbmsec.dtx
% and the derived files
%    hypbmsec.sty, hypbmsec.pdf, hypbmsec.ins, hypbmsec.drv.
%
% Distribution:
%    CTAN:macros/latex/contrib/oberdiek/hypbmsec.dtx
%    CTAN:macros/latex/contrib/oberdiek/hypbmsec.pdf
%
% Unpacking:
%    (a) If hypbmsec.ins is present:
%           tex hypbmsec.ins
%    (b) Without hypbmsec.ins:
%           tex hypbmsec.dtx
%    (c) If you insist on using LaTeX
%           latex \let\install=y\input{hypbmsec.dtx}
%        (quote the arguments according to the demands of your shell)
%
% Documentation:
%    (a) If hypbmsec.drv is present:
%           latex hypbmsec.drv
%    (b) Without hypbmsec.drv:
%           latex hypbmsec.dtx; ...
%    The class ltxdoc loads the configuration file ltxdoc.cfg
%    if available. Here you can specify further options, e.g.
%    use A4 as paper format:
%       \PassOptionsToClass{a4paper}{article}
%
%    Programm calls to get the documentation (example):
%       pdflatex hypbmsec.dtx
%       makeindex -s gind.ist hypbmsec.idx
%       pdflatex hypbmsec.dtx
%       makeindex -s gind.ist hypbmsec.idx
%       pdflatex hypbmsec.dtx
%
% Installation:
%    TDS:tex/latex/oberdiek/hypbmsec.sty
%    TDS:doc/latex/oberdiek/hypbmsec.pdf
%    TDS:source/latex/oberdiek/hypbmsec.dtx
%
%<*ignore>
\begingroup
  \catcode123=1 %
  \catcode125=2 %
  \def\x{LaTeX2e}%
\expandafter\endgroup
\ifcase 0\ifx\install y1\fi\expandafter
         \ifx\csname processbatchFile\endcsname\relax\else1\fi
         \ifx\fmtname\x\else 1\fi\relax
\else\csname fi\endcsname
%</ignore>
%<*install>
\input docstrip.tex
\Msg{************************************************************************}
\Msg{* Installation}
\Msg{* Package: hypbmsec 2016/05/16 v2.5 Bookmarks in sectioning commands (HO)}
\Msg{************************************************************************}

\keepsilent
\askforoverwritefalse

\let\MetaPrefix\relax
\preamble

This is a generated file.

Project: hypbmsec
Version: 2016/05/16 v2.5

Copyright (C) 1998-2000, 2006, 2007 by
   Heiko Oberdiek <heiko.oberdiek at googlemail.com>

This work may be distributed and/or modified under the
conditions of the LaTeX Project Public License, either
version 1.3c of this license or (at your option) any later
version. This version of this license is in
   http://www.latex-project.org/lppl/lppl-1-3c.txt
and the latest version of this license is in
   http://www.latex-project.org/lppl.txt
and version 1.3 or later is part of all distributions of
LaTeX version 2005/12/01 or later.

This work has the LPPL maintenance status "maintained".

This Current Maintainer of this work is Heiko Oberdiek.

This work consists of the main source file hypbmsec.dtx
and the derived files
   hypbmsec.sty, hypbmsec.pdf, hypbmsec.ins, hypbmsec.drv.

\endpreamble
\let\MetaPrefix\DoubleperCent

\generate{%
  \file{hypbmsec.ins}{\from{hypbmsec.dtx}{install}}%
  \file{hypbmsec.drv}{\from{hypbmsec.dtx}{driver}}%
  \usedir{tex/latex/oberdiek}%
  \file{hypbmsec.sty}{\from{hypbmsec.dtx}{package}}%
  \nopreamble
  \nopostamble
  \usedir{source/latex/oberdiek/catalogue}%
  \file{hypbmsec.xml}{\from{hypbmsec.dtx}{catalogue}}%
}

\catcode32=13\relax% active space
\let =\space%
\Msg{************************************************************************}
\Msg{*}
\Msg{* To finish the installation you have to move the following}
\Msg{* file into a directory searched by TeX:}
\Msg{*}
\Msg{*     hypbmsec.sty}
\Msg{*}
\Msg{* To produce the documentation run the file `hypbmsec.drv'}
\Msg{* through LaTeX.}
\Msg{*}
\Msg{* Happy TeXing!}
\Msg{*}
\Msg{************************************************************************}

\endbatchfile
%</install>
%<*ignore>
\fi
%</ignore>
%<*driver>
\NeedsTeXFormat{LaTeX2e}
\ProvidesFile{hypbmsec.drv}%
  [2016/05/16 v2.5 Bookmarks in sectioning commands (HO)]%
\documentclass{ltxdoc}
\usepackage{holtxdoc}[2011/11/22]
\begin{document}
  \DocInput{hypbmsec.dtx}%
\end{document}
%</driver>
% \fi
%
%
% \CharacterTable
%  {Upper-case    \A\B\C\D\E\F\G\H\I\J\K\L\M\N\O\P\Q\R\S\T\U\V\W\X\Y\Z
%   Lower-case    \a\b\c\d\e\f\g\h\i\j\k\l\m\n\o\p\q\r\s\t\u\v\w\x\y\z
%   Digits        \0\1\2\3\4\5\6\7\8\9
%   Exclamation   \!     Double quote  \"     Hash (number) \#
%   Dollar        \$     Percent       \%     Ampersand     \&
%   Acute accent  \'     Left paren    \(     Right paren   \)
%   Asterisk      \*     Plus          \+     Comma         \,
%   Minus         \-     Point         \.     Solidus       \/
%   Colon         \:     Semicolon     \;     Less than     \<
%   Equals        \=     Greater than  \>     Question mark \?
%   Commercial at \@     Left bracket  \[     Backslash     \\
%   Right bracket \]     Circumflex    \^     Underscore    \_
%   Grave accent  \`     Left brace    \{     Vertical bar  \|
%   Right brace   \}     Tilde         \~}
%
% \GetFileInfo{hypbmsec.drv}
%
% \title{The \xpackage{hypbmsec} package}
% \date{2016/05/16 v2.5}
% \author{Heiko Oberdiek\thanks
% {Please report any issues at https://github.com/ho-tex/oberdiek/issues}\\
% \xemail{heiko.oberdiek at googlemail.com}}
%
% \maketitle
%
% \begin{abstract}
% This package expands the syntax of the sectioning commands. If the
% argument of the sectioning commands isn't usable as outline entry,
% a replacement for the bookmarks can be given.
% \end{abstract}
%
% \tableofcontents
%
% \newcommand{\type}[1]{\textsf{#1}}
%
% ^^A No thread support.
% \newenvironment{article}[1]{}{}
%
% \section{Usage}
%
% \subsection{Bookmarks restrictions}\label{sec:restrictions}
%    Outline entries (bookmarks) are written to a file and have
%    to obey the pdf specification.
%    Therefore they have several restrictions:
%    \begin{itemize}
%    \item Bookmarks have to be encoded in PDFDocEncoding^^A
%          \footnote{\Package{hyperref} doesn't support
%            Unicode.}.
%    \item They should only expand to letters and spaces.
%    \item The result of expansion have to be a valid pdf string.
%    \item Stomach commands like \cmd{\relax}, box commands, math,
%          assignments, or definitions aren't allowed.
%    \item Short entries are recommended, which allow a clear view.
%    \end{itemize}
%
% \subsection{\texorpdfstring{\cmd{\texorpdfstring}}{^^A
%    \textbackslash texorpdfstring}}
%    The generic way in package \Package{hyperref} is the use
%    of \cmd{\texorpdfstring}^^A
%    \footnote{In versions of \Package{hyperref} below 6.54 see
%      \cmd{\ifbookmark}.}:
%    \begin{quote}
%\begin{verbatim}
%\section{Pythagoras:
%  \texorpdfstring{$a^2+b^2=c^2}{%
%    a\texttwosuperior\ + b\texttwosuperior\ =
%    c\texttwosuperior}%
%}
%\end{verbatim}
%    \end{quote}
%
% \subsection{Sectioning commands}
%    The package \Package{hyperref} automatically generates
%    bookmarks from the sectioning commands,
%    unless it is suppressed by an option.
%    Commands that structure the text are here called
%    ``sectioning commands'':
%    \begin{quote}
%    \cmd{\part}, \cmd{\chapter},\\
%    \cmd{\section}, \cmd{\subsection}, \cmd{\subsubsection},\\
%    \cmd{\paragraph}, \cmd{\subparagraph}
%    \end{quote}
%
% \subsection{Places\texorpdfstring{ for sectioning strings}{}}
%    \label{sec:places}
%    The argument(s) of these commands are used on several places:
%    \begin{description}
%    \item[\type{text}]
%      The current text without restrictions.
%    \item[\type{toc}]
%      The headlines and the table of contents with the
%      restrictions of ``moving arguments''.
%    \item[\type{out}]
%      The outlines with many restrictions: The outline
%      have to expand to a valid pdf string with PDFDocEncoding
%      (see \ref{sec:restrictions}).
%    \end{description}
%
% \subsection{\texorpdfstring{Solution with o}{O}ptional arguments}
%    If the user wants to use a footnote within a sectioning command,
%    the \LaTeX{} solution is an optional argument:
%    \begin{quote}
%      |\section[Title]{Title\footnote{Footnote text}}|
%    \end{quote}
%    Now |Title| without the footnote is used in the headlines and
%    the table of contents. Also \Package{hyperref} uses it for the
%    bookmarks.
%
%    This package \Package{\filename} offers two possibilities to
%    specify a separate outline entry:
%    \begin{itemize}
%    \item An additional second optional argument in square brackets.
%    \item An additional optional argument in parentheses (in
%          assoziation with a pdf string that is internally surrounded
%          by parentheses, too).
%    \end{itemize}
%    Because \Package{\filename} stores the original meaning of the
%    sectioning commands and uses them again, there should be no
%    problems with packages that redefine the sectioning commands, if
%    these packages doesn't change the syntax.
%
% \subsection{Syntax}
%    The following examples show the syntax of the sectioning
%    commands. For the places the strings appear the abbreviations
%    are used, that are introduced in \ref{sec:places}.
%
% \subsubsection{\texorpdfstring{Star form}{^^A
%    \textbackslash section*\{\}}}
%    The behaviour of the star form isn't changed. The string
%    appears only in the current text:
%    \begin{article}{Syntax}
%    \begin{quote}
%      |\section*{text}|
%    \end{quote}
%    \end{article}
%
% \subsubsection{\texorpdfstring{Without optional arguments}{^^A
%    \textbackslash section\{\}}}
%    The normal case, the string in the mandatory argument is
%    used for all places:
%    \begin{article}{Syntax}
%    \begin{quote}
%      |\section{text, toc, out}|
%    \end{quote}
%    \end{article}
%
% \subsubsection{\texorpdfstring{One optional argument}{^^A
%    \textbackslash section[]\{\}}}
%    Also the form with one optional parameter in square brackets isn't
%    new; for the bookmarks the optional parameter is used:
%    \begin{article}{Syntax}
%    \begin{quote}
%      |\section[toc, out]{text}|
%    \end{quote}
%    \end{article}
%
% \subsubsection{\texorpdfstring{Two optional arguments}{^^A
%    \textbackslash section[][out]\{\}}}\label{sec:two}
%    The second optional parameter in square brackets is introduced
%    by this package to specify an outline entry:
%    \begin{article}{Syntax}
%    \begin{quote}
%      |\section[toc][out]{text}|
%    \end{quote}
%    \end{article}
%
% \subsubsection{\texorpdfstring{Optional argument in parentheses}{^^A
%    \textbackslash section(out)\{\}}}
%    Often the \type{toc} and the \type{text} string would be the same.
%    With the method of the two optional arguments in square brackets
%    (see \ref{sec:two}) this string must be given twice,
%    if the user only wants to specify a different outline entry.
%    Therefore this package offers another possibility:
%    In association with the internal representation in the pdf file
%    an outline entry can be given in parentheses.
%    So the package can easily distinguish between
%    the two forms of optional arguments and the order does not matter:
%    \begin{article}{Syntax}
%    \begin{quote}
%      |\section(out){toc, text}|\\
%      |\section[toc](out){text}|\\
%      |\section(out)[toc]{text}|
%    \end{quote}
%    \end{article}
%
% \subsection{Without \Package{hyperref}}
%    Package \Package{\filename} uses \Package{hyperref} for support of
%    the bookmarks, but this package is not required.
%    If \Package{hyperref} isn't loaded, or
%    is called with a driver that doesn't support bookmarks,
%    package \Package{\filename} shouldn't be removed,
%    because this would lead to
%    a wrong syntax of the sectioning commands.
%    In any cases package \Package{\filename}
%    supports its syntax and ignores the outline entries,
%    if there are no code for bookmarks.
%    So it is possible to write texts,
%    that are processed with several drivers to get different output
%    formats.
%
% \subsection{Protecting parentheses}
%    If the string itself contains parentheses, they have to be hidden
%    from \TeX's argument parsing mechanism.
%    The argument should be surrounded
%    by curly braces:
%    \begin{quote}
%      |\section({outlines(bookmarks)}){text, toc}|
%    \end{quote}
%    With version 6.54 of \Package{hyperref} the other standard method
%    works, too: The closing parenthesis is protected:
%    \begin{quote}
%      |\section(outlines(bookmarks{)}){text, toc}|
%    \end{quote}
%
% \StopEventually{
% }
%
% \section{Implementation}
%    \begin{macrocode}
%<*package>
%    \end{macrocode}
%    Package identification.
%    \begin{macrocode}
\NeedsTeXFormat{LaTeX2e}
\ProvidesPackage{hypbmsec}%
  [2016/05/16 v2.5 Bookmarks in sectioning commands (HO)]
%    \end{macrocode}
%
%    Because of redifining the sectioning commands, it is dangerous
%    to reload the package several times.
%    \begin{macrocode}
\@ifundefined{hbs@do}{}{%
  \PackageInfo{hypbmsec}{Package 'hypbmsec' is already loaded}%
  \endinput
}
%    \end{macrocode}
%
%    \begin{macro}{\hbs@do}
%    The redefined sectioning commands calls \cmd{\hbs@do}. It does
%    \begin{itemize}
%    \item handle the star case.
%    \item resets the macros that store the entries for the outlines
%          (\cmd{\hbs@bmstring}) and table of contents (\cmd{\hbs@tocstring}).
%    \item store the sectioning command |#1| in \cmd{\hbs@seccmd}
%          for later reuse.
%    \item at last call \cmd{\hbs@checkarg} that scans and interprets the
%          parameters of the redefined sectioning command.
%    \end{itemize}
%    \begin{macrocode}
\def\hbs@do#1{%
  \@ifstar{#1*}{%
    \let\hbs@tocstring\relax
    \let\hbs@bmstring\relax
    \let\hbs@seccmd#1%
    \hbs@checkarg
  }%
}
%    \end{macrocode}
%    \end{macro}
%
%    \begin{macro}{\hbs@checkarg}
%    \cmd{\hbs@checkarg} determines the type of the next argument:
%    \begin{itemize}
%    \item An optional argument in square brackets can be an entry
%          for the table of contents or the bookmarks. It will be
%          read by \cmd{\hbs@getsquare}
%    \item An optional argument in parentheses is an outline entry.
%          This is worked off by \cmd{\hbs@getbookmark}.
%    \item If there are no more optional arguments, \cmd{\hbs@process}
%          reads the mandatory argument and calls the original
%          sectioning commands.
%    \end{itemize}
%    \begin{macrocode}
\def\hbs@checkarg{%
  \@ifnextchar[\hbs@getsquare{%
    \@ifnextchar(\hbs@getbookmark\hbs@process
  }%
}
%    \end{macrocode}
%    \end{macro}
%
%    \begin{macro}{\hbs@getsquare}
%    \cmd{\hbs@getsquare} reads an optional argument in square
%    brackets and determines, if this is an entry for the table
%    of contents or the bookmarks.
%    \begin{macrocode}
\long\def\hbs@getsquare[#1]{%
  \ifx\hbs@tocstring\relax
    \def\hbs@tocstring{#1}%
  \else
    \hbs@bmdef{#1}%
  \fi
  \hbs@checkarg
}
%    \end{macrocode}
%    \end{macro}
%
%    \begin{macro}{\hbs@getbookmark}
%    \cmd{\hbs@getbookmark} reads an outline entry in parentheses.
%    \begin{macrocode}
\def\hbs@getbookmark(#1){%
  \hbs@bmdef{#1}%
  \hbs@checkarg
}
%    \end{macrocode}
%    \end{macro}
%
%    \begin{macro}{\hbs@bmdef}
%    The command \cmd{\hbs@bmdef} save the bookmark entry in
%    parameter |#1| in the macro \cmd{\hbs@bmstring} and catches
%    the case, if the user has given several outline strings.
%    \begin{macrocode}
\def\hbs@bmdef#1{%
  \ifx\hbs@bmstring\relax
    \def\hbs@bmstring{#1}%
  \else
    \PackageError{hypbmsec}{%
      Sectioning command with too many parameters%
    }{%
      You can only give one outline entry.%
    }%
  \fi
}
%    \end{macrocode}
%    \end{macro}
%
%    \begin{macro}{\hbs@process}
%    The parameter |#1| is the mandatory argument of the sectioning
%    commands. \cmd{\hbs@process} calls the original sectioning command
%    stored in \cmd{\hbs@seccmd} with arguments that depend of which
%    optional argument are used previously.
%    \begin{macrocode}
\long\def\hbs@process#1{%
  \ifx\hbs@tocstring\relax
    \ifx\hbs@bmstring\relax
      \hbs@seccmd{#1}%
    \else
      \begingroup
        \def\x##1{\endgroup
          \hbs@seccmd{\texorpdfstring{#1}{##1}}%
        }%
      \expandafter\x\expandafter{\hbs@bmstring}%
    \fi
  \else
    \ifx\hbs@bmstring\relax
      \expandafter\hbs@seccmd\expandafter[%
        \expandafter{\hbs@tocstring}%
      ]{#1}%
    \else
      \expandafter\expandafter\expandafter
      \hbs@seccmd\expandafter\expandafter\expandafter[%
        \expandafter\expandafter\expandafter
        \texorpdfstring
        \expandafter\expandafter\expandafter{%
          \expandafter\hbs@tocstring\expandafter
        }\expandafter{%
          \hbs@bmstring
        }%
      ]{#1}%
    \fi
  \fi
}
%    \end{macrocode}
%    \end{macro}
%
%    We have to check, whether package \Package{hyperref} is loaded
%    and have to provide a definition for \cmd{\texorpdfstring}.
%    Because \Package{hyperref} can be loaded after this package,
%    we do the work later (\cmd{\AtBeginDocument}).
%
%    This code only checks versions of \Package{hyperref} that
%    define \cmd{\ifbookmark} (v6.4x until v6.53) or
%    \cmd{\texorpdfstring} (v6.54 and above). Older versions aren't
%    supported.
%    \begin{macrocode}
\AtBeginDocument{%
  \@ifundefined{texorpdfstring}{%
    \@ifundefined{ifbookmark}{%
      \let\texorpdfstring\@firstoftwo
      \@ifpackageloaded{hyperref}{%
        \PackageInfo{hypbmsec}{%
          \ifx\hy@driver\@empty
            Default driver %
          \else
            '\hy@driver' %
          \fi
          of hyperref not supported,\MessageBreak
          bookmark parameters will be ignored%
        }%
      }{%
        \PackageInfo{hypbmsec}{%
          Package hyperref not loaded,\MessageBreak
          bookmark parameters will be ignored%
        }%
      }%
    }%
    {%
      \newcommand\texorpdfstring[2]{\ifbookmark{#2}{#1}}%
      \PackageWarningNoLine{hypbmsec}{%
        Old hyperref version found,\MessageBreak
        update of hyperref recommended%
      }%
    }%
  }{}%
%    \end{macrocode}
%
%    Other packages are allowed to redefine the sectioning commands,
%    if they does not change the syntax. Therefore the redefinitons
%    of this package should be done after the other packages.
%    \begin{macrocode}
  \let\hbs@part         \part
  \let\hbs@section      \section
  \let\hbs@subsection   \subsection
  \let\hbs@subsubsection\subsubsection
  \let\hbs@paragraph    \paragraph
  \let\hbs@subparagraph \subparagraph
  \renewcommand\part         {\hbs@do\hbs@part}%
  \renewcommand\section      {\hbs@do\hbs@section}%
  \renewcommand\subsection   {\hbs@do\hbs@subsection}%
  \renewcommand\subsubsection{\hbs@do\hbs@subsubsection}%
  \renewcommand\paragraph    {\hbs@do\hbs@paragraph}%
  \renewcommand\subparagraph {\hbs@do\hbs@subparagraph}%
  \begingroup\expandafter\expandafter\expandafter\endgroup
  \expandafter\ifx\csname chapter\endcsname\relax\else
    \let\hbs@chapter    \chapter
    \renewcommand\chapter    {\hbs@do\hbs@chapter}%
  \fi
}
%    \end{macrocode}
%
%    \begin{macrocode}
%</package>
%    \end{macrocode}
%
% \section{Installation}
%
% \subsection{Download}
%
% \paragraph{Package.} This package is available on
% CTAN\footnote{\url{http://ctan.org/pkg/hypbmsec}}:
% \begin{description}
% \item[\CTAN{macros/latex/contrib/oberdiek/hypbmsec.dtx}] The source file.
% \item[\CTAN{macros/latex/contrib/oberdiek/hypbmsec.pdf}] Documentation.
% \end{description}
%
%
% \paragraph{Bundle.} All the packages of the bundle `oberdiek'
% are also available in a TDS compliant ZIP archive. There
% the packages are already unpacked and the documentation files
% are generated. The files and directories obey the TDS standard.
% \begin{description}
% \item[\CTAN{install/macros/latex/contrib/oberdiek.tds.zip}]
% \end{description}
% \emph{TDS} refers to the standard ``A Directory Structure
% for \TeX\ Files'' (\CTAN{tds/tds.pdf}). Directories
% with \xfile{texmf} in their name are usually organized this way.
%
% \subsection{Bundle installation}
%
% \paragraph{Unpacking.} Unpack the \xfile{oberdiek.tds.zip} in the
% TDS tree (also known as \xfile{texmf} tree) of your choice.
% Example (linux):
% \begin{quote}
%   |unzip oberdiek.tds.zip -d ~/texmf|
% \end{quote}
%
% \paragraph{Script installation.}
% Check the directory \xfile{TDS:scripts/oberdiek/} for
% scripts that need further installation steps.
% Package \xpackage{attachfile2} comes with the Perl script
% \xfile{pdfatfi.pl} that should be installed in such a way
% that it can be called as \texttt{pdfatfi}.
% Example (linux):
% \begin{quote}
%   |chmod +x scripts/oberdiek/pdfatfi.pl|\\
%   |cp scripts/oberdiek/pdfatfi.pl /usr/local/bin/|
% \end{quote}
%
% \subsection{Package installation}
%
% \paragraph{Unpacking.} The \xfile{.dtx} file is a self-extracting
% \docstrip\ archive. The files are extracted by running the
% \xfile{.dtx} through \plainTeX:
% \begin{quote}
%   \verb|tex hypbmsec.dtx|
% \end{quote}
%
% \paragraph{TDS.} Now the different files must be moved into
% the different directories in your installation TDS tree
% (also known as \xfile{texmf} tree):
% \begin{quote}
% \def\t{^^A
% \begin{tabular}{@{}>{\ttfamily}l@{ $\rightarrow$ }>{\ttfamily}l@{}}
%   hypbmsec.sty & tex/latex/oberdiek/hypbmsec.sty\\
%   hypbmsec.pdf & doc/latex/oberdiek/hypbmsec.pdf\\
%   hypbmsec.dtx & source/latex/oberdiek/hypbmsec.dtx\\
% \end{tabular}^^A
% }^^A
% \sbox0{\t}^^A
% \ifdim\wd0>\linewidth
%   \begingroup
%     \advance\linewidth by\leftmargin
%     \advance\linewidth by\rightmargin
%   \edef\x{\endgroup
%     \def\noexpand\lw{\the\linewidth}^^A
%   }\x
%   \def\lwbox{^^A
%     \leavevmode
%     \hbox to \linewidth{^^A
%       \kern-\leftmargin\relax
%       \hss
%       \usebox0
%       \hss
%       \kern-\rightmargin\relax
%     }^^A
%   }^^A
%   \ifdim\wd0>\lw
%     \sbox0{\small\t}^^A
%     \ifdim\wd0>\linewidth
%       \ifdim\wd0>\lw
%         \sbox0{\footnotesize\t}^^A
%         \ifdim\wd0>\linewidth
%           \ifdim\wd0>\lw
%             \sbox0{\scriptsize\t}^^A
%             \ifdim\wd0>\linewidth
%               \ifdim\wd0>\lw
%                 \sbox0{\tiny\t}^^A
%                 \ifdim\wd0>\linewidth
%                   \lwbox
%                 \else
%                   \usebox0
%                 \fi
%               \else
%                 \lwbox
%               \fi
%             \else
%               \usebox0
%             \fi
%           \else
%             \lwbox
%           \fi
%         \else
%           \usebox0
%         \fi
%       \else
%         \lwbox
%       \fi
%     \else
%       \usebox0
%     \fi
%   \else
%     \lwbox
%   \fi
% \else
%   \usebox0
% \fi
% \end{quote}
% If you have a \xfile{docstrip.cfg} that configures and enables \docstrip's
% TDS installing feature, then some files can already be in the right
% place, see the documentation of \docstrip.
%
% \subsection{Refresh file name databases}
%
% If your \TeX~distribution
% (\teTeX, \mikTeX, \dots) relies on file name databases, you must refresh
% these. For example, \teTeX\ users run \verb|texhash| or
% \verb|mktexlsr|.
%
% \subsection{Some details for the interested}
%
% \paragraph{Attached source.}
%
% The PDF documentation on CTAN also includes the
% \xfile{.dtx} source file. It can be extracted by
% AcrobatReader 6 or higher. Another option is \textsf{pdftk},
% e.g. unpack the file into the current directory:
% \begin{quote}
%   \verb|pdftk hypbmsec.pdf unpack_files output .|
% \end{quote}
%
% \paragraph{Unpacking with \LaTeX.}
% The \xfile{.dtx} chooses its action depending on the format:
% \begin{description}
% \item[\plainTeX:] Run \docstrip\ and extract the files.
% \item[\LaTeX:] Generate the documentation.
% \end{description}
% If you insist on using \LaTeX\ for \docstrip\ (really,
% \docstrip\ does not need \LaTeX), then inform the autodetect routine
% about your intention:
% \begin{quote}
%   \verb|latex \let\install=y\input{hypbmsec.dtx}|
% \end{quote}
% Do not forget to quote the argument according to the demands
% of your shell.
%
% \paragraph{Generating the documentation.}
% You can use both the \xfile{.dtx} or the \xfile{.drv} to generate
% the documentation. The process can be configured by the
% configuration file \xfile{ltxdoc.cfg}. For instance, put this
% line into this file, if you want to have A4 as paper format:
% \begin{quote}
%   \verb|\PassOptionsToClass{a4paper}{article}|
% \end{quote}
% An example follows how to generate the
% documentation with pdf\LaTeX:
% \begin{quote}
%\begin{verbatim}
%pdflatex hypbmsec.dtx
%makeindex -s gind.ist hypbmsec.idx
%pdflatex hypbmsec.dtx
%makeindex -s gind.ist hypbmsec.idx
%pdflatex hypbmsec.dtx
%\end{verbatim}
% \end{quote}
%
% \section{Catalogue}
%
% The following XML file can be used as source for the
% \href{http://mirror.ctan.org/help/Catalogue/catalogue.html}{\TeX\ Catalogue}.
% The elements \texttt{caption} and \texttt{description} are imported
% from the original XML file from the Catalogue.
% The name of the XML file in the Catalogue is \xfile{hypbmsec.xml}.
%    \begin{macrocode}
%<*catalogue>
<?xml version='1.0' encoding='us-ascii'?>
<!DOCTYPE entry SYSTEM 'catalogue.dtd'>
<entry datestamp='$Date$' modifier='$Author$' id='hypbmsec'>
  <name>hypbmsec</name>
  <caption>Hypertext bookmarks in sectioning commands.</caption>
  <authorref id='auth:oberdiek'/>
  <copyright owner='Heiko Oberdiek' year='1998-2000,2006,2007'/>
  <license type='lppl1.3'/>
  <version number='2.5'/>
  <description>
    Bookmark entries can be given as another argument to the LaTeX
    sectioning commands. The <xref refid='hyperref'>hyperref</xref>
    package is required to get the bookmarks, but the syntax
    works without it.
    <p/>
    This package is part of the <xref refid='oberdiek'>oberdiek</xref>
    bundle.
  </description>
  <documentation details='Package documentation'
      href='ctan:/macros/latex/contrib/oberdiek/hypbmsec.pdf'/>
  <ctan file='true' path='/macros/latex/contrib/oberdiek/hypbmsec.dtx'/>
  <miktex location='oberdiek'/>
  <texlive location='oberdiek'/>
  <install path='/macros/latex/contrib/oberdiek/oberdiek.tds.zip'/>
</entry>
%</catalogue>
%    \end{macrocode}
%
% \begin{History}
%   \begin{Version}{1998/11/20 v1.0}
%   \item
%     First version.
%   \item
%     It merges package \xpackage{hysecopt} and
%   \item
%     package \xpackage{hypbmpar}.
%   \item
%     Published for the DANTE'99 meeting^^A
%     \URL{}{http://dante99.cs.uni-dortmund.de/handouts/oberdiek/hypbmsec.sty}.
%   \end{Version}
%   \begin{Version}{1999/04/12 v2.0}
%   \item
%     Adaptation to \Package{hyperref} version 6.54.
%   \item
%     Documentation in dtx format.
%   \item
%     Copyright: LPPL (\CTAN{macros/latex/base/lppl.txt})
%   \item
%     First CTAN release.
%   \end{Version}
%   \begin{Version}{2000/03/22 v2.1}
%   \item
%     Bug fix in redefinition of \cmd{\chapter}.
%   \item
%     Copyright: LPPL 1.2
%   \end{Version}
%   \begin{Version}{2006/02/20 v2.2}
%   \item
%     Code is not changed.
%   \item
%     New DTX framework.
%   \item
%     LPPL 1.3
%   \end{Version}
%   \begin{Version}{2007/03/05 v2.3}
%   \item
%     Bug fix: Expand \cs{hbs@tocstring} and \cs{hbs@bmstring} before
%     calling \cs{hbs@seccmd}.
%   \end{Version}
%   \begin{Version}{2007/04/11 v2.4}
%   \item
%     Line ends sanitized.
%   \end{Version}
%   \begin{Version}{2016/05/16 v2.5}
%   \item
%     Documentation updates.
%   \end{Version}
% \end{History}
%
% \PrintIndex
%
% \Finale
\endinput

%        (quote the arguments according to the demands of your shell)
%
% Documentation:
%    (a) If hypbmsec.drv is present:
%           latex hypbmsec.drv
%    (b) Without hypbmsec.drv:
%           latex hypbmsec.dtx; ...
%    The class ltxdoc loads the configuration file ltxdoc.cfg
%    if available. Here you can specify further options, e.g.
%    use A4 as paper format:
%       \PassOptionsToClass{a4paper}{article}
%
%    Programm calls to get the documentation (example):
%       pdflatex hypbmsec.dtx
%       makeindex -s gind.ist hypbmsec.idx
%       pdflatex hypbmsec.dtx
%       makeindex -s gind.ist hypbmsec.idx
%       pdflatex hypbmsec.dtx
%
% Installation:
%    TDS:tex/latex/oberdiek/hypbmsec.sty
%    TDS:doc/latex/oberdiek/hypbmsec.pdf
%    TDS:source/latex/oberdiek/hypbmsec.dtx
%
%<*ignore>
\begingroup
  \catcode123=1 %
  \catcode125=2 %
  \def\x{LaTeX2e}%
\expandafter\endgroup
\ifcase 0\ifx\install y1\fi\expandafter
         \ifx\csname processbatchFile\endcsname\relax\else1\fi
         \ifx\fmtname\x\else 1\fi\relax
\else\csname fi\endcsname
%</ignore>
%<*install>
\input docstrip.tex
\Msg{************************************************************************}
\Msg{* Installation}
\Msg{* Package: hypbmsec 2016/05/16 v2.5 Bookmarks in sectioning commands (HO)}
\Msg{************************************************************************}

\keepsilent
\askforoverwritefalse

\let\MetaPrefix\relax
\preamble

This is a generated file.

Project: hypbmsec
Version: 2016/05/16 v2.5

Copyright (C) 1998-2000, 2006, 2007 by
   Heiko Oberdiek <heiko.oberdiek at googlemail.com>

This work may be distributed and/or modified under the
conditions of the LaTeX Project Public License, either
version 1.3c of this license or (at your option) any later
version. This version of this license is in
   http://www.latex-project.org/lppl/lppl-1-3c.txt
and the latest version of this license is in
   http://www.latex-project.org/lppl.txt
and version 1.3 or later is part of all distributions of
LaTeX version 2005/12/01 or later.

This work has the LPPL maintenance status "maintained".

This Current Maintainer of this work is Heiko Oberdiek.

This work consists of the main source file hypbmsec.dtx
and the derived files
   hypbmsec.sty, hypbmsec.pdf, hypbmsec.ins, hypbmsec.drv.

\endpreamble
\let\MetaPrefix\DoubleperCent

\generate{%
  \file{hypbmsec.ins}{\from{hypbmsec.dtx}{install}}%
  \file{hypbmsec.drv}{\from{hypbmsec.dtx}{driver}}%
  \usedir{tex/latex/oberdiek}%
  \file{hypbmsec.sty}{\from{hypbmsec.dtx}{package}}%
  \nopreamble
  \nopostamble
  \usedir{source/latex/oberdiek/catalogue}%
  \file{hypbmsec.xml}{\from{hypbmsec.dtx}{catalogue}}%
}

\catcode32=13\relax% active space
\let =\space%
\Msg{************************************************************************}
\Msg{*}
\Msg{* To finish the installation you have to move the following}
\Msg{* file into a directory searched by TeX:}
\Msg{*}
\Msg{*     hypbmsec.sty}
\Msg{*}
\Msg{* To produce the documentation run the file `hypbmsec.drv'}
\Msg{* through LaTeX.}
\Msg{*}
\Msg{* Happy TeXing!}
\Msg{*}
\Msg{************************************************************************}

\endbatchfile
%</install>
%<*ignore>
\fi
%</ignore>
%<*driver>
\NeedsTeXFormat{LaTeX2e}
\ProvidesFile{hypbmsec.drv}%
  [2016/05/16 v2.5 Bookmarks in sectioning commands (HO)]%
\documentclass{ltxdoc}
\usepackage{holtxdoc}[2011/11/22]
\begin{document}
  \DocInput{hypbmsec.dtx}%
\end{document}
%</driver>
% \fi
%
%
% \CharacterTable
%  {Upper-case    \A\B\C\D\E\F\G\H\I\J\K\L\M\N\O\P\Q\R\S\T\U\V\W\X\Y\Z
%   Lower-case    \a\b\c\d\e\f\g\h\i\j\k\l\m\n\o\p\q\r\s\t\u\v\w\x\y\z
%   Digits        \0\1\2\3\4\5\6\7\8\9
%   Exclamation   \!     Double quote  \"     Hash (number) \#
%   Dollar        \$     Percent       \%     Ampersand     \&
%   Acute accent  \'     Left paren    \(     Right paren   \)
%   Asterisk      \*     Plus          \+     Comma         \,
%   Minus         \-     Point         \.     Solidus       \/
%   Colon         \:     Semicolon     \;     Less than     \<
%   Equals        \=     Greater than  \>     Question mark \?
%   Commercial at \@     Left bracket  \[     Backslash     \\
%   Right bracket \]     Circumflex    \^     Underscore    \_
%   Grave accent  \`     Left brace    \{     Vertical bar  \|
%   Right brace   \}     Tilde         \~}
%
% \GetFileInfo{hypbmsec.drv}
%
% \title{The \xpackage{hypbmsec} package}
% \date{2016/05/16 v2.5}
% \author{Heiko Oberdiek\thanks
% {Please report any issues at https://github.com/ho-tex/oberdiek/issues}\\
% \xemail{heiko.oberdiek at googlemail.com}}
%
% \maketitle
%
% \begin{abstract}
% This package expands the syntax of the sectioning commands. If the
% argument of the sectioning commands isn't usable as outline entry,
% a replacement for the bookmarks can be given.
% \end{abstract}
%
% \tableofcontents
%
% \newcommand{\type}[1]{\textsf{#1}}
%
% ^^A No thread support.
% \newenvironment{article}[1]{}{}
%
% \section{Usage}
%
% \subsection{Bookmarks restrictions}\label{sec:restrictions}
%    Outline entries (bookmarks) are written to a file and have
%    to obey the pdf specification.
%    Therefore they have several restrictions:
%    \begin{itemize}
%    \item Bookmarks have to be encoded in PDFDocEncoding^^A
%          \footnote{\Package{hyperref} doesn't support
%            Unicode.}.
%    \item They should only expand to letters and spaces.
%    \item The result of expansion have to be a valid pdf string.
%    \item Stomach commands like \cmd{\relax}, box commands, math,
%          assignments, or definitions aren't allowed.
%    \item Short entries are recommended, which allow a clear view.
%    \end{itemize}
%
% \subsection{\texorpdfstring{\cmd{\texorpdfstring}}{^^A
%    \textbackslash texorpdfstring}}
%    The generic way in package \Package{hyperref} is the use
%    of \cmd{\texorpdfstring}^^A
%    \footnote{In versions of \Package{hyperref} below 6.54 see
%      \cmd{\ifbookmark}.}:
%    \begin{quote}
%\begin{verbatim}
%\section{Pythagoras:
%  \texorpdfstring{$a^2+b^2=c^2}{%
%    a\texttwosuperior\ + b\texttwosuperior\ =
%    c\texttwosuperior}%
%}
%\end{verbatim}
%    \end{quote}
%
% \subsection{Sectioning commands}
%    The package \Package{hyperref} automatically generates
%    bookmarks from the sectioning commands,
%    unless it is suppressed by an option.
%    Commands that structure the text are here called
%    ``sectioning commands'':
%    \begin{quote}
%    \cmd{\part}, \cmd{\chapter},\\
%    \cmd{\section}, \cmd{\subsection}, \cmd{\subsubsection},\\
%    \cmd{\paragraph}, \cmd{\subparagraph}
%    \end{quote}
%
% \subsection{Places\texorpdfstring{ for sectioning strings}{}}
%    \label{sec:places}
%    The argument(s) of these commands are used on several places:
%    \begin{description}
%    \item[\type{text}]
%      The current text without restrictions.
%    \item[\type{toc}]
%      The headlines and the table of contents with the
%      restrictions of ``moving arguments''.
%    \item[\type{out}]
%      The outlines with many restrictions: The outline
%      have to expand to a valid pdf string with PDFDocEncoding
%      (see \ref{sec:restrictions}).
%    \end{description}
%
% \subsection{\texorpdfstring{Solution with o}{O}ptional arguments}
%    If the user wants to use a footnote within a sectioning command,
%    the \LaTeX{} solution is an optional argument:
%    \begin{quote}
%      |\section[Title]{Title\footnote{Footnote text}}|
%    \end{quote}
%    Now |Title| without the footnote is used in the headlines and
%    the table of contents. Also \Package{hyperref} uses it for the
%    bookmarks.
%
%    This package \Package{\filename} offers two possibilities to
%    specify a separate outline entry:
%    \begin{itemize}
%    \item An additional second optional argument in square brackets.
%    \item An additional optional argument in parentheses (in
%          assoziation with a pdf string that is internally surrounded
%          by parentheses, too).
%    \end{itemize}
%    Because \Package{\filename} stores the original meaning of the
%    sectioning commands and uses them again, there should be no
%    problems with packages that redefine the sectioning commands, if
%    these packages doesn't change the syntax.
%
% \subsection{Syntax}
%    The following examples show the syntax of the sectioning
%    commands. For the places the strings appear the abbreviations
%    are used, that are introduced in \ref{sec:places}.
%
% \subsubsection{\texorpdfstring{Star form}{^^A
%    \textbackslash section*\{\}}}
%    The behaviour of the star form isn't changed. The string
%    appears only in the current text:
%    \begin{article}{Syntax}
%    \begin{quote}
%      |\section*{text}|
%    \end{quote}
%    \end{article}
%
% \subsubsection{\texorpdfstring{Without optional arguments}{^^A
%    \textbackslash section\{\}}}
%    The normal case, the string in the mandatory argument is
%    used for all places:
%    \begin{article}{Syntax}
%    \begin{quote}
%      |\section{text, toc, out}|
%    \end{quote}
%    \end{article}
%
% \subsubsection{\texorpdfstring{One optional argument}{^^A
%    \textbackslash section[]\{\}}}
%    Also the form with one optional parameter in square brackets isn't
%    new; for the bookmarks the optional parameter is used:
%    \begin{article}{Syntax}
%    \begin{quote}
%      |\section[toc, out]{text}|
%    \end{quote}
%    \end{article}
%
% \subsubsection{\texorpdfstring{Two optional arguments}{^^A
%    \textbackslash section[][out]\{\}}}\label{sec:two}
%    The second optional parameter in square brackets is introduced
%    by this package to specify an outline entry:
%    \begin{article}{Syntax}
%    \begin{quote}
%      |\section[toc][out]{text}|
%    \end{quote}
%    \end{article}
%
% \subsubsection{\texorpdfstring{Optional argument in parentheses}{^^A
%    \textbackslash section(out)\{\}}}
%    Often the \type{toc} and the \type{text} string would be the same.
%    With the method of the two optional arguments in square brackets
%    (see \ref{sec:two}) this string must be given twice,
%    if the user only wants to specify a different outline entry.
%    Therefore this package offers another possibility:
%    In association with the internal representation in the pdf file
%    an outline entry can be given in parentheses.
%    So the package can easily distinguish between
%    the two forms of optional arguments and the order does not matter:
%    \begin{article}{Syntax}
%    \begin{quote}
%      |\section(out){toc, text}|\\
%      |\section[toc](out){text}|\\
%      |\section(out)[toc]{text}|
%    \end{quote}
%    \end{article}
%
% \subsection{Without \Package{hyperref}}
%    Package \Package{\filename} uses \Package{hyperref} for support of
%    the bookmarks, but this package is not required.
%    If \Package{hyperref} isn't loaded, or
%    is called with a driver that doesn't support bookmarks,
%    package \Package{\filename} shouldn't be removed,
%    because this would lead to
%    a wrong syntax of the sectioning commands.
%    In any cases package \Package{\filename}
%    supports its syntax and ignores the outline entries,
%    if there are no code for bookmarks.
%    So it is possible to write texts,
%    that are processed with several drivers to get different output
%    formats.
%
% \subsection{Protecting parentheses}
%    If the string itself contains parentheses, they have to be hidden
%    from \TeX's argument parsing mechanism.
%    The argument should be surrounded
%    by curly braces:
%    \begin{quote}
%      |\section({outlines(bookmarks)}){text, toc}|
%    \end{quote}
%    With version 6.54 of \Package{hyperref} the other standard method
%    works, too: The closing parenthesis is protected:
%    \begin{quote}
%      |\section(outlines(bookmarks{)}){text, toc}|
%    \end{quote}
%
% \StopEventually{
% }
%
% \section{Implementation}
%    \begin{macrocode}
%<*package>
%    \end{macrocode}
%    Package identification.
%    \begin{macrocode}
\NeedsTeXFormat{LaTeX2e}
\ProvidesPackage{hypbmsec}%
  [2016/05/16 v2.5 Bookmarks in sectioning commands (HO)]
%    \end{macrocode}
%
%    Because of redifining the sectioning commands, it is dangerous
%    to reload the package several times.
%    \begin{macrocode}
\@ifundefined{hbs@do}{}{%
  \PackageInfo{hypbmsec}{Package 'hypbmsec' is already loaded}%
  \endinput
}
%    \end{macrocode}
%
%    \begin{macro}{\hbs@do}
%    The redefined sectioning commands calls \cmd{\hbs@do}. It does
%    \begin{itemize}
%    \item handle the star case.
%    \item resets the macros that store the entries for the outlines
%          (\cmd{\hbs@bmstring}) and table of contents (\cmd{\hbs@tocstring}).
%    \item store the sectioning command |#1| in \cmd{\hbs@seccmd}
%          for later reuse.
%    \item at last call \cmd{\hbs@checkarg} that scans and interprets the
%          parameters of the redefined sectioning command.
%    \end{itemize}
%    \begin{macrocode}
\def\hbs@do#1{%
  \@ifstar{#1*}{%
    \let\hbs@tocstring\relax
    \let\hbs@bmstring\relax
    \let\hbs@seccmd#1%
    \hbs@checkarg
  }%
}
%    \end{macrocode}
%    \end{macro}
%
%    \begin{macro}{\hbs@checkarg}
%    \cmd{\hbs@checkarg} determines the type of the next argument:
%    \begin{itemize}
%    \item An optional argument in square brackets can be an entry
%          for the table of contents or the bookmarks. It will be
%          read by \cmd{\hbs@getsquare}
%    \item An optional argument in parentheses is an outline entry.
%          This is worked off by \cmd{\hbs@getbookmark}.
%    \item If there are no more optional arguments, \cmd{\hbs@process}
%          reads the mandatory argument and calls the original
%          sectioning commands.
%    \end{itemize}
%    \begin{macrocode}
\def\hbs@checkarg{%
  \@ifnextchar[\hbs@getsquare{%
    \@ifnextchar(\hbs@getbookmark\hbs@process
  }%
}
%    \end{macrocode}
%    \end{macro}
%
%    \begin{macro}{\hbs@getsquare}
%    \cmd{\hbs@getsquare} reads an optional argument in square
%    brackets and determines, if this is an entry for the table
%    of contents or the bookmarks.
%    \begin{macrocode}
\long\def\hbs@getsquare[#1]{%
  \ifx\hbs@tocstring\relax
    \def\hbs@tocstring{#1}%
  \else
    \hbs@bmdef{#1}%
  \fi
  \hbs@checkarg
}
%    \end{macrocode}
%    \end{macro}
%
%    \begin{macro}{\hbs@getbookmark}
%    \cmd{\hbs@getbookmark} reads an outline entry in parentheses.
%    \begin{macrocode}
\def\hbs@getbookmark(#1){%
  \hbs@bmdef{#1}%
  \hbs@checkarg
}
%    \end{macrocode}
%    \end{macro}
%
%    \begin{macro}{\hbs@bmdef}
%    The command \cmd{\hbs@bmdef} save the bookmark entry in
%    parameter |#1| in the macro \cmd{\hbs@bmstring} and catches
%    the case, if the user has given several outline strings.
%    \begin{macrocode}
\def\hbs@bmdef#1{%
  \ifx\hbs@bmstring\relax
    \def\hbs@bmstring{#1}%
  \else
    \PackageError{hypbmsec}{%
      Sectioning command with too many parameters%
    }{%
      You can only give one outline entry.%
    }%
  \fi
}
%    \end{macrocode}
%    \end{macro}
%
%    \begin{macro}{\hbs@process}
%    The parameter |#1| is the mandatory argument of the sectioning
%    commands. \cmd{\hbs@process} calls the original sectioning command
%    stored in \cmd{\hbs@seccmd} with arguments that depend of which
%    optional argument are used previously.
%    \begin{macrocode}
\long\def\hbs@process#1{%
  \ifx\hbs@tocstring\relax
    \ifx\hbs@bmstring\relax
      \hbs@seccmd{#1}%
    \else
      \begingroup
        \def\x##1{\endgroup
          \hbs@seccmd{\texorpdfstring{#1}{##1}}%
        }%
      \expandafter\x\expandafter{\hbs@bmstring}%
    \fi
  \else
    \ifx\hbs@bmstring\relax
      \expandafter\hbs@seccmd\expandafter[%
        \expandafter{\hbs@tocstring}%
      ]{#1}%
    \else
      \expandafter\expandafter\expandafter
      \hbs@seccmd\expandafter\expandafter\expandafter[%
        \expandafter\expandafter\expandafter
        \texorpdfstring
        \expandafter\expandafter\expandafter{%
          \expandafter\hbs@tocstring\expandafter
        }\expandafter{%
          \hbs@bmstring
        }%
      ]{#1}%
    \fi
  \fi
}
%    \end{macrocode}
%    \end{macro}
%
%    We have to check, whether package \Package{hyperref} is loaded
%    and have to provide a definition for \cmd{\texorpdfstring}.
%    Because \Package{hyperref} can be loaded after this package,
%    we do the work later (\cmd{\AtBeginDocument}).
%
%    This code only checks versions of \Package{hyperref} that
%    define \cmd{\ifbookmark} (v6.4x until v6.53) or
%    \cmd{\texorpdfstring} (v6.54 and above). Older versions aren't
%    supported.
%    \begin{macrocode}
\AtBeginDocument{%
  \@ifundefined{texorpdfstring}{%
    \@ifundefined{ifbookmark}{%
      \let\texorpdfstring\@firstoftwo
      \@ifpackageloaded{hyperref}{%
        \PackageInfo{hypbmsec}{%
          \ifx\hy@driver\@empty
            Default driver %
          \else
            '\hy@driver' %
          \fi
          of hyperref not supported,\MessageBreak
          bookmark parameters will be ignored%
        }%
      }{%
        \PackageInfo{hypbmsec}{%
          Package hyperref not loaded,\MessageBreak
          bookmark parameters will be ignored%
        }%
      }%
    }%
    {%
      \newcommand\texorpdfstring[2]{\ifbookmark{#2}{#1}}%
      \PackageWarningNoLine{hypbmsec}{%
        Old hyperref version found,\MessageBreak
        update of hyperref recommended%
      }%
    }%
  }{}%
%    \end{macrocode}
%
%    Other packages are allowed to redefine the sectioning commands,
%    if they does not change the syntax. Therefore the redefinitons
%    of this package should be done after the other packages.
%    \begin{macrocode}
  \let\hbs@part         \part
  \let\hbs@section      \section
  \let\hbs@subsection   \subsection
  \let\hbs@subsubsection\subsubsection
  \let\hbs@paragraph    \paragraph
  \let\hbs@subparagraph \subparagraph
  \renewcommand\part         {\hbs@do\hbs@part}%
  \renewcommand\section      {\hbs@do\hbs@section}%
  \renewcommand\subsection   {\hbs@do\hbs@subsection}%
  \renewcommand\subsubsection{\hbs@do\hbs@subsubsection}%
  \renewcommand\paragraph    {\hbs@do\hbs@paragraph}%
  \renewcommand\subparagraph {\hbs@do\hbs@subparagraph}%
  \begingroup\expandafter\expandafter\expandafter\endgroup
  \expandafter\ifx\csname chapter\endcsname\relax\else
    \let\hbs@chapter    \chapter
    \renewcommand\chapter    {\hbs@do\hbs@chapter}%
  \fi
}
%    \end{macrocode}
%
%    \begin{macrocode}
%</package>
%    \end{macrocode}
%
% \section{Installation}
%
% \subsection{Download}
%
% \paragraph{Package.} This package is available on
% CTAN\footnote{\url{http://ctan.org/pkg/hypbmsec}}:
% \begin{description}
% \item[\CTAN{macros/latex/contrib/oberdiek/hypbmsec.dtx}] The source file.
% \item[\CTAN{macros/latex/contrib/oberdiek/hypbmsec.pdf}] Documentation.
% \end{description}
%
%
% \paragraph{Bundle.} All the packages of the bundle `oberdiek'
% are also available in a TDS compliant ZIP archive. There
% the packages are already unpacked and the documentation files
% are generated. The files and directories obey the TDS standard.
% \begin{description}
% \item[\CTAN{install/macros/latex/contrib/oberdiek.tds.zip}]
% \end{description}
% \emph{TDS} refers to the standard ``A Directory Structure
% for \TeX\ Files'' (\CTAN{tds/tds.pdf}). Directories
% with \xfile{texmf} in their name are usually organized this way.
%
% \subsection{Bundle installation}
%
% \paragraph{Unpacking.} Unpack the \xfile{oberdiek.tds.zip} in the
% TDS tree (also known as \xfile{texmf} tree) of your choice.
% Example (linux):
% \begin{quote}
%   |unzip oberdiek.tds.zip -d ~/texmf|
% \end{quote}
%
% \paragraph{Script installation.}
% Check the directory \xfile{TDS:scripts/oberdiek/} for
% scripts that need further installation steps.
% Package \xpackage{attachfile2} comes with the Perl script
% \xfile{pdfatfi.pl} that should be installed in such a way
% that it can be called as \texttt{pdfatfi}.
% Example (linux):
% \begin{quote}
%   |chmod +x scripts/oberdiek/pdfatfi.pl|\\
%   |cp scripts/oberdiek/pdfatfi.pl /usr/local/bin/|
% \end{quote}
%
% \subsection{Package installation}
%
% \paragraph{Unpacking.} The \xfile{.dtx} file is a self-extracting
% \docstrip\ archive. The files are extracted by running the
% \xfile{.dtx} through \plainTeX:
% \begin{quote}
%   \verb|tex hypbmsec.dtx|
% \end{quote}
%
% \paragraph{TDS.} Now the different files must be moved into
% the different directories in your installation TDS tree
% (also known as \xfile{texmf} tree):
% \begin{quote}
% \def\t{^^A
% \begin{tabular}{@{}>{\ttfamily}l@{ $\rightarrow$ }>{\ttfamily}l@{}}
%   hypbmsec.sty & tex/latex/oberdiek/hypbmsec.sty\\
%   hypbmsec.pdf & doc/latex/oberdiek/hypbmsec.pdf\\
%   hypbmsec.dtx & source/latex/oberdiek/hypbmsec.dtx\\
% \end{tabular}^^A
% }^^A
% \sbox0{\t}^^A
% \ifdim\wd0>\linewidth
%   \begingroup
%     \advance\linewidth by\leftmargin
%     \advance\linewidth by\rightmargin
%   \edef\x{\endgroup
%     \def\noexpand\lw{\the\linewidth}^^A
%   }\x
%   \def\lwbox{^^A
%     \leavevmode
%     \hbox to \linewidth{^^A
%       \kern-\leftmargin\relax
%       \hss
%       \usebox0
%       \hss
%       \kern-\rightmargin\relax
%     }^^A
%   }^^A
%   \ifdim\wd0>\lw
%     \sbox0{\small\t}^^A
%     \ifdim\wd0>\linewidth
%       \ifdim\wd0>\lw
%         \sbox0{\footnotesize\t}^^A
%         \ifdim\wd0>\linewidth
%           \ifdim\wd0>\lw
%             \sbox0{\scriptsize\t}^^A
%             \ifdim\wd0>\linewidth
%               \ifdim\wd0>\lw
%                 \sbox0{\tiny\t}^^A
%                 \ifdim\wd0>\linewidth
%                   \lwbox
%                 \else
%                   \usebox0
%                 \fi
%               \else
%                 \lwbox
%               \fi
%             \else
%               \usebox0
%             \fi
%           \else
%             \lwbox
%           \fi
%         \else
%           \usebox0
%         \fi
%       \else
%         \lwbox
%       \fi
%     \else
%       \usebox0
%     \fi
%   \else
%     \lwbox
%   \fi
% \else
%   \usebox0
% \fi
% \end{quote}
% If you have a \xfile{docstrip.cfg} that configures and enables \docstrip's
% TDS installing feature, then some files can already be in the right
% place, see the documentation of \docstrip.
%
% \subsection{Refresh file name databases}
%
% If your \TeX~distribution
% (\teTeX, \mikTeX, \dots) relies on file name databases, you must refresh
% these. For example, \teTeX\ users run \verb|texhash| or
% \verb|mktexlsr|.
%
% \subsection{Some details for the interested}
%
% \paragraph{Attached source.}
%
% The PDF documentation on CTAN also includes the
% \xfile{.dtx} source file. It can be extracted by
% AcrobatReader 6 or higher. Another option is \textsf{pdftk},
% e.g. unpack the file into the current directory:
% \begin{quote}
%   \verb|pdftk hypbmsec.pdf unpack_files output .|
% \end{quote}
%
% \paragraph{Unpacking with \LaTeX.}
% The \xfile{.dtx} chooses its action depending on the format:
% \begin{description}
% \item[\plainTeX:] Run \docstrip\ and extract the files.
% \item[\LaTeX:] Generate the documentation.
% \end{description}
% If you insist on using \LaTeX\ for \docstrip\ (really,
% \docstrip\ does not need \LaTeX), then inform the autodetect routine
% about your intention:
% \begin{quote}
%   \verb|latex \let\install=y% \iffalse meta-comment
%
% File: hypbmsec.dtx
% Version: 2016/05/16 v2.5
% Info: Bookmarks in sectioning commands
%
% Copyright (C) 1998-2000, 2006, 2007 by
%    Heiko Oberdiek <heiko.oberdiek at googlemail.com>
%    2016
%    https://github.com/ho-tex/oberdiek/issues
%
% This work may be distributed and/or modified under the
% conditions of the LaTeX Project Public License, either
% version 1.3c of this license or (at your option) any later
% version. This version of this license is in
%    http://www.latex-project.org/lppl/lppl-1-3c.txt
% and the latest version of this license is in
%    http://www.latex-project.org/lppl.txt
% and version 1.3 or later is part of all distributions of
% LaTeX version 2005/12/01 or later.
%
% This work has the LPPL maintenance status "maintained".
%
% This Current Maintainer of this work is Heiko Oberdiek.
%
% This work consists of the main source file hypbmsec.dtx
% and the derived files
%    hypbmsec.sty, hypbmsec.pdf, hypbmsec.ins, hypbmsec.drv.
%
% Distribution:
%    CTAN:macros/latex/contrib/oberdiek/hypbmsec.dtx
%    CTAN:macros/latex/contrib/oberdiek/hypbmsec.pdf
%
% Unpacking:
%    (a) If hypbmsec.ins is present:
%           tex hypbmsec.ins
%    (b) Without hypbmsec.ins:
%           tex hypbmsec.dtx
%    (c) If you insist on using LaTeX
%           latex \let\install=y\input{hypbmsec.dtx}
%        (quote the arguments according to the demands of your shell)
%
% Documentation:
%    (a) If hypbmsec.drv is present:
%           latex hypbmsec.drv
%    (b) Without hypbmsec.drv:
%           latex hypbmsec.dtx; ...
%    The class ltxdoc loads the configuration file ltxdoc.cfg
%    if available. Here you can specify further options, e.g.
%    use A4 as paper format:
%       \PassOptionsToClass{a4paper}{article}
%
%    Programm calls to get the documentation (example):
%       pdflatex hypbmsec.dtx
%       makeindex -s gind.ist hypbmsec.idx
%       pdflatex hypbmsec.dtx
%       makeindex -s gind.ist hypbmsec.idx
%       pdflatex hypbmsec.dtx
%
% Installation:
%    TDS:tex/latex/oberdiek/hypbmsec.sty
%    TDS:doc/latex/oberdiek/hypbmsec.pdf
%    TDS:source/latex/oberdiek/hypbmsec.dtx
%
%<*ignore>
\begingroup
  \catcode123=1 %
  \catcode125=2 %
  \def\x{LaTeX2e}%
\expandafter\endgroup
\ifcase 0\ifx\install y1\fi\expandafter
         \ifx\csname processbatchFile\endcsname\relax\else1\fi
         \ifx\fmtname\x\else 1\fi\relax
\else\csname fi\endcsname
%</ignore>
%<*install>
\input docstrip.tex
\Msg{************************************************************************}
\Msg{* Installation}
\Msg{* Package: hypbmsec 2016/05/16 v2.5 Bookmarks in sectioning commands (HO)}
\Msg{************************************************************************}

\keepsilent
\askforoverwritefalse

\let\MetaPrefix\relax
\preamble

This is a generated file.

Project: hypbmsec
Version: 2016/05/16 v2.5

Copyright (C) 1998-2000, 2006, 2007 by
   Heiko Oberdiek <heiko.oberdiek at googlemail.com>

This work may be distributed and/or modified under the
conditions of the LaTeX Project Public License, either
version 1.3c of this license or (at your option) any later
version. This version of this license is in
   http://www.latex-project.org/lppl/lppl-1-3c.txt
and the latest version of this license is in
   http://www.latex-project.org/lppl.txt
and version 1.3 or later is part of all distributions of
LaTeX version 2005/12/01 or later.

This work has the LPPL maintenance status "maintained".

This Current Maintainer of this work is Heiko Oberdiek.

This work consists of the main source file hypbmsec.dtx
and the derived files
   hypbmsec.sty, hypbmsec.pdf, hypbmsec.ins, hypbmsec.drv.

\endpreamble
\let\MetaPrefix\DoubleperCent

\generate{%
  \file{hypbmsec.ins}{\from{hypbmsec.dtx}{install}}%
  \file{hypbmsec.drv}{\from{hypbmsec.dtx}{driver}}%
  \usedir{tex/latex/oberdiek}%
  \file{hypbmsec.sty}{\from{hypbmsec.dtx}{package}}%
  \nopreamble
  \nopostamble
  \usedir{source/latex/oberdiek/catalogue}%
  \file{hypbmsec.xml}{\from{hypbmsec.dtx}{catalogue}}%
}

\catcode32=13\relax% active space
\let =\space%
\Msg{************************************************************************}
\Msg{*}
\Msg{* To finish the installation you have to move the following}
\Msg{* file into a directory searched by TeX:}
\Msg{*}
\Msg{*     hypbmsec.sty}
\Msg{*}
\Msg{* To produce the documentation run the file `hypbmsec.drv'}
\Msg{* through LaTeX.}
\Msg{*}
\Msg{* Happy TeXing!}
\Msg{*}
\Msg{************************************************************************}

\endbatchfile
%</install>
%<*ignore>
\fi
%</ignore>
%<*driver>
\NeedsTeXFormat{LaTeX2e}
\ProvidesFile{hypbmsec.drv}%
  [2016/05/16 v2.5 Bookmarks in sectioning commands (HO)]%
\documentclass{ltxdoc}
\usepackage{holtxdoc}[2011/11/22]
\begin{document}
  \DocInput{hypbmsec.dtx}%
\end{document}
%</driver>
% \fi
%
%
% \CharacterTable
%  {Upper-case    \A\B\C\D\E\F\G\H\I\J\K\L\M\N\O\P\Q\R\S\T\U\V\W\X\Y\Z
%   Lower-case    \a\b\c\d\e\f\g\h\i\j\k\l\m\n\o\p\q\r\s\t\u\v\w\x\y\z
%   Digits        \0\1\2\3\4\5\6\7\8\9
%   Exclamation   \!     Double quote  \"     Hash (number) \#
%   Dollar        \$     Percent       \%     Ampersand     \&
%   Acute accent  \'     Left paren    \(     Right paren   \)
%   Asterisk      \*     Plus          \+     Comma         \,
%   Minus         \-     Point         \.     Solidus       \/
%   Colon         \:     Semicolon     \;     Less than     \<
%   Equals        \=     Greater than  \>     Question mark \?
%   Commercial at \@     Left bracket  \[     Backslash     \\
%   Right bracket \]     Circumflex    \^     Underscore    \_
%   Grave accent  \`     Left brace    \{     Vertical bar  \|
%   Right brace   \}     Tilde         \~}
%
% \GetFileInfo{hypbmsec.drv}
%
% \title{The \xpackage{hypbmsec} package}
% \date{2016/05/16 v2.5}
% \author{Heiko Oberdiek\thanks
% {Please report any issues at https://github.com/ho-tex/oberdiek/issues}\\
% \xemail{heiko.oberdiek at googlemail.com}}
%
% \maketitle
%
% \begin{abstract}
% This package expands the syntax of the sectioning commands. If the
% argument of the sectioning commands isn't usable as outline entry,
% a replacement for the bookmarks can be given.
% \end{abstract}
%
% \tableofcontents
%
% \newcommand{\type}[1]{\textsf{#1}}
%
% ^^A No thread support.
% \newenvironment{article}[1]{}{}
%
% \section{Usage}
%
% \subsection{Bookmarks restrictions}\label{sec:restrictions}
%    Outline entries (bookmarks) are written to a file and have
%    to obey the pdf specification.
%    Therefore they have several restrictions:
%    \begin{itemize}
%    \item Bookmarks have to be encoded in PDFDocEncoding^^A
%          \footnote{\Package{hyperref} doesn't support
%            Unicode.}.
%    \item They should only expand to letters and spaces.
%    \item The result of expansion have to be a valid pdf string.
%    \item Stomach commands like \cmd{\relax}, box commands, math,
%          assignments, or definitions aren't allowed.
%    \item Short entries are recommended, which allow a clear view.
%    \end{itemize}
%
% \subsection{\texorpdfstring{\cmd{\texorpdfstring}}{^^A
%    \textbackslash texorpdfstring}}
%    The generic way in package \Package{hyperref} is the use
%    of \cmd{\texorpdfstring}^^A
%    \footnote{In versions of \Package{hyperref} below 6.54 see
%      \cmd{\ifbookmark}.}:
%    \begin{quote}
%\begin{verbatim}
%\section{Pythagoras:
%  \texorpdfstring{$a^2+b^2=c^2}{%
%    a\texttwosuperior\ + b\texttwosuperior\ =
%    c\texttwosuperior}%
%}
%\end{verbatim}
%    \end{quote}
%
% \subsection{Sectioning commands}
%    The package \Package{hyperref} automatically generates
%    bookmarks from the sectioning commands,
%    unless it is suppressed by an option.
%    Commands that structure the text are here called
%    ``sectioning commands'':
%    \begin{quote}
%    \cmd{\part}, \cmd{\chapter},\\
%    \cmd{\section}, \cmd{\subsection}, \cmd{\subsubsection},\\
%    \cmd{\paragraph}, \cmd{\subparagraph}
%    \end{quote}
%
% \subsection{Places\texorpdfstring{ for sectioning strings}{}}
%    \label{sec:places}
%    The argument(s) of these commands are used on several places:
%    \begin{description}
%    \item[\type{text}]
%      The current text without restrictions.
%    \item[\type{toc}]
%      The headlines and the table of contents with the
%      restrictions of ``moving arguments''.
%    \item[\type{out}]
%      The outlines with many restrictions: The outline
%      have to expand to a valid pdf string with PDFDocEncoding
%      (see \ref{sec:restrictions}).
%    \end{description}
%
% \subsection{\texorpdfstring{Solution with o}{O}ptional arguments}
%    If the user wants to use a footnote within a sectioning command,
%    the \LaTeX{} solution is an optional argument:
%    \begin{quote}
%      |\section[Title]{Title\footnote{Footnote text}}|
%    \end{quote}
%    Now |Title| without the footnote is used in the headlines and
%    the table of contents. Also \Package{hyperref} uses it for the
%    bookmarks.
%
%    This package \Package{\filename} offers two possibilities to
%    specify a separate outline entry:
%    \begin{itemize}
%    \item An additional second optional argument in square brackets.
%    \item An additional optional argument in parentheses (in
%          assoziation with a pdf string that is internally surrounded
%          by parentheses, too).
%    \end{itemize}
%    Because \Package{\filename} stores the original meaning of the
%    sectioning commands and uses them again, there should be no
%    problems with packages that redefine the sectioning commands, if
%    these packages doesn't change the syntax.
%
% \subsection{Syntax}
%    The following examples show the syntax of the sectioning
%    commands. For the places the strings appear the abbreviations
%    are used, that are introduced in \ref{sec:places}.
%
% \subsubsection{\texorpdfstring{Star form}{^^A
%    \textbackslash section*\{\}}}
%    The behaviour of the star form isn't changed. The string
%    appears only in the current text:
%    \begin{article}{Syntax}
%    \begin{quote}
%      |\section*{text}|
%    \end{quote}
%    \end{article}
%
% \subsubsection{\texorpdfstring{Without optional arguments}{^^A
%    \textbackslash section\{\}}}
%    The normal case, the string in the mandatory argument is
%    used for all places:
%    \begin{article}{Syntax}
%    \begin{quote}
%      |\section{text, toc, out}|
%    \end{quote}
%    \end{article}
%
% \subsubsection{\texorpdfstring{One optional argument}{^^A
%    \textbackslash section[]\{\}}}
%    Also the form with one optional parameter in square brackets isn't
%    new; for the bookmarks the optional parameter is used:
%    \begin{article}{Syntax}
%    \begin{quote}
%      |\section[toc, out]{text}|
%    \end{quote}
%    \end{article}
%
% \subsubsection{\texorpdfstring{Two optional arguments}{^^A
%    \textbackslash section[][out]\{\}}}\label{sec:two}
%    The second optional parameter in square brackets is introduced
%    by this package to specify an outline entry:
%    \begin{article}{Syntax}
%    \begin{quote}
%      |\section[toc][out]{text}|
%    \end{quote}
%    \end{article}
%
% \subsubsection{\texorpdfstring{Optional argument in parentheses}{^^A
%    \textbackslash section(out)\{\}}}
%    Often the \type{toc} and the \type{text} string would be the same.
%    With the method of the two optional arguments in square brackets
%    (see \ref{sec:two}) this string must be given twice,
%    if the user only wants to specify a different outline entry.
%    Therefore this package offers another possibility:
%    In association with the internal representation in the pdf file
%    an outline entry can be given in parentheses.
%    So the package can easily distinguish between
%    the two forms of optional arguments and the order does not matter:
%    \begin{article}{Syntax}
%    \begin{quote}
%      |\section(out){toc, text}|\\
%      |\section[toc](out){text}|\\
%      |\section(out)[toc]{text}|
%    \end{quote}
%    \end{article}
%
% \subsection{Without \Package{hyperref}}
%    Package \Package{\filename} uses \Package{hyperref} for support of
%    the bookmarks, but this package is not required.
%    If \Package{hyperref} isn't loaded, or
%    is called with a driver that doesn't support bookmarks,
%    package \Package{\filename} shouldn't be removed,
%    because this would lead to
%    a wrong syntax of the sectioning commands.
%    In any cases package \Package{\filename}
%    supports its syntax and ignores the outline entries,
%    if there are no code for bookmarks.
%    So it is possible to write texts,
%    that are processed with several drivers to get different output
%    formats.
%
% \subsection{Protecting parentheses}
%    If the string itself contains parentheses, they have to be hidden
%    from \TeX's argument parsing mechanism.
%    The argument should be surrounded
%    by curly braces:
%    \begin{quote}
%      |\section({outlines(bookmarks)}){text, toc}|
%    \end{quote}
%    With version 6.54 of \Package{hyperref} the other standard method
%    works, too: The closing parenthesis is protected:
%    \begin{quote}
%      |\section(outlines(bookmarks{)}){text, toc}|
%    \end{quote}
%
% \StopEventually{
% }
%
% \section{Implementation}
%    \begin{macrocode}
%<*package>
%    \end{macrocode}
%    Package identification.
%    \begin{macrocode}
\NeedsTeXFormat{LaTeX2e}
\ProvidesPackage{hypbmsec}%
  [2016/05/16 v2.5 Bookmarks in sectioning commands (HO)]
%    \end{macrocode}
%
%    Because of redifining the sectioning commands, it is dangerous
%    to reload the package several times.
%    \begin{macrocode}
\@ifundefined{hbs@do}{}{%
  \PackageInfo{hypbmsec}{Package 'hypbmsec' is already loaded}%
  \endinput
}
%    \end{macrocode}
%
%    \begin{macro}{\hbs@do}
%    The redefined sectioning commands calls \cmd{\hbs@do}. It does
%    \begin{itemize}
%    \item handle the star case.
%    \item resets the macros that store the entries for the outlines
%          (\cmd{\hbs@bmstring}) and table of contents (\cmd{\hbs@tocstring}).
%    \item store the sectioning command |#1| in \cmd{\hbs@seccmd}
%          for later reuse.
%    \item at last call \cmd{\hbs@checkarg} that scans and interprets the
%          parameters of the redefined sectioning command.
%    \end{itemize}
%    \begin{macrocode}
\def\hbs@do#1{%
  \@ifstar{#1*}{%
    \let\hbs@tocstring\relax
    \let\hbs@bmstring\relax
    \let\hbs@seccmd#1%
    \hbs@checkarg
  }%
}
%    \end{macrocode}
%    \end{macro}
%
%    \begin{macro}{\hbs@checkarg}
%    \cmd{\hbs@checkarg} determines the type of the next argument:
%    \begin{itemize}
%    \item An optional argument in square brackets can be an entry
%          for the table of contents or the bookmarks. It will be
%          read by \cmd{\hbs@getsquare}
%    \item An optional argument in parentheses is an outline entry.
%          This is worked off by \cmd{\hbs@getbookmark}.
%    \item If there are no more optional arguments, \cmd{\hbs@process}
%          reads the mandatory argument and calls the original
%          sectioning commands.
%    \end{itemize}
%    \begin{macrocode}
\def\hbs@checkarg{%
  \@ifnextchar[\hbs@getsquare{%
    \@ifnextchar(\hbs@getbookmark\hbs@process
  }%
}
%    \end{macrocode}
%    \end{macro}
%
%    \begin{macro}{\hbs@getsquare}
%    \cmd{\hbs@getsquare} reads an optional argument in square
%    brackets and determines, if this is an entry for the table
%    of contents or the bookmarks.
%    \begin{macrocode}
\long\def\hbs@getsquare[#1]{%
  \ifx\hbs@tocstring\relax
    \def\hbs@tocstring{#1}%
  \else
    \hbs@bmdef{#1}%
  \fi
  \hbs@checkarg
}
%    \end{macrocode}
%    \end{macro}
%
%    \begin{macro}{\hbs@getbookmark}
%    \cmd{\hbs@getbookmark} reads an outline entry in parentheses.
%    \begin{macrocode}
\def\hbs@getbookmark(#1){%
  \hbs@bmdef{#1}%
  \hbs@checkarg
}
%    \end{macrocode}
%    \end{macro}
%
%    \begin{macro}{\hbs@bmdef}
%    The command \cmd{\hbs@bmdef} save the bookmark entry in
%    parameter |#1| in the macro \cmd{\hbs@bmstring} and catches
%    the case, if the user has given several outline strings.
%    \begin{macrocode}
\def\hbs@bmdef#1{%
  \ifx\hbs@bmstring\relax
    \def\hbs@bmstring{#1}%
  \else
    \PackageError{hypbmsec}{%
      Sectioning command with too many parameters%
    }{%
      You can only give one outline entry.%
    }%
  \fi
}
%    \end{macrocode}
%    \end{macro}
%
%    \begin{macro}{\hbs@process}
%    The parameter |#1| is the mandatory argument of the sectioning
%    commands. \cmd{\hbs@process} calls the original sectioning command
%    stored in \cmd{\hbs@seccmd} with arguments that depend of which
%    optional argument are used previously.
%    \begin{macrocode}
\long\def\hbs@process#1{%
  \ifx\hbs@tocstring\relax
    \ifx\hbs@bmstring\relax
      \hbs@seccmd{#1}%
    \else
      \begingroup
        \def\x##1{\endgroup
          \hbs@seccmd{\texorpdfstring{#1}{##1}}%
        }%
      \expandafter\x\expandafter{\hbs@bmstring}%
    \fi
  \else
    \ifx\hbs@bmstring\relax
      \expandafter\hbs@seccmd\expandafter[%
        \expandafter{\hbs@tocstring}%
      ]{#1}%
    \else
      \expandafter\expandafter\expandafter
      \hbs@seccmd\expandafter\expandafter\expandafter[%
        \expandafter\expandafter\expandafter
        \texorpdfstring
        \expandafter\expandafter\expandafter{%
          \expandafter\hbs@tocstring\expandafter
        }\expandafter{%
          \hbs@bmstring
        }%
      ]{#1}%
    \fi
  \fi
}
%    \end{macrocode}
%    \end{macro}
%
%    We have to check, whether package \Package{hyperref} is loaded
%    and have to provide a definition for \cmd{\texorpdfstring}.
%    Because \Package{hyperref} can be loaded after this package,
%    we do the work later (\cmd{\AtBeginDocument}).
%
%    This code only checks versions of \Package{hyperref} that
%    define \cmd{\ifbookmark} (v6.4x until v6.53) or
%    \cmd{\texorpdfstring} (v6.54 and above). Older versions aren't
%    supported.
%    \begin{macrocode}
\AtBeginDocument{%
  \@ifundefined{texorpdfstring}{%
    \@ifundefined{ifbookmark}{%
      \let\texorpdfstring\@firstoftwo
      \@ifpackageloaded{hyperref}{%
        \PackageInfo{hypbmsec}{%
          \ifx\hy@driver\@empty
            Default driver %
          \else
            '\hy@driver' %
          \fi
          of hyperref not supported,\MessageBreak
          bookmark parameters will be ignored%
        }%
      }{%
        \PackageInfo{hypbmsec}{%
          Package hyperref not loaded,\MessageBreak
          bookmark parameters will be ignored%
        }%
      }%
    }%
    {%
      \newcommand\texorpdfstring[2]{\ifbookmark{#2}{#1}}%
      \PackageWarningNoLine{hypbmsec}{%
        Old hyperref version found,\MessageBreak
        update of hyperref recommended%
      }%
    }%
  }{}%
%    \end{macrocode}
%
%    Other packages are allowed to redefine the sectioning commands,
%    if they does not change the syntax. Therefore the redefinitons
%    of this package should be done after the other packages.
%    \begin{macrocode}
  \let\hbs@part         \part
  \let\hbs@section      \section
  \let\hbs@subsection   \subsection
  \let\hbs@subsubsection\subsubsection
  \let\hbs@paragraph    \paragraph
  \let\hbs@subparagraph \subparagraph
  \renewcommand\part         {\hbs@do\hbs@part}%
  \renewcommand\section      {\hbs@do\hbs@section}%
  \renewcommand\subsection   {\hbs@do\hbs@subsection}%
  \renewcommand\subsubsection{\hbs@do\hbs@subsubsection}%
  \renewcommand\paragraph    {\hbs@do\hbs@paragraph}%
  \renewcommand\subparagraph {\hbs@do\hbs@subparagraph}%
  \begingroup\expandafter\expandafter\expandafter\endgroup
  \expandafter\ifx\csname chapter\endcsname\relax\else
    \let\hbs@chapter    \chapter
    \renewcommand\chapter    {\hbs@do\hbs@chapter}%
  \fi
}
%    \end{macrocode}
%
%    \begin{macrocode}
%</package>
%    \end{macrocode}
%
% \section{Installation}
%
% \subsection{Download}
%
% \paragraph{Package.} This package is available on
% CTAN\footnote{\url{http://ctan.org/pkg/hypbmsec}}:
% \begin{description}
% \item[\CTAN{macros/latex/contrib/oberdiek/hypbmsec.dtx}] The source file.
% \item[\CTAN{macros/latex/contrib/oberdiek/hypbmsec.pdf}] Documentation.
% \end{description}
%
%
% \paragraph{Bundle.} All the packages of the bundle `oberdiek'
% are also available in a TDS compliant ZIP archive. There
% the packages are already unpacked and the documentation files
% are generated. The files and directories obey the TDS standard.
% \begin{description}
% \item[\CTAN{install/macros/latex/contrib/oberdiek.tds.zip}]
% \end{description}
% \emph{TDS} refers to the standard ``A Directory Structure
% for \TeX\ Files'' (\CTAN{tds/tds.pdf}). Directories
% with \xfile{texmf} in their name are usually organized this way.
%
% \subsection{Bundle installation}
%
% \paragraph{Unpacking.} Unpack the \xfile{oberdiek.tds.zip} in the
% TDS tree (also known as \xfile{texmf} tree) of your choice.
% Example (linux):
% \begin{quote}
%   |unzip oberdiek.tds.zip -d ~/texmf|
% \end{quote}
%
% \paragraph{Script installation.}
% Check the directory \xfile{TDS:scripts/oberdiek/} for
% scripts that need further installation steps.
% Package \xpackage{attachfile2} comes with the Perl script
% \xfile{pdfatfi.pl} that should be installed in such a way
% that it can be called as \texttt{pdfatfi}.
% Example (linux):
% \begin{quote}
%   |chmod +x scripts/oberdiek/pdfatfi.pl|\\
%   |cp scripts/oberdiek/pdfatfi.pl /usr/local/bin/|
% \end{quote}
%
% \subsection{Package installation}
%
% \paragraph{Unpacking.} The \xfile{.dtx} file is a self-extracting
% \docstrip\ archive. The files are extracted by running the
% \xfile{.dtx} through \plainTeX:
% \begin{quote}
%   \verb|tex hypbmsec.dtx|
% \end{quote}
%
% \paragraph{TDS.} Now the different files must be moved into
% the different directories in your installation TDS tree
% (also known as \xfile{texmf} tree):
% \begin{quote}
% \def\t{^^A
% \begin{tabular}{@{}>{\ttfamily}l@{ $\rightarrow$ }>{\ttfamily}l@{}}
%   hypbmsec.sty & tex/latex/oberdiek/hypbmsec.sty\\
%   hypbmsec.pdf & doc/latex/oberdiek/hypbmsec.pdf\\
%   hypbmsec.dtx & source/latex/oberdiek/hypbmsec.dtx\\
% \end{tabular}^^A
% }^^A
% \sbox0{\t}^^A
% \ifdim\wd0>\linewidth
%   \begingroup
%     \advance\linewidth by\leftmargin
%     \advance\linewidth by\rightmargin
%   \edef\x{\endgroup
%     \def\noexpand\lw{\the\linewidth}^^A
%   }\x
%   \def\lwbox{^^A
%     \leavevmode
%     \hbox to \linewidth{^^A
%       \kern-\leftmargin\relax
%       \hss
%       \usebox0
%       \hss
%       \kern-\rightmargin\relax
%     }^^A
%   }^^A
%   \ifdim\wd0>\lw
%     \sbox0{\small\t}^^A
%     \ifdim\wd0>\linewidth
%       \ifdim\wd0>\lw
%         \sbox0{\footnotesize\t}^^A
%         \ifdim\wd0>\linewidth
%           \ifdim\wd0>\lw
%             \sbox0{\scriptsize\t}^^A
%             \ifdim\wd0>\linewidth
%               \ifdim\wd0>\lw
%                 \sbox0{\tiny\t}^^A
%                 \ifdim\wd0>\linewidth
%                   \lwbox
%                 \else
%                   \usebox0
%                 \fi
%               \else
%                 \lwbox
%               \fi
%             \else
%               \usebox0
%             \fi
%           \else
%             \lwbox
%           \fi
%         \else
%           \usebox0
%         \fi
%       \else
%         \lwbox
%       \fi
%     \else
%       \usebox0
%     \fi
%   \else
%     \lwbox
%   \fi
% \else
%   \usebox0
% \fi
% \end{quote}
% If you have a \xfile{docstrip.cfg} that configures and enables \docstrip's
% TDS installing feature, then some files can already be in the right
% place, see the documentation of \docstrip.
%
% \subsection{Refresh file name databases}
%
% If your \TeX~distribution
% (\teTeX, \mikTeX, \dots) relies on file name databases, you must refresh
% these. For example, \teTeX\ users run \verb|texhash| or
% \verb|mktexlsr|.
%
% \subsection{Some details for the interested}
%
% \paragraph{Attached source.}
%
% The PDF documentation on CTAN also includes the
% \xfile{.dtx} source file. It can be extracted by
% AcrobatReader 6 or higher. Another option is \textsf{pdftk},
% e.g. unpack the file into the current directory:
% \begin{quote}
%   \verb|pdftk hypbmsec.pdf unpack_files output .|
% \end{quote}
%
% \paragraph{Unpacking with \LaTeX.}
% The \xfile{.dtx} chooses its action depending on the format:
% \begin{description}
% \item[\plainTeX:] Run \docstrip\ and extract the files.
% \item[\LaTeX:] Generate the documentation.
% \end{description}
% If you insist on using \LaTeX\ for \docstrip\ (really,
% \docstrip\ does not need \LaTeX), then inform the autodetect routine
% about your intention:
% \begin{quote}
%   \verb|latex \let\install=y\input{hypbmsec.dtx}|
% \end{quote}
% Do not forget to quote the argument according to the demands
% of your shell.
%
% \paragraph{Generating the documentation.}
% You can use both the \xfile{.dtx} or the \xfile{.drv} to generate
% the documentation. The process can be configured by the
% configuration file \xfile{ltxdoc.cfg}. For instance, put this
% line into this file, if you want to have A4 as paper format:
% \begin{quote}
%   \verb|\PassOptionsToClass{a4paper}{article}|
% \end{quote}
% An example follows how to generate the
% documentation with pdf\LaTeX:
% \begin{quote}
%\begin{verbatim}
%pdflatex hypbmsec.dtx
%makeindex -s gind.ist hypbmsec.idx
%pdflatex hypbmsec.dtx
%makeindex -s gind.ist hypbmsec.idx
%pdflatex hypbmsec.dtx
%\end{verbatim}
% \end{quote}
%
% \section{Catalogue}
%
% The following XML file can be used as source for the
% \href{http://mirror.ctan.org/help/Catalogue/catalogue.html}{\TeX\ Catalogue}.
% The elements \texttt{caption} and \texttt{description} are imported
% from the original XML file from the Catalogue.
% The name of the XML file in the Catalogue is \xfile{hypbmsec.xml}.
%    \begin{macrocode}
%<*catalogue>
<?xml version='1.0' encoding='us-ascii'?>
<!DOCTYPE entry SYSTEM 'catalogue.dtd'>
<entry datestamp='$Date$' modifier='$Author$' id='hypbmsec'>
  <name>hypbmsec</name>
  <caption>Hypertext bookmarks in sectioning commands.</caption>
  <authorref id='auth:oberdiek'/>
  <copyright owner='Heiko Oberdiek' year='1998-2000,2006,2007'/>
  <license type='lppl1.3'/>
  <version number='2.5'/>
  <description>
    Bookmark entries can be given as another argument to the LaTeX
    sectioning commands. The <xref refid='hyperref'>hyperref</xref>
    package is required to get the bookmarks, but the syntax
    works without it.
    <p/>
    This package is part of the <xref refid='oberdiek'>oberdiek</xref>
    bundle.
  </description>
  <documentation details='Package documentation'
      href='ctan:/macros/latex/contrib/oberdiek/hypbmsec.pdf'/>
  <ctan file='true' path='/macros/latex/contrib/oberdiek/hypbmsec.dtx'/>
  <miktex location='oberdiek'/>
  <texlive location='oberdiek'/>
  <install path='/macros/latex/contrib/oberdiek/oberdiek.tds.zip'/>
</entry>
%</catalogue>
%    \end{macrocode}
%
% \begin{History}
%   \begin{Version}{1998/11/20 v1.0}
%   \item
%     First version.
%   \item
%     It merges package \xpackage{hysecopt} and
%   \item
%     package \xpackage{hypbmpar}.
%   \item
%     Published for the DANTE'99 meeting^^A
%     \URL{}{http://dante99.cs.uni-dortmund.de/handouts/oberdiek/hypbmsec.sty}.
%   \end{Version}
%   \begin{Version}{1999/04/12 v2.0}
%   \item
%     Adaptation to \Package{hyperref} version 6.54.
%   \item
%     Documentation in dtx format.
%   \item
%     Copyright: LPPL (\CTAN{macros/latex/base/lppl.txt})
%   \item
%     First CTAN release.
%   \end{Version}
%   \begin{Version}{2000/03/22 v2.1}
%   \item
%     Bug fix in redefinition of \cmd{\chapter}.
%   \item
%     Copyright: LPPL 1.2
%   \end{Version}
%   \begin{Version}{2006/02/20 v2.2}
%   \item
%     Code is not changed.
%   \item
%     New DTX framework.
%   \item
%     LPPL 1.3
%   \end{Version}
%   \begin{Version}{2007/03/05 v2.3}
%   \item
%     Bug fix: Expand \cs{hbs@tocstring} and \cs{hbs@bmstring} before
%     calling \cs{hbs@seccmd}.
%   \end{Version}
%   \begin{Version}{2007/04/11 v2.4}
%   \item
%     Line ends sanitized.
%   \end{Version}
%   \begin{Version}{2016/05/16 v2.5}
%   \item
%     Documentation updates.
%   \end{Version}
% \end{History}
%
% \PrintIndex
%
% \Finale
\endinput
|
% \end{quote}
% Do not forget to quote the argument according to the demands
% of your shell.
%
% \paragraph{Generating the documentation.}
% You can use both the \xfile{.dtx} or the \xfile{.drv} to generate
% the documentation. The process can be configured by the
% configuration file \xfile{ltxdoc.cfg}. For instance, put this
% line into this file, if you want to have A4 as paper format:
% \begin{quote}
%   \verb|\PassOptionsToClass{a4paper}{article}|
% \end{quote}
% An example follows how to generate the
% documentation with pdf\LaTeX:
% \begin{quote}
%\begin{verbatim}
%pdflatex hypbmsec.dtx
%makeindex -s gind.ist hypbmsec.idx
%pdflatex hypbmsec.dtx
%makeindex -s gind.ist hypbmsec.idx
%pdflatex hypbmsec.dtx
%\end{verbatim}
% \end{quote}
%
% \section{Catalogue}
%
% The following XML file can be used as source for the
% \href{http://mirror.ctan.org/help/Catalogue/catalogue.html}{\TeX\ Catalogue}.
% The elements \texttt{caption} and \texttt{description} are imported
% from the original XML file from the Catalogue.
% The name of the XML file in the Catalogue is \xfile{hypbmsec.xml}.
%    \begin{macrocode}
%<*catalogue>
<?xml version='1.0' encoding='us-ascii'?>
<!DOCTYPE entry SYSTEM 'catalogue.dtd'>
<entry datestamp='$Date$' modifier='$Author$' id='hypbmsec'>
  <name>hypbmsec</name>
  <caption>Hypertext bookmarks in sectioning commands.</caption>
  <authorref id='auth:oberdiek'/>
  <copyright owner='Heiko Oberdiek' year='1998-2000,2006,2007'/>
  <license type='lppl1.3'/>
  <version number='2.5'/>
  <description>
    Bookmark entries can be given as another argument to the LaTeX
    sectioning commands. The <xref refid='hyperref'>hyperref</xref>
    package is required to get the bookmarks, but the syntax
    works without it.
    <p/>
    This package is part of the <xref refid='oberdiek'>oberdiek</xref>
    bundle.
  </description>
  <documentation details='Package documentation'
      href='ctan:/macros/latex/contrib/oberdiek/hypbmsec.pdf'/>
  <ctan file='true' path='/macros/latex/contrib/oberdiek/hypbmsec.dtx'/>
  <miktex location='oberdiek'/>
  <texlive location='oberdiek'/>
  <install path='/macros/latex/contrib/oberdiek/oberdiek.tds.zip'/>
</entry>
%</catalogue>
%    \end{macrocode}
%
% \begin{History}
%   \begin{Version}{1998/11/20 v1.0}
%   \item
%     First version.
%   \item
%     It merges package \xpackage{hysecopt} and
%   \item
%     package \xpackage{hypbmpar}.
%   \item
%     Published for the DANTE'99 meeting^^A
%     \URL{}{http://dante99.cs.uni-dortmund.de/handouts/oberdiek/hypbmsec.sty}.
%   \end{Version}
%   \begin{Version}{1999/04/12 v2.0}
%   \item
%     Adaptation to \Package{hyperref} version 6.54.
%   \item
%     Documentation in dtx format.
%   \item
%     Copyright: LPPL (\CTAN{macros/latex/base/lppl.txt})
%   \item
%     First CTAN release.
%   \end{Version}
%   \begin{Version}{2000/03/22 v2.1}
%   \item
%     Bug fix in redefinition of \cmd{\chapter}.
%   \item
%     Copyright: LPPL 1.2
%   \end{Version}
%   \begin{Version}{2006/02/20 v2.2}
%   \item
%     Code is not changed.
%   \item
%     New DTX framework.
%   \item
%     LPPL 1.3
%   \end{Version}
%   \begin{Version}{2007/03/05 v2.3}
%   \item
%     Bug fix: Expand \cs{hbs@tocstring} and \cs{hbs@bmstring} before
%     calling \cs{hbs@seccmd}.
%   \end{Version}
%   \begin{Version}{2007/04/11 v2.4}
%   \item
%     Line ends sanitized.
%   \end{Version}
%   \begin{Version}{2016/05/16 v2.5}
%   \item
%     Documentation updates.
%   \end{Version}
% \end{History}
%
% \PrintIndex
%
% \Finale
\endinput
|
% \end{quote}
% Do not forget to quote the argument according to the demands
% of your shell.
%
% \paragraph{Generating the documentation.}
% You can use both the \xfile{.dtx} or the \xfile{.drv} to generate
% the documentation. The process can be configured by the
% configuration file \xfile{ltxdoc.cfg}. For instance, put this
% line into this file, if you want to have A4 as paper format:
% \begin{quote}
%   \verb|\PassOptionsToClass{a4paper}{article}|
% \end{quote}
% An example follows how to generate the
% documentation with pdf\LaTeX:
% \begin{quote}
%\begin{verbatim}
%pdflatex hypbmsec.dtx
%makeindex -s gind.ist hypbmsec.idx
%pdflatex hypbmsec.dtx
%makeindex -s gind.ist hypbmsec.idx
%pdflatex hypbmsec.dtx
%\end{verbatim}
% \end{quote}
%
% \section{Catalogue}
%
% The following XML file can be used as source for the
% \href{http://mirror.ctan.org/help/Catalogue/catalogue.html}{\TeX\ Catalogue}.
% The elements \texttt{caption} and \texttt{description} are imported
% from the original XML file from the Catalogue.
% The name of the XML file in the Catalogue is \xfile{hypbmsec.xml}.
%    \begin{macrocode}
%<*catalogue>
<?xml version='1.0' encoding='us-ascii'?>
<!DOCTYPE entry SYSTEM 'catalogue.dtd'>
<entry datestamp='$Date$' modifier='$Author$' id='hypbmsec'>
  <name>hypbmsec</name>
  <caption>Hypertext bookmarks in sectioning commands.</caption>
  <authorref id='auth:oberdiek'/>
  <copyright owner='Heiko Oberdiek' year='1998-2000,2006,2007'/>
  <license type='lppl1.3'/>
  <version number='2.5'/>
  <description>
    Bookmark entries can be given as another argument to the LaTeX
    sectioning commands. The <xref refid='hyperref'>hyperref</xref>
    package is required to get the bookmarks, but the syntax
    works without it.
    <p/>
    This package is part of the <xref refid='oberdiek'>oberdiek</xref>
    bundle.
  </description>
  <documentation details='Package documentation'
      href='ctan:/macros/latex/contrib/oberdiek/hypbmsec.pdf'/>
  <ctan file='true' path='/macros/latex/contrib/oberdiek/hypbmsec.dtx'/>
  <miktex location='oberdiek'/>
  <texlive location='oberdiek'/>
  <install path='/macros/latex/contrib/oberdiek/oberdiek.tds.zip'/>
</entry>
%</catalogue>
%    \end{macrocode}
%
% \begin{History}
%   \begin{Version}{1998/11/20 v1.0}
%   \item
%     First version.
%   \item
%     It merges package \xpackage{hysecopt} and
%   \item
%     package \xpackage{hypbmpar}.
%   \item
%     Published for the DANTE'99 meeting^^A
%     \URL{}{http://dante99.cs.uni-dortmund.de/handouts/oberdiek/hypbmsec.sty}.
%   \end{Version}
%   \begin{Version}{1999/04/12 v2.0}
%   \item
%     Adaptation to \Package{hyperref} version 6.54.
%   \item
%     Documentation in dtx format.
%   \item
%     Copyright: LPPL (\CTAN{macros/latex/base/lppl.txt})
%   \item
%     First CTAN release.
%   \end{Version}
%   \begin{Version}{2000/03/22 v2.1}
%   \item
%     Bug fix in redefinition of \cmd{\chapter}.
%   \item
%     Copyright: LPPL 1.2
%   \end{Version}
%   \begin{Version}{2006/02/20 v2.2}
%   \item
%     Code is not changed.
%   \item
%     New DTX framework.
%   \item
%     LPPL 1.3
%   \end{Version}
%   \begin{Version}{2007/03/05 v2.3}
%   \item
%     Bug fix: Expand \cs{hbs@tocstring} and \cs{hbs@bmstring} before
%     calling \cs{hbs@seccmd}.
%   \end{Version}
%   \begin{Version}{2007/04/11 v2.4}
%   \item
%     Line ends sanitized.
%   \end{Version}
%   \begin{Version}{2016/05/16 v2.5}
%   \item
%     Documentation updates.
%   \end{Version}
% \end{History}
%
% \PrintIndex
%
% \Finale
\endinput

%        (quote the arguments according to the demands of your shell)
%
% Documentation:
%    (a) If hypbmsec.drv is present:
%           latex hypbmsec.drv
%    (b) Without hypbmsec.drv:
%           latex hypbmsec.dtx; ...
%    The class ltxdoc loads the configuration file ltxdoc.cfg
%    if available. Here you can specify further options, e.g.
%    use A4 as paper format:
%       \PassOptionsToClass{a4paper}{article}
%
%    Programm calls to get the documentation (example):
%       pdflatex hypbmsec.dtx
%       makeindex -s gind.ist hypbmsec.idx
%       pdflatex hypbmsec.dtx
%       makeindex -s gind.ist hypbmsec.idx
%       pdflatex hypbmsec.dtx
%
% Installation:
%    TDS:tex/latex/oberdiek/hypbmsec.sty
%    TDS:doc/latex/oberdiek/hypbmsec.pdf
%    TDS:source/latex/oberdiek/hypbmsec.dtx
%
%<*ignore>
\begingroup
  \catcode123=1 %
  \catcode125=2 %
  \def\x{LaTeX2e}%
\expandafter\endgroup
\ifcase 0\ifx\install y1\fi\expandafter
         \ifx\csname processbatchFile\endcsname\relax\else1\fi
         \ifx\fmtname\x\else 1\fi\relax
\else\csname fi\endcsname
%</ignore>
%<*install>
\input docstrip.tex
\Msg{************************************************************************}
\Msg{* Installation}
\Msg{* Package: hypbmsec 2016/05/16 v2.5 Bookmarks in sectioning commands (HO)}
\Msg{************************************************************************}

\keepsilent
\askforoverwritefalse

\let\MetaPrefix\relax
\preamble

This is a generated file.

Project: hypbmsec
Version: 2016/05/16 v2.5

Copyright (C) 1998-2000, 2006, 2007 by
   Heiko Oberdiek <heiko.oberdiek at googlemail.com>

This work may be distributed and/or modified under the
conditions of the LaTeX Project Public License, either
version 1.3c of this license or (at your option) any later
version. This version of this license is in
   http://www.latex-project.org/lppl/lppl-1-3c.txt
and the latest version of this license is in
   http://www.latex-project.org/lppl.txt
and version 1.3 or later is part of all distributions of
LaTeX version 2005/12/01 or later.

This work has the LPPL maintenance status "maintained".

This Current Maintainer of this work is Heiko Oberdiek.

This work consists of the main source file hypbmsec.dtx
and the derived files
   hypbmsec.sty, hypbmsec.pdf, hypbmsec.ins, hypbmsec.drv.

\endpreamble
\let\MetaPrefix\DoubleperCent

\generate{%
  \file{hypbmsec.ins}{\from{hypbmsec.dtx}{install}}%
  \file{hypbmsec.drv}{\from{hypbmsec.dtx}{driver}}%
  \usedir{tex/latex/oberdiek}%
  \file{hypbmsec.sty}{\from{hypbmsec.dtx}{package}}%
  \nopreamble
  \nopostamble
  \usedir{source/latex/oberdiek/catalogue}%
  \file{hypbmsec.xml}{\from{hypbmsec.dtx}{catalogue}}%
}

\catcode32=13\relax% active space
\let =\space%
\Msg{************************************************************************}
\Msg{*}
\Msg{* To finish the installation you have to move the following}
\Msg{* file into a directory searched by TeX:}
\Msg{*}
\Msg{*     hypbmsec.sty}
\Msg{*}
\Msg{* To produce the documentation run the file `hypbmsec.drv'}
\Msg{* through LaTeX.}
\Msg{*}
\Msg{* Happy TeXing!}
\Msg{*}
\Msg{************************************************************************}

\endbatchfile
%</install>
%<*ignore>
\fi
%</ignore>
%<*driver>
\NeedsTeXFormat{LaTeX2e}
\ProvidesFile{hypbmsec.drv}%
  [2016/05/16 v2.5 Bookmarks in sectioning commands (HO)]%
\documentclass{ltxdoc}
\usepackage{holtxdoc}[2011/11/22]
\begin{document}
  \DocInput{hypbmsec.dtx}%
\end{document}
%</driver>
% \fi
%
%
% \CharacterTable
%  {Upper-case    \A\B\C\D\E\F\G\H\I\J\K\L\M\N\O\P\Q\R\S\T\U\V\W\X\Y\Z
%   Lower-case    \a\b\c\d\e\f\g\h\i\j\k\l\m\n\o\p\q\r\s\t\u\v\w\x\y\z
%   Digits        \0\1\2\3\4\5\6\7\8\9
%   Exclamation   \!     Double quote  \"     Hash (number) \#
%   Dollar        \$     Percent       \%     Ampersand     \&
%   Acute accent  \'     Left paren    \(     Right paren   \)
%   Asterisk      \*     Plus          \+     Comma         \,
%   Minus         \-     Point         \.     Solidus       \/
%   Colon         \:     Semicolon     \;     Less than     \<
%   Equals        \=     Greater than  \>     Question mark \?
%   Commercial at \@     Left bracket  \[     Backslash     \\
%   Right bracket \]     Circumflex    \^     Underscore    \_
%   Grave accent  \`     Left brace    \{     Vertical bar  \|
%   Right brace   \}     Tilde         \~}
%
% \GetFileInfo{hypbmsec.drv}
%
% \title{The \xpackage{hypbmsec} package}
% \date{2016/05/16 v2.5}
% \author{Heiko Oberdiek\thanks
% {Please report any issues at https://github.com/ho-tex/oberdiek/issues}\\
% \xemail{heiko.oberdiek at googlemail.com}}
%
% \maketitle
%
% \begin{abstract}
% This package expands the syntax of the sectioning commands. If the
% argument of the sectioning commands isn't usable as outline entry,
% a replacement for the bookmarks can be given.
% \end{abstract}
%
% \tableofcontents
%
% \newcommand{\type}[1]{\textsf{#1}}
%
% ^^A No thread support.
% \newenvironment{article}[1]{}{}
%
% \section{Usage}
%
% \subsection{Bookmarks restrictions}\label{sec:restrictions}
%    Outline entries (bookmarks) are written to a file and have
%    to obey the pdf specification.
%    Therefore they have several restrictions:
%    \begin{itemize}
%    \item Bookmarks have to be encoded in PDFDocEncoding^^A
%          \footnote{\Package{hyperref} doesn't support
%            Unicode.}.
%    \item They should only expand to letters and spaces.
%    \item The result of expansion have to be a valid pdf string.
%    \item Stomach commands like \cmd{\relax}, box commands, math,
%          assignments, or definitions aren't allowed.
%    \item Short entries are recommended, which allow a clear view.
%    \end{itemize}
%
% \subsection{\texorpdfstring{\cmd{\texorpdfstring}}{^^A
%    \textbackslash texorpdfstring}}
%    The generic way in package \Package{hyperref} is the use
%    of \cmd{\texorpdfstring}^^A
%    \footnote{In versions of \Package{hyperref} below 6.54 see
%      \cmd{\ifbookmark}.}:
%    \begin{quote}
%\begin{verbatim}
%\section{Pythagoras:
%  \texorpdfstring{$a^2+b^2=c^2}{%
%    a\texttwosuperior\ + b\texttwosuperior\ =
%    c\texttwosuperior}%
%}
%\end{verbatim}
%    \end{quote}
%
% \subsection{Sectioning commands}
%    The package \Package{hyperref} automatically generates
%    bookmarks from the sectioning commands,
%    unless it is suppressed by an option.
%    Commands that structure the text are here called
%    ``sectioning commands'':
%    \begin{quote}
%    \cmd{\part}, \cmd{\chapter},\\
%    \cmd{\section}, \cmd{\subsection}, \cmd{\subsubsection},\\
%    \cmd{\paragraph}, \cmd{\subparagraph}
%    \end{quote}
%
% \subsection{Places\texorpdfstring{ for sectioning strings}{}}
%    \label{sec:places}
%    The argument(s) of these commands are used on several places:
%    \begin{description}
%    \item[\type{text}]
%      The current text without restrictions.
%    \item[\type{toc}]
%      The headlines and the table of contents with the
%      restrictions of ``moving arguments''.
%    \item[\type{out}]
%      The outlines with many restrictions: The outline
%      have to expand to a valid pdf string with PDFDocEncoding
%      (see \ref{sec:restrictions}).
%    \end{description}
%
% \subsection{\texorpdfstring{Solution with o}{O}ptional arguments}
%    If the user wants to use a footnote within a sectioning command,
%    the \LaTeX{} solution is an optional argument:
%    \begin{quote}
%      |\section[Title]{Title\footnote{Footnote text}}|
%    \end{quote}
%    Now |Title| without the footnote is used in the headlines and
%    the table of contents. Also \Package{hyperref} uses it for the
%    bookmarks.
%
%    This package \Package{\filename} offers two possibilities to
%    specify a separate outline entry:
%    \begin{itemize}
%    \item An additional second optional argument in square brackets.
%    \item An additional optional argument in parentheses (in
%          assoziation with a pdf string that is internally surrounded
%          by parentheses, too).
%    \end{itemize}
%    Because \Package{\filename} stores the original meaning of the
%    sectioning commands and uses them again, there should be no
%    problems with packages that redefine the sectioning commands, if
%    these packages doesn't change the syntax.
%
% \subsection{Syntax}
%    The following examples show the syntax of the sectioning
%    commands. For the places the strings appear the abbreviations
%    are used, that are introduced in \ref{sec:places}.
%
% \subsubsection{\texorpdfstring{Star form}{^^A
%    \textbackslash section*\{\}}}
%    The behaviour of the star form isn't changed. The string
%    appears only in the current text:
%    \begin{article}{Syntax}
%    \begin{quote}
%      |\section*{text}|
%    \end{quote}
%    \end{article}
%
% \subsubsection{\texorpdfstring{Without optional arguments}{^^A
%    \textbackslash section\{\}}}
%    The normal case, the string in the mandatory argument is
%    used for all places:
%    \begin{article}{Syntax}
%    \begin{quote}
%      |\section{text, toc, out}|
%    \end{quote}
%    \end{article}
%
% \subsubsection{\texorpdfstring{One optional argument}{^^A
%    \textbackslash section[]\{\}}}
%    Also the form with one optional parameter in square brackets isn't
%    new; for the bookmarks the optional parameter is used:
%    \begin{article}{Syntax}
%    \begin{quote}
%      |\section[toc, out]{text}|
%    \end{quote}
%    \end{article}
%
% \subsubsection{\texorpdfstring{Two optional arguments}{^^A
%    \textbackslash section[][out]\{\}}}\label{sec:two}
%    The second optional parameter in square brackets is introduced
%    by this package to specify an outline entry:
%    \begin{article}{Syntax}
%    \begin{quote}
%      |\section[toc][out]{text}|
%    \end{quote}
%    \end{article}
%
% \subsubsection{\texorpdfstring{Optional argument in parentheses}{^^A
%    \textbackslash section(out)\{\}}}
%    Often the \type{toc} and the \type{text} string would be the same.
%    With the method of the two optional arguments in square brackets
%    (see \ref{sec:two}) this string must be given twice,
%    if the user only wants to specify a different outline entry.
%    Therefore this package offers another possibility:
%    In association with the internal representation in the pdf file
%    an outline entry can be given in parentheses.
%    So the package can easily distinguish between
%    the two forms of optional arguments and the order does not matter:
%    \begin{article}{Syntax}
%    \begin{quote}
%      |\section(out){toc, text}|\\
%      |\section[toc](out){text}|\\
%      |\section(out)[toc]{text}|
%    \end{quote}
%    \end{article}
%
% \subsection{Without \Package{hyperref}}
%    Package \Package{\filename} uses \Package{hyperref} for support of
%    the bookmarks, but this package is not required.
%    If \Package{hyperref} isn't loaded, or
%    is called with a driver that doesn't support bookmarks,
%    package \Package{\filename} shouldn't be removed,
%    because this would lead to
%    a wrong syntax of the sectioning commands.
%    In any cases package \Package{\filename}
%    supports its syntax and ignores the outline entries,
%    if there are no code for bookmarks.
%    So it is possible to write texts,
%    that are processed with several drivers to get different output
%    formats.
%
% \subsection{Protecting parentheses}
%    If the string itself contains parentheses, they have to be hidden
%    from \TeX's argument parsing mechanism.
%    The argument should be surrounded
%    by curly braces:
%    \begin{quote}
%      |\section({outlines(bookmarks)}){text, toc}|
%    \end{quote}
%    With version 6.54 of \Package{hyperref} the other standard method
%    works, too: The closing parenthesis is protected:
%    \begin{quote}
%      |\section(outlines(bookmarks{)}){text, toc}|
%    \end{quote}
%
% \StopEventually{
% }
%
% \section{Implementation}
%    \begin{macrocode}
%<*package>
%    \end{macrocode}
%    Package identification.
%    \begin{macrocode}
\NeedsTeXFormat{LaTeX2e}
\ProvidesPackage{hypbmsec}%
  [2016/05/16 v2.5 Bookmarks in sectioning commands (HO)]
%    \end{macrocode}
%
%    Because of redifining the sectioning commands, it is dangerous
%    to reload the package several times.
%    \begin{macrocode}
\@ifundefined{hbs@do}{}{%
  \PackageInfo{hypbmsec}{Package 'hypbmsec' is already loaded}%
  \endinput
}
%    \end{macrocode}
%
%    \begin{macro}{\hbs@do}
%    The redefined sectioning commands calls \cmd{\hbs@do}. It does
%    \begin{itemize}
%    \item handle the star case.
%    \item resets the macros that store the entries for the outlines
%          (\cmd{\hbs@bmstring}) and table of contents (\cmd{\hbs@tocstring}).
%    \item store the sectioning command |#1| in \cmd{\hbs@seccmd}
%          for later reuse.
%    \item at last call \cmd{\hbs@checkarg} that scans and interprets the
%          parameters of the redefined sectioning command.
%    \end{itemize}
%    \begin{macrocode}
\def\hbs@do#1{%
  \@ifstar{#1*}{%
    \let\hbs@tocstring\relax
    \let\hbs@bmstring\relax
    \let\hbs@seccmd#1%
    \hbs@checkarg
  }%
}
%    \end{macrocode}
%    \end{macro}
%
%    \begin{macro}{\hbs@checkarg}
%    \cmd{\hbs@checkarg} determines the type of the next argument:
%    \begin{itemize}
%    \item An optional argument in square brackets can be an entry
%          for the table of contents or the bookmarks. It will be
%          read by \cmd{\hbs@getsquare}
%    \item An optional argument in parentheses is an outline entry.
%          This is worked off by \cmd{\hbs@getbookmark}.
%    \item If there are no more optional arguments, \cmd{\hbs@process}
%          reads the mandatory argument and calls the original
%          sectioning commands.
%    \end{itemize}
%    \begin{macrocode}
\def\hbs@checkarg{%
  \@ifnextchar[\hbs@getsquare{%
    \@ifnextchar(\hbs@getbookmark\hbs@process
  }%
}
%    \end{macrocode}
%    \end{macro}
%
%    \begin{macro}{\hbs@getsquare}
%    \cmd{\hbs@getsquare} reads an optional argument in square
%    brackets and determines, if this is an entry for the table
%    of contents or the bookmarks.
%    \begin{macrocode}
\long\def\hbs@getsquare[#1]{%
  \ifx\hbs@tocstring\relax
    \def\hbs@tocstring{#1}%
  \else
    \hbs@bmdef{#1}%
  \fi
  \hbs@checkarg
}
%    \end{macrocode}
%    \end{macro}
%
%    \begin{macro}{\hbs@getbookmark}
%    \cmd{\hbs@getbookmark} reads an outline entry in parentheses.
%    \begin{macrocode}
\def\hbs@getbookmark(#1){%
  \hbs@bmdef{#1}%
  \hbs@checkarg
}
%    \end{macrocode}
%    \end{macro}
%
%    \begin{macro}{\hbs@bmdef}
%    The command \cmd{\hbs@bmdef} save the bookmark entry in
%    parameter |#1| in the macro \cmd{\hbs@bmstring} and catches
%    the case, if the user has given several outline strings.
%    \begin{macrocode}
\def\hbs@bmdef#1{%
  \ifx\hbs@bmstring\relax
    \def\hbs@bmstring{#1}%
  \else
    \PackageError{hypbmsec}{%
      Sectioning command with too many parameters%
    }{%
      You can only give one outline entry.%
    }%
  \fi
}
%    \end{macrocode}
%    \end{macro}
%
%    \begin{macro}{\hbs@process}
%    The parameter |#1| is the mandatory argument of the sectioning
%    commands. \cmd{\hbs@process} calls the original sectioning command
%    stored in \cmd{\hbs@seccmd} with arguments that depend of which
%    optional argument are used previously.
%    \begin{macrocode}
\long\def\hbs@process#1{%
  \ifx\hbs@tocstring\relax
    \ifx\hbs@bmstring\relax
      \hbs@seccmd{#1}%
    \else
      \begingroup
        \def\x##1{\endgroup
          \hbs@seccmd{\texorpdfstring{#1}{##1}}%
        }%
      \expandafter\x\expandafter{\hbs@bmstring}%
    \fi
  \else
    \ifx\hbs@bmstring\relax
      \expandafter\hbs@seccmd\expandafter[%
        \expandafter{\hbs@tocstring}%
      ]{#1}%
    \else
      \expandafter\expandafter\expandafter
      \hbs@seccmd\expandafter\expandafter\expandafter[%
        \expandafter\expandafter\expandafter
        \texorpdfstring
        \expandafter\expandafter\expandafter{%
          \expandafter\hbs@tocstring\expandafter
        }\expandafter{%
          \hbs@bmstring
        }%
      ]{#1}%
    \fi
  \fi
}
%    \end{macrocode}
%    \end{macro}
%
%    We have to check, whether package \Package{hyperref} is loaded
%    and have to provide a definition for \cmd{\texorpdfstring}.
%    Because \Package{hyperref} can be loaded after this package,
%    we do the work later (\cmd{\AtBeginDocument}).
%
%    This code only checks versions of \Package{hyperref} that
%    define \cmd{\ifbookmark} (v6.4x until v6.53) or
%    \cmd{\texorpdfstring} (v6.54 and above). Older versions aren't
%    supported.
%    \begin{macrocode}
\AtBeginDocument{%
  \@ifundefined{texorpdfstring}{%
    \@ifundefined{ifbookmark}{%
      \let\texorpdfstring\@firstoftwo
      \@ifpackageloaded{hyperref}{%
        \PackageInfo{hypbmsec}{%
          \ifx\hy@driver\@empty
            Default driver %
          \else
            '\hy@driver' %
          \fi
          of hyperref not supported,\MessageBreak
          bookmark parameters will be ignored%
        }%
      }{%
        \PackageInfo{hypbmsec}{%
          Package hyperref not loaded,\MessageBreak
          bookmark parameters will be ignored%
        }%
      }%
    }%
    {%
      \newcommand\texorpdfstring[2]{\ifbookmark{#2}{#1}}%
      \PackageWarningNoLine{hypbmsec}{%
        Old hyperref version found,\MessageBreak
        update of hyperref recommended%
      }%
    }%
  }{}%
%    \end{macrocode}
%
%    Other packages are allowed to redefine the sectioning commands,
%    if they does not change the syntax. Therefore the redefinitons
%    of this package should be done after the other packages.
%    \begin{macrocode}
  \let\hbs@part         \part
  \let\hbs@section      \section
  \let\hbs@subsection   \subsection
  \let\hbs@subsubsection\subsubsection
  \let\hbs@paragraph    \paragraph
  \let\hbs@subparagraph \subparagraph
  \renewcommand\part         {\hbs@do\hbs@part}%
  \renewcommand\section      {\hbs@do\hbs@section}%
  \renewcommand\subsection   {\hbs@do\hbs@subsection}%
  \renewcommand\subsubsection{\hbs@do\hbs@subsubsection}%
  \renewcommand\paragraph    {\hbs@do\hbs@paragraph}%
  \renewcommand\subparagraph {\hbs@do\hbs@subparagraph}%
  \begingroup\expandafter\expandafter\expandafter\endgroup
  \expandafter\ifx\csname chapter\endcsname\relax\else
    \let\hbs@chapter    \chapter
    \renewcommand\chapter    {\hbs@do\hbs@chapter}%
  \fi
}
%    \end{macrocode}
%
%    \begin{macrocode}
%</package>
%    \end{macrocode}
%
% \section{Installation}
%
% \subsection{Download}
%
% \paragraph{Package.} This package is available on
% CTAN\footnote{\url{http://ctan.org/pkg/hypbmsec}}:
% \begin{description}
% \item[\CTAN{macros/latex/contrib/oberdiek/hypbmsec.dtx}] The source file.
% \item[\CTAN{macros/latex/contrib/oberdiek/hypbmsec.pdf}] Documentation.
% \end{description}
%
%
% \paragraph{Bundle.} All the packages of the bundle `oberdiek'
% are also available in a TDS compliant ZIP archive. There
% the packages are already unpacked and the documentation files
% are generated. The files and directories obey the TDS standard.
% \begin{description}
% \item[\CTAN{install/macros/latex/contrib/oberdiek.tds.zip}]
% \end{description}
% \emph{TDS} refers to the standard ``A Directory Structure
% for \TeX\ Files'' (\CTAN{tds/tds.pdf}). Directories
% with \xfile{texmf} in their name are usually organized this way.
%
% \subsection{Bundle installation}
%
% \paragraph{Unpacking.} Unpack the \xfile{oberdiek.tds.zip} in the
% TDS tree (also known as \xfile{texmf} tree) of your choice.
% Example (linux):
% \begin{quote}
%   |unzip oberdiek.tds.zip -d ~/texmf|
% \end{quote}
%
% \paragraph{Script installation.}
% Check the directory \xfile{TDS:scripts/oberdiek/} for
% scripts that need further installation steps.
% Package \xpackage{attachfile2} comes with the Perl script
% \xfile{pdfatfi.pl} that should be installed in such a way
% that it can be called as \texttt{pdfatfi}.
% Example (linux):
% \begin{quote}
%   |chmod +x scripts/oberdiek/pdfatfi.pl|\\
%   |cp scripts/oberdiek/pdfatfi.pl /usr/local/bin/|
% \end{quote}
%
% \subsection{Package installation}
%
% \paragraph{Unpacking.} The \xfile{.dtx} file is a self-extracting
% \docstrip\ archive. The files are extracted by running the
% \xfile{.dtx} through \plainTeX:
% \begin{quote}
%   \verb|tex hypbmsec.dtx|
% \end{quote}
%
% \paragraph{TDS.} Now the different files must be moved into
% the different directories in your installation TDS tree
% (also known as \xfile{texmf} tree):
% \begin{quote}
% \def\t{^^A
% \begin{tabular}{@{}>{\ttfamily}l@{ $\rightarrow$ }>{\ttfamily}l@{}}
%   hypbmsec.sty & tex/latex/oberdiek/hypbmsec.sty\\
%   hypbmsec.pdf & doc/latex/oberdiek/hypbmsec.pdf\\
%   hypbmsec.dtx & source/latex/oberdiek/hypbmsec.dtx\\
% \end{tabular}^^A
% }^^A
% \sbox0{\t}^^A
% \ifdim\wd0>\linewidth
%   \begingroup
%     \advance\linewidth by\leftmargin
%     \advance\linewidth by\rightmargin
%   \edef\x{\endgroup
%     \def\noexpand\lw{\the\linewidth}^^A
%   }\x
%   \def\lwbox{^^A
%     \leavevmode
%     \hbox to \linewidth{^^A
%       \kern-\leftmargin\relax
%       \hss
%       \usebox0
%       \hss
%       \kern-\rightmargin\relax
%     }^^A
%   }^^A
%   \ifdim\wd0>\lw
%     \sbox0{\small\t}^^A
%     \ifdim\wd0>\linewidth
%       \ifdim\wd0>\lw
%         \sbox0{\footnotesize\t}^^A
%         \ifdim\wd0>\linewidth
%           \ifdim\wd0>\lw
%             \sbox0{\scriptsize\t}^^A
%             \ifdim\wd0>\linewidth
%               \ifdim\wd0>\lw
%                 \sbox0{\tiny\t}^^A
%                 \ifdim\wd0>\linewidth
%                   \lwbox
%                 \else
%                   \usebox0
%                 \fi
%               \else
%                 \lwbox
%               \fi
%             \else
%               \usebox0
%             \fi
%           \else
%             \lwbox
%           \fi
%         \else
%           \usebox0
%         \fi
%       \else
%         \lwbox
%       \fi
%     \else
%       \usebox0
%     \fi
%   \else
%     \lwbox
%   \fi
% \else
%   \usebox0
% \fi
% \end{quote}
% If you have a \xfile{docstrip.cfg} that configures and enables \docstrip's
% TDS installing feature, then some files can already be in the right
% place, see the documentation of \docstrip.
%
% \subsection{Refresh file name databases}
%
% If your \TeX~distribution
% (\teTeX, \mikTeX, \dots) relies on file name databases, you must refresh
% these. For example, \teTeX\ users run \verb|texhash| or
% \verb|mktexlsr|.
%
% \subsection{Some details for the interested}
%
% \paragraph{Attached source.}
%
% The PDF documentation on CTAN also includes the
% \xfile{.dtx} source file. It can be extracted by
% AcrobatReader 6 or higher. Another option is \textsf{pdftk},
% e.g. unpack the file into the current directory:
% \begin{quote}
%   \verb|pdftk hypbmsec.pdf unpack_files output .|
% \end{quote}
%
% \paragraph{Unpacking with \LaTeX.}
% The \xfile{.dtx} chooses its action depending on the format:
% \begin{description}
% \item[\plainTeX:] Run \docstrip\ and extract the files.
% \item[\LaTeX:] Generate the documentation.
% \end{description}
% If you insist on using \LaTeX\ for \docstrip\ (really,
% \docstrip\ does not need \LaTeX), then inform the autodetect routine
% about your intention:
% \begin{quote}
%   \verb|latex \let\install=y% \iffalse meta-comment
%
% File: hypbmsec.dtx
% Version: 2016/05/16 v2.5
% Info: Bookmarks in sectioning commands
%
% Copyright (C) 1998-2000, 2006, 2007 by
%    Heiko Oberdiek <heiko.oberdiek at googlemail.com>
%    2016
%    https://github.com/ho-tex/oberdiek/issues
%
% This work may be distributed and/or modified under the
% conditions of the LaTeX Project Public License, either
% version 1.3c of this license or (at your option) any later
% version. This version of this license is in
%    http://www.latex-project.org/lppl/lppl-1-3c.txt
% and the latest version of this license is in
%    http://www.latex-project.org/lppl.txt
% and version 1.3 or later is part of all distributions of
% LaTeX version 2005/12/01 or later.
%
% This work has the LPPL maintenance status "maintained".
%
% This Current Maintainer of this work is Heiko Oberdiek.
%
% This work consists of the main source file hypbmsec.dtx
% and the derived files
%    hypbmsec.sty, hypbmsec.pdf, hypbmsec.ins, hypbmsec.drv.
%
% Distribution:
%    CTAN:macros/latex/contrib/oberdiek/hypbmsec.dtx
%    CTAN:macros/latex/contrib/oberdiek/hypbmsec.pdf
%
% Unpacking:
%    (a) If hypbmsec.ins is present:
%           tex hypbmsec.ins
%    (b) Without hypbmsec.ins:
%           tex hypbmsec.dtx
%    (c) If you insist on using LaTeX
%           latex \let\install=y% \iffalse meta-comment
%
% File: hypbmsec.dtx
% Version: 2016/05/16 v2.5
% Info: Bookmarks in sectioning commands
%
% Copyright (C) 1998-2000, 2006, 2007 by
%    Heiko Oberdiek <heiko.oberdiek at googlemail.com>
%    2016
%    https://github.com/ho-tex/oberdiek/issues
%
% This work may be distributed and/or modified under the
% conditions of the LaTeX Project Public License, either
% version 1.3c of this license or (at your option) any later
% version. This version of this license is in
%    http://www.latex-project.org/lppl/lppl-1-3c.txt
% and the latest version of this license is in
%    http://www.latex-project.org/lppl.txt
% and version 1.3 or later is part of all distributions of
% LaTeX version 2005/12/01 or later.
%
% This work has the LPPL maintenance status "maintained".
%
% This Current Maintainer of this work is Heiko Oberdiek.
%
% This work consists of the main source file hypbmsec.dtx
% and the derived files
%    hypbmsec.sty, hypbmsec.pdf, hypbmsec.ins, hypbmsec.drv.
%
% Distribution:
%    CTAN:macros/latex/contrib/oberdiek/hypbmsec.dtx
%    CTAN:macros/latex/contrib/oberdiek/hypbmsec.pdf
%
% Unpacking:
%    (a) If hypbmsec.ins is present:
%           tex hypbmsec.ins
%    (b) Without hypbmsec.ins:
%           tex hypbmsec.dtx
%    (c) If you insist on using LaTeX
%           latex \let\install=y% \iffalse meta-comment
%
% File: hypbmsec.dtx
% Version: 2016/05/16 v2.5
% Info: Bookmarks in sectioning commands
%
% Copyright (C) 1998-2000, 2006, 2007 by
%    Heiko Oberdiek <heiko.oberdiek at googlemail.com>
%    2016
%    https://github.com/ho-tex/oberdiek/issues
%
% This work may be distributed and/or modified under the
% conditions of the LaTeX Project Public License, either
% version 1.3c of this license or (at your option) any later
% version. This version of this license is in
%    http://www.latex-project.org/lppl/lppl-1-3c.txt
% and the latest version of this license is in
%    http://www.latex-project.org/lppl.txt
% and version 1.3 or later is part of all distributions of
% LaTeX version 2005/12/01 or later.
%
% This work has the LPPL maintenance status "maintained".
%
% This Current Maintainer of this work is Heiko Oberdiek.
%
% This work consists of the main source file hypbmsec.dtx
% and the derived files
%    hypbmsec.sty, hypbmsec.pdf, hypbmsec.ins, hypbmsec.drv.
%
% Distribution:
%    CTAN:macros/latex/contrib/oberdiek/hypbmsec.dtx
%    CTAN:macros/latex/contrib/oberdiek/hypbmsec.pdf
%
% Unpacking:
%    (a) If hypbmsec.ins is present:
%           tex hypbmsec.ins
%    (b) Without hypbmsec.ins:
%           tex hypbmsec.dtx
%    (c) If you insist on using LaTeX
%           latex \let\install=y\input{hypbmsec.dtx}
%        (quote the arguments according to the demands of your shell)
%
% Documentation:
%    (a) If hypbmsec.drv is present:
%           latex hypbmsec.drv
%    (b) Without hypbmsec.drv:
%           latex hypbmsec.dtx; ...
%    The class ltxdoc loads the configuration file ltxdoc.cfg
%    if available. Here you can specify further options, e.g.
%    use A4 as paper format:
%       \PassOptionsToClass{a4paper}{article}
%
%    Programm calls to get the documentation (example):
%       pdflatex hypbmsec.dtx
%       makeindex -s gind.ist hypbmsec.idx
%       pdflatex hypbmsec.dtx
%       makeindex -s gind.ist hypbmsec.idx
%       pdflatex hypbmsec.dtx
%
% Installation:
%    TDS:tex/latex/oberdiek/hypbmsec.sty
%    TDS:doc/latex/oberdiek/hypbmsec.pdf
%    TDS:source/latex/oberdiek/hypbmsec.dtx
%
%<*ignore>
\begingroup
  \catcode123=1 %
  \catcode125=2 %
  \def\x{LaTeX2e}%
\expandafter\endgroup
\ifcase 0\ifx\install y1\fi\expandafter
         \ifx\csname processbatchFile\endcsname\relax\else1\fi
         \ifx\fmtname\x\else 1\fi\relax
\else\csname fi\endcsname
%</ignore>
%<*install>
\input docstrip.tex
\Msg{************************************************************************}
\Msg{* Installation}
\Msg{* Package: hypbmsec 2016/05/16 v2.5 Bookmarks in sectioning commands (HO)}
\Msg{************************************************************************}

\keepsilent
\askforoverwritefalse

\let\MetaPrefix\relax
\preamble

This is a generated file.

Project: hypbmsec
Version: 2016/05/16 v2.5

Copyright (C) 1998-2000, 2006, 2007 by
   Heiko Oberdiek <heiko.oberdiek at googlemail.com>

This work may be distributed and/or modified under the
conditions of the LaTeX Project Public License, either
version 1.3c of this license or (at your option) any later
version. This version of this license is in
   http://www.latex-project.org/lppl/lppl-1-3c.txt
and the latest version of this license is in
   http://www.latex-project.org/lppl.txt
and version 1.3 or later is part of all distributions of
LaTeX version 2005/12/01 or later.

This work has the LPPL maintenance status "maintained".

This Current Maintainer of this work is Heiko Oberdiek.

This work consists of the main source file hypbmsec.dtx
and the derived files
   hypbmsec.sty, hypbmsec.pdf, hypbmsec.ins, hypbmsec.drv.

\endpreamble
\let\MetaPrefix\DoubleperCent

\generate{%
  \file{hypbmsec.ins}{\from{hypbmsec.dtx}{install}}%
  \file{hypbmsec.drv}{\from{hypbmsec.dtx}{driver}}%
  \usedir{tex/latex/oberdiek}%
  \file{hypbmsec.sty}{\from{hypbmsec.dtx}{package}}%
  \nopreamble
  \nopostamble
  \usedir{source/latex/oberdiek/catalogue}%
  \file{hypbmsec.xml}{\from{hypbmsec.dtx}{catalogue}}%
}

\catcode32=13\relax% active space
\let =\space%
\Msg{************************************************************************}
\Msg{*}
\Msg{* To finish the installation you have to move the following}
\Msg{* file into a directory searched by TeX:}
\Msg{*}
\Msg{*     hypbmsec.sty}
\Msg{*}
\Msg{* To produce the documentation run the file `hypbmsec.drv'}
\Msg{* through LaTeX.}
\Msg{*}
\Msg{* Happy TeXing!}
\Msg{*}
\Msg{************************************************************************}

\endbatchfile
%</install>
%<*ignore>
\fi
%</ignore>
%<*driver>
\NeedsTeXFormat{LaTeX2e}
\ProvidesFile{hypbmsec.drv}%
  [2016/05/16 v2.5 Bookmarks in sectioning commands (HO)]%
\documentclass{ltxdoc}
\usepackage{holtxdoc}[2011/11/22]
\begin{document}
  \DocInput{hypbmsec.dtx}%
\end{document}
%</driver>
% \fi
%
%
% \CharacterTable
%  {Upper-case    \A\B\C\D\E\F\G\H\I\J\K\L\M\N\O\P\Q\R\S\T\U\V\W\X\Y\Z
%   Lower-case    \a\b\c\d\e\f\g\h\i\j\k\l\m\n\o\p\q\r\s\t\u\v\w\x\y\z
%   Digits        \0\1\2\3\4\5\6\7\8\9
%   Exclamation   \!     Double quote  \"     Hash (number) \#
%   Dollar        \$     Percent       \%     Ampersand     \&
%   Acute accent  \'     Left paren    \(     Right paren   \)
%   Asterisk      \*     Plus          \+     Comma         \,
%   Minus         \-     Point         \.     Solidus       \/
%   Colon         \:     Semicolon     \;     Less than     \<
%   Equals        \=     Greater than  \>     Question mark \?
%   Commercial at \@     Left bracket  \[     Backslash     \\
%   Right bracket \]     Circumflex    \^     Underscore    \_
%   Grave accent  \`     Left brace    \{     Vertical bar  \|
%   Right brace   \}     Tilde         \~}
%
% \GetFileInfo{hypbmsec.drv}
%
% \title{The \xpackage{hypbmsec} package}
% \date{2016/05/16 v2.5}
% \author{Heiko Oberdiek\thanks
% {Please report any issues at https://github.com/ho-tex/oberdiek/issues}\\
% \xemail{heiko.oberdiek at googlemail.com}}
%
% \maketitle
%
% \begin{abstract}
% This package expands the syntax of the sectioning commands. If the
% argument of the sectioning commands isn't usable as outline entry,
% a replacement for the bookmarks can be given.
% \end{abstract}
%
% \tableofcontents
%
% \newcommand{\type}[1]{\textsf{#1}}
%
% ^^A No thread support.
% \newenvironment{article}[1]{}{}
%
% \section{Usage}
%
% \subsection{Bookmarks restrictions}\label{sec:restrictions}
%    Outline entries (bookmarks) are written to a file and have
%    to obey the pdf specification.
%    Therefore they have several restrictions:
%    \begin{itemize}
%    \item Bookmarks have to be encoded in PDFDocEncoding^^A
%          \footnote{\Package{hyperref} doesn't support
%            Unicode.}.
%    \item They should only expand to letters and spaces.
%    \item The result of expansion have to be a valid pdf string.
%    \item Stomach commands like \cmd{\relax}, box commands, math,
%          assignments, or definitions aren't allowed.
%    \item Short entries are recommended, which allow a clear view.
%    \end{itemize}
%
% \subsection{\texorpdfstring{\cmd{\texorpdfstring}}{^^A
%    \textbackslash texorpdfstring}}
%    The generic way in package \Package{hyperref} is the use
%    of \cmd{\texorpdfstring}^^A
%    \footnote{In versions of \Package{hyperref} below 6.54 see
%      \cmd{\ifbookmark}.}:
%    \begin{quote}
%\begin{verbatim}
%\section{Pythagoras:
%  \texorpdfstring{$a^2+b^2=c^2}{%
%    a\texttwosuperior\ + b\texttwosuperior\ =
%    c\texttwosuperior}%
%}
%\end{verbatim}
%    \end{quote}
%
% \subsection{Sectioning commands}
%    The package \Package{hyperref} automatically generates
%    bookmarks from the sectioning commands,
%    unless it is suppressed by an option.
%    Commands that structure the text are here called
%    ``sectioning commands'':
%    \begin{quote}
%    \cmd{\part}, \cmd{\chapter},\\
%    \cmd{\section}, \cmd{\subsection}, \cmd{\subsubsection},\\
%    \cmd{\paragraph}, \cmd{\subparagraph}
%    \end{quote}
%
% \subsection{Places\texorpdfstring{ for sectioning strings}{}}
%    \label{sec:places}
%    The argument(s) of these commands are used on several places:
%    \begin{description}
%    \item[\type{text}]
%      The current text without restrictions.
%    \item[\type{toc}]
%      The headlines and the table of contents with the
%      restrictions of ``moving arguments''.
%    \item[\type{out}]
%      The outlines with many restrictions: The outline
%      have to expand to a valid pdf string with PDFDocEncoding
%      (see \ref{sec:restrictions}).
%    \end{description}
%
% \subsection{\texorpdfstring{Solution with o}{O}ptional arguments}
%    If the user wants to use a footnote within a sectioning command,
%    the \LaTeX{} solution is an optional argument:
%    \begin{quote}
%      |\section[Title]{Title\footnote{Footnote text}}|
%    \end{quote}
%    Now |Title| without the footnote is used in the headlines and
%    the table of contents. Also \Package{hyperref} uses it for the
%    bookmarks.
%
%    This package \Package{\filename} offers two possibilities to
%    specify a separate outline entry:
%    \begin{itemize}
%    \item An additional second optional argument in square brackets.
%    \item An additional optional argument in parentheses (in
%          assoziation with a pdf string that is internally surrounded
%          by parentheses, too).
%    \end{itemize}
%    Because \Package{\filename} stores the original meaning of the
%    sectioning commands and uses them again, there should be no
%    problems with packages that redefine the sectioning commands, if
%    these packages doesn't change the syntax.
%
% \subsection{Syntax}
%    The following examples show the syntax of the sectioning
%    commands. For the places the strings appear the abbreviations
%    are used, that are introduced in \ref{sec:places}.
%
% \subsubsection{\texorpdfstring{Star form}{^^A
%    \textbackslash section*\{\}}}
%    The behaviour of the star form isn't changed. The string
%    appears only in the current text:
%    \begin{article}{Syntax}
%    \begin{quote}
%      |\section*{text}|
%    \end{quote}
%    \end{article}
%
% \subsubsection{\texorpdfstring{Without optional arguments}{^^A
%    \textbackslash section\{\}}}
%    The normal case, the string in the mandatory argument is
%    used for all places:
%    \begin{article}{Syntax}
%    \begin{quote}
%      |\section{text, toc, out}|
%    \end{quote}
%    \end{article}
%
% \subsubsection{\texorpdfstring{One optional argument}{^^A
%    \textbackslash section[]\{\}}}
%    Also the form with one optional parameter in square brackets isn't
%    new; for the bookmarks the optional parameter is used:
%    \begin{article}{Syntax}
%    \begin{quote}
%      |\section[toc, out]{text}|
%    \end{quote}
%    \end{article}
%
% \subsubsection{\texorpdfstring{Two optional arguments}{^^A
%    \textbackslash section[][out]\{\}}}\label{sec:two}
%    The second optional parameter in square brackets is introduced
%    by this package to specify an outline entry:
%    \begin{article}{Syntax}
%    \begin{quote}
%      |\section[toc][out]{text}|
%    \end{quote}
%    \end{article}
%
% \subsubsection{\texorpdfstring{Optional argument in parentheses}{^^A
%    \textbackslash section(out)\{\}}}
%    Often the \type{toc} and the \type{text} string would be the same.
%    With the method of the two optional arguments in square brackets
%    (see \ref{sec:two}) this string must be given twice,
%    if the user only wants to specify a different outline entry.
%    Therefore this package offers another possibility:
%    In association with the internal representation in the pdf file
%    an outline entry can be given in parentheses.
%    So the package can easily distinguish between
%    the two forms of optional arguments and the order does not matter:
%    \begin{article}{Syntax}
%    \begin{quote}
%      |\section(out){toc, text}|\\
%      |\section[toc](out){text}|\\
%      |\section(out)[toc]{text}|
%    \end{quote}
%    \end{article}
%
% \subsection{Without \Package{hyperref}}
%    Package \Package{\filename} uses \Package{hyperref} for support of
%    the bookmarks, but this package is not required.
%    If \Package{hyperref} isn't loaded, or
%    is called with a driver that doesn't support bookmarks,
%    package \Package{\filename} shouldn't be removed,
%    because this would lead to
%    a wrong syntax of the sectioning commands.
%    In any cases package \Package{\filename}
%    supports its syntax and ignores the outline entries,
%    if there are no code for bookmarks.
%    So it is possible to write texts,
%    that are processed with several drivers to get different output
%    formats.
%
% \subsection{Protecting parentheses}
%    If the string itself contains parentheses, they have to be hidden
%    from \TeX's argument parsing mechanism.
%    The argument should be surrounded
%    by curly braces:
%    \begin{quote}
%      |\section({outlines(bookmarks)}){text, toc}|
%    \end{quote}
%    With version 6.54 of \Package{hyperref} the other standard method
%    works, too: The closing parenthesis is protected:
%    \begin{quote}
%      |\section(outlines(bookmarks{)}){text, toc}|
%    \end{quote}
%
% \StopEventually{
% }
%
% \section{Implementation}
%    \begin{macrocode}
%<*package>
%    \end{macrocode}
%    Package identification.
%    \begin{macrocode}
\NeedsTeXFormat{LaTeX2e}
\ProvidesPackage{hypbmsec}%
  [2016/05/16 v2.5 Bookmarks in sectioning commands (HO)]
%    \end{macrocode}
%
%    Because of redifining the sectioning commands, it is dangerous
%    to reload the package several times.
%    \begin{macrocode}
\@ifundefined{hbs@do}{}{%
  \PackageInfo{hypbmsec}{Package 'hypbmsec' is already loaded}%
  \endinput
}
%    \end{macrocode}
%
%    \begin{macro}{\hbs@do}
%    The redefined sectioning commands calls \cmd{\hbs@do}. It does
%    \begin{itemize}
%    \item handle the star case.
%    \item resets the macros that store the entries for the outlines
%          (\cmd{\hbs@bmstring}) and table of contents (\cmd{\hbs@tocstring}).
%    \item store the sectioning command |#1| in \cmd{\hbs@seccmd}
%          for later reuse.
%    \item at last call \cmd{\hbs@checkarg} that scans and interprets the
%          parameters of the redefined sectioning command.
%    \end{itemize}
%    \begin{macrocode}
\def\hbs@do#1{%
  \@ifstar{#1*}{%
    \let\hbs@tocstring\relax
    \let\hbs@bmstring\relax
    \let\hbs@seccmd#1%
    \hbs@checkarg
  }%
}
%    \end{macrocode}
%    \end{macro}
%
%    \begin{macro}{\hbs@checkarg}
%    \cmd{\hbs@checkarg} determines the type of the next argument:
%    \begin{itemize}
%    \item An optional argument in square brackets can be an entry
%          for the table of contents or the bookmarks. It will be
%          read by \cmd{\hbs@getsquare}
%    \item An optional argument in parentheses is an outline entry.
%          This is worked off by \cmd{\hbs@getbookmark}.
%    \item If there are no more optional arguments, \cmd{\hbs@process}
%          reads the mandatory argument and calls the original
%          sectioning commands.
%    \end{itemize}
%    \begin{macrocode}
\def\hbs@checkarg{%
  \@ifnextchar[\hbs@getsquare{%
    \@ifnextchar(\hbs@getbookmark\hbs@process
  }%
}
%    \end{macrocode}
%    \end{macro}
%
%    \begin{macro}{\hbs@getsquare}
%    \cmd{\hbs@getsquare} reads an optional argument in square
%    brackets and determines, if this is an entry for the table
%    of contents or the bookmarks.
%    \begin{macrocode}
\long\def\hbs@getsquare[#1]{%
  \ifx\hbs@tocstring\relax
    \def\hbs@tocstring{#1}%
  \else
    \hbs@bmdef{#1}%
  \fi
  \hbs@checkarg
}
%    \end{macrocode}
%    \end{macro}
%
%    \begin{macro}{\hbs@getbookmark}
%    \cmd{\hbs@getbookmark} reads an outline entry in parentheses.
%    \begin{macrocode}
\def\hbs@getbookmark(#1){%
  \hbs@bmdef{#1}%
  \hbs@checkarg
}
%    \end{macrocode}
%    \end{macro}
%
%    \begin{macro}{\hbs@bmdef}
%    The command \cmd{\hbs@bmdef} save the bookmark entry in
%    parameter |#1| in the macro \cmd{\hbs@bmstring} and catches
%    the case, if the user has given several outline strings.
%    \begin{macrocode}
\def\hbs@bmdef#1{%
  \ifx\hbs@bmstring\relax
    \def\hbs@bmstring{#1}%
  \else
    \PackageError{hypbmsec}{%
      Sectioning command with too many parameters%
    }{%
      You can only give one outline entry.%
    }%
  \fi
}
%    \end{macrocode}
%    \end{macro}
%
%    \begin{macro}{\hbs@process}
%    The parameter |#1| is the mandatory argument of the sectioning
%    commands. \cmd{\hbs@process} calls the original sectioning command
%    stored in \cmd{\hbs@seccmd} with arguments that depend of which
%    optional argument are used previously.
%    \begin{macrocode}
\long\def\hbs@process#1{%
  \ifx\hbs@tocstring\relax
    \ifx\hbs@bmstring\relax
      \hbs@seccmd{#1}%
    \else
      \begingroup
        \def\x##1{\endgroup
          \hbs@seccmd{\texorpdfstring{#1}{##1}}%
        }%
      \expandafter\x\expandafter{\hbs@bmstring}%
    \fi
  \else
    \ifx\hbs@bmstring\relax
      \expandafter\hbs@seccmd\expandafter[%
        \expandafter{\hbs@tocstring}%
      ]{#1}%
    \else
      \expandafter\expandafter\expandafter
      \hbs@seccmd\expandafter\expandafter\expandafter[%
        \expandafter\expandafter\expandafter
        \texorpdfstring
        \expandafter\expandafter\expandafter{%
          \expandafter\hbs@tocstring\expandafter
        }\expandafter{%
          \hbs@bmstring
        }%
      ]{#1}%
    \fi
  \fi
}
%    \end{macrocode}
%    \end{macro}
%
%    We have to check, whether package \Package{hyperref} is loaded
%    and have to provide a definition for \cmd{\texorpdfstring}.
%    Because \Package{hyperref} can be loaded after this package,
%    we do the work later (\cmd{\AtBeginDocument}).
%
%    This code only checks versions of \Package{hyperref} that
%    define \cmd{\ifbookmark} (v6.4x until v6.53) or
%    \cmd{\texorpdfstring} (v6.54 and above). Older versions aren't
%    supported.
%    \begin{macrocode}
\AtBeginDocument{%
  \@ifundefined{texorpdfstring}{%
    \@ifundefined{ifbookmark}{%
      \let\texorpdfstring\@firstoftwo
      \@ifpackageloaded{hyperref}{%
        \PackageInfo{hypbmsec}{%
          \ifx\hy@driver\@empty
            Default driver %
          \else
            '\hy@driver' %
          \fi
          of hyperref not supported,\MessageBreak
          bookmark parameters will be ignored%
        }%
      }{%
        \PackageInfo{hypbmsec}{%
          Package hyperref not loaded,\MessageBreak
          bookmark parameters will be ignored%
        }%
      }%
    }%
    {%
      \newcommand\texorpdfstring[2]{\ifbookmark{#2}{#1}}%
      \PackageWarningNoLine{hypbmsec}{%
        Old hyperref version found,\MessageBreak
        update of hyperref recommended%
      }%
    }%
  }{}%
%    \end{macrocode}
%
%    Other packages are allowed to redefine the sectioning commands,
%    if they does not change the syntax. Therefore the redefinitons
%    of this package should be done after the other packages.
%    \begin{macrocode}
  \let\hbs@part         \part
  \let\hbs@section      \section
  \let\hbs@subsection   \subsection
  \let\hbs@subsubsection\subsubsection
  \let\hbs@paragraph    \paragraph
  \let\hbs@subparagraph \subparagraph
  \renewcommand\part         {\hbs@do\hbs@part}%
  \renewcommand\section      {\hbs@do\hbs@section}%
  \renewcommand\subsection   {\hbs@do\hbs@subsection}%
  \renewcommand\subsubsection{\hbs@do\hbs@subsubsection}%
  \renewcommand\paragraph    {\hbs@do\hbs@paragraph}%
  \renewcommand\subparagraph {\hbs@do\hbs@subparagraph}%
  \begingroup\expandafter\expandafter\expandafter\endgroup
  \expandafter\ifx\csname chapter\endcsname\relax\else
    \let\hbs@chapter    \chapter
    \renewcommand\chapter    {\hbs@do\hbs@chapter}%
  \fi
}
%    \end{macrocode}
%
%    \begin{macrocode}
%</package>
%    \end{macrocode}
%
% \section{Installation}
%
% \subsection{Download}
%
% \paragraph{Package.} This package is available on
% CTAN\footnote{\url{http://ctan.org/pkg/hypbmsec}}:
% \begin{description}
% \item[\CTAN{macros/latex/contrib/oberdiek/hypbmsec.dtx}] The source file.
% \item[\CTAN{macros/latex/contrib/oberdiek/hypbmsec.pdf}] Documentation.
% \end{description}
%
%
% \paragraph{Bundle.} All the packages of the bundle `oberdiek'
% are also available in a TDS compliant ZIP archive. There
% the packages are already unpacked and the documentation files
% are generated. The files and directories obey the TDS standard.
% \begin{description}
% \item[\CTAN{install/macros/latex/contrib/oberdiek.tds.zip}]
% \end{description}
% \emph{TDS} refers to the standard ``A Directory Structure
% for \TeX\ Files'' (\CTAN{tds/tds.pdf}). Directories
% with \xfile{texmf} in their name are usually organized this way.
%
% \subsection{Bundle installation}
%
% \paragraph{Unpacking.} Unpack the \xfile{oberdiek.tds.zip} in the
% TDS tree (also known as \xfile{texmf} tree) of your choice.
% Example (linux):
% \begin{quote}
%   |unzip oberdiek.tds.zip -d ~/texmf|
% \end{quote}
%
% \paragraph{Script installation.}
% Check the directory \xfile{TDS:scripts/oberdiek/} for
% scripts that need further installation steps.
% Package \xpackage{attachfile2} comes with the Perl script
% \xfile{pdfatfi.pl} that should be installed in such a way
% that it can be called as \texttt{pdfatfi}.
% Example (linux):
% \begin{quote}
%   |chmod +x scripts/oberdiek/pdfatfi.pl|\\
%   |cp scripts/oberdiek/pdfatfi.pl /usr/local/bin/|
% \end{quote}
%
% \subsection{Package installation}
%
% \paragraph{Unpacking.} The \xfile{.dtx} file is a self-extracting
% \docstrip\ archive. The files are extracted by running the
% \xfile{.dtx} through \plainTeX:
% \begin{quote}
%   \verb|tex hypbmsec.dtx|
% \end{quote}
%
% \paragraph{TDS.} Now the different files must be moved into
% the different directories in your installation TDS tree
% (also known as \xfile{texmf} tree):
% \begin{quote}
% \def\t{^^A
% \begin{tabular}{@{}>{\ttfamily}l@{ $\rightarrow$ }>{\ttfamily}l@{}}
%   hypbmsec.sty & tex/latex/oberdiek/hypbmsec.sty\\
%   hypbmsec.pdf & doc/latex/oberdiek/hypbmsec.pdf\\
%   hypbmsec.dtx & source/latex/oberdiek/hypbmsec.dtx\\
% \end{tabular}^^A
% }^^A
% \sbox0{\t}^^A
% \ifdim\wd0>\linewidth
%   \begingroup
%     \advance\linewidth by\leftmargin
%     \advance\linewidth by\rightmargin
%   \edef\x{\endgroup
%     \def\noexpand\lw{\the\linewidth}^^A
%   }\x
%   \def\lwbox{^^A
%     \leavevmode
%     \hbox to \linewidth{^^A
%       \kern-\leftmargin\relax
%       \hss
%       \usebox0
%       \hss
%       \kern-\rightmargin\relax
%     }^^A
%   }^^A
%   \ifdim\wd0>\lw
%     \sbox0{\small\t}^^A
%     \ifdim\wd0>\linewidth
%       \ifdim\wd0>\lw
%         \sbox0{\footnotesize\t}^^A
%         \ifdim\wd0>\linewidth
%           \ifdim\wd0>\lw
%             \sbox0{\scriptsize\t}^^A
%             \ifdim\wd0>\linewidth
%               \ifdim\wd0>\lw
%                 \sbox0{\tiny\t}^^A
%                 \ifdim\wd0>\linewidth
%                   \lwbox
%                 \else
%                   \usebox0
%                 \fi
%               \else
%                 \lwbox
%               \fi
%             \else
%               \usebox0
%             \fi
%           \else
%             \lwbox
%           \fi
%         \else
%           \usebox0
%         \fi
%       \else
%         \lwbox
%       \fi
%     \else
%       \usebox0
%     \fi
%   \else
%     \lwbox
%   \fi
% \else
%   \usebox0
% \fi
% \end{quote}
% If you have a \xfile{docstrip.cfg} that configures and enables \docstrip's
% TDS installing feature, then some files can already be in the right
% place, see the documentation of \docstrip.
%
% \subsection{Refresh file name databases}
%
% If your \TeX~distribution
% (\teTeX, \mikTeX, \dots) relies on file name databases, you must refresh
% these. For example, \teTeX\ users run \verb|texhash| or
% \verb|mktexlsr|.
%
% \subsection{Some details for the interested}
%
% \paragraph{Attached source.}
%
% The PDF documentation on CTAN also includes the
% \xfile{.dtx} source file. It can be extracted by
% AcrobatReader 6 or higher. Another option is \textsf{pdftk},
% e.g. unpack the file into the current directory:
% \begin{quote}
%   \verb|pdftk hypbmsec.pdf unpack_files output .|
% \end{quote}
%
% \paragraph{Unpacking with \LaTeX.}
% The \xfile{.dtx} chooses its action depending on the format:
% \begin{description}
% \item[\plainTeX:] Run \docstrip\ and extract the files.
% \item[\LaTeX:] Generate the documentation.
% \end{description}
% If you insist on using \LaTeX\ for \docstrip\ (really,
% \docstrip\ does not need \LaTeX), then inform the autodetect routine
% about your intention:
% \begin{quote}
%   \verb|latex \let\install=y\input{hypbmsec.dtx}|
% \end{quote}
% Do not forget to quote the argument according to the demands
% of your shell.
%
% \paragraph{Generating the documentation.}
% You can use both the \xfile{.dtx} or the \xfile{.drv} to generate
% the documentation. The process can be configured by the
% configuration file \xfile{ltxdoc.cfg}. For instance, put this
% line into this file, if you want to have A4 as paper format:
% \begin{quote}
%   \verb|\PassOptionsToClass{a4paper}{article}|
% \end{quote}
% An example follows how to generate the
% documentation with pdf\LaTeX:
% \begin{quote}
%\begin{verbatim}
%pdflatex hypbmsec.dtx
%makeindex -s gind.ist hypbmsec.idx
%pdflatex hypbmsec.dtx
%makeindex -s gind.ist hypbmsec.idx
%pdflatex hypbmsec.dtx
%\end{verbatim}
% \end{quote}
%
% \section{Catalogue}
%
% The following XML file can be used as source for the
% \href{http://mirror.ctan.org/help/Catalogue/catalogue.html}{\TeX\ Catalogue}.
% The elements \texttt{caption} and \texttt{description} are imported
% from the original XML file from the Catalogue.
% The name of the XML file in the Catalogue is \xfile{hypbmsec.xml}.
%    \begin{macrocode}
%<*catalogue>
<?xml version='1.0' encoding='us-ascii'?>
<!DOCTYPE entry SYSTEM 'catalogue.dtd'>
<entry datestamp='$Date$' modifier='$Author$' id='hypbmsec'>
  <name>hypbmsec</name>
  <caption>Hypertext bookmarks in sectioning commands.</caption>
  <authorref id='auth:oberdiek'/>
  <copyright owner='Heiko Oberdiek' year='1998-2000,2006,2007'/>
  <license type='lppl1.3'/>
  <version number='2.5'/>
  <description>
    Bookmark entries can be given as another argument to the LaTeX
    sectioning commands. The <xref refid='hyperref'>hyperref</xref>
    package is required to get the bookmarks, but the syntax
    works without it.
    <p/>
    This package is part of the <xref refid='oberdiek'>oberdiek</xref>
    bundle.
  </description>
  <documentation details='Package documentation'
      href='ctan:/macros/latex/contrib/oberdiek/hypbmsec.pdf'/>
  <ctan file='true' path='/macros/latex/contrib/oberdiek/hypbmsec.dtx'/>
  <miktex location='oberdiek'/>
  <texlive location='oberdiek'/>
  <install path='/macros/latex/contrib/oberdiek/oberdiek.tds.zip'/>
</entry>
%</catalogue>
%    \end{macrocode}
%
% \begin{History}
%   \begin{Version}{1998/11/20 v1.0}
%   \item
%     First version.
%   \item
%     It merges package \xpackage{hysecopt} and
%   \item
%     package \xpackage{hypbmpar}.
%   \item
%     Published for the DANTE'99 meeting^^A
%     \URL{}{http://dante99.cs.uni-dortmund.de/handouts/oberdiek/hypbmsec.sty}.
%   \end{Version}
%   \begin{Version}{1999/04/12 v2.0}
%   \item
%     Adaptation to \Package{hyperref} version 6.54.
%   \item
%     Documentation in dtx format.
%   \item
%     Copyright: LPPL (\CTAN{macros/latex/base/lppl.txt})
%   \item
%     First CTAN release.
%   \end{Version}
%   \begin{Version}{2000/03/22 v2.1}
%   \item
%     Bug fix in redefinition of \cmd{\chapter}.
%   \item
%     Copyright: LPPL 1.2
%   \end{Version}
%   \begin{Version}{2006/02/20 v2.2}
%   \item
%     Code is not changed.
%   \item
%     New DTX framework.
%   \item
%     LPPL 1.3
%   \end{Version}
%   \begin{Version}{2007/03/05 v2.3}
%   \item
%     Bug fix: Expand \cs{hbs@tocstring} and \cs{hbs@bmstring} before
%     calling \cs{hbs@seccmd}.
%   \end{Version}
%   \begin{Version}{2007/04/11 v2.4}
%   \item
%     Line ends sanitized.
%   \end{Version}
%   \begin{Version}{2016/05/16 v2.5}
%   \item
%     Documentation updates.
%   \end{Version}
% \end{History}
%
% \PrintIndex
%
% \Finale
\endinput

%        (quote the arguments according to the demands of your shell)
%
% Documentation:
%    (a) If hypbmsec.drv is present:
%           latex hypbmsec.drv
%    (b) Without hypbmsec.drv:
%           latex hypbmsec.dtx; ...
%    The class ltxdoc loads the configuration file ltxdoc.cfg
%    if available. Here you can specify further options, e.g.
%    use A4 as paper format:
%       \PassOptionsToClass{a4paper}{article}
%
%    Programm calls to get the documentation (example):
%       pdflatex hypbmsec.dtx
%       makeindex -s gind.ist hypbmsec.idx
%       pdflatex hypbmsec.dtx
%       makeindex -s gind.ist hypbmsec.idx
%       pdflatex hypbmsec.dtx
%
% Installation:
%    TDS:tex/latex/oberdiek/hypbmsec.sty
%    TDS:doc/latex/oberdiek/hypbmsec.pdf
%    TDS:source/latex/oberdiek/hypbmsec.dtx
%
%<*ignore>
\begingroup
  \catcode123=1 %
  \catcode125=2 %
  \def\x{LaTeX2e}%
\expandafter\endgroup
\ifcase 0\ifx\install y1\fi\expandafter
         \ifx\csname processbatchFile\endcsname\relax\else1\fi
         \ifx\fmtname\x\else 1\fi\relax
\else\csname fi\endcsname
%</ignore>
%<*install>
\input docstrip.tex
\Msg{************************************************************************}
\Msg{* Installation}
\Msg{* Package: hypbmsec 2016/05/16 v2.5 Bookmarks in sectioning commands (HO)}
\Msg{************************************************************************}

\keepsilent
\askforoverwritefalse

\let\MetaPrefix\relax
\preamble

This is a generated file.

Project: hypbmsec
Version: 2016/05/16 v2.5

Copyright (C) 1998-2000, 2006, 2007 by
   Heiko Oberdiek <heiko.oberdiek at googlemail.com>

This work may be distributed and/or modified under the
conditions of the LaTeX Project Public License, either
version 1.3c of this license or (at your option) any later
version. This version of this license is in
   http://www.latex-project.org/lppl/lppl-1-3c.txt
and the latest version of this license is in
   http://www.latex-project.org/lppl.txt
and version 1.3 or later is part of all distributions of
LaTeX version 2005/12/01 or later.

This work has the LPPL maintenance status "maintained".

This Current Maintainer of this work is Heiko Oberdiek.

This work consists of the main source file hypbmsec.dtx
and the derived files
   hypbmsec.sty, hypbmsec.pdf, hypbmsec.ins, hypbmsec.drv.

\endpreamble
\let\MetaPrefix\DoubleperCent

\generate{%
  \file{hypbmsec.ins}{\from{hypbmsec.dtx}{install}}%
  \file{hypbmsec.drv}{\from{hypbmsec.dtx}{driver}}%
  \usedir{tex/latex/oberdiek}%
  \file{hypbmsec.sty}{\from{hypbmsec.dtx}{package}}%
  \nopreamble
  \nopostamble
  \usedir{source/latex/oberdiek/catalogue}%
  \file{hypbmsec.xml}{\from{hypbmsec.dtx}{catalogue}}%
}

\catcode32=13\relax% active space
\let =\space%
\Msg{************************************************************************}
\Msg{*}
\Msg{* To finish the installation you have to move the following}
\Msg{* file into a directory searched by TeX:}
\Msg{*}
\Msg{*     hypbmsec.sty}
\Msg{*}
\Msg{* To produce the documentation run the file `hypbmsec.drv'}
\Msg{* through LaTeX.}
\Msg{*}
\Msg{* Happy TeXing!}
\Msg{*}
\Msg{************************************************************************}

\endbatchfile
%</install>
%<*ignore>
\fi
%</ignore>
%<*driver>
\NeedsTeXFormat{LaTeX2e}
\ProvidesFile{hypbmsec.drv}%
  [2016/05/16 v2.5 Bookmarks in sectioning commands (HO)]%
\documentclass{ltxdoc}
\usepackage{holtxdoc}[2011/11/22]
\begin{document}
  \DocInput{hypbmsec.dtx}%
\end{document}
%</driver>
% \fi
%
%
% \CharacterTable
%  {Upper-case    \A\B\C\D\E\F\G\H\I\J\K\L\M\N\O\P\Q\R\S\T\U\V\W\X\Y\Z
%   Lower-case    \a\b\c\d\e\f\g\h\i\j\k\l\m\n\o\p\q\r\s\t\u\v\w\x\y\z
%   Digits        \0\1\2\3\4\5\6\7\8\9
%   Exclamation   \!     Double quote  \"     Hash (number) \#
%   Dollar        \$     Percent       \%     Ampersand     \&
%   Acute accent  \'     Left paren    \(     Right paren   \)
%   Asterisk      \*     Plus          \+     Comma         \,
%   Minus         \-     Point         \.     Solidus       \/
%   Colon         \:     Semicolon     \;     Less than     \<
%   Equals        \=     Greater than  \>     Question mark \?
%   Commercial at \@     Left bracket  \[     Backslash     \\
%   Right bracket \]     Circumflex    \^     Underscore    \_
%   Grave accent  \`     Left brace    \{     Vertical bar  \|
%   Right brace   \}     Tilde         \~}
%
% \GetFileInfo{hypbmsec.drv}
%
% \title{The \xpackage{hypbmsec} package}
% \date{2016/05/16 v2.5}
% \author{Heiko Oberdiek\thanks
% {Please report any issues at https://github.com/ho-tex/oberdiek/issues}\\
% \xemail{heiko.oberdiek at googlemail.com}}
%
% \maketitle
%
% \begin{abstract}
% This package expands the syntax of the sectioning commands. If the
% argument of the sectioning commands isn't usable as outline entry,
% a replacement for the bookmarks can be given.
% \end{abstract}
%
% \tableofcontents
%
% \newcommand{\type}[1]{\textsf{#1}}
%
% ^^A No thread support.
% \newenvironment{article}[1]{}{}
%
% \section{Usage}
%
% \subsection{Bookmarks restrictions}\label{sec:restrictions}
%    Outline entries (bookmarks) are written to a file and have
%    to obey the pdf specification.
%    Therefore they have several restrictions:
%    \begin{itemize}
%    \item Bookmarks have to be encoded in PDFDocEncoding^^A
%          \footnote{\Package{hyperref} doesn't support
%            Unicode.}.
%    \item They should only expand to letters and spaces.
%    \item The result of expansion have to be a valid pdf string.
%    \item Stomach commands like \cmd{\relax}, box commands, math,
%          assignments, or definitions aren't allowed.
%    \item Short entries are recommended, which allow a clear view.
%    \end{itemize}
%
% \subsection{\texorpdfstring{\cmd{\texorpdfstring}}{^^A
%    \textbackslash texorpdfstring}}
%    The generic way in package \Package{hyperref} is the use
%    of \cmd{\texorpdfstring}^^A
%    \footnote{In versions of \Package{hyperref} below 6.54 see
%      \cmd{\ifbookmark}.}:
%    \begin{quote}
%\begin{verbatim}
%\section{Pythagoras:
%  \texorpdfstring{$a^2+b^2=c^2}{%
%    a\texttwosuperior\ + b\texttwosuperior\ =
%    c\texttwosuperior}%
%}
%\end{verbatim}
%    \end{quote}
%
% \subsection{Sectioning commands}
%    The package \Package{hyperref} automatically generates
%    bookmarks from the sectioning commands,
%    unless it is suppressed by an option.
%    Commands that structure the text are here called
%    ``sectioning commands'':
%    \begin{quote}
%    \cmd{\part}, \cmd{\chapter},\\
%    \cmd{\section}, \cmd{\subsection}, \cmd{\subsubsection},\\
%    \cmd{\paragraph}, \cmd{\subparagraph}
%    \end{quote}
%
% \subsection{Places\texorpdfstring{ for sectioning strings}{}}
%    \label{sec:places}
%    The argument(s) of these commands are used on several places:
%    \begin{description}
%    \item[\type{text}]
%      The current text without restrictions.
%    \item[\type{toc}]
%      The headlines and the table of contents with the
%      restrictions of ``moving arguments''.
%    \item[\type{out}]
%      The outlines with many restrictions: The outline
%      have to expand to a valid pdf string with PDFDocEncoding
%      (see \ref{sec:restrictions}).
%    \end{description}
%
% \subsection{\texorpdfstring{Solution with o}{O}ptional arguments}
%    If the user wants to use a footnote within a sectioning command,
%    the \LaTeX{} solution is an optional argument:
%    \begin{quote}
%      |\section[Title]{Title\footnote{Footnote text}}|
%    \end{quote}
%    Now |Title| without the footnote is used in the headlines and
%    the table of contents. Also \Package{hyperref} uses it for the
%    bookmarks.
%
%    This package \Package{\filename} offers two possibilities to
%    specify a separate outline entry:
%    \begin{itemize}
%    \item An additional second optional argument in square brackets.
%    \item An additional optional argument in parentheses (in
%          assoziation with a pdf string that is internally surrounded
%          by parentheses, too).
%    \end{itemize}
%    Because \Package{\filename} stores the original meaning of the
%    sectioning commands and uses them again, there should be no
%    problems with packages that redefine the sectioning commands, if
%    these packages doesn't change the syntax.
%
% \subsection{Syntax}
%    The following examples show the syntax of the sectioning
%    commands. For the places the strings appear the abbreviations
%    are used, that are introduced in \ref{sec:places}.
%
% \subsubsection{\texorpdfstring{Star form}{^^A
%    \textbackslash section*\{\}}}
%    The behaviour of the star form isn't changed. The string
%    appears only in the current text:
%    \begin{article}{Syntax}
%    \begin{quote}
%      |\section*{text}|
%    \end{quote}
%    \end{article}
%
% \subsubsection{\texorpdfstring{Without optional arguments}{^^A
%    \textbackslash section\{\}}}
%    The normal case, the string in the mandatory argument is
%    used for all places:
%    \begin{article}{Syntax}
%    \begin{quote}
%      |\section{text, toc, out}|
%    \end{quote}
%    \end{article}
%
% \subsubsection{\texorpdfstring{One optional argument}{^^A
%    \textbackslash section[]\{\}}}
%    Also the form with one optional parameter in square brackets isn't
%    new; for the bookmarks the optional parameter is used:
%    \begin{article}{Syntax}
%    \begin{quote}
%      |\section[toc, out]{text}|
%    \end{quote}
%    \end{article}
%
% \subsubsection{\texorpdfstring{Two optional arguments}{^^A
%    \textbackslash section[][out]\{\}}}\label{sec:two}
%    The second optional parameter in square brackets is introduced
%    by this package to specify an outline entry:
%    \begin{article}{Syntax}
%    \begin{quote}
%      |\section[toc][out]{text}|
%    \end{quote}
%    \end{article}
%
% \subsubsection{\texorpdfstring{Optional argument in parentheses}{^^A
%    \textbackslash section(out)\{\}}}
%    Often the \type{toc} and the \type{text} string would be the same.
%    With the method of the two optional arguments in square brackets
%    (see \ref{sec:two}) this string must be given twice,
%    if the user only wants to specify a different outline entry.
%    Therefore this package offers another possibility:
%    In association with the internal representation in the pdf file
%    an outline entry can be given in parentheses.
%    So the package can easily distinguish between
%    the two forms of optional arguments and the order does not matter:
%    \begin{article}{Syntax}
%    \begin{quote}
%      |\section(out){toc, text}|\\
%      |\section[toc](out){text}|\\
%      |\section(out)[toc]{text}|
%    \end{quote}
%    \end{article}
%
% \subsection{Without \Package{hyperref}}
%    Package \Package{\filename} uses \Package{hyperref} for support of
%    the bookmarks, but this package is not required.
%    If \Package{hyperref} isn't loaded, or
%    is called with a driver that doesn't support bookmarks,
%    package \Package{\filename} shouldn't be removed,
%    because this would lead to
%    a wrong syntax of the sectioning commands.
%    In any cases package \Package{\filename}
%    supports its syntax and ignores the outline entries,
%    if there are no code for bookmarks.
%    So it is possible to write texts,
%    that are processed with several drivers to get different output
%    formats.
%
% \subsection{Protecting parentheses}
%    If the string itself contains parentheses, they have to be hidden
%    from \TeX's argument parsing mechanism.
%    The argument should be surrounded
%    by curly braces:
%    \begin{quote}
%      |\section({outlines(bookmarks)}){text, toc}|
%    \end{quote}
%    With version 6.54 of \Package{hyperref} the other standard method
%    works, too: The closing parenthesis is protected:
%    \begin{quote}
%      |\section(outlines(bookmarks{)}){text, toc}|
%    \end{quote}
%
% \StopEventually{
% }
%
% \section{Implementation}
%    \begin{macrocode}
%<*package>
%    \end{macrocode}
%    Package identification.
%    \begin{macrocode}
\NeedsTeXFormat{LaTeX2e}
\ProvidesPackage{hypbmsec}%
  [2016/05/16 v2.5 Bookmarks in sectioning commands (HO)]
%    \end{macrocode}
%
%    Because of redifining the sectioning commands, it is dangerous
%    to reload the package several times.
%    \begin{macrocode}
\@ifundefined{hbs@do}{}{%
  \PackageInfo{hypbmsec}{Package 'hypbmsec' is already loaded}%
  \endinput
}
%    \end{macrocode}
%
%    \begin{macro}{\hbs@do}
%    The redefined sectioning commands calls \cmd{\hbs@do}. It does
%    \begin{itemize}
%    \item handle the star case.
%    \item resets the macros that store the entries for the outlines
%          (\cmd{\hbs@bmstring}) and table of contents (\cmd{\hbs@tocstring}).
%    \item store the sectioning command |#1| in \cmd{\hbs@seccmd}
%          for later reuse.
%    \item at last call \cmd{\hbs@checkarg} that scans and interprets the
%          parameters of the redefined sectioning command.
%    \end{itemize}
%    \begin{macrocode}
\def\hbs@do#1{%
  \@ifstar{#1*}{%
    \let\hbs@tocstring\relax
    \let\hbs@bmstring\relax
    \let\hbs@seccmd#1%
    \hbs@checkarg
  }%
}
%    \end{macrocode}
%    \end{macro}
%
%    \begin{macro}{\hbs@checkarg}
%    \cmd{\hbs@checkarg} determines the type of the next argument:
%    \begin{itemize}
%    \item An optional argument in square brackets can be an entry
%          for the table of contents or the bookmarks. It will be
%          read by \cmd{\hbs@getsquare}
%    \item An optional argument in parentheses is an outline entry.
%          This is worked off by \cmd{\hbs@getbookmark}.
%    \item If there are no more optional arguments, \cmd{\hbs@process}
%          reads the mandatory argument and calls the original
%          sectioning commands.
%    \end{itemize}
%    \begin{macrocode}
\def\hbs@checkarg{%
  \@ifnextchar[\hbs@getsquare{%
    \@ifnextchar(\hbs@getbookmark\hbs@process
  }%
}
%    \end{macrocode}
%    \end{macro}
%
%    \begin{macro}{\hbs@getsquare}
%    \cmd{\hbs@getsquare} reads an optional argument in square
%    brackets and determines, if this is an entry for the table
%    of contents or the bookmarks.
%    \begin{macrocode}
\long\def\hbs@getsquare[#1]{%
  \ifx\hbs@tocstring\relax
    \def\hbs@tocstring{#1}%
  \else
    \hbs@bmdef{#1}%
  \fi
  \hbs@checkarg
}
%    \end{macrocode}
%    \end{macro}
%
%    \begin{macro}{\hbs@getbookmark}
%    \cmd{\hbs@getbookmark} reads an outline entry in parentheses.
%    \begin{macrocode}
\def\hbs@getbookmark(#1){%
  \hbs@bmdef{#1}%
  \hbs@checkarg
}
%    \end{macrocode}
%    \end{macro}
%
%    \begin{macro}{\hbs@bmdef}
%    The command \cmd{\hbs@bmdef} save the bookmark entry in
%    parameter |#1| in the macro \cmd{\hbs@bmstring} and catches
%    the case, if the user has given several outline strings.
%    \begin{macrocode}
\def\hbs@bmdef#1{%
  \ifx\hbs@bmstring\relax
    \def\hbs@bmstring{#1}%
  \else
    \PackageError{hypbmsec}{%
      Sectioning command with too many parameters%
    }{%
      You can only give one outline entry.%
    }%
  \fi
}
%    \end{macrocode}
%    \end{macro}
%
%    \begin{macro}{\hbs@process}
%    The parameter |#1| is the mandatory argument of the sectioning
%    commands. \cmd{\hbs@process} calls the original sectioning command
%    stored in \cmd{\hbs@seccmd} with arguments that depend of which
%    optional argument are used previously.
%    \begin{macrocode}
\long\def\hbs@process#1{%
  \ifx\hbs@tocstring\relax
    \ifx\hbs@bmstring\relax
      \hbs@seccmd{#1}%
    \else
      \begingroup
        \def\x##1{\endgroup
          \hbs@seccmd{\texorpdfstring{#1}{##1}}%
        }%
      \expandafter\x\expandafter{\hbs@bmstring}%
    \fi
  \else
    \ifx\hbs@bmstring\relax
      \expandafter\hbs@seccmd\expandafter[%
        \expandafter{\hbs@tocstring}%
      ]{#1}%
    \else
      \expandafter\expandafter\expandafter
      \hbs@seccmd\expandafter\expandafter\expandafter[%
        \expandafter\expandafter\expandafter
        \texorpdfstring
        \expandafter\expandafter\expandafter{%
          \expandafter\hbs@tocstring\expandafter
        }\expandafter{%
          \hbs@bmstring
        }%
      ]{#1}%
    \fi
  \fi
}
%    \end{macrocode}
%    \end{macro}
%
%    We have to check, whether package \Package{hyperref} is loaded
%    and have to provide a definition for \cmd{\texorpdfstring}.
%    Because \Package{hyperref} can be loaded after this package,
%    we do the work later (\cmd{\AtBeginDocument}).
%
%    This code only checks versions of \Package{hyperref} that
%    define \cmd{\ifbookmark} (v6.4x until v6.53) or
%    \cmd{\texorpdfstring} (v6.54 and above). Older versions aren't
%    supported.
%    \begin{macrocode}
\AtBeginDocument{%
  \@ifundefined{texorpdfstring}{%
    \@ifundefined{ifbookmark}{%
      \let\texorpdfstring\@firstoftwo
      \@ifpackageloaded{hyperref}{%
        \PackageInfo{hypbmsec}{%
          \ifx\hy@driver\@empty
            Default driver %
          \else
            '\hy@driver' %
          \fi
          of hyperref not supported,\MessageBreak
          bookmark parameters will be ignored%
        }%
      }{%
        \PackageInfo{hypbmsec}{%
          Package hyperref not loaded,\MessageBreak
          bookmark parameters will be ignored%
        }%
      }%
    }%
    {%
      \newcommand\texorpdfstring[2]{\ifbookmark{#2}{#1}}%
      \PackageWarningNoLine{hypbmsec}{%
        Old hyperref version found,\MessageBreak
        update of hyperref recommended%
      }%
    }%
  }{}%
%    \end{macrocode}
%
%    Other packages are allowed to redefine the sectioning commands,
%    if they does not change the syntax. Therefore the redefinitons
%    of this package should be done after the other packages.
%    \begin{macrocode}
  \let\hbs@part         \part
  \let\hbs@section      \section
  \let\hbs@subsection   \subsection
  \let\hbs@subsubsection\subsubsection
  \let\hbs@paragraph    \paragraph
  \let\hbs@subparagraph \subparagraph
  \renewcommand\part         {\hbs@do\hbs@part}%
  \renewcommand\section      {\hbs@do\hbs@section}%
  \renewcommand\subsection   {\hbs@do\hbs@subsection}%
  \renewcommand\subsubsection{\hbs@do\hbs@subsubsection}%
  \renewcommand\paragraph    {\hbs@do\hbs@paragraph}%
  \renewcommand\subparagraph {\hbs@do\hbs@subparagraph}%
  \begingroup\expandafter\expandafter\expandafter\endgroup
  \expandafter\ifx\csname chapter\endcsname\relax\else
    \let\hbs@chapter    \chapter
    \renewcommand\chapter    {\hbs@do\hbs@chapter}%
  \fi
}
%    \end{macrocode}
%
%    \begin{macrocode}
%</package>
%    \end{macrocode}
%
% \section{Installation}
%
% \subsection{Download}
%
% \paragraph{Package.} This package is available on
% CTAN\footnote{\url{http://ctan.org/pkg/hypbmsec}}:
% \begin{description}
% \item[\CTAN{macros/latex/contrib/oberdiek/hypbmsec.dtx}] The source file.
% \item[\CTAN{macros/latex/contrib/oberdiek/hypbmsec.pdf}] Documentation.
% \end{description}
%
%
% \paragraph{Bundle.} All the packages of the bundle `oberdiek'
% are also available in a TDS compliant ZIP archive. There
% the packages are already unpacked and the documentation files
% are generated. The files and directories obey the TDS standard.
% \begin{description}
% \item[\CTAN{install/macros/latex/contrib/oberdiek.tds.zip}]
% \end{description}
% \emph{TDS} refers to the standard ``A Directory Structure
% for \TeX\ Files'' (\CTAN{tds/tds.pdf}). Directories
% with \xfile{texmf} in their name are usually organized this way.
%
% \subsection{Bundle installation}
%
% \paragraph{Unpacking.} Unpack the \xfile{oberdiek.tds.zip} in the
% TDS tree (also known as \xfile{texmf} tree) of your choice.
% Example (linux):
% \begin{quote}
%   |unzip oberdiek.tds.zip -d ~/texmf|
% \end{quote}
%
% \paragraph{Script installation.}
% Check the directory \xfile{TDS:scripts/oberdiek/} for
% scripts that need further installation steps.
% Package \xpackage{attachfile2} comes with the Perl script
% \xfile{pdfatfi.pl} that should be installed in such a way
% that it can be called as \texttt{pdfatfi}.
% Example (linux):
% \begin{quote}
%   |chmod +x scripts/oberdiek/pdfatfi.pl|\\
%   |cp scripts/oberdiek/pdfatfi.pl /usr/local/bin/|
% \end{quote}
%
% \subsection{Package installation}
%
% \paragraph{Unpacking.} The \xfile{.dtx} file is a self-extracting
% \docstrip\ archive. The files are extracted by running the
% \xfile{.dtx} through \plainTeX:
% \begin{quote}
%   \verb|tex hypbmsec.dtx|
% \end{quote}
%
% \paragraph{TDS.} Now the different files must be moved into
% the different directories in your installation TDS tree
% (also known as \xfile{texmf} tree):
% \begin{quote}
% \def\t{^^A
% \begin{tabular}{@{}>{\ttfamily}l@{ $\rightarrow$ }>{\ttfamily}l@{}}
%   hypbmsec.sty & tex/latex/oberdiek/hypbmsec.sty\\
%   hypbmsec.pdf & doc/latex/oberdiek/hypbmsec.pdf\\
%   hypbmsec.dtx & source/latex/oberdiek/hypbmsec.dtx\\
% \end{tabular}^^A
% }^^A
% \sbox0{\t}^^A
% \ifdim\wd0>\linewidth
%   \begingroup
%     \advance\linewidth by\leftmargin
%     \advance\linewidth by\rightmargin
%   \edef\x{\endgroup
%     \def\noexpand\lw{\the\linewidth}^^A
%   }\x
%   \def\lwbox{^^A
%     \leavevmode
%     \hbox to \linewidth{^^A
%       \kern-\leftmargin\relax
%       \hss
%       \usebox0
%       \hss
%       \kern-\rightmargin\relax
%     }^^A
%   }^^A
%   \ifdim\wd0>\lw
%     \sbox0{\small\t}^^A
%     \ifdim\wd0>\linewidth
%       \ifdim\wd0>\lw
%         \sbox0{\footnotesize\t}^^A
%         \ifdim\wd0>\linewidth
%           \ifdim\wd0>\lw
%             \sbox0{\scriptsize\t}^^A
%             \ifdim\wd0>\linewidth
%               \ifdim\wd0>\lw
%                 \sbox0{\tiny\t}^^A
%                 \ifdim\wd0>\linewidth
%                   \lwbox
%                 \else
%                   \usebox0
%                 \fi
%               \else
%                 \lwbox
%               \fi
%             \else
%               \usebox0
%             \fi
%           \else
%             \lwbox
%           \fi
%         \else
%           \usebox0
%         \fi
%       \else
%         \lwbox
%       \fi
%     \else
%       \usebox0
%     \fi
%   \else
%     \lwbox
%   \fi
% \else
%   \usebox0
% \fi
% \end{quote}
% If you have a \xfile{docstrip.cfg} that configures and enables \docstrip's
% TDS installing feature, then some files can already be in the right
% place, see the documentation of \docstrip.
%
% \subsection{Refresh file name databases}
%
% If your \TeX~distribution
% (\teTeX, \mikTeX, \dots) relies on file name databases, you must refresh
% these. For example, \teTeX\ users run \verb|texhash| or
% \verb|mktexlsr|.
%
% \subsection{Some details for the interested}
%
% \paragraph{Attached source.}
%
% The PDF documentation on CTAN also includes the
% \xfile{.dtx} source file. It can be extracted by
% AcrobatReader 6 or higher. Another option is \textsf{pdftk},
% e.g. unpack the file into the current directory:
% \begin{quote}
%   \verb|pdftk hypbmsec.pdf unpack_files output .|
% \end{quote}
%
% \paragraph{Unpacking with \LaTeX.}
% The \xfile{.dtx} chooses its action depending on the format:
% \begin{description}
% \item[\plainTeX:] Run \docstrip\ and extract the files.
% \item[\LaTeX:] Generate the documentation.
% \end{description}
% If you insist on using \LaTeX\ for \docstrip\ (really,
% \docstrip\ does not need \LaTeX), then inform the autodetect routine
% about your intention:
% \begin{quote}
%   \verb|latex \let\install=y% \iffalse meta-comment
%
% File: hypbmsec.dtx
% Version: 2016/05/16 v2.5
% Info: Bookmarks in sectioning commands
%
% Copyright (C) 1998-2000, 2006, 2007 by
%    Heiko Oberdiek <heiko.oberdiek at googlemail.com>
%    2016
%    https://github.com/ho-tex/oberdiek/issues
%
% This work may be distributed and/or modified under the
% conditions of the LaTeX Project Public License, either
% version 1.3c of this license or (at your option) any later
% version. This version of this license is in
%    http://www.latex-project.org/lppl/lppl-1-3c.txt
% and the latest version of this license is in
%    http://www.latex-project.org/lppl.txt
% and version 1.3 or later is part of all distributions of
% LaTeX version 2005/12/01 or later.
%
% This work has the LPPL maintenance status "maintained".
%
% This Current Maintainer of this work is Heiko Oberdiek.
%
% This work consists of the main source file hypbmsec.dtx
% and the derived files
%    hypbmsec.sty, hypbmsec.pdf, hypbmsec.ins, hypbmsec.drv.
%
% Distribution:
%    CTAN:macros/latex/contrib/oberdiek/hypbmsec.dtx
%    CTAN:macros/latex/contrib/oberdiek/hypbmsec.pdf
%
% Unpacking:
%    (a) If hypbmsec.ins is present:
%           tex hypbmsec.ins
%    (b) Without hypbmsec.ins:
%           tex hypbmsec.dtx
%    (c) If you insist on using LaTeX
%           latex \let\install=y\input{hypbmsec.dtx}
%        (quote the arguments according to the demands of your shell)
%
% Documentation:
%    (a) If hypbmsec.drv is present:
%           latex hypbmsec.drv
%    (b) Without hypbmsec.drv:
%           latex hypbmsec.dtx; ...
%    The class ltxdoc loads the configuration file ltxdoc.cfg
%    if available. Here you can specify further options, e.g.
%    use A4 as paper format:
%       \PassOptionsToClass{a4paper}{article}
%
%    Programm calls to get the documentation (example):
%       pdflatex hypbmsec.dtx
%       makeindex -s gind.ist hypbmsec.idx
%       pdflatex hypbmsec.dtx
%       makeindex -s gind.ist hypbmsec.idx
%       pdflatex hypbmsec.dtx
%
% Installation:
%    TDS:tex/latex/oberdiek/hypbmsec.sty
%    TDS:doc/latex/oberdiek/hypbmsec.pdf
%    TDS:source/latex/oberdiek/hypbmsec.dtx
%
%<*ignore>
\begingroup
  \catcode123=1 %
  \catcode125=2 %
  \def\x{LaTeX2e}%
\expandafter\endgroup
\ifcase 0\ifx\install y1\fi\expandafter
         \ifx\csname processbatchFile\endcsname\relax\else1\fi
         \ifx\fmtname\x\else 1\fi\relax
\else\csname fi\endcsname
%</ignore>
%<*install>
\input docstrip.tex
\Msg{************************************************************************}
\Msg{* Installation}
\Msg{* Package: hypbmsec 2016/05/16 v2.5 Bookmarks in sectioning commands (HO)}
\Msg{************************************************************************}

\keepsilent
\askforoverwritefalse

\let\MetaPrefix\relax
\preamble

This is a generated file.

Project: hypbmsec
Version: 2016/05/16 v2.5

Copyright (C) 1998-2000, 2006, 2007 by
   Heiko Oberdiek <heiko.oberdiek at googlemail.com>

This work may be distributed and/or modified under the
conditions of the LaTeX Project Public License, either
version 1.3c of this license or (at your option) any later
version. This version of this license is in
   http://www.latex-project.org/lppl/lppl-1-3c.txt
and the latest version of this license is in
   http://www.latex-project.org/lppl.txt
and version 1.3 or later is part of all distributions of
LaTeX version 2005/12/01 or later.

This work has the LPPL maintenance status "maintained".

This Current Maintainer of this work is Heiko Oberdiek.

This work consists of the main source file hypbmsec.dtx
and the derived files
   hypbmsec.sty, hypbmsec.pdf, hypbmsec.ins, hypbmsec.drv.

\endpreamble
\let\MetaPrefix\DoubleperCent

\generate{%
  \file{hypbmsec.ins}{\from{hypbmsec.dtx}{install}}%
  \file{hypbmsec.drv}{\from{hypbmsec.dtx}{driver}}%
  \usedir{tex/latex/oberdiek}%
  \file{hypbmsec.sty}{\from{hypbmsec.dtx}{package}}%
  \nopreamble
  \nopostamble
  \usedir{source/latex/oberdiek/catalogue}%
  \file{hypbmsec.xml}{\from{hypbmsec.dtx}{catalogue}}%
}

\catcode32=13\relax% active space
\let =\space%
\Msg{************************************************************************}
\Msg{*}
\Msg{* To finish the installation you have to move the following}
\Msg{* file into a directory searched by TeX:}
\Msg{*}
\Msg{*     hypbmsec.sty}
\Msg{*}
\Msg{* To produce the documentation run the file `hypbmsec.drv'}
\Msg{* through LaTeX.}
\Msg{*}
\Msg{* Happy TeXing!}
\Msg{*}
\Msg{************************************************************************}

\endbatchfile
%</install>
%<*ignore>
\fi
%</ignore>
%<*driver>
\NeedsTeXFormat{LaTeX2e}
\ProvidesFile{hypbmsec.drv}%
  [2016/05/16 v2.5 Bookmarks in sectioning commands (HO)]%
\documentclass{ltxdoc}
\usepackage{holtxdoc}[2011/11/22]
\begin{document}
  \DocInput{hypbmsec.dtx}%
\end{document}
%</driver>
% \fi
%
%
% \CharacterTable
%  {Upper-case    \A\B\C\D\E\F\G\H\I\J\K\L\M\N\O\P\Q\R\S\T\U\V\W\X\Y\Z
%   Lower-case    \a\b\c\d\e\f\g\h\i\j\k\l\m\n\o\p\q\r\s\t\u\v\w\x\y\z
%   Digits        \0\1\2\3\4\5\6\7\8\9
%   Exclamation   \!     Double quote  \"     Hash (number) \#
%   Dollar        \$     Percent       \%     Ampersand     \&
%   Acute accent  \'     Left paren    \(     Right paren   \)
%   Asterisk      \*     Plus          \+     Comma         \,
%   Minus         \-     Point         \.     Solidus       \/
%   Colon         \:     Semicolon     \;     Less than     \<
%   Equals        \=     Greater than  \>     Question mark \?
%   Commercial at \@     Left bracket  \[     Backslash     \\
%   Right bracket \]     Circumflex    \^     Underscore    \_
%   Grave accent  \`     Left brace    \{     Vertical bar  \|
%   Right brace   \}     Tilde         \~}
%
% \GetFileInfo{hypbmsec.drv}
%
% \title{The \xpackage{hypbmsec} package}
% \date{2016/05/16 v2.5}
% \author{Heiko Oberdiek\thanks
% {Please report any issues at https://github.com/ho-tex/oberdiek/issues}\\
% \xemail{heiko.oberdiek at googlemail.com}}
%
% \maketitle
%
% \begin{abstract}
% This package expands the syntax of the sectioning commands. If the
% argument of the sectioning commands isn't usable as outline entry,
% a replacement for the bookmarks can be given.
% \end{abstract}
%
% \tableofcontents
%
% \newcommand{\type}[1]{\textsf{#1}}
%
% ^^A No thread support.
% \newenvironment{article}[1]{}{}
%
% \section{Usage}
%
% \subsection{Bookmarks restrictions}\label{sec:restrictions}
%    Outline entries (bookmarks) are written to a file and have
%    to obey the pdf specification.
%    Therefore they have several restrictions:
%    \begin{itemize}
%    \item Bookmarks have to be encoded in PDFDocEncoding^^A
%          \footnote{\Package{hyperref} doesn't support
%            Unicode.}.
%    \item They should only expand to letters and spaces.
%    \item The result of expansion have to be a valid pdf string.
%    \item Stomach commands like \cmd{\relax}, box commands, math,
%          assignments, or definitions aren't allowed.
%    \item Short entries are recommended, which allow a clear view.
%    \end{itemize}
%
% \subsection{\texorpdfstring{\cmd{\texorpdfstring}}{^^A
%    \textbackslash texorpdfstring}}
%    The generic way in package \Package{hyperref} is the use
%    of \cmd{\texorpdfstring}^^A
%    \footnote{In versions of \Package{hyperref} below 6.54 see
%      \cmd{\ifbookmark}.}:
%    \begin{quote}
%\begin{verbatim}
%\section{Pythagoras:
%  \texorpdfstring{$a^2+b^2=c^2}{%
%    a\texttwosuperior\ + b\texttwosuperior\ =
%    c\texttwosuperior}%
%}
%\end{verbatim}
%    \end{quote}
%
% \subsection{Sectioning commands}
%    The package \Package{hyperref} automatically generates
%    bookmarks from the sectioning commands,
%    unless it is suppressed by an option.
%    Commands that structure the text are here called
%    ``sectioning commands'':
%    \begin{quote}
%    \cmd{\part}, \cmd{\chapter},\\
%    \cmd{\section}, \cmd{\subsection}, \cmd{\subsubsection},\\
%    \cmd{\paragraph}, \cmd{\subparagraph}
%    \end{quote}
%
% \subsection{Places\texorpdfstring{ for sectioning strings}{}}
%    \label{sec:places}
%    The argument(s) of these commands are used on several places:
%    \begin{description}
%    \item[\type{text}]
%      The current text without restrictions.
%    \item[\type{toc}]
%      The headlines and the table of contents with the
%      restrictions of ``moving arguments''.
%    \item[\type{out}]
%      The outlines with many restrictions: The outline
%      have to expand to a valid pdf string with PDFDocEncoding
%      (see \ref{sec:restrictions}).
%    \end{description}
%
% \subsection{\texorpdfstring{Solution with o}{O}ptional arguments}
%    If the user wants to use a footnote within a sectioning command,
%    the \LaTeX{} solution is an optional argument:
%    \begin{quote}
%      |\section[Title]{Title\footnote{Footnote text}}|
%    \end{quote}
%    Now |Title| without the footnote is used in the headlines and
%    the table of contents. Also \Package{hyperref} uses it for the
%    bookmarks.
%
%    This package \Package{\filename} offers two possibilities to
%    specify a separate outline entry:
%    \begin{itemize}
%    \item An additional second optional argument in square brackets.
%    \item An additional optional argument in parentheses (in
%          assoziation with a pdf string that is internally surrounded
%          by parentheses, too).
%    \end{itemize}
%    Because \Package{\filename} stores the original meaning of the
%    sectioning commands and uses them again, there should be no
%    problems with packages that redefine the sectioning commands, if
%    these packages doesn't change the syntax.
%
% \subsection{Syntax}
%    The following examples show the syntax of the sectioning
%    commands. For the places the strings appear the abbreviations
%    are used, that are introduced in \ref{sec:places}.
%
% \subsubsection{\texorpdfstring{Star form}{^^A
%    \textbackslash section*\{\}}}
%    The behaviour of the star form isn't changed. The string
%    appears only in the current text:
%    \begin{article}{Syntax}
%    \begin{quote}
%      |\section*{text}|
%    \end{quote}
%    \end{article}
%
% \subsubsection{\texorpdfstring{Without optional arguments}{^^A
%    \textbackslash section\{\}}}
%    The normal case, the string in the mandatory argument is
%    used for all places:
%    \begin{article}{Syntax}
%    \begin{quote}
%      |\section{text, toc, out}|
%    \end{quote}
%    \end{article}
%
% \subsubsection{\texorpdfstring{One optional argument}{^^A
%    \textbackslash section[]\{\}}}
%    Also the form with one optional parameter in square brackets isn't
%    new; for the bookmarks the optional parameter is used:
%    \begin{article}{Syntax}
%    \begin{quote}
%      |\section[toc, out]{text}|
%    \end{quote}
%    \end{article}
%
% \subsubsection{\texorpdfstring{Two optional arguments}{^^A
%    \textbackslash section[][out]\{\}}}\label{sec:two}
%    The second optional parameter in square brackets is introduced
%    by this package to specify an outline entry:
%    \begin{article}{Syntax}
%    \begin{quote}
%      |\section[toc][out]{text}|
%    \end{quote}
%    \end{article}
%
% \subsubsection{\texorpdfstring{Optional argument in parentheses}{^^A
%    \textbackslash section(out)\{\}}}
%    Often the \type{toc} and the \type{text} string would be the same.
%    With the method of the two optional arguments in square brackets
%    (see \ref{sec:two}) this string must be given twice,
%    if the user only wants to specify a different outline entry.
%    Therefore this package offers another possibility:
%    In association with the internal representation in the pdf file
%    an outline entry can be given in parentheses.
%    So the package can easily distinguish between
%    the two forms of optional arguments and the order does not matter:
%    \begin{article}{Syntax}
%    \begin{quote}
%      |\section(out){toc, text}|\\
%      |\section[toc](out){text}|\\
%      |\section(out)[toc]{text}|
%    \end{quote}
%    \end{article}
%
% \subsection{Without \Package{hyperref}}
%    Package \Package{\filename} uses \Package{hyperref} for support of
%    the bookmarks, but this package is not required.
%    If \Package{hyperref} isn't loaded, or
%    is called with a driver that doesn't support bookmarks,
%    package \Package{\filename} shouldn't be removed,
%    because this would lead to
%    a wrong syntax of the sectioning commands.
%    In any cases package \Package{\filename}
%    supports its syntax and ignores the outline entries,
%    if there are no code for bookmarks.
%    So it is possible to write texts,
%    that are processed with several drivers to get different output
%    formats.
%
% \subsection{Protecting parentheses}
%    If the string itself contains parentheses, they have to be hidden
%    from \TeX's argument parsing mechanism.
%    The argument should be surrounded
%    by curly braces:
%    \begin{quote}
%      |\section({outlines(bookmarks)}){text, toc}|
%    \end{quote}
%    With version 6.54 of \Package{hyperref} the other standard method
%    works, too: The closing parenthesis is protected:
%    \begin{quote}
%      |\section(outlines(bookmarks{)}){text, toc}|
%    \end{quote}
%
% \StopEventually{
% }
%
% \section{Implementation}
%    \begin{macrocode}
%<*package>
%    \end{macrocode}
%    Package identification.
%    \begin{macrocode}
\NeedsTeXFormat{LaTeX2e}
\ProvidesPackage{hypbmsec}%
  [2016/05/16 v2.5 Bookmarks in sectioning commands (HO)]
%    \end{macrocode}
%
%    Because of redifining the sectioning commands, it is dangerous
%    to reload the package several times.
%    \begin{macrocode}
\@ifundefined{hbs@do}{}{%
  \PackageInfo{hypbmsec}{Package 'hypbmsec' is already loaded}%
  \endinput
}
%    \end{macrocode}
%
%    \begin{macro}{\hbs@do}
%    The redefined sectioning commands calls \cmd{\hbs@do}. It does
%    \begin{itemize}
%    \item handle the star case.
%    \item resets the macros that store the entries for the outlines
%          (\cmd{\hbs@bmstring}) and table of contents (\cmd{\hbs@tocstring}).
%    \item store the sectioning command |#1| in \cmd{\hbs@seccmd}
%          for later reuse.
%    \item at last call \cmd{\hbs@checkarg} that scans and interprets the
%          parameters of the redefined sectioning command.
%    \end{itemize}
%    \begin{macrocode}
\def\hbs@do#1{%
  \@ifstar{#1*}{%
    \let\hbs@tocstring\relax
    \let\hbs@bmstring\relax
    \let\hbs@seccmd#1%
    \hbs@checkarg
  }%
}
%    \end{macrocode}
%    \end{macro}
%
%    \begin{macro}{\hbs@checkarg}
%    \cmd{\hbs@checkarg} determines the type of the next argument:
%    \begin{itemize}
%    \item An optional argument in square brackets can be an entry
%          for the table of contents or the bookmarks. It will be
%          read by \cmd{\hbs@getsquare}
%    \item An optional argument in parentheses is an outline entry.
%          This is worked off by \cmd{\hbs@getbookmark}.
%    \item If there are no more optional arguments, \cmd{\hbs@process}
%          reads the mandatory argument and calls the original
%          sectioning commands.
%    \end{itemize}
%    \begin{macrocode}
\def\hbs@checkarg{%
  \@ifnextchar[\hbs@getsquare{%
    \@ifnextchar(\hbs@getbookmark\hbs@process
  }%
}
%    \end{macrocode}
%    \end{macro}
%
%    \begin{macro}{\hbs@getsquare}
%    \cmd{\hbs@getsquare} reads an optional argument in square
%    brackets and determines, if this is an entry for the table
%    of contents or the bookmarks.
%    \begin{macrocode}
\long\def\hbs@getsquare[#1]{%
  \ifx\hbs@tocstring\relax
    \def\hbs@tocstring{#1}%
  \else
    \hbs@bmdef{#1}%
  \fi
  \hbs@checkarg
}
%    \end{macrocode}
%    \end{macro}
%
%    \begin{macro}{\hbs@getbookmark}
%    \cmd{\hbs@getbookmark} reads an outline entry in parentheses.
%    \begin{macrocode}
\def\hbs@getbookmark(#1){%
  \hbs@bmdef{#1}%
  \hbs@checkarg
}
%    \end{macrocode}
%    \end{macro}
%
%    \begin{macro}{\hbs@bmdef}
%    The command \cmd{\hbs@bmdef} save the bookmark entry in
%    parameter |#1| in the macro \cmd{\hbs@bmstring} and catches
%    the case, if the user has given several outline strings.
%    \begin{macrocode}
\def\hbs@bmdef#1{%
  \ifx\hbs@bmstring\relax
    \def\hbs@bmstring{#1}%
  \else
    \PackageError{hypbmsec}{%
      Sectioning command with too many parameters%
    }{%
      You can only give one outline entry.%
    }%
  \fi
}
%    \end{macrocode}
%    \end{macro}
%
%    \begin{macro}{\hbs@process}
%    The parameter |#1| is the mandatory argument of the sectioning
%    commands. \cmd{\hbs@process} calls the original sectioning command
%    stored in \cmd{\hbs@seccmd} with arguments that depend of which
%    optional argument are used previously.
%    \begin{macrocode}
\long\def\hbs@process#1{%
  \ifx\hbs@tocstring\relax
    \ifx\hbs@bmstring\relax
      \hbs@seccmd{#1}%
    \else
      \begingroup
        \def\x##1{\endgroup
          \hbs@seccmd{\texorpdfstring{#1}{##1}}%
        }%
      \expandafter\x\expandafter{\hbs@bmstring}%
    \fi
  \else
    \ifx\hbs@bmstring\relax
      \expandafter\hbs@seccmd\expandafter[%
        \expandafter{\hbs@tocstring}%
      ]{#1}%
    \else
      \expandafter\expandafter\expandafter
      \hbs@seccmd\expandafter\expandafter\expandafter[%
        \expandafter\expandafter\expandafter
        \texorpdfstring
        \expandafter\expandafter\expandafter{%
          \expandafter\hbs@tocstring\expandafter
        }\expandafter{%
          \hbs@bmstring
        }%
      ]{#1}%
    \fi
  \fi
}
%    \end{macrocode}
%    \end{macro}
%
%    We have to check, whether package \Package{hyperref} is loaded
%    and have to provide a definition for \cmd{\texorpdfstring}.
%    Because \Package{hyperref} can be loaded after this package,
%    we do the work later (\cmd{\AtBeginDocument}).
%
%    This code only checks versions of \Package{hyperref} that
%    define \cmd{\ifbookmark} (v6.4x until v6.53) or
%    \cmd{\texorpdfstring} (v6.54 and above). Older versions aren't
%    supported.
%    \begin{macrocode}
\AtBeginDocument{%
  \@ifundefined{texorpdfstring}{%
    \@ifundefined{ifbookmark}{%
      \let\texorpdfstring\@firstoftwo
      \@ifpackageloaded{hyperref}{%
        \PackageInfo{hypbmsec}{%
          \ifx\hy@driver\@empty
            Default driver %
          \else
            '\hy@driver' %
          \fi
          of hyperref not supported,\MessageBreak
          bookmark parameters will be ignored%
        }%
      }{%
        \PackageInfo{hypbmsec}{%
          Package hyperref not loaded,\MessageBreak
          bookmark parameters will be ignored%
        }%
      }%
    }%
    {%
      \newcommand\texorpdfstring[2]{\ifbookmark{#2}{#1}}%
      \PackageWarningNoLine{hypbmsec}{%
        Old hyperref version found,\MessageBreak
        update of hyperref recommended%
      }%
    }%
  }{}%
%    \end{macrocode}
%
%    Other packages are allowed to redefine the sectioning commands,
%    if they does not change the syntax. Therefore the redefinitons
%    of this package should be done after the other packages.
%    \begin{macrocode}
  \let\hbs@part         \part
  \let\hbs@section      \section
  \let\hbs@subsection   \subsection
  \let\hbs@subsubsection\subsubsection
  \let\hbs@paragraph    \paragraph
  \let\hbs@subparagraph \subparagraph
  \renewcommand\part         {\hbs@do\hbs@part}%
  \renewcommand\section      {\hbs@do\hbs@section}%
  \renewcommand\subsection   {\hbs@do\hbs@subsection}%
  \renewcommand\subsubsection{\hbs@do\hbs@subsubsection}%
  \renewcommand\paragraph    {\hbs@do\hbs@paragraph}%
  \renewcommand\subparagraph {\hbs@do\hbs@subparagraph}%
  \begingroup\expandafter\expandafter\expandafter\endgroup
  \expandafter\ifx\csname chapter\endcsname\relax\else
    \let\hbs@chapter    \chapter
    \renewcommand\chapter    {\hbs@do\hbs@chapter}%
  \fi
}
%    \end{macrocode}
%
%    \begin{macrocode}
%</package>
%    \end{macrocode}
%
% \section{Installation}
%
% \subsection{Download}
%
% \paragraph{Package.} This package is available on
% CTAN\footnote{\url{http://ctan.org/pkg/hypbmsec}}:
% \begin{description}
% \item[\CTAN{macros/latex/contrib/oberdiek/hypbmsec.dtx}] The source file.
% \item[\CTAN{macros/latex/contrib/oberdiek/hypbmsec.pdf}] Documentation.
% \end{description}
%
%
% \paragraph{Bundle.} All the packages of the bundle `oberdiek'
% are also available in a TDS compliant ZIP archive. There
% the packages are already unpacked and the documentation files
% are generated. The files and directories obey the TDS standard.
% \begin{description}
% \item[\CTAN{install/macros/latex/contrib/oberdiek.tds.zip}]
% \end{description}
% \emph{TDS} refers to the standard ``A Directory Structure
% for \TeX\ Files'' (\CTAN{tds/tds.pdf}). Directories
% with \xfile{texmf} in their name are usually organized this way.
%
% \subsection{Bundle installation}
%
% \paragraph{Unpacking.} Unpack the \xfile{oberdiek.tds.zip} in the
% TDS tree (also known as \xfile{texmf} tree) of your choice.
% Example (linux):
% \begin{quote}
%   |unzip oberdiek.tds.zip -d ~/texmf|
% \end{quote}
%
% \paragraph{Script installation.}
% Check the directory \xfile{TDS:scripts/oberdiek/} for
% scripts that need further installation steps.
% Package \xpackage{attachfile2} comes with the Perl script
% \xfile{pdfatfi.pl} that should be installed in such a way
% that it can be called as \texttt{pdfatfi}.
% Example (linux):
% \begin{quote}
%   |chmod +x scripts/oberdiek/pdfatfi.pl|\\
%   |cp scripts/oberdiek/pdfatfi.pl /usr/local/bin/|
% \end{quote}
%
% \subsection{Package installation}
%
% \paragraph{Unpacking.} The \xfile{.dtx} file is a self-extracting
% \docstrip\ archive. The files are extracted by running the
% \xfile{.dtx} through \plainTeX:
% \begin{quote}
%   \verb|tex hypbmsec.dtx|
% \end{quote}
%
% \paragraph{TDS.} Now the different files must be moved into
% the different directories in your installation TDS tree
% (also known as \xfile{texmf} tree):
% \begin{quote}
% \def\t{^^A
% \begin{tabular}{@{}>{\ttfamily}l@{ $\rightarrow$ }>{\ttfamily}l@{}}
%   hypbmsec.sty & tex/latex/oberdiek/hypbmsec.sty\\
%   hypbmsec.pdf & doc/latex/oberdiek/hypbmsec.pdf\\
%   hypbmsec.dtx & source/latex/oberdiek/hypbmsec.dtx\\
% \end{tabular}^^A
% }^^A
% \sbox0{\t}^^A
% \ifdim\wd0>\linewidth
%   \begingroup
%     \advance\linewidth by\leftmargin
%     \advance\linewidth by\rightmargin
%   \edef\x{\endgroup
%     \def\noexpand\lw{\the\linewidth}^^A
%   }\x
%   \def\lwbox{^^A
%     \leavevmode
%     \hbox to \linewidth{^^A
%       \kern-\leftmargin\relax
%       \hss
%       \usebox0
%       \hss
%       \kern-\rightmargin\relax
%     }^^A
%   }^^A
%   \ifdim\wd0>\lw
%     \sbox0{\small\t}^^A
%     \ifdim\wd0>\linewidth
%       \ifdim\wd0>\lw
%         \sbox0{\footnotesize\t}^^A
%         \ifdim\wd0>\linewidth
%           \ifdim\wd0>\lw
%             \sbox0{\scriptsize\t}^^A
%             \ifdim\wd0>\linewidth
%               \ifdim\wd0>\lw
%                 \sbox0{\tiny\t}^^A
%                 \ifdim\wd0>\linewidth
%                   \lwbox
%                 \else
%                   \usebox0
%                 \fi
%               \else
%                 \lwbox
%               \fi
%             \else
%               \usebox0
%             \fi
%           \else
%             \lwbox
%           \fi
%         \else
%           \usebox0
%         \fi
%       \else
%         \lwbox
%       \fi
%     \else
%       \usebox0
%     \fi
%   \else
%     \lwbox
%   \fi
% \else
%   \usebox0
% \fi
% \end{quote}
% If you have a \xfile{docstrip.cfg} that configures and enables \docstrip's
% TDS installing feature, then some files can already be in the right
% place, see the documentation of \docstrip.
%
% \subsection{Refresh file name databases}
%
% If your \TeX~distribution
% (\teTeX, \mikTeX, \dots) relies on file name databases, you must refresh
% these. For example, \teTeX\ users run \verb|texhash| or
% \verb|mktexlsr|.
%
% \subsection{Some details for the interested}
%
% \paragraph{Attached source.}
%
% The PDF documentation on CTAN also includes the
% \xfile{.dtx} source file. It can be extracted by
% AcrobatReader 6 or higher. Another option is \textsf{pdftk},
% e.g. unpack the file into the current directory:
% \begin{quote}
%   \verb|pdftk hypbmsec.pdf unpack_files output .|
% \end{quote}
%
% \paragraph{Unpacking with \LaTeX.}
% The \xfile{.dtx} chooses its action depending on the format:
% \begin{description}
% \item[\plainTeX:] Run \docstrip\ and extract the files.
% \item[\LaTeX:] Generate the documentation.
% \end{description}
% If you insist on using \LaTeX\ for \docstrip\ (really,
% \docstrip\ does not need \LaTeX), then inform the autodetect routine
% about your intention:
% \begin{quote}
%   \verb|latex \let\install=y\input{hypbmsec.dtx}|
% \end{quote}
% Do not forget to quote the argument according to the demands
% of your shell.
%
% \paragraph{Generating the documentation.}
% You can use both the \xfile{.dtx} or the \xfile{.drv} to generate
% the documentation. The process can be configured by the
% configuration file \xfile{ltxdoc.cfg}. For instance, put this
% line into this file, if you want to have A4 as paper format:
% \begin{quote}
%   \verb|\PassOptionsToClass{a4paper}{article}|
% \end{quote}
% An example follows how to generate the
% documentation with pdf\LaTeX:
% \begin{quote}
%\begin{verbatim}
%pdflatex hypbmsec.dtx
%makeindex -s gind.ist hypbmsec.idx
%pdflatex hypbmsec.dtx
%makeindex -s gind.ist hypbmsec.idx
%pdflatex hypbmsec.dtx
%\end{verbatim}
% \end{quote}
%
% \section{Catalogue}
%
% The following XML file can be used as source for the
% \href{http://mirror.ctan.org/help/Catalogue/catalogue.html}{\TeX\ Catalogue}.
% The elements \texttt{caption} and \texttt{description} are imported
% from the original XML file from the Catalogue.
% The name of the XML file in the Catalogue is \xfile{hypbmsec.xml}.
%    \begin{macrocode}
%<*catalogue>
<?xml version='1.0' encoding='us-ascii'?>
<!DOCTYPE entry SYSTEM 'catalogue.dtd'>
<entry datestamp='$Date$' modifier='$Author$' id='hypbmsec'>
  <name>hypbmsec</name>
  <caption>Hypertext bookmarks in sectioning commands.</caption>
  <authorref id='auth:oberdiek'/>
  <copyright owner='Heiko Oberdiek' year='1998-2000,2006,2007'/>
  <license type='lppl1.3'/>
  <version number='2.5'/>
  <description>
    Bookmark entries can be given as another argument to the LaTeX
    sectioning commands. The <xref refid='hyperref'>hyperref</xref>
    package is required to get the bookmarks, but the syntax
    works without it.
    <p/>
    This package is part of the <xref refid='oberdiek'>oberdiek</xref>
    bundle.
  </description>
  <documentation details='Package documentation'
      href='ctan:/macros/latex/contrib/oberdiek/hypbmsec.pdf'/>
  <ctan file='true' path='/macros/latex/contrib/oberdiek/hypbmsec.dtx'/>
  <miktex location='oberdiek'/>
  <texlive location='oberdiek'/>
  <install path='/macros/latex/contrib/oberdiek/oberdiek.tds.zip'/>
</entry>
%</catalogue>
%    \end{macrocode}
%
% \begin{History}
%   \begin{Version}{1998/11/20 v1.0}
%   \item
%     First version.
%   \item
%     It merges package \xpackage{hysecopt} and
%   \item
%     package \xpackage{hypbmpar}.
%   \item
%     Published for the DANTE'99 meeting^^A
%     \URL{}{http://dante99.cs.uni-dortmund.de/handouts/oberdiek/hypbmsec.sty}.
%   \end{Version}
%   \begin{Version}{1999/04/12 v2.0}
%   \item
%     Adaptation to \Package{hyperref} version 6.54.
%   \item
%     Documentation in dtx format.
%   \item
%     Copyright: LPPL (\CTAN{macros/latex/base/lppl.txt})
%   \item
%     First CTAN release.
%   \end{Version}
%   \begin{Version}{2000/03/22 v2.1}
%   \item
%     Bug fix in redefinition of \cmd{\chapter}.
%   \item
%     Copyright: LPPL 1.2
%   \end{Version}
%   \begin{Version}{2006/02/20 v2.2}
%   \item
%     Code is not changed.
%   \item
%     New DTX framework.
%   \item
%     LPPL 1.3
%   \end{Version}
%   \begin{Version}{2007/03/05 v2.3}
%   \item
%     Bug fix: Expand \cs{hbs@tocstring} and \cs{hbs@bmstring} before
%     calling \cs{hbs@seccmd}.
%   \end{Version}
%   \begin{Version}{2007/04/11 v2.4}
%   \item
%     Line ends sanitized.
%   \end{Version}
%   \begin{Version}{2016/05/16 v2.5}
%   \item
%     Documentation updates.
%   \end{Version}
% \end{History}
%
% \PrintIndex
%
% \Finale
\endinput
|
% \end{quote}
% Do not forget to quote the argument according to the demands
% of your shell.
%
% \paragraph{Generating the documentation.}
% You can use both the \xfile{.dtx} or the \xfile{.drv} to generate
% the documentation. The process can be configured by the
% configuration file \xfile{ltxdoc.cfg}. For instance, put this
% line into this file, if you want to have A4 as paper format:
% \begin{quote}
%   \verb|\PassOptionsToClass{a4paper}{article}|
% \end{quote}
% An example follows how to generate the
% documentation with pdf\LaTeX:
% \begin{quote}
%\begin{verbatim}
%pdflatex hypbmsec.dtx
%makeindex -s gind.ist hypbmsec.idx
%pdflatex hypbmsec.dtx
%makeindex -s gind.ist hypbmsec.idx
%pdflatex hypbmsec.dtx
%\end{verbatim}
% \end{quote}
%
% \section{Catalogue}
%
% The following XML file can be used as source for the
% \href{http://mirror.ctan.org/help/Catalogue/catalogue.html}{\TeX\ Catalogue}.
% The elements \texttt{caption} and \texttt{description} are imported
% from the original XML file from the Catalogue.
% The name of the XML file in the Catalogue is \xfile{hypbmsec.xml}.
%    \begin{macrocode}
%<*catalogue>
<?xml version='1.0' encoding='us-ascii'?>
<!DOCTYPE entry SYSTEM 'catalogue.dtd'>
<entry datestamp='$Date$' modifier='$Author$' id='hypbmsec'>
  <name>hypbmsec</name>
  <caption>Hypertext bookmarks in sectioning commands.</caption>
  <authorref id='auth:oberdiek'/>
  <copyright owner='Heiko Oberdiek' year='1998-2000,2006,2007'/>
  <license type='lppl1.3'/>
  <version number='2.5'/>
  <description>
    Bookmark entries can be given as another argument to the LaTeX
    sectioning commands. The <xref refid='hyperref'>hyperref</xref>
    package is required to get the bookmarks, but the syntax
    works without it.
    <p/>
    This package is part of the <xref refid='oberdiek'>oberdiek</xref>
    bundle.
  </description>
  <documentation details='Package documentation'
      href='ctan:/macros/latex/contrib/oberdiek/hypbmsec.pdf'/>
  <ctan file='true' path='/macros/latex/contrib/oberdiek/hypbmsec.dtx'/>
  <miktex location='oberdiek'/>
  <texlive location='oberdiek'/>
  <install path='/macros/latex/contrib/oberdiek/oberdiek.tds.zip'/>
</entry>
%</catalogue>
%    \end{macrocode}
%
% \begin{History}
%   \begin{Version}{1998/11/20 v1.0}
%   \item
%     First version.
%   \item
%     It merges package \xpackage{hysecopt} and
%   \item
%     package \xpackage{hypbmpar}.
%   \item
%     Published for the DANTE'99 meeting^^A
%     \URL{}{http://dante99.cs.uni-dortmund.de/handouts/oberdiek/hypbmsec.sty}.
%   \end{Version}
%   \begin{Version}{1999/04/12 v2.0}
%   \item
%     Adaptation to \Package{hyperref} version 6.54.
%   \item
%     Documentation in dtx format.
%   \item
%     Copyright: LPPL (\CTAN{macros/latex/base/lppl.txt})
%   \item
%     First CTAN release.
%   \end{Version}
%   \begin{Version}{2000/03/22 v2.1}
%   \item
%     Bug fix in redefinition of \cmd{\chapter}.
%   \item
%     Copyright: LPPL 1.2
%   \end{Version}
%   \begin{Version}{2006/02/20 v2.2}
%   \item
%     Code is not changed.
%   \item
%     New DTX framework.
%   \item
%     LPPL 1.3
%   \end{Version}
%   \begin{Version}{2007/03/05 v2.3}
%   \item
%     Bug fix: Expand \cs{hbs@tocstring} and \cs{hbs@bmstring} before
%     calling \cs{hbs@seccmd}.
%   \end{Version}
%   \begin{Version}{2007/04/11 v2.4}
%   \item
%     Line ends sanitized.
%   \end{Version}
%   \begin{Version}{2016/05/16 v2.5}
%   \item
%     Documentation updates.
%   \end{Version}
% \end{History}
%
% \PrintIndex
%
% \Finale
\endinput

%        (quote the arguments according to the demands of your shell)
%
% Documentation:
%    (a) If hypbmsec.drv is present:
%           latex hypbmsec.drv
%    (b) Without hypbmsec.drv:
%           latex hypbmsec.dtx; ...
%    The class ltxdoc loads the configuration file ltxdoc.cfg
%    if available. Here you can specify further options, e.g.
%    use A4 as paper format:
%       \PassOptionsToClass{a4paper}{article}
%
%    Programm calls to get the documentation (example):
%       pdflatex hypbmsec.dtx
%       makeindex -s gind.ist hypbmsec.idx
%       pdflatex hypbmsec.dtx
%       makeindex -s gind.ist hypbmsec.idx
%       pdflatex hypbmsec.dtx
%
% Installation:
%    TDS:tex/latex/oberdiek/hypbmsec.sty
%    TDS:doc/latex/oberdiek/hypbmsec.pdf
%    TDS:source/latex/oberdiek/hypbmsec.dtx
%
%<*ignore>
\begingroup
  \catcode123=1 %
  \catcode125=2 %
  \def\x{LaTeX2e}%
\expandafter\endgroup
\ifcase 0\ifx\install y1\fi\expandafter
         \ifx\csname processbatchFile\endcsname\relax\else1\fi
         \ifx\fmtname\x\else 1\fi\relax
\else\csname fi\endcsname
%</ignore>
%<*install>
\input docstrip.tex
\Msg{************************************************************************}
\Msg{* Installation}
\Msg{* Package: hypbmsec 2016/05/16 v2.5 Bookmarks in sectioning commands (HO)}
\Msg{************************************************************************}

\keepsilent
\askforoverwritefalse

\let\MetaPrefix\relax
\preamble

This is a generated file.

Project: hypbmsec
Version: 2016/05/16 v2.5

Copyright (C) 1998-2000, 2006, 2007 by
   Heiko Oberdiek <heiko.oberdiek at googlemail.com>

This work may be distributed and/or modified under the
conditions of the LaTeX Project Public License, either
version 1.3c of this license or (at your option) any later
version. This version of this license is in
   http://www.latex-project.org/lppl/lppl-1-3c.txt
and the latest version of this license is in
   http://www.latex-project.org/lppl.txt
and version 1.3 or later is part of all distributions of
LaTeX version 2005/12/01 or later.

This work has the LPPL maintenance status "maintained".

This Current Maintainer of this work is Heiko Oberdiek.

This work consists of the main source file hypbmsec.dtx
and the derived files
   hypbmsec.sty, hypbmsec.pdf, hypbmsec.ins, hypbmsec.drv.

\endpreamble
\let\MetaPrefix\DoubleperCent

\generate{%
  \file{hypbmsec.ins}{\from{hypbmsec.dtx}{install}}%
  \file{hypbmsec.drv}{\from{hypbmsec.dtx}{driver}}%
  \usedir{tex/latex/oberdiek}%
  \file{hypbmsec.sty}{\from{hypbmsec.dtx}{package}}%
  \nopreamble
  \nopostamble
  \usedir{source/latex/oberdiek/catalogue}%
  \file{hypbmsec.xml}{\from{hypbmsec.dtx}{catalogue}}%
}

\catcode32=13\relax% active space
\let =\space%
\Msg{************************************************************************}
\Msg{*}
\Msg{* To finish the installation you have to move the following}
\Msg{* file into a directory searched by TeX:}
\Msg{*}
\Msg{*     hypbmsec.sty}
\Msg{*}
\Msg{* To produce the documentation run the file `hypbmsec.drv'}
\Msg{* through LaTeX.}
\Msg{*}
\Msg{* Happy TeXing!}
\Msg{*}
\Msg{************************************************************************}

\endbatchfile
%</install>
%<*ignore>
\fi
%</ignore>
%<*driver>
\NeedsTeXFormat{LaTeX2e}
\ProvidesFile{hypbmsec.drv}%
  [2016/05/16 v2.5 Bookmarks in sectioning commands (HO)]%
\documentclass{ltxdoc}
\usepackage{holtxdoc}[2011/11/22]
\begin{document}
  \DocInput{hypbmsec.dtx}%
\end{document}
%</driver>
% \fi
%
%
% \CharacterTable
%  {Upper-case    \A\B\C\D\E\F\G\H\I\J\K\L\M\N\O\P\Q\R\S\T\U\V\W\X\Y\Z
%   Lower-case    \a\b\c\d\e\f\g\h\i\j\k\l\m\n\o\p\q\r\s\t\u\v\w\x\y\z
%   Digits        \0\1\2\3\4\5\6\7\8\9
%   Exclamation   \!     Double quote  \"     Hash (number) \#
%   Dollar        \$     Percent       \%     Ampersand     \&
%   Acute accent  \'     Left paren    \(     Right paren   \)
%   Asterisk      \*     Plus          \+     Comma         \,
%   Minus         \-     Point         \.     Solidus       \/
%   Colon         \:     Semicolon     \;     Less than     \<
%   Equals        \=     Greater than  \>     Question mark \?
%   Commercial at \@     Left bracket  \[     Backslash     \\
%   Right bracket \]     Circumflex    \^     Underscore    \_
%   Grave accent  \`     Left brace    \{     Vertical bar  \|
%   Right brace   \}     Tilde         \~}
%
% \GetFileInfo{hypbmsec.drv}
%
% \title{The \xpackage{hypbmsec} package}
% \date{2016/05/16 v2.5}
% \author{Heiko Oberdiek\thanks
% {Please report any issues at https://github.com/ho-tex/oberdiek/issues}\\
% \xemail{heiko.oberdiek at googlemail.com}}
%
% \maketitle
%
% \begin{abstract}
% This package expands the syntax of the sectioning commands. If the
% argument of the sectioning commands isn't usable as outline entry,
% a replacement for the bookmarks can be given.
% \end{abstract}
%
% \tableofcontents
%
% \newcommand{\type}[1]{\textsf{#1}}
%
% ^^A No thread support.
% \newenvironment{article}[1]{}{}
%
% \section{Usage}
%
% \subsection{Bookmarks restrictions}\label{sec:restrictions}
%    Outline entries (bookmarks) are written to a file and have
%    to obey the pdf specification.
%    Therefore they have several restrictions:
%    \begin{itemize}
%    \item Bookmarks have to be encoded in PDFDocEncoding^^A
%          \footnote{\Package{hyperref} doesn't support
%            Unicode.}.
%    \item They should only expand to letters and spaces.
%    \item The result of expansion have to be a valid pdf string.
%    \item Stomach commands like \cmd{\relax}, box commands, math,
%          assignments, or definitions aren't allowed.
%    \item Short entries are recommended, which allow a clear view.
%    \end{itemize}
%
% \subsection{\texorpdfstring{\cmd{\texorpdfstring}}{^^A
%    \textbackslash texorpdfstring}}
%    The generic way in package \Package{hyperref} is the use
%    of \cmd{\texorpdfstring}^^A
%    \footnote{In versions of \Package{hyperref} below 6.54 see
%      \cmd{\ifbookmark}.}:
%    \begin{quote}
%\begin{verbatim}
%\section{Pythagoras:
%  \texorpdfstring{$a^2+b^2=c^2}{%
%    a\texttwosuperior\ + b\texttwosuperior\ =
%    c\texttwosuperior}%
%}
%\end{verbatim}
%    \end{quote}
%
% \subsection{Sectioning commands}
%    The package \Package{hyperref} automatically generates
%    bookmarks from the sectioning commands,
%    unless it is suppressed by an option.
%    Commands that structure the text are here called
%    ``sectioning commands'':
%    \begin{quote}
%    \cmd{\part}, \cmd{\chapter},\\
%    \cmd{\section}, \cmd{\subsection}, \cmd{\subsubsection},\\
%    \cmd{\paragraph}, \cmd{\subparagraph}
%    \end{quote}
%
% \subsection{Places\texorpdfstring{ for sectioning strings}{}}
%    \label{sec:places}
%    The argument(s) of these commands are used on several places:
%    \begin{description}
%    \item[\type{text}]
%      The current text without restrictions.
%    \item[\type{toc}]
%      The headlines and the table of contents with the
%      restrictions of ``moving arguments''.
%    \item[\type{out}]
%      The outlines with many restrictions: The outline
%      have to expand to a valid pdf string with PDFDocEncoding
%      (see \ref{sec:restrictions}).
%    \end{description}
%
% \subsection{\texorpdfstring{Solution with o}{O}ptional arguments}
%    If the user wants to use a footnote within a sectioning command,
%    the \LaTeX{} solution is an optional argument:
%    \begin{quote}
%      |\section[Title]{Title\footnote{Footnote text}}|
%    \end{quote}
%    Now |Title| without the footnote is used in the headlines and
%    the table of contents. Also \Package{hyperref} uses it for the
%    bookmarks.
%
%    This package \Package{\filename} offers two possibilities to
%    specify a separate outline entry:
%    \begin{itemize}
%    \item An additional second optional argument in square brackets.
%    \item An additional optional argument in parentheses (in
%          assoziation with a pdf string that is internally surrounded
%          by parentheses, too).
%    \end{itemize}
%    Because \Package{\filename} stores the original meaning of the
%    sectioning commands and uses them again, there should be no
%    problems with packages that redefine the sectioning commands, if
%    these packages doesn't change the syntax.
%
% \subsection{Syntax}
%    The following examples show the syntax of the sectioning
%    commands. For the places the strings appear the abbreviations
%    are used, that are introduced in \ref{sec:places}.
%
% \subsubsection{\texorpdfstring{Star form}{^^A
%    \textbackslash section*\{\}}}
%    The behaviour of the star form isn't changed. The string
%    appears only in the current text:
%    \begin{article}{Syntax}
%    \begin{quote}
%      |\section*{text}|
%    \end{quote}
%    \end{article}
%
% \subsubsection{\texorpdfstring{Without optional arguments}{^^A
%    \textbackslash section\{\}}}
%    The normal case, the string in the mandatory argument is
%    used for all places:
%    \begin{article}{Syntax}
%    \begin{quote}
%      |\section{text, toc, out}|
%    \end{quote}
%    \end{article}
%
% \subsubsection{\texorpdfstring{One optional argument}{^^A
%    \textbackslash section[]\{\}}}
%    Also the form with one optional parameter in square brackets isn't
%    new; for the bookmarks the optional parameter is used:
%    \begin{article}{Syntax}
%    \begin{quote}
%      |\section[toc, out]{text}|
%    \end{quote}
%    \end{article}
%
% \subsubsection{\texorpdfstring{Two optional arguments}{^^A
%    \textbackslash section[][out]\{\}}}\label{sec:two}
%    The second optional parameter in square brackets is introduced
%    by this package to specify an outline entry:
%    \begin{article}{Syntax}
%    \begin{quote}
%      |\section[toc][out]{text}|
%    \end{quote}
%    \end{article}
%
% \subsubsection{\texorpdfstring{Optional argument in parentheses}{^^A
%    \textbackslash section(out)\{\}}}
%    Often the \type{toc} and the \type{text} string would be the same.
%    With the method of the two optional arguments in square brackets
%    (see \ref{sec:two}) this string must be given twice,
%    if the user only wants to specify a different outline entry.
%    Therefore this package offers another possibility:
%    In association with the internal representation in the pdf file
%    an outline entry can be given in parentheses.
%    So the package can easily distinguish between
%    the two forms of optional arguments and the order does not matter:
%    \begin{article}{Syntax}
%    \begin{quote}
%      |\section(out){toc, text}|\\
%      |\section[toc](out){text}|\\
%      |\section(out)[toc]{text}|
%    \end{quote}
%    \end{article}
%
% \subsection{Without \Package{hyperref}}
%    Package \Package{\filename} uses \Package{hyperref} for support of
%    the bookmarks, but this package is not required.
%    If \Package{hyperref} isn't loaded, or
%    is called with a driver that doesn't support bookmarks,
%    package \Package{\filename} shouldn't be removed,
%    because this would lead to
%    a wrong syntax of the sectioning commands.
%    In any cases package \Package{\filename}
%    supports its syntax and ignores the outline entries,
%    if there are no code for bookmarks.
%    So it is possible to write texts,
%    that are processed with several drivers to get different output
%    formats.
%
% \subsection{Protecting parentheses}
%    If the string itself contains parentheses, they have to be hidden
%    from \TeX's argument parsing mechanism.
%    The argument should be surrounded
%    by curly braces:
%    \begin{quote}
%      |\section({outlines(bookmarks)}){text, toc}|
%    \end{quote}
%    With version 6.54 of \Package{hyperref} the other standard method
%    works, too: The closing parenthesis is protected:
%    \begin{quote}
%      |\section(outlines(bookmarks{)}){text, toc}|
%    \end{quote}
%
% \StopEventually{
% }
%
% \section{Implementation}
%    \begin{macrocode}
%<*package>
%    \end{macrocode}
%    Package identification.
%    \begin{macrocode}
\NeedsTeXFormat{LaTeX2e}
\ProvidesPackage{hypbmsec}%
  [2016/05/16 v2.5 Bookmarks in sectioning commands (HO)]
%    \end{macrocode}
%
%    Because of redifining the sectioning commands, it is dangerous
%    to reload the package several times.
%    \begin{macrocode}
\@ifundefined{hbs@do}{}{%
  \PackageInfo{hypbmsec}{Package 'hypbmsec' is already loaded}%
  \endinput
}
%    \end{macrocode}
%
%    \begin{macro}{\hbs@do}
%    The redefined sectioning commands calls \cmd{\hbs@do}. It does
%    \begin{itemize}
%    \item handle the star case.
%    \item resets the macros that store the entries for the outlines
%          (\cmd{\hbs@bmstring}) and table of contents (\cmd{\hbs@tocstring}).
%    \item store the sectioning command |#1| in \cmd{\hbs@seccmd}
%          for later reuse.
%    \item at last call \cmd{\hbs@checkarg} that scans and interprets the
%          parameters of the redefined sectioning command.
%    \end{itemize}
%    \begin{macrocode}
\def\hbs@do#1{%
  \@ifstar{#1*}{%
    \let\hbs@tocstring\relax
    \let\hbs@bmstring\relax
    \let\hbs@seccmd#1%
    \hbs@checkarg
  }%
}
%    \end{macrocode}
%    \end{macro}
%
%    \begin{macro}{\hbs@checkarg}
%    \cmd{\hbs@checkarg} determines the type of the next argument:
%    \begin{itemize}
%    \item An optional argument in square brackets can be an entry
%          for the table of contents or the bookmarks. It will be
%          read by \cmd{\hbs@getsquare}
%    \item An optional argument in parentheses is an outline entry.
%          This is worked off by \cmd{\hbs@getbookmark}.
%    \item If there are no more optional arguments, \cmd{\hbs@process}
%          reads the mandatory argument and calls the original
%          sectioning commands.
%    \end{itemize}
%    \begin{macrocode}
\def\hbs@checkarg{%
  \@ifnextchar[\hbs@getsquare{%
    \@ifnextchar(\hbs@getbookmark\hbs@process
  }%
}
%    \end{macrocode}
%    \end{macro}
%
%    \begin{macro}{\hbs@getsquare}
%    \cmd{\hbs@getsquare} reads an optional argument in square
%    brackets and determines, if this is an entry for the table
%    of contents or the bookmarks.
%    \begin{macrocode}
\long\def\hbs@getsquare[#1]{%
  \ifx\hbs@tocstring\relax
    \def\hbs@tocstring{#1}%
  \else
    \hbs@bmdef{#1}%
  \fi
  \hbs@checkarg
}
%    \end{macrocode}
%    \end{macro}
%
%    \begin{macro}{\hbs@getbookmark}
%    \cmd{\hbs@getbookmark} reads an outline entry in parentheses.
%    \begin{macrocode}
\def\hbs@getbookmark(#1){%
  \hbs@bmdef{#1}%
  \hbs@checkarg
}
%    \end{macrocode}
%    \end{macro}
%
%    \begin{macro}{\hbs@bmdef}
%    The command \cmd{\hbs@bmdef} save the bookmark entry in
%    parameter |#1| in the macro \cmd{\hbs@bmstring} and catches
%    the case, if the user has given several outline strings.
%    \begin{macrocode}
\def\hbs@bmdef#1{%
  \ifx\hbs@bmstring\relax
    \def\hbs@bmstring{#1}%
  \else
    \PackageError{hypbmsec}{%
      Sectioning command with too many parameters%
    }{%
      You can only give one outline entry.%
    }%
  \fi
}
%    \end{macrocode}
%    \end{macro}
%
%    \begin{macro}{\hbs@process}
%    The parameter |#1| is the mandatory argument of the sectioning
%    commands. \cmd{\hbs@process} calls the original sectioning command
%    stored in \cmd{\hbs@seccmd} with arguments that depend of which
%    optional argument are used previously.
%    \begin{macrocode}
\long\def\hbs@process#1{%
  \ifx\hbs@tocstring\relax
    \ifx\hbs@bmstring\relax
      \hbs@seccmd{#1}%
    \else
      \begingroup
        \def\x##1{\endgroup
          \hbs@seccmd{\texorpdfstring{#1}{##1}}%
        }%
      \expandafter\x\expandafter{\hbs@bmstring}%
    \fi
  \else
    \ifx\hbs@bmstring\relax
      \expandafter\hbs@seccmd\expandafter[%
        \expandafter{\hbs@tocstring}%
      ]{#1}%
    \else
      \expandafter\expandafter\expandafter
      \hbs@seccmd\expandafter\expandafter\expandafter[%
        \expandafter\expandafter\expandafter
        \texorpdfstring
        \expandafter\expandafter\expandafter{%
          \expandafter\hbs@tocstring\expandafter
        }\expandafter{%
          \hbs@bmstring
        }%
      ]{#1}%
    \fi
  \fi
}
%    \end{macrocode}
%    \end{macro}
%
%    We have to check, whether package \Package{hyperref} is loaded
%    and have to provide a definition for \cmd{\texorpdfstring}.
%    Because \Package{hyperref} can be loaded after this package,
%    we do the work later (\cmd{\AtBeginDocument}).
%
%    This code only checks versions of \Package{hyperref} that
%    define \cmd{\ifbookmark} (v6.4x until v6.53) or
%    \cmd{\texorpdfstring} (v6.54 and above). Older versions aren't
%    supported.
%    \begin{macrocode}
\AtBeginDocument{%
  \@ifundefined{texorpdfstring}{%
    \@ifundefined{ifbookmark}{%
      \let\texorpdfstring\@firstoftwo
      \@ifpackageloaded{hyperref}{%
        \PackageInfo{hypbmsec}{%
          \ifx\hy@driver\@empty
            Default driver %
          \else
            '\hy@driver' %
          \fi
          of hyperref not supported,\MessageBreak
          bookmark parameters will be ignored%
        }%
      }{%
        \PackageInfo{hypbmsec}{%
          Package hyperref not loaded,\MessageBreak
          bookmark parameters will be ignored%
        }%
      }%
    }%
    {%
      \newcommand\texorpdfstring[2]{\ifbookmark{#2}{#1}}%
      \PackageWarningNoLine{hypbmsec}{%
        Old hyperref version found,\MessageBreak
        update of hyperref recommended%
      }%
    }%
  }{}%
%    \end{macrocode}
%
%    Other packages are allowed to redefine the sectioning commands,
%    if they does not change the syntax. Therefore the redefinitons
%    of this package should be done after the other packages.
%    \begin{macrocode}
  \let\hbs@part         \part
  \let\hbs@section      \section
  \let\hbs@subsection   \subsection
  \let\hbs@subsubsection\subsubsection
  \let\hbs@paragraph    \paragraph
  \let\hbs@subparagraph \subparagraph
  \renewcommand\part         {\hbs@do\hbs@part}%
  \renewcommand\section      {\hbs@do\hbs@section}%
  \renewcommand\subsection   {\hbs@do\hbs@subsection}%
  \renewcommand\subsubsection{\hbs@do\hbs@subsubsection}%
  \renewcommand\paragraph    {\hbs@do\hbs@paragraph}%
  \renewcommand\subparagraph {\hbs@do\hbs@subparagraph}%
  \begingroup\expandafter\expandafter\expandafter\endgroup
  \expandafter\ifx\csname chapter\endcsname\relax\else
    \let\hbs@chapter    \chapter
    \renewcommand\chapter    {\hbs@do\hbs@chapter}%
  \fi
}
%    \end{macrocode}
%
%    \begin{macrocode}
%</package>
%    \end{macrocode}
%
% \section{Installation}
%
% \subsection{Download}
%
% \paragraph{Package.} This package is available on
% CTAN\footnote{\url{http://ctan.org/pkg/hypbmsec}}:
% \begin{description}
% \item[\CTAN{macros/latex/contrib/oberdiek/hypbmsec.dtx}] The source file.
% \item[\CTAN{macros/latex/contrib/oberdiek/hypbmsec.pdf}] Documentation.
% \end{description}
%
%
% \paragraph{Bundle.} All the packages of the bundle `oberdiek'
% are also available in a TDS compliant ZIP archive. There
% the packages are already unpacked and the documentation files
% are generated. The files and directories obey the TDS standard.
% \begin{description}
% \item[\CTAN{install/macros/latex/contrib/oberdiek.tds.zip}]
% \end{description}
% \emph{TDS} refers to the standard ``A Directory Structure
% for \TeX\ Files'' (\CTAN{tds/tds.pdf}). Directories
% with \xfile{texmf} in their name are usually organized this way.
%
% \subsection{Bundle installation}
%
% \paragraph{Unpacking.} Unpack the \xfile{oberdiek.tds.zip} in the
% TDS tree (also known as \xfile{texmf} tree) of your choice.
% Example (linux):
% \begin{quote}
%   |unzip oberdiek.tds.zip -d ~/texmf|
% \end{quote}
%
% \paragraph{Script installation.}
% Check the directory \xfile{TDS:scripts/oberdiek/} for
% scripts that need further installation steps.
% Package \xpackage{attachfile2} comes with the Perl script
% \xfile{pdfatfi.pl} that should be installed in such a way
% that it can be called as \texttt{pdfatfi}.
% Example (linux):
% \begin{quote}
%   |chmod +x scripts/oberdiek/pdfatfi.pl|\\
%   |cp scripts/oberdiek/pdfatfi.pl /usr/local/bin/|
% \end{quote}
%
% \subsection{Package installation}
%
% \paragraph{Unpacking.} The \xfile{.dtx} file is a self-extracting
% \docstrip\ archive. The files are extracted by running the
% \xfile{.dtx} through \plainTeX:
% \begin{quote}
%   \verb|tex hypbmsec.dtx|
% \end{quote}
%
% \paragraph{TDS.} Now the different files must be moved into
% the different directories in your installation TDS tree
% (also known as \xfile{texmf} tree):
% \begin{quote}
% \def\t{^^A
% \begin{tabular}{@{}>{\ttfamily}l@{ $\rightarrow$ }>{\ttfamily}l@{}}
%   hypbmsec.sty & tex/latex/oberdiek/hypbmsec.sty\\
%   hypbmsec.pdf & doc/latex/oberdiek/hypbmsec.pdf\\
%   hypbmsec.dtx & source/latex/oberdiek/hypbmsec.dtx\\
% \end{tabular}^^A
% }^^A
% \sbox0{\t}^^A
% \ifdim\wd0>\linewidth
%   \begingroup
%     \advance\linewidth by\leftmargin
%     \advance\linewidth by\rightmargin
%   \edef\x{\endgroup
%     \def\noexpand\lw{\the\linewidth}^^A
%   }\x
%   \def\lwbox{^^A
%     \leavevmode
%     \hbox to \linewidth{^^A
%       \kern-\leftmargin\relax
%       \hss
%       \usebox0
%       \hss
%       \kern-\rightmargin\relax
%     }^^A
%   }^^A
%   \ifdim\wd0>\lw
%     \sbox0{\small\t}^^A
%     \ifdim\wd0>\linewidth
%       \ifdim\wd0>\lw
%         \sbox0{\footnotesize\t}^^A
%         \ifdim\wd0>\linewidth
%           \ifdim\wd0>\lw
%             \sbox0{\scriptsize\t}^^A
%             \ifdim\wd0>\linewidth
%               \ifdim\wd0>\lw
%                 \sbox0{\tiny\t}^^A
%                 \ifdim\wd0>\linewidth
%                   \lwbox
%                 \else
%                   \usebox0
%                 \fi
%               \else
%                 \lwbox
%               \fi
%             \else
%               \usebox0
%             \fi
%           \else
%             \lwbox
%           \fi
%         \else
%           \usebox0
%         \fi
%       \else
%         \lwbox
%       \fi
%     \else
%       \usebox0
%     \fi
%   \else
%     \lwbox
%   \fi
% \else
%   \usebox0
% \fi
% \end{quote}
% If you have a \xfile{docstrip.cfg} that configures and enables \docstrip's
% TDS installing feature, then some files can already be in the right
% place, see the documentation of \docstrip.
%
% \subsection{Refresh file name databases}
%
% If your \TeX~distribution
% (\teTeX, \mikTeX, \dots) relies on file name databases, you must refresh
% these. For example, \teTeX\ users run \verb|texhash| or
% \verb|mktexlsr|.
%
% \subsection{Some details for the interested}
%
% \paragraph{Attached source.}
%
% The PDF documentation on CTAN also includes the
% \xfile{.dtx} source file. It can be extracted by
% AcrobatReader 6 or higher. Another option is \textsf{pdftk},
% e.g. unpack the file into the current directory:
% \begin{quote}
%   \verb|pdftk hypbmsec.pdf unpack_files output .|
% \end{quote}
%
% \paragraph{Unpacking with \LaTeX.}
% The \xfile{.dtx} chooses its action depending on the format:
% \begin{description}
% \item[\plainTeX:] Run \docstrip\ and extract the files.
% \item[\LaTeX:] Generate the documentation.
% \end{description}
% If you insist on using \LaTeX\ for \docstrip\ (really,
% \docstrip\ does not need \LaTeX), then inform the autodetect routine
% about your intention:
% \begin{quote}
%   \verb|latex \let\install=y% \iffalse meta-comment
%
% File: hypbmsec.dtx
% Version: 2016/05/16 v2.5
% Info: Bookmarks in sectioning commands
%
% Copyright (C) 1998-2000, 2006, 2007 by
%    Heiko Oberdiek <heiko.oberdiek at googlemail.com>
%    2016
%    https://github.com/ho-tex/oberdiek/issues
%
% This work may be distributed and/or modified under the
% conditions of the LaTeX Project Public License, either
% version 1.3c of this license or (at your option) any later
% version. This version of this license is in
%    http://www.latex-project.org/lppl/lppl-1-3c.txt
% and the latest version of this license is in
%    http://www.latex-project.org/lppl.txt
% and version 1.3 or later is part of all distributions of
% LaTeX version 2005/12/01 or later.
%
% This work has the LPPL maintenance status "maintained".
%
% This Current Maintainer of this work is Heiko Oberdiek.
%
% This work consists of the main source file hypbmsec.dtx
% and the derived files
%    hypbmsec.sty, hypbmsec.pdf, hypbmsec.ins, hypbmsec.drv.
%
% Distribution:
%    CTAN:macros/latex/contrib/oberdiek/hypbmsec.dtx
%    CTAN:macros/latex/contrib/oberdiek/hypbmsec.pdf
%
% Unpacking:
%    (a) If hypbmsec.ins is present:
%           tex hypbmsec.ins
%    (b) Without hypbmsec.ins:
%           tex hypbmsec.dtx
%    (c) If you insist on using LaTeX
%           latex \let\install=y% \iffalse meta-comment
%
% File: hypbmsec.dtx
% Version: 2016/05/16 v2.5
% Info: Bookmarks in sectioning commands
%
% Copyright (C) 1998-2000, 2006, 2007 by
%    Heiko Oberdiek <heiko.oberdiek at googlemail.com>
%    2016
%    https://github.com/ho-tex/oberdiek/issues
%
% This work may be distributed and/or modified under the
% conditions of the LaTeX Project Public License, either
% version 1.3c of this license or (at your option) any later
% version. This version of this license is in
%    http://www.latex-project.org/lppl/lppl-1-3c.txt
% and the latest version of this license is in
%    http://www.latex-project.org/lppl.txt
% and version 1.3 or later is part of all distributions of
% LaTeX version 2005/12/01 or later.
%
% This work has the LPPL maintenance status "maintained".
%
% This Current Maintainer of this work is Heiko Oberdiek.
%
% This work consists of the main source file hypbmsec.dtx
% and the derived files
%    hypbmsec.sty, hypbmsec.pdf, hypbmsec.ins, hypbmsec.drv.
%
% Distribution:
%    CTAN:macros/latex/contrib/oberdiek/hypbmsec.dtx
%    CTAN:macros/latex/contrib/oberdiek/hypbmsec.pdf
%
% Unpacking:
%    (a) If hypbmsec.ins is present:
%           tex hypbmsec.ins
%    (b) Without hypbmsec.ins:
%           tex hypbmsec.dtx
%    (c) If you insist on using LaTeX
%           latex \let\install=y\input{hypbmsec.dtx}
%        (quote the arguments according to the demands of your shell)
%
% Documentation:
%    (a) If hypbmsec.drv is present:
%           latex hypbmsec.drv
%    (b) Without hypbmsec.drv:
%           latex hypbmsec.dtx; ...
%    The class ltxdoc loads the configuration file ltxdoc.cfg
%    if available. Here you can specify further options, e.g.
%    use A4 as paper format:
%       \PassOptionsToClass{a4paper}{article}
%
%    Programm calls to get the documentation (example):
%       pdflatex hypbmsec.dtx
%       makeindex -s gind.ist hypbmsec.idx
%       pdflatex hypbmsec.dtx
%       makeindex -s gind.ist hypbmsec.idx
%       pdflatex hypbmsec.dtx
%
% Installation:
%    TDS:tex/latex/oberdiek/hypbmsec.sty
%    TDS:doc/latex/oberdiek/hypbmsec.pdf
%    TDS:source/latex/oberdiek/hypbmsec.dtx
%
%<*ignore>
\begingroup
  \catcode123=1 %
  \catcode125=2 %
  \def\x{LaTeX2e}%
\expandafter\endgroup
\ifcase 0\ifx\install y1\fi\expandafter
         \ifx\csname processbatchFile\endcsname\relax\else1\fi
         \ifx\fmtname\x\else 1\fi\relax
\else\csname fi\endcsname
%</ignore>
%<*install>
\input docstrip.tex
\Msg{************************************************************************}
\Msg{* Installation}
\Msg{* Package: hypbmsec 2016/05/16 v2.5 Bookmarks in sectioning commands (HO)}
\Msg{************************************************************************}

\keepsilent
\askforoverwritefalse

\let\MetaPrefix\relax
\preamble

This is a generated file.

Project: hypbmsec
Version: 2016/05/16 v2.5

Copyright (C) 1998-2000, 2006, 2007 by
   Heiko Oberdiek <heiko.oberdiek at googlemail.com>

This work may be distributed and/or modified under the
conditions of the LaTeX Project Public License, either
version 1.3c of this license or (at your option) any later
version. This version of this license is in
   http://www.latex-project.org/lppl/lppl-1-3c.txt
and the latest version of this license is in
   http://www.latex-project.org/lppl.txt
and version 1.3 or later is part of all distributions of
LaTeX version 2005/12/01 or later.

This work has the LPPL maintenance status "maintained".

This Current Maintainer of this work is Heiko Oberdiek.

This work consists of the main source file hypbmsec.dtx
and the derived files
   hypbmsec.sty, hypbmsec.pdf, hypbmsec.ins, hypbmsec.drv.

\endpreamble
\let\MetaPrefix\DoubleperCent

\generate{%
  \file{hypbmsec.ins}{\from{hypbmsec.dtx}{install}}%
  \file{hypbmsec.drv}{\from{hypbmsec.dtx}{driver}}%
  \usedir{tex/latex/oberdiek}%
  \file{hypbmsec.sty}{\from{hypbmsec.dtx}{package}}%
  \nopreamble
  \nopostamble
  \usedir{source/latex/oberdiek/catalogue}%
  \file{hypbmsec.xml}{\from{hypbmsec.dtx}{catalogue}}%
}

\catcode32=13\relax% active space
\let =\space%
\Msg{************************************************************************}
\Msg{*}
\Msg{* To finish the installation you have to move the following}
\Msg{* file into a directory searched by TeX:}
\Msg{*}
\Msg{*     hypbmsec.sty}
\Msg{*}
\Msg{* To produce the documentation run the file `hypbmsec.drv'}
\Msg{* through LaTeX.}
\Msg{*}
\Msg{* Happy TeXing!}
\Msg{*}
\Msg{************************************************************************}

\endbatchfile
%</install>
%<*ignore>
\fi
%</ignore>
%<*driver>
\NeedsTeXFormat{LaTeX2e}
\ProvidesFile{hypbmsec.drv}%
  [2016/05/16 v2.5 Bookmarks in sectioning commands (HO)]%
\documentclass{ltxdoc}
\usepackage{holtxdoc}[2011/11/22]
\begin{document}
  \DocInput{hypbmsec.dtx}%
\end{document}
%</driver>
% \fi
%
%
% \CharacterTable
%  {Upper-case    \A\B\C\D\E\F\G\H\I\J\K\L\M\N\O\P\Q\R\S\T\U\V\W\X\Y\Z
%   Lower-case    \a\b\c\d\e\f\g\h\i\j\k\l\m\n\o\p\q\r\s\t\u\v\w\x\y\z
%   Digits        \0\1\2\3\4\5\6\7\8\9
%   Exclamation   \!     Double quote  \"     Hash (number) \#
%   Dollar        \$     Percent       \%     Ampersand     \&
%   Acute accent  \'     Left paren    \(     Right paren   \)
%   Asterisk      \*     Plus          \+     Comma         \,
%   Minus         \-     Point         \.     Solidus       \/
%   Colon         \:     Semicolon     \;     Less than     \<
%   Equals        \=     Greater than  \>     Question mark \?
%   Commercial at \@     Left bracket  \[     Backslash     \\
%   Right bracket \]     Circumflex    \^     Underscore    \_
%   Grave accent  \`     Left brace    \{     Vertical bar  \|
%   Right brace   \}     Tilde         \~}
%
% \GetFileInfo{hypbmsec.drv}
%
% \title{The \xpackage{hypbmsec} package}
% \date{2016/05/16 v2.5}
% \author{Heiko Oberdiek\thanks
% {Please report any issues at https://github.com/ho-tex/oberdiek/issues}\\
% \xemail{heiko.oberdiek at googlemail.com}}
%
% \maketitle
%
% \begin{abstract}
% This package expands the syntax of the sectioning commands. If the
% argument of the sectioning commands isn't usable as outline entry,
% a replacement for the bookmarks can be given.
% \end{abstract}
%
% \tableofcontents
%
% \newcommand{\type}[1]{\textsf{#1}}
%
% ^^A No thread support.
% \newenvironment{article}[1]{}{}
%
% \section{Usage}
%
% \subsection{Bookmarks restrictions}\label{sec:restrictions}
%    Outline entries (bookmarks) are written to a file and have
%    to obey the pdf specification.
%    Therefore they have several restrictions:
%    \begin{itemize}
%    \item Bookmarks have to be encoded in PDFDocEncoding^^A
%          \footnote{\Package{hyperref} doesn't support
%            Unicode.}.
%    \item They should only expand to letters and spaces.
%    \item The result of expansion have to be a valid pdf string.
%    \item Stomach commands like \cmd{\relax}, box commands, math,
%          assignments, or definitions aren't allowed.
%    \item Short entries are recommended, which allow a clear view.
%    \end{itemize}
%
% \subsection{\texorpdfstring{\cmd{\texorpdfstring}}{^^A
%    \textbackslash texorpdfstring}}
%    The generic way in package \Package{hyperref} is the use
%    of \cmd{\texorpdfstring}^^A
%    \footnote{In versions of \Package{hyperref} below 6.54 see
%      \cmd{\ifbookmark}.}:
%    \begin{quote}
%\begin{verbatim}
%\section{Pythagoras:
%  \texorpdfstring{$a^2+b^2=c^2}{%
%    a\texttwosuperior\ + b\texttwosuperior\ =
%    c\texttwosuperior}%
%}
%\end{verbatim}
%    \end{quote}
%
% \subsection{Sectioning commands}
%    The package \Package{hyperref} automatically generates
%    bookmarks from the sectioning commands,
%    unless it is suppressed by an option.
%    Commands that structure the text are here called
%    ``sectioning commands'':
%    \begin{quote}
%    \cmd{\part}, \cmd{\chapter},\\
%    \cmd{\section}, \cmd{\subsection}, \cmd{\subsubsection},\\
%    \cmd{\paragraph}, \cmd{\subparagraph}
%    \end{quote}
%
% \subsection{Places\texorpdfstring{ for sectioning strings}{}}
%    \label{sec:places}
%    The argument(s) of these commands are used on several places:
%    \begin{description}
%    \item[\type{text}]
%      The current text without restrictions.
%    \item[\type{toc}]
%      The headlines and the table of contents with the
%      restrictions of ``moving arguments''.
%    \item[\type{out}]
%      The outlines with many restrictions: The outline
%      have to expand to a valid pdf string with PDFDocEncoding
%      (see \ref{sec:restrictions}).
%    \end{description}
%
% \subsection{\texorpdfstring{Solution with o}{O}ptional arguments}
%    If the user wants to use a footnote within a sectioning command,
%    the \LaTeX{} solution is an optional argument:
%    \begin{quote}
%      |\section[Title]{Title\footnote{Footnote text}}|
%    \end{quote}
%    Now |Title| without the footnote is used in the headlines and
%    the table of contents. Also \Package{hyperref} uses it for the
%    bookmarks.
%
%    This package \Package{\filename} offers two possibilities to
%    specify a separate outline entry:
%    \begin{itemize}
%    \item An additional second optional argument in square brackets.
%    \item An additional optional argument in parentheses (in
%          assoziation with a pdf string that is internally surrounded
%          by parentheses, too).
%    \end{itemize}
%    Because \Package{\filename} stores the original meaning of the
%    sectioning commands and uses them again, there should be no
%    problems with packages that redefine the sectioning commands, if
%    these packages doesn't change the syntax.
%
% \subsection{Syntax}
%    The following examples show the syntax of the sectioning
%    commands. For the places the strings appear the abbreviations
%    are used, that are introduced in \ref{sec:places}.
%
% \subsubsection{\texorpdfstring{Star form}{^^A
%    \textbackslash section*\{\}}}
%    The behaviour of the star form isn't changed. The string
%    appears only in the current text:
%    \begin{article}{Syntax}
%    \begin{quote}
%      |\section*{text}|
%    \end{quote}
%    \end{article}
%
% \subsubsection{\texorpdfstring{Without optional arguments}{^^A
%    \textbackslash section\{\}}}
%    The normal case, the string in the mandatory argument is
%    used for all places:
%    \begin{article}{Syntax}
%    \begin{quote}
%      |\section{text, toc, out}|
%    \end{quote}
%    \end{article}
%
% \subsubsection{\texorpdfstring{One optional argument}{^^A
%    \textbackslash section[]\{\}}}
%    Also the form with one optional parameter in square brackets isn't
%    new; for the bookmarks the optional parameter is used:
%    \begin{article}{Syntax}
%    \begin{quote}
%      |\section[toc, out]{text}|
%    \end{quote}
%    \end{article}
%
% \subsubsection{\texorpdfstring{Two optional arguments}{^^A
%    \textbackslash section[][out]\{\}}}\label{sec:two}
%    The second optional parameter in square brackets is introduced
%    by this package to specify an outline entry:
%    \begin{article}{Syntax}
%    \begin{quote}
%      |\section[toc][out]{text}|
%    \end{quote}
%    \end{article}
%
% \subsubsection{\texorpdfstring{Optional argument in parentheses}{^^A
%    \textbackslash section(out)\{\}}}
%    Often the \type{toc} and the \type{text} string would be the same.
%    With the method of the two optional arguments in square brackets
%    (see \ref{sec:two}) this string must be given twice,
%    if the user only wants to specify a different outline entry.
%    Therefore this package offers another possibility:
%    In association with the internal representation in the pdf file
%    an outline entry can be given in parentheses.
%    So the package can easily distinguish between
%    the two forms of optional arguments and the order does not matter:
%    \begin{article}{Syntax}
%    \begin{quote}
%      |\section(out){toc, text}|\\
%      |\section[toc](out){text}|\\
%      |\section(out)[toc]{text}|
%    \end{quote}
%    \end{article}
%
% \subsection{Without \Package{hyperref}}
%    Package \Package{\filename} uses \Package{hyperref} for support of
%    the bookmarks, but this package is not required.
%    If \Package{hyperref} isn't loaded, or
%    is called with a driver that doesn't support bookmarks,
%    package \Package{\filename} shouldn't be removed,
%    because this would lead to
%    a wrong syntax of the sectioning commands.
%    In any cases package \Package{\filename}
%    supports its syntax and ignores the outline entries,
%    if there are no code for bookmarks.
%    So it is possible to write texts,
%    that are processed with several drivers to get different output
%    formats.
%
% \subsection{Protecting parentheses}
%    If the string itself contains parentheses, they have to be hidden
%    from \TeX's argument parsing mechanism.
%    The argument should be surrounded
%    by curly braces:
%    \begin{quote}
%      |\section({outlines(bookmarks)}){text, toc}|
%    \end{quote}
%    With version 6.54 of \Package{hyperref} the other standard method
%    works, too: The closing parenthesis is protected:
%    \begin{quote}
%      |\section(outlines(bookmarks{)}){text, toc}|
%    \end{quote}
%
% \StopEventually{
% }
%
% \section{Implementation}
%    \begin{macrocode}
%<*package>
%    \end{macrocode}
%    Package identification.
%    \begin{macrocode}
\NeedsTeXFormat{LaTeX2e}
\ProvidesPackage{hypbmsec}%
  [2016/05/16 v2.5 Bookmarks in sectioning commands (HO)]
%    \end{macrocode}
%
%    Because of redifining the sectioning commands, it is dangerous
%    to reload the package several times.
%    \begin{macrocode}
\@ifundefined{hbs@do}{}{%
  \PackageInfo{hypbmsec}{Package 'hypbmsec' is already loaded}%
  \endinput
}
%    \end{macrocode}
%
%    \begin{macro}{\hbs@do}
%    The redefined sectioning commands calls \cmd{\hbs@do}. It does
%    \begin{itemize}
%    \item handle the star case.
%    \item resets the macros that store the entries for the outlines
%          (\cmd{\hbs@bmstring}) and table of contents (\cmd{\hbs@tocstring}).
%    \item store the sectioning command |#1| in \cmd{\hbs@seccmd}
%          for later reuse.
%    \item at last call \cmd{\hbs@checkarg} that scans and interprets the
%          parameters of the redefined sectioning command.
%    \end{itemize}
%    \begin{macrocode}
\def\hbs@do#1{%
  \@ifstar{#1*}{%
    \let\hbs@tocstring\relax
    \let\hbs@bmstring\relax
    \let\hbs@seccmd#1%
    \hbs@checkarg
  }%
}
%    \end{macrocode}
%    \end{macro}
%
%    \begin{macro}{\hbs@checkarg}
%    \cmd{\hbs@checkarg} determines the type of the next argument:
%    \begin{itemize}
%    \item An optional argument in square brackets can be an entry
%          for the table of contents or the bookmarks. It will be
%          read by \cmd{\hbs@getsquare}
%    \item An optional argument in parentheses is an outline entry.
%          This is worked off by \cmd{\hbs@getbookmark}.
%    \item If there are no more optional arguments, \cmd{\hbs@process}
%          reads the mandatory argument and calls the original
%          sectioning commands.
%    \end{itemize}
%    \begin{macrocode}
\def\hbs@checkarg{%
  \@ifnextchar[\hbs@getsquare{%
    \@ifnextchar(\hbs@getbookmark\hbs@process
  }%
}
%    \end{macrocode}
%    \end{macro}
%
%    \begin{macro}{\hbs@getsquare}
%    \cmd{\hbs@getsquare} reads an optional argument in square
%    brackets and determines, if this is an entry for the table
%    of contents or the bookmarks.
%    \begin{macrocode}
\long\def\hbs@getsquare[#1]{%
  \ifx\hbs@tocstring\relax
    \def\hbs@tocstring{#1}%
  \else
    \hbs@bmdef{#1}%
  \fi
  \hbs@checkarg
}
%    \end{macrocode}
%    \end{macro}
%
%    \begin{macro}{\hbs@getbookmark}
%    \cmd{\hbs@getbookmark} reads an outline entry in parentheses.
%    \begin{macrocode}
\def\hbs@getbookmark(#1){%
  \hbs@bmdef{#1}%
  \hbs@checkarg
}
%    \end{macrocode}
%    \end{macro}
%
%    \begin{macro}{\hbs@bmdef}
%    The command \cmd{\hbs@bmdef} save the bookmark entry in
%    parameter |#1| in the macro \cmd{\hbs@bmstring} and catches
%    the case, if the user has given several outline strings.
%    \begin{macrocode}
\def\hbs@bmdef#1{%
  \ifx\hbs@bmstring\relax
    \def\hbs@bmstring{#1}%
  \else
    \PackageError{hypbmsec}{%
      Sectioning command with too many parameters%
    }{%
      You can only give one outline entry.%
    }%
  \fi
}
%    \end{macrocode}
%    \end{macro}
%
%    \begin{macro}{\hbs@process}
%    The parameter |#1| is the mandatory argument of the sectioning
%    commands. \cmd{\hbs@process} calls the original sectioning command
%    stored in \cmd{\hbs@seccmd} with arguments that depend of which
%    optional argument are used previously.
%    \begin{macrocode}
\long\def\hbs@process#1{%
  \ifx\hbs@tocstring\relax
    \ifx\hbs@bmstring\relax
      \hbs@seccmd{#1}%
    \else
      \begingroup
        \def\x##1{\endgroup
          \hbs@seccmd{\texorpdfstring{#1}{##1}}%
        }%
      \expandafter\x\expandafter{\hbs@bmstring}%
    \fi
  \else
    \ifx\hbs@bmstring\relax
      \expandafter\hbs@seccmd\expandafter[%
        \expandafter{\hbs@tocstring}%
      ]{#1}%
    \else
      \expandafter\expandafter\expandafter
      \hbs@seccmd\expandafter\expandafter\expandafter[%
        \expandafter\expandafter\expandafter
        \texorpdfstring
        \expandafter\expandafter\expandafter{%
          \expandafter\hbs@tocstring\expandafter
        }\expandafter{%
          \hbs@bmstring
        }%
      ]{#1}%
    \fi
  \fi
}
%    \end{macrocode}
%    \end{macro}
%
%    We have to check, whether package \Package{hyperref} is loaded
%    and have to provide a definition for \cmd{\texorpdfstring}.
%    Because \Package{hyperref} can be loaded after this package,
%    we do the work later (\cmd{\AtBeginDocument}).
%
%    This code only checks versions of \Package{hyperref} that
%    define \cmd{\ifbookmark} (v6.4x until v6.53) or
%    \cmd{\texorpdfstring} (v6.54 and above). Older versions aren't
%    supported.
%    \begin{macrocode}
\AtBeginDocument{%
  \@ifundefined{texorpdfstring}{%
    \@ifundefined{ifbookmark}{%
      \let\texorpdfstring\@firstoftwo
      \@ifpackageloaded{hyperref}{%
        \PackageInfo{hypbmsec}{%
          \ifx\hy@driver\@empty
            Default driver %
          \else
            '\hy@driver' %
          \fi
          of hyperref not supported,\MessageBreak
          bookmark parameters will be ignored%
        }%
      }{%
        \PackageInfo{hypbmsec}{%
          Package hyperref not loaded,\MessageBreak
          bookmark parameters will be ignored%
        }%
      }%
    }%
    {%
      \newcommand\texorpdfstring[2]{\ifbookmark{#2}{#1}}%
      \PackageWarningNoLine{hypbmsec}{%
        Old hyperref version found,\MessageBreak
        update of hyperref recommended%
      }%
    }%
  }{}%
%    \end{macrocode}
%
%    Other packages are allowed to redefine the sectioning commands,
%    if they does not change the syntax. Therefore the redefinitons
%    of this package should be done after the other packages.
%    \begin{macrocode}
  \let\hbs@part         \part
  \let\hbs@section      \section
  \let\hbs@subsection   \subsection
  \let\hbs@subsubsection\subsubsection
  \let\hbs@paragraph    \paragraph
  \let\hbs@subparagraph \subparagraph
  \renewcommand\part         {\hbs@do\hbs@part}%
  \renewcommand\section      {\hbs@do\hbs@section}%
  \renewcommand\subsection   {\hbs@do\hbs@subsection}%
  \renewcommand\subsubsection{\hbs@do\hbs@subsubsection}%
  \renewcommand\paragraph    {\hbs@do\hbs@paragraph}%
  \renewcommand\subparagraph {\hbs@do\hbs@subparagraph}%
  \begingroup\expandafter\expandafter\expandafter\endgroup
  \expandafter\ifx\csname chapter\endcsname\relax\else
    \let\hbs@chapter    \chapter
    \renewcommand\chapter    {\hbs@do\hbs@chapter}%
  \fi
}
%    \end{macrocode}
%
%    \begin{macrocode}
%</package>
%    \end{macrocode}
%
% \section{Installation}
%
% \subsection{Download}
%
% \paragraph{Package.} This package is available on
% CTAN\footnote{\url{http://ctan.org/pkg/hypbmsec}}:
% \begin{description}
% \item[\CTAN{macros/latex/contrib/oberdiek/hypbmsec.dtx}] The source file.
% \item[\CTAN{macros/latex/contrib/oberdiek/hypbmsec.pdf}] Documentation.
% \end{description}
%
%
% \paragraph{Bundle.} All the packages of the bundle `oberdiek'
% are also available in a TDS compliant ZIP archive. There
% the packages are already unpacked and the documentation files
% are generated. The files and directories obey the TDS standard.
% \begin{description}
% \item[\CTAN{install/macros/latex/contrib/oberdiek.tds.zip}]
% \end{description}
% \emph{TDS} refers to the standard ``A Directory Structure
% for \TeX\ Files'' (\CTAN{tds/tds.pdf}). Directories
% with \xfile{texmf} in their name are usually organized this way.
%
% \subsection{Bundle installation}
%
% \paragraph{Unpacking.} Unpack the \xfile{oberdiek.tds.zip} in the
% TDS tree (also known as \xfile{texmf} tree) of your choice.
% Example (linux):
% \begin{quote}
%   |unzip oberdiek.tds.zip -d ~/texmf|
% \end{quote}
%
% \paragraph{Script installation.}
% Check the directory \xfile{TDS:scripts/oberdiek/} for
% scripts that need further installation steps.
% Package \xpackage{attachfile2} comes with the Perl script
% \xfile{pdfatfi.pl} that should be installed in such a way
% that it can be called as \texttt{pdfatfi}.
% Example (linux):
% \begin{quote}
%   |chmod +x scripts/oberdiek/pdfatfi.pl|\\
%   |cp scripts/oberdiek/pdfatfi.pl /usr/local/bin/|
% \end{quote}
%
% \subsection{Package installation}
%
% \paragraph{Unpacking.} The \xfile{.dtx} file is a self-extracting
% \docstrip\ archive. The files are extracted by running the
% \xfile{.dtx} through \plainTeX:
% \begin{quote}
%   \verb|tex hypbmsec.dtx|
% \end{quote}
%
% \paragraph{TDS.} Now the different files must be moved into
% the different directories in your installation TDS tree
% (also known as \xfile{texmf} tree):
% \begin{quote}
% \def\t{^^A
% \begin{tabular}{@{}>{\ttfamily}l@{ $\rightarrow$ }>{\ttfamily}l@{}}
%   hypbmsec.sty & tex/latex/oberdiek/hypbmsec.sty\\
%   hypbmsec.pdf & doc/latex/oberdiek/hypbmsec.pdf\\
%   hypbmsec.dtx & source/latex/oberdiek/hypbmsec.dtx\\
% \end{tabular}^^A
% }^^A
% \sbox0{\t}^^A
% \ifdim\wd0>\linewidth
%   \begingroup
%     \advance\linewidth by\leftmargin
%     \advance\linewidth by\rightmargin
%   \edef\x{\endgroup
%     \def\noexpand\lw{\the\linewidth}^^A
%   }\x
%   \def\lwbox{^^A
%     \leavevmode
%     \hbox to \linewidth{^^A
%       \kern-\leftmargin\relax
%       \hss
%       \usebox0
%       \hss
%       \kern-\rightmargin\relax
%     }^^A
%   }^^A
%   \ifdim\wd0>\lw
%     \sbox0{\small\t}^^A
%     \ifdim\wd0>\linewidth
%       \ifdim\wd0>\lw
%         \sbox0{\footnotesize\t}^^A
%         \ifdim\wd0>\linewidth
%           \ifdim\wd0>\lw
%             \sbox0{\scriptsize\t}^^A
%             \ifdim\wd0>\linewidth
%               \ifdim\wd0>\lw
%                 \sbox0{\tiny\t}^^A
%                 \ifdim\wd0>\linewidth
%                   \lwbox
%                 \else
%                   \usebox0
%                 \fi
%               \else
%                 \lwbox
%               \fi
%             \else
%               \usebox0
%             \fi
%           \else
%             \lwbox
%           \fi
%         \else
%           \usebox0
%         \fi
%       \else
%         \lwbox
%       \fi
%     \else
%       \usebox0
%     \fi
%   \else
%     \lwbox
%   \fi
% \else
%   \usebox0
% \fi
% \end{quote}
% If you have a \xfile{docstrip.cfg} that configures and enables \docstrip's
% TDS installing feature, then some files can already be in the right
% place, see the documentation of \docstrip.
%
% \subsection{Refresh file name databases}
%
% If your \TeX~distribution
% (\teTeX, \mikTeX, \dots) relies on file name databases, you must refresh
% these. For example, \teTeX\ users run \verb|texhash| or
% \verb|mktexlsr|.
%
% \subsection{Some details for the interested}
%
% \paragraph{Attached source.}
%
% The PDF documentation on CTAN also includes the
% \xfile{.dtx} source file. It can be extracted by
% AcrobatReader 6 or higher. Another option is \textsf{pdftk},
% e.g. unpack the file into the current directory:
% \begin{quote}
%   \verb|pdftk hypbmsec.pdf unpack_files output .|
% \end{quote}
%
% \paragraph{Unpacking with \LaTeX.}
% The \xfile{.dtx} chooses its action depending on the format:
% \begin{description}
% \item[\plainTeX:] Run \docstrip\ and extract the files.
% \item[\LaTeX:] Generate the documentation.
% \end{description}
% If you insist on using \LaTeX\ for \docstrip\ (really,
% \docstrip\ does not need \LaTeX), then inform the autodetect routine
% about your intention:
% \begin{quote}
%   \verb|latex \let\install=y\input{hypbmsec.dtx}|
% \end{quote}
% Do not forget to quote the argument according to the demands
% of your shell.
%
% \paragraph{Generating the documentation.}
% You can use both the \xfile{.dtx} or the \xfile{.drv} to generate
% the documentation. The process can be configured by the
% configuration file \xfile{ltxdoc.cfg}. For instance, put this
% line into this file, if you want to have A4 as paper format:
% \begin{quote}
%   \verb|\PassOptionsToClass{a4paper}{article}|
% \end{quote}
% An example follows how to generate the
% documentation with pdf\LaTeX:
% \begin{quote}
%\begin{verbatim}
%pdflatex hypbmsec.dtx
%makeindex -s gind.ist hypbmsec.idx
%pdflatex hypbmsec.dtx
%makeindex -s gind.ist hypbmsec.idx
%pdflatex hypbmsec.dtx
%\end{verbatim}
% \end{quote}
%
% \section{Catalogue}
%
% The following XML file can be used as source for the
% \href{http://mirror.ctan.org/help/Catalogue/catalogue.html}{\TeX\ Catalogue}.
% The elements \texttt{caption} and \texttt{description} are imported
% from the original XML file from the Catalogue.
% The name of the XML file in the Catalogue is \xfile{hypbmsec.xml}.
%    \begin{macrocode}
%<*catalogue>
<?xml version='1.0' encoding='us-ascii'?>
<!DOCTYPE entry SYSTEM 'catalogue.dtd'>
<entry datestamp='$Date$' modifier='$Author$' id='hypbmsec'>
  <name>hypbmsec</name>
  <caption>Hypertext bookmarks in sectioning commands.</caption>
  <authorref id='auth:oberdiek'/>
  <copyright owner='Heiko Oberdiek' year='1998-2000,2006,2007'/>
  <license type='lppl1.3'/>
  <version number='2.5'/>
  <description>
    Bookmark entries can be given as another argument to the LaTeX
    sectioning commands. The <xref refid='hyperref'>hyperref</xref>
    package is required to get the bookmarks, but the syntax
    works without it.
    <p/>
    This package is part of the <xref refid='oberdiek'>oberdiek</xref>
    bundle.
  </description>
  <documentation details='Package documentation'
      href='ctan:/macros/latex/contrib/oberdiek/hypbmsec.pdf'/>
  <ctan file='true' path='/macros/latex/contrib/oberdiek/hypbmsec.dtx'/>
  <miktex location='oberdiek'/>
  <texlive location='oberdiek'/>
  <install path='/macros/latex/contrib/oberdiek/oberdiek.tds.zip'/>
</entry>
%</catalogue>
%    \end{macrocode}
%
% \begin{History}
%   \begin{Version}{1998/11/20 v1.0}
%   \item
%     First version.
%   \item
%     It merges package \xpackage{hysecopt} and
%   \item
%     package \xpackage{hypbmpar}.
%   \item
%     Published for the DANTE'99 meeting^^A
%     \URL{}{http://dante99.cs.uni-dortmund.de/handouts/oberdiek/hypbmsec.sty}.
%   \end{Version}
%   \begin{Version}{1999/04/12 v2.0}
%   \item
%     Adaptation to \Package{hyperref} version 6.54.
%   \item
%     Documentation in dtx format.
%   \item
%     Copyright: LPPL (\CTAN{macros/latex/base/lppl.txt})
%   \item
%     First CTAN release.
%   \end{Version}
%   \begin{Version}{2000/03/22 v2.1}
%   \item
%     Bug fix in redefinition of \cmd{\chapter}.
%   \item
%     Copyright: LPPL 1.2
%   \end{Version}
%   \begin{Version}{2006/02/20 v2.2}
%   \item
%     Code is not changed.
%   \item
%     New DTX framework.
%   \item
%     LPPL 1.3
%   \end{Version}
%   \begin{Version}{2007/03/05 v2.3}
%   \item
%     Bug fix: Expand \cs{hbs@tocstring} and \cs{hbs@bmstring} before
%     calling \cs{hbs@seccmd}.
%   \end{Version}
%   \begin{Version}{2007/04/11 v2.4}
%   \item
%     Line ends sanitized.
%   \end{Version}
%   \begin{Version}{2016/05/16 v2.5}
%   \item
%     Documentation updates.
%   \end{Version}
% \end{History}
%
% \PrintIndex
%
% \Finale
\endinput

%        (quote the arguments according to the demands of your shell)
%
% Documentation:
%    (a) If hypbmsec.drv is present:
%           latex hypbmsec.drv
%    (b) Without hypbmsec.drv:
%           latex hypbmsec.dtx; ...
%    The class ltxdoc loads the configuration file ltxdoc.cfg
%    if available. Here you can specify further options, e.g.
%    use A4 as paper format:
%       \PassOptionsToClass{a4paper}{article}
%
%    Programm calls to get the documentation (example):
%       pdflatex hypbmsec.dtx
%       makeindex -s gind.ist hypbmsec.idx
%       pdflatex hypbmsec.dtx
%       makeindex -s gind.ist hypbmsec.idx
%       pdflatex hypbmsec.dtx
%
% Installation:
%    TDS:tex/latex/oberdiek/hypbmsec.sty
%    TDS:doc/latex/oberdiek/hypbmsec.pdf
%    TDS:source/latex/oberdiek/hypbmsec.dtx
%
%<*ignore>
\begingroup
  \catcode123=1 %
  \catcode125=2 %
  \def\x{LaTeX2e}%
\expandafter\endgroup
\ifcase 0\ifx\install y1\fi\expandafter
         \ifx\csname processbatchFile\endcsname\relax\else1\fi
         \ifx\fmtname\x\else 1\fi\relax
\else\csname fi\endcsname
%</ignore>
%<*install>
\input docstrip.tex
\Msg{************************************************************************}
\Msg{* Installation}
\Msg{* Package: hypbmsec 2016/05/16 v2.5 Bookmarks in sectioning commands (HO)}
\Msg{************************************************************************}

\keepsilent
\askforoverwritefalse

\let\MetaPrefix\relax
\preamble

This is a generated file.

Project: hypbmsec
Version: 2016/05/16 v2.5

Copyright (C) 1998-2000, 2006, 2007 by
   Heiko Oberdiek <heiko.oberdiek at googlemail.com>

This work may be distributed and/or modified under the
conditions of the LaTeX Project Public License, either
version 1.3c of this license or (at your option) any later
version. This version of this license is in
   http://www.latex-project.org/lppl/lppl-1-3c.txt
and the latest version of this license is in
   http://www.latex-project.org/lppl.txt
and version 1.3 or later is part of all distributions of
LaTeX version 2005/12/01 or later.

This work has the LPPL maintenance status "maintained".

This Current Maintainer of this work is Heiko Oberdiek.

This work consists of the main source file hypbmsec.dtx
and the derived files
   hypbmsec.sty, hypbmsec.pdf, hypbmsec.ins, hypbmsec.drv.

\endpreamble
\let\MetaPrefix\DoubleperCent

\generate{%
  \file{hypbmsec.ins}{\from{hypbmsec.dtx}{install}}%
  \file{hypbmsec.drv}{\from{hypbmsec.dtx}{driver}}%
  \usedir{tex/latex/oberdiek}%
  \file{hypbmsec.sty}{\from{hypbmsec.dtx}{package}}%
  \nopreamble
  \nopostamble
  \usedir{source/latex/oberdiek/catalogue}%
  \file{hypbmsec.xml}{\from{hypbmsec.dtx}{catalogue}}%
}

\catcode32=13\relax% active space
\let =\space%
\Msg{************************************************************************}
\Msg{*}
\Msg{* To finish the installation you have to move the following}
\Msg{* file into a directory searched by TeX:}
\Msg{*}
\Msg{*     hypbmsec.sty}
\Msg{*}
\Msg{* To produce the documentation run the file `hypbmsec.drv'}
\Msg{* through LaTeX.}
\Msg{*}
\Msg{* Happy TeXing!}
\Msg{*}
\Msg{************************************************************************}

\endbatchfile
%</install>
%<*ignore>
\fi
%</ignore>
%<*driver>
\NeedsTeXFormat{LaTeX2e}
\ProvidesFile{hypbmsec.drv}%
  [2016/05/16 v2.5 Bookmarks in sectioning commands (HO)]%
\documentclass{ltxdoc}
\usepackage{holtxdoc}[2011/11/22]
\begin{document}
  \DocInput{hypbmsec.dtx}%
\end{document}
%</driver>
% \fi
%
%
% \CharacterTable
%  {Upper-case    \A\B\C\D\E\F\G\H\I\J\K\L\M\N\O\P\Q\R\S\T\U\V\W\X\Y\Z
%   Lower-case    \a\b\c\d\e\f\g\h\i\j\k\l\m\n\o\p\q\r\s\t\u\v\w\x\y\z
%   Digits        \0\1\2\3\4\5\6\7\8\9
%   Exclamation   \!     Double quote  \"     Hash (number) \#
%   Dollar        \$     Percent       \%     Ampersand     \&
%   Acute accent  \'     Left paren    \(     Right paren   \)
%   Asterisk      \*     Plus          \+     Comma         \,
%   Minus         \-     Point         \.     Solidus       \/
%   Colon         \:     Semicolon     \;     Less than     \<
%   Equals        \=     Greater than  \>     Question mark \?
%   Commercial at \@     Left bracket  \[     Backslash     \\
%   Right bracket \]     Circumflex    \^     Underscore    \_
%   Grave accent  \`     Left brace    \{     Vertical bar  \|
%   Right brace   \}     Tilde         \~}
%
% \GetFileInfo{hypbmsec.drv}
%
% \title{The \xpackage{hypbmsec} package}
% \date{2016/05/16 v2.5}
% \author{Heiko Oberdiek\thanks
% {Please report any issues at https://github.com/ho-tex/oberdiek/issues}\\
% \xemail{heiko.oberdiek at googlemail.com}}
%
% \maketitle
%
% \begin{abstract}
% This package expands the syntax of the sectioning commands. If the
% argument of the sectioning commands isn't usable as outline entry,
% a replacement for the bookmarks can be given.
% \end{abstract}
%
% \tableofcontents
%
% \newcommand{\type}[1]{\textsf{#1}}
%
% ^^A No thread support.
% \newenvironment{article}[1]{}{}
%
% \section{Usage}
%
% \subsection{Bookmarks restrictions}\label{sec:restrictions}
%    Outline entries (bookmarks) are written to a file and have
%    to obey the pdf specification.
%    Therefore they have several restrictions:
%    \begin{itemize}
%    \item Bookmarks have to be encoded in PDFDocEncoding^^A
%          \footnote{\Package{hyperref} doesn't support
%            Unicode.}.
%    \item They should only expand to letters and spaces.
%    \item The result of expansion have to be a valid pdf string.
%    \item Stomach commands like \cmd{\relax}, box commands, math,
%          assignments, or definitions aren't allowed.
%    \item Short entries are recommended, which allow a clear view.
%    \end{itemize}
%
% \subsection{\texorpdfstring{\cmd{\texorpdfstring}}{^^A
%    \textbackslash texorpdfstring}}
%    The generic way in package \Package{hyperref} is the use
%    of \cmd{\texorpdfstring}^^A
%    \footnote{In versions of \Package{hyperref} below 6.54 see
%      \cmd{\ifbookmark}.}:
%    \begin{quote}
%\begin{verbatim}
%\section{Pythagoras:
%  \texorpdfstring{$a^2+b^2=c^2}{%
%    a\texttwosuperior\ + b\texttwosuperior\ =
%    c\texttwosuperior}%
%}
%\end{verbatim}
%    \end{quote}
%
% \subsection{Sectioning commands}
%    The package \Package{hyperref} automatically generates
%    bookmarks from the sectioning commands,
%    unless it is suppressed by an option.
%    Commands that structure the text are here called
%    ``sectioning commands'':
%    \begin{quote}
%    \cmd{\part}, \cmd{\chapter},\\
%    \cmd{\section}, \cmd{\subsection}, \cmd{\subsubsection},\\
%    \cmd{\paragraph}, \cmd{\subparagraph}
%    \end{quote}
%
% \subsection{Places\texorpdfstring{ for sectioning strings}{}}
%    \label{sec:places}
%    The argument(s) of these commands are used on several places:
%    \begin{description}
%    \item[\type{text}]
%      The current text without restrictions.
%    \item[\type{toc}]
%      The headlines and the table of contents with the
%      restrictions of ``moving arguments''.
%    \item[\type{out}]
%      The outlines with many restrictions: The outline
%      have to expand to a valid pdf string with PDFDocEncoding
%      (see \ref{sec:restrictions}).
%    \end{description}
%
% \subsection{\texorpdfstring{Solution with o}{O}ptional arguments}
%    If the user wants to use a footnote within a sectioning command,
%    the \LaTeX{} solution is an optional argument:
%    \begin{quote}
%      |\section[Title]{Title\footnote{Footnote text}}|
%    \end{quote}
%    Now |Title| without the footnote is used in the headlines and
%    the table of contents. Also \Package{hyperref} uses it for the
%    bookmarks.
%
%    This package \Package{\filename} offers two possibilities to
%    specify a separate outline entry:
%    \begin{itemize}
%    \item An additional second optional argument in square brackets.
%    \item An additional optional argument in parentheses (in
%          assoziation with a pdf string that is internally surrounded
%          by parentheses, too).
%    \end{itemize}
%    Because \Package{\filename} stores the original meaning of the
%    sectioning commands and uses them again, there should be no
%    problems with packages that redefine the sectioning commands, if
%    these packages doesn't change the syntax.
%
% \subsection{Syntax}
%    The following examples show the syntax of the sectioning
%    commands. For the places the strings appear the abbreviations
%    are used, that are introduced in \ref{sec:places}.
%
% \subsubsection{\texorpdfstring{Star form}{^^A
%    \textbackslash section*\{\}}}
%    The behaviour of the star form isn't changed. The string
%    appears only in the current text:
%    \begin{article}{Syntax}
%    \begin{quote}
%      |\section*{text}|
%    \end{quote}
%    \end{article}
%
% \subsubsection{\texorpdfstring{Without optional arguments}{^^A
%    \textbackslash section\{\}}}
%    The normal case, the string in the mandatory argument is
%    used for all places:
%    \begin{article}{Syntax}
%    \begin{quote}
%      |\section{text, toc, out}|
%    \end{quote}
%    \end{article}
%
% \subsubsection{\texorpdfstring{One optional argument}{^^A
%    \textbackslash section[]\{\}}}
%    Also the form with one optional parameter in square brackets isn't
%    new; for the bookmarks the optional parameter is used:
%    \begin{article}{Syntax}
%    \begin{quote}
%      |\section[toc, out]{text}|
%    \end{quote}
%    \end{article}
%
% \subsubsection{\texorpdfstring{Two optional arguments}{^^A
%    \textbackslash section[][out]\{\}}}\label{sec:two}
%    The second optional parameter in square brackets is introduced
%    by this package to specify an outline entry:
%    \begin{article}{Syntax}
%    \begin{quote}
%      |\section[toc][out]{text}|
%    \end{quote}
%    \end{article}
%
% \subsubsection{\texorpdfstring{Optional argument in parentheses}{^^A
%    \textbackslash section(out)\{\}}}
%    Often the \type{toc} and the \type{text} string would be the same.
%    With the method of the two optional arguments in square brackets
%    (see \ref{sec:two}) this string must be given twice,
%    if the user only wants to specify a different outline entry.
%    Therefore this package offers another possibility:
%    In association with the internal representation in the pdf file
%    an outline entry can be given in parentheses.
%    So the package can easily distinguish between
%    the two forms of optional arguments and the order does not matter:
%    \begin{article}{Syntax}
%    \begin{quote}
%      |\section(out){toc, text}|\\
%      |\section[toc](out){text}|\\
%      |\section(out)[toc]{text}|
%    \end{quote}
%    \end{article}
%
% \subsection{Without \Package{hyperref}}
%    Package \Package{\filename} uses \Package{hyperref} for support of
%    the bookmarks, but this package is not required.
%    If \Package{hyperref} isn't loaded, or
%    is called with a driver that doesn't support bookmarks,
%    package \Package{\filename} shouldn't be removed,
%    because this would lead to
%    a wrong syntax of the sectioning commands.
%    In any cases package \Package{\filename}
%    supports its syntax and ignores the outline entries,
%    if there are no code for bookmarks.
%    So it is possible to write texts,
%    that are processed with several drivers to get different output
%    formats.
%
% \subsection{Protecting parentheses}
%    If the string itself contains parentheses, they have to be hidden
%    from \TeX's argument parsing mechanism.
%    The argument should be surrounded
%    by curly braces:
%    \begin{quote}
%      |\section({outlines(bookmarks)}){text, toc}|
%    \end{quote}
%    With version 6.54 of \Package{hyperref} the other standard method
%    works, too: The closing parenthesis is protected:
%    \begin{quote}
%      |\section(outlines(bookmarks{)}){text, toc}|
%    \end{quote}
%
% \StopEventually{
% }
%
% \section{Implementation}
%    \begin{macrocode}
%<*package>
%    \end{macrocode}
%    Package identification.
%    \begin{macrocode}
\NeedsTeXFormat{LaTeX2e}
\ProvidesPackage{hypbmsec}%
  [2016/05/16 v2.5 Bookmarks in sectioning commands (HO)]
%    \end{macrocode}
%
%    Because of redifining the sectioning commands, it is dangerous
%    to reload the package several times.
%    \begin{macrocode}
\@ifundefined{hbs@do}{}{%
  \PackageInfo{hypbmsec}{Package 'hypbmsec' is already loaded}%
  \endinput
}
%    \end{macrocode}
%
%    \begin{macro}{\hbs@do}
%    The redefined sectioning commands calls \cmd{\hbs@do}. It does
%    \begin{itemize}
%    \item handle the star case.
%    \item resets the macros that store the entries for the outlines
%          (\cmd{\hbs@bmstring}) and table of contents (\cmd{\hbs@tocstring}).
%    \item store the sectioning command |#1| in \cmd{\hbs@seccmd}
%          for later reuse.
%    \item at last call \cmd{\hbs@checkarg} that scans and interprets the
%          parameters of the redefined sectioning command.
%    \end{itemize}
%    \begin{macrocode}
\def\hbs@do#1{%
  \@ifstar{#1*}{%
    \let\hbs@tocstring\relax
    \let\hbs@bmstring\relax
    \let\hbs@seccmd#1%
    \hbs@checkarg
  }%
}
%    \end{macrocode}
%    \end{macro}
%
%    \begin{macro}{\hbs@checkarg}
%    \cmd{\hbs@checkarg} determines the type of the next argument:
%    \begin{itemize}
%    \item An optional argument in square brackets can be an entry
%          for the table of contents or the bookmarks. It will be
%          read by \cmd{\hbs@getsquare}
%    \item An optional argument in parentheses is an outline entry.
%          This is worked off by \cmd{\hbs@getbookmark}.
%    \item If there are no more optional arguments, \cmd{\hbs@process}
%          reads the mandatory argument and calls the original
%          sectioning commands.
%    \end{itemize}
%    \begin{macrocode}
\def\hbs@checkarg{%
  \@ifnextchar[\hbs@getsquare{%
    \@ifnextchar(\hbs@getbookmark\hbs@process
  }%
}
%    \end{macrocode}
%    \end{macro}
%
%    \begin{macro}{\hbs@getsquare}
%    \cmd{\hbs@getsquare} reads an optional argument in square
%    brackets and determines, if this is an entry for the table
%    of contents or the bookmarks.
%    \begin{macrocode}
\long\def\hbs@getsquare[#1]{%
  \ifx\hbs@tocstring\relax
    \def\hbs@tocstring{#1}%
  \else
    \hbs@bmdef{#1}%
  \fi
  \hbs@checkarg
}
%    \end{macrocode}
%    \end{macro}
%
%    \begin{macro}{\hbs@getbookmark}
%    \cmd{\hbs@getbookmark} reads an outline entry in parentheses.
%    \begin{macrocode}
\def\hbs@getbookmark(#1){%
  \hbs@bmdef{#1}%
  \hbs@checkarg
}
%    \end{macrocode}
%    \end{macro}
%
%    \begin{macro}{\hbs@bmdef}
%    The command \cmd{\hbs@bmdef} save the bookmark entry in
%    parameter |#1| in the macro \cmd{\hbs@bmstring} and catches
%    the case, if the user has given several outline strings.
%    \begin{macrocode}
\def\hbs@bmdef#1{%
  \ifx\hbs@bmstring\relax
    \def\hbs@bmstring{#1}%
  \else
    \PackageError{hypbmsec}{%
      Sectioning command with too many parameters%
    }{%
      You can only give one outline entry.%
    }%
  \fi
}
%    \end{macrocode}
%    \end{macro}
%
%    \begin{macro}{\hbs@process}
%    The parameter |#1| is the mandatory argument of the sectioning
%    commands. \cmd{\hbs@process} calls the original sectioning command
%    stored in \cmd{\hbs@seccmd} with arguments that depend of which
%    optional argument are used previously.
%    \begin{macrocode}
\long\def\hbs@process#1{%
  \ifx\hbs@tocstring\relax
    \ifx\hbs@bmstring\relax
      \hbs@seccmd{#1}%
    \else
      \begingroup
        \def\x##1{\endgroup
          \hbs@seccmd{\texorpdfstring{#1}{##1}}%
        }%
      \expandafter\x\expandafter{\hbs@bmstring}%
    \fi
  \else
    \ifx\hbs@bmstring\relax
      \expandafter\hbs@seccmd\expandafter[%
        \expandafter{\hbs@tocstring}%
      ]{#1}%
    \else
      \expandafter\expandafter\expandafter
      \hbs@seccmd\expandafter\expandafter\expandafter[%
        \expandafter\expandafter\expandafter
        \texorpdfstring
        \expandafter\expandafter\expandafter{%
          \expandafter\hbs@tocstring\expandafter
        }\expandafter{%
          \hbs@bmstring
        }%
      ]{#1}%
    \fi
  \fi
}
%    \end{macrocode}
%    \end{macro}
%
%    We have to check, whether package \Package{hyperref} is loaded
%    and have to provide a definition for \cmd{\texorpdfstring}.
%    Because \Package{hyperref} can be loaded after this package,
%    we do the work later (\cmd{\AtBeginDocument}).
%
%    This code only checks versions of \Package{hyperref} that
%    define \cmd{\ifbookmark} (v6.4x until v6.53) or
%    \cmd{\texorpdfstring} (v6.54 and above). Older versions aren't
%    supported.
%    \begin{macrocode}
\AtBeginDocument{%
  \@ifundefined{texorpdfstring}{%
    \@ifundefined{ifbookmark}{%
      \let\texorpdfstring\@firstoftwo
      \@ifpackageloaded{hyperref}{%
        \PackageInfo{hypbmsec}{%
          \ifx\hy@driver\@empty
            Default driver %
          \else
            '\hy@driver' %
          \fi
          of hyperref not supported,\MessageBreak
          bookmark parameters will be ignored%
        }%
      }{%
        \PackageInfo{hypbmsec}{%
          Package hyperref not loaded,\MessageBreak
          bookmark parameters will be ignored%
        }%
      }%
    }%
    {%
      \newcommand\texorpdfstring[2]{\ifbookmark{#2}{#1}}%
      \PackageWarningNoLine{hypbmsec}{%
        Old hyperref version found,\MessageBreak
        update of hyperref recommended%
      }%
    }%
  }{}%
%    \end{macrocode}
%
%    Other packages are allowed to redefine the sectioning commands,
%    if they does not change the syntax. Therefore the redefinitons
%    of this package should be done after the other packages.
%    \begin{macrocode}
  \let\hbs@part         \part
  \let\hbs@section      \section
  \let\hbs@subsection   \subsection
  \let\hbs@subsubsection\subsubsection
  \let\hbs@paragraph    \paragraph
  \let\hbs@subparagraph \subparagraph
  \renewcommand\part         {\hbs@do\hbs@part}%
  \renewcommand\section      {\hbs@do\hbs@section}%
  \renewcommand\subsection   {\hbs@do\hbs@subsection}%
  \renewcommand\subsubsection{\hbs@do\hbs@subsubsection}%
  \renewcommand\paragraph    {\hbs@do\hbs@paragraph}%
  \renewcommand\subparagraph {\hbs@do\hbs@subparagraph}%
  \begingroup\expandafter\expandafter\expandafter\endgroup
  \expandafter\ifx\csname chapter\endcsname\relax\else
    \let\hbs@chapter    \chapter
    \renewcommand\chapter    {\hbs@do\hbs@chapter}%
  \fi
}
%    \end{macrocode}
%
%    \begin{macrocode}
%</package>
%    \end{macrocode}
%
% \section{Installation}
%
% \subsection{Download}
%
% \paragraph{Package.} This package is available on
% CTAN\footnote{\url{http://ctan.org/pkg/hypbmsec}}:
% \begin{description}
% \item[\CTAN{macros/latex/contrib/oberdiek/hypbmsec.dtx}] The source file.
% \item[\CTAN{macros/latex/contrib/oberdiek/hypbmsec.pdf}] Documentation.
% \end{description}
%
%
% \paragraph{Bundle.} All the packages of the bundle `oberdiek'
% are also available in a TDS compliant ZIP archive. There
% the packages are already unpacked and the documentation files
% are generated. The files and directories obey the TDS standard.
% \begin{description}
% \item[\CTAN{install/macros/latex/contrib/oberdiek.tds.zip}]
% \end{description}
% \emph{TDS} refers to the standard ``A Directory Structure
% for \TeX\ Files'' (\CTAN{tds/tds.pdf}). Directories
% with \xfile{texmf} in their name are usually organized this way.
%
% \subsection{Bundle installation}
%
% \paragraph{Unpacking.} Unpack the \xfile{oberdiek.tds.zip} in the
% TDS tree (also known as \xfile{texmf} tree) of your choice.
% Example (linux):
% \begin{quote}
%   |unzip oberdiek.tds.zip -d ~/texmf|
% \end{quote}
%
% \paragraph{Script installation.}
% Check the directory \xfile{TDS:scripts/oberdiek/} for
% scripts that need further installation steps.
% Package \xpackage{attachfile2} comes with the Perl script
% \xfile{pdfatfi.pl} that should be installed in such a way
% that it can be called as \texttt{pdfatfi}.
% Example (linux):
% \begin{quote}
%   |chmod +x scripts/oberdiek/pdfatfi.pl|\\
%   |cp scripts/oberdiek/pdfatfi.pl /usr/local/bin/|
% \end{quote}
%
% \subsection{Package installation}
%
% \paragraph{Unpacking.} The \xfile{.dtx} file is a self-extracting
% \docstrip\ archive. The files are extracted by running the
% \xfile{.dtx} through \plainTeX:
% \begin{quote}
%   \verb|tex hypbmsec.dtx|
% \end{quote}
%
% \paragraph{TDS.} Now the different files must be moved into
% the different directories in your installation TDS tree
% (also known as \xfile{texmf} tree):
% \begin{quote}
% \def\t{^^A
% \begin{tabular}{@{}>{\ttfamily}l@{ $\rightarrow$ }>{\ttfamily}l@{}}
%   hypbmsec.sty & tex/latex/oberdiek/hypbmsec.sty\\
%   hypbmsec.pdf & doc/latex/oberdiek/hypbmsec.pdf\\
%   hypbmsec.dtx & source/latex/oberdiek/hypbmsec.dtx\\
% \end{tabular}^^A
% }^^A
% \sbox0{\t}^^A
% \ifdim\wd0>\linewidth
%   \begingroup
%     \advance\linewidth by\leftmargin
%     \advance\linewidth by\rightmargin
%   \edef\x{\endgroup
%     \def\noexpand\lw{\the\linewidth}^^A
%   }\x
%   \def\lwbox{^^A
%     \leavevmode
%     \hbox to \linewidth{^^A
%       \kern-\leftmargin\relax
%       \hss
%       \usebox0
%       \hss
%       \kern-\rightmargin\relax
%     }^^A
%   }^^A
%   \ifdim\wd0>\lw
%     \sbox0{\small\t}^^A
%     \ifdim\wd0>\linewidth
%       \ifdim\wd0>\lw
%         \sbox0{\footnotesize\t}^^A
%         \ifdim\wd0>\linewidth
%           \ifdim\wd0>\lw
%             \sbox0{\scriptsize\t}^^A
%             \ifdim\wd0>\linewidth
%               \ifdim\wd0>\lw
%                 \sbox0{\tiny\t}^^A
%                 \ifdim\wd0>\linewidth
%                   \lwbox
%                 \else
%                   \usebox0
%                 \fi
%               \else
%                 \lwbox
%               \fi
%             \else
%               \usebox0
%             \fi
%           \else
%             \lwbox
%           \fi
%         \else
%           \usebox0
%         \fi
%       \else
%         \lwbox
%       \fi
%     \else
%       \usebox0
%     \fi
%   \else
%     \lwbox
%   \fi
% \else
%   \usebox0
% \fi
% \end{quote}
% If you have a \xfile{docstrip.cfg} that configures and enables \docstrip's
% TDS installing feature, then some files can already be in the right
% place, see the documentation of \docstrip.
%
% \subsection{Refresh file name databases}
%
% If your \TeX~distribution
% (\teTeX, \mikTeX, \dots) relies on file name databases, you must refresh
% these. For example, \teTeX\ users run \verb|texhash| or
% \verb|mktexlsr|.
%
% \subsection{Some details for the interested}
%
% \paragraph{Attached source.}
%
% The PDF documentation on CTAN also includes the
% \xfile{.dtx} source file. It can be extracted by
% AcrobatReader 6 or higher. Another option is \textsf{pdftk},
% e.g. unpack the file into the current directory:
% \begin{quote}
%   \verb|pdftk hypbmsec.pdf unpack_files output .|
% \end{quote}
%
% \paragraph{Unpacking with \LaTeX.}
% The \xfile{.dtx} chooses its action depending on the format:
% \begin{description}
% \item[\plainTeX:] Run \docstrip\ and extract the files.
% \item[\LaTeX:] Generate the documentation.
% \end{description}
% If you insist on using \LaTeX\ for \docstrip\ (really,
% \docstrip\ does not need \LaTeX), then inform the autodetect routine
% about your intention:
% \begin{quote}
%   \verb|latex \let\install=y% \iffalse meta-comment
%
% File: hypbmsec.dtx
% Version: 2016/05/16 v2.5
% Info: Bookmarks in sectioning commands
%
% Copyright (C) 1998-2000, 2006, 2007 by
%    Heiko Oberdiek <heiko.oberdiek at googlemail.com>
%    2016
%    https://github.com/ho-tex/oberdiek/issues
%
% This work may be distributed and/or modified under the
% conditions of the LaTeX Project Public License, either
% version 1.3c of this license or (at your option) any later
% version. This version of this license is in
%    http://www.latex-project.org/lppl/lppl-1-3c.txt
% and the latest version of this license is in
%    http://www.latex-project.org/lppl.txt
% and version 1.3 or later is part of all distributions of
% LaTeX version 2005/12/01 or later.
%
% This work has the LPPL maintenance status "maintained".
%
% This Current Maintainer of this work is Heiko Oberdiek.
%
% This work consists of the main source file hypbmsec.dtx
% and the derived files
%    hypbmsec.sty, hypbmsec.pdf, hypbmsec.ins, hypbmsec.drv.
%
% Distribution:
%    CTAN:macros/latex/contrib/oberdiek/hypbmsec.dtx
%    CTAN:macros/latex/contrib/oberdiek/hypbmsec.pdf
%
% Unpacking:
%    (a) If hypbmsec.ins is present:
%           tex hypbmsec.ins
%    (b) Without hypbmsec.ins:
%           tex hypbmsec.dtx
%    (c) If you insist on using LaTeX
%           latex \let\install=y\input{hypbmsec.dtx}
%        (quote the arguments according to the demands of your shell)
%
% Documentation:
%    (a) If hypbmsec.drv is present:
%           latex hypbmsec.drv
%    (b) Without hypbmsec.drv:
%           latex hypbmsec.dtx; ...
%    The class ltxdoc loads the configuration file ltxdoc.cfg
%    if available. Here you can specify further options, e.g.
%    use A4 as paper format:
%       \PassOptionsToClass{a4paper}{article}
%
%    Programm calls to get the documentation (example):
%       pdflatex hypbmsec.dtx
%       makeindex -s gind.ist hypbmsec.idx
%       pdflatex hypbmsec.dtx
%       makeindex -s gind.ist hypbmsec.idx
%       pdflatex hypbmsec.dtx
%
% Installation:
%    TDS:tex/latex/oberdiek/hypbmsec.sty
%    TDS:doc/latex/oberdiek/hypbmsec.pdf
%    TDS:source/latex/oberdiek/hypbmsec.dtx
%
%<*ignore>
\begingroup
  \catcode123=1 %
  \catcode125=2 %
  \def\x{LaTeX2e}%
\expandafter\endgroup
\ifcase 0\ifx\install y1\fi\expandafter
         \ifx\csname processbatchFile\endcsname\relax\else1\fi
         \ifx\fmtname\x\else 1\fi\relax
\else\csname fi\endcsname
%</ignore>
%<*install>
\input docstrip.tex
\Msg{************************************************************************}
\Msg{* Installation}
\Msg{* Package: hypbmsec 2016/05/16 v2.5 Bookmarks in sectioning commands (HO)}
\Msg{************************************************************************}

\keepsilent
\askforoverwritefalse

\let\MetaPrefix\relax
\preamble

This is a generated file.

Project: hypbmsec
Version: 2016/05/16 v2.5

Copyright (C) 1998-2000, 2006, 2007 by
   Heiko Oberdiek <heiko.oberdiek at googlemail.com>

This work may be distributed and/or modified under the
conditions of the LaTeX Project Public License, either
version 1.3c of this license or (at your option) any later
version. This version of this license is in
   http://www.latex-project.org/lppl/lppl-1-3c.txt
and the latest version of this license is in
   http://www.latex-project.org/lppl.txt
and version 1.3 or later is part of all distributions of
LaTeX version 2005/12/01 or later.

This work has the LPPL maintenance status "maintained".

This Current Maintainer of this work is Heiko Oberdiek.

This work consists of the main source file hypbmsec.dtx
and the derived files
   hypbmsec.sty, hypbmsec.pdf, hypbmsec.ins, hypbmsec.drv.

\endpreamble
\let\MetaPrefix\DoubleperCent

\generate{%
  \file{hypbmsec.ins}{\from{hypbmsec.dtx}{install}}%
  \file{hypbmsec.drv}{\from{hypbmsec.dtx}{driver}}%
  \usedir{tex/latex/oberdiek}%
  \file{hypbmsec.sty}{\from{hypbmsec.dtx}{package}}%
  \nopreamble
  \nopostamble
  \usedir{source/latex/oberdiek/catalogue}%
  \file{hypbmsec.xml}{\from{hypbmsec.dtx}{catalogue}}%
}

\catcode32=13\relax% active space
\let =\space%
\Msg{************************************************************************}
\Msg{*}
\Msg{* To finish the installation you have to move the following}
\Msg{* file into a directory searched by TeX:}
\Msg{*}
\Msg{*     hypbmsec.sty}
\Msg{*}
\Msg{* To produce the documentation run the file `hypbmsec.drv'}
\Msg{* through LaTeX.}
\Msg{*}
\Msg{* Happy TeXing!}
\Msg{*}
\Msg{************************************************************************}

\endbatchfile
%</install>
%<*ignore>
\fi
%</ignore>
%<*driver>
\NeedsTeXFormat{LaTeX2e}
\ProvidesFile{hypbmsec.drv}%
  [2016/05/16 v2.5 Bookmarks in sectioning commands (HO)]%
\documentclass{ltxdoc}
\usepackage{holtxdoc}[2011/11/22]
\begin{document}
  \DocInput{hypbmsec.dtx}%
\end{document}
%</driver>
% \fi
%
%
% \CharacterTable
%  {Upper-case    \A\B\C\D\E\F\G\H\I\J\K\L\M\N\O\P\Q\R\S\T\U\V\W\X\Y\Z
%   Lower-case    \a\b\c\d\e\f\g\h\i\j\k\l\m\n\o\p\q\r\s\t\u\v\w\x\y\z
%   Digits        \0\1\2\3\4\5\6\7\8\9
%   Exclamation   \!     Double quote  \"     Hash (number) \#
%   Dollar        \$     Percent       \%     Ampersand     \&
%   Acute accent  \'     Left paren    \(     Right paren   \)
%   Asterisk      \*     Plus          \+     Comma         \,
%   Minus         \-     Point         \.     Solidus       \/
%   Colon         \:     Semicolon     \;     Less than     \<
%   Equals        \=     Greater than  \>     Question mark \?
%   Commercial at \@     Left bracket  \[     Backslash     \\
%   Right bracket \]     Circumflex    \^     Underscore    \_
%   Grave accent  \`     Left brace    \{     Vertical bar  \|
%   Right brace   \}     Tilde         \~}
%
% \GetFileInfo{hypbmsec.drv}
%
% \title{The \xpackage{hypbmsec} package}
% \date{2016/05/16 v2.5}
% \author{Heiko Oberdiek\thanks
% {Please report any issues at https://github.com/ho-tex/oberdiek/issues}\\
% \xemail{heiko.oberdiek at googlemail.com}}
%
% \maketitle
%
% \begin{abstract}
% This package expands the syntax of the sectioning commands. If the
% argument of the sectioning commands isn't usable as outline entry,
% a replacement for the bookmarks can be given.
% \end{abstract}
%
% \tableofcontents
%
% \newcommand{\type}[1]{\textsf{#1}}
%
% ^^A No thread support.
% \newenvironment{article}[1]{}{}
%
% \section{Usage}
%
% \subsection{Bookmarks restrictions}\label{sec:restrictions}
%    Outline entries (bookmarks) are written to a file and have
%    to obey the pdf specification.
%    Therefore they have several restrictions:
%    \begin{itemize}
%    \item Bookmarks have to be encoded in PDFDocEncoding^^A
%          \footnote{\Package{hyperref} doesn't support
%            Unicode.}.
%    \item They should only expand to letters and spaces.
%    \item The result of expansion have to be a valid pdf string.
%    \item Stomach commands like \cmd{\relax}, box commands, math,
%          assignments, or definitions aren't allowed.
%    \item Short entries are recommended, which allow a clear view.
%    \end{itemize}
%
% \subsection{\texorpdfstring{\cmd{\texorpdfstring}}{^^A
%    \textbackslash texorpdfstring}}
%    The generic way in package \Package{hyperref} is the use
%    of \cmd{\texorpdfstring}^^A
%    \footnote{In versions of \Package{hyperref} below 6.54 see
%      \cmd{\ifbookmark}.}:
%    \begin{quote}
%\begin{verbatim}
%\section{Pythagoras:
%  \texorpdfstring{$a^2+b^2=c^2}{%
%    a\texttwosuperior\ + b\texttwosuperior\ =
%    c\texttwosuperior}%
%}
%\end{verbatim}
%    \end{quote}
%
% \subsection{Sectioning commands}
%    The package \Package{hyperref} automatically generates
%    bookmarks from the sectioning commands,
%    unless it is suppressed by an option.
%    Commands that structure the text are here called
%    ``sectioning commands'':
%    \begin{quote}
%    \cmd{\part}, \cmd{\chapter},\\
%    \cmd{\section}, \cmd{\subsection}, \cmd{\subsubsection},\\
%    \cmd{\paragraph}, \cmd{\subparagraph}
%    \end{quote}
%
% \subsection{Places\texorpdfstring{ for sectioning strings}{}}
%    \label{sec:places}
%    The argument(s) of these commands are used on several places:
%    \begin{description}
%    \item[\type{text}]
%      The current text without restrictions.
%    \item[\type{toc}]
%      The headlines and the table of contents with the
%      restrictions of ``moving arguments''.
%    \item[\type{out}]
%      The outlines with many restrictions: The outline
%      have to expand to a valid pdf string with PDFDocEncoding
%      (see \ref{sec:restrictions}).
%    \end{description}
%
% \subsection{\texorpdfstring{Solution with o}{O}ptional arguments}
%    If the user wants to use a footnote within a sectioning command,
%    the \LaTeX{} solution is an optional argument:
%    \begin{quote}
%      |\section[Title]{Title\footnote{Footnote text}}|
%    \end{quote}
%    Now |Title| without the footnote is used in the headlines and
%    the table of contents. Also \Package{hyperref} uses it for the
%    bookmarks.
%
%    This package \Package{\filename} offers two possibilities to
%    specify a separate outline entry:
%    \begin{itemize}
%    \item An additional second optional argument in square brackets.
%    \item An additional optional argument in parentheses (in
%          assoziation with a pdf string that is internally surrounded
%          by parentheses, too).
%    \end{itemize}
%    Because \Package{\filename} stores the original meaning of the
%    sectioning commands and uses them again, there should be no
%    problems with packages that redefine the sectioning commands, if
%    these packages doesn't change the syntax.
%
% \subsection{Syntax}
%    The following examples show the syntax of the sectioning
%    commands. For the places the strings appear the abbreviations
%    are used, that are introduced in \ref{sec:places}.
%
% \subsubsection{\texorpdfstring{Star form}{^^A
%    \textbackslash section*\{\}}}
%    The behaviour of the star form isn't changed. The string
%    appears only in the current text:
%    \begin{article}{Syntax}
%    \begin{quote}
%      |\section*{text}|
%    \end{quote}
%    \end{article}
%
% \subsubsection{\texorpdfstring{Without optional arguments}{^^A
%    \textbackslash section\{\}}}
%    The normal case, the string in the mandatory argument is
%    used for all places:
%    \begin{article}{Syntax}
%    \begin{quote}
%      |\section{text, toc, out}|
%    \end{quote}
%    \end{article}
%
% \subsubsection{\texorpdfstring{One optional argument}{^^A
%    \textbackslash section[]\{\}}}
%    Also the form with one optional parameter in square brackets isn't
%    new; for the bookmarks the optional parameter is used:
%    \begin{article}{Syntax}
%    \begin{quote}
%      |\section[toc, out]{text}|
%    \end{quote}
%    \end{article}
%
% \subsubsection{\texorpdfstring{Two optional arguments}{^^A
%    \textbackslash section[][out]\{\}}}\label{sec:two}
%    The second optional parameter in square brackets is introduced
%    by this package to specify an outline entry:
%    \begin{article}{Syntax}
%    \begin{quote}
%      |\section[toc][out]{text}|
%    \end{quote}
%    \end{article}
%
% \subsubsection{\texorpdfstring{Optional argument in parentheses}{^^A
%    \textbackslash section(out)\{\}}}
%    Often the \type{toc} and the \type{text} string would be the same.
%    With the method of the two optional arguments in square brackets
%    (see \ref{sec:two}) this string must be given twice,
%    if the user only wants to specify a different outline entry.
%    Therefore this package offers another possibility:
%    In association with the internal representation in the pdf file
%    an outline entry can be given in parentheses.
%    So the package can easily distinguish between
%    the two forms of optional arguments and the order does not matter:
%    \begin{article}{Syntax}
%    \begin{quote}
%      |\section(out){toc, text}|\\
%      |\section[toc](out){text}|\\
%      |\section(out)[toc]{text}|
%    \end{quote}
%    \end{article}
%
% \subsection{Without \Package{hyperref}}
%    Package \Package{\filename} uses \Package{hyperref} for support of
%    the bookmarks, but this package is not required.
%    If \Package{hyperref} isn't loaded, or
%    is called with a driver that doesn't support bookmarks,
%    package \Package{\filename} shouldn't be removed,
%    because this would lead to
%    a wrong syntax of the sectioning commands.
%    In any cases package \Package{\filename}
%    supports its syntax and ignores the outline entries,
%    if there are no code for bookmarks.
%    So it is possible to write texts,
%    that are processed with several drivers to get different output
%    formats.
%
% \subsection{Protecting parentheses}
%    If the string itself contains parentheses, they have to be hidden
%    from \TeX's argument parsing mechanism.
%    The argument should be surrounded
%    by curly braces:
%    \begin{quote}
%      |\section({outlines(bookmarks)}){text, toc}|
%    \end{quote}
%    With version 6.54 of \Package{hyperref} the other standard method
%    works, too: The closing parenthesis is protected:
%    \begin{quote}
%      |\section(outlines(bookmarks{)}){text, toc}|
%    \end{quote}
%
% \StopEventually{
% }
%
% \section{Implementation}
%    \begin{macrocode}
%<*package>
%    \end{macrocode}
%    Package identification.
%    \begin{macrocode}
\NeedsTeXFormat{LaTeX2e}
\ProvidesPackage{hypbmsec}%
  [2016/05/16 v2.5 Bookmarks in sectioning commands (HO)]
%    \end{macrocode}
%
%    Because of redifining the sectioning commands, it is dangerous
%    to reload the package several times.
%    \begin{macrocode}
\@ifundefined{hbs@do}{}{%
  \PackageInfo{hypbmsec}{Package 'hypbmsec' is already loaded}%
  \endinput
}
%    \end{macrocode}
%
%    \begin{macro}{\hbs@do}
%    The redefined sectioning commands calls \cmd{\hbs@do}. It does
%    \begin{itemize}
%    \item handle the star case.
%    \item resets the macros that store the entries for the outlines
%          (\cmd{\hbs@bmstring}) and table of contents (\cmd{\hbs@tocstring}).
%    \item store the sectioning command |#1| in \cmd{\hbs@seccmd}
%          for later reuse.
%    \item at last call \cmd{\hbs@checkarg} that scans and interprets the
%          parameters of the redefined sectioning command.
%    \end{itemize}
%    \begin{macrocode}
\def\hbs@do#1{%
  \@ifstar{#1*}{%
    \let\hbs@tocstring\relax
    \let\hbs@bmstring\relax
    \let\hbs@seccmd#1%
    \hbs@checkarg
  }%
}
%    \end{macrocode}
%    \end{macro}
%
%    \begin{macro}{\hbs@checkarg}
%    \cmd{\hbs@checkarg} determines the type of the next argument:
%    \begin{itemize}
%    \item An optional argument in square brackets can be an entry
%          for the table of contents or the bookmarks. It will be
%          read by \cmd{\hbs@getsquare}
%    \item An optional argument in parentheses is an outline entry.
%          This is worked off by \cmd{\hbs@getbookmark}.
%    \item If there are no more optional arguments, \cmd{\hbs@process}
%          reads the mandatory argument and calls the original
%          sectioning commands.
%    \end{itemize}
%    \begin{macrocode}
\def\hbs@checkarg{%
  \@ifnextchar[\hbs@getsquare{%
    \@ifnextchar(\hbs@getbookmark\hbs@process
  }%
}
%    \end{macrocode}
%    \end{macro}
%
%    \begin{macro}{\hbs@getsquare}
%    \cmd{\hbs@getsquare} reads an optional argument in square
%    brackets and determines, if this is an entry for the table
%    of contents or the bookmarks.
%    \begin{macrocode}
\long\def\hbs@getsquare[#1]{%
  \ifx\hbs@tocstring\relax
    \def\hbs@tocstring{#1}%
  \else
    \hbs@bmdef{#1}%
  \fi
  \hbs@checkarg
}
%    \end{macrocode}
%    \end{macro}
%
%    \begin{macro}{\hbs@getbookmark}
%    \cmd{\hbs@getbookmark} reads an outline entry in parentheses.
%    \begin{macrocode}
\def\hbs@getbookmark(#1){%
  \hbs@bmdef{#1}%
  \hbs@checkarg
}
%    \end{macrocode}
%    \end{macro}
%
%    \begin{macro}{\hbs@bmdef}
%    The command \cmd{\hbs@bmdef} save the bookmark entry in
%    parameter |#1| in the macro \cmd{\hbs@bmstring} and catches
%    the case, if the user has given several outline strings.
%    \begin{macrocode}
\def\hbs@bmdef#1{%
  \ifx\hbs@bmstring\relax
    \def\hbs@bmstring{#1}%
  \else
    \PackageError{hypbmsec}{%
      Sectioning command with too many parameters%
    }{%
      You can only give one outline entry.%
    }%
  \fi
}
%    \end{macrocode}
%    \end{macro}
%
%    \begin{macro}{\hbs@process}
%    The parameter |#1| is the mandatory argument of the sectioning
%    commands. \cmd{\hbs@process} calls the original sectioning command
%    stored in \cmd{\hbs@seccmd} with arguments that depend of which
%    optional argument are used previously.
%    \begin{macrocode}
\long\def\hbs@process#1{%
  \ifx\hbs@tocstring\relax
    \ifx\hbs@bmstring\relax
      \hbs@seccmd{#1}%
    \else
      \begingroup
        \def\x##1{\endgroup
          \hbs@seccmd{\texorpdfstring{#1}{##1}}%
        }%
      \expandafter\x\expandafter{\hbs@bmstring}%
    \fi
  \else
    \ifx\hbs@bmstring\relax
      \expandafter\hbs@seccmd\expandafter[%
        \expandafter{\hbs@tocstring}%
      ]{#1}%
    \else
      \expandafter\expandafter\expandafter
      \hbs@seccmd\expandafter\expandafter\expandafter[%
        \expandafter\expandafter\expandafter
        \texorpdfstring
        \expandafter\expandafter\expandafter{%
          \expandafter\hbs@tocstring\expandafter
        }\expandafter{%
          \hbs@bmstring
        }%
      ]{#1}%
    \fi
  \fi
}
%    \end{macrocode}
%    \end{macro}
%
%    We have to check, whether package \Package{hyperref} is loaded
%    and have to provide a definition for \cmd{\texorpdfstring}.
%    Because \Package{hyperref} can be loaded after this package,
%    we do the work later (\cmd{\AtBeginDocument}).
%
%    This code only checks versions of \Package{hyperref} that
%    define \cmd{\ifbookmark} (v6.4x until v6.53) or
%    \cmd{\texorpdfstring} (v6.54 and above). Older versions aren't
%    supported.
%    \begin{macrocode}
\AtBeginDocument{%
  \@ifundefined{texorpdfstring}{%
    \@ifundefined{ifbookmark}{%
      \let\texorpdfstring\@firstoftwo
      \@ifpackageloaded{hyperref}{%
        \PackageInfo{hypbmsec}{%
          \ifx\hy@driver\@empty
            Default driver %
          \else
            '\hy@driver' %
          \fi
          of hyperref not supported,\MessageBreak
          bookmark parameters will be ignored%
        }%
      }{%
        \PackageInfo{hypbmsec}{%
          Package hyperref not loaded,\MessageBreak
          bookmark parameters will be ignored%
        }%
      }%
    }%
    {%
      \newcommand\texorpdfstring[2]{\ifbookmark{#2}{#1}}%
      \PackageWarningNoLine{hypbmsec}{%
        Old hyperref version found,\MessageBreak
        update of hyperref recommended%
      }%
    }%
  }{}%
%    \end{macrocode}
%
%    Other packages are allowed to redefine the sectioning commands,
%    if they does not change the syntax. Therefore the redefinitons
%    of this package should be done after the other packages.
%    \begin{macrocode}
  \let\hbs@part         \part
  \let\hbs@section      \section
  \let\hbs@subsection   \subsection
  \let\hbs@subsubsection\subsubsection
  \let\hbs@paragraph    \paragraph
  \let\hbs@subparagraph \subparagraph
  \renewcommand\part         {\hbs@do\hbs@part}%
  \renewcommand\section      {\hbs@do\hbs@section}%
  \renewcommand\subsection   {\hbs@do\hbs@subsection}%
  \renewcommand\subsubsection{\hbs@do\hbs@subsubsection}%
  \renewcommand\paragraph    {\hbs@do\hbs@paragraph}%
  \renewcommand\subparagraph {\hbs@do\hbs@subparagraph}%
  \begingroup\expandafter\expandafter\expandafter\endgroup
  \expandafter\ifx\csname chapter\endcsname\relax\else
    \let\hbs@chapter    \chapter
    \renewcommand\chapter    {\hbs@do\hbs@chapter}%
  \fi
}
%    \end{macrocode}
%
%    \begin{macrocode}
%</package>
%    \end{macrocode}
%
% \section{Installation}
%
% \subsection{Download}
%
% \paragraph{Package.} This package is available on
% CTAN\footnote{\url{http://ctan.org/pkg/hypbmsec}}:
% \begin{description}
% \item[\CTAN{macros/latex/contrib/oberdiek/hypbmsec.dtx}] The source file.
% \item[\CTAN{macros/latex/contrib/oberdiek/hypbmsec.pdf}] Documentation.
% \end{description}
%
%
% \paragraph{Bundle.} All the packages of the bundle `oberdiek'
% are also available in a TDS compliant ZIP archive. There
% the packages are already unpacked and the documentation files
% are generated. The files and directories obey the TDS standard.
% \begin{description}
% \item[\CTAN{install/macros/latex/contrib/oberdiek.tds.zip}]
% \end{description}
% \emph{TDS} refers to the standard ``A Directory Structure
% for \TeX\ Files'' (\CTAN{tds/tds.pdf}). Directories
% with \xfile{texmf} in their name are usually organized this way.
%
% \subsection{Bundle installation}
%
% \paragraph{Unpacking.} Unpack the \xfile{oberdiek.tds.zip} in the
% TDS tree (also known as \xfile{texmf} tree) of your choice.
% Example (linux):
% \begin{quote}
%   |unzip oberdiek.tds.zip -d ~/texmf|
% \end{quote}
%
% \paragraph{Script installation.}
% Check the directory \xfile{TDS:scripts/oberdiek/} for
% scripts that need further installation steps.
% Package \xpackage{attachfile2} comes with the Perl script
% \xfile{pdfatfi.pl} that should be installed in such a way
% that it can be called as \texttt{pdfatfi}.
% Example (linux):
% \begin{quote}
%   |chmod +x scripts/oberdiek/pdfatfi.pl|\\
%   |cp scripts/oberdiek/pdfatfi.pl /usr/local/bin/|
% \end{quote}
%
% \subsection{Package installation}
%
% \paragraph{Unpacking.} The \xfile{.dtx} file is a self-extracting
% \docstrip\ archive. The files are extracted by running the
% \xfile{.dtx} through \plainTeX:
% \begin{quote}
%   \verb|tex hypbmsec.dtx|
% \end{quote}
%
% \paragraph{TDS.} Now the different files must be moved into
% the different directories in your installation TDS tree
% (also known as \xfile{texmf} tree):
% \begin{quote}
% \def\t{^^A
% \begin{tabular}{@{}>{\ttfamily}l@{ $\rightarrow$ }>{\ttfamily}l@{}}
%   hypbmsec.sty & tex/latex/oberdiek/hypbmsec.sty\\
%   hypbmsec.pdf & doc/latex/oberdiek/hypbmsec.pdf\\
%   hypbmsec.dtx & source/latex/oberdiek/hypbmsec.dtx\\
% \end{tabular}^^A
% }^^A
% \sbox0{\t}^^A
% \ifdim\wd0>\linewidth
%   \begingroup
%     \advance\linewidth by\leftmargin
%     \advance\linewidth by\rightmargin
%   \edef\x{\endgroup
%     \def\noexpand\lw{\the\linewidth}^^A
%   }\x
%   \def\lwbox{^^A
%     \leavevmode
%     \hbox to \linewidth{^^A
%       \kern-\leftmargin\relax
%       \hss
%       \usebox0
%       \hss
%       \kern-\rightmargin\relax
%     }^^A
%   }^^A
%   \ifdim\wd0>\lw
%     \sbox0{\small\t}^^A
%     \ifdim\wd0>\linewidth
%       \ifdim\wd0>\lw
%         \sbox0{\footnotesize\t}^^A
%         \ifdim\wd0>\linewidth
%           \ifdim\wd0>\lw
%             \sbox0{\scriptsize\t}^^A
%             \ifdim\wd0>\linewidth
%               \ifdim\wd0>\lw
%                 \sbox0{\tiny\t}^^A
%                 \ifdim\wd0>\linewidth
%                   \lwbox
%                 \else
%                   \usebox0
%                 \fi
%               \else
%                 \lwbox
%               \fi
%             \else
%               \usebox0
%             \fi
%           \else
%             \lwbox
%           \fi
%         \else
%           \usebox0
%         \fi
%       \else
%         \lwbox
%       \fi
%     \else
%       \usebox0
%     \fi
%   \else
%     \lwbox
%   \fi
% \else
%   \usebox0
% \fi
% \end{quote}
% If you have a \xfile{docstrip.cfg} that configures and enables \docstrip's
% TDS installing feature, then some files can already be in the right
% place, see the documentation of \docstrip.
%
% \subsection{Refresh file name databases}
%
% If your \TeX~distribution
% (\teTeX, \mikTeX, \dots) relies on file name databases, you must refresh
% these. For example, \teTeX\ users run \verb|texhash| or
% \verb|mktexlsr|.
%
% \subsection{Some details for the interested}
%
% \paragraph{Attached source.}
%
% The PDF documentation on CTAN also includes the
% \xfile{.dtx} source file. It can be extracted by
% AcrobatReader 6 or higher. Another option is \textsf{pdftk},
% e.g. unpack the file into the current directory:
% \begin{quote}
%   \verb|pdftk hypbmsec.pdf unpack_files output .|
% \end{quote}
%
% \paragraph{Unpacking with \LaTeX.}
% The \xfile{.dtx} chooses its action depending on the format:
% \begin{description}
% \item[\plainTeX:] Run \docstrip\ and extract the files.
% \item[\LaTeX:] Generate the documentation.
% \end{description}
% If you insist on using \LaTeX\ for \docstrip\ (really,
% \docstrip\ does not need \LaTeX), then inform the autodetect routine
% about your intention:
% \begin{quote}
%   \verb|latex \let\install=y\input{hypbmsec.dtx}|
% \end{quote}
% Do not forget to quote the argument according to the demands
% of your shell.
%
% \paragraph{Generating the documentation.}
% You can use both the \xfile{.dtx} or the \xfile{.drv} to generate
% the documentation. The process can be configured by the
% configuration file \xfile{ltxdoc.cfg}. For instance, put this
% line into this file, if you want to have A4 as paper format:
% \begin{quote}
%   \verb|\PassOptionsToClass{a4paper}{article}|
% \end{quote}
% An example follows how to generate the
% documentation with pdf\LaTeX:
% \begin{quote}
%\begin{verbatim}
%pdflatex hypbmsec.dtx
%makeindex -s gind.ist hypbmsec.idx
%pdflatex hypbmsec.dtx
%makeindex -s gind.ist hypbmsec.idx
%pdflatex hypbmsec.dtx
%\end{verbatim}
% \end{quote}
%
% \section{Catalogue}
%
% The following XML file can be used as source for the
% \href{http://mirror.ctan.org/help/Catalogue/catalogue.html}{\TeX\ Catalogue}.
% The elements \texttt{caption} and \texttt{description} are imported
% from the original XML file from the Catalogue.
% The name of the XML file in the Catalogue is \xfile{hypbmsec.xml}.
%    \begin{macrocode}
%<*catalogue>
<?xml version='1.0' encoding='us-ascii'?>
<!DOCTYPE entry SYSTEM 'catalogue.dtd'>
<entry datestamp='$Date$' modifier='$Author$' id='hypbmsec'>
  <name>hypbmsec</name>
  <caption>Hypertext bookmarks in sectioning commands.</caption>
  <authorref id='auth:oberdiek'/>
  <copyright owner='Heiko Oberdiek' year='1998-2000,2006,2007'/>
  <license type='lppl1.3'/>
  <version number='2.5'/>
  <description>
    Bookmark entries can be given as another argument to the LaTeX
    sectioning commands. The <xref refid='hyperref'>hyperref</xref>
    package is required to get the bookmarks, but the syntax
    works without it.
    <p/>
    This package is part of the <xref refid='oberdiek'>oberdiek</xref>
    bundle.
  </description>
  <documentation details='Package documentation'
      href='ctan:/macros/latex/contrib/oberdiek/hypbmsec.pdf'/>
  <ctan file='true' path='/macros/latex/contrib/oberdiek/hypbmsec.dtx'/>
  <miktex location='oberdiek'/>
  <texlive location='oberdiek'/>
  <install path='/macros/latex/contrib/oberdiek/oberdiek.tds.zip'/>
</entry>
%</catalogue>
%    \end{macrocode}
%
% \begin{History}
%   \begin{Version}{1998/11/20 v1.0}
%   \item
%     First version.
%   \item
%     It merges package \xpackage{hysecopt} and
%   \item
%     package \xpackage{hypbmpar}.
%   \item
%     Published for the DANTE'99 meeting^^A
%     \URL{}{http://dante99.cs.uni-dortmund.de/handouts/oberdiek/hypbmsec.sty}.
%   \end{Version}
%   \begin{Version}{1999/04/12 v2.0}
%   \item
%     Adaptation to \Package{hyperref} version 6.54.
%   \item
%     Documentation in dtx format.
%   \item
%     Copyright: LPPL (\CTAN{macros/latex/base/lppl.txt})
%   \item
%     First CTAN release.
%   \end{Version}
%   \begin{Version}{2000/03/22 v2.1}
%   \item
%     Bug fix in redefinition of \cmd{\chapter}.
%   \item
%     Copyright: LPPL 1.2
%   \end{Version}
%   \begin{Version}{2006/02/20 v2.2}
%   \item
%     Code is not changed.
%   \item
%     New DTX framework.
%   \item
%     LPPL 1.3
%   \end{Version}
%   \begin{Version}{2007/03/05 v2.3}
%   \item
%     Bug fix: Expand \cs{hbs@tocstring} and \cs{hbs@bmstring} before
%     calling \cs{hbs@seccmd}.
%   \end{Version}
%   \begin{Version}{2007/04/11 v2.4}
%   \item
%     Line ends sanitized.
%   \end{Version}
%   \begin{Version}{2016/05/16 v2.5}
%   \item
%     Documentation updates.
%   \end{Version}
% \end{History}
%
% \PrintIndex
%
% \Finale
\endinput
|
% \end{quote}
% Do not forget to quote the argument according to the demands
% of your shell.
%
% \paragraph{Generating the documentation.}
% You can use both the \xfile{.dtx} or the \xfile{.drv} to generate
% the documentation. The process can be configured by the
% configuration file \xfile{ltxdoc.cfg}. For instance, put this
% line into this file, if you want to have A4 as paper format:
% \begin{quote}
%   \verb|\PassOptionsToClass{a4paper}{article}|
% \end{quote}
% An example follows how to generate the
% documentation with pdf\LaTeX:
% \begin{quote}
%\begin{verbatim}
%pdflatex hypbmsec.dtx
%makeindex -s gind.ist hypbmsec.idx
%pdflatex hypbmsec.dtx
%makeindex -s gind.ist hypbmsec.idx
%pdflatex hypbmsec.dtx
%\end{verbatim}
% \end{quote}
%
% \section{Catalogue}
%
% The following XML file can be used as source for the
% \href{http://mirror.ctan.org/help/Catalogue/catalogue.html}{\TeX\ Catalogue}.
% The elements \texttt{caption} and \texttt{description} are imported
% from the original XML file from the Catalogue.
% The name of the XML file in the Catalogue is \xfile{hypbmsec.xml}.
%    \begin{macrocode}
%<*catalogue>
<?xml version='1.0' encoding='us-ascii'?>
<!DOCTYPE entry SYSTEM 'catalogue.dtd'>
<entry datestamp='$Date$' modifier='$Author$' id='hypbmsec'>
  <name>hypbmsec</name>
  <caption>Hypertext bookmarks in sectioning commands.</caption>
  <authorref id='auth:oberdiek'/>
  <copyright owner='Heiko Oberdiek' year='1998-2000,2006,2007'/>
  <license type='lppl1.3'/>
  <version number='2.5'/>
  <description>
    Bookmark entries can be given as another argument to the LaTeX
    sectioning commands. The <xref refid='hyperref'>hyperref</xref>
    package is required to get the bookmarks, but the syntax
    works without it.
    <p/>
    This package is part of the <xref refid='oberdiek'>oberdiek</xref>
    bundle.
  </description>
  <documentation details='Package documentation'
      href='ctan:/macros/latex/contrib/oberdiek/hypbmsec.pdf'/>
  <ctan file='true' path='/macros/latex/contrib/oberdiek/hypbmsec.dtx'/>
  <miktex location='oberdiek'/>
  <texlive location='oberdiek'/>
  <install path='/macros/latex/contrib/oberdiek/oberdiek.tds.zip'/>
</entry>
%</catalogue>
%    \end{macrocode}
%
% \begin{History}
%   \begin{Version}{1998/11/20 v1.0}
%   \item
%     First version.
%   \item
%     It merges package \xpackage{hysecopt} and
%   \item
%     package \xpackage{hypbmpar}.
%   \item
%     Published for the DANTE'99 meeting^^A
%     \URL{}{http://dante99.cs.uni-dortmund.de/handouts/oberdiek/hypbmsec.sty}.
%   \end{Version}
%   \begin{Version}{1999/04/12 v2.0}
%   \item
%     Adaptation to \Package{hyperref} version 6.54.
%   \item
%     Documentation in dtx format.
%   \item
%     Copyright: LPPL (\CTAN{macros/latex/base/lppl.txt})
%   \item
%     First CTAN release.
%   \end{Version}
%   \begin{Version}{2000/03/22 v2.1}
%   \item
%     Bug fix in redefinition of \cmd{\chapter}.
%   \item
%     Copyright: LPPL 1.2
%   \end{Version}
%   \begin{Version}{2006/02/20 v2.2}
%   \item
%     Code is not changed.
%   \item
%     New DTX framework.
%   \item
%     LPPL 1.3
%   \end{Version}
%   \begin{Version}{2007/03/05 v2.3}
%   \item
%     Bug fix: Expand \cs{hbs@tocstring} and \cs{hbs@bmstring} before
%     calling \cs{hbs@seccmd}.
%   \end{Version}
%   \begin{Version}{2007/04/11 v2.4}
%   \item
%     Line ends sanitized.
%   \end{Version}
%   \begin{Version}{2016/05/16 v2.5}
%   \item
%     Documentation updates.
%   \end{Version}
% \end{History}
%
% \PrintIndex
%
% \Finale
\endinput
|
% \end{quote}
% Do not forget to quote the argument according to the demands
% of your shell.
%
% \paragraph{Generating the documentation.}
% You can use both the \xfile{.dtx} or the \xfile{.drv} to generate
% the documentation. The process can be configured by the
% configuration file \xfile{ltxdoc.cfg}. For instance, put this
% line into this file, if you want to have A4 as paper format:
% \begin{quote}
%   \verb|\PassOptionsToClass{a4paper}{article}|
% \end{quote}
% An example follows how to generate the
% documentation with pdf\LaTeX:
% \begin{quote}
%\begin{verbatim}
%pdflatex hypbmsec.dtx
%makeindex -s gind.ist hypbmsec.idx
%pdflatex hypbmsec.dtx
%makeindex -s gind.ist hypbmsec.idx
%pdflatex hypbmsec.dtx
%\end{verbatim}
% \end{quote}
%
% \section{Catalogue}
%
% The following XML file can be used as source for the
% \href{http://mirror.ctan.org/help/Catalogue/catalogue.html}{\TeX\ Catalogue}.
% The elements \texttt{caption} and \texttt{description} are imported
% from the original XML file from the Catalogue.
% The name of the XML file in the Catalogue is \xfile{hypbmsec.xml}.
%    \begin{macrocode}
%<*catalogue>
<?xml version='1.0' encoding='us-ascii'?>
<!DOCTYPE entry SYSTEM 'catalogue.dtd'>
<entry datestamp='$Date$' modifier='$Author$' id='hypbmsec'>
  <name>hypbmsec</name>
  <caption>Hypertext bookmarks in sectioning commands.</caption>
  <authorref id='auth:oberdiek'/>
  <copyright owner='Heiko Oberdiek' year='1998-2000,2006,2007'/>
  <license type='lppl1.3'/>
  <version number='2.5'/>
  <description>
    Bookmark entries can be given as another argument to the LaTeX
    sectioning commands. The <xref refid='hyperref'>hyperref</xref>
    package is required to get the bookmarks, but the syntax
    works without it.
    <p/>
    This package is part of the <xref refid='oberdiek'>oberdiek</xref>
    bundle.
  </description>
  <documentation details='Package documentation'
      href='ctan:/macros/latex/contrib/oberdiek/hypbmsec.pdf'/>
  <ctan file='true' path='/macros/latex/contrib/oberdiek/hypbmsec.dtx'/>
  <miktex location='oberdiek'/>
  <texlive location='oberdiek'/>
  <install path='/macros/latex/contrib/oberdiek/oberdiek.tds.zip'/>
</entry>
%</catalogue>
%    \end{macrocode}
%
% \begin{History}
%   \begin{Version}{1998/11/20 v1.0}
%   \item
%     First version.
%   \item
%     It merges package \xpackage{hysecopt} and
%   \item
%     package \xpackage{hypbmpar}.
%   \item
%     Published for the DANTE'99 meeting^^A
%     \URL{}{http://dante99.cs.uni-dortmund.de/handouts/oberdiek/hypbmsec.sty}.
%   \end{Version}
%   \begin{Version}{1999/04/12 v2.0}
%   \item
%     Adaptation to \Package{hyperref} version 6.54.
%   \item
%     Documentation in dtx format.
%   \item
%     Copyright: LPPL (\CTAN{macros/latex/base/lppl.txt})
%   \item
%     First CTAN release.
%   \end{Version}
%   \begin{Version}{2000/03/22 v2.1}
%   \item
%     Bug fix in redefinition of \cmd{\chapter}.
%   \item
%     Copyright: LPPL 1.2
%   \end{Version}
%   \begin{Version}{2006/02/20 v2.2}
%   \item
%     Code is not changed.
%   \item
%     New DTX framework.
%   \item
%     LPPL 1.3
%   \end{Version}
%   \begin{Version}{2007/03/05 v2.3}
%   \item
%     Bug fix: Expand \cs{hbs@tocstring} and \cs{hbs@bmstring} before
%     calling \cs{hbs@seccmd}.
%   \end{Version}
%   \begin{Version}{2007/04/11 v2.4}
%   \item
%     Line ends sanitized.
%   \end{Version}
%   \begin{Version}{2016/05/16 v2.5}
%   \item
%     Documentation updates.
%   \end{Version}
% \end{History}
%
% \PrintIndex
%
% \Finale
\endinput
|
% \end{quote}
% Do not forget to quote the argument according to the demands
% of your shell.
%
% \paragraph{Generating the documentation.}
% You can use both the \xfile{.dtx} or the \xfile{.drv} to generate
% the documentation. The process can be configured by the
% configuration file \xfile{ltxdoc.cfg}. For instance, put this
% line into this file, if you want to have A4 as paper format:
% \begin{quote}
%   \verb|\PassOptionsToClass{a4paper}{article}|
% \end{quote}
% An example follows how to generate the
% documentation with pdf\LaTeX:
% \begin{quote}
%\begin{verbatim}
%pdflatex hypbmsec.dtx
%makeindex -s gind.ist hypbmsec.idx
%pdflatex hypbmsec.dtx
%makeindex -s gind.ist hypbmsec.idx
%pdflatex hypbmsec.dtx
%\end{verbatim}
% \end{quote}
%
% \section{Catalogue}
%
% The following XML file can be used as source for the
% \href{http://mirror.ctan.org/help/Catalogue/catalogue.html}{\TeX\ Catalogue}.
% The elements \texttt{caption} and \texttt{description} are imported
% from the original XML file from the Catalogue.
% The name of the XML file in the Catalogue is \xfile{hypbmsec.xml}.
%    \begin{macrocode}
%<*catalogue>
<?xml version='1.0' encoding='us-ascii'?>
<!DOCTYPE entry SYSTEM 'catalogue.dtd'>
<entry datestamp='$Date$' modifier='$Author$' id='hypbmsec'>
  <name>hypbmsec</name>
  <caption>Hypertext bookmarks in sectioning commands.</caption>
  <authorref id='auth:oberdiek'/>
  <copyright owner='Heiko Oberdiek' year='1998-2000,2006,2007'/>
  <license type='lppl1.3'/>
  <version number='2.5'/>
  <description>
    Bookmark entries can be given as another argument to the LaTeX
    sectioning commands. The <xref refid='hyperref'>hyperref</xref>
    package is required to get the bookmarks, but the syntax
    works without it.
    <p/>
    This package is part of the <xref refid='oberdiek'>oberdiek</xref>
    bundle.
  </description>
  <documentation details='Package documentation'
      href='ctan:/macros/latex/contrib/oberdiek/hypbmsec.pdf'/>
  <ctan file='true' path='/macros/latex/contrib/oberdiek/hypbmsec.dtx'/>
  <miktex location='oberdiek'/>
  <texlive location='oberdiek'/>
  <install path='/macros/latex/contrib/oberdiek/oberdiek.tds.zip'/>
</entry>
%</catalogue>
%    \end{macrocode}
%
% \begin{History}
%   \begin{Version}{1998/11/20 v1.0}
%   \item
%     First version.
%   \item
%     It merges package \xpackage{hysecopt} and
%   \item
%     package \xpackage{hypbmpar}.
%   \item
%     Published for the DANTE'99 meeting^^A
%     \URL{}{http://dante99.cs.uni-dortmund.de/handouts/oberdiek/hypbmsec.sty}.
%   \end{Version}
%   \begin{Version}{1999/04/12 v2.0}
%   \item
%     Adaptation to \Package{hyperref} version 6.54.
%   \item
%     Documentation in dtx format.
%   \item
%     Copyright: LPPL (\CTAN{macros/latex/base/lppl.txt})
%   \item
%     First CTAN release.
%   \end{Version}
%   \begin{Version}{2000/03/22 v2.1}
%   \item
%     Bug fix in redefinition of \cmd{\chapter}.
%   \item
%     Copyright: LPPL 1.2
%   \end{Version}
%   \begin{Version}{2006/02/20 v2.2}
%   \item
%     Code is not changed.
%   \item
%     New DTX framework.
%   \item
%     LPPL 1.3
%   \end{Version}
%   \begin{Version}{2007/03/05 v2.3}
%   \item
%     Bug fix: Expand \cs{hbs@tocstring} and \cs{hbs@bmstring} before
%     calling \cs{hbs@seccmd}.
%   \end{Version}
%   \begin{Version}{2007/04/11 v2.4}
%   \item
%     Line ends sanitized.
%   \end{Version}
%   \begin{Version}{2016/05/16 v2.5}
%   \item
%     Documentation updates.
%   \end{Version}
% \end{History}
%
% \PrintIndex
%
% \Finale
\endinput
