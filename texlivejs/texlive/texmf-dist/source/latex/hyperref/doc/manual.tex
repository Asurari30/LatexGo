% This is the manual for the LaTeX hyperref package.
%
% Copyright (C) 1998, 2003 Sebastian Rahtz.
%
% Permission is granted to copy, distribute and/or modify this document
% under the terms of the GNU Free Documentation License, Version 1.1 or
% any later version published by the Free Software Foundation; with no
% Invariant Sections, with no Front-Cover Texts, and with no Back-Cover
% Texts.  A copy of the license is included in the section entitled
% ``GNU Free Documentation License.''
%
% Manual updates:
% * Steve Peter and Karl Berry, 7/03.
% * Heiko Oberdiek, 2006-2012.
% * HO-support 2017
%

\def\mydate{January 2017}

\RequirePackage{ifpdf}
\ifpdf % We are running pdfTeX in pdf mode
\ifx\directlua\undefinded
\documentclass[pdftex]{article}
\else
\documentclass[luatex]{article}
\fi
\else
\documentclass{article}
\fi

\usepackage{etex}% for \eTeX

\usepackage{pifont}
\usepackage{calc}

\usepackage{hologo}

\def\OzTeX{O\kern-0.03em z\kern-0.15em \TeX}

\newcommand*{\cs}[1]{%
  \texttt{\textbackslash #1}%
}
\newcommand*{\xpackage}[1]{\textsf{#1}}

% from doc.sty
\makeatletter
\ifx\l@nohyphenation\@undefined
\newlanguage\l@nohyphenation
\fi
\ifx\l@nohyphenation\@undefined
  \newlanguage\l@nohyphenation
\fi
\DeclareRobustCommand\meta[1]{%
  \ensuremath\langle
  \ifmmode \expandafter \nfss@text \fi
  {%
    \meta@font@select
    \edef\meta@hyphen@restore
      {\hyphenchar\the\font\the\hyphenchar\font}%
    \hyphenchar\font\m@ne
    \language\l@nohyphenation
    #1\/%
    \meta@hyphen@restore
  }%
  \ensuremath\rangle
}
\def\meta@font@select{\ttfamily\itshape}
\makeatother

% Page layout.
\advance\textwidth by 1.1in
\advance\oddsidemargin by -.55in
\advance\evensidemargin by -.55in
%
\advance\textheight by 1in
\advance\topmargin by -.5in
\advance\footskip by -.5in
%
\pagestyle{headings}
%
% Avoid some overfull boxes.
\emergencystretch=.1\hsize
\hbadness = 3000

% these are from lshort.sty, but lshort.sty pulls in so many other
% packages it seems cleaner to just include them here.
%
\newcommand{\bs}{\symbol{'134}}%Print backslash
\newcommand{\ci}[1]{\texttt{\bs#1}}

\makeatletter
\@ifpackageloaded{tex4ht}{%
  % separate definition for HTML case to avoid
  % nasty borders with double horizontal lines with
  % large gaps.
  \newsavebox{\cmdsyntaxbox}%
  \newenvironment{cmdsyntax}{%
    \par
    % \small
    \addvspace{3.2ex plus 0.8ex minus 0.2ex}%
    \vskip -\parskip
    \noindent
    \begin{lrbox}{\cmdsyntaxbox}%
      \begin{tabular}{l}%
        \rule{0pt}{1em}%
        \ignorespaces
  }{%
      \end{tabular}%
    \end{lrbox}%
    \fbox{\usebox{\cmdsyntaxbox}}%
    \par
    \nopagebreak
    \addvspace{3.2ex plus 0.8ex minus 0.2ex}%
    \vskip -\parskip
  }%
}{%
  \newenvironment{cmdsyntax}{%
    \par
    \small
    \addvspace{3.2ex plus 0.8ex minus 0.2ex}%
    \vskip -\parskip
    \noindent
    \begin{tabular}{|l|}%
      \hline
      \rule{0pt}{1em}%
      \ignorespaces
  }{%
      \\%
      \hline
    \end{tabular}%
    \par
    \nopagebreak
    \addvspace{3.2ex plus 0.8ex minus 0.2ex}%
    \vskip -\parskip
  }%
}
\makeatother

\usepackage{array,longtable}
\usepackage{ifluatex,ifxetex}
\ifnum 0\ifluatex 1\else\ifxetex 1\fi\fi=0 %
  \usepackage[T1]{fontenc}%
  \usepackage{lmodern}%
  \renewcommand*{\ttdefault}{lmvtt}%
\else
  \usepackage{fontspec}%
  \renewcommand*{\ttdefault}{lmvtt}%
\fi

\newcommand*{\Quote}[1]{\textquotedblleft#1\textquotedblright}

\def\Hanh{H\`an Th\^e\llap{\raise 0.5ex\hbox{\'{}}} Th\`anh}

\ifpdf
  \usepackage[%
%    pdftex,% might be luatex, just allow automatic default
    colorlinks,%
    hyperindex,%
    plainpages=false,%
    bookmarksopen,%
    bookmarksnumbered,
    pdfusetitle,%
  ]{hyperref}
  %%?? \def\pdfBorderAttrs{/Border [0 0 0] } % No border arround Links
  \usepackage{thumbpdf}
\else
  \usepackage{hyperref}
\fi

\makeatletter
\@ifpackageloaded{tex4ht}{%
}{%
  \usepackage{bmhydoc}%
}
\makeatother

\title{Hypertext marks in \hologo{LaTeX}: a manual for \xpackage{hyperref}}
\author{Sebastian Rahtz \and  Heiko Oberdiek}
\date{\mydate}

\begin{document}

% comes out too close to the toc, and we know it's page one anyway.
\thispagestyle{empty}
\maketitle
\tableofcontents
\setcounter{tocdepth}{2}% for bookmark levels

\section{Introduction}

The package derives from, and builds on, the work of the Hyper\hologo{TeX}
project, described at \url{http://xxx.lanl.gov/hypertex/}. It extends
the functionality of all the \hologo{LaTeX} cross-referencing commands
(including the table of contents, bibliographies etc) to produce
\cs{special} commands which a driver can turn into hypertext links;
it also provides new commands to allow the user to write \emph{ad hoc}
hypertext links, including those to external documents and URLs.

This manual provides a brief overview of the \xpackage{hyperref}
package. For more details, you should read the additional documentation
distributed with the package, as well as the complete documentation by
processing \texttt{hyperref.dtx}. You should also read the chapter on
\xpackage{hyperref} in \textit{The \hologo{LaTeX} Web Companion}, where you will
find additional examples.

The Hyper\hologo{TeX} specification\footnote{This is borrowed from an article
by Arthur Smith.} says that conformant viewers/translators must
recognize the following set of \cs{special} constructs:

\begin{description}
\item[href:] \verb|html:<a href = "href_string">|
\item[name:] \verb|html:<a name = "name_string">|
\item[end:] \verb|html:</a>|
\item[image:] \verb|html:<img src = "href_string">|
\item[base\_name:] \verb|html:<base href = "href_string">|
\end{description}

The \emph{href}, \emph{name} and \emph{end} commands are used to do the
basic hypertext operations of establishing links between sections of
documents. The \emph{image} command is intended (as with current HTML
viewers) to place an image of arbitrary graphical format on the page in
the current location. The \emph{base\_name} command is be used to
communicate to the DVI viewer the full (URL) location of the current
document so that files specified by relative URLs may be retrieved
correctly.

The \emph{href} and \emph{name} commands must be paired with an
\emph{end} command later in the \TeX\ file---the \TeX\ commands between
the two ends of a pair form an \emph{anchor} in the document. In the
case of an \emph{href} command, the \emph{anchor} is to be highlighted
in the \emph{DVI viewer}, and when clicked on will cause the scene to
shift to the destination specified by \emph{href\_string}. The
\emph{anchor} associated with a name command represents a possible
location to which other hypertext links may refer, either as local
references (of the form \verb|href="#name_string"| with the
\emph{name\_string} identical to the one in the name command) or as part
of a URL (of the form \emph{URL\#name\_string}). Here
\emph{href\_string} is a valid URL or local identifier, while
\emph{name\_string} could be any string at all: the only caveat is that
`$\verb|"|$' characters should be escaped with a backslash
($\backslash$), and if it looks like a URL name it may cause problems.

However, the drivers intended to produce \emph{only} PDF use literal
PostScript or PDF \verb|\special| commands. The commands are defined in
configuration files for different drivers, selected by package options;
at present, the following drivers are supported:

\begin{description}
\item[hypertex] DVI processors conforming to the Hyper\TeX\ guidelines (i.e.\ \textsf{xdvi}, \textsf{dvips} (with
the \textsf{-z} option), \textsf{\OzTeX}, and \textsf{Textures})
\item[dvips] produces \verb|\special| commands tailored for \textsf{dvips}
\item[dvipsone] produces \verb|\special| commands tailored for \textsf{dvipsone}
\item[ps2pdf] a special case of output suitable for processing by earlier versions of Ghost\-script's
PDF writer; this is basically the same as that for \textsf{dvips}, but a few variations remained before version 5.21
\item[tex4ht] produces \verb|\special| commands for use with \textsf{\TeX4ht}
\item[pdftex] pdf\TeX, \Hanh{}'s \TeX{} variant that writes PDF directly
\item[dvipdfm] produces \verb|\special| commands for Mark Wicks' DVI to PDF driver \textsf{dvipdfm}
\item[dvipdfmx] produces \verb|\special| commands for driver
     \textsf{dvipdfmx}, a successor of \textsf{dvipdfm}
\item[dviwindo] produces \verb|\special| commands that Y\&Y's Windows previewer interprets as hypertext jumps within the previewer
\item[vtex] produces \verb|\special| commands that MicroPress' HTML and
     PDF-producing \TeX\ variants interpret as hypertext jumps within the
     previewer
\item[textures] produces \verb|\special| commands that \textsf{Textures} interprets as hypertext jumps within the previewer
\item[xetex] produces \verb|\special| commands for Xe\TeX{}
\end{description}

Output from \textsf{dvips} or \textsf{dvipsone} must be processed using
Acrobat Distiller to obtain a PDF file.\footnote{Make sure you turn off
the partial font downloading supported by \textsf{dvips} and
\textsf{dvipsone} in favor of Distiller's own system.} The result is
generally preferable to that produced by using the \textsf{hypertex}
driver, and then processing with \textsf{dvips -z}, but the DVI file is
not portable. The main advantage of using the Hyper\TeX\ \ci{special}
commands is that you can also use the document in hypertext DVI viewers,
such as \textsf{xdvi}.

\begin{description}
\item[driverfallback]
  If a driver is not given and cannot be autodetected, then use
  the driver option, given as value to this option \textsf{driverfallback}.
  Example:
  \begin{quote}
    \texttt{driverfallback=dvipdfm}
  \end{quote}
  Autodetected drivers (\textsf{pdftex}, \textsf{xetex}, \textsf{vtex},
  \textsf{vtexpdfmark}) are recognized from within \TeX\ and
  therefore cannot be given as value to option \textsf{driverfallback}.
  However a DVI driver program is run after the \TeX\ run is finished.
  Thus it cannot be detected at \TeX\ macro level. Then package
  \xpackage{hyperref}
  uses the driver, given by \textsf{driverfallback}. If the driver
  is already specified or can be autodetected, then option
  \textsf{driverfallback} is ignored.
\end{description}

\section{Implicit behavior}

This package can be used with more or less any normal \LaTeX\ document
by specifying in the document preamble

\begin{verbatim}
\usepackage{hyperref}
\end{verbatim}

Make sure it comes \emph{last} of your loaded packages, to give it a
fighting chance of not being over-written, since its job is to redefine
many \LaTeX\ commands. Hopefully you will find that all cross-references
work correctly as hypertext. For example, \ci{section} commands will
produce a bookmark and a link, whereas \ci{section*} commands will only
show links when paired with a corresponding \ci{addcontentsline}
command.

In addition, the \texttt{hyperindex} option (see below) attempts to make
items in the index by hyperlinked back to the text, and the option
\texttt{backref} inserts extra `back' links into the bibliography for
each entry. Other options control the appearance of links, and give
extra control over PDF output. For example, \texttt{colorlinks}, as its
name well implies, colors the links instead of using boxes; this is the
option used in this document.


\section{Package options}

All user-configurable aspects of \xpackage{hyperref} are set using a
single `key=value' scheme (using the \xpackage{keyval} package) with the
key \texttt{Hyp}. The options can be set either in the optional argument
to the \cs{usepackage} command, or using the \cs{hypersetup}
macro. When the package is loaded, a file \texttt{hyperref.cfg} is read
if it can be found, and this is a convenient place to set options on a
site-wide basis.

As an example, the behavior of a particular file could be controlled by:
\begin{itemize}

\item	a site-wide \texttt{hyperref.cfg} setting up the look of links,
adding backreferencing, and setting a PDF display default:

\begin{verbatim}
\hypersetup{backref,
pdfpagemode=FullScreen,
colorlinks=true}
\end{verbatim}

\item	A global option in the file, which is passed down to
\textsf{hyperref}:

\begin{verbatim}
\documentclass[dvips]{article}
\end{verbatim}

\item	File-specific options in the \cs{usepackage} commands, which
override the ones set in \texttt{hyperref.cfg}:

\begin{verbatim}
\usepackage[colorlinks=false]{hyperref}
\hypersetup{pdftitle={A Perfect Day}}
\end{verbatim}
\end{itemize}

As seen in the previous example, information entries
(pdftitle, pdfauthor, \dots) should be set after the package is loaded.
Otherwise \LaTeX\ expands the values of these options prematurely.
Also \LaTeX\ strips spaces in options. Especially option `pdfborder'
requires some care. Curly braces protect the value, if given
as package option. They are not necessary in \verb|\hypersetup|.

\begin{verbatim}
\usepackage[pdfborder={0 0 0}]{hyperref}
\hypersetup{pdfborder=0 0 0}
\end{verbatim}

Package `kvoptions-patch' patches \LaTeX\ to make it aware
of key value options and to prevent premature value expansions.

Some options can be given at any time, but many are restricted: before
\verb|\begin{document}|, only in \verb|\usepackage[...]{hyperref}|,
before first use, etc.

In the key descriptions that follow, many options do not need a value,
as they default to the value true if used. These are the ones classed as
`boolean'. The values true and false can always be specified, however.

\subsection{General options}

Firstly, the options to specify general behavior and page size.

\medskip
\noindent\begin{longtable}{>{\ttfamily}ll>{\itshape}ll}
draft          & boolean & false & all hypertext options are turned off \\
final          & boolean & true  & all hypertext options are turned on \\
debug          & boolean & false & extra diagnostic messages are printed in \\
               &         &       & the log file \\
verbose        & boolean & false & same as debug \\
implicit       & boolean & true  & redefines \LaTeX\ internals \\
setpagesize    & boolean & true  & sets page size by special driver commands
\end{longtable}

\subsection{Options for destination names}

Destinations names (also anchor, target or link names) are internal
names that identify a position on a page in the document. They
are used in link targets for inner document links or the bookmarks,
for example.

Usually anchor are set, if \cs{refstepcounter} is called.
Thus there is a counter name and value. Both are used to
construct the destination name. By default the counter value
follows the counter name separated by a dot. Example for
the fourth chapter:
\begin{quote}
  \verb|chapter.4|
\end{quote}
This scheme is used by:
\begin{description}
\item[\cs{autoref}] displays the description label for the
  reference depending on the counter name.
\item[\cs{hyperpage}] is used by the index to get
page links. Page anchor setting (\verb|pageanchor|) must not
be turned off.
\end{description}

It is very important that the destination names are unique,
because two destinations must not share the same name.
The counter value \cs{the<counter>} is not always unique
for the counter. For example, table and figures can be numbered
inside the chapter without having the chapter number in their
number. Therefore \xpackage{hyperref} has introduced \cs{theH<counter>}
that allows a unique counter value without messing up with
the appearance of the counter number. For example, the number
of the second table in the third chapter might be printed
as \texttt{2}, the result of \cs{thetable}. But the
destination name \texttt{table.2.4} is unique because it
has used \cs{theHtable} that gives \verb|2.4| in this case.

Often the user do not need to set \cs{theH<counter>}. Defaults
for standard cases (chapter, \dots) are provided. And after \xpackage{hyperref}
is loaded, new counters with parent counters also define
\cs{theH<counter>} automatically, if \cs{newcounter}, \cs{@addtoreset}
or \cs{numberwithin} of package \xpackage{amsmath} are used.

Usually problems with duplicate destination names can be solved
by an appropriate definition of \cs{theH<counter>}. If option
\texttt{hypertexnames} is disabled, then a unique artificial
number is used instead of the counter value. In case of page
anchors the absolute page anchor is used. With option \texttt{plainpages}
the page anchors use the arabic form. In both latter cases \cs{hyperpage}
for index links is affected and might not work properly.

If an unnumbered entity gets an anchor (starred forms of
chapters, sections, \dots) or \cs{phantomsection} is used,
then the dummy counter name \texttt{section*} and an artificial
unique number is used.

If the final PDF file is going to be merged with another file, than
the destination names might clash, because both documents might
contain \texttt{chapter.1} or \texttt{page.1}. Also \xpackage{hyperref}
sets anchor with name \texttt{Doc-Start} at the begin of the document.
This can be resolved by redefining \cs{HyperDestNameFilter}.
Package \xpackage{hyperref} calls this macro each time, it uses a
destination name.
The macro must be expandable and expects the destination name
as only argument. As example, the macro is redefined to add
a prefix to all destination names:
\begin{quote}
\begin{verbatim}
\renewcommand*{\HyperDestNameFilter}[1]{\jobname-#1}
\end{verbatim}
\end{quote}
In document \texttt{docA} the destination name \texttt{chapter.2}
becomes \texttt{docA-chapter.2}.

Destination names can also be used from the outside in URIs(, if the
driver has not removed or changed them), for example:
\begin{quote}
\begin{verbatim}
http://somewhere/path/file.pdf#nameddest=chapter.4
\end{verbatim}
\end{quote}
However using a number seems unhappy. If another chapter is added
before, the number changes. But it is very difficult to pass
a new name for the destination to the anchor setting process that
is usually deep hidden in the internals. The first name of
\cs{label} after the anchor setting seems a good approximation:
\begin{quote}
\begin{verbatim}
  \section{Introduction}
  \label{intro}
\end{verbatim}
\end{quote}
Option \texttt{destlabel} checks for each \cs{label}, if there is
a new destination name active and replaces the destination
name by the label name. Because the destination name is already in use
because of the anchor setting, the new name is recorded in the \texttt{.aux}
file and used in the subsequent \hologo{LaTeX} run. The renaming is done by
a redefinition of \cs{HyperDestNameFilter}. That leaves the old
destination names intact (e.g., they are needed for \cs{autoref}).
This redefinition is also available as \cs{HyperDestLabelReplace},
thus that an own redefinition can use it.
The following example also adds a prefix for \emph{all} destination names:
\begin{quote}
\begin{verbatim}
\renewcommand*{\HyperDestNameFilter}[1]{%
  \jobname-\HyperDestLabelReplace{#1}%
}
\end{verbatim}
\end{quote}
The other case that only files prefixed that do not have a corresponding
\cs{label} is more complicate, because \cs{HyperDestLabelReplace} needs
the unmodified destination name as argument. This is solved by an
expandable string test (\cs{pdfstrcmp} of \hologo{pdfTeX}
or \cs{strcmp} of \hologo{XeTeX}, package \xpackage{pdftexcmds} also supports
\hologo{LuaTeX}):
\begin{quote}
\begin{verbatim}
\usepackage{pdftexcmds}
\makeatletter
\renewcommand*{\HyperDestNameFilter}[1]{%
  \ifcase\pdf@strcmp{#1}{\HyperDestLabelReplace{#1}} %
    \jobname-#1%
  \else
    \HyperDestLabelReplace{#1}%
  \fi
}
\makeatother
\end{verbatim}
\end{quote}

With option \texttt{destlabel} destinations can also named manually,
if the destination is not yet renamed:
\begin{quote}
\verb|\HyperDestRename{|\meta{destination}\verb|}{|\meta{newname}\verb|}|
\end{quote}

Hint: Anchors can also be named and set by \cs{hypertarget}.

\medskip
\noindent\begin{longtable}{>{\ttfamily}ll>{\itshape}ll}
destlabel      & boolean & false & destinations are named by first \cs{label}\\
               &         &       & after anchor creation\\
hypertexnames  & boolean & true  & use guessable names for links \\
naturalnames   & boolean & false & use \LaTeX-computed names for links \\
plainpages     & boolean & false & Forces page anchors to be named by the Arabic form \\
               &         &       & of the page number, rather than the formatted form. \\
\end{longtable}

\subsection{Configuration options}

\noindent\begin{longtable}{>{\ttfamily}ll>{\itshape}lp{7cm}}
raiselinks & boolean & true  & In the hypertex driver, the height of links is normally calculated by the driver as
                               simply the base line of contained text; this options forces \verb|\special| commands to
                               reflect the real height of the link (which could contain a graphic) \\
breaklinks & boolean & false & Allows link text to break across lines; since this cannot be accommodated in PDF, it is
                               only set true by default if the pdftex driver is used. This makes links on multiple lines
                               into different PDF links to the same target. \\
pageanchor & boolean & true  & Determines whether every page is given an implicit anchor at the top left corner. If this
                               is turned off, \verb|\printindex| will not contain
                               valid hyperlinks. \\
nesting    & boolean & false & Allows links to be nested; no drivers currently support this.
\end{longtable}

Note for option \verb|breaklinks|:
The correct value is automatically set according to the driver features.
It can be overwritten for drivers that do not support broken links.
However, at any case, the link area will be wrong and displaced.

\subsection{Backend drivers}

If no driver is specified, the package tries to find a driver in
the following order:
\begin{enumerate}
\item Autodetection, some \TeX\ processors can be detected at \TeX\ macro
  level (pdf\TeX, Xe\TeX, V\TeX).
\item Option \textsf{driverfallback}. If this option is set, its value
  is taken as driver option.
\item Macro \cs{Hy@defaultdriver}. The macro takes a driver file
  name (without file extension).
\item Package default is \textsf{hypertex}.
\end{enumerate}
Many distributions are using a driver file \texttt{hypertex.cfg} that
define \cs{Hy@defaultdriver} with \texttt{hdvips}. This is recommended
because driver \textsf{dvips} provides much more features than
\textsf{hypertex} for PDF generation.

\noindent\begin{longtable}{>{\ttfamily}lp{.8\hsize}}
driverfallback & Its value is used as driver option\\
            & if the driver is not given or autodetected.\\
dvipdfm     & Sets up \textsf{hyperref} for use with the \textsf{dvipdfm} driver.\\
dvipdfmx    & Sets up \textsf{hyperref} for use with the \textsf{dvipdfmx} driver.\\
dvips       & Sets up \textsf{hyperref} for use with the \textsf{dvips} driver. \\
dvipsone    & Sets up \textsf{hyperref} for use with the \textsf{dvipsone} driver. \\
dviwindo    & Sets up \textsf{hyperref} for use with the \textsf{dviwindo} Windows previewer. \\
hypertex    & Sets up \textsf{hyperref} for use with the Hyper\TeX-compliant drivers. \\
latex2html  & Redefines a few macros for compatibility with \textsf{latex2html}. \\
nativepdf   & An alias for \textsf{dvips} \\
pdfmark     & An alias for \textsf{dvips} \\
pdftex      & Sets up \textsf{hyperref} for use with the \textsf{pdftex} program.\\
ps2pdf      & Redefines a few macros for compatibility with
              Ghostscript's PDF writer, otherwise identical to
              \textsf{dvips}. \\
tex4ht      & For use with \textsf{\TeX4ht} \\
textures    & For use with \textsf{Textures} \\
vtex        & For use with MicroPress' \textsf{VTeX}; the PDF
                       and HTML backends are detected automatically. \\
vtexpdfmark & For use with \textsf{VTeX}'s PostScript backend. \\
xetex       & For use with Xe\TeX (using backend for dvipdfm).
\end{longtable}
\smallskip

If you use \textsf{dviwindo}, you may need to redefine the macro
\ci{wwwbrowser} (the default is \verb|C:\netscape\netscape|) to tell
\textsf{dviwindo} what program to launch. Thus, users of Internet
Explorer might add something like this to hyperref.cfg:

\begin{verbatim}
\renewcommand{\wwwbrowser}{C:\string\Program\space
  Files\string\Plus!\string\Microsoft\space
  Internet\string\iexplore.exe}
\end{verbatim}

\subsection{Extension options}
\noindent\begin{longtable}{>{\ttfamily}ll>{\itshape}lp{6cm}}
extension      & text    &         & Set the file extension (e.g.\ \textsf{dvi}) which will be appended to file links
                                     created if you use the \xpackage{xr} package. \\
hyperfigures   & boolean &         & \\
backref        & text    & false   & Adds `backlink' text to the end of each item in the bibliography, as a list of
                                     section numbers. This can only work properly \emph{if} there is a blank line after
                                     each \verb|\bibitem|. Supported values are \verb|section|, \verb|slide|, \verb|page|,
                                     \verb|none|, or \verb|false|. If no value is given, \verb|section| is taken as default.\\
pagebackref    & boolean & false   & Adds `backlink' text to the end of each item in the bibliography, as a list of page
                                     numbers. \\
hyperindex     & boolean & true    & Makes the page numbers of index entries into hyperlinks. Relays on unique
                                     page anchors (\verb|pageanchor|, \ldots)\\
                                     \verb|pageanchors| and \verb|plainpages=false|. \\
hyperfootnotes & boolean & true    & Makes the footnote marks into hyperlinks to the footnote text.
                                     Easily broken \ldots\\
encap          &         &         & Sets encap character for hyperindex \\
linktoc        & text    & section & make text (\verb|section|), page number (\verb|page|), both (\verb|all|) or nothing (\verb|none|) be link on TOC, LOF and LOT \\
linktocpage    & boolean & false   & make page number, not text, be link on TOC, LOF and LOT \\
breaklinks     & boolean & false   & allow links to break over lines by making links over multiple lines into PDF links to
                                     the same target \\
colorlinks     & boolean & false   & Colors the text of links and anchors. The colors chosen depend on the the type of
                                     link. At present the only types of link distinguished are citations, page references,
                                     URLs, local file references, and other links.
                                     Unlike colored boxes, the colored
                                     text remains when printing.\\
linkcolor      & color   & red     & Color for normal internal links. \\
anchorcolor    & color   & black   & Color for anchor text. \\
citecolor      & color   & green   & Color for bibliographical citations in text. \\
filecolor      & color   & cyan    & Color for URLs which open local files. \\
menucolor      & color   & red     & Color for Acrobat menu items. \\
runcolor       & color   & filecolor & Color for run links (launch annotations). \\
urlcolor       & color   & magenta & Color for linked URLs. \\
allcolors      & color   &         & Set all color options (without border and field options).\\
frenchlinks    & boolean & false   & Use small caps instead of color for links.\\
hidelinks      &         &         & Hide links (removing color and border). \\
\end{longtable} \smallskip

Note that all color names must be defined before use, following the
normal system of the standard \LaTeX\ \xpackage{color} package.

\subsection{PDF-specific display options}
\noindent\begin{longtable}{@{}>{\ttfamily}ll>{\itshape}lp{7.5cm}@{}}
bookmarks          & boolean   & true   & A set of Acrobat bookmarks are written, in a manner similar to the
                                           table of contents, requiring two passes of \LaTeX. Some postprocessing
                                           of the bookmark file (file extension \texttt{.out}) may be needed to
                                           translate \LaTeX\ codes, since bookmarks must be written in  PDFEncoding.
                                           To aid this  process, the \texttt{.out} file is not rewritten by \LaTeX\
                                           if it is edited to contain a line \verb|\let\WriteBookmarks\relax| \\
bookmarksopen      & boolean   & false   & If Acrobat bookmarks are requested, show them with all the subtrees
                                           expanded. \\
bookmarksopenlevel & parameter &         & level (\ci{maxdimen}) to which bookmarks are open \\
bookmarksnumbered  & boolean   & false   & If Acrobat bookmarks are requested, include section numbers. \\
bookmarkstype      & text      & toc     & to specify which `toc' file to mimic \\
CJKbookmarks       & boolean   & false   &
    This option should be used to produce CJK bookmarks.
    Package \verb|hyperref|
    supports both normal and preprocessed mode of the \xpackage{CJK} package;
    during the creation of bookmarks, it simply replaces CJK's macros
    with special versions which expand to the corresponding character
    codes.  Note that without the `unicode' option of hyperref you get
    PDF files which actually violate the PDF specification because
    non-Unicode character codes are used -- some PDF readers localized
    for CJK languages (most notably Acroread itself) support this.
    Also note that option `CJKbookmarks' cannot be used together
    with option `unicode'.

    No mechanism is provided to translate non-Unicode bookmarks to
    Unicode; for portable PDF documents only Unicode encoding should
    be used.\\
pdfhighlight       & name      & /I      & How link buttons behave when selected; /I is for inverse (the default);
                                           the other possibilities are /N (no effect), /O (outline), and /P (inset
                                           highlighting). \\
citebordercolor    & RGB color & 0 1 0   & The color of the box around citations \\
filebordercolor    & RGB color & 0 .5 .5 & The color of the box around links to files \\
linkbordercolor    & RGB color & 1 0 0   & The color of the box around normal links \\
menubordercolor    & RGB color & 1 0 0   & The color of the box around Acrobat menu links \\
urlbordercolor     & RGB color & 0 1 1   & The color of the box around links to URLs \\
runbordercolor     & RGB color & 0 .7 .7 & Color of border around `run' links \\
allbordercolors    &           &         & Set all border color options \\
pdfborder          &           & 0 0 1   & The style of box around links; defaults to a box with lines of 1pt thickness,
                                           but the colorlinks option resets it to produce no border.
\end{longtable}

Note that the color of link borders can be specified \emph{only} as 3
numbers in the range 0..1, giving an RGB color. You cannot use colors
defined in \TeX. Since version 6.76a this is no longer true.
Especially with the help of package \xpackage{xcolor} the usual
color specifications of package \xpackage{(x)color} can be used.
For further information see description of package \xpackage{hycolor}.

The bookmark commands are stored in a file called
\textit{jobname}\texttt{.out}. The files is not processed by \LaTeX\ so
any markup is passed through. You can postprocess this file as needed;
as an aid for this, the \texttt{.out} file is not overwritten on the
next \TeX\ run if it is edited to contain the line \\

\begin{verbatim}
\let\WriteBookmarks\relax
\end{verbatim}

\subsection{PDF display and information options}
\noindent\begin{longtable}{@{}>{\ttfamily}l>{\raggedright}p{\widthof{key value}}>{\itshape}lp{7cm}@{}}
baseurl            & URL     &       & Sets the base URL of the PDF document \\
pdfpagemode        & text    & empty & Determines how the file is opening in Acrobat; the possibilities are
                                       \texttt{UseNone}, \texttt{UseThumbs} (show thumbnails), \texttt{UseOutlines}
                                       (show bookmarks), \texttt{FullScreen}, \texttt{UseOC} (PDF 1.5),
                                       and \texttt{UseAttachments} (PDF 1.6). If no mode if explicitly chosen, but the
                                       bookmarks option is set, \texttt{UseOutlines} is used. \\
pdftitle           & text    &       & Sets the document information Title field \\
pdfauthor          & text    &       & Sets the document information Author field \\
pdfsubject         & text    &       & Sets the document information Subject field \\
pdfcreator         & text    &       & Sets the document information Creator field \\
pdfproducer        & text    &       & Sets the document information Producer field \\
pdfkeywords        & text    &       & Sets the document information Keywords field \\
pdftrapped         & text    & empty & Sets the document information Trapped entry.\\
&&& Possible values are \texttt{True}, \texttt{False} and \texttt{Unknown}.\\
&&& An empty value means, the entry is not set.\\
pdfinfo            & key value list & empty & Alternative interface for setting the
                                              document information.\\
pdfview            & text    & XYZ   & Sets the default PDF `view' for each link \\
pdfstartpage       & text    & 1     & Determines on which page the PDF file is opened. \\
pdfstartview       & text    & Fit   & Set the startup page view \\
pdfremotestartview & text    & Fit   & Set the startup page view of remote PDF files \\
pdfpagescrop       & n n n n &       & Sets the default PDF crop box for pages. This should be a set of four numbers \\
pdfcenterwindow    & boolean & false & position the document window in the center of the screen \\
pdfdirection       & text    & empty & direction setting \\
pdfdisplaydoctitle & boolean & false & display document title instead of
                                       file name in title bar\\
pdfduplex          & text    & empty & paper handling option for print dialog\\
pdffitwindow       & boolean & false & resize document window to fit document size \\
pdflang            & text    & relax & PDF language identifier (RFC 3066)\\
pdfmenubar         & boolean & true  & make PDF viewer's menu bar visible \\
pdfnewwindow       & boolean & false & make links that open another PDF file start a new window \\
pdfnonfullscreenpagemode
                   & boolean & empty & page mode setting on exiting
                                       full-screen mode\\
pdfnumcopies       & integer & empty & number of printed copies \\
pdfpagelayout      & text    & empty & set layout of PDF pages \\
pdfpagelabels      & boolean & true  & set PDF page labels \\
pdfpagetransition  & text    & empty & set PDF page transition style \\
pdfpicktraybypdfsize & text & empty & set option for print dialog \\
pdfprintarea       & text    & empty & set /PrintArea of viewer preferences \\
pdfprintclip       & text    & empty & set /PrintClip of viewer preferences \\
pdfprintpagerange  & n n (n n)*
                             & empty & set /PrintPageRange of viewer
                                       preferences\\
pdfprintscaling    & text    & empty & page scaling option for print dialog
                                       (option /PrintScaling of viewer
                                       preferences, PDF 1.6);
                                       valid values are \texttt{None} and
                                       \texttt{AppDefault} \\
pdftoolbar         & boolean & true  & make PDF toolbar visible \\
pdfviewarea        & text    & empty & set /ViewArea of viewer preferences \\
pdfviewclip        & text    & empty & set /ViewClip of viewer preferences \\
pdfwindowui        & boolean & true  & make PDF user interface elements visible \\
unicode            & boolean & false & Unicode encoded PDF strings
\end{longtable}

Each link in Acrobat carries its own magnification level, which is set
using PDF coordinate space, which is not the same as \TeX's. The unit
is bp and the origin is in the lower left corner. See also
\verb|\hypercalcbp| that is explained on page \pageref{hypercalcbp}.
pdf\TeX\
works by supplying default values for \texttt{XYZ} (horizontal $\times$
vertical $\times$ zoom) and \texttt{FitBH}. However, drivers using
\texttt{pdfmark} do not supply defaults, so \textsf{hyperref} passes in
a value of -32768, which causes Acrobat to set (usually) sensible
defaults. The following are possible values for the \texttt{pdfview},
\texttt{pdfstartview} and \texttt{pdfremotestartview} parameters.

\noindent\begin{longtable}{>{\ttfamily}l>{\itshape}lp{7cm}}
XYZ   & left top zoom         & Sets a coordinate and a zoom factor. If any one is null, the source link value is used.
                                \textit{null null null} will give the same values as the current page. \\
Fit   &                       & Fits the page to the window. \\
FitH  & top                   & Fits the width of the page to the window. \\
FitV  & left                  & Fits the height of the page to the window. \\
FitR  & left bottom right top & Fits the rectangle specified by the four coordinates to the window. \\
FitB  &                       & Fits the page bounding box to the window. \\
FitBH & top                   & Fits the width of the page bounding box to the window. \\
FitBV & left                  & Fits the height of the page bounding box to the window. \\
\end{longtable}

The \texttt{pdfpagelayout} can be one of the following values.

\noindent\begin{longtable}{>{\ttfamily}lp{10cm}}
SinglePage     & Displays a single page; advancing flips the page \\
OneColumn      & Displays the document in one column; continuous scrolling. \\
TwoColumnLeft  & Displays the document in two columns, odd-numbered pages to the left. \\
TwoColumnRight & Displays the document in two columns, odd-numbered pages to the right.\\
TwoPageLeft    & Displays two pages, odd-numbered pages to the left (since PDF 1.5).\\
TwoPageRight   & Displays two pages, odd-numbered pages to the right (since PDF 1.5).
\end{longtable}

Finally, the \texttt{pdfpagetransition} can be one of the following
values, where \textit{/Di} stands for direction of motion in degrees,
generally in 90$^{\circ}$\ steps, \textit{/Dm} is a horizontal
(\texttt{/H}) or vertical (\texttt{/V}) dimension (e.g.\ \texttt{Blinds
/Dm /V}), and \textit{/M} is for motion, either in (\texttt{/I}) or out
(\texttt{/O}).

\noindent\begin{longtable}{>{\ttfamily}llp{8.5cm}}
Blinds   & /Dm    & Multiple lines distributed evenly across the screen sweep in the same direction to reveal the new
                    page. \\
Box      & /M     & A box sweeps in or out. \\
Dissolve &        & The page image dissolves in a piecemeal fashion to reveal the new page. \\
Glitter  & /Di    & Similar to Dissolve, except the effect sweeps across the screen. \\
Split    & /Dm /M & Two lines sweep across the screen to reveal the new page. \\
Wipe     & /Di    & A single line sweeps across the screen to reveal the new page.
\end{longtable}

\subsection{Option \texttt{pdfinfo}}

The information entries can be set using \texttt{pdftitle},
\texttt{pdfsubject}, \dots. Option \texttt{pdfinfo} provides an alternative
interface. It takes a key value list. The key names are the names that
appear in the PDF information dictionary directly. Known keys such as
\texttt{Title}, \texttt{Subject}, \texttt{Trapped} and other are mapped to
options \texttt{pdftitle}, \texttt{subject}, \texttt{trapped}, \dots
Unknown keys are added to the information dictionary. Their values are text
strings (see PDF specification).
Example:
\begin{quote}
\begin{verbatim}
\hypersetup{
  pdfinfo={
    Title={My Title},
    Subject={My Subject},
    NewKey={Foobar},
    % ...
  }
}
\end{verbatim}
\end{quote}

\subsection{Big alphabetical list}

The following is a complete listing of available options for
\textsf{hyperref}, arranged alphabetically.

\noindent\begin{longtable}{@{}>{\ttfamily}llp{7cm}@{}}
anchorcolor        & \textit{black}         & set color of anchors \\
backref            & \textit{false}         & do bibliographical back references \\
baseurl            & \textit{empty}         & set base URL for document \\
bookmarks          & \textit{true}          & make bookmarks \\
bookmarksnumbered  & \textit{false}         & put section numbers in bookmarks \\
bookmarksopen      & \textit{false}         & open up bookmark tree \\
bookmarksopenlevel & \ttfamily\ci{maxdimen} & level to which bookmarks are open \\
bookmarkstype      & \textit{toc}           & to specify which `toc' file to mimic \\
breaklinks         & \textit{false}         & allow links to break over lines \\
CJKbookmarks       & \textit{false}         & to produce CJK bookmarks\\
citebordercolor    & \textit{0 1 0}         & color of border around cites \\
citecolor          & \textit{green}         & color of citation links \\
colorlinks         & \textit{false}         & color links \\
                   & \textit{true}          & (\textsf{tex4ht}, \textsf{dviwindo}) \\
debug              & \textit{false}         & provide details of anchors defined; same as verbose \\
destlabel          & \textit{false}         & destinations are named by the first \verb|\label| after the anchor creation \\
draft              & \textit{false}         & do not do any hyperlinking \\
dvipdfm            &                        & use \textsf{dvipdfm} backend \\
dvipdfmx           &                        & use \textsf{dvipdfmx} backend \\
dvips              &                        & use \textsf{dvips} backend \\
dvipsone           &                        & use \textsf{dvipsone} backend \\
dviwindo           &                        & use \textsf{dviwindo} backend \\
encap              &                        & to set encap character for hyperindex \\
extension          & \textit{dvi}           & suffix of linked files \\
filebordercolor    & \textit{0 .5 .5}       & color of border around file links \\
filecolor          & \textit{cyan}          & color of file links \\
final              & \textit{true}          & opposite of option draft \\
frenchlinks        & \textit{false}         & use small caps instead of color for links \\
hyperfigures       & \textit{false}         & make figures hyper links \\
hyperfootnotes     & \textit{true}          & set up hyperlinked footnotes \\
hyperindex         & \textit{true}          & set up hyperlinked indices \\
hypertex           &                        & use \textsf{Hyper\TeX} backend \\
hypertexnames      & \textit{true}          & use guessable names for links \\
implicit           & \textit{true}          & redefine \LaTeX\ internals \\
latex2html         &                        & use \textsf{\LaTeX2HTML} backend \\
linkbordercolor    & \textit{1 0 0}         & color of border around links \\
linkcolor          & \textit{red}           & color of links \\
linktoc            & \textit{section}       & make text be link on TOC, LOF and LOT \\
linktocpage        & \textit{false}         & make page number, not text, be link on TOC, LOF and LOT \\
menubordercolor    & \textit{1 0 0}         & color of border around menu links \\
menucolor          & \textit{red}           & color for menu links \\
nativepdf          & \textit{false}         & an alias for \textsf{dvips} \\
naturalnames       & \textit{false}         & use \LaTeX-computed names for links \\
nesting            & \textit{false}         & allow nesting of links \\
pageanchor         & \textit{true}          & put an anchor on every page \\
pagebackref        & \textit{false}         & backreference by page number \\
pdfauthor          & \textit{empty}         & text for PDF Author field \\
pdfborder          & \textit{0 0 1}         & width of PDF link border \\
                   & \textit{0 0 0}         & (\texttt{colorlinks)} \\
pdfcenterwindow    & \textit{false}         & position the document window in the center of the screen \\
pdfcreator         & \textit{LaTeX with}    & text for PDF Creator field \\
                   & \textit{hyperref}      & \\
                   & \textit{package}       & \\
pdfdirection       & \textit{empty}         & direction setting \\
pdfdisplaydoctitle & \textit{false}         & display document title instead
                                              of file name in title bar\\
pdfduplex          & \textit{empty}         & paper handling option for
                                              print dialog\\
pdffitwindow       & \textit{false}         & resize document window to fit document size \\
pdfhighlight       & \textit{/I}            & set highlighting of PDF links \\
pdfinfo            & \textit{empty}         & alternative interface for setting document information\\
pdfkeywords        & \textit{empty}         & text for PDF Keywords field \\
pdflang            & \textit{relax}         & PDF language identifier (RFC 3066) \\
pdfmark            & \textit{false}         & an alias for \textsf{dvips} \\
pdfmenubar         & \textit{true}          & make PDF viewer's menu bar visible \\
pdfnewwindow       & \textit{false}         & make links that open another PDF \\
                   &                        & file start a new window \\
pdfnonfullscreenpagemode
                   & \textit{empty}         & page mode setting on exiting
                                              full-screen mode\\
pdfnumcopies       & \textit{empty}         & number of printed copies\\
pdfpagelayout      & \textit{empty}         & set layout of PDF pages \\
pdfpagemode        & \textit{empty}         & set default mode of PDF display \\
pdfpagelabels      & \textit{true}          & set PDF page labels \\
pdfpagescrop       & \textit{empty}         & set crop size of PDF document \\
pdfpagetransition  & \textit{empty}         & set PDF page transition style \\
pdfpicktraybypdfsize
                   & \textit{empty}         & set option for print dialog \\
pdfprintarea       & \textit{empty}         & set /PrintArea of viewer preferences \\
pdfprintclip       & \textit{empty}         & set /PrintClip of viewer preferences \\
pdfprintpagerange  & \textit{empty}         & set /PrintPageRange of viewer preferences \\
pdfprintscaling    & \textit{empty}         & page scaling option for print dialog \\
pdfproducer        & \textit{empty}         & text for PDF Producer field \\
pdfremotestartview & \textit{Fit}           & starting view of remote PDF documents \\
pdfstartpage       & \textit{1}             & page at which PDF document opens \\
pdfstartview       & \textit{Fit}           & starting view of PDF document \\
pdfsubject         & \textit{empty}         & text for PDF Subject field \\
pdftex             &                        & use \textsf{pdf\TeX} backend \\
pdftitle           & \textit{empty}         & text for PDF Title field \\
pdftoolbar         & \textit{true}          & make PDF toolbar visible \\
pdftrapped         & \textit{empty} & Sets the document information Trapped entry.
  Possible values are \texttt{True}, \texttt{False} and \texttt{Unknown}.
  An empty value means, the entry is not set.\\
pdfview            & \textit{XYZ}           & PDF `view' when on link traversal \\
pdfviewarea        & \textit{empty}         & set /ViewArea of viewer preferences \\
pdfviewclip        & \textit{empty}         & set /ViewClip of viewer preferences \\
pdfwindowui        & \textit{true}          & make PDF user interface elements visible \\
plainpages         & \textit{false}         & do page number anchors as plain Arabic \\
ps2pdf             &                        & use \textsf{ps2pdf} backend \\
raiselinks         & \textit{false}         & raise up links (for \textsf{Hyper\TeX} backend) \\
runbordercolor     & \textit{0 .7 .7}       & color of border around `run' links \\
runcolor           & \textit{filecolor}     & color of `run' links\\
setpagesize        & \textit{true}          & set page size by special driver commands \\
tex4ht             &                        & use \textsf{\TeX4ht} backend \\
textures           &                        & use \textsf{Textures} backend \\
unicode            & \textit{false}         & Unicode encoded pdf strings \\
urlbordercolor     & \textit{0 1 1}         & color of border around URL links \\
urlcolor           & \textit{magenta}       & color of URL links \\
verbose            & \textit{false}         & be chatty \\
vtex               &                        & use \textsf{VTeX} backend\\
xetex              &                        & use \textsf{Xe\TeX} backend\\
\end{longtable}

\section{Additional user macros}

If you need to make references to URLs, or write explicit links, the
following low-level user macros are provided:

\begin{cmdsyntax}
\ci{href}\verb|[|\emph{options}\verb|]|\verb|{|\emph{URL}\verb|}{|\emph{text}\verb|}|
\end{cmdsyntax}

\noindent The \emph{text} is made a hyperlink to the \emph{URL}; this
must be a full URL (relative to the base URL, if that is defined). The
special characters \# and \~{} do \emph{not} need to be escaped in any
way.

The optional argument \emph{options} recognizes the hyperref options
\texttt{pdfremotestartview}, \texttt{pdfnewwindow} and the following
key value options:
\begin{description}
\item[\texttt{page}:] Specifies the start page number of remote
PDF documents. First page is \verb|1|.
\item[\texttt{ismap}:] Boolean key, if set to |true|, the
URL should appended by the coordinates as query parameters by
the PDF viewer.
\item[\texttt{nextactionraw}:] The value of key |/Next| of
action dictionaries, see PDF specification.
\end{description}

\begin{cmdsyntax}
\ci{url}\verb|{|\emph{URL}\verb|}|
\end{cmdsyntax}

\noindent Similar to
\ci{href}\verb|{|\emph{URL}\verb|}{|\ci{nolinkurl}\verb|{|\emph{URL}\verb|}}|.
Depending on the driver \verb|\href| also tries to detect the link type.
Thus the result can be a url link, file link, \dots

\begin{cmdsyntax}
\ci{nolinkurl}\verb|{|\emph{URL}\verb|}|
\end{cmdsyntax}

\noindent Write \emph{URL} in the same way as \verb|\url|,
  without creating a hyperlink.

\begin{cmdsyntax}
\ci{hyperbaseurl}\verb|{|\emph{URL}\verb|}|
\end{cmdsyntax}

\noindent A base \emph{URL} is established, which is prepended to other
specified URLs, to make it easier to write portable documents.

\begin{cmdsyntax}
\ci{hyperimage}\verb|{|\emph{imageURL}\verb|}{|\emph{text}\verb|}|
\end{cmdsyntax}

\noindent The link to the image referenced by the URL is inserted, using
\emph{text} as the anchor.

  For drivers that produce HTML, the image itself is inserted by the
browser, with the \emph{text} being ignored completely.

\begin{cmdsyntax}
\ci{hyperdef}\verb|{|\emph{category}\verb|}{|\emph{name}\verb|}{|\emph{text}\verb|}|
\end{cmdsyntax}

\noindent A target area of the document (the \emph{text}) is marked, and
given the name \emph{category.name}

\begin{cmdsyntax}
\ci{hyperref}\verb|{|\emph{URL}\verb|}{|\emph{category}\verb|}{|\emph{name}\verb|}{|\emph{text}\verb|}|
\end{cmdsyntax}

\noindent \emph{text} is made into a link to \emph{URL\#category.name}

\begin{cmdsyntax}
\ci{hyperref}\verb|[|\emph{label}\verb|]{|\emph{text}\verb|}|
\end{cmdsyntax}

\noindent
\emph{text} is made into a link to the same place as
\verb|\ref{|\emph{label}\verb|}| would be linked.

\begin{cmdsyntax}
\ci{hyperlink}\verb|{|\emph{name}\verb|}{|\emph{text}\verb|}|
\end{cmdsyntax}
\begin{cmdsyntax}
\ci{hypertarget}\verb|{|\emph{name}\verb|}{|\emph{text}\verb|}|
\end{cmdsyntax}

\noindent A simple internal link is created with \verb|\hypertarget|,
with two parameters of an anchor \emph{name}, and anchor
\emph{text}. \verb|\hyperlink| has two arguments, the name of a
hypertext object defined somewhere by \verb|\hypertarget|, and the
\emph{text} which be used as the link on the page.

Note that in HTML parlance, the \verb|\hyperlink| command inserts a
notional \# in front of each link, making it relative to the current
testdocument; \verb|\href| expects a full URL.

\begin{cmdsyntax}
\ci{phantomsection}
\end{cmdsyntax}

\noindent
This sets an anchor at this location. It works similar to
\verb|\hypertarget{}{}| with an automatically chosen anchor name.
Often it is used in conjunction
with \verb|\addcontentsline| for sectionlike things (index, bibliography,
preface). \verb|\addcontentsline| refers to the latest previous location
where an anchor is set. Example:
\begin{quote}
\begin{verbatim}
\cleardoublepage
\phantomsection
\addcontentsline{toc}{chapter}{\indexname}
\printindex
\end{verbatim}
\end{quote}
Now the entry in the table of contents (and bookmarks) for the
index points to the start of the index page, not to a location
before this page.

\begin{cmdsyntax}
\ci{autoref}\verb|{|\emph{label}\verb|}|
\end{cmdsyntax}

\noindent
This is a replacement for the usual \verb|\ref| command that places a
contextual label in front of the reference. This gives your users a
bigger target to click for hyperlinks (e.g.\ `section 2' instead of
merely the number `2').

The label is worked out from the context of the original \verb|\label|
command by \textsf{hyperref} by using the macros listed below (shown
with their default values). The macros can be (re)defined in documents
using \verb|\(re)newcommand|; note that some of these macros are already
defined in the standard document classes. The mixture of lowercase and
uppercase initial letters is deliberate and corresponds to the author's
practice.

For each macro below, \textsf{hyperref} checks \ci{*autorefname} before
\ci{*name}.  For instance, it looks for \ci{figureautorefname} before
\ci{figurename}.

\noindent\begin{longtable}{lp{10cm}}
\textit{Macro}         & \textit{Default} \\
\ci{figurename}        & Figure \\
\ci{tablename}         & Table \\
\ci{partname}          & Part \\
\ci{appendixname}      & Appendix \\
\ci{equationname}      & Equation \\
\ci{Itemname}          & item \\
\ci{chaptername}       & chapter \\
\ci{sectionname}       & section \\
\ci{subsectionname}    & subsection \\
\ci{subsubsectionname} & subsubsection \\
\ci{paragraphname}     & paragraph \\
\ci{Hfootnotename}     & footnote \\
\ci{AMSname}           & Equation \\
\ci{theoremname}       & Theorem\\
\ci{page}              & page\\
\end{longtable}

Example for a redefinition if \textsf{babel} is used:
\begin{quote}
\begin{verbatim}
\usepackage[ngerman]{babel}
\addto\extrasngerman{%
  \def\subsectionautorefname{Unterkapitel}%
}
\end{verbatim}
\end{quote}

Hint: \cs{autoref} works via the counter name that the reference
is based on. Sometimes \cs{autoref} chooses the wrong name, if
the counter is used for different things. For example, it happens
with \cs{newtheorem} if a lemma shares a counter with theorems.
Then package \xpackage{aliascnt} provides a method to generate a
simulated second counter that allows the differentiation between
theorems and lemmas:
\begin{quote}
\begin{verbatim}
\documentclass{article}

\usepackage{aliascnt}
\usepackage{hyperref}

\newtheorem{theorem}{Theorem}

\newaliascnt{lemma}{theorem}
\newtheorem{lemma}[lemma]{Lemma}
\aliascntresetthe{lemma}

\providecommand*{\lemmaautorefname}{Lemma}

\begin{document}

We will use \autoref{a} to prove \autoref{b}.

\begin{lemma}\label{a}
  Nobody knows.
\end{lemma}

\begin{theorem}\label{b}
  Nobody is right.
\end{theorem}.

\end{document}
\end{verbatim}
\end{quote}

\begin{cmdsyntax}
\ci{autopageref}\verb|{|\emph{label}\verb|}|
\end{cmdsyntax}

\noindent
It replaces \verb|\pageref| and adds the name for page in front of
the page reference. First \ci{pageautorefname} is checked before
\ci{pagename}.

For instances where you want a reference to use the correct counter, but
not to create a link, there are starred forms:

\begin{cmdsyntax}
\ci{ref*}\verb|{|\emph{label}\verb|}|
\end{cmdsyntax}

\begin{cmdsyntax}
\ci{pageref*}\verb|{|\emph{label}\verb|}|
\end{cmdsyntax}

\begin{cmdsyntax}
\ci{autoref*}\verb|{|\emph{label}\verb|}|
\end{cmdsyntax}

\begin{cmdsyntax}
\ci{autopageref*}\verb|{|\emph{label}\verb|}|
\end{cmdsyntax}

A typical use would be to write
\begin{verbatim}
\hyperref[other]{that nice section (\ref*{other}) we read before}
\end{verbatim}

We want \verb|\ref*{other}| to generate the correct number, but not to
form a link, since we do this ourselves with \ci{hyperref}.

\begin{cmdsyntax}
\ci{pdfstringdef}\verb|{|\emph{macroname}\verb|}{|\emph{\TeX string}\verb|}|
\end{cmdsyntax}

\ci{pdfstringdef} returns a macro containing the PDF string. (Currently
this is done globally, but do not rely on it.) All the following tasks,
definitions and redefinitions are made in a group to keep them local:

\begin{itemize}
\item Switching to PD1 or PU encoding
\item Defining the \Quote{octal sequence commands} (\verb|\345|): \verb|\edef\3{\string\3}|
\item Special glyphs of \TeX: \verb|\{|, \verb|\%|, \verb|\&|, \verb|\space|, \verb|\dots|, etc.
\item National glyphs (\textsf{german.sty}, \textsf{french.sty}, etc.)
\item Logos: \verb|\TeX|, \verb|\eTeX|, \verb|\MF|, etc.
\item Disabling commands that do not provide useful functionality in bookmarks:
\verb|\label|, \verb|\index|, \verb|\glossary|, \verb|\discretionary|, \verb|\def|, \verb|\let|, etc.
\item \LaTeX's font commands like \verb|\textbf|, etc.
\item Support for \verb|\xspace| provided by the \xpackage{xspace} package
\end{itemize}

In addition, parentheses are protected to avoid the danger of unsafe
unbalanced parentheses in the PDF string. For further details, see Heiko
Oberdiek's Euro\TeX\ paper distributed with \textsf{hyperref}.

\subsection{Bookmark macros}

\subsubsection{Setting bookmarks}

Usually \textsf{hyperref} automatically adds bookmarks for
\verb|\section| and similar macros. But they can also set manually.

\begin{cmdsyntax}
\ci{pdfbookmark}\verb|[|\emph{level}\verb|]{|text\verb|}{|\emph{name}\verb|}|
\end{cmdsyntax}
creates a bookmark with the specified text and at the given level (default is
0). As name for the internal anchor name is used (in conjunction with
level). Therefore the name must be unique (similar to \verb|\label|).

\begin{cmdsyntax}
\ci{currentpdfbookmark}\verb|{|\emph{text}\verb|}{|\emph{name}\verb|}|
\end{cmdsyntax}
creates a bookmark at the current level.

\begin{cmdsyntax}
\ci{subpdfbookmark}\verb|{|\emph{text}\verb|}{|\emph{name}\verb|}|
\end{cmdsyntax}
creates a bookmark one step down in the bookmark hierarchy.
Internally the current level is increased by one.

\begin{cmdsyntax}
\ci{belowpdfbookmark}\verb|{|\emph{text}\verb|}{|\emph{name}\verb|}|
\end{cmdsyntax}
creates a bookmark below the current bookmark level. However after the
command the current bookmark level has not changed.

\noindent \textbf{Hint:} Package \textsf{bookmark} replaces
\textsf{hyperref}'s bookmark organization by a new algorithm:
\begin{itemize}
\item Usually only one \LaTeX\ run is needed.
\item More control over the bookmark appearance (color, font).
\item Different bookmark actions are supported (external file links,
  URLs, \dots).
\end{itemize}
Therefore I recommend using this package.

\subsubsection{Replacement macros}

\textsf{hyperref} takes the text for bookmarks from the arguments of
commands like \ci{section}, which can contain things like math, colors,
or font changes, none of which will display in bookmarks as is.

\begin{cmdsyntax}
\ci{texorpdfstring}\verb|{|\emph{\TeX string}\verb|}{|\emph{PDFstring}\verb|}|
\end{cmdsyntax}

For example,
\begin{verbatim}
\section{Pythagoras:
  \texorpdfstring{$ a^2 + b^2 = c^2 $}{%
    a\texttwosuperior\ + b\texttwosuperior\ =
    c\texttwosuperior
  }%
}
\section{\texorpdfstring{\textcolor{red}}{}{Red} Mars}
\end{verbatim}

\ci{pdfstringdef} executes the hook \pdfstringdefPreHook before it
expands the string. Therefore, you can use this hook to perform
additional tasks or to disable additional commands.

\begin{verbatim}
\expandafter\def\expandafter\pdfstringdefPreHook
\expandafter{%
  \pdfstringdefPreHook
  \renewcommand{\mycommand}[1]{}%
}
\end{verbatim}

However, for disabling commands, an easier way is via
\ci{pdfstringdefDisableCommands}, which adds its argument to the
definition of \ci{pdfstringdefPreHook} (`@' can here be used as letter in
command names):

\begin{verbatim}
\pdfstringdefDisableCommands{%
  \let~\textasciitilde
  \def\url{\pdfstringdefWarn\url}%
  \let\textcolor\@gobble
}
\end{verbatim}

\subsection{Utility macros}

\label{hypercalcbp}
\begin{cmdsyntax}
\ci{hypercalcbp}\verb|{|\emph{dimen specification}\verb|}|
\end{cmdsyntax}
\noindent
\verb|\hypercalcbp| takes a \TeX\ dimen specification and
converts it to bp and returns the number without the unit.
This is useful for options \verb|pdfview|, \verb|pdfstartview|
and \verb|pdfremotestartview|.
Example:
\begin{quote}
\begin{verbatim}
\hypersetup{
  pdfstartview={FitBH \hypercalcbp{\paperheight-\topmargin-1in
    -\headheight-\headsep}
}
\end{verbatim}
\end{quote}
The origin of the PDF coordinate system is the lower left corner.

Note, for calculations you need either package \xpackage{calc} or
\hologo{eTeX}. Nowadays the latter should automatically be enabled
for \hologo{LaTeX} formats. Users without \hologo{eTeX}, please, look
in the source documentation \verb|hyperref.dtx| for further
limitations.

Also \cs{hypercalcbp} cannot be used in option specifications
of \cs{documentclass} and \cs{usepackage}, because
\hologo{LaTeX} expands the option lists of these commands. However
package \xpackage{hyperref} is not yet loaded and an undefined control
sequence error would arise.

\section{Acrobat-specific behavior}
If you want to access the menu options of Acrobat Reader or Exchange, the following
macro is provided in the appropriate drivers:

\begin{cmdsyntax}
\ci{Acrobatmenu}\verb|{|\emph{menuoption}\verb|}{|\emph{text}\verb|}|
\end{cmdsyntax}

\noindent The \emph{text} is used to create a button which activates the appropriate \emph{menuoption}. The following table lists the option names you can use---comparison of this with the menus in Acrobat Reader or Exchange will show what they do. Obviously some are only appropriate to Exchange.

\medskip
\noindent\begin{longtable}{lp{9cm}}
File                          & Open, Close, Scan, Save, SaveAs, Optimizer:SaveAsOpt, Print, PageSetup, Quit \\
File$\rightarrow$Import       & ImportImage, ImportNotes, AcroForm:ImportFDF \\
File$\rightarrow$Export       & ExportNotes, AcroForm:ExportFDF \\
File$\rightarrow$DocumentInfo & GeneralInfo, OpenInfo, FontsInfo, SecurityInfo, Weblink:Base, AutoIndex:DocInfo \\
File$\rightarrow$Preferences  & GeneralPrefs, NotePrefs, FullScreenPrefs, Weblink:Prefs, AcroSearch:Preferences(Windows)
                                or, AcroSearch:Prefs(Mac), Cpt:Capture \\
Edit                          & Undo, Cut, Copy, Paste, Clear, SelectAll, Ole:CopyFile, TouchUp:TextAttributes,
                                TouchUp:FitTextToSelection, TouchUp:ShowLineMarkers, TouchUp:ShowCaptureSuspects,
                                TouchUp:FindSuspect, \\
                              & Properties \\
Edit$\rightarrow$Fields       & AcroForm:Duplicate, AcroForm:TabOrder \\
Document                      & Cpt:CapturePages, AcroForm:Actions, CropPages, RotatePages, InsertPages, ExtractPages,
                                ReplacePages, DeletePages, NewBookmark, SetBookmarkDest, CreateAllThumbs,
                                DeleteAllThumbs \\
View                          & ActualSize, FitVisible, FitWidth, FitPage, ZoomTo, FullScreen, FirstPage, PrevPage,
                                NextPage, LastPage, GoToPage, GoBack, GoForward, SinglePage, OneColumn, TwoColumns,
                                ArticleThreads, PageOnly, ShowBookmarks, ShowThumbs \\
Tools                         & Hand, ZoomIn, ZoomOut, SelectText, SelectGraphics, Note, Link, Thread, AcroForm:Tool,
                                Acro\_Movie:MoviePlayer, TouchUp:TextTool, Find, FindAgain, FindNextNote,
                                CreateNotesFile \\
Tools$\rightarrow$Search      & AcroSrch:Query, AcroSrch:Indexes, AcroSrch:Results, AcroSrch:Assist, AcroSrch:PrevDoc,
                                AcroSrch:PrevHit, AcroSrch:NextHit, AcroSrch:NextDoc \\
Window                        & ShowHideToolBar, ShowHideMenuBar, ShowHideClipboard, Cascade, TileHorizontal,
                                TileVertical, CloseAll \\
Help                          & HelpUserGuide, HelpTutorial, HelpExchange, HelpScan, HelpCapture, HelpPDFWriter,
                                HelpDistiller, HelpSearch, HelpCatalog, HelpReader, Weblink:Home \\
Help(Windows)                 & About
\end{longtable}

\section{PDF and HTML forms}
You must put your fields inside a \texttt{Form} environment (only one per file).

There are six macros to prepare fields:

\begin{cmdsyntax}
\ci{TextField}\verb|[|\emph{parameters}\verb|]{|\emph{label}\verb|}|
\end{cmdsyntax}

\begin{cmdsyntax}
\ci{CheckBox}\verb|[|\emph{parameters}\verb|]{|\emph{label}\verb|}|
\end{cmdsyntax}

\begin{cmdsyntax}
\ci{ChoiceMenu}\verb|[|\emph{parameters}\verb|]{|\emph{label}\verb|}{|\emph{choices}\verb|}|
\end{cmdsyntax}

\begin{cmdsyntax}
\ci{PushButton}\verb|[|\emph{parameters}\verb|]{|\emph{label}\verb|}|
\end{cmdsyntax}

\begin{cmdsyntax}
\ci{Submit}\verb|[|\emph{parameters}\verb|]{|\emph{label}\verb|}|
\end{cmdsyntax}

\begin{cmdsyntax}
\ci{Reset}\verb|[|\emph{parameters}\verb|]{|\emph{label}\verb|}|
\end{cmdsyntax}

The way forms and their labels are laid out is determined by:
\begin{cmdsyntax}
\ci{LayoutTextField}\verb|{|\emph{label}\verb|}{|\emph{field}\verb|}|
\end{cmdsyntax}

\begin{cmdsyntax}
\ci{LayoutChoiceField}\verb|{|\emph{label}\verb|}{|\emph{field}\verb|}|
\end{cmdsyntax}

\begin{cmdsyntax}
\ci{LayoutCheckField}\verb|{|\emph{label}\verb|}{|\emph{field}\verb|}|
\end{cmdsyntax}

These macros default to \#1 \#2

What is actually shown in as the field is determined by:
\begin{cmdsyntax}
\ci{MakeRadioField}\verb|{|\emph{width}\verb|}{|\emph{height}\verb|}|
\end{cmdsyntax}

\begin{cmdsyntax}
\ci{MakeCheckField}\verb|{|\emph{width}\verb|}{|\emph{height}\verb|}|
\end{cmdsyntax}
\begin{cmdsyntax}
\ci{MakeTextField}\verb|{|\emph{width}\verb|}{|\emph{height}\verb|}|
\end{cmdsyntax}
\begin{cmdsyntax}
\ci{MakeChoiceField}\verb|{|\emph{width}\verb|}{|\emph{height}\verb|}|
\end{cmdsyntax}

\begin{cmdsyntax}
\ci{MakeButtonField}\verb|{|\emph{text}\verb|}|
\end{cmdsyntax}

These macros default to \verb|\vbox to #2{\hbox to #1{\hfill}\vfill}|, except the
last, which defaults to \#1; it is used for buttons, and the special \ci{Submit} and \ci{Reset}
macros.

You may also want to redefine the following macros:
\begin{verbatim}
\def\DefaultHeightofSubmit{12pt}
\def\DefaultWidthofSubmit{2cm}
\def\DefaultHeightofReset{12pt}
\def\DefaultWidthofReset{2cm}
\def\DefaultHeightofCheckBox{0.8\baselineskip}
\def\DefaultWidthofCheckBox{0.8\baselineskip}
\def\DefaultHeightofChoiceMenu{0.8\baselineskip}
\def\DefaultWidthofChoiceMenu{0.8\baselineskip}
\def\DefaultHeightofText{\baselineskip}
\def\DefaultHeightofTextMultiline{4\baselineskip}
\def\DefaultWidthofText{3cm}
\end{verbatim}

\subsection{Forms environment parameters}

\smallskip\noindent\begin{longtable}{>{\ttfamily}l>{\itshape}lp{9cm}}
action   & URL  & The URL that will receive the form data if a \textsf{Submit} button is included in the form \\
encoding & name & The encoding for the string set to the URL; FDF-encoding is usual, and \texttt{html} is the only
                  valid value \\
method   & name & Used only when generating HTML; values can be \texttt{post} or \texttt{get} \\
\end{longtable}

\subsection{Forms optional parameters}
Note that all colors must be expressed as RGB triples, in the range 0..1 (i.e.\ \texttt{color=0 0
0.5})

\smallskip\noindent\begin{longtable}{>{\ttfamily}ll>{\itshape}ll}
accesskey       & key     &       & (as per HTML) \\
align           & number  & 0     & alignment within text field; 0 is left-aligned,\\*
                &         &       & 1 is centered, 2 is right-aligned. \\
altname         & name    &       & alternative name,\\*
                &         &       & the name shown in the user interface\\
backgroundcolor &         &       & color of box \\
bordercolor     &         &       & color of border \\
bordersep       &         &       & box border gap \\
borderwidth     &         & 1     & width of box border, the value is a dimension\\
                &         &       & or a number with default unit bp\\
calculate       &         &       & JavaScript code to calculate the value of the field \\
charsize        & dimen   &       & font size of field text \\
checkboxsymbol  & char    & 4 (\ding{\number`4}) & symbol used for check boxes (ZapfDingbats), \\
&&& the value is a character or \cs{ding}\verb|{|\texttt{\itshape number}\verb|}|, \\
&&& see package \xpackage{pifont} from bundle \xpackage{psnfss} \\
checked         & boolean & false & whether option selected by default \\
color           &         &       & color of text in box \\
combo           & boolean & false & choice list is `combo' style \\
default         &         &       & default value \\
disabled        & boolean & false & field disabled \\
format          &         &       & JavaScript code to format the field \\
height          & dimen   &       & height of field box \\
hidden          & boolean & false & field hidden \\
keystroke       &         &       & JavaScript code to control the keystrokes on entry \\
mappingname     & name    &       & the mapping name to be used when exporting\\*
                &         &       & the field data\\
maxlen          & number  & 0     & number of characters allowed in text field \\
menulength      & number  & 4     & number of elements shown in list \\
multiline       & boolean & false & whether text box is multiline \\
name            & name    &       & name of field (defaults to label) \\
onblur          &         &       & JavaScript code \\
onchange        &         &       & JavaScript code \\
onclick         &         &       & JavaScript code \\
ondblclick      &         &       & JavaScript code \\
onfocus         &         &       & JavaScript code \\
onkeydown       &         &       & JavaScript code \\
onkeypress      &         &       & JavaScript code \\
onkeyup         &         &       & JavaScript code \\
onmousedown     &         &       & JavaScript code \\
onmousemove     &         &       & JavaScript code \\
onmouseout      &         &       & JavaScript code \\
onmouseover     &         &       & JavaScript code \\
onmouseup       &         &       & JavaScript code \\
onselect        &         &       & JavaScript code \\
password        & boolean & false & text field is `password' style \\
popdown         & boolean & false & choice list is `popdown' style \\
radio           & boolean & false & choice list is `radio' style \\
radiosymbol     & char    & H (\ding{\number`H}) & symbol used for radio fields (ZapfDingbats), \\
&&& the value is a character or \cs{ding}\verb|{|\texttt{\itshape number}\verb|}|, \\
&&& see package \xpackage{pifont} from bundle \xpackage{psnfss} \\
readonly        & boolean & false & field is readonly \\
rotation        & number  & 0     & rotation of the widget annotation \\*
                &         &       & (degree, counterclockwise, multiple of 90)\\
tabkey          &         &       & (as per HTML) \\
validate        &         &       & JavaScript code to validate the entry \\
value           &         &       & initial value \\
width           & dimen   &       & width of field box
\end{longtable}

\section{Defining a new driver}
A hyperref driver has to provide definitions for eight macros:

\smallskip
\noindent 1. \verb|\hyper@anchor|

\noindent 2. \verb|\hyper@link|

\noindent 3. \verb|\hyper@linkfile|

\noindent 4. \verb|\hyper@linkurl|

\noindent 5. \verb|\hyper@anchorstart|

\noindent 6. \verb|\hyper@anchorend|

\noindent 7. \verb|\hyper@linkstart|

\noindent 8. \verb|\hyper@linkend|
\smallskip

The draft option defines the macros as follows
\qquad\begin{verbatim}
\let\hyper@@anchor\@gobble
\gdef\hyper@link##1##2##3{##3}%
\def\hyper@linkurl##1##2{##1}%
\def\hyper@linkfile##1##2##3{##1}%
\let\hyper@anchorstart\@gobble
\let\hyper@anchorend\@empty
\let\hyper@linkstart\@gobbletwo
\let\hyper@linkend\@empty
\end{verbatim}

\section{Special support for other packages}

Package \xpackage{hyperref} aims to cooperate with other packages, but there are
several possible sources for conflict, such as

\begin{itemize}

\item Packages that manipulate the bibliographic mechanism. Peter
William's \xpackage{harvard} package is supported. However, the
recommended package is Patrick Daly's \xpackage{natbib} package that has
specific \xpackage{hyperref} hooks to allow reliable interaction. This
package covers a very wide variety of layouts and citation styles, all
of which work with \xpackage{hyperref}.

\item Packages that typeset the contents of the \ci{label} and \ci{ref}
macros, such as \xpackage{showkeys}. Since the \xpackage{hyperref} package
redefines these commands, you must set \texttt{implicit=false} for these
packages to work.

\item Packages that do anything serious with the index.
\end{itemize}

The \xpackage{hyperref} package is distributed with variants on two useful
packages designed to work especially well with it. These are \xpackage{xr}
and \xpackage{minitoc}, which support crossdocument links using \hologo{LaTeX}'s
normal \cs{label}/\cs{ref} mechanisms and per-chapter tables of contents,
respectively.

\section{History and acknowledgments}

The original authors of \textsf{hyperbasics.tex} and
\textsf{hypertex.sty}, from which this package descends, are Tanmoy
Bhattacharya and Thorsten Ohl. Package \xpackage{hyperref}
started as a simple port of their work to \hologo{LaTeXe} standards, but
eventually I rewrote nearly everything, because I didn't understand a
lot of the original, and was only interested in getting it to work with
\hologo{LaTeX}. I would like to thank Arthur Smith, Tanmoy Bhattacharya, Mark
Doyle, Paul Ginsparg, David Carlisle, T.\ V.\ Raman and Leslie Lamport
for comments, requests, thoughts and code to get the package into its
first useable state. Various other people are mentioned at the point in
the source where I had to change the code in later versions because of
problems they found.

Tanmoy found a great many of the bugs, and (even better) often provided
fixes, which has made the package more robust. The days spent on
Rev\TeX\ are entirely due to him! The investigations of Bill Moss
into the later versions including
native PDF support uncovered a good many bugs, and his testing is
appreciated. Hans Hagen provided a lot of
insight into PDF.

Berthold Horn provided help, encouragement and sponsorship for the
\textsf{dvipsone} and \textsf{dviwindo} drivers. Sergey Lesenko provided
the changes needed for \textsf{dvipdf}, and \Hanh{} supplied all the
information needed for \textsf{pdftex}. Patrick Daly kindly updated his
\xpackage{natbib} package to allow easy integration with
\xpackage{hyperref}. Michael Mehlich's \xpackage{hyper} package (developed
in parallel with \textsf{hyperref}) showed me solutions for some
problems. Hopefully the two packages will combine one day.

The forms creation section owes a great deal to: T.\ V.\ Raman, for
encouragement, support and ideas; Thomas Merz, whose book \emph{Web
Publishing with Acrobat/PDF} provided crucial insights; D.\ P.\ Story,
whose detailed article about pdfmarks and forms solved many practical
problems; and Hans Hagen, who explained how to do it in \textsf{pdftex}.

Steve Peter recreated the manual source in July 2003 after it had been
lost.

Especial extra thanks to David Carlisle for the \xpackage{backref} module,
the ps2pdf and dviwindo support, frequent general rewrites of my bad
code, and for working on changes to the \xpackage{xr} package to suit
\xpackage{hyperref}.

\begingroup
  \makeatletter
  \let\chapter=\section
  % subsections goes into bookmarks but not toc
  \hypersetup{bookmarksopenlevel=1}
  \addtocontents{toc}{\protect\setcounter{tocdepth}{1}}
  % The \section command acts as \subsection.
  % Additionally the title is converted to lowercase except
  % for the first letter.
  \def\section{%
    \let\section\lc@subsection
    \lc@subsection
  }
  \def\lc@subsection{%
    \@ifstar{\def\mystar{*}\lc@sec}%
            {\let\mystar\@empty\lc@sec}%
  }
  \def\lc@sec#1{%
    \lc@@sec#1\@nil
  }
  \def\lc@@sec#1#2\@nil{%
    \begingroup
      \def\a{#1}%
      \lowercase{%
        \edef\x{\endgroup
          \noexpand\subsection\mystar{\a#2}%
        }%
      }%
    \x
  }
  % This file is a chapter.  It must be included in a larger document to work
% properly.

\chapter{GNU Free Documentation License}

Version 1.2, November 2002\\


 Copyright \copyright\ 2000,2001,2002  Free Software Foundation, Inc.\\
     59 Temple Place, Suite 330, Boston, MA  02111-1307  USA\\
 Everyone is permitted to copy and distribute verbatim copies
 of this license document, but changing it is not allowed.


\section*{PREAMBLE}

The purpose of this License is to make a manual, textbook, or other
functional and useful document ``free'' in the sense of freedom: to
assure everyone the effective freedom to copy and redistribute it,
with or without modifying it, either commercially or noncommercially.
Secondarily, this License preserves for the author and publisher a way
to get credit for their work, while not being considered responsible
for modifications made by others.

This License is a kind of ``copyleft'', which means that derivative
works of the document must themselves be free in the same sense.  It
complements the GNU General Public License, which is a copyleft
license designed for free software.

We have designed this License in order to use it for manuals for free
software, because free software needs free documentation: a free
program should come with manuals providing the same freedoms that the
software does.  But this License is not limited to software manuals;
it can be used for any textual work, regardless of subject matter or
whether it is published as a printed book.  We recommend this License
principally for works whose purpose is instruction or reference.


\section{APPLICABILITY AND DEFINITIONS}
\label{applicability}

This License applies to any manual or other work, in any medium, that
contains a notice placed by the copyright holder saying it can be
distributed under the terms of this License.  Such a notice grants a
world-wide, royalty-free license, unlimited in duration, to use that
work under the conditions stated herein.  The ``Document'', below,
refers to any such manual or work.  Any member of the public is a
licensee, and is addressed as ``you''.  You accept the license if you
copy, modify or distribute the work in a way requiring permission
under copyright law.

A ``Modified Version'' of the Document means any work containing the
Document or a portion of it, either copied verbatim, or with
modifications and/or translated into another language.

A ``Secondary Section'' is a named appendix or a front-matter section of
the Document that deals exclusively with the relationship of the
publishers or authors of the Document to the Document's overall subject
(or to related matters) and contains nothing that could fall directly
within that overall subject.  (Thus, if the Document is in part a
textbook of mathematics, a Secondary Section may not explain any
mathematics.)  The relationship could be a matter of historical
connection with the subject or with related matters, or of legal,
commercial, philosophical, ethical or political position regarding
them.

The ``Invariant Sections'' are certain Secondary Sections whose titles
are designated, as being those of Invariant Sections, in the notice
that says that the Document is released under this License.  If a
section does not fit the above definition of Secondary then it is not
allowed to be designated as Invariant.  The Document may contain zero
Invariant Sections.  If the Document does not identify any Invariant
Sections then there are none.

The ``Cover Texts'' are certain short passages of text that are listed,
as Front-Cover Texts or Back-Cover Texts, in the notice that says that
the Document is released under this License.  A Front-Cover Text may
be at most 5 words, and a Back-Cover Text may be at most 25 words.

A ``Transparent'' copy of the Document means a machine-readable copy,
represented in a format whose specification is available to the
general public, that is suitable for revising the document
straightforwardly with generic text editors or (for images composed of
pixels) generic paint programs or (for drawings) some widely available
drawing editor, and that is suitable for input to text formatters or
for automatic translation to a variety of formats suitable for input
to text formatters.  A copy made in an otherwise Transparent file
format whose markup, or absence of markup, has been arranged to thwart
or discourage subsequent modification by readers is not Transparent.
An image format is not Transparent if used for any substantial amount
of text.  A copy that is not ``Transparent'' is called ``Opaque''.

Examples of suitable formats for Transparent copies include plain
ASCII without markup, Texinfo input format, \LaTeX\ input format, SGML
or XML using a publicly available DTD, and standard-conforming simple
HTML, PostScript or PDF designed for human modification.  Examples of
transparent image formats include PNG, XCF and JPG.  Opaque formats
include proprietary formats that can be read and edited only by
proprietary word processors, SGML or XML for which the DTD and/or
processing tools are not generally available, and the
machine-generated HTML, PostScript or PDF produced by some word
processors for output purposes only.

The ``Title Page'' means, for a printed book, the title page itself,
plus such following pages as are needed to hold, legibly, the material
this License requires to appear in the title page.  For works in
formats which do not have any title page as such, ``Title Page'' means
the text near the most prominent appearance of the work's title,
preceding the beginning of the body of the text.

A section ``Entitled XYZ'' means a named subunit of the Document whose
title either is precisely XYZ or contains XYZ in parentheses following
text that translates XYZ in another language.  (Here XYZ stands for a
specific section name mentioned below, such as ``Acknowledgements'',
``Dedications'', ``Endorsements'', or ``History''.)  To ``Preserve the Title''
of such a section when you modify the Document means that it remains a
section ``Entitled XYZ'' according to this definition.

The Document may include Warranty Disclaimers next to the notice which
states that this License applies to the Document.  These Warranty
Disclaimers are considered to be included by reference in this
License, but only as regards disclaiming warranties: any other
implication that these Warranty Disclaimers may have is void and has
no effect on the meaning of this License.


\section{VERBATIM COPYING}
\label{verbatim}

You may copy and distribute the Document in any medium, either
commercially or noncommercially, provided that this License, the
copyright notices, and the license notice saying this License applies
to the Document are reproduced in all copies, and that you add no other
conditions whatsoever to those of this License.  You may not use
technical measures to obstruct or control the reading or further
copying of the copies you make or distribute.  However, you may accept
compensation in exchange for copies.  If you distribute a large enough
number of copies you must also follow the conditions in
section~\ref{copying}.

You may also lend copies, under the same conditions stated above, and
you may publicly display copies.


\section{COPYING IN QUANTITY}
\label{copying}

If you publish printed copies (or copies in media that commonly have
printed covers) of the Document, numbering more than 100, and the
Document's license notice requires Cover Texts, you must enclose the
copies in covers that carry, clearly and legibly, all these Cover
Texts: Front-Cover Texts on the front cover, and Back-Cover Texts on
the back cover.  Both covers must also clearly and legibly identify
you as the publisher of these copies.  The front cover must present
the full title with all words of the title equally prominent and
visible.  You may add other material on the covers in addition.
Copying with changes limited to the covers, as long as they preserve
the title of the Document and satisfy these conditions, can be treated
as verbatim copying in other respects.

If the required texts for either cover are too voluminous to fit
legibly, you should put the first ones listed (as many as fit
reasonably) on the actual cover, and continue the rest onto adjacent
pages.

If you publish or distribute Opaque copies of the Document numbering
more than 100, you must either include a machine-readable Transparent
copy along with each Opaque copy, or state in or with each Opaque copy
a computer-network location from which the general network-using
public has access to download using public-standard network protocols
a complete Transparent copy of the Document, free of added material.
If you use the latter option, you must take reasonably prudent steps,
when you begin distribution of Opaque copies in quantity, to ensure
that this Transparent copy will remain thus accessible at the stated
location until at least one year after the last time you distribute an
Opaque copy (directly or through your agents or retailers) of that
edition to the public.

It is requested, but not required, that you contact the authors of the
Document well before redistributing any large number of copies, to give
them a chance to provide you with an updated version of the Document.


\section{MODIFICATIONS}
\label{modifications}

You may copy and distribute a Modified Version of the Document under
the conditions of sections~\ref{verbatim} and \ref{copying} above,
provided that you release
the Modified Version under precisely this License, with the Modified
Version filling the role of the Document, thus licensing distribution
and modification of the Modified Version to whoever possesses a copy
of it.  In addition, you must do these things in the Modified Version:

\renewcommand{\labelenumi}{\Alph{enumi}.}
\begin{enumerate}
\item Use in the Title Page (and on the covers, if any) a title distinct
   from that of the Document, and from those of previous versions
   (which should, if there were any, be listed in the History section
   of the Document).  You may use the same title as a previous version
   if the original publisher of that version gives permission.
\item List on the Title Page, as authors, one or more persons or entities
   responsible for authorship of the modifications in the Modified
   Version, together with at least five of the principal authors of the
   Document (all of its principal authors, if it has fewer than five),
   unless they release you from this requirement.
\item State on the Title page the name of the publisher of the
   Modified Version, as the publisher.
\item Preserve all the copyright notices of the Document.
\item Add an appropriate copyright notice for your modifications
   adjacent to the other copyright notices.
\item Include, immediately after the copyright notices, a license notice
   giving the public permission to use the Modified Version under the
   terms of this License, in the form shown in the Addendum below.
\item Preserve in that license notice the full lists of Invariant Sections
   and required Cover Texts given in the Document's license notice.
\item Include an unaltered copy of this License.
\item Preserve the section Entitled ``History'', Preserve its Title, and add
   to it an item stating at least the title, year, new authors, and
   publisher of the Modified Version as given on the Title Page.  If
   there is no section Entitled ``History'' in the Document, create one
   stating the title, year, authors, and publisher of the Document as
   given on its Title Page, then add an item describing the Modified
   Version as stated in the previous sentence.
\item Preserve the network location, if any, given in the Document for
   public access to a Transparent copy of the Document, and likewise
   the network locations given in the Document for previous versions
   it was based on.  These may be placed in the ``History'' section.
   You may omit a network location for a work that was published at
   least four years before the Document itself, or if the original
   publisher of the version it refers to gives permission.
\item For any section Entitled ``Acknowledgements'' or ``Dedications'',
   Preserve the Title of the section, and preserve in the section all
   the substance and tone of each of the contributor acknowledgements
   and/or dedications given therein.
\item Preserve all the Invariant Sections of the Document,
   unaltered in their text and in their titles.  Section numbers
   or the equivalent are not considered part of the section titles.
\item Delete any section Entitled ``Endorsements''.  Such a section
   may not be included in the Modified Version.
\item Do not retitle any existing section to be Entitled ``Endorsements''
   or to conflict in title with any Invariant Section.
\item Preserve any Warranty Disclaimers.

\end{enumerate}

If the Modified Version includes new front-matter sections or
appendices that qualify as Secondary Sections and contain no material
copied from the Document, you may at your option designate some or all
of these sections as invariant.  To do this, add their titles to the
list of Invariant Sections in the Modified Version's license notice.
These titles must be distinct from any other section titles.

You may add a section Entitled ``Endorsements'', provided it contains
nothing but endorsements of your Modified Version by various
parties--for example, statements of peer review or that the text has
been approved by an organization as the authoritative definition of a
standard.

You may add a passage of up to five words as a Front-Cover Text, and a
passage of up to 25 words as a Back-Cover Text, to the end of the list
of Cover Texts in the Modified Version.  Only one passage of
Front-Cover Text and one of Back-Cover Text may be added by (or
through arrangements made by) any one entity.  If the Document already
includes a cover text for the same cover, previously added by you or
by arrangement made by the same entity you are acting on behalf of,
you may not add another; but you may replace the old one, on explicit
permission from the previous publisher that added the old one.

The author(s) and publisher(s) of the Document do not by this License
give permission to use their names for publicity for or to assert or
imply endorsement of any Modified Version.


\section{COMBINING DOCUMENTS}
\label{combining}

You may combine the Document with other documents released under this
License, under the terms defined in section~\ref{modifications}
above for modified
versions, provided that you include in the combination all of the
Invariant Sections of all of the original documents, unmodified, and
list them all as Invariant Sections of your combined work in its
license notice, and that you preserve all their Warranty Disclaimers.

The combined work need only contain one copy of this License, and
multiple identical Invariant Sections may be replaced with a single
copy.  If there are multiple Invariant Sections with the same name but
different contents, make the title of each such section unique by
adding at the end of it, in parentheses, the name of the original
author or publisher of that section if known, or else a unique number.
Make the same adjustment to the section titles in the list of
Invariant Sections in the license notice of the combined work.

In the combination, you must combine any sections Entitled ``History''
in the various original documents, forming one section Entitled
``History''; likewise combine any sections Entitled ``Acknowledgements'',
and any sections Entitled ``Dedications''.  You must delete all sections
Entitled ``Endorsements''.


\section{COLLECTIONS OF DOCUMENTS}
\label{collections}

You may make a collection consisting of the Document and other documents
released under this License, and replace the individual copies of this
License in the various documents with a single copy that is included in
the collection, provided that you follow the rules of this License for
verbatim copying of each of the documents in all other respects.

You may extract a single document from such a collection, and distribute
it individually under this License, provided you insert a copy of this
License into the extracted document, and follow this License in all
other respects regarding verbatim copying of that document.


\section{AGGREGATION WITH INDEPENDENT WORKS}
\label{aggregation}

A compilation of the Document or its derivatives with other separate
and independent documents or works, in or on a volume of a storage or
distribution medium, is called an ``aggregate'' if the copyright
resulting from the compilation is not used to limit the legal rights
of the compilation's users beyond what the individual works permit.
When the Document is included in an aggregate, this License does not
apply to the other works in the aggregate which are not themselves
derivative works of the Document.

If the Cover Text requirement of section~\ref{copying} is applicable to
these copies of the Document, then if the Document is less than one half
of the entire aggregate, the Document's Cover Texts may be placed on
covers that bracket the Document within the aggregate, or the
electronic equivalent of covers if the Document is in electronic form.
Otherwise they must appear on printed covers that bracket the whole
aggregate.


\section{TRANSLATION}
\label{translation}

Translation is considered a kind of modification, so you may
distribute translations of the Document under the terms of
section~\ref{modifications}.
Replacing Invariant Sections with translations requires special
permission from their copyright holders, but you may include
translations of some or all Invariant Sections in addition to the
original versions of these Invariant Sections.  You may include a
translation of this License, and all the license notices in the
Document, and any Warranty Disclaimers, provided that you also include
the original English version of this License and the original versions
of those notices and disclaimers.  In case of a disagreement between
the translation and the original version of this License or a notice
or disclaimer, the original version will prevail.

If a section in the Document is Entitled ``Acknowledgements'',
``Dedications'', or ``History'', the requirement
(section~\ref{modifications}) to Preserve
its Title (section~\ref{applicability}) will typically require
changing the actual title.


\section{TERMINATION}
\label{termination}

You may not copy, modify, sublicense, or distribute the Document except
as expressly provided for under this License.  Any other attempt to
copy, modify, sublicense or distribute the Document is void, and will
automatically terminate your rights under this License.  However,
parties who have received copies, or rights, from you under this
License will not have their licenses terminated so long as such
parties remain in full compliance.


\section{FUTURE REVISIONS OF THIS LICENSE}
\label{future}

The Free Software Foundation may publish new, revised versions
of the GNU Free Documentation License from time to time.  Such new
versions will be similar in spirit to the present version, but may
differ in detail to address new problems or concerns.  See
http://www.gnu.org/copyleft/.

Each version of the License is given a distinguishing version number.
If the Document specifies that a particular numbered version of this
License ``or any later version'' applies to it, you have the option of
following the terms and conditions either of that specified version or
of any later version that has been published (not as a draft) by the
Free Software Foundation.  If the Document does not specify a version
number of this License, you may choose any version ever published (not
as a draft) by the Free Software Foundation.


\section*{ADDENDUM: How to use this License for your documents}

To use this License in a document you have written, include a copy of
the License in the document and put the following copyright and
license notices just after the title page:

\begin{quote}
    Copyright \copyright\ YEAR  YOUR NAME.
    Permission is granted to copy, distribute and/or modify this document
    under the terms of the GNU Free Documentation License, Version 1.2
    or any later version published by the Free Software Foundation;
    with no Invariant Sections, no Front-Cover Texts, and no Back-Cover Texts.
    A copy of the license is included in the section entitled ``GNU
    Free Documentation License''.
\end{quote}

If you have Invariant Sections, Front-Cover Texts and Back-Cover Texts,
replace the ``with...Texts.'' line with this:

    with the Invariant Sections being LIST THEIR TITLES, with the
    Front-Cover Texts being LIST, and with the Back-Cover Texts being LIST.

If you have Invariant Sections without Cover Texts, or some other
combination of the three, merge those two alternatives to suit the
situation.

If your document contains nontrivial examples of program code, we
recommend releasing these examples in parallel under your choice of
free software license, such as the GNU General Public License,
to permit their use in free software.

\endgroup

\end{document}
