\ProvidesFile{langcode.tex}[2012/09/20 documenting langcode.sty]
\title{\pkgtitle{langcode.sty}{%
       Simple Language-Dependent Settings\\
       Based on Language Codes}} 
{ \RequirePackage{makedoc} \ProcessLineMessage{}
  \MakeJobDoc{17}%% 2011/11/23
  {\SectionLevelThreeParseInput}  }             %% 2012/09/17
\documentclass[fleqn]{article}
% \usepackage{inputtrc} \dotracinginputs
version https://git-lfs.github.com/spec/v1
oid sha256:ded97341479a4e2387b08b2fca0a903a98527805994be66f525785a0cdabece0
size 1168
 %% shared formatting settings
% \ReadPackageInfos{langcode}
% \usepackage{langcode,catchdq,ngerman} \originalTeX
\usepackage{catchdq,langcode,ngerman} \originalTeX
%  \show\endqtd
\MDkeywords{languages other than English; German, macro programming 
            (programming structures), hypertext}
\sloppy
% \listfiles
\begin{document}
\maketitle
\begin{MDabstract}
'langcode.sty' in the first instance provides a command 
$$|\uselangcode{<chars>}|$$ to adjust language-dependent settings, 
such as key words, typographical conventions, and language codes 
(\acro{\Wikiref{ISO-639-1}}).
% it is intended to be a kind of ``leight-weight" \ctanpkgref{babel}. 
% It uses \ctanpkgref{dowith} for adjustments and 
% \ctanpkgref{plainpkg} for use with both \LaTeX\ and Plain \TeX.
An author frequently writing documents in two or more languages 
can use the same commands independently of the language, 
provided they are gathered in a list macro to be 
used by the \ctanpkgref{dowith} package.
If `\<cmd>' is in the list, it is set to work like 
`\<chars><cmd>', and a macro `\langcode' will expand to 
<chars> (the respective tokens), usable in \acro{URL}s.---The 
package is ``generic," based on \ctanpkgref{plainpkg}.
The code has been used with \ctanpkgref{morehype} and 
'catchdq' (\ctanpkgref{catcodes}), but may be useful more generally.
\MDaddtoabstract{Related packages} \ctanpkgref{babel}, \ctanpkgref{polyglossia}
% \ctanpkgref{morehype}, 'catchdq' (\ctanpkgref{catcodes}), 
% \ctanpkgref{dowith} 
\end{MDabstract}
\newpage
\tableofcontents
% \newpage
% \section{Features and Usage}
\section{Installing and Calling}
The file 'langcode.sty' is provided ready, installation only requires
putting it somewhere where \TeX\ finds it
(which may need updating the filename data
 base).\urlfoot{ukfaqref}{inst-wlcf}           %% corr. 2011/02/08
The packages \ctanpkgref{dowith}, \ctanpkgref{plainpkg}, 
and 'stacklet' (\ctanpkgref{catcodes}) must be installed as well.

As to calling (loading): 'langcode' is a ``\pkg{plainpkg} package" 
in the sense of the 
\ctanpkgref{plainpkg}\,\foothttpurlref{ctan.org/pkg/plainpkg} 
documentation that you may consult for details.
So roughly, 
\begin{itemize}
  \item load it by \ |\usepackage{langcode}| \ if you can, 
  \item otherwise by \ |\RequirePackage{langcode}| \\
        (perhaps from within another ``\pkg{plainpkg} package"), 
  \item or by \ |\input langcode.sty| 
  \item or even by \ |version https://git-lfs.github.com/spec/v1
oid sha256:52991069b2648d7ef8ad3a691871b5c282d2aa98d44b3e45237ad6fc1dc88892
size 5917
|~\dots
\end{itemize}

%   \pagebreak
% \section{The Package File}
\section{Header---\pkg{plainpkg} and Legalese}
On the right hand side, that `plainpkg.tex' is loaded, 
before the package version is declared, for ``generic" function:
version https://git-lfs.github.com/spec/v1
oid sha256:5da818b237f3177d3005ed3759b7396d8e5b88a730a635516190f1c3def92095
size 3232


\end{document}

VERSION HISTORY

2012/09/17  for v0.1    very first 
2012/09/20              extended ...
