% \iffalse meta-comment
%
% This is file `umlaute.dtx'.
%
% Copyright (C) 1994-2006 Axel Sommerfeldt (caption@sommerfee.de)
% 
% --------------------------------------------------------------------------
% 
% This work may be distributed and/or modified under the
% conditions of the LaTeX Project Public License, either version 1.3
% of this license or (at your option) any later version.
% The latest version of this license is in
%   http://www.latex-project.org/lppl.txt
% and version 1.3 or later is part of all distributions of LaTeX
% version 2003/12/01 or later.
% 
% This work has the LPPL maintenance status "maintained".
% 
% This Current Maintainer of this work is Axel Sommerfeldt.
% 
% This work consists of the files umlaute.ins, umlaute.dtx
% and the derived files umlaute.sty, atari.def, pc850.def,
% isolatin.def, mac.def, and roman8.def.
%
% \fi
% \CheckSum{556}
%
% \iffalse
%<*driver>
\NeedsTeXFormat{LaTeX2e}[1994/12/01]
\documentclass{ltxdoc}
\setlength{\parindent}{0pt}
\setlength{\parskip}{\smallskipamount}
%
\ifx\pdfoutput\undefined\else
  \ifcase\pdfoutput\else
    \usepackage{mathptmx,courier}
    \usepackage[scaled=0.90]{helvet}
    \addtolength\marginparwidth{15pt}
  \fi
\fi
%
\usepackage{umlaute}
\usepackage{hyperref}
%
%<+driver>\OnlyDescription
%
\begin{document}
  \DocInput{umlaute.dtx}
\end{document}
%</driver>
% \fi
%
% \newcommand*{\purerm}[1]{{\upshape\mdseries\rmfamily #1}}
% \newcommand*{\puresf}[1]{{\upshape\mdseries\sffamily #1}}
% \newcommand*{\purett}[1]{{\upshape\mdseries\ttfamily #1}}
% \let\package\puresf\def\thispackage{\package{umlaute}}
% \let\env\purett \let\opt\purett
%
% \ProvideTextCommandDefault{\texttrademark}{\ensuremath{^\mathrm{TM}}}
%
% \changes{v1.0}{1994/10/09}{First release}
% \changes{v1.1}{1994/11/04}{New options `amigaos', `dos', `macos', and `tos'}
% \changes{v1.1}{1994/11/04}{The option `ansi' is default now}
% \changes{v1.2}{1994/11/28}{Option `cork' removed}
% \changes{v1.2}{1994/11/28}{Options `850' and `roman8' added}
% \changes{v1.2}{1994/11/28}{French characters added}
% \changes{v1.2}{1994/11/28}{This package works now without babel, too}
% \changes{v2.0}{1995/01/23}{umlaute.sty is now obsolete and just reads input\-enc.sty}
% \changes{v2.0a}{1995/02/05}{First release of definition files for `inputenc'}
% \changes{v2.1}{2006/01/29}{Made the whole package even more obsolete}
%
% \GetFileInfo{umlaute.sty}
% \title{The \thispackage\ package\thanks{This package has version number
%        \fileversion, last revised \filedate.}
%        \footnote{The name of this package is for historical reasons: The first
%        versions supported only the german characters, called `Umlaute'}}
% \author{Axel Sommerfeldt\\\href{mailto:caption@sommerfee.de}{\texttt{caption@sommerfee.de}}}
% \date{2006/01/29}
% \maketitle
%
% \begin{abstract}
% Once apon a time, there was a package called \textsf{umlaute} which offered the portability
% of \LaTeXe\ documents even with special characters (with ASCII codes $\ge 128$)
% and supported many input character encodings like
% \texttt{Atari}, \texttt{ISO 8859/1 (Latin1)}, \texttt{Apple Macintosh},
% \texttt{PC codepage 850}, and \texttt{Roman8}.
% \end{abstract}
%
% \section*{This package is obsolete!}
% This package was superseeded by the \package{inputenc} package~\cite{inputenc}
% which is included in any \LaTeXe\ system since December 1994.
% Therefore this package is no longer supported;
% so please don't use \thispackage, just use \package{inputenc} instead:
%
% \section{The user interface}
% Type
% \begin{quote}
% |\usepackage|\oarg{encoding name}|{inputenc}|
% \end{quote}
% in the preamble of your document.
% (For further information about the \package{inputenc} package consult
% the documented file |inputenc.dtx| which comes with every \LaTeXe .)
%
% Here comes a \meta{encoding name} translation table:
%
% \DeleteShortVerb{\|}
% \begin{tabular}{|l@{\hspace{2em}$\Rightarrow$\hspace{2em}}l|}
% \hline
% \textbf{umlaute package} & \textbf{inputenc package}\\\hline
% \verb|atari|    & \textit{no replacement yet}\\
% \verb|isolatin| & \verb|latin1|\\
% \verb|mac|      & \verb|applemac|\\
% \verb|pc850|    & \verb|cp850|\\
% \verb|roman8|   & \textit{no replacement yet}\\\hline
% \end{tabular}
% \MakeShortVerb{\|}
%
% Note that there are no encoding tables for the \textit{ATARI\texttrademark\ ST(E)/TT/Falcon}
% and \textit{HP\texttrademark\ Roman8} included in the \texttt{inputenc} package yet,
% so for these encodings you still need this package.
%
% \section{Notes for em\TeX\ users}
% If you are using em\TeX\, you have to make sure that you don't have
% any character translation table built into your \LaTeXe\ format
% (with option |/c|). Instead, you have to use the option |/8| to allow
% 8 bit character codes in your \LaTeX\ document, so a valid command line
% to build the \LaTeXe\ format would be
% \begin{quote} |tex386 /i /8 latex.ltx|\ . \end{quote}
% If you don't follow this, the \package{inputenc} package won't work properly!
% You won't be able to compile documents with other character encodings than yours!
%
% \section{Notes for CS-\TeX\ and Multi\TeX\ users}
% CS-\TeX\ has a build-in character translation table which converts the
% Atari TOS characters to Cork encoded characters. Therefore, the package
% \package{inputenc} doesn't work properly with CS-\TeX !
%
% There are two different things you can do about it:
% \begin{enumerate}
% \item Don't use the \package{inputenc} package!\\
%  With the character translation table inside you can use quite much special
%  characters by selecting the |T1| fontencoding (via
%  |\usepackage[T1]{fontenc}| in the preamble of your document).
%  Just before you give your documents away, just add the line
%  |\usepackage[atari]{inputenc}|, so the documents will be compiled
%  correctly on other implementations.
%
%  Of course, this is only a (quite bad) workaround!
%  Newertheless, you are not able to translate \LaTeX\ documents with foreign
%  character encodings codes at all, so this solution isn't recommended!
%
% \item Patch your CS-\TeX\ to get a \TeX\ with normal behaviour!\\
%  This is easier than you might think: just search for the hex codes
%  \begin{quote}|00 01 02 03| \ldots\ |7E 7F FC DC C2 C3| \ldots\ |BE 9E|\end{quote}
%  in the files |INITEX_L.TTP| and (|TT-|)|TEX_L.TTP|
%  and replace them with the hex codes
%  \begin{quote}|00 01 02 03| \ldots\ |7E 7F 80 81 82 83| \ldots\ |FE FF| .\end{quote}
%  The uuencoded ZIP archive |cs_patch.uue| with an already compiled patch
%  program is provided with this package.\footnote{Thanks to Markus Kohm for
%  this patch program!}
%
%  Of course, this solution is recommended.
% \end{enumerate}
%
% \StopEventually{
%   \begin{thebibliography}{9}
%   \bibitem{inputenc}
%   Alan Jeffrey \& Frank Mittelbach: \emph{An input encoding package for \LaTeXe\ (v0.99a)}, 2001/07/10
%   \end{thebibliography}
% }
%
% \DoNotIndex{\atcode,\catcode,\char,\chardef,\csname,\def,\endcsname,\else,\empty}
% \DoNotIndex{\expandafter,\fi,\fontencoding,\fontfamily,\frac,\ifx,\lccode}
% \DoNotIndex{\makeatletter,\mathrm,\protect,\providecommand,\relax,\selectfont}
% \DoNotIndex{\space,\typeout,\uccode}
% \DoNotIndex{\AtBeginDocument,\DeclareOption,\ExecuteOptions,\NeedsTeXFormat}
% \DoNotIndex{\PassOptionsToPackage,\ProcessOptions,\ProvidesFile,\ProvidesPackage}
% \DoNotIndex{\ProvideTextCommandDefault,\RequirePackage}
% \DoNotIndex{\",\',\^,\`,\=,\\,\c,\i,\ss,\v,\@}
% \DoNotIndex{\AA,\aa,\AE,\ae,\alpha,\approx,\beta,\bullet,\cdot,\circ}
% \DoNotIndex{\copyright,\dag,\ddag,\Delta,\delta,\DH,\dh,\div}
% \DoNotIndex{\equiv,\Gamma,\ge,\guillemotleft,\guillemotright,\guilsinglleft}
% \DoNotIndex{\guilsinglright,\infty,\int,\invneg,\ldots,\le,\mu,\neg,\neq}
% \DoNotIndex{\O,\o,\OE,\oe,\Omega,\omega,\P,\partial,\Phi,\Pi,\pi,\pm}
% \DoNotIndex{\quotedblbase,\quotesinglbase,\S,\Sigma,\sigma,\sqrt,\SS,\ss}
% \DoNotIndex{\tau,\textbullet,\textemdash,\textendash,\textexclamdown}
% \DoNotIndex{\textperiodcentered,\textquestiondown,\textsterling,\TH,\th}
% \DoNotIndex{\Theta,\times,\varoint}
% \DoNotIndex{\@as@@xx}
%
% \clearpage
% \setlength{\parskip}{0pt plus 1pt}
%
% \section{The \thispackage\ package}
% The current version of the \thispackage\ package just emulates
% the old options and loads the \package{inputenc} package.
%    \begin{macrocode}
%<*package>
\NeedsTeXFormat{LaTeX2e}[1994/12/01]
\ProvidesPackage{umlaute}[2006/01/29 v2.1 umlaute package (AS)]
\PackageWarning{umlaute}{%
  This package has been superseeded by the `inputenc' package}
%    \end{macrocode}
%    \begin{macrocode}
\DeclareOption{iso}{\PassOptionsToPackage{latin1}{inputenc}}
\DeclareOption{ansi}{\PassOptionsToPackage{latin1}{inputenc}}
\DeclareOption{850}{\PassOptionsToPackage{cp850}{inputenc}}
%\DeclareOption{roman8}{\PassOptionsToPackage{roman8}{inputenc}}
\DeclareOption{amigaos}{\\PassOptionsToPackage{latin1}{inputenc}}
\DeclareOption{amiga}{\PassOptionsToPackage{latin1}{inputenc}}
%\DeclareOption{atari}{\PassOptionsToPackage{atari}{inputenc}}
\DeclareOption{dos}{\PassOptionsToPackage{cp850}{inputenc}}
\DeclareOption{ibmpc}{\PassOptionsToPackage{cp850}{inputenc}}
\DeclareOption{hpux}{\ExecuteOptions{roman8}}
\DeclareOption{macos}{\PassOptionsToPackage{applemac}{inputenc}}
\DeclareOption{mac}{\PassOptionsToPackage{applemac}{inputenc}}
\DeclareOption{tos}{\PassOptionsToPackage{atari}{inputenc}}
\DeclareOption{windows}{\PassOptionsToPackage{latin1}{inputenc}}
\DeclareOption{nogerman}{} % just declared for compatibility with v1.2
\DeclareOption*{\PassOptionsToPackage{\CurrentOption}{inputenc}}
\ProcessOptions*
\RequirePackage{inputenc}
%</package>
%    \end{macrocode}
%
% \section{The input encoding file \texttt{atari.def} for ATARI\texttrademark\ ST(E)/TT/Falcon}
%
% \texttt{atari.def} is similar as \texttt{cp437.def} but:
% \begin{itemize}
%   \item Character 158 is defined as |\ss| instead of |\textpeseta|.
%   \item Characters 176---233 are defined additionally
%   \item Character 254 is defined as |\maththreesuperior| instead of |\textblacksquare|.
% \end{itemize}
%
%    \begin{macrocode}
%<*atari>
\ProvidesFile{atari.def}[2006/01/29 v2.1 Input encoding file (AS)]
\makeatletter
%    \end{macrocode}
%    \begin{macrocode}
\ProvideTextCommandDefault{\textdegree}{\ensuremath{{^\circ}}}
\ProvideTextCommandDefault{\textonehalf}{\ensuremath{\frac12}}
\ProvideTextCommandDefault{\textonequarter}{\ensuremath{\frac14}}
\ProvideTextCommandDefault{\textthreequarters}{\ensuremath{\frac34}}
\ProvideTextCommandDefault{\texttrademark}{\ensuremath{^\mathrm{TM}}}
\ProvideTextCommandDefault{\textflorin}{\textit{f}}
\ProvideTextCommandDefault{\textpeseta}{Pt}
\ProvideTextCommandDefault{\textblacksquare}
   {\vrule \@width .3em \@height .4em \@depth -.1em\relax}
\ProvideTextCommandDefault{\textcent}
   {\TextSymbolUnavailable\textcent}
\ProvideTextCommandDefault{\textyen}
   {\TextSymbolUnavailable\textyen}
\ProvideTextCommandDefault{\textcurrency}
   {\TextSymbolUnavailable\textcurrency}
\ProvideTextCommandDefault{\textbrokenbar}
   {\TextSymbolUnavailable\textbrokenbar}
%    \end{macrocode}
%    \begin{macrocode}
\providecommand{\mathonesuperior}{{^1}}
\providecommand{\mathtwosuperior}{{^2}}
\providecommand{\maththreesuperior}{{^3}}
\providecommand{\mathnsuperior}{{^n}}
%    \end{macrocode}
%    \begin{macrocode}
\DeclareInputText{128}{{\c C}}
\DeclareInputText{129}{\"u}
\DeclareInputText{130}{\@tabacckludge'e}
\DeclareInputText{131}{\^a}
\DeclareInputText{132}{\"a}
\DeclareInputText{133}{\@tabacckludge`a}
\DeclareInputText{134}{\r a}
\DeclareInputText{135}{{\c c}}
\DeclareInputText{136}{\^e}
\DeclareInputText{137}{\"e}
\DeclareInputText{138}{\@tabacckludge`e}
\DeclareInputText{139}{\"\i}
\DeclareInputText{140}{\^\i}
\DeclareInputText{141}{\@tabacckludge`\i}
\DeclareInputText{142}{\"A}
\DeclareInputText{143}{\r A}
\DeclareInputText{144}{\@tabacckludge'E}
\DeclareInputText{145}{\ae}
\DeclareInputText{146}{\AE}
\DeclareInputText{147}{\^o}
\DeclareInputText{148}{\"o}
\DeclareInputText{149}{\@tabacckludge`o}
\DeclareInputText{150}{\^u}
\DeclareInputText{151}{\@tabacckludge`u}
\DeclareInputText{152}{\"y}
\DeclareInputText{153}{\"O}
\DeclareInputText{154}{\"U}
\DeclareInputText{155}{\textcent}
\DeclareInputText{156}{\pounds}
\DeclareInputText{157}{\textyen}
\DeclareInputText{158}{\ss} % german sz
\DeclareInputText{159}{\textflorin}
\DeclareInputText{160}{\@tabacckludge'a}
\DeclareInputText{161}{\@tabacckludge'\i}
\DeclareInputText{162}{\@tabacckludge'o}
\DeclareInputText{163}{\@tabacckludge'u}
\DeclareInputText{164}{\~n}
\DeclareInputText{165}{\~N}
\DeclareInputText{166}{\textordfeminine}
\DeclareInputText{167}{\textordmasculine}
\DeclareInputText{168}{\textquestiondown}
\DeclareInputMath{170}{\lnot}
\DeclareInputText{171}{\textonehalf}
\DeclareInputText{172}{\textonequarter}
\DeclareInputText{173}{\textexclamdown}
\DeclareInputText{174}{\guillemotleft}
\DeclareInputText{175}{\guillemotright}
\DeclareInputText{176}{\~a}
\DeclareInputText{177}{\~o}
\DeclareInputText{178}{\O}
\DeclareInputText{179}{\o}
\DeclareInputText{180}{\oe}
\DeclareInputText{181}{\OE}
\DeclareInputText{182}{\@tabacckludge`A}
\DeclareInputText{183}{\~A}
\DeclareInputText{184}{\~O}
\DeclareInputText{185}{\"{}}
\DeclareInputText{186}{\'{}}
\DeclareInputText{187}{\dag}
\DeclareInputText{188}{\P}
\DeclareInputText{189}{\copyright}
\DeclareInputText{190}{\textregistered}
\DeclareInputText{191}{\texttrademark}
\DeclareInputText{221}{\S}
\DeclareInputText{222}{\^{}}
\DeclareInputMath{223}{\infty}
\DeclareInputMath{224}{\alpha}
\DeclareInputMath{225}{\beta}
\DeclareInputMath{226}{\Gamma}
\DeclareInputMath{227}{\pi}
\DeclareInputMath{228}{\Sigma}
\DeclareInputMath{229}{\sigma}
\DeclareInputMath{230}{\mu}
\DeclareInputMath{231}{\gamma}
\DeclareInputMath{232}{\Phi}
\DeclareInputMath{233}{\theta}
\DeclareInputMath{234}{\Omega}
\DeclareInputMath{235}{\delta}
\DeclareInputMath{236}{\infty}
\DeclareInputMath{237}{\phi}
\DeclareInputMath{238}{\varepsilon}
\DeclareInputMath{239}{\cap}
\DeclareInputMath{240}{\equiv}
\DeclareInputMath{241}{\pm}
\DeclareInputMath{242}{\ge}
\DeclareInputMath{243}{\le}
\DeclareInputMath{246}{\div}
\DeclareInputMath{247}{\approx}
\DeclareInputText{248}{\textdegree}
\DeclareInputText{249}{\textperiodcentered}
\DeclareInputText{250}{\textbullet}
\DeclareInputMath{251}{\surd}
\DeclareInputMath{252}{\mathnsuperior}
\DeclareInputMath{253}{\mathtwosuperior}
\DeclareInputText{254}{\maththreesuperior}
\DeclareInputText{255}{\nobreakspace}
%    \end{macrocode}
%    \begin{macrocode}
\makeatother
%</atari>
%    \end{macrocode}
%
% \section{The input encoding file \texttt{isolatin.def}}
%
%    \begin{macrocode}
%<*isolatin>
\ProvidesFile{isolatin.def}[2006/01/29 v2.1 Input encoding file (AS)]
\PackageWarning{umlaute}{%
  This input encoding has been superseeded by the `latin1' encoding}
\input latin1.def
%</isolatin>
%    \end{macrocode}
%
% \section{The input encoding file \texttt{mac.def}}
%
%    \begin{macrocode}
%<*mac>
\ProvidesFile{mac.def}[2006/01/29 v2.1 Input encoding file (AS)]
\PackageWarning{umlaute}{%
  This input encoding has been superseeded by the `applemac' encoding}
\input applemac.def
%</mac>
%    \end{macrocode}
%
% \section{The input encoding file \texttt{pc850.def}}
%
%    \begin{macrocode}
%<*pc850>
\ProvidesFile{pc850.def}[2006/01/29 v2.1 Input encoding file (AS)]
\PackageWarning{umlaute}{%
  This input encoding has been superseeded by the `cp850' encoding}
\input cp850.def
%</pc850>
%    \end{macrocode}
%
% \section{The input encoding file \texttt{roman8.def} for HP-UX\texttrademark }
%
%    \begin{macrocode}
%<*roman8>
\ProvidesFile{roman8.def}[2006/01/29 v2.1 Input encoding file (AS)]
\makeatletter
%    \end{macrocode}
%    \begin{macrocode}
\ProvideTextCommandDefault{\textdegree}{\ensuremath{{^\circ}}}
\ProvideTextCommandDefault{\textonehalf}{\ensuremath{\frac12}}
\ProvideTextCommandDefault{\textonequarter}{\ensuremath{\frac14}}
\ProvideTextCommandDefault{\textthreequarters}{\ensuremath{\frac34}}
\ProvideTextCommandDefault{\texttrademark}{\ensuremath{^\mathrm{TM}}}
\ProvideTextCommandDefault{\textflorin}{\textit{f}}
\ProvideTextCommandDefault{\textpeseta}{Pt}
\ProvideTextCommandDefault{\textblacksquare}
   {\vrule \@width .3em \@height .4em \@depth -.1em\relax}
\ProvideTextCommandDefault{\textcent}
   {\TextSymbolUnavailable\textcent}
\ProvideTextCommandDefault{\textyen}
   {\TextSymbolUnavailable\textyen}
\ProvideTextCommandDefault{\textcurrency}
   {\TextSymbolUnavailable\textcurrency}
\ProvideTextCommandDefault{\textbrokenbar}
   {\TextSymbolUnavailable\textbrokenbar}
%    \end{macrocode}
%    \begin{macrocode}
\providecommand{\mathonesuperior}{{^1}}
\providecommand{\mathtwosuperior}{{^2}}
\providecommand{\maththreesuperior}{{^3}}
\providecommand{\mathnsuperior}{{^n}}
%    \end{macrocode}
%    \begin{macrocode}
\DeclareInputText{160}{~}
\DeclareInputText{161}{\@tabacckludge`A}
\DeclareInputText{162}{\^A}
\DeclareInputText{163}{\@tabacckludge`E}
\DeclareInputText{164}{\^E}
\DeclareInputText{165}{\"E}
\DeclareInputText{166}{\^I}
\DeclareInputText{167}{\"I}
\DeclareInputText{168}{\'{}}
\DeclareInputText{169}{\`{}}
\DeclareInputText{170}{\^{}}
\DeclareInputText{171}{\"{}}
\DeclareInputText{172}{\~{}}
\DeclareInputText{173}{\@tabacckludge`U}
\DeclareInputText{174}{\^U}
\DeclareInputText{175}{\textlira}
\DeclareInputText{176}{\={}}
\DeclareInputText{177}{\@tabacckludge'Y}
\DeclareInputText{178}{\@tabacckludge'y}
\DeclareInputMath{179}{\mathdegree}
\DeclareInputText{180}{{\c C}}
\DeclareInputText{181}{{\c c}}
\DeclareInputText{182}{\~N}
\DeclareInputText{183}{\~n}
\DeclareInputText{184}{\textexclamdown}
\DeclareInputText{185}{\textquestiondown}
\DeclareInputText{186}{\textcurrency}
\DeclareInputText{187}{\textsterling}
\DeclareInputText{188}{\textyen}
\DeclareInputText{189}{\S}
\DeclareInputText{190}{\textgulden}
\DeclareInputText{191}{\textcent}
\DeclareInputText{192}{\^a}
\DeclareInputText{193}{\^e}
\DeclareInputText{194}{\^o}
\DeclareInputText{195}{\^u}
\DeclareInputText{196}{\@tabacckludge'a}
\DeclareInputText{197}{\@tabacckludge'e}
\DeclareInputText{198}{\@tabacckludge'o}
\DeclareInputText{199}{\@tabacckludge'u}
\DeclareInputText{200}{\@tabacckludge`a}
\DeclareInputText{201}{\@tabacckludge`e}
\DeclareInputText{202}{\@tabacckludge`o}
\DeclareInputText{203}{\@tabacckludge`u}
\DeclareInputText{204}{\"a}
\DeclareInputText{205}{\"e}
\DeclareInputText{206}{\"o}
\DeclareInputText{207}{\"u}
\DeclareInputText{208}{\AA}
\DeclareInputText{209}{\^\i}
\DeclareInputText{210}{\O}
\DeclareInputText{211}{\AE}
\DeclareInputText{212}{\aa}
\DeclareInputText{213}{\@tabacckludge'\i}
\DeclareInputText{214}{\o}
\DeclareInputText{215}{\ae}
\DeclareInputText{216}{\"A}
\DeclareInputText{217}{\@tabacckludge`\i}
\DeclareInputText{218}{\"O}
\DeclareInputText{219}{\"U}
\DeclareInputText{220}{\@tabacckludge'E}
\DeclareInputText{221}{\"\i}
\DeclareInputText{222}{\ss}
\DeclareInputText{223}{\^O}
\DeclareInputText{224}{\@tabacckludge'A}
\DeclareInputText{225}{\~A}
\DeclareInputText{226}{\~a}
\DeclareInputText{227}{\DH}
\DeclareInputText{228}{\dh}
\DeclareInputText{229}{\@tabacckludge'I}
\DeclareInputText{230}{\@tabacckludge`I}
\DeclareInputText{231}{\@tabacckludge'O}
\DeclareInputText{232}{\@tabacckludge`O}
\DeclareInputText{233}{\~O}
\DeclareInputText{234}{\~o}
\DeclareInputText{235}{\v{S}}
\DeclareInputText{236}{\v{s}}
\DeclareInputText{237}{\@tabacckludge'U}
\DeclareInputText{238}{\"Y}
\DeclareInputText{239}{\"y}
\DeclareInputText{240}{\TH}
\DeclareInputText{241}{\th}
\DeclareInputText{242}{\textperiodcentered}
\DeclareInputMath{243}{\mu}
\DeclareInputText{244}{\P}
\DeclareInputText{245}{\textthreequarters}
\DeclareInputText{246}{\textemdash}
\DeclareInputText{247}{\textonequarter}
\DeclareInputText{248}{\textonehalf}
\DeclareInputMath{249}{\mathordmasculine}
\DeclareInputMath{250}{\mathordfeminine}
\DeclareInputText{251}{\guillemotleft}
\DeclareInputText{252}{\textblacksquare}
\DeclareInputText{253}{\guillemotright}
\DeclareInputMath{254}{\pm}
\DeclareInputText{255}{\nobreakspace}
%    \end{macrocode}
%    \begin{macrocode}
\makeatother
%</roman8>
%    \end{macrocode}
%
% \Finale
%
\endinput
