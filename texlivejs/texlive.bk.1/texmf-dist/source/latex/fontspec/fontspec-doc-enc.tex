%%^^A%%  fontspec-doc-enc.tex -- part of FONTSPEC <wspr.io/fontspec>

\documentclass[a4paper]{l3doc}
\usepackage{fontspec-doc-style}
\showexamplesfalse
\begin{document}

\part{Commands for accents and symbols (`encodings')}
\label{part:enc}

\textbf{The functionality described in this section is experimental.}

In the pre-Unicode era, significant work was required by \LaTeX\ to ensure that
input characters in the source could be interpreted correctly depending on file encoding,
and that glyphs in the output were selected correctly depending on the font encoding.
With Unicode, we have the luxury of a single file and font encoding that is used for both input
and output.

While this may provide some illusion that we could get away simply with typing
Unicode text and receive correct output, this is not always the case.
For a start, hyphenation in particular is language-specific, so tags should be used
when switch between languages in a document.
The \pkg{babel} and \pkg{polyglossia} packages both provide features for this.

Multilingual documents will often use different fonts for different languages,
not just for style, but for the more pragmatic reason that fonts do not all contain
the same glyphs. (In fact, only test fonts such as Code2000 provide
anywhere near the full Unicode coverage.)
Indeed, certain fonts may be perfect for a certain application but miss a handful
of necessary diacritics or accented letters.
In these cases, \pkg{fontspec} can leverage the font encoding technology built
into \LaTeX2\ to provide on a per-font basis either provide fallback options or
error messages when a desired accent or symbol is not available.
However, at present
these features can only be provided for input using \LaTeX\ commands rather
than Unicode input; for example, typing |\`e| instead of |è| or |\textcopyright|
instead of |©| in the source file.

The most widely-used encoding in \LaTeXe\ was |T1| with companion `|TS1|' symbols
provided by the \pkg{textcomp} package.
These encodings provided glyphs to typeset text in a variety of western European languages.
As with most legacy \LaTeXe\ input methods, accents and symbols were input using
encoding-dependent commands such as |\`e| as described above.
As of 2017, in \LaTeXe\ on \XeTeX\ and \LuaTeX, the default encoding is |TU|,
which uses Unicode for input and output.
The |TU| encoding provides appropriate encoding-dependent definitions for input commands
to match the coverage of the |T1+TS1| encodings.
Wider coverage is not provided by default since (a)~each font will provide different glyph coverage, and
(b)~it is expected that most users will be writing with direct Unicode input.

For those users who do need finer-grained control, \pkg{fontspec} provides an
interface for a more extensible system.


\section{A new Unicode-based encoding from scratch}

Let's say you need to provide support for a document originally written with fonts
in the |OT2| encoding, which contains encoding-dependent commands for Cyrillic letters.
An example from the |OT2| encoding definition file (|ot2enc.def|) reads:
\begin{Verbatim}[numbers=left,firstnumber=57]
\DeclareTextSymbol{\CYRIE}{OT2}{5}
\DeclareTextSymbol{\CYRDJE}{OT2}{6}
\DeclareTextSymbol{\CYRTSHE}{OT2}{7}
\DeclareTextSymbol{\cyrnje}{OT2}{8}
\DeclareTextSymbol{\cyrlje}{OT2}{9}
\DeclareTextSymbol{\cyrdzhe}{OT2}{10}
\end{Verbatim}

To recreate this encoding in a form suitable for \pkg{fontspec}, create a new file
named, say, |fontrange-cyr.def| and populate it with
\begin{Verbatim}
...
\DeclareTextSymbol{\CYRIE}  {\LastDeclaredEncoding}{"0404}
\DeclareTextSymbol{\CYRDJE} {\LastDeclaredEncoding}{"0402}
\DeclareTextSymbol{\CYRTSHE}{\LastDeclaredEncoding}{"040B}
\DeclareTextSymbol{\cyrnje} {\LastDeclaredEncoding}{"045A}
\DeclareTextSymbol{\cyrlje} {\LastDeclaredEncoding}{"0459}
\DeclareTextSymbol{\cyrdzhe}{\LastDeclaredEncoding}{"045F}
...
\end{Verbatim}
The numbers |"0404|, |"0402|, \dots, are the Unicode slots (in hexadecimal)
of each glyph respectively.
The \pkg{fontspec} package provides a number of shorthands to simplify this style of input; in this case,
you could also write
\begin{Verbatim}
\EncodingSymbol{\CYRIE}{"0404}
...
\end{Verbatim}

To use this encoding in a \pkg{fontspec} font, you would first add this to your preamble:
\begin{Verbatim}
\DeclareUnicodeEncoding{unicyr}{
  \input{fontrange-cyr.def}
}
\end{Verbatim}
Then follow it up with a font loading call such as
\begin{Verbatim}
\setmainfont{...}[NFSSEncoding=unicyr]
\end{Verbatim}
The first argument |unicyr| is the name of the `encoding' to use in the
font family. (There's nothing special about the name chosen but it must be unique.)
The second argument to |\DeclareUnicodeEncoding| also allows adjustments to be made
for per-font changes.
We'll cover this use case in the next section.


\section{Adjusting a pre-existing encoding}

There are three reasons to adjust a pre-existing encoding:
to add, to remove, and to redefine some symbols, letters, and/or accents.

When adding symbols, etc., simply write
\begin{Verbatim}
\DeclareUnicodeEncoding{unicyr}{
  %%
%% This is file `tuenc.def',
%% generated with the docstrip utility.
%%
%% The original source files were:
%%
%% ltoutenc.dtx  (with options: `TU')
%% 
%% This is a generated file.
%% 
%% The source is maintained by the LaTeX Project team and bug
%% reports for it can be opened at http://latex-project.org/bugs.html
%% (but please observe conditions on bug reports sent to that address!)
%% 
%% 
%% Copyright 1993-2016
%% The LaTeX3 Project and any individual authors listed elsewhere
%% in this file.
%% 
%% This file was generated from file(s) of the LaTeX base system.
%% --------------------------------------------------------------
%% 
%% It may be distributed and/or modified under the
%% conditions of the LaTeX Project Public License, either version 1.3c
%% of this license or (at your option) any later version.
%% The latest version of this license is in
%%    http://www.latex-project.org/lppl.txt
%% and version 1.3c or later is part of all distributions of LaTeX
%% version 2005/12/01 or later.
%% 
%% This file has the LPPL maintenance status "maintained".
%% 
%% This file may only be distributed together with a copy of the LaTeX
%% base system. You may however distribute the LaTeX base system without
%% such generated files.
%% 
%% The list of all files belonging to the LaTeX base distribution is
%% given in the file `manifest.txt'. See also `legal.txt' for additional
%% information.
%% 
%% The list of derived (unpacked) files belonging to the distribution
%% and covered by LPPL is defined by the unpacking scripts (with
%% extension .ins) which are part of the distribution.
%%% From File: ltoutenc.dtx
\ProvidesFile{tuenc.def}
 [2017/02/22 v2.0g
         Standard LaTeX file]
\providecommand\UnicodeEncodingName{TU}
\begingroup\expandafter\expandafter\expandafter\endgroup
\expandafter\ifx\csname XeTeXrevision\endcsname\relax
  \begingroup\expandafter\expandafter\expandafter\endgroup
  \expandafter\ifx\csname directlua\endcsname\relax
    \PackageWarningNoLine{fontenc}
      {\UnicodeEncodingName\space
       encoding is only available with XeTeX and LuaTeX.\MessageBreak
       Defaulting to T1 encoding}
      \def\encodingdefault{T1}
    \expandafter\expandafter\expandafter\endinput
  \else
    \def\UnicodeFontTeXLigatures{+tlig;}
    \def\reserved@a#1{%
      \def\@remove@tlig##1{\@remove@tlig@##1\@nil#1\@nil\relax}
      \def\@remove@tlig@##1#1{\@remove@tlig@@##1}}
    \edef\reserved@b{\detokenize{+tlig;}}
    \expandafter\reserved@a\expandafter{\reserved@b}
    \def\@remove@tlig@@#1\@nil#2\relax{#1}
    \def\remove@tlig#1{%
      \begingroup
      \font\remove@tlig
      \expandafter\@remove@tlig\expandafter{\fontname\font}%
      \remove@tlig
      \char#1\relax
      \endgroup
    }
  \fi
\else
  \def\UnicodeFontTeXLigatures{mapping=tex-text;}
  \def\remove@tlig#1{\XeTeXglyph\numexpr\XeTeXcharglyph#1\relax}
\fi
\def\UnicodeFontFile#1#2{"[#1]:#2"}
\def\UnicodeFontName#1#2{"#1:#2"}
\DeclareFontEncoding\UnicodeEncodingName{}{}
\def\add@unicode@accent#1#2{%
  \if\relax\detokenize{#2}\relax^^a0\else#2\fi
  \char#1\relax}
\def\DeclareUnicodeAccent#1#2#3{%
  \DeclareTextCommand{#1}{#2}{\add@unicode@accent{#3}}%
}
\DeclareTextCommand\textquotesingle \UnicodeEncodingName{%
                                                \remove@tlig{"0027}}
\DeclareTextCommand\textasciigrave  \UnicodeEncodingName{%
                                                \remove@tlig{"0060}}
\DeclareTextCommand\textquotedbl    \UnicodeEncodingName{%
                                                \remove@tlig{"0022}}
\DeclareTextSymbol{\textdollar}          \UnicodeEncodingName{"0024}
\DeclareTextSymbol{\textless}            \UnicodeEncodingName{"003C}
\DeclareTextSymbol{\textgreater}         \UnicodeEncodingName{"003E}
\DeclareTextSymbol{\textbackslash}       \UnicodeEncodingName{"005C}
\DeclareTextSymbol{\textasciicircum}     \UnicodeEncodingName{"005E}
\DeclareTextSymbol{\textunderscore}      \UnicodeEncodingName{"005F}
\DeclareTextSymbol{\textbraceleft}       \UnicodeEncodingName{"007B}
\DeclareTextSymbol{\textbar}             \UnicodeEncodingName{"007C}
\DeclareTextSymbol{\textbraceright}      \UnicodeEncodingName{"007D}
\DeclareTextSymbol{\textasciitilde}      \UnicodeEncodingName{"007E}
\DeclareTextSymbol{\textexclamdown}      \UnicodeEncodingName{"00A1}
\DeclareTextSymbol{\textcent}            \UnicodeEncodingName{"00A2}
\DeclareTextSymbol{\textsterling}        \UnicodeEncodingName{"00A3}
\DeclareTextSymbol{\textcurrency}        \UnicodeEncodingName{"00A4}
\DeclareTextSymbol{\textyen}             \UnicodeEncodingName{"00A5}
\DeclareTextSymbol{\textbrokenbar}       \UnicodeEncodingName{"00A6}
\DeclareTextSymbol{\textsection}         \UnicodeEncodingName{"00A7}
\DeclareTextSymbol{\textasciidieresis}   \UnicodeEncodingName{"00A8}
\DeclareTextSymbol{\textcopyright}       \UnicodeEncodingName{"00A9}
\DeclareTextSymbol{\textordfeminine}     \UnicodeEncodingName{"00AA}
\DeclareTextSymbol{\guillemotleft}       \UnicodeEncodingName{"00AB}
\DeclareTextSymbol{\textlnot}            \UnicodeEncodingName{"00AC}
\DeclareTextSymbol{\textregistered}      \UnicodeEncodingName{"00AE}
\DeclareTextSymbol{\textasciimacron}     \UnicodeEncodingName{"00AF}
\DeclareTextSymbol{\textdegree}          \UnicodeEncodingName{"00B0}
\DeclareTextSymbol{\textpm}              \UnicodeEncodingName{"00B1}
\DeclareTextSymbol{\texttwosuperior}     \UnicodeEncodingName{"00B2}
\DeclareTextSymbol{\textthreesuperior}   \UnicodeEncodingName{"00B3}
\DeclareTextSymbol{\textasciiacute}      \UnicodeEncodingName{"00B4}
\DeclareTextSymbol{\textmu}              \UnicodeEncodingName{"00B5}
\DeclareTextSymbol{\textparagraph}       \UnicodeEncodingName{"00B6}
\DeclareTextSymbol{\textperiodcentered}  \UnicodeEncodingName{"00B7}
\DeclareTextSymbol{\textonesuperior}     \UnicodeEncodingName{"00B9}
\DeclareTextSymbol{\textordmasculine}    \UnicodeEncodingName{"00BA}
\DeclareTextSymbol{\guillemotright}      \UnicodeEncodingName{"00BB}
\DeclareTextSymbol{\textonequarter}      \UnicodeEncodingName{"00BC}
\DeclareTextSymbol{\textonehalf}         \UnicodeEncodingName{"00BD}
\DeclareTextSymbol{\textthreequarters}   \UnicodeEncodingName{"00BE}
\DeclareTextSymbol{\textquestiondown}    \UnicodeEncodingName{"00BF}
\DeclareTextSymbol{\AE}                  \UnicodeEncodingName{"00C6}
\DeclareTextSymbol{\DH}                  \UnicodeEncodingName{"00D0}
\DeclareTextSymbol{\texttimes}           \UnicodeEncodingName{"00D7}
\DeclareTextSymbol{\O}                   \UnicodeEncodingName{"00D8}
\DeclareTextSymbol{\TH}                  \UnicodeEncodingName{"00DE}
\DeclareTextSymbol{\ss}                  \UnicodeEncodingName{"00DF}
\DeclareTextSymbol{\ae}                  \UnicodeEncodingName{"00E6}
\DeclareTextSymbol{\dh}                  \UnicodeEncodingName{"00F0}
\DeclareTextSymbol{\textdiv}             \UnicodeEncodingName{"00F7}
\DeclareTextSymbol{\o}                   \UnicodeEncodingName{"00F8}
\DeclareTextSymbol{\th}                  \UnicodeEncodingName{"00FE}
\DeclareTextSymbol{\DJ}                  \UnicodeEncodingName{"0110}
\DeclareTextSymbol{\dj}                  \UnicodeEncodingName{"0111}
\DeclareTextSymbol{\i}                   \UnicodeEncodingName{"0131}
\DeclareTextSymbol{\IJ}                  \UnicodeEncodingName{"0132}
\DeclareTextSymbol{\ij}                  \UnicodeEncodingName{"0133}
\DeclareTextSymbol{\L}                   \UnicodeEncodingName{"0141}
\DeclareTextSymbol{\l}                   \UnicodeEncodingName{"0142}
\DeclareTextSymbol{\NG}                  \UnicodeEncodingName{"014A}
\DeclareTextSymbol{\ng}                  \UnicodeEncodingName{"014B}
\DeclareTextSymbol{\OE}                  \UnicodeEncodingName{"0152}
\DeclareTextSymbol{\oe}                  \UnicodeEncodingName{"0153}
\DeclareTextSymbol{\textflorin}          \UnicodeEncodingName{"0192}
\DeclareTextComposite{\=}             \UnicodeEncodingName{Y}{"0232}
\DeclareTextComposite{\=}             \UnicodeEncodingName{y}{"0232}
\DeclareTextSymbol{\j}                   \UnicodeEncodingName{"0237}
\DeclareTextSymbol{\textasciicaron}      \UnicodeEncodingName{"02C7}
\DeclareTextSymbol{\textasciibreve}      \UnicodeEncodingName{"02D8}
\DeclareTextSymbol{\textacutedbl}        \UnicodeEncodingName{"02DD}
\DeclareTextSymbol{\textgravedbl}        \UnicodeEncodingName{"02F5}
\DeclareTextSymbol{\texttildelow}        \UnicodeEncodingName{"02F7}
\DeclareTextSymbol{\textbaht}            \UnicodeEncodingName{"0E3F}
\DeclareTextComposite{\=}             \UnicodeEncodingName{G}{"1E20}
\DeclareTextComposite{\=}             \UnicodeEncodingName{g}{"1E21}
\DeclareTextSymbol{\SS}                  \UnicodeEncodingName{"1E9E}
\DeclareTextSymbol{\textcompwordmark}    \UnicodeEncodingName{"200C}
\DeclareTextSymbol{\textendash}          \UnicodeEncodingName{"2013}
\DeclareTextSymbol{\textemdash}          \UnicodeEncodingName{"2014}
\DeclareTextSymbol{\textbardbl}          \UnicodeEncodingName{"2016}
\DeclareTextSymbol{\textquoteleft}       \UnicodeEncodingName{"2018}
\DeclareTextSymbol{\textquoteright}      \UnicodeEncodingName{"2019}
\DeclareTextSymbol{\quotesinglbase}      \UnicodeEncodingName{"201A}
\DeclareTextSymbol{\textquotedblleft}    \UnicodeEncodingName{"201C}
\DeclareTextSymbol{\textquotedblright}   \UnicodeEncodingName{"201D}
\DeclareTextSymbol{\quotedblbase}        \UnicodeEncodingName{"201E}
\DeclareTextSymbol{\textdagger}          \UnicodeEncodingName{"2020}
\DeclareTextSymbol{\textdaggerdbl}       \UnicodeEncodingName{"2021}
\DeclareTextSymbol{\textbullet}          \UnicodeEncodingName{"2022}
\DeclareTextSymbol{\textellipsis}        \UnicodeEncodingName{"2026}
\DeclareTextSymbol{\textperthousand}     \UnicodeEncodingName{"2030}
\DeclareTextSymbol{\textpertenthousand}  \UnicodeEncodingName{"2031}
\DeclareTextSymbol{\guilsinglleft}       \UnicodeEncodingName{"2039}
\DeclareTextSymbol{\guilsinglright}      \UnicodeEncodingName{"203A}
\DeclareTextSymbol{\textreferencemark}   \UnicodeEncodingName{"203B}
\DeclareTextSymbol{\textinterrobang}     \UnicodeEncodingName{"203D}
\DeclareTextSymbol{\textfractionsolidus} \UnicodeEncodingName{"2044}
\DeclareTextSymbol{\textlquill}          \UnicodeEncodingName{"2045}
\DeclareTextSymbol{\textrquill}          \UnicodeEncodingName{"2046}
\DeclareTextSymbol{\textdiscount}        \UnicodeEncodingName{"2052}
\DeclareTextSymbol{\textcolonmonetary}   \UnicodeEncodingName{"20A1}
\DeclareTextSymbol{\textlira}            \UnicodeEncodingName{"20A4}
\DeclareTextSymbol{\textnaira}           \UnicodeEncodingName{"20A6}
\DeclareTextSymbol{\textwon}             \UnicodeEncodingName{"20A9}
\DeclareTextSymbol{\textdong}            \UnicodeEncodingName{"20AB}
\DeclareTextSymbol{\texteuro}            \UnicodeEncodingName{"20AC}
\DeclareTextSymbol{\textpeso}            \UnicodeEncodingName{"20B1}
\DeclareTextSymbol{\textcelsius}         \UnicodeEncodingName{"2103}
\DeclareTextSymbol{\textnumero}          \UnicodeEncodingName{"2116}
\DeclareTextSymbol{\textcircledP}        \UnicodeEncodingName{"2117}
\DeclareTextSymbol{\textrecipe}          \UnicodeEncodingName{"211E}
\DeclareTextSymbol{\textservicemark}     \UnicodeEncodingName{"2120}
\DeclareTextSymbol{\texttrademark}       \UnicodeEncodingName{"2122}
\DeclareTextSymbol{\textohm}             \UnicodeEncodingName{"2126}
\DeclareTextSymbol{\textmho}             \UnicodeEncodingName{"2127}
\DeclareTextSymbol{\textestimated}       \UnicodeEncodingName{"212E}
\DeclareTextSymbol{\textleftarrow}       \UnicodeEncodingName{"2190}
\DeclareTextSymbol{\textuparrow}         \UnicodeEncodingName{"2191}
\DeclareTextSymbol{\textrightarrow}      \UnicodeEncodingName{"2192}
\DeclareTextSymbol{\textdownarrow}       \UnicodeEncodingName{"2193}
\DeclareTextSymbol{\textminus}           \UnicodeEncodingName{"2212}
\DeclareTextCommand{\textasteriskcentered}\UnicodeEncodingName{%
  \iffontchar\font"2217 \char"2217 \else
    \begingroup
      \fontsize
       {\the\dimexpr1.2\dimexpr\f@size pt\relax}%
       {\f@baselineskip}%
      \selectfont
      \raisebox{-0.6ex}[\dimexpr\height-0.6ex][0pt]{*}%
    \endgroup
  \fi
}
\DeclareTextSymbol{\textsurd}            \UnicodeEncodingName{"221A}
\DeclareTextSymbol{\textlangle}          \UnicodeEncodingName{"2329}
\DeclareTextSymbol{\textrangle}          \UnicodeEncodingName{"232A}
\DeclareTextSymbol{\textblank}           \UnicodeEncodingName{"2422}
\DeclareTextSymbol{\textvisiblespace}    \UnicodeEncodingName{"2423}
\DeclareTextSymbol{\textopenbullet}      \UnicodeEncodingName{"25E6}
\DeclareTextSymbol{\textbigcircle}       \UnicodeEncodingName{"25EF}
\DeclareTextSymbol{\textmusicalnote}     \UnicodeEncodingName{"266A}
\DeclareTextSymbol{\textmarried}         \UnicodeEncodingName{"26AD}
\DeclareTextSymbol{\textdivorced}        \UnicodeEncodingName{"26AE}
\DeclareTextSymbol{\textinterrobangdown} \UnicodeEncodingName{"2E18}
\DeclareUnicodeAccent{\`}                \UnicodeEncodingName{"0300}
\DeclareUnicodeAccent{\'}                \UnicodeEncodingName{"0301}
\DeclareUnicodeAccent{\^}                \UnicodeEncodingName{"0302}
\DeclareUnicodeAccent{\~}                \UnicodeEncodingName{"0303}
\DeclareUnicodeAccent{\"}                \UnicodeEncodingName{"0308}
\DeclareUnicodeAccent{\H}                \UnicodeEncodingName{"030B}
\DeclareUnicodeAccent{\r}                \UnicodeEncodingName{"030A}
\DeclareUnicodeAccent{\v}                \UnicodeEncodingName{"030C}
\DeclareUnicodeAccent{\u}                \UnicodeEncodingName{"0306}
\DeclareUnicodeAccent{\=}                \UnicodeEncodingName{"0304}
\DeclareUnicodeAccent{\.}                \UnicodeEncodingName{"0307}
\DeclareUnicodeAccent{\b}                \UnicodeEncodingName{"0332}
\DeclareUnicodeAccent{\c}                \UnicodeEncodingName{"0327}
\DeclareUnicodeAccent{\d}                \UnicodeEncodingName{"0323}
\DeclareUnicodeAccent{\k}                \UnicodeEncodingName{"0328}
\DeclareTextComposite{\^}             \UnicodeEncodingName {}{"005E}
\DeclareTextComposite{\~}             \UnicodeEncodingName {}{"007E}
\DeclareTextComposite{\`}             \UnicodeEncodingName{A}{"00C0}
\DeclareTextComposite{\'}             \UnicodeEncodingName{A}{"00C1}
\DeclareTextComposite{\^}             \UnicodeEncodingName{A}{"00C2}
\DeclareTextComposite{\~}             \UnicodeEncodingName{A}{"00C3}
\DeclareTextComposite{\"}             \UnicodeEncodingName{A}{"00C4}
\DeclareTextComposite{\r}             \UnicodeEncodingName{A}{"00C5}
\DeclareTextComposite{\c}             \UnicodeEncodingName{C}{"00C7}
\DeclareTextComposite{\`}             \UnicodeEncodingName{E}{"00C8}
\DeclareTextComposite{\'}             \UnicodeEncodingName{E}{"00C9}
\DeclareTextComposite{\^}             \UnicodeEncodingName{E}{"00CA}
\DeclareTextComposite{\"}             \UnicodeEncodingName{E}{"00CB}
\DeclareTextComposite{\`}             \UnicodeEncodingName{I}{"00CC}
\DeclareTextComposite{\'}             \UnicodeEncodingName{I}{"00CD}
\DeclareTextComposite{\^}             \UnicodeEncodingName{I}{"00CE}
\DeclareTextComposite{\"}             \UnicodeEncodingName{I}{"00CF}
\DeclareTextComposite{\~}             \UnicodeEncodingName{N}{"00D1}
\DeclareTextComposite{\`}             \UnicodeEncodingName{O}{"00D2}
\DeclareTextComposite{\'}             \UnicodeEncodingName{O}{"00D3}
\DeclareTextComposite{\^}             \UnicodeEncodingName{O}{"00D4}
\DeclareTextComposite{\~}             \UnicodeEncodingName{O}{"00D5}
\DeclareTextComposite{\"}             \UnicodeEncodingName{O}{"00D6}
\DeclareTextComposite{\`}             \UnicodeEncodingName{U}{"00D9}
\DeclareTextComposite{\'}             \UnicodeEncodingName{U}{"00DA}
\DeclareTextComposite{\^}             \UnicodeEncodingName{U}{"00DB}
\DeclareTextComposite{\"}             \UnicodeEncodingName{U}{"00DC}
\DeclareTextComposite{\'}             \UnicodeEncodingName{Y}{"00DD}
\DeclareTextComposite{\`}             \UnicodeEncodingName{a}{"00E0}
\DeclareTextComposite{\'}             \UnicodeEncodingName{a}{"00E1}
\DeclareTextComposite{\^}             \UnicodeEncodingName{a}{"00E2}
\DeclareTextComposite{\~}             \UnicodeEncodingName{a}{"00E3}
\DeclareTextComposite{\"}             \UnicodeEncodingName{a}{"00E4}
\DeclareTextComposite{\r}             \UnicodeEncodingName{a}{"00E5}
\DeclareTextComposite{\c}             \UnicodeEncodingName{c}{"00E7}
\DeclareTextComposite{\`}             \UnicodeEncodingName{e}{"00E8}
\DeclareTextComposite{\'}             \UnicodeEncodingName{e}{"00E9}
\DeclareTextComposite{\^}             \UnicodeEncodingName{e}{"00EA}
\DeclareTextComposite{\"}             \UnicodeEncodingName{e}{"00EB}
\DeclareTextComposite{\`}             \UnicodeEncodingName\i {"00EC}
\DeclareTextComposite{\`}             \UnicodeEncodingName{i}{"00EC}
\DeclareTextComposite{\'}             \UnicodeEncodingName\i {"00ED}
\DeclareTextComposite{\'}             \UnicodeEncodingName{i}{"00ED}
\DeclareTextComposite{\^}             \UnicodeEncodingName\i {"00EE}
\DeclareTextComposite{\^}             \UnicodeEncodingName{i}{"00EE}
\DeclareTextComposite{\"}             \UnicodeEncodingName\i {"00EF}
\DeclareTextComposite{\"}             \UnicodeEncodingName{i}{"00EF}
\DeclareTextComposite{\~}             \UnicodeEncodingName{n}{"00F1}
\DeclareTextComposite{\`}             \UnicodeEncodingName{o}{"00F2}
\DeclareTextComposite{\'}             \UnicodeEncodingName{o}{"00F3}
\DeclareTextComposite{\^}             \UnicodeEncodingName{o}{"00F4}
\DeclareTextComposite{\~}             \UnicodeEncodingName{o}{"00F5}
\DeclareTextComposite{\"}             \UnicodeEncodingName{o}{"00F6}
\DeclareTextComposite{\`}             \UnicodeEncodingName{u}{"00F9}
\DeclareTextComposite{\'}             \UnicodeEncodingName{u}{"00FA}
\DeclareTextComposite{\^}             \UnicodeEncodingName{u}{"00FB}
\DeclareTextComposite{\"}             \UnicodeEncodingName{u}{"00FC}
\DeclareTextComposite{\'}             \UnicodeEncodingName{y}{"00FD}
\DeclareTextComposite{\"}             \UnicodeEncodingName{y}{"00FF}
\DeclareTextComposite{\=}             \UnicodeEncodingName{A}{"0100}
\DeclareTextComposite{\=}             \UnicodeEncodingName{a}{"0101}
\DeclareTextComposite{\u}             \UnicodeEncodingName{A}{"0102}
\DeclareTextComposite{\u}             \UnicodeEncodingName{a}{"0103}
\DeclareTextComposite{\k}             \UnicodeEncodingName{A}{"0104}
\DeclareTextComposite{\k}             \UnicodeEncodingName{a}{"0105}
\DeclareTextComposite{\'}             \UnicodeEncodingName{C}{"0106}
\DeclareTextComposite{\'}             \UnicodeEncodingName{c}{"0107}
\DeclareTextComposite{\^}             \UnicodeEncodingName{C}{"0108}
\DeclareTextComposite{\^}             \UnicodeEncodingName{c}{"0109}
\DeclareTextComposite{\.}             \UnicodeEncodingName{C}{"010A}
\DeclareTextComposite{\.}             \UnicodeEncodingName{c}{"010B}
\DeclareTextComposite{\v}             \UnicodeEncodingName{C}{"010C}
\DeclareTextComposite{\v}             \UnicodeEncodingName{c}{"010D}
\DeclareTextComposite{\v}             \UnicodeEncodingName{D}{"010E}
\DeclareTextComposite{\v}             \UnicodeEncodingName{d}{"010F}
\DeclareTextComposite{\=}             \UnicodeEncodingName{E}{"0112}
\DeclareTextComposite{\=}             \UnicodeEncodingName{e}{"0113}
\DeclareTextComposite{\u}             \UnicodeEncodingName{E}{"0114}
\DeclareTextComposite{\u}             \UnicodeEncodingName{e}{"0115}
\DeclareTextComposite{\.}             \UnicodeEncodingName{E}{"0116}
\DeclareTextComposite{\.}             \UnicodeEncodingName{e}{"0117}
\DeclareTextComposite{\k}             \UnicodeEncodingName{E}{"0118}
\DeclareTextComposite{\k}             \UnicodeEncodingName{e}{"0119}
\DeclareTextComposite{\v}             \UnicodeEncodingName{E}{"011A}
\DeclareTextComposite{\v}             \UnicodeEncodingName{e}{"011B}
\DeclareTextComposite{\^}             \UnicodeEncodingName{G}{"011C}
\DeclareTextComposite{\^}             \UnicodeEncodingName{g}{"011D}
\DeclareTextComposite{\u}             \UnicodeEncodingName{G}{"011E}
\DeclareTextComposite{\u}             \UnicodeEncodingName{g}{"011F}
\DeclareTextComposite{\.}             \UnicodeEncodingName{G}{"0120}
\DeclareTextComposite{\.}             \UnicodeEncodingName{g}{"0121}
\DeclareTextComposite{\c}             \UnicodeEncodingName{G}{"0122}
\DeclareTextComposite{\c}             \UnicodeEncodingName{g}{"0123}
\DeclareTextComposite{\^}             \UnicodeEncodingName{H}{"0124}
\DeclareTextComposite{\^}             \UnicodeEncodingName{h}{"0125}
\DeclareTextComposite{\~}             \UnicodeEncodingName{I}{"0128}
\DeclareTextComposite{\~}             \UnicodeEncodingName\i {"0129}
\DeclareTextComposite{\~}             \UnicodeEncodingName{i}{"0129}
\DeclareTextComposite{\=}             \UnicodeEncodingName{I}{"012A}
\DeclareTextComposite{\=}             \UnicodeEncodingName\i {"012B}
\DeclareTextComposite{\=}             \UnicodeEncodingName{i}{"012B}
\DeclareTextComposite{\u}             \UnicodeEncodingName{I}{"012C}
\DeclareTextComposite{\u}             \UnicodeEncodingName\i {"012D}
\DeclareTextComposite{\u}             \UnicodeEncodingName{i}{"012D}
\DeclareTextComposite{\k}             \UnicodeEncodingName{I}{"012E}
\DeclareTextComposite{\k}             \UnicodeEncodingName\i {"012F}
\DeclareTextComposite{\k}             \UnicodeEncodingName{i}{"012F}
\DeclareTextComposite{\.}             \UnicodeEncodingName{I}{"0130}
\DeclareTextComposite{\^}             \UnicodeEncodingName{J}{"0134}
\DeclareTextComposite{\^}             \UnicodeEncodingName\j {"0135}
\DeclareTextComposite{\^}             \UnicodeEncodingName{j}{"0135}
\DeclareTextComposite{\c}             \UnicodeEncodingName{K}{"0136}
\DeclareTextComposite{\c}             \UnicodeEncodingName{k}{"0137}
\DeclareTextComposite{\'}             \UnicodeEncodingName{L}{"0139}
\DeclareTextComposite{\'}             \UnicodeEncodingName{l}{"013A}
\DeclareTextComposite{\c}             \UnicodeEncodingName{L}{"013B}
\DeclareTextComposite{\c}             \UnicodeEncodingName{l}{"013C}
\DeclareTextComposite{\v}             \UnicodeEncodingName{L}{"013D}
\DeclareTextComposite{\v}             \UnicodeEncodingName{l}{"013E}
\DeclareTextComposite{\'}             \UnicodeEncodingName{N}{"0143}
\DeclareTextComposite{\'}             \UnicodeEncodingName{n}{"0144}
\DeclareTextComposite{\c}             \UnicodeEncodingName{N}{"0145}
\DeclareTextComposite{\c}             \UnicodeEncodingName{n}{"0146}
\DeclareTextComposite{\v}             \UnicodeEncodingName{N}{"0147}
\DeclareTextComposite{\v}             \UnicodeEncodingName{n}{"0148}
\DeclareTextComposite{\=}             \UnicodeEncodingName{O}{"014C}
\DeclareTextComposite{\=}             \UnicodeEncodingName{o}{"014D}
\DeclareTextComposite{\u}             \UnicodeEncodingName{O}{"014E}
\DeclareTextComposite{\u}             \UnicodeEncodingName{o}{"014F}
\DeclareTextComposite{\H}             \UnicodeEncodingName{O}{"0150}
\DeclareTextComposite{\H}             \UnicodeEncodingName{o}{"0151}
\DeclareTextComposite{\'}             \UnicodeEncodingName{R}{"0154}
\DeclareTextComposite{\'}             \UnicodeEncodingName{r}{"0155}
\DeclareTextComposite{\c}             \UnicodeEncodingName{R}{"0156}
\DeclareTextComposite{\c}             \UnicodeEncodingName{r}{"0157}
\DeclareTextComposite{\v}             \UnicodeEncodingName{R}{"0158}
\DeclareTextComposite{\v}             \UnicodeEncodingName{r}{"0159}
\DeclareTextComposite{\'}             \UnicodeEncodingName{S}{"015A}
\DeclareTextComposite{\'}             \UnicodeEncodingName{s}{"015B}
\DeclareTextComposite{\^}             \UnicodeEncodingName{S}{"015C}
\DeclareTextComposite{\^}             \UnicodeEncodingName{s}{"015D}
\DeclareTextComposite{\c}             \UnicodeEncodingName{S}{"015E}
\DeclareTextComposite{\c}             \UnicodeEncodingName{s}{"015F}
\DeclareTextComposite{\v}             \UnicodeEncodingName{S}{"0160}
\DeclareTextComposite{\v}             \UnicodeEncodingName{s}{"0161}
\DeclareTextComposite{\c}             \UnicodeEncodingName{T}{"0162}
\DeclareTextComposite{\c}             \UnicodeEncodingName{t}{"0163}
\DeclareTextComposite{\v}             \UnicodeEncodingName{T}{"0164}
\DeclareTextComposite{\v}             \UnicodeEncodingName{t}{"0165}
\DeclareTextComposite{\~}             \UnicodeEncodingName{U}{"0168}
\DeclareTextComposite{\~}             \UnicodeEncodingName{u}{"0169}
\DeclareTextComposite{\=}             \UnicodeEncodingName{U}{"016A}
\DeclareTextComposite{\=}             \UnicodeEncodingName{u}{"016B}
\DeclareTextComposite{\u}             \UnicodeEncodingName{U}{"016C}
\DeclareTextComposite{\u}             \UnicodeEncodingName{u}{"016D}
\DeclareTextComposite{\r}             \UnicodeEncodingName{U}{"016E}
\DeclareTextComposite{\r}             \UnicodeEncodingName{u}{"016F}
\DeclareTextComposite{\H}             \UnicodeEncodingName{U}{"0170}
\DeclareTextComposite{\H}             \UnicodeEncodingName{u}{"0171}
\DeclareTextComposite{\k}             \UnicodeEncodingName{U}{"0172}
\DeclareTextComposite{\k}             \UnicodeEncodingName{u}{"0173}
\DeclareTextComposite{\^}             \UnicodeEncodingName{W}{"0174}
\DeclareTextComposite{\^}             \UnicodeEncodingName{w}{"0175}
\DeclareTextComposite{\^}             \UnicodeEncodingName{Y}{"0176}
\DeclareTextComposite{\^}             \UnicodeEncodingName{y}{"0177}
\DeclareTextComposite{\"}             \UnicodeEncodingName{Y}{"0178}
\DeclareTextComposite{\'}             \UnicodeEncodingName{Z}{"0179}
\DeclareTextComposite{\'}             \UnicodeEncodingName{z}{"017A}
\DeclareTextComposite{\.}             \UnicodeEncodingName{Z}{"017B}
\DeclareTextComposite{\.}             \UnicodeEncodingName{z}{"017C}
\DeclareTextComposite{\v}             \UnicodeEncodingName{Z}{"017D}
\DeclareTextComposite{\v}             \UnicodeEncodingName{z}{"017E}
\DeclareTextComposite{\v}             \UnicodeEncodingName{A}{"01CD}
\DeclareTextComposite{\v}             \UnicodeEncodingName{a}{"01CE}
\DeclareTextComposite{\v}             \UnicodeEncodingName{I}{"01CF}
\DeclareTextComposite{\v}             \UnicodeEncodingName\i {"01D0}
\DeclareTextComposite{\v}             \UnicodeEncodingName{i}{"01D0}
\DeclareTextComposite{\v}             \UnicodeEncodingName{O}{"01D1}
\DeclareTextComposite{\v}             \UnicodeEncodingName{o}{"01D2}
\DeclareTextComposite{\v}             \UnicodeEncodingName{U}{"01D3}
\DeclareTextComposite{\v}             \UnicodeEncodingName{u}{"01D4}
\DeclareTextComposite{\=}             \UnicodeEncodingName\AE{"01E2}
\DeclareTextComposite{\=}             \UnicodeEncodingName\ae{"01E3}
\DeclareTextComposite{\v}             \UnicodeEncodingName{G}{"01E6}
\DeclareTextComposite{\v}             \UnicodeEncodingName{g}{"01E7}
\DeclareTextComposite{\v}             \UnicodeEncodingName{K}{"01E8}
\DeclareTextComposite{\v}             \UnicodeEncodingName{k}{"01E9}
\DeclareTextComposite{\k}             \UnicodeEncodingName{O}{"01EA}
\DeclareTextComposite{\k}             \UnicodeEncodingName{o}{"01EB}
\DeclareTextComposite{\v}             \UnicodeEncodingName\j {"01F0}
\DeclareTextComposite{\v}             \UnicodeEncodingName{j}{"01F0}
\DeclareTextComposite{\'}             \UnicodeEncodingName{G}{"01F4}
\DeclareTextComposite{\'}             \UnicodeEncodingName{g}{"01F5}
\DeclareTextComposite{\textcommabelow}\UnicodeEncodingName{S}{"0218}
\DeclareTextComposite{\textcommabelow}\UnicodeEncodingName{s}{"0219}
\DeclareTextComposite{\textcommabelow}\UnicodeEncodingName{T}{"021A}
\DeclareTextComposite{\textcommabelow}\UnicodeEncodingName{t}{"021B}
\DeclareTextComposite{\.}             \UnicodeEncodingName{B}{"1E02}
\DeclareTextComposite{\.}             \UnicodeEncodingName{b}{"1E03}
\endinput
%%
%% End of file `tuenc.def'.

  \input{fontrange-cyr.def}
  \EncodingSymbol{\textruble}{"20BD}
}
\end{Verbatim}
Of course if you consistently add a number of symbols to an encoding it would be
a good idea to create a new |fontrange-XX.def| file to suit your needs.

When removing symbols, use the |\UndeclareSymbol|\marg{cmd} command.
For example, if you a loading a font that you know is missing, say, the interrobang
(not that unusual a situation), you might write:
\begin{Verbatim}
\DeclareUnicodeEncoding{nobang}{
  %%
%% This is file `tuenc.def',
%% generated with the docstrip utility.
%%
%% The original source files were:
%%
%% ltoutenc.dtx  (with options: `TU')
%% 
%% This is a generated file.
%% 
%% The source is maintained by the LaTeX Project team and bug
%% reports for it can be opened at http://latex-project.org/bugs.html
%% (but please observe conditions on bug reports sent to that address!)
%% 
%% 
%% Copyright 1993-2016
%% The LaTeX3 Project and any individual authors listed elsewhere
%% in this file.
%% 
%% This file was generated from file(s) of the LaTeX base system.
%% --------------------------------------------------------------
%% 
%% It may be distributed and/or modified under the
%% conditions of the LaTeX Project Public License, either version 1.3c
%% of this license or (at your option) any later version.
%% The latest version of this license is in
%%    http://www.latex-project.org/lppl.txt
%% and version 1.3c or later is part of all distributions of LaTeX
%% version 2005/12/01 or later.
%% 
%% This file has the LPPL maintenance status "maintained".
%% 
%% This file may only be distributed together with a copy of the LaTeX
%% base system. You may however distribute the LaTeX base system without
%% such generated files.
%% 
%% The list of all files belonging to the LaTeX base distribution is
%% given in the file `manifest.txt'. See also `legal.txt' for additional
%% information.
%% 
%% The list of derived (unpacked) files belonging to the distribution
%% and covered by LPPL is defined by the unpacking scripts (with
%% extension .ins) which are part of the distribution.
%%% From File: ltoutenc.dtx
\ProvidesFile{tuenc.def}
 [2017/02/22 v2.0g
         Standard LaTeX file]
\providecommand\UnicodeEncodingName{TU}
\begingroup\expandafter\expandafter\expandafter\endgroup
\expandafter\ifx\csname XeTeXrevision\endcsname\relax
  \begingroup\expandafter\expandafter\expandafter\endgroup
  \expandafter\ifx\csname directlua\endcsname\relax
    \PackageWarningNoLine{fontenc}
      {\UnicodeEncodingName\space
       encoding is only available with XeTeX and LuaTeX.\MessageBreak
       Defaulting to T1 encoding}
      \def\encodingdefault{T1}
    \expandafter\expandafter\expandafter\endinput
  \else
    \def\UnicodeFontTeXLigatures{+tlig;}
    \def\reserved@a#1{%
      \def\@remove@tlig##1{\@remove@tlig@##1\@nil#1\@nil\relax}
      \def\@remove@tlig@##1#1{\@remove@tlig@@##1}}
    \edef\reserved@b{\detokenize{+tlig;}}
    \expandafter\reserved@a\expandafter{\reserved@b}
    \def\@remove@tlig@@#1\@nil#2\relax{#1}
    \def\remove@tlig#1{%
      \begingroup
      \font\remove@tlig
      \expandafter\@remove@tlig\expandafter{\fontname\font}%
      \remove@tlig
      \char#1\relax
      \endgroup
    }
  \fi
\else
  \def\UnicodeFontTeXLigatures{mapping=tex-text;}
  \def\remove@tlig#1{\XeTeXglyph\numexpr\XeTeXcharglyph#1\relax}
\fi
\def\UnicodeFontFile#1#2{"[#1]:#2"}
\def\UnicodeFontName#1#2{"#1:#2"}
\DeclareFontEncoding\UnicodeEncodingName{}{}
\def\add@unicode@accent#1#2{%
  \if\relax\detokenize{#2}\relax^^a0\else#2\fi
  \char#1\relax}
\def\DeclareUnicodeAccent#1#2#3{%
  \DeclareTextCommand{#1}{#2}{\add@unicode@accent{#3}}%
}
\DeclareTextCommand\textquotesingle \UnicodeEncodingName{%
                                                \remove@tlig{"0027}}
\DeclareTextCommand\textasciigrave  \UnicodeEncodingName{%
                                                \remove@tlig{"0060}}
\DeclareTextCommand\textquotedbl    \UnicodeEncodingName{%
                                                \remove@tlig{"0022}}
\DeclareTextSymbol{\textdollar}          \UnicodeEncodingName{"0024}
\DeclareTextSymbol{\textless}            \UnicodeEncodingName{"003C}
\DeclareTextSymbol{\textgreater}         \UnicodeEncodingName{"003E}
\DeclareTextSymbol{\textbackslash}       \UnicodeEncodingName{"005C}
\DeclareTextSymbol{\textasciicircum}     \UnicodeEncodingName{"005E}
\DeclareTextSymbol{\textunderscore}      \UnicodeEncodingName{"005F}
\DeclareTextSymbol{\textbraceleft}       \UnicodeEncodingName{"007B}
\DeclareTextSymbol{\textbar}             \UnicodeEncodingName{"007C}
\DeclareTextSymbol{\textbraceright}      \UnicodeEncodingName{"007D}
\DeclareTextSymbol{\textasciitilde}      \UnicodeEncodingName{"007E}
\DeclareTextSymbol{\textexclamdown}      \UnicodeEncodingName{"00A1}
\DeclareTextSymbol{\textcent}            \UnicodeEncodingName{"00A2}
\DeclareTextSymbol{\textsterling}        \UnicodeEncodingName{"00A3}
\DeclareTextSymbol{\textcurrency}        \UnicodeEncodingName{"00A4}
\DeclareTextSymbol{\textyen}             \UnicodeEncodingName{"00A5}
\DeclareTextSymbol{\textbrokenbar}       \UnicodeEncodingName{"00A6}
\DeclareTextSymbol{\textsection}         \UnicodeEncodingName{"00A7}
\DeclareTextSymbol{\textasciidieresis}   \UnicodeEncodingName{"00A8}
\DeclareTextSymbol{\textcopyright}       \UnicodeEncodingName{"00A9}
\DeclareTextSymbol{\textordfeminine}     \UnicodeEncodingName{"00AA}
\DeclareTextSymbol{\guillemotleft}       \UnicodeEncodingName{"00AB}
\DeclareTextSymbol{\textlnot}            \UnicodeEncodingName{"00AC}
\DeclareTextSymbol{\textregistered}      \UnicodeEncodingName{"00AE}
\DeclareTextSymbol{\textasciimacron}     \UnicodeEncodingName{"00AF}
\DeclareTextSymbol{\textdegree}          \UnicodeEncodingName{"00B0}
\DeclareTextSymbol{\textpm}              \UnicodeEncodingName{"00B1}
\DeclareTextSymbol{\texttwosuperior}     \UnicodeEncodingName{"00B2}
\DeclareTextSymbol{\textthreesuperior}   \UnicodeEncodingName{"00B3}
\DeclareTextSymbol{\textasciiacute}      \UnicodeEncodingName{"00B4}
\DeclareTextSymbol{\textmu}              \UnicodeEncodingName{"00B5}
\DeclareTextSymbol{\textparagraph}       \UnicodeEncodingName{"00B6}
\DeclareTextSymbol{\textperiodcentered}  \UnicodeEncodingName{"00B7}
\DeclareTextSymbol{\textonesuperior}     \UnicodeEncodingName{"00B9}
\DeclareTextSymbol{\textordmasculine}    \UnicodeEncodingName{"00BA}
\DeclareTextSymbol{\guillemotright}      \UnicodeEncodingName{"00BB}
\DeclareTextSymbol{\textonequarter}      \UnicodeEncodingName{"00BC}
\DeclareTextSymbol{\textonehalf}         \UnicodeEncodingName{"00BD}
\DeclareTextSymbol{\textthreequarters}   \UnicodeEncodingName{"00BE}
\DeclareTextSymbol{\textquestiondown}    \UnicodeEncodingName{"00BF}
\DeclareTextSymbol{\AE}                  \UnicodeEncodingName{"00C6}
\DeclareTextSymbol{\DH}                  \UnicodeEncodingName{"00D0}
\DeclareTextSymbol{\texttimes}           \UnicodeEncodingName{"00D7}
\DeclareTextSymbol{\O}                   \UnicodeEncodingName{"00D8}
\DeclareTextSymbol{\TH}                  \UnicodeEncodingName{"00DE}
\DeclareTextSymbol{\ss}                  \UnicodeEncodingName{"00DF}
\DeclareTextSymbol{\ae}                  \UnicodeEncodingName{"00E6}
\DeclareTextSymbol{\dh}                  \UnicodeEncodingName{"00F0}
\DeclareTextSymbol{\textdiv}             \UnicodeEncodingName{"00F7}
\DeclareTextSymbol{\o}                   \UnicodeEncodingName{"00F8}
\DeclareTextSymbol{\th}                  \UnicodeEncodingName{"00FE}
\DeclareTextSymbol{\DJ}                  \UnicodeEncodingName{"0110}
\DeclareTextSymbol{\dj}                  \UnicodeEncodingName{"0111}
\DeclareTextSymbol{\i}                   \UnicodeEncodingName{"0131}
\DeclareTextSymbol{\IJ}                  \UnicodeEncodingName{"0132}
\DeclareTextSymbol{\ij}                  \UnicodeEncodingName{"0133}
\DeclareTextSymbol{\L}                   \UnicodeEncodingName{"0141}
\DeclareTextSymbol{\l}                   \UnicodeEncodingName{"0142}
\DeclareTextSymbol{\NG}                  \UnicodeEncodingName{"014A}
\DeclareTextSymbol{\ng}                  \UnicodeEncodingName{"014B}
\DeclareTextSymbol{\OE}                  \UnicodeEncodingName{"0152}
\DeclareTextSymbol{\oe}                  \UnicodeEncodingName{"0153}
\DeclareTextSymbol{\textflorin}          \UnicodeEncodingName{"0192}
\DeclareTextComposite{\=}             \UnicodeEncodingName{Y}{"0232}
\DeclareTextComposite{\=}             \UnicodeEncodingName{y}{"0232}
\DeclareTextSymbol{\j}                   \UnicodeEncodingName{"0237}
\DeclareTextSymbol{\textasciicaron}      \UnicodeEncodingName{"02C7}
\DeclareTextSymbol{\textasciibreve}      \UnicodeEncodingName{"02D8}
\DeclareTextSymbol{\textacutedbl}        \UnicodeEncodingName{"02DD}
\DeclareTextSymbol{\textgravedbl}        \UnicodeEncodingName{"02F5}
\DeclareTextSymbol{\texttildelow}        \UnicodeEncodingName{"02F7}
\DeclareTextSymbol{\textbaht}            \UnicodeEncodingName{"0E3F}
\DeclareTextComposite{\=}             \UnicodeEncodingName{G}{"1E20}
\DeclareTextComposite{\=}             \UnicodeEncodingName{g}{"1E21}
\DeclareTextSymbol{\SS}                  \UnicodeEncodingName{"1E9E}
\DeclareTextSymbol{\textcompwordmark}    \UnicodeEncodingName{"200C}
\DeclareTextSymbol{\textendash}          \UnicodeEncodingName{"2013}
\DeclareTextSymbol{\textemdash}          \UnicodeEncodingName{"2014}
\DeclareTextSymbol{\textbardbl}          \UnicodeEncodingName{"2016}
\DeclareTextSymbol{\textquoteleft}       \UnicodeEncodingName{"2018}
\DeclareTextSymbol{\textquoteright}      \UnicodeEncodingName{"2019}
\DeclareTextSymbol{\quotesinglbase}      \UnicodeEncodingName{"201A}
\DeclareTextSymbol{\textquotedblleft}    \UnicodeEncodingName{"201C}
\DeclareTextSymbol{\textquotedblright}   \UnicodeEncodingName{"201D}
\DeclareTextSymbol{\quotedblbase}        \UnicodeEncodingName{"201E}
\DeclareTextSymbol{\textdagger}          \UnicodeEncodingName{"2020}
\DeclareTextSymbol{\textdaggerdbl}       \UnicodeEncodingName{"2021}
\DeclareTextSymbol{\textbullet}          \UnicodeEncodingName{"2022}
\DeclareTextSymbol{\textellipsis}        \UnicodeEncodingName{"2026}
\DeclareTextSymbol{\textperthousand}     \UnicodeEncodingName{"2030}
\DeclareTextSymbol{\textpertenthousand}  \UnicodeEncodingName{"2031}
\DeclareTextSymbol{\guilsinglleft}       \UnicodeEncodingName{"2039}
\DeclareTextSymbol{\guilsinglright}      \UnicodeEncodingName{"203A}
\DeclareTextSymbol{\textreferencemark}   \UnicodeEncodingName{"203B}
\DeclareTextSymbol{\textinterrobang}     \UnicodeEncodingName{"203D}
\DeclareTextSymbol{\textfractionsolidus} \UnicodeEncodingName{"2044}
\DeclareTextSymbol{\textlquill}          \UnicodeEncodingName{"2045}
\DeclareTextSymbol{\textrquill}          \UnicodeEncodingName{"2046}
\DeclareTextSymbol{\textdiscount}        \UnicodeEncodingName{"2052}
\DeclareTextSymbol{\textcolonmonetary}   \UnicodeEncodingName{"20A1}
\DeclareTextSymbol{\textlira}            \UnicodeEncodingName{"20A4}
\DeclareTextSymbol{\textnaira}           \UnicodeEncodingName{"20A6}
\DeclareTextSymbol{\textwon}             \UnicodeEncodingName{"20A9}
\DeclareTextSymbol{\textdong}            \UnicodeEncodingName{"20AB}
\DeclareTextSymbol{\texteuro}            \UnicodeEncodingName{"20AC}
\DeclareTextSymbol{\textpeso}            \UnicodeEncodingName{"20B1}
\DeclareTextSymbol{\textcelsius}         \UnicodeEncodingName{"2103}
\DeclareTextSymbol{\textnumero}          \UnicodeEncodingName{"2116}
\DeclareTextSymbol{\textcircledP}        \UnicodeEncodingName{"2117}
\DeclareTextSymbol{\textrecipe}          \UnicodeEncodingName{"211E}
\DeclareTextSymbol{\textservicemark}     \UnicodeEncodingName{"2120}
\DeclareTextSymbol{\texttrademark}       \UnicodeEncodingName{"2122}
\DeclareTextSymbol{\textohm}             \UnicodeEncodingName{"2126}
\DeclareTextSymbol{\textmho}             \UnicodeEncodingName{"2127}
\DeclareTextSymbol{\textestimated}       \UnicodeEncodingName{"212E}
\DeclareTextSymbol{\textleftarrow}       \UnicodeEncodingName{"2190}
\DeclareTextSymbol{\textuparrow}         \UnicodeEncodingName{"2191}
\DeclareTextSymbol{\textrightarrow}      \UnicodeEncodingName{"2192}
\DeclareTextSymbol{\textdownarrow}       \UnicodeEncodingName{"2193}
\DeclareTextSymbol{\textminus}           \UnicodeEncodingName{"2212}
\DeclareTextCommand{\textasteriskcentered}\UnicodeEncodingName{%
  \iffontchar\font"2217 \char"2217 \else
    \begingroup
      \fontsize
       {\the\dimexpr1.2\dimexpr\f@size pt\relax}%
       {\f@baselineskip}%
      \selectfont
      \raisebox{-0.6ex}[\dimexpr\height-0.6ex][0pt]{*}%
    \endgroup
  \fi
}
\DeclareTextSymbol{\textsurd}            \UnicodeEncodingName{"221A}
\DeclareTextSymbol{\textlangle}          \UnicodeEncodingName{"2329}
\DeclareTextSymbol{\textrangle}          \UnicodeEncodingName{"232A}
\DeclareTextSymbol{\textblank}           \UnicodeEncodingName{"2422}
\DeclareTextSymbol{\textvisiblespace}    \UnicodeEncodingName{"2423}
\DeclareTextSymbol{\textopenbullet}      \UnicodeEncodingName{"25E6}
\DeclareTextSymbol{\textbigcircle}       \UnicodeEncodingName{"25EF}
\DeclareTextSymbol{\textmusicalnote}     \UnicodeEncodingName{"266A}
\DeclareTextSymbol{\textmarried}         \UnicodeEncodingName{"26AD}
\DeclareTextSymbol{\textdivorced}        \UnicodeEncodingName{"26AE}
\DeclareTextSymbol{\textinterrobangdown} \UnicodeEncodingName{"2E18}
\DeclareUnicodeAccent{\`}                \UnicodeEncodingName{"0300}
\DeclareUnicodeAccent{\'}                \UnicodeEncodingName{"0301}
\DeclareUnicodeAccent{\^}                \UnicodeEncodingName{"0302}
\DeclareUnicodeAccent{\~}                \UnicodeEncodingName{"0303}
\DeclareUnicodeAccent{\"}                \UnicodeEncodingName{"0308}
\DeclareUnicodeAccent{\H}                \UnicodeEncodingName{"030B}
\DeclareUnicodeAccent{\r}                \UnicodeEncodingName{"030A}
\DeclareUnicodeAccent{\v}                \UnicodeEncodingName{"030C}
\DeclareUnicodeAccent{\u}                \UnicodeEncodingName{"0306}
\DeclareUnicodeAccent{\=}                \UnicodeEncodingName{"0304}
\DeclareUnicodeAccent{\.}                \UnicodeEncodingName{"0307}
\DeclareUnicodeAccent{\b}                \UnicodeEncodingName{"0332}
\DeclareUnicodeAccent{\c}                \UnicodeEncodingName{"0327}
\DeclareUnicodeAccent{\d}                \UnicodeEncodingName{"0323}
\DeclareUnicodeAccent{\k}                \UnicodeEncodingName{"0328}
\DeclareTextComposite{\^}             \UnicodeEncodingName {}{"005E}
\DeclareTextComposite{\~}             \UnicodeEncodingName {}{"007E}
\DeclareTextComposite{\`}             \UnicodeEncodingName{A}{"00C0}
\DeclareTextComposite{\'}             \UnicodeEncodingName{A}{"00C1}
\DeclareTextComposite{\^}             \UnicodeEncodingName{A}{"00C2}
\DeclareTextComposite{\~}             \UnicodeEncodingName{A}{"00C3}
\DeclareTextComposite{\"}             \UnicodeEncodingName{A}{"00C4}
\DeclareTextComposite{\r}             \UnicodeEncodingName{A}{"00C5}
\DeclareTextComposite{\c}             \UnicodeEncodingName{C}{"00C7}
\DeclareTextComposite{\`}             \UnicodeEncodingName{E}{"00C8}
\DeclareTextComposite{\'}             \UnicodeEncodingName{E}{"00C9}
\DeclareTextComposite{\^}             \UnicodeEncodingName{E}{"00CA}
\DeclareTextComposite{\"}             \UnicodeEncodingName{E}{"00CB}
\DeclareTextComposite{\`}             \UnicodeEncodingName{I}{"00CC}
\DeclareTextComposite{\'}             \UnicodeEncodingName{I}{"00CD}
\DeclareTextComposite{\^}             \UnicodeEncodingName{I}{"00CE}
\DeclareTextComposite{\"}             \UnicodeEncodingName{I}{"00CF}
\DeclareTextComposite{\~}             \UnicodeEncodingName{N}{"00D1}
\DeclareTextComposite{\`}             \UnicodeEncodingName{O}{"00D2}
\DeclareTextComposite{\'}             \UnicodeEncodingName{O}{"00D3}
\DeclareTextComposite{\^}             \UnicodeEncodingName{O}{"00D4}
\DeclareTextComposite{\~}             \UnicodeEncodingName{O}{"00D5}
\DeclareTextComposite{\"}             \UnicodeEncodingName{O}{"00D6}
\DeclareTextComposite{\`}             \UnicodeEncodingName{U}{"00D9}
\DeclareTextComposite{\'}             \UnicodeEncodingName{U}{"00DA}
\DeclareTextComposite{\^}             \UnicodeEncodingName{U}{"00DB}
\DeclareTextComposite{\"}             \UnicodeEncodingName{U}{"00DC}
\DeclareTextComposite{\'}             \UnicodeEncodingName{Y}{"00DD}
\DeclareTextComposite{\`}             \UnicodeEncodingName{a}{"00E0}
\DeclareTextComposite{\'}             \UnicodeEncodingName{a}{"00E1}
\DeclareTextComposite{\^}             \UnicodeEncodingName{a}{"00E2}
\DeclareTextComposite{\~}             \UnicodeEncodingName{a}{"00E3}
\DeclareTextComposite{\"}             \UnicodeEncodingName{a}{"00E4}
\DeclareTextComposite{\r}             \UnicodeEncodingName{a}{"00E5}
\DeclareTextComposite{\c}             \UnicodeEncodingName{c}{"00E7}
\DeclareTextComposite{\`}             \UnicodeEncodingName{e}{"00E8}
\DeclareTextComposite{\'}             \UnicodeEncodingName{e}{"00E9}
\DeclareTextComposite{\^}             \UnicodeEncodingName{e}{"00EA}
\DeclareTextComposite{\"}             \UnicodeEncodingName{e}{"00EB}
\DeclareTextComposite{\`}             \UnicodeEncodingName\i {"00EC}
\DeclareTextComposite{\`}             \UnicodeEncodingName{i}{"00EC}
\DeclareTextComposite{\'}             \UnicodeEncodingName\i {"00ED}
\DeclareTextComposite{\'}             \UnicodeEncodingName{i}{"00ED}
\DeclareTextComposite{\^}             \UnicodeEncodingName\i {"00EE}
\DeclareTextComposite{\^}             \UnicodeEncodingName{i}{"00EE}
\DeclareTextComposite{\"}             \UnicodeEncodingName\i {"00EF}
\DeclareTextComposite{\"}             \UnicodeEncodingName{i}{"00EF}
\DeclareTextComposite{\~}             \UnicodeEncodingName{n}{"00F1}
\DeclareTextComposite{\`}             \UnicodeEncodingName{o}{"00F2}
\DeclareTextComposite{\'}             \UnicodeEncodingName{o}{"00F3}
\DeclareTextComposite{\^}             \UnicodeEncodingName{o}{"00F4}
\DeclareTextComposite{\~}             \UnicodeEncodingName{o}{"00F5}
\DeclareTextComposite{\"}             \UnicodeEncodingName{o}{"00F6}
\DeclareTextComposite{\`}             \UnicodeEncodingName{u}{"00F9}
\DeclareTextComposite{\'}             \UnicodeEncodingName{u}{"00FA}
\DeclareTextComposite{\^}             \UnicodeEncodingName{u}{"00FB}
\DeclareTextComposite{\"}             \UnicodeEncodingName{u}{"00FC}
\DeclareTextComposite{\'}             \UnicodeEncodingName{y}{"00FD}
\DeclareTextComposite{\"}             \UnicodeEncodingName{y}{"00FF}
\DeclareTextComposite{\=}             \UnicodeEncodingName{A}{"0100}
\DeclareTextComposite{\=}             \UnicodeEncodingName{a}{"0101}
\DeclareTextComposite{\u}             \UnicodeEncodingName{A}{"0102}
\DeclareTextComposite{\u}             \UnicodeEncodingName{a}{"0103}
\DeclareTextComposite{\k}             \UnicodeEncodingName{A}{"0104}
\DeclareTextComposite{\k}             \UnicodeEncodingName{a}{"0105}
\DeclareTextComposite{\'}             \UnicodeEncodingName{C}{"0106}
\DeclareTextComposite{\'}             \UnicodeEncodingName{c}{"0107}
\DeclareTextComposite{\^}             \UnicodeEncodingName{C}{"0108}
\DeclareTextComposite{\^}             \UnicodeEncodingName{c}{"0109}
\DeclareTextComposite{\.}             \UnicodeEncodingName{C}{"010A}
\DeclareTextComposite{\.}             \UnicodeEncodingName{c}{"010B}
\DeclareTextComposite{\v}             \UnicodeEncodingName{C}{"010C}
\DeclareTextComposite{\v}             \UnicodeEncodingName{c}{"010D}
\DeclareTextComposite{\v}             \UnicodeEncodingName{D}{"010E}
\DeclareTextComposite{\v}             \UnicodeEncodingName{d}{"010F}
\DeclareTextComposite{\=}             \UnicodeEncodingName{E}{"0112}
\DeclareTextComposite{\=}             \UnicodeEncodingName{e}{"0113}
\DeclareTextComposite{\u}             \UnicodeEncodingName{E}{"0114}
\DeclareTextComposite{\u}             \UnicodeEncodingName{e}{"0115}
\DeclareTextComposite{\.}             \UnicodeEncodingName{E}{"0116}
\DeclareTextComposite{\.}             \UnicodeEncodingName{e}{"0117}
\DeclareTextComposite{\k}             \UnicodeEncodingName{E}{"0118}
\DeclareTextComposite{\k}             \UnicodeEncodingName{e}{"0119}
\DeclareTextComposite{\v}             \UnicodeEncodingName{E}{"011A}
\DeclareTextComposite{\v}             \UnicodeEncodingName{e}{"011B}
\DeclareTextComposite{\^}             \UnicodeEncodingName{G}{"011C}
\DeclareTextComposite{\^}             \UnicodeEncodingName{g}{"011D}
\DeclareTextComposite{\u}             \UnicodeEncodingName{G}{"011E}
\DeclareTextComposite{\u}             \UnicodeEncodingName{g}{"011F}
\DeclareTextComposite{\.}             \UnicodeEncodingName{G}{"0120}
\DeclareTextComposite{\.}             \UnicodeEncodingName{g}{"0121}
\DeclareTextComposite{\c}             \UnicodeEncodingName{G}{"0122}
\DeclareTextComposite{\c}             \UnicodeEncodingName{g}{"0123}
\DeclareTextComposite{\^}             \UnicodeEncodingName{H}{"0124}
\DeclareTextComposite{\^}             \UnicodeEncodingName{h}{"0125}
\DeclareTextComposite{\~}             \UnicodeEncodingName{I}{"0128}
\DeclareTextComposite{\~}             \UnicodeEncodingName\i {"0129}
\DeclareTextComposite{\~}             \UnicodeEncodingName{i}{"0129}
\DeclareTextComposite{\=}             \UnicodeEncodingName{I}{"012A}
\DeclareTextComposite{\=}             \UnicodeEncodingName\i {"012B}
\DeclareTextComposite{\=}             \UnicodeEncodingName{i}{"012B}
\DeclareTextComposite{\u}             \UnicodeEncodingName{I}{"012C}
\DeclareTextComposite{\u}             \UnicodeEncodingName\i {"012D}
\DeclareTextComposite{\u}             \UnicodeEncodingName{i}{"012D}
\DeclareTextComposite{\k}             \UnicodeEncodingName{I}{"012E}
\DeclareTextComposite{\k}             \UnicodeEncodingName\i {"012F}
\DeclareTextComposite{\k}             \UnicodeEncodingName{i}{"012F}
\DeclareTextComposite{\.}             \UnicodeEncodingName{I}{"0130}
\DeclareTextComposite{\^}             \UnicodeEncodingName{J}{"0134}
\DeclareTextComposite{\^}             \UnicodeEncodingName\j {"0135}
\DeclareTextComposite{\^}             \UnicodeEncodingName{j}{"0135}
\DeclareTextComposite{\c}             \UnicodeEncodingName{K}{"0136}
\DeclareTextComposite{\c}             \UnicodeEncodingName{k}{"0137}
\DeclareTextComposite{\'}             \UnicodeEncodingName{L}{"0139}
\DeclareTextComposite{\'}             \UnicodeEncodingName{l}{"013A}
\DeclareTextComposite{\c}             \UnicodeEncodingName{L}{"013B}
\DeclareTextComposite{\c}             \UnicodeEncodingName{l}{"013C}
\DeclareTextComposite{\v}             \UnicodeEncodingName{L}{"013D}
\DeclareTextComposite{\v}             \UnicodeEncodingName{l}{"013E}
\DeclareTextComposite{\'}             \UnicodeEncodingName{N}{"0143}
\DeclareTextComposite{\'}             \UnicodeEncodingName{n}{"0144}
\DeclareTextComposite{\c}             \UnicodeEncodingName{N}{"0145}
\DeclareTextComposite{\c}             \UnicodeEncodingName{n}{"0146}
\DeclareTextComposite{\v}             \UnicodeEncodingName{N}{"0147}
\DeclareTextComposite{\v}             \UnicodeEncodingName{n}{"0148}
\DeclareTextComposite{\=}             \UnicodeEncodingName{O}{"014C}
\DeclareTextComposite{\=}             \UnicodeEncodingName{o}{"014D}
\DeclareTextComposite{\u}             \UnicodeEncodingName{O}{"014E}
\DeclareTextComposite{\u}             \UnicodeEncodingName{o}{"014F}
\DeclareTextComposite{\H}             \UnicodeEncodingName{O}{"0150}
\DeclareTextComposite{\H}             \UnicodeEncodingName{o}{"0151}
\DeclareTextComposite{\'}             \UnicodeEncodingName{R}{"0154}
\DeclareTextComposite{\'}             \UnicodeEncodingName{r}{"0155}
\DeclareTextComposite{\c}             \UnicodeEncodingName{R}{"0156}
\DeclareTextComposite{\c}             \UnicodeEncodingName{r}{"0157}
\DeclareTextComposite{\v}             \UnicodeEncodingName{R}{"0158}
\DeclareTextComposite{\v}             \UnicodeEncodingName{r}{"0159}
\DeclareTextComposite{\'}             \UnicodeEncodingName{S}{"015A}
\DeclareTextComposite{\'}             \UnicodeEncodingName{s}{"015B}
\DeclareTextComposite{\^}             \UnicodeEncodingName{S}{"015C}
\DeclareTextComposite{\^}             \UnicodeEncodingName{s}{"015D}
\DeclareTextComposite{\c}             \UnicodeEncodingName{S}{"015E}
\DeclareTextComposite{\c}             \UnicodeEncodingName{s}{"015F}
\DeclareTextComposite{\v}             \UnicodeEncodingName{S}{"0160}
\DeclareTextComposite{\v}             \UnicodeEncodingName{s}{"0161}
\DeclareTextComposite{\c}             \UnicodeEncodingName{T}{"0162}
\DeclareTextComposite{\c}             \UnicodeEncodingName{t}{"0163}
\DeclareTextComposite{\v}             \UnicodeEncodingName{T}{"0164}
\DeclareTextComposite{\v}             \UnicodeEncodingName{t}{"0165}
\DeclareTextComposite{\~}             \UnicodeEncodingName{U}{"0168}
\DeclareTextComposite{\~}             \UnicodeEncodingName{u}{"0169}
\DeclareTextComposite{\=}             \UnicodeEncodingName{U}{"016A}
\DeclareTextComposite{\=}             \UnicodeEncodingName{u}{"016B}
\DeclareTextComposite{\u}             \UnicodeEncodingName{U}{"016C}
\DeclareTextComposite{\u}             \UnicodeEncodingName{u}{"016D}
\DeclareTextComposite{\r}             \UnicodeEncodingName{U}{"016E}
\DeclareTextComposite{\r}             \UnicodeEncodingName{u}{"016F}
\DeclareTextComposite{\H}             \UnicodeEncodingName{U}{"0170}
\DeclareTextComposite{\H}             \UnicodeEncodingName{u}{"0171}
\DeclareTextComposite{\k}             \UnicodeEncodingName{U}{"0172}
\DeclareTextComposite{\k}             \UnicodeEncodingName{u}{"0173}
\DeclareTextComposite{\^}             \UnicodeEncodingName{W}{"0174}
\DeclareTextComposite{\^}             \UnicodeEncodingName{w}{"0175}
\DeclareTextComposite{\^}             \UnicodeEncodingName{Y}{"0176}
\DeclareTextComposite{\^}             \UnicodeEncodingName{y}{"0177}
\DeclareTextComposite{\"}             \UnicodeEncodingName{Y}{"0178}
\DeclareTextComposite{\'}             \UnicodeEncodingName{Z}{"0179}
\DeclareTextComposite{\'}             \UnicodeEncodingName{z}{"017A}
\DeclareTextComposite{\.}             \UnicodeEncodingName{Z}{"017B}
\DeclareTextComposite{\.}             \UnicodeEncodingName{z}{"017C}
\DeclareTextComposite{\v}             \UnicodeEncodingName{Z}{"017D}
\DeclareTextComposite{\v}             \UnicodeEncodingName{z}{"017E}
\DeclareTextComposite{\v}             \UnicodeEncodingName{A}{"01CD}
\DeclareTextComposite{\v}             \UnicodeEncodingName{a}{"01CE}
\DeclareTextComposite{\v}             \UnicodeEncodingName{I}{"01CF}
\DeclareTextComposite{\v}             \UnicodeEncodingName\i {"01D0}
\DeclareTextComposite{\v}             \UnicodeEncodingName{i}{"01D0}
\DeclareTextComposite{\v}             \UnicodeEncodingName{O}{"01D1}
\DeclareTextComposite{\v}             \UnicodeEncodingName{o}{"01D2}
\DeclareTextComposite{\v}             \UnicodeEncodingName{U}{"01D3}
\DeclareTextComposite{\v}             \UnicodeEncodingName{u}{"01D4}
\DeclareTextComposite{\=}             \UnicodeEncodingName\AE{"01E2}
\DeclareTextComposite{\=}             \UnicodeEncodingName\ae{"01E3}
\DeclareTextComposite{\v}             \UnicodeEncodingName{G}{"01E6}
\DeclareTextComposite{\v}             \UnicodeEncodingName{g}{"01E7}
\DeclareTextComposite{\v}             \UnicodeEncodingName{K}{"01E8}
\DeclareTextComposite{\v}             \UnicodeEncodingName{k}{"01E9}
\DeclareTextComposite{\k}             \UnicodeEncodingName{O}{"01EA}
\DeclareTextComposite{\k}             \UnicodeEncodingName{o}{"01EB}
\DeclareTextComposite{\v}             \UnicodeEncodingName\j {"01F0}
\DeclareTextComposite{\v}             \UnicodeEncodingName{j}{"01F0}
\DeclareTextComposite{\'}             \UnicodeEncodingName{G}{"01F4}
\DeclareTextComposite{\'}             \UnicodeEncodingName{g}{"01F5}
\DeclareTextComposite{\textcommabelow}\UnicodeEncodingName{S}{"0218}
\DeclareTextComposite{\textcommabelow}\UnicodeEncodingName{s}{"0219}
\DeclareTextComposite{\textcommabelow}\UnicodeEncodingName{T}{"021A}
\DeclareTextComposite{\textcommabelow}\UnicodeEncodingName{t}{"021B}
\DeclareTextComposite{\.}             \UnicodeEncodingName{B}{"1E02}
\DeclareTextComposite{\.}             \UnicodeEncodingName{b}{"1E03}
\endinput
%%
%% End of file `tuenc.def'.

  \UndeclareSymbol\textinterrobang
}
\end{Verbatim}
Provided that you use the command |\textinterrobang| to typeset this symbol,
it will appear in fonts with the default encoding, while in any font loaded with
the |nobang| encoding an attempt to access the symbol will either use the default
fallback definition or return an error, depending on the symbol being undeclared.


The third use case is to redefine a symbol or accent. The most common use case
in this scenario is to adjust a specific accent command to either fine-tune its placement
or to `fake' it entirely.
For example, the underdot diacritic is used in typeset Sanskrit,
but it is not necessarily included as an accent symbol is all fonts.
By default the underdot is defined in |TU| as:
\begin{Verbatim}
\EncodingAccent{\d}{"0323}
\end{Verbatim}
For fonts with a missing (or poorly-spaced) |"0323| accent glyph, the `traditional' \TeX\ fake accent
construction could be used instead:
\begin{Verbatim}
\DeclareUnicodeEncoding{fakeacc}{
  %%
%% This is file `tuenc.def',
%% generated with the docstrip utility.
%%
%% The original source files were:
%%
%% ltoutenc.dtx  (with options: `TU')
%% 
%% This is a generated file.
%% 
%% The source is maintained by the LaTeX Project team and bug
%% reports for it can be opened at http://latex-project.org/bugs.html
%% (but please observe conditions on bug reports sent to that address!)
%% 
%% 
%% Copyright 1993-2016
%% The LaTeX3 Project and any individual authors listed elsewhere
%% in this file.
%% 
%% This file was generated from file(s) of the LaTeX base system.
%% --------------------------------------------------------------
%% 
%% It may be distributed and/or modified under the
%% conditions of the LaTeX Project Public License, either version 1.3c
%% of this license or (at your option) any later version.
%% The latest version of this license is in
%%    http://www.latex-project.org/lppl.txt
%% and version 1.3c or later is part of all distributions of LaTeX
%% version 2005/12/01 or later.
%% 
%% This file has the LPPL maintenance status "maintained".
%% 
%% This file may only be distributed together with a copy of the LaTeX
%% base system. You may however distribute the LaTeX base system without
%% such generated files.
%% 
%% The list of all files belonging to the LaTeX base distribution is
%% given in the file `manifest.txt'. See also `legal.txt' for additional
%% information.
%% 
%% The list of derived (unpacked) files belonging to the distribution
%% and covered by LPPL is defined by the unpacking scripts (with
%% extension .ins) which are part of the distribution.
%%% From File: ltoutenc.dtx
\ProvidesFile{tuenc.def}
 [2017/02/22 v2.0g
         Standard LaTeX file]
\providecommand\UnicodeEncodingName{TU}
\begingroup\expandafter\expandafter\expandafter\endgroup
\expandafter\ifx\csname XeTeXrevision\endcsname\relax
  \begingroup\expandafter\expandafter\expandafter\endgroup
  \expandafter\ifx\csname directlua\endcsname\relax
    \PackageWarningNoLine{fontenc}
      {\UnicodeEncodingName\space
       encoding is only available with XeTeX and LuaTeX.\MessageBreak
       Defaulting to T1 encoding}
      \def\encodingdefault{T1}
    \expandafter\expandafter\expandafter\endinput
  \else
    \def\UnicodeFontTeXLigatures{+tlig;}
    \def\reserved@a#1{%
      \def\@remove@tlig##1{\@remove@tlig@##1\@nil#1\@nil\relax}
      \def\@remove@tlig@##1#1{\@remove@tlig@@##1}}
    \edef\reserved@b{\detokenize{+tlig;}}
    \expandafter\reserved@a\expandafter{\reserved@b}
    \def\@remove@tlig@@#1\@nil#2\relax{#1}
    \def\remove@tlig#1{%
      \begingroup
      \font\remove@tlig
      \expandafter\@remove@tlig\expandafter{\fontname\font}%
      \remove@tlig
      \char#1\relax
      \endgroup
    }
  \fi
\else
  \def\UnicodeFontTeXLigatures{mapping=tex-text;}
  \def\remove@tlig#1{\XeTeXglyph\numexpr\XeTeXcharglyph#1\relax}
\fi
\def\UnicodeFontFile#1#2{"[#1]:#2"}
\def\UnicodeFontName#1#2{"#1:#2"}
\DeclareFontEncoding\UnicodeEncodingName{}{}
\def\add@unicode@accent#1#2{%
  \if\relax\detokenize{#2}\relax^^a0\else#2\fi
  \char#1\relax}
\def\DeclareUnicodeAccent#1#2#3{%
  \DeclareTextCommand{#1}{#2}{\add@unicode@accent{#3}}%
}
\DeclareTextCommand\textquotesingle \UnicodeEncodingName{%
                                                \remove@tlig{"0027}}
\DeclareTextCommand\textasciigrave  \UnicodeEncodingName{%
                                                \remove@tlig{"0060}}
\DeclareTextCommand\textquotedbl    \UnicodeEncodingName{%
                                                \remove@tlig{"0022}}
\DeclareTextSymbol{\textdollar}          \UnicodeEncodingName{"0024}
\DeclareTextSymbol{\textless}            \UnicodeEncodingName{"003C}
\DeclareTextSymbol{\textgreater}         \UnicodeEncodingName{"003E}
\DeclareTextSymbol{\textbackslash}       \UnicodeEncodingName{"005C}
\DeclareTextSymbol{\textasciicircum}     \UnicodeEncodingName{"005E}
\DeclareTextSymbol{\textunderscore}      \UnicodeEncodingName{"005F}
\DeclareTextSymbol{\textbraceleft}       \UnicodeEncodingName{"007B}
\DeclareTextSymbol{\textbar}             \UnicodeEncodingName{"007C}
\DeclareTextSymbol{\textbraceright}      \UnicodeEncodingName{"007D}
\DeclareTextSymbol{\textasciitilde}      \UnicodeEncodingName{"007E}
\DeclareTextSymbol{\textexclamdown}      \UnicodeEncodingName{"00A1}
\DeclareTextSymbol{\textcent}            \UnicodeEncodingName{"00A2}
\DeclareTextSymbol{\textsterling}        \UnicodeEncodingName{"00A3}
\DeclareTextSymbol{\textcurrency}        \UnicodeEncodingName{"00A4}
\DeclareTextSymbol{\textyen}             \UnicodeEncodingName{"00A5}
\DeclareTextSymbol{\textbrokenbar}       \UnicodeEncodingName{"00A6}
\DeclareTextSymbol{\textsection}         \UnicodeEncodingName{"00A7}
\DeclareTextSymbol{\textasciidieresis}   \UnicodeEncodingName{"00A8}
\DeclareTextSymbol{\textcopyright}       \UnicodeEncodingName{"00A9}
\DeclareTextSymbol{\textordfeminine}     \UnicodeEncodingName{"00AA}
\DeclareTextSymbol{\guillemotleft}       \UnicodeEncodingName{"00AB}
\DeclareTextSymbol{\textlnot}            \UnicodeEncodingName{"00AC}
\DeclareTextSymbol{\textregistered}      \UnicodeEncodingName{"00AE}
\DeclareTextSymbol{\textasciimacron}     \UnicodeEncodingName{"00AF}
\DeclareTextSymbol{\textdegree}          \UnicodeEncodingName{"00B0}
\DeclareTextSymbol{\textpm}              \UnicodeEncodingName{"00B1}
\DeclareTextSymbol{\texttwosuperior}     \UnicodeEncodingName{"00B2}
\DeclareTextSymbol{\textthreesuperior}   \UnicodeEncodingName{"00B3}
\DeclareTextSymbol{\textasciiacute}      \UnicodeEncodingName{"00B4}
\DeclareTextSymbol{\textmu}              \UnicodeEncodingName{"00B5}
\DeclareTextSymbol{\textparagraph}       \UnicodeEncodingName{"00B6}
\DeclareTextSymbol{\textperiodcentered}  \UnicodeEncodingName{"00B7}
\DeclareTextSymbol{\textonesuperior}     \UnicodeEncodingName{"00B9}
\DeclareTextSymbol{\textordmasculine}    \UnicodeEncodingName{"00BA}
\DeclareTextSymbol{\guillemotright}      \UnicodeEncodingName{"00BB}
\DeclareTextSymbol{\textonequarter}      \UnicodeEncodingName{"00BC}
\DeclareTextSymbol{\textonehalf}         \UnicodeEncodingName{"00BD}
\DeclareTextSymbol{\textthreequarters}   \UnicodeEncodingName{"00BE}
\DeclareTextSymbol{\textquestiondown}    \UnicodeEncodingName{"00BF}
\DeclareTextSymbol{\AE}                  \UnicodeEncodingName{"00C6}
\DeclareTextSymbol{\DH}                  \UnicodeEncodingName{"00D0}
\DeclareTextSymbol{\texttimes}           \UnicodeEncodingName{"00D7}
\DeclareTextSymbol{\O}                   \UnicodeEncodingName{"00D8}
\DeclareTextSymbol{\TH}                  \UnicodeEncodingName{"00DE}
\DeclareTextSymbol{\ss}                  \UnicodeEncodingName{"00DF}
\DeclareTextSymbol{\ae}                  \UnicodeEncodingName{"00E6}
\DeclareTextSymbol{\dh}                  \UnicodeEncodingName{"00F0}
\DeclareTextSymbol{\textdiv}             \UnicodeEncodingName{"00F7}
\DeclareTextSymbol{\o}                   \UnicodeEncodingName{"00F8}
\DeclareTextSymbol{\th}                  \UnicodeEncodingName{"00FE}
\DeclareTextSymbol{\DJ}                  \UnicodeEncodingName{"0110}
\DeclareTextSymbol{\dj}                  \UnicodeEncodingName{"0111}
\DeclareTextSymbol{\i}                   \UnicodeEncodingName{"0131}
\DeclareTextSymbol{\IJ}                  \UnicodeEncodingName{"0132}
\DeclareTextSymbol{\ij}                  \UnicodeEncodingName{"0133}
\DeclareTextSymbol{\L}                   \UnicodeEncodingName{"0141}
\DeclareTextSymbol{\l}                   \UnicodeEncodingName{"0142}
\DeclareTextSymbol{\NG}                  \UnicodeEncodingName{"014A}
\DeclareTextSymbol{\ng}                  \UnicodeEncodingName{"014B}
\DeclareTextSymbol{\OE}                  \UnicodeEncodingName{"0152}
\DeclareTextSymbol{\oe}                  \UnicodeEncodingName{"0153}
\DeclareTextSymbol{\textflorin}          \UnicodeEncodingName{"0192}
\DeclareTextComposite{\=}             \UnicodeEncodingName{Y}{"0232}
\DeclareTextComposite{\=}             \UnicodeEncodingName{y}{"0232}
\DeclareTextSymbol{\j}                   \UnicodeEncodingName{"0237}
\DeclareTextSymbol{\textasciicaron}      \UnicodeEncodingName{"02C7}
\DeclareTextSymbol{\textasciibreve}      \UnicodeEncodingName{"02D8}
\DeclareTextSymbol{\textacutedbl}        \UnicodeEncodingName{"02DD}
\DeclareTextSymbol{\textgravedbl}        \UnicodeEncodingName{"02F5}
\DeclareTextSymbol{\texttildelow}        \UnicodeEncodingName{"02F7}
\DeclareTextSymbol{\textbaht}            \UnicodeEncodingName{"0E3F}
\DeclareTextComposite{\=}             \UnicodeEncodingName{G}{"1E20}
\DeclareTextComposite{\=}             \UnicodeEncodingName{g}{"1E21}
\DeclareTextSymbol{\SS}                  \UnicodeEncodingName{"1E9E}
\DeclareTextSymbol{\textcompwordmark}    \UnicodeEncodingName{"200C}
\DeclareTextSymbol{\textendash}          \UnicodeEncodingName{"2013}
\DeclareTextSymbol{\textemdash}          \UnicodeEncodingName{"2014}
\DeclareTextSymbol{\textbardbl}          \UnicodeEncodingName{"2016}
\DeclareTextSymbol{\textquoteleft}       \UnicodeEncodingName{"2018}
\DeclareTextSymbol{\textquoteright}      \UnicodeEncodingName{"2019}
\DeclareTextSymbol{\quotesinglbase}      \UnicodeEncodingName{"201A}
\DeclareTextSymbol{\textquotedblleft}    \UnicodeEncodingName{"201C}
\DeclareTextSymbol{\textquotedblright}   \UnicodeEncodingName{"201D}
\DeclareTextSymbol{\quotedblbase}        \UnicodeEncodingName{"201E}
\DeclareTextSymbol{\textdagger}          \UnicodeEncodingName{"2020}
\DeclareTextSymbol{\textdaggerdbl}       \UnicodeEncodingName{"2021}
\DeclareTextSymbol{\textbullet}          \UnicodeEncodingName{"2022}
\DeclareTextSymbol{\textellipsis}        \UnicodeEncodingName{"2026}
\DeclareTextSymbol{\textperthousand}     \UnicodeEncodingName{"2030}
\DeclareTextSymbol{\textpertenthousand}  \UnicodeEncodingName{"2031}
\DeclareTextSymbol{\guilsinglleft}       \UnicodeEncodingName{"2039}
\DeclareTextSymbol{\guilsinglright}      \UnicodeEncodingName{"203A}
\DeclareTextSymbol{\textreferencemark}   \UnicodeEncodingName{"203B}
\DeclareTextSymbol{\textinterrobang}     \UnicodeEncodingName{"203D}
\DeclareTextSymbol{\textfractionsolidus} \UnicodeEncodingName{"2044}
\DeclareTextSymbol{\textlquill}          \UnicodeEncodingName{"2045}
\DeclareTextSymbol{\textrquill}          \UnicodeEncodingName{"2046}
\DeclareTextSymbol{\textdiscount}        \UnicodeEncodingName{"2052}
\DeclareTextSymbol{\textcolonmonetary}   \UnicodeEncodingName{"20A1}
\DeclareTextSymbol{\textlira}            \UnicodeEncodingName{"20A4}
\DeclareTextSymbol{\textnaira}           \UnicodeEncodingName{"20A6}
\DeclareTextSymbol{\textwon}             \UnicodeEncodingName{"20A9}
\DeclareTextSymbol{\textdong}            \UnicodeEncodingName{"20AB}
\DeclareTextSymbol{\texteuro}            \UnicodeEncodingName{"20AC}
\DeclareTextSymbol{\textpeso}            \UnicodeEncodingName{"20B1}
\DeclareTextSymbol{\textcelsius}         \UnicodeEncodingName{"2103}
\DeclareTextSymbol{\textnumero}          \UnicodeEncodingName{"2116}
\DeclareTextSymbol{\textcircledP}        \UnicodeEncodingName{"2117}
\DeclareTextSymbol{\textrecipe}          \UnicodeEncodingName{"211E}
\DeclareTextSymbol{\textservicemark}     \UnicodeEncodingName{"2120}
\DeclareTextSymbol{\texttrademark}       \UnicodeEncodingName{"2122}
\DeclareTextSymbol{\textohm}             \UnicodeEncodingName{"2126}
\DeclareTextSymbol{\textmho}             \UnicodeEncodingName{"2127}
\DeclareTextSymbol{\textestimated}       \UnicodeEncodingName{"212E}
\DeclareTextSymbol{\textleftarrow}       \UnicodeEncodingName{"2190}
\DeclareTextSymbol{\textuparrow}         \UnicodeEncodingName{"2191}
\DeclareTextSymbol{\textrightarrow}      \UnicodeEncodingName{"2192}
\DeclareTextSymbol{\textdownarrow}       \UnicodeEncodingName{"2193}
\DeclareTextSymbol{\textminus}           \UnicodeEncodingName{"2212}
\DeclareTextCommand{\textasteriskcentered}\UnicodeEncodingName{%
  \iffontchar\font"2217 \char"2217 \else
    \begingroup
      \fontsize
       {\the\dimexpr1.2\dimexpr\f@size pt\relax}%
       {\f@baselineskip}%
      \selectfont
      \raisebox{-0.6ex}[\dimexpr\height-0.6ex][0pt]{*}%
    \endgroup
  \fi
}
\DeclareTextSymbol{\textsurd}            \UnicodeEncodingName{"221A}
\DeclareTextSymbol{\textlangle}          \UnicodeEncodingName{"2329}
\DeclareTextSymbol{\textrangle}          \UnicodeEncodingName{"232A}
\DeclareTextSymbol{\textblank}           \UnicodeEncodingName{"2422}
\DeclareTextSymbol{\textvisiblespace}    \UnicodeEncodingName{"2423}
\DeclareTextSymbol{\textopenbullet}      \UnicodeEncodingName{"25E6}
\DeclareTextSymbol{\textbigcircle}       \UnicodeEncodingName{"25EF}
\DeclareTextSymbol{\textmusicalnote}     \UnicodeEncodingName{"266A}
\DeclareTextSymbol{\textmarried}         \UnicodeEncodingName{"26AD}
\DeclareTextSymbol{\textdivorced}        \UnicodeEncodingName{"26AE}
\DeclareTextSymbol{\textinterrobangdown} \UnicodeEncodingName{"2E18}
\DeclareUnicodeAccent{\`}                \UnicodeEncodingName{"0300}
\DeclareUnicodeAccent{\'}                \UnicodeEncodingName{"0301}
\DeclareUnicodeAccent{\^}                \UnicodeEncodingName{"0302}
\DeclareUnicodeAccent{\~}                \UnicodeEncodingName{"0303}
\DeclareUnicodeAccent{\"}                \UnicodeEncodingName{"0308}
\DeclareUnicodeAccent{\H}                \UnicodeEncodingName{"030B}
\DeclareUnicodeAccent{\r}                \UnicodeEncodingName{"030A}
\DeclareUnicodeAccent{\v}                \UnicodeEncodingName{"030C}
\DeclareUnicodeAccent{\u}                \UnicodeEncodingName{"0306}
\DeclareUnicodeAccent{\=}                \UnicodeEncodingName{"0304}
\DeclareUnicodeAccent{\.}                \UnicodeEncodingName{"0307}
\DeclareUnicodeAccent{\b}                \UnicodeEncodingName{"0332}
\DeclareUnicodeAccent{\c}                \UnicodeEncodingName{"0327}
\DeclareUnicodeAccent{\d}                \UnicodeEncodingName{"0323}
\DeclareUnicodeAccent{\k}                \UnicodeEncodingName{"0328}
\DeclareTextComposite{\^}             \UnicodeEncodingName {}{"005E}
\DeclareTextComposite{\~}             \UnicodeEncodingName {}{"007E}
\DeclareTextComposite{\`}             \UnicodeEncodingName{A}{"00C0}
\DeclareTextComposite{\'}             \UnicodeEncodingName{A}{"00C1}
\DeclareTextComposite{\^}             \UnicodeEncodingName{A}{"00C2}
\DeclareTextComposite{\~}             \UnicodeEncodingName{A}{"00C3}
\DeclareTextComposite{\"}             \UnicodeEncodingName{A}{"00C4}
\DeclareTextComposite{\r}             \UnicodeEncodingName{A}{"00C5}
\DeclareTextComposite{\c}             \UnicodeEncodingName{C}{"00C7}
\DeclareTextComposite{\`}             \UnicodeEncodingName{E}{"00C8}
\DeclareTextComposite{\'}             \UnicodeEncodingName{E}{"00C9}
\DeclareTextComposite{\^}             \UnicodeEncodingName{E}{"00CA}
\DeclareTextComposite{\"}             \UnicodeEncodingName{E}{"00CB}
\DeclareTextComposite{\`}             \UnicodeEncodingName{I}{"00CC}
\DeclareTextComposite{\'}             \UnicodeEncodingName{I}{"00CD}
\DeclareTextComposite{\^}             \UnicodeEncodingName{I}{"00CE}
\DeclareTextComposite{\"}             \UnicodeEncodingName{I}{"00CF}
\DeclareTextComposite{\~}             \UnicodeEncodingName{N}{"00D1}
\DeclareTextComposite{\`}             \UnicodeEncodingName{O}{"00D2}
\DeclareTextComposite{\'}             \UnicodeEncodingName{O}{"00D3}
\DeclareTextComposite{\^}             \UnicodeEncodingName{O}{"00D4}
\DeclareTextComposite{\~}             \UnicodeEncodingName{O}{"00D5}
\DeclareTextComposite{\"}             \UnicodeEncodingName{O}{"00D6}
\DeclareTextComposite{\`}             \UnicodeEncodingName{U}{"00D9}
\DeclareTextComposite{\'}             \UnicodeEncodingName{U}{"00DA}
\DeclareTextComposite{\^}             \UnicodeEncodingName{U}{"00DB}
\DeclareTextComposite{\"}             \UnicodeEncodingName{U}{"00DC}
\DeclareTextComposite{\'}             \UnicodeEncodingName{Y}{"00DD}
\DeclareTextComposite{\`}             \UnicodeEncodingName{a}{"00E0}
\DeclareTextComposite{\'}             \UnicodeEncodingName{a}{"00E1}
\DeclareTextComposite{\^}             \UnicodeEncodingName{a}{"00E2}
\DeclareTextComposite{\~}             \UnicodeEncodingName{a}{"00E3}
\DeclareTextComposite{\"}             \UnicodeEncodingName{a}{"00E4}
\DeclareTextComposite{\r}             \UnicodeEncodingName{a}{"00E5}
\DeclareTextComposite{\c}             \UnicodeEncodingName{c}{"00E7}
\DeclareTextComposite{\`}             \UnicodeEncodingName{e}{"00E8}
\DeclareTextComposite{\'}             \UnicodeEncodingName{e}{"00E9}
\DeclareTextComposite{\^}             \UnicodeEncodingName{e}{"00EA}
\DeclareTextComposite{\"}             \UnicodeEncodingName{e}{"00EB}
\DeclareTextComposite{\`}             \UnicodeEncodingName\i {"00EC}
\DeclareTextComposite{\`}             \UnicodeEncodingName{i}{"00EC}
\DeclareTextComposite{\'}             \UnicodeEncodingName\i {"00ED}
\DeclareTextComposite{\'}             \UnicodeEncodingName{i}{"00ED}
\DeclareTextComposite{\^}             \UnicodeEncodingName\i {"00EE}
\DeclareTextComposite{\^}             \UnicodeEncodingName{i}{"00EE}
\DeclareTextComposite{\"}             \UnicodeEncodingName\i {"00EF}
\DeclareTextComposite{\"}             \UnicodeEncodingName{i}{"00EF}
\DeclareTextComposite{\~}             \UnicodeEncodingName{n}{"00F1}
\DeclareTextComposite{\`}             \UnicodeEncodingName{o}{"00F2}
\DeclareTextComposite{\'}             \UnicodeEncodingName{o}{"00F3}
\DeclareTextComposite{\^}             \UnicodeEncodingName{o}{"00F4}
\DeclareTextComposite{\~}             \UnicodeEncodingName{o}{"00F5}
\DeclareTextComposite{\"}             \UnicodeEncodingName{o}{"00F6}
\DeclareTextComposite{\`}             \UnicodeEncodingName{u}{"00F9}
\DeclareTextComposite{\'}             \UnicodeEncodingName{u}{"00FA}
\DeclareTextComposite{\^}             \UnicodeEncodingName{u}{"00FB}
\DeclareTextComposite{\"}             \UnicodeEncodingName{u}{"00FC}
\DeclareTextComposite{\'}             \UnicodeEncodingName{y}{"00FD}
\DeclareTextComposite{\"}             \UnicodeEncodingName{y}{"00FF}
\DeclareTextComposite{\=}             \UnicodeEncodingName{A}{"0100}
\DeclareTextComposite{\=}             \UnicodeEncodingName{a}{"0101}
\DeclareTextComposite{\u}             \UnicodeEncodingName{A}{"0102}
\DeclareTextComposite{\u}             \UnicodeEncodingName{a}{"0103}
\DeclareTextComposite{\k}             \UnicodeEncodingName{A}{"0104}
\DeclareTextComposite{\k}             \UnicodeEncodingName{a}{"0105}
\DeclareTextComposite{\'}             \UnicodeEncodingName{C}{"0106}
\DeclareTextComposite{\'}             \UnicodeEncodingName{c}{"0107}
\DeclareTextComposite{\^}             \UnicodeEncodingName{C}{"0108}
\DeclareTextComposite{\^}             \UnicodeEncodingName{c}{"0109}
\DeclareTextComposite{\.}             \UnicodeEncodingName{C}{"010A}
\DeclareTextComposite{\.}             \UnicodeEncodingName{c}{"010B}
\DeclareTextComposite{\v}             \UnicodeEncodingName{C}{"010C}
\DeclareTextComposite{\v}             \UnicodeEncodingName{c}{"010D}
\DeclareTextComposite{\v}             \UnicodeEncodingName{D}{"010E}
\DeclareTextComposite{\v}             \UnicodeEncodingName{d}{"010F}
\DeclareTextComposite{\=}             \UnicodeEncodingName{E}{"0112}
\DeclareTextComposite{\=}             \UnicodeEncodingName{e}{"0113}
\DeclareTextComposite{\u}             \UnicodeEncodingName{E}{"0114}
\DeclareTextComposite{\u}             \UnicodeEncodingName{e}{"0115}
\DeclareTextComposite{\.}             \UnicodeEncodingName{E}{"0116}
\DeclareTextComposite{\.}             \UnicodeEncodingName{e}{"0117}
\DeclareTextComposite{\k}             \UnicodeEncodingName{E}{"0118}
\DeclareTextComposite{\k}             \UnicodeEncodingName{e}{"0119}
\DeclareTextComposite{\v}             \UnicodeEncodingName{E}{"011A}
\DeclareTextComposite{\v}             \UnicodeEncodingName{e}{"011B}
\DeclareTextComposite{\^}             \UnicodeEncodingName{G}{"011C}
\DeclareTextComposite{\^}             \UnicodeEncodingName{g}{"011D}
\DeclareTextComposite{\u}             \UnicodeEncodingName{G}{"011E}
\DeclareTextComposite{\u}             \UnicodeEncodingName{g}{"011F}
\DeclareTextComposite{\.}             \UnicodeEncodingName{G}{"0120}
\DeclareTextComposite{\.}             \UnicodeEncodingName{g}{"0121}
\DeclareTextComposite{\c}             \UnicodeEncodingName{G}{"0122}
\DeclareTextComposite{\c}             \UnicodeEncodingName{g}{"0123}
\DeclareTextComposite{\^}             \UnicodeEncodingName{H}{"0124}
\DeclareTextComposite{\^}             \UnicodeEncodingName{h}{"0125}
\DeclareTextComposite{\~}             \UnicodeEncodingName{I}{"0128}
\DeclareTextComposite{\~}             \UnicodeEncodingName\i {"0129}
\DeclareTextComposite{\~}             \UnicodeEncodingName{i}{"0129}
\DeclareTextComposite{\=}             \UnicodeEncodingName{I}{"012A}
\DeclareTextComposite{\=}             \UnicodeEncodingName\i {"012B}
\DeclareTextComposite{\=}             \UnicodeEncodingName{i}{"012B}
\DeclareTextComposite{\u}             \UnicodeEncodingName{I}{"012C}
\DeclareTextComposite{\u}             \UnicodeEncodingName\i {"012D}
\DeclareTextComposite{\u}             \UnicodeEncodingName{i}{"012D}
\DeclareTextComposite{\k}             \UnicodeEncodingName{I}{"012E}
\DeclareTextComposite{\k}             \UnicodeEncodingName\i {"012F}
\DeclareTextComposite{\k}             \UnicodeEncodingName{i}{"012F}
\DeclareTextComposite{\.}             \UnicodeEncodingName{I}{"0130}
\DeclareTextComposite{\^}             \UnicodeEncodingName{J}{"0134}
\DeclareTextComposite{\^}             \UnicodeEncodingName\j {"0135}
\DeclareTextComposite{\^}             \UnicodeEncodingName{j}{"0135}
\DeclareTextComposite{\c}             \UnicodeEncodingName{K}{"0136}
\DeclareTextComposite{\c}             \UnicodeEncodingName{k}{"0137}
\DeclareTextComposite{\'}             \UnicodeEncodingName{L}{"0139}
\DeclareTextComposite{\'}             \UnicodeEncodingName{l}{"013A}
\DeclareTextComposite{\c}             \UnicodeEncodingName{L}{"013B}
\DeclareTextComposite{\c}             \UnicodeEncodingName{l}{"013C}
\DeclareTextComposite{\v}             \UnicodeEncodingName{L}{"013D}
\DeclareTextComposite{\v}             \UnicodeEncodingName{l}{"013E}
\DeclareTextComposite{\'}             \UnicodeEncodingName{N}{"0143}
\DeclareTextComposite{\'}             \UnicodeEncodingName{n}{"0144}
\DeclareTextComposite{\c}             \UnicodeEncodingName{N}{"0145}
\DeclareTextComposite{\c}             \UnicodeEncodingName{n}{"0146}
\DeclareTextComposite{\v}             \UnicodeEncodingName{N}{"0147}
\DeclareTextComposite{\v}             \UnicodeEncodingName{n}{"0148}
\DeclareTextComposite{\=}             \UnicodeEncodingName{O}{"014C}
\DeclareTextComposite{\=}             \UnicodeEncodingName{o}{"014D}
\DeclareTextComposite{\u}             \UnicodeEncodingName{O}{"014E}
\DeclareTextComposite{\u}             \UnicodeEncodingName{o}{"014F}
\DeclareTextComposite{\H}             \UnicodeEncodingName{O}{"0150}
\DeclareTextComposite{\H}             \UnicodeEncodingName{o}{"0151}
\DeclareTextComposite{\'}             \UnicodeEncodingName{R}{"0154}
\DeclareTextComposite{\'}             \UnicodeEncodingName{r}{"0155}
\DeclareTextComposite{\c}             \UnicodeEncodingName{R}{"0156}
\DeclareTextComposite{\c}             \UnicodeEncodingName{r}{"0157}
\DeclareTextComposite{\v}             \UnicodeEncodingName{R}{"0158}
\DeclareTextComposite{\v}             \UnicodeEncodingName{r}{"0159}
\DeclareTextComposite{\'}             \UnicodeEncodingName{S}{"015A}
\DeclareTextComposite{\'}             \UnicodeEncodingName{s}{"015B}
\DeclareTextComposite{\^}             \UnicodeEncodingName{S}{"015C}
\DeclareTextComposite{\^}             \UnicodeEncodingName{s}{"015D}
\DeclareTextComposite{\c}             \UnicodeEncodingName{S}{"015E}
\DeclareTextComposite{\c}             \UnicodeEncodingName{s}{"015F}
\DeclareTextComposite{\v}             \UnicodeEncodingName{S}{"0160}
\DeclareTextComposite{\v}             \UnicodeEncodingName{s}{"0161}
\DeclareTextComposite{\c}             \UnicodeEncodingName{T}{"0162}
\DeclareTextComposite{\c}             \UnicodeEncodingName{t}{"0163}
\DeclareTextComposite{\v}             \UnicodeEncodingName{T}{"0164}
\DeclareTextComposite{\v}             \UnicodeEncodingName{t}{"0165}
\DeclareTextComposite{\~}             \UnicodeEncodingName{U}{"0168}
\DeclareTextComposite{\~}             \UnicodeEncodingName{u}{"0169}
\DeclareTextComposite{\=}             \UnicodeEncodingName{U}{"016A}
\DeclareTextComposite{\=}             \UnicodeEncodingName{u}{"016B}
\DeclareTextComposite{\u}             \UnicodeEncodingName{U}{"016C}
\DeclareTextComposite{\u}             \UnicodeEncodingName{u}{"016D}
\DeclareTextComposite{\r}             \UnicodeEncodingName{U}{"016E}
\DeclareTextComposite{\r}             \UnicodeEncodingName{u}{"016F}
\DeclareTextComposite{\H}             \UnicodeEncodingName{U}{"0170}
\DeclareTextComposite{\H}             \UnicodeEncodingName{u}{"0171}
\DeclareTextComposite{\k}             \UnicodeEncodingName{U}{"0172}
\DeclareTextComposite{\k}             \UnicodeEncodingName{u}{"0173}
\DeclareTextComposite{\^}             \UnicodeEncodingName{W}{"0174}
\DeclareTextComposite{\^}             \UnicodeEncodingName{w}{"0175}
\DeclareTextComposite{\^}             \UnicodeEncodingName{Y}{"0176}
\DeclareTextComposite{\^}             \UnicodeEncodingName{y}{"0177}
\DeclareTextComposite{\"}             \UnicodeEncodingName{Y}{"0178}
\DeclareTextComposite{\'}             \UnicodeEncodingName{Z}{"0179}
\DeclareTextComposite{\'}             \UnicodeEncodingName{z}{"017A}
\DeclareTextComposite{\.}             \UnicodeEncodingName{Z}{"017B}
\DeclareTextComposite{\.}             \UnicodeEncodingName{z}{"017C}
\DeclareTextComposite{\v}             \UnicodeEncodingName{Z}{"017D}
\DeclareTextComposite{\v}             \UnicodeEncodingName{z}{"017E}
\DeclareTextComposite{\v}             \UnicodeEncodingName{A}{"01CD}
\DeclareTextComposite{\v}             \UnicodeEncodingName{a}{"01CE}
\DeclareTextComposite{\v}             \UnicodeEncodingName{I}{"01CF}
\DeclareTextComposite{\v}             \UnicodeEncodingName\i {"01D0}
\DeclareTextComposite{\v}             \UnicodeEncodingName{i}{"01D0}
\DeclareTextComposite{\v}             \UnicodeEncodingName{O}{"01D1}
\DeclareTextComposite{\v}             \UnicodeEncodingName{o}{"01D2}
\DeclareTextComposite{\v}             \UnicodeEncodingName{U}{"01D3}
\DeclareTextComposite{\v}             \UnicodeEncodingName{u}{"01D4}
\DeclareTextComposite{\=}             \UnicodeEncodingName\AE{"01E2}
\DeclareTextComposite{\=}             \UnicodeEncodingName\ae{"01E3}
\DeclareTextComposite{\v}             \UnicodeEncodingName{G}{"01E6}
\DeclareTextComposite{\v}             \UnicodeEncodingName{g}{"01E7}
\DeclareTextComposite{\v}             \UnicodeEncodingName{K}{"01E8}
\DeclareTextComposite{\v}             \UnicodeEncodingName{k}{"01E9}
\DeclareTextComposite{\k}             \UnicodeEncodingName{O}{"01EA}
\DeclareTextComposite{\k}             \UnicodeEncodingName{o}{"01EB}
\DeclareTextComposite{\v}             \UnicodeEncodingName\j {"01F0}
\DeclareTextComposite{\v}             \UnicodeEncodingName{j}{"01F0}
\DeclareTextComposite{\'}             \UnicodeEncodingName{G}{"01F4}
\DeclareTextComposite{\'}             \UnicodeEncodingName{g}{"01F5}
\DeclareTextComposite{\textcommabelow}\UnicodeEncodingName{S}{"0218}
\DeclareTextComposite{\textcommabelow}\UnicodeEncodingName{s}{"0219}
\DeclareTextComposite{\textcommabelow}\UnicodeEncodingName{T}{"021A}
\DeclareTextComposite{\textcommabelow}\UnicodeEncodingName{t}{"021B}
\DeclareTextComposite{\.}             \UnicodeEncodingName{B}{"1E02}
\DeclareTextComposite{\.}             \UnicodeEncodingName{b}{"1E03}
\endinput
%%
%% End of file `tuenc.def'.

  \EncodingCommand{\d}[1]{%
    \hmode@bgroup
      \o@lign{\relax#1\crcr\hidewidth\ltx@sh@ft{-1ex}.\hidewidth}%
    \egroup
  }
}
\end{Verbatim}
This would be set up in a document as such:
\begin{Verbatim}
\newfontfamily\sanskitfont{CharisSIL}
\newfontfamily\titlefont{Posterama}[NFSSEncoding=fakeacc]
\end{Verbatim}
Then later in the document, no additional work is needed:
\begin{Verbatim}
...{\titlefont   kalita\d m}...  % <- uses fake accent
...{\sanskitfont kalita\d m}...  % <- uses real accent
\end{Verbatim}
To reiterate from above, typing this input with Unicode text (`|kalitaṃ|')
will \emph{bypass} this encoding mechanism and you will receive only what is contained
literally within the font.


\section{Summary of commands}

The \LaTeXe\ kernel provides the following font encoding commands suitable for Unicode encodings:
\begin{quote}\obeylines
  \cs{DeclareTextCommand}\marg{command}\marg{encoding}\oarg{num}\oarg{default}\marg{code}
  \cs{DeclareUnicodeAccent}\marg{command}\marg{encoding}\marg{slot}
  \cs{DeclareTextSymbol}\marg{command}\marg{encoding}\marg{slot}
  \cs{DeclareTextComposite}\marg{command}\marg{encoding}\marg{letter}\marg{slot}
  \cs{DeclareTextCompositeCommand}\marg{command}\marg{encoding}\marg{letter}\marg{code}
  \cs{UndeclareTextCommand}\marg{command}\marg{encoding}
\end{quote}
See |fntguide.pdf| for full documentation of these.
As shown above, the following shorthands are provided by \pkg{fontspec} to simplify
the process of defining Unicode font range encodings:
\begin{quote}\obeylines
  \cs{EncodingCommand}\marg{command}\oarg{num}\oarg{default}\marg{code}
  \cs{EncodingAccent}\marg{command}\marg{code}
  \cs{EncodingSymbol}\marg{command}\marg{code}
  \cs{EncodingComposite}\marg{command}\marg{letter}\marg{slot}
  \cs{EncodingCompositeCommand}\marg{command}\marg{letter}\marg{code}
  \cs{UndeclareSymbol}\marg{command}
  \cs{UndeclareAccent}\marg{command}
  \cs{UndeclareCommand}\marg{command}
  \cs{UndeclareComposite}\marg{command}\marg{letter}
\end{quote}

\end{document}


% /©
% ------------------------------------------------
% The FONTSPEC package  <wspr.io/fontspec>
% ------------------------------------------------
% Copyright  2004-2019  Will Robertson, LPPL "maintainer"
% Copyright  2009-2015  Khaled Hosny
% Copyright  2013       Philipp Gesang
% Copyright  2013-2016  Joseph Wright
% ------------------------------------------------
% This package is free software and may be redistributed and/or modified under
% the conditions of the LaTeX Project Public License, version 1.3c or higher
% (your choice): <http://www.latex-project.org/lppl/>.
% ------------------------------------------------
% ©/
