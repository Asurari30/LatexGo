% 
% This is file `caption-eng.tex'.
% 
% $Id$
% $HeadURL$
%
% Copyright (C) 1994-2012 Axel Sommerfeldt (axel.sommerfeldt@f-m.fm)
% 
% --------------------------------------------------------------------------
% 
% This work may be distributed and/or modified under the
% conditions of the LaTeX Project Public License, either version 1.3
% of this license or (at your option) any later version.
% The latest version of this license is in
%   http://www.latex-project.org/lppl.txt
% and version 1.3 or later is part of all distributions of LaTeX
% version 2003/12/01 or later.
% 
% This work has the LPPL maintenance status "maintained".
% 
% This Current Maintainer of this work is Axel Sommerfeldt.
% 
% This work consists of the files caption.ins, caption.dtx, caption2.dtx,
% caption3.dtx, bicaption.dtx, ltcaption.dtx, subcaption.dtx, and newfloat.dtx,
% the derived files caption.sty, caption2.sty, caption3.sty,
% bicaption.sty, ltcaption.sty, subcaption.sty, and newfloat.sty,
% and the user manuals caption-deu.tex, caption-eng.tex, and caption-rus.tex.
% 
\NeedsTeXFormat{LaTeX2e}[1994/12/01]
\ProvidesFile{caption-eng.tex}[2011/11/02 v3.2 The caption package]
\hbadness=9999 \newcount\hbadness \hfuzz=20pt % Make TeX shut up.
%\errorcontextlines=3

\documentclass[USenglish]{ltxdoc}
\setlength\parindent{0pt}
\setlength\parskip{\smallskipamount}
\setlength\leftmargini{2em}% default = 2.5em
\makeatletter\g@addto@macro\MacroFont{\normalcolor}\makeatother

\newcommand\LineBreak{\linebreak[3]}
\newcommand\PageBreak{\pagebreak[3]}
\usepackage{ifpdf}
\ifpdf
  \usepackage{mathptmx,courier}
  \usepackage[scaled=0.90]{helvet}
  \addtolength\marginparwidth{15pt}
  \ifdim\paperheight=297mm % a4paper
    \renewcommand\LineBreak{\\}
    \renewcommand\PageBreak{\clearpage}
  \fi
\fi

\usepackage[bottom]{footmisc}
\usepackage{graphicx,longtable,setspace}

\usepackage{hypdoc}
\ifpdf\usepackage{hypdestopt}\fi
\hypersetup{pdfkeywords={LaTeX, package, caption},pdfstartpage={},pdfstartview={}}
\hypersetup{breaklinks=true}

\usepackage[listof=0,hypcap=false]{caption}[2008/04/01]

\DeclareRobustCommand*\eTeX{\texorpdfstring
  {\leavevmode\hbox{$\varepsilon$}-\TeX}%
  {e-TeX}}
\DeclareRobustCommand*\AmS{\texorpdfstring
  {{\protect\usefont{OMS}{cmsy}{m}{n}A\kern-.1667em\lower.5ex\hbox{M}\kern-.125emS}}%
  {AMS}}
\DeclareRobustCommand*\KOMAScript{\texorpdfstring
  {\textsf{K\kern.05em O\kern.05em M\kern.05em A\kern.1em-\kern.1em Script}}%
  {KOMA-Script}}
\DeclareRobustCommand*\NTG{NTG}
\DeclareRobustCommand*\SmF{SMF}

\usepackage{marvosym}
\makeatletter
\newcommand*\INFO{\@ifstar{\@INFO{}}{\@INFO{\vbox to \ht\strutbox}}}
\newcommand*\@INFO[1]{\MARGINSYM{#1{\LARGE\Info}}}
\makeatother

\newcommand*\MARGINSYM[1]{\hskip 1sp \marginpar{\raggedleft\textcolor{blue}{{#1}}}}
\newcommand*\NEW[2]{}%\hskip 1sp \marginpar{\footnotesize\sffamily\raggedleft#1\\#2}}

% \ContinuedFloat
\DeclareCaptionLabelFormat{continued1}{Continued #1~#2}
\DeclareCaptionLabelFormat{continued2}{#1~#2\alph{ContinuedFloat}}
% \DeclareCaptionLabelFormat
\DeclareCaptionLabelFormat{bf-parens}{(\textbf{#2})}
% \DeclareCaptionStyle
\DeclareCaptionStyle{mystyle}[margin=5mm,justification=centering]%
                    {font=footnotesize,labelfont=sc,margin={10mm,0mm}}
% Example 1
\DeclareCaptionFormat{myformat1}{#1#2\\#3}
\newlength\myindention
\DeclareCaptionFormat{myformat2}{#1#2\\\hspace*\myindention#3}
\DeclareCaptionOption{myindention}{\setlength\myindention{#1}}
% Example 2
\DeclareCaptionFormat{reverse}{#3#2#1}
\DeclareCaptionLabelFormat{fullparens}{(\bothIfFirst{#1}{~}#2)}
\DeclareCaptionLabelSeparator{fill}{\hfill}
% Example 3
\DeclareCaptionFormat{llap}{\llap{#1#2}#3\par}
\DeclareCaptionFormat{llapx}{\llap{\makebox[5em][l]{#1}}#3\par}
% Example 4
\DeclareCaptionLabelFormat{andtable}{#1~#2 \& \tablename~\thetable}

\newcommand*\purerm[1]{{\upshape\mdseries\rmfamily #1}}
\newcommand*\puresf[1]{{\upshape\mdseries\sffamily #1}}
\newcommand*\purett[1]{{\upshape\mdseries\ttfamily #1}}
\let\class\puresf \let\package\puresf

\newcommand*\csmarg[1]{\texttt{\char`\{#1\char`\}}}
\newcommand*\csoarg[1]{\texttt{\char`\[#1\char`\]}}
\newcommand*\version[2][]{$v#2$}
\newcommand*\x{\discretionary{}{}{}}

\newenvironment{Annotation}%
  {\ifvmode\else\unskip\par\fi\pagebreak[2]%
   \smallskip
   \small(\ignorespaces}{\unskip)\par}
\newenvironment{Annotation*}%
  {\ifvmode\else\unskip\par\fi\pagebreak[2]%
   \small(\ignorespaces}{\unskip)\par}

\newenvironment{Example}%
  {\ifvmode\else\unskip\par\fi\pagebreak[2]%
   \minipage{\linewidth}\smallskip}%
  {\smallskip\endminipage\par}

\makeatletter
\newcommand\example{\@ifstar
  {\@example{belowskip=\abovecaptionskip}}%
  {\@example{}}}
\def\@example#1{%
  \@testopt{\@@example{#1}}{figure}}
\long\def\@@example#1[#2]#3#4{%
  \begingroup
    \captionsetup{#1,size=small,margin={\leftmargini,10pt},#3}%
    \captionof{#2}[]{#4}%
  \endgroup}
\makeatother

\newenvironment{Expert}%
  {\ifvmode\else\unskip\par\fi\pagebreak[3]%
   \smallskip
   \footnotesize\ignorespaces}{\par}
\newenvironment{Expert*}%
  {\ifvmode\else\unskip\par\fi\pagebreak[3]%
   \footnotesize\ignorespaces}{\par}

\newenvironment{Note}[1][\DefaultNoteText]%
  {\ifvmode\else\unskip\par\fi
   \smallskip
   \small\emph{#1:}~\ignorespaces}{\par}
\newenvironment{Note*}[1][\DefaultNoteText]%
  {\ifvmode\else\unskip\par\fi
   \small\emph{#1:}~\ignorespaces}{\par}

\newenvironment{Options}[1]%
  {\list{}{\renewcommand\makelabel[1]{\texttt{##1}\hfil}%
   \settowidth\labelwidth{\texttt{#1\space}}%
   \setlength\leftmargin{10pt}%
   \addtolength\leftmargin{\labelwidth}%
   \addtolength\leftmargin{\labelsep}}}%
  {\endlist}

\makeatletter
\newcommand*\Ref{\@ifstar{\@Ref\ref}{\@Ref\autoref}}
\newcommand*\@Ref[2]{#1{#2}: \textit{\nameref{#2}}}
\newcommand*\SEE[3][]{\nopagebreak{#1(#2 #3)}}
\newcommand*\SeeUserDefined[1][]{\See{\Ref{declare}#1}}
\makeatother

\begin{document}
\let\subsectionautorefname\sectionautorefname
\let\subsubsectionautorefname\sectionautorefname

\def\thispackage{the \package{caption} package}
\def\Thispackage{The \package{caption} package}

\newcommand*\DefaultNoteText{Note}
\newcommand*\NEWfeature{\NEW{New feature}}
\newcommand*\NEWdescription{\NEW{New description}}
\makeatletter
\newcommand*\See{\@ifstar{\SEE{See}}{\SEE[\small]{See}}}
\newcommand*\see{\@ifstar{\SEE{see}}{\SEE[\small]{see}}}
\makeatother

% --------------------------------------------------------------------------- %

\GetFileInfo{caption-eng.tex}
\let\docdate\filedate
\GetFileInfo{caption.sty}

\title{Customizing captions of floating environments}
\author{Axel Sommerfeldt\\
        \url{https://gitlab.com/axelsommerfeldt/caption}}
\date{\docdate}
\maketitle

% --------------------------------------------------------------------------- %

\begin{abstract}
\Thispackage\ offers customization of captions
in floating environments such |figure| and |table| and cooperates with
many other packages.
\par\smallskip
\textit{Please note:} Many document classes already have build-in options and
commands for customizing captions.
If these possibilities are sufficient for you, there is usually no need for
you to use the \package{caption} package at all.
And if you are just interested in using the command \cs{captionof}, loading of
the very small \package{capt-of} package is usually sufficient.
\end{abstract}

\newcommand\exampletext{%
  White sand beaches. The pink smoothness of the conch shell. A sea abundant
  with possibilities. Duty-free shops filled with Europe's finest gifts and
  perfumes. Play your favorite game of golf amidst the tropical greens on one
  of the many championship courses.}

% --------------------------------------------------------------------------- %

\section*{Introduction}

Within the standard \LaTeX\ document classes captions haven't received the
attention they deserve. Simply typeset as an ordinary paragraph there is no
remarkable visual difference from the rest of the text, like here:

\example*{size=normalsize,margin=0pt}{\exampletext}

There should be possibilities to change this; for example, it would be nice
to make the text of the caption a little bit smaller as the normal
text, add an extra margin, typeset the caption label with the same font
family and shape as your headings etc. Just like this one:

\example*{size=small,margin=10pt,labelfont=bf,labelsep=endash}{\exampletext}

You can do this easily with this package as there are many ready-to-use
caption formatting options, but you are free to define your very own stuff, too.

\bigskip

\INFO\begin{minipage}[t]{\textwidth}
\small
Please note that \thispackage\ is only controlling the look \& feel of the
captions. It does \emph{not} control the placement of the captions.
(But you could do so by using other packages like the \package{floatrow}
 package\cite{floatrow}.)
\end{minipage}

% --------------------------------------------------------------------------- %

\clearpage
\tableofcontents

% --------------------------------------------------------------------------- %

\clearpage
\section{Using this package}
\label{usage}

\DescribeMacro{\usepackage}
Insert
\begin{quote}
  |\usepackage|\oarg{options}|{caption}[|\texttt{\filedate}|]|
\end{quote}
into the preamble of your document, i.e.~the part of your document
between |\document|\-|class| and |\begin{doc|\-|u|\-|ment}|.
The options control how your captions will look like; e.g.,
\begin{quote}
  |\usepackage[margin=10pt,font=small,labelfont=bf,|\\
  |            labelsep=endash]{caption}|%
  %|[|\texttt{\filedate}|]|
\end{quote}
would result in captions looking like the second one in the introduction.

\DescribeMacro{\captionsetup}
For a later change of options \thispackage\ provides the command
\begin{quote}
  |\captionsetup|\oarg{float type}\marg{options}\quad.
\end{quote}
So
\begin{quote}
  |\usepackage[margin=10pt,font=small,labelfont=bf]{caption}|
\end{quote}
and
\begin{quote}
  |\usepackage{caption}|\\
  |\captionsetup{margin=10pt,font=small,labelfont=bf}|
\end{quote}
are equal in their results.

It's good to know that |\caption|\-|setup| has an effect on the current
environment only. So if you want to change settings for the
current |figure| or |table| only, just place the |\caption|\-|setup| command
inside the |figure| or |table| right before the |\caption| command.
For example
\begin{quote}
  |\begin{figure}|\\
  |  |\ldots\\
  |  \captionsetup{singlelinecheck=off}|\\
  |  \caption{|\ldots|}|\\
  |\end{figure}|
\end{quote}
switches the single-line-check off, but only for this |figure|,
so all the other captions remain untouched.

\begin{Annotation}
For a detailed description of |\caption|\-|setup| see \Ref{captionsetup}.
\end{Annotation}

% --------------------------------------------------------------------------- %

\clearpage
\section{Options}
\label{options}

\def\OptionLabel{RaggedRight}
\def\UserDefined{\ldots}

% --------------------------------------------------------------------------- %

\subsection{Formatting}
\label{formats}

\DescribeMacro{format=}
A figure or table caption mainly consists of three parts: the caption label,
which says if this object is a `Figure' or `Table' and what number is
associated with it, the caption text itself, which is normally a short
description of contents, and the caption separator which separates the text
from the label.

The \textit{caption format} determines how this information will be presented;
it is specified with the option
\begin{quote}
  |format=|\meta{format name}\quad,
\end{quote}
having the name of the caption format as its argument.

There are two standard caption formats:
%\footnote{You have the option to
%define your own ones, too. See section \ref{declare}:
%\textit{``Do it yourself!''} for details.}

\begin{Options}{\OptionLabel}
  \item[plain]%\NEWdescription{v3.0h}
  Typesets the captions as a normal paragraph.

  \item[hang]
  Indents the caption text, so it will `hang' under the first line of the text.

  \item[\UserDefined]
  Own formats can be defined using |\Declare|\-|Caption|\-|Format|.\LineBreak
  \SeeUserDefined
\end{Options}

\begin{Example}
  An example: Specifying the option
  \begin{quote}
    |format=hang|
  \end{quote}
  yields captions like this:
  \example{format=hang}{\exampletext}
\end{Example}

\pagebreak[3]
\DescribeMacro{indention=}
For both formats (\texttt{plain} and \texttt{hang}) you can setup an extra
indention starting at the second line of the caption. You do this with the
option
\begin{quote}
  |indention=|\meta{amount}\quad.
\end{quote}

Two examples:

\begin{Example}
  \begin{quote}
    |format=plain,indention=.5cm|
  \end{quote}
  \captionsetup{skip=0pt}
  \example{format=plain,indention=.5cm}{\exampletext}
\end{Example}

\begin{Example}
  \begin{quote}
    |format=hang,indention=-0.5cm|
  \end{quote}
  \captionsetup{skip=0pt}
  \example{format=hang,indention=-0.5cm}{\exampletext}
\end{Example}

\bigskip

\pagebreak[3]
\DescribeMacro{labelformat=}
With the option
\nopagebreak[3]
\begin{quote}
  |labelformat=|\meta{label format name}
\end{quote}
\nopagebreak[3]
%\NEWdescription{v3.0e}
you specify how the caption label will be typeset.
There are five standard caption label formats:

\begin{Options}{\OptionLabel}
  \item[default]
  The caption label will be typeset as specified by the document class,
  usually this means the name and the number (like \texttt{simple}).
  (This is the default behaviour.)

  \item[empty]
  The caption label will be empty.
\iffalse
  (This option makes sense when used together with other options like
   \texttt{labelsep=none}.)
\fi

  \item[simple]
  The caption label will be typeset as a name and a number.

  \item[brace]\NEWfeature{v3.1f}
  The caption label will be closed with a single (right) brace.

  \item[parens]
  The number of the caption label will be typeset in parentheses.

  \item[\UserDefined]
  Own label formats can be defined using |\Declare|\-|Caption|\-|Label|\-|Format|.
  \SeeUserDefined
\end{Options}

\begin{Example}
  An example: Using the options
  \begin{quote}
    |format=plain,labelformat=parens,labelsep=quad|
  \end{quote}
  gives captions like this one:
  \example{format=plain,labelformat=parens,labelsep=quad}{\exampletext}
\end{Example}

\medskip

\begin{Note*}
Some environments, e.g.~the |algorithm| environment offered by the
\package{algorithm2e} package, might react allergic to a change of the
caption label format.
\end{Note*}

\bigskip

\pagebreak[3]
\DescribeMacro{labelsep=}
With the option
\begin{quote}
  |labelsep=|\meta{label separator name}
\end{quote}
you specify what caption separator will be used.\footnote{%
  If the caption label or the caption text is empty,
  no separator will be used.}
You can choose one of the following:

\begin{Options}{\OptionLabel}
  \item[none]
  There is no caption separator.
\iffalse
  (This option makes sense when used together with other options like
   \texttt{labelformat=empty}.)
\fi

  \item[colon]
  The caption label and text will be separated by a colon and a space.

  \item[period]
  The caption label and text will be separated by a period and a space.

  \item[space]
  The caption label and text will be separated by a single space.

  \item[quad]
  The caption label and text will be separated by a |\quad|.

  \item[newline]
  The caption label and text will be separated by a line break (|\\|).
  Please note that this separator does not work with all caption
  formats (e.g.~|format=|\x|hang|), and you'll get an error message
  trying such combination of options.

  \item[endash]\NEWfeature{v3.0h}
  The caption label and text will be separated by an en-dash,
  surrounded by spaces (\verb*| -- |).

%\showhyphens{separator}
  \item[\UserDefined]
  Own separators can be defined using |\Declare|\-|Caption|\-|Label|\-|Sep|\-|a|\-|ra|\-|tor|.
  \SeeUserDefined
\end{Options}

Three examples:
\begin{Example}
  \begin{quote}
    |format=plain,labelsep=period|
  \end{quote}
  \captionsetup{skip=0pt}
  \example{format=plain,labelsep=period}{\exampletext}
\end{Example}

\begin{Example}
  \begin{quote}
    |format=plain,labelsep=newline,singlelinecheck=false|
  \end{quote}
  \captionsetup{skip=0pt}
  \example{format=plain,labelsep=newline,singlelinecheck=false}{\exampletext}
\end{Example}

\begin{Example}
  \begin{quote}
    |format=plain,labelsep=endash|
  \end{quote}
  \captionsetup{skip=0pt}
  \example{format=plain,labelsep=endash}{\exampletext}
\end{Example}

\bigskip

\pagebreak[3]
\DescribeMacro{textformat=}\NEWfeature{v3.0l}
With the option
\nopagebreak[3]
\begin{quote}
  |textformat=|\meta{text format name}
\end{quote}
\nopagebreak[3]
you specify how the caption text will be typeset.
There are two standard caption text formats:

\begin{Options}{\OptionLabel}
  \item[empty]
  The caption text will be suppressed.

  \item[simple]
  The caption text will be typeset as it is.

  \item[period]
  The caption text will be followed by a period.

  \item[\UserDefined]
  Own text formats can be defined using |\Declare|\-|Caption|\-|Text|\-|Format|.
  \SeeUserDefined
\end{Options}

% --------------------------------------------------------------------------- %

\PageBreak
\subsection{Justification}
\label{justification}

\DescribeMacro{justification=}
As addition to the caption format you could also specify a
\emph{caption justification}; it is specified with the option
\begin{quote}
  |justification=|\meta{justification name}\quad.
\end{quote}

You can choose one of the following:

\begin{Options}{\OptionLabel}
  \item[justified]
  Typesets the caption as a normal paragraph.

  \item[centering]
  Each line of the caption will be centered.

%  \item[Centering]

  \item[centerlast]
  The last line of each paragraph of the caption text will be centered.

  \item[centerfirst]
  Only the first line of the caption will be centered.

  \item[raggedright]
  Each line of the caption will be moved to the left margin.

\iffalse
  \item[RaggedRight]
  Each line of the caption will be moved to the left margin using
  the command |\RaggedRight| from the \package{ragged2e} package.
\else
  \item[RaggedRight]
  Each line of the caption will be moved to the left margin, too.
  But this time the command |\RaggedRight| of the \package{ragged2e}
  package will be used to achieve this.
  The main difference to |raggedright| is that the word breaking
  algorithm of \TeX\ will work inside captions.\footnote{%
    The need for the \package{ragged2e} package will be detected at
    run-time, therefore you maybe need a second \LaTeX\ run if this
    option is used for the first time.}
\fi

  \item[raggedleft]
  Each line of the caption will be moved to the right margin.

%  \item[RaggedLeft]
%  Each line of the caption will be moved to the right margin using
%  the command |\RaggedLeft| from the \package{ragged2e} package.

%\showhyphens{justification}
  \item[\UserDefined]
  Own justifications can be defined using |\Declare|\-|Caption|\-|Jus|\-|ti|\-|fi|\-|ca|\-|tion|.
  \SeeUserDefined
\end{Options}

Three examples:
\begin{Example}
  \begin{quote}
    |format=plain,justification=centerlast|
  \end{quote}
  \captionsetup{skip=0pt}
  \example{format=plain,justification=centerlast}{\exampletext}
\end{Example}

\begin{Example}
  \begin{quote}
    |format=hang,justification=raggedright|
  \end{quote}
  \captionsetup{skip=0pt}
  \example{format=hang,justification=raggedright}{\exampletext}
\end{Example}

\begin{Example}
  \begin{quote}
    |format=plain,labelsep=newline,justification=centering|
  \end{quote}
  \captionsetup{skip=0pt}
  \example*{format=plain,labelsep=newline,justification=centering}{\exampletext}
\end{Example}

\smallskip

\PageBreak
\DescribeMacro{singlelinecheck=}
The standard \LaTeX\ document classes (\class{article}, \class{report}, and
\class{book}) automatically center a caption if it fits in one single line:

\example*{}{A short caption.}

\INFO
\Thispackage\ adapts this behavior and therefore usually ignores the
justification \& indention you have set with |justification=| \&
|indention=| in such case.
But you can switch this special treatment of such short captions off
with the option
\begin{quote}
  |singlelinecheck=|\meta{bool}\quad.
\end{quote}
Using |false|, |no|, |off| or |0| for \meta{bool} switches the
extra centering off:
\begin{quote}
  |singlelinecheck=false|
\end{quote}
Doing so the above short caption would look like

\begingroup
  \captionsetup{type=figure}
  \ContinuedFloat
  \example*{singlelinecheck=false}{A short caption.}
\endgroup

You switch the extra centering on again by using |true|, |yes|, |on| or |1|
for \meta{bool}. (The default is on.)

% --------------------------------------------------------------------------- %

\subsection{Fonts}
\label{fonts}

\DescribeMacro{font=}
\DescribeMacro{labelfont=}
\DescribeMacro{textfont=}
There are three font options which affects different parts of the caption:
One affecting the whole caption (|font|), one which only affects the caption
label and separator (|label|\-|font|) and at least one which only affects the
caption text (|text|\-|font|).
You set them up using the options
\begin{quote}\begin{tabular}{@{}r@{}ll}
  |font=|      & \marg{font options} & ,\\
  |labelfont=| & \marg{font options} & , and\\
  |textfont=|  & \marg{font options} & ,\\
\end{tabular}\end{quote}
where \meta{font options} is a list of comma separated font options.

And these are the available font options:

\begin{Options}{stretch=\meta{amount}}
  \item[scriptsize]   {\scriptsize Very small size}
  \item[footnotesize] {\footnotesize The size usually used for footnotes}
  \item[small]        {\small Small size}
  \item[normalsize]   {\normalsize Normal size}
  \item[large]        {\large Large size}
  \item[Large]        {\Large Even larger size}
\end{Options}
\vspace{0pt}\pagebreak[3]
\begin{Options}{stretch=\meta{amount}}
  \item[normalfont]   {\normalfont Normal shape \& series \& family}

  \item[up]           {\upshape Upright shape}
  \item[it]           {\itshape Italic shape}
  \item[sl]           {\slshape Slanted shape}
  \item[sc]           {\scshape Small Caps shape}

  \item[md]           {\mdseries Medium series}
  \item[bf]           {\bfseries Bold series}

  \item[rm]           {\rmfamily Roman family}
  \item[sf]           {\sffamily Sans Serif family}
  \item[tt]           {\ttfamily Typewriter family}
\end{Options}
\vspace{0pt}\pagebreak[3]
\begin{Options}{stretch=\meta{amount}}
  \item[singlespacing]  Single spacing \See{\Ref{setspace}}
  \item[onehalfspacing] One-and-a-half spacing \See{\Ref{setspace}}
  \item[doublespacing]  Double spacing \See{\Ref{setspace}}
  \item[stretch=\meta{amount}] |\setstretch|\marg{amount} \See{\Ref{setspace}}
\end{Options}
\vspace{0pt}\pagebreak[3]
\begin{Options}{stretch=\meta{amount}}
  \item[normalcolor]          |\normalcolor|
  \item[color=\meta{colour}]  |\color|\marg{colour}
       {\small(If the \package{color} or \package{xcolor} package is loaded,
        see \Ref{declare} for an example)}
\end{Options}
\vspace{0pt}\pagebreak[3]
\begin{Options}{stretch=\meta{amount}}
  \item[normal] The combination of the options |normal|\-|color|,
                |normal|\-|font|, |normal|\-|size|, and |single|\-|spacing|

  \item[\UserDefined]
  Own font options can be defined using |\Declare|\-|Caption|\-|Font|.
  \SeeUserDefined
\end{Options}

If you use only one of these options you can omit the braces;
e.g., the options
%\begin{quote}
  |font={small}|
%\end{quote}
and
%\begin{quote}
  |font=small|
%\end{quote}
will give the same result.

Three examples:
\begin{Example}
  \begin{quote}
    |font=it,labelfont=bf|
  \end{quote}
  \captionsetup{skip=0pt}
  \example{font=it,labelfont=bf}{\exampletext}
\end{Example}

\begin{Example}
  \begin{quote}
    |labelfont=bf,textfont=it|
  \end{quote}
  \captionsetup{skip=0pt}
  \example{labelfont=bf,textfont=it}{\exampletext}
\end{Example}

\begin{Example}
  \begin{quote}
    |font={small,stretch=0.80}|
  \end{quote}
  \captionsetup{skip=0pt}
  \example{font={small,stretch=0.80}}{\exampletext}
\end{Example}

\medskip

\pagebreak[3]
\DescribeMacro{font+=}
\DescribeMacro{labelfont+=}
\DescribeMacro{textfont+=}
\NEWfeature{v3.1f}
You can also add font options to the current ones, so for example
\begin{quote}
  |\captionsetup{font=small}|\\
  |\captionsetup{font+=it}|
\end{quote}
is identical to
\begin{quote}
  |\captionsetup{font={small,it}}|
\end{quote}

% --------------------------------------------------------------------------- %

\subsection{Margins and further paragraph options}
\label{margins}

\DescribeMacro{margin=}
\DescribeMacro{width=}
For all captions you can specify \emph{either} an extra margin \emph{or}
a fixed width:~\footnote{Only fixed widths are supported here; if you are
looking for a way to limit the width of the caption to the width of the
figure or table, please take a look at the \package{floatrow}\cite{floatrow}
or \package{threeparttable}\cite{threeparttable} package.}
\begin{quote}\begin{tabular}{@{}r@{}ll}
  |margin=| & \meta{amount} & \emph{--or--}\\
  |margin=| & |{|\meta{left amount}|,|\meta{right amount}|}| & \emph{--or--}\\
  |width=|  & \meta{amount} & \\
\end{tabular}\end{quote}
\NEWfeature{v3.1}
If you specify just one \meta{amount} for the margin, it will be used for
both, the left and right margin, e.g.~|margin=|\x|10pt| is equivalent to
|margin=|\x|{10pt,10pt}|.
In two-side documents the left and right margin will be swapped on even pages.
\DescribeMacro{oneside}
\DescribeMacro{twoside}
To prevent this you can specify the option |oneside| additionally,
e.g.~|\caption|\-|setup{margin=|\x|{0pt,10pt},|\x|oneside}|.\par
But if you are specifying a width, then both, the left and the right margin,
will have the same amount.

Three examples will illustrating this:
\begin{Example}
  \begin{quote}
    |margin=10pt|
  \end{quote}
  \captionsetup{skip=0pt}
  \example{margin=10pt}{\exampletext}
\end{Example}

\begin{Example}
  \begin{quote}
    |margin={1cm,0cm}|
  \end{quote}
  \captionsetup{skip=0pt}
  \example{margin={1cm,0cm}}{\exampletext}
\end{Example}

\begin{Example}
  \begin{quote}
    |width=.75\textwidth|
  \end{quote}
  \captionsetup{skip=0pt}
  \example{width=.75\textwidth}{\exampletext}
\end{Example}

\begin{Note}
When the caption is placed beside the contents (for example in a |SC|\-|figure|
environment offered by the \package{sidecap} package\cite{sidecap}) or the
figure is an in-text figure (for example in a |wrap|\-|figure| environment
offered by the \package{wrapfig} package\cite{wrapfig}), the margin setting
will be automatically reset to |0pt| at the very beginning of the environment.
But if you really want to setup an extra margin for these environments, you can
do so by setting this margin either inside the environment itself, of by
specifying a margin for this particular environment,
e.g.~|\caption|\-|setup[SC|\-|figure]{margin=|\x|10pt}|.
\end{Note}

\begin{Expert}
\DescribeMacro{margin*=}\NEWfeature{v3.1}
There is also a starred variant of the |margin=| option, |margin*=|, which only
changes the margin if no width was set.
\end{Expert}

\begin{Expert}
\DescribeMacro{minmargin=}
\DescribeMacro{maxmargin=}\NEWfeature{v3.1}
You can also set a minimum or maximum margin amount. This can be useful for
limiting the margin amount in smaller environments, e.g.~|minipage|s.
For example the \SmF\ document classes limit the margin amount to
|maxmargin=|\x|0.1\linewidth|. (See \Ref{SMF})
\end{Expert}

\medskip

\pagebreak[3]
\DescribeMacro{parskip=}
This option is useful for captions containing more than one paragraph.
It specifies the extra vertical space inserted between them:
\begin{quote}
  |parskip=|\meta{amount}
\end{quote}
One example:
\begin{Example}
  \begin{quote}
    |margin=10pt,parskip=5pt|
  \end{quote}
  \captionsetup{skip=0pt}
  \example{margin=10pt,parskip=5pt}{%
    First paragraph of the caption. This one contains some test, just to
    show how these options affect the layout of the caption.\par
    Second paragraph of the caption. This one contains some text, too, to
    show how these options affect the layout of the caption.}
\end{Example}

\pagebreak[3]
\DescribeMacro{hangindent=}
The option
\begin{quote}
  |hangindent=|\meta{amount}
\end{quote}
is for setting up a hanging indention starting from the second line of each
paragraph. If the caption contains just a single paragraph, using this option
leads to the same result as the option |indention=|\meta{amount} you already
know about.
But if the caption contains multiple paragraphs you will notice the difference:

\begin{Example}
  \begin{quote}
    |format=hang,indention=-.5cm|
  \end{quote}
  \captionsetup{skip=0pt}
  \example{format=hang,indention=-.5cm}{%
    First paragraph of the caption. This one contains some test, just to
    show how these options affect the layout of the caption.\par
    Second paragraph of the caption. This one contains some text, too, to
    show how these options affect the layout of the caption.}
\end{Example}

\begin{Example}
  \begin{quote}
    |format=hang,hangindent=-.5cm|
  \end{quote}
  \captionsetup{skip=0pt}
  \example{format=hang,hangindent=-.5cm}{%
    First paragraph of the caption. This one contains some test, just to
    show how these options affect the layout of the caption.\par
    Second paragraph of the caption. This one contains some text, too, to
    show how these options affect the layout of the caption.}
\end{Example}

\begin{Note}
If your caption contains more than one paragraph, you have
to specify an alternative caption for the list-of-figures using the
optional argument of |\caption| or |\caption|\-|of|, otherwise you will
get an error message.
\end{Note}

% --------------------------------------------------------------------------- %

\PageBreak
\subsection{Styles}
\label{style}

\DescribeMacro{style=}
A suitable combination of caption options is called \emph{caption style}.
You can compare them more or less to page styles which you set up with
|\page|\-|style|;
the caption style provides all settings for a whole caption layout.

You switch to an already defined caption style with the option
\begin{quote}
  |style=|\meta{style name}\quad.
\end{quote}
\Thispackage\ pre-defines two styles: |base| and |default|.

\NEWfeature{v3.1}
The |base| style puts all options you already know about to values reflecting
the look of the captions when using one of the base \LaTeX\ document classes
\class{article}, \class{report}, and \class{book}.
This means that specifying the option
\begin{quote}
  |style=base|
\end{quote}
has the same effect as specifying all these options:
\begin{quote}
  |format=plain,labelformat=default,labelsep=colon,|\\
  |justification=justified,font={},labelfont={},|\\
  |textfont={},margin=0pt,indention=0pt|\\
  |parindent=0pt,hangindent=0pt,singlelinecheck=true|
\end{quote}
\begin{Annotation*}
But |justification=centering,indention=0pt| will be set if
the caption fits into a single line.
\end{Annotation*}

In contrast the |default| style follows the default values, reflecting
the look of the captions given by the document class you actually use.
This style is selected by default and represents these options:
\begin{quote}
  |format=default,labelformat=default,labelsep=default,|\\
  |justification=default,font=default,labelfont=default,|\\
  |textfont=default,margin=0pt,indention=0pt|\\
  |parindent=0pt,hangindent=0pt,singlelinecheck=true|
\end{quote}
\begin{Annotation*}
But again |justification=centering,indention=0pt| will be set if
the caption fits into a single line.
\end{Annotation*}

So if you use one of the base \LaTeX\ document classes \class{article},
\class{report}, or \class{book}, both caption styles, |base| and |default|,
point to (nearly) the same settings.

\begin{Note}
Own caption styles can be defined using |\Declare|\-|Caption|\-|Style|.\LineBreak
\SeeUserDefined
\end{Note}

% --------------------------------------------------------------------------- %

\subsection{Skips}
\label{skips}

\DescribeMacro{skip=}\NEWfeature{v3.0d}
The vertical space between the caption and the figure or table contents is
controlled by the option
\begin{quote}
  |skip=|\meta{amount}\quad.
\end{quote}
The standard \LaTeX\ document classes \class{article}, \class{report} and
\class{book} preset it to |skip=|\x|10pt|, but other document classes may
use a different amount.

\bigskip

\pagebreak[3]
\DescribeMacro{position=}
The |\caption| command offered by \LaTeX{} has a design flaw:
The command does not know if it stands on the beginning of the figure or
table, or at the end. Therefore it does not know where to put the space
separating the caption from the content of the figure or table.
While the standard implementation always puts the space above the caption
in floating environments (and inconsistently below the caption in
|long|\-|table|s), the implementation offered by this package is more
flexible:
By giving the option
\begin{quote}
  |position=top|\quad or\quad |position=above|
\end{quote}
it's assumed that the caption is standing at the \emph{top} of the
environment and therefore the space setup with |skip=|\x\meta{amount}
is placed below the caption.
(Please note that |position=|\x|top| does \emph{NOT} mean that the
caption is actually placed at the top of the figure or table.
Instead the caption is usually placed where you place the |\caption|
command.)
But with
\begin{quote}
  |position=bottom|\quad or\quad |position=below|
\end{quote}
it's assumed that the caption is standing at the \emph{bottom} of the
environment and therefore the space is placed above the caption.
And finally with
\begin{quote}
  |position=auto|\quad {\small(which is the default setting)}
\end{quote}
\thispackage\ tries its best to determine the actual position of
the caption on its own. Please note that while this is successfully in
most cases, it could give wrong results under rare circumstances.

\medskip

\pagebreak[3]
\DescribeMacro{figureposition=}%\NEWfeature{v3.0a}
\DescribeMacro{tableposition=}%\NEWfeature{v3.0a}
The |position| option is especially useful when used together with the
optional argument of the |\caption|\-|setup| command.
\See{also \Ref{captionsetup}}\par
For example
\begin{quote}
  |\captionsetup[table]{position=above}|
\end{quote}
causes all captions within tables to be treated as captions \emph{above}
the table (regarding spacing around it).
Because this is a very common setting, \thispackage\ offers
the abbreviating options |figure|\-|position=|\x\meta{pos} and
|table|\-|position=|\x\meta{pos}, e.g.
\begin{quote}
  |\usepackage[|\ldots|,tableposition=top]{caption}|
\end{quote}
is equivalent to
\begin{quote}
  |\usepackage[|\ldots|]{caption}|\\
  |\captionsetup[table]{position=top}|
\end{quote}

\medskip

\INFO
Please note that the options |skip=|, |position=|, |figure|\-|position=|,
and |table|\-|position=| do not always have an effect.
Since it's a matter of the document class to supply the environments
|figure| and |table|, it could use its very own spacing, and could
decide for itself if the caption will be typeset as ``top'' or ``bottom''
caption.
For example the \KOMAScript\ document classes support the |skip=|
setting, but will always typeset |figure| captions as ``bottom'' captions,
and |table| captions are dependent on the global option
|table|\-|captions|\-|above| resp.\ |table|\-|captions|\-|below|.
\See{\Ref{KOMA}}

Furthermore some packages control the behavior of the spacing above and
below the caption for themself, e.g.~the \package{float}, the
\package{floatrow}, and the \package{supertabular} package.

\bigskip

\begin{Expert*}
Internally the skip between caption and contents is represented by
|\above|\-|caption|\-|skip| (which is always set above the caption in
\LaTeX{}s implementation).
But there is a second value |\below|\-|caption|\-|skip| (usually set to
|0pt| by default) which is set below the caption in \LaTeX{}s implementation,
but on the other side than |\above|\-|caption|\-|skip| by this package.
So technically speaking, this package swaps the meaning of these two skips
when |position=|\x|top| is set.
Please note that there are several packages around which do the same trick
(like the \package{ftcap}, the \package{nonfloat}, and the \package{topcap}
package), so the usage of the \package{caption} option |position=| is not
supported if one of these packages will be used, too.
\end{Expert*}

% --------------------------------------------------------------------------- %

\subsection{Lists}
\label{lists}

\DescribeMacro{list=}\NEWfeature{v3.1}
The |\caption| command usually places an entry in the List of Figures
resp.~List of Tables. You can either suppress that individually by giving an
empty optional argument to |\caption| {\small(see \Ref{caption})}, or
programmatically by saying
\begin{quote}
  |list=no|\quad(or any other boolean value instead of |no|)\quad.~\footnote{%
  Please note that the \package{subfig} package\cite{subfig} is not supporting
  this option, it uses the counters \texttt{lofdepth} \& \texttt{lotdepth} for
  this purpose instead.}
\end{quote}

\pagebreak[3]
\DescribeMacro{listformat=}\NEWfeature{v3.1}
With the option
\nopagebreak[3]
\begin{quote}
  |listformat=|\meta{list format name}
\end{quote}
\nopagebreak[3]
you can specify how the |figure| or |table| number will be typeset within
the List of Figures resp.~List of Tables.
There are five standard caption list formats:

\begin{Options}{\OptionLabel}
  \item[empty]
  No number will be typeset.

  \item[simple]
  The number will be typeset with label prefix.

  \item[parens]
  The number will be typeset in parentheses, with label prefix.

  \item[subsimple]
  Same as |simple|, but without label prefix. (default)

  \item[subparens]
  Same as |parens|, but without label prefix.

  \item[\UserDefined]
  Own list formats can be defined using |\Declare|\-|Caption|\-|List|\-|Format|.
  \SeeUserDefined
\end{Options}

The prefix ($=$|\p@figure| resp.~|\p@table|), which runs ahead of the
number ($=$|\the|\-|figure| resp.~|\the|\-|table|) in lists (e.g.~List of
Figure/Table) and references, is usually empty,
so the list formats |simple| and |subsimple| gives equal results;
same with |parens| and |subparens|.
But this can be different for sub-figures or sub-tables listed in the List of
Figures resp.~List of Tables, or when the label prefix is redefined for a
different purpose.\footnote{%
  Sub-figures and sub-tables can be typeset using the \package{subcaption}
  or \package{subfig} package.}

% --------------------------------------------------------------------------- %

\subsection{Names}
\label{names}

\DescribeMacro{name=}\NEWfeature{v3.1f}
The option
\begin{quote}
  |name=|\meta{name}
\end{quote}
changes the name of the \emph{current} environment.
This could be useful in conjunction with the optional argument of
|\caption|\-|setup|, e.g.
\begin{quote}
  |\captionsetup[wrapfigure]{name=Fig.}|
\end{quote}
changes the name to ``Fig.'' for all |wrap|\-|figure|s (while all the other
figure captions will still have ``Figure'' as name).

% --------------------------------------------------------------------------- %

\subsection{Types}
\label{types}

\DescribeMacro{type=}\NEWfeature{v3.0d}
The |\caption| command can typeset captions for different types,
e.g.~|figure| and |table|. If you try to use the |\caption| command outside
these environments you will get an error message, because it does not know what
kind of caption do you want to have here. But in such situations you can set
the caption type manually with
\begin{quote}
  |type=|\meta{float type}
\end{quote}
prior to the usage of the |\caption| command (and other commands like
|\Continued|\-|Float|, or |\sub|\-|caption|\-|box| offered by the
\package{subcaption} package, or |\sub|\-|float| offered by the
\package{subfig} package\cite{subfig}),
for example within an non-floating environment like |minipage|:
\begin{quote}
  |\noindent\begin{minipage}{\textwidth}|\\
  |  \captionsetup{type=figure}|\\
  |  \subfloat{|\ldots|}|\\
  |  |\ldots\\
  |  \caption{|\ldots|}|\\
  |\end{minipage}|
\end{quote}

\begin{Expert}
There is also a starred variant of this option, |type*=|\meta{float type},
which behaves different than |type=|\meta{float type} if the
\package{hyperref} package\cite{hyperref} is loaded: While |type=| sets an
hyperlink anchor (if |hypcap=|\x|true| is set), |type*=| does not.
(See also \Ref{hyperref}\,)
\end{Expert}
\begin{Expert}
\emph{\DefaultNoteText:} Please don't re-define the internal macro |\@captype|
for yourself, like suggested by some documentations, always use
|\caption|\-|setup{type=|\x\ldots|}| instead.
\end{Expert}

\medskip

Own float types can be defined with
 |\Declare|\-|Floating|\-|Environment| offered by the \package{newfloat} package,
 |\new|\-|float| offered by the \package{float} package\cite{float}, or
 |\Declare|\-|New|\-|Float|\-|Type| offered by the \package{floatrow} package\cite{float}.

\medskip

\INFO % \NEWdescription{v3.1}
Please note that you should use the option |type=| only
\emph{inside} boxes or environments (like |\par|\-|box| or |mini|\-|page|),
at best one where no page break could happen between contents and caption.
Furthermore some visual side-effects (e.g.~mixed-up figure and table settings
regarding captions) could occur without using a box or environment,
therefore a warning message will be issued if you try to do so.\footnote{%
You only get this warning message if you use \eTeX\ as underlying \TeX\ engine.}

% --------------------------------------------------------------------------- %

\clearpage
\section{Commands}

\subsection{Typesetting captions}
\label{caption}
\label{captionlistentry}

\DescribeMacro{\caption}
The command
\begin{quote}
  |\caption|\oarg{list entry}\marg{heading}
\end{quote}
typesets the caption inside a floating environment like |figure| or |table|.
Well, you already know this, but \thispackage\ offers an extension:
If you leave the argument \meta{list entry} empty, no entry in the list of
figures or tables will be made. For example:
\begin{quote}
  |\caption[]{A figure without list entry.}|
\end{quote}

\begin{Expert}
Please remember that the \meta{heading} is a so-called \emph{moving} argument,
if no \meta{list entry} has been given. But if a \meta{list entry} is given,
this argument is moving instead.
``\emph{Moving} argument'' means that the argument will be written to the
list-of file, make it appearing in the ``List of Figures'' resp.
``List of Tables'', too.
\emph{Moving} arguments are not allowed to contain \emph{fragile} commands,
everything must be \emph{robust}, otherwise the argument could get
\emph{broken}, resulting in strange errors at the next \LaTeX\ run.
Some \emph{fragile} commands could be \emph{protected} by a leading |\protect|,
own definitions could get defined with |\Declare|\-|Robust|\-|Command| instead of
|\new|\-|command| to make them \emph{robust}.

An example: |\caption{${}^{137}_{\phantom{1}55}$Cs}| will cause errors since
|\phantom| is \emph{fragile}.
So we have either have to use the optional argument \meta{list entry}
(e.g.~|\caption[${}^|\x|{137}_|\x|{55}$|\x|Cs]|\x|{${}^|\x|{137}_|\x|{\phantom{1}55}$|\x|Cs}|)
or add |\protect| to prevent the |\phantom| command from getting \emph{broken}:
|\caption{${}^|\x|{137}_|\x|{|\textcolor{blue}{\cs{protect}}|\phantom{1}55}$|\x|Cs}|.

But sometimes even this is not sufficient. The reason behind is the so-called
single-line-check: It puts the \meta{heading} into a horizontal box to determine
the width of the caption, and this could cause error messages, too.
An example:
|\caption{A| |scheme.| |\[V_{C}| |\sim| |\left| |\{| |\begin{array}{cc}|
|E_{g}| |&| |\textrm{p-n}| |\\| |e\phi_{B}| |&| |\textrm{M-S}| |\end{array}|
|\right.| |\]}|.
Using |\caption[A| |scheme]{|\ldots|}| is not sufficient here, it's still
leading to errors. (``\texttt{Missing \$ inserted.}'')
So we have to put a |\caption|\-|setup{single|\-|line|\-|check=|\x|off}|
just in front of the |\caption| command additionally.

For more information about \emph{moving} arguments and \emph{fragile} \&
\emph{robust} commands, take a closer look at your \LaTeX\ manual or visit
\url{http://www-h.eng.cam.ac.uk/help/tpl/textprocessing/teTeX/latex/latex2e-html/fragile.html}.
\end{Expert}

\medskip

\pagebreak[3]
\DescribeMacro{\caption*}
The \package{longtable} package defines the command |\caption*| which
typesets the caption without label and without entry in the list of tables.
An example:
\begin{quote}
  |\begin{longtable}{|\ldots|}|\\
  |  \caption*{A table}\\|\\
  |  |\ldots\\
  |\end{longtable}|
\end{quote}
looks like
\DeleteShortVerb{\|}
\vskip\intextsep
\begin{minipage}{\linewidth}
  \captionsetup{type=table,position=t}
  \caption*{A table}
  \centering
  \begin{tabular}{r|rr}
      & x & y \\\hline
    a & 1 & 2 \\
    b & 3 & 4 \\
  \end{tabular}
\end{minipage}
\vskip\intextsep
\MakeShortVerb{\|}
\Thispackage\ offers this feature, too, so you can use this command now within
every floating environment like |figure| or |table|, like:
\begin{quote}
  |\begin{table}|\\
  |  \caption*{A table}|\\
% |  \begin{tabular}{|\ldots|}|\\
% |    |\ldots\\
% |  \end{tabular}|\\
  |  |\ldots\\
  |\end{table}|
\end{quote}

\pagebreak[3]
\DescribeMacro{\captionof}
\DescribeMacro{\captionof*}
Sometimes you want to typeset a caption \emph{outside} a floating environment,
putting a figure within a non-floating |mini|\-|page| for instance.
For this purpose \thispackage\ offers the command
\begin{quote}
  |\captionof|\marg{float type}\oarg{list entry}\marg{heading}\quad.
\end{quote}
Note that the first argument, the \meta{float type}, is mandatory here, because
the |\captionof| command needs to know which name to put into the caption label
(e.g. ``Figure'' or ``Table'') and in which list to put the contents entry.
An example:
\begin{quote}
% |\captionof{figure}{A figure}|\\
  |\captionof{table}{A table}|
\end{quote}
typesets captions like this:
\begin{Example}
% \begingroup
%   \captionof{figure}{A figure}
% \endgroup
  \captionsetup{position=top}
% \begingroup
    \captionof{table}{A table}
% \endgroup
\end{Example}

The star variant |\caption|\-|of*| has the same behavior as the |\caption*|
command:
It typesets the caption without label and without entry to the list of figures
or tables.

\INFO % \NEWdescription{v3.1}
Since |\caption|\-|of| uses the option |type| internally, the same
restrictions as for the |type| option apply here, so you should use both
|\caption|\-|of| and |\caption|\-|of*| only \emph{inside} boxes or
environments, too.
\See{\Ref{types}}

\bigskip

\pagebreak[3]
\DescribeMacro{\captionlistentry}\NEWfeature{v3.1}
Under certain circumstances it could be useful to make a list-of-figure
(or table) entry on its own.
This could be achieved with
\begin{quote}
  |\captionlistentry|\oarg{float type}\marg{list entry}\quad.
\end{quote}

One example: It's quite easy to have a |long|\-|table| with captions above the
contents and a single list entry which points to the first page of the table:
\begin{quote}
  |\begin{longtable}{|\ldots|}|\\
  |  \caption{|\ldots|}\\|\\
  |\endfirsthead|\\
  |  \caption[]{|\ldots|}\\|\\
  |\endhead|\\
  |  |\ldots
\end{quote}
But since the \package{longtable} package does not offer an
|\end|\-|first|\-|foot| command, you cannot easily have captions \emph{below}
the table contents and a single list entry which points to the first page of
the table.
Here is were the |\caption|\-|list|\-|entry| command could be used:
\begin{quote}
  |\begin{longtable}{|\ldots|}|\\
  |  \caption[]{|\ldots|}\\|\\
  |\endfoot|\\
  |  \captionlistentry{|\ldots|}|\\
  |  |\ldots
\end{quote}

\begin{Annotation*}
Another example can be found in \Ref{examples}.
\end{Annotation*}

\begin{Expert}
There is also a starred variant, |\caption|\-|list|\-|entry*|, which does not
increment the \meta{float type} counter.
(Note that inside |long|\-|table| environments |\caption|\-|list|\-|entry|
never increments the |table| counter. See also \Ref{longtable}.)
\end{Expert}

\begin{Expert}
Please note that \meta{list entry} is a \emph{moving} argument, so everything
it contains must be \emph{robust}. (See also explanation of |\caption|)
\end{Expert}

\subsection{Setting options}
\label{captionsetup}

\DescribeMacro{\captionsetup}
We already know the |\caption|\-|setup| command (see \Ref{usage}),
but this time we get enlighten about its optional argument
\meta{float type}.

Remember, the syntax of this command is
\begin{quote}
  |\captionsetup|\oarg{float type}\marg{options}\quad.
\end{quote}
If a \meta{float type} gets specified, all the \meta{options} don't
change anything at this time. Instead they only get marked for a later use,
when a caption inside of a floating environment of the particular type
\meta{float type} gets typeset.
For example
\begin{quote}
  |\captionsetup[figure]|\marg{options}
\end{quote}
forces captions within a |figure| environment to use the given \meta{options}.

Here comes an example to illustrate this:
\begin{quote}
  |\captionsetup{font=small}|\\
  |\captionsetup[figure]{labelfont=bf,textfont=it}|
\end{quote}
gives captions like this:
\begin{Example}
  \captionsetup{font=small}
  \captionsetup[figure]{labelfont=bf,textfont=it}
  \begingroup
    \captionof{figure}{A figure}
  \endgroup
  \captionsetup{position=top}
  \begingroup
    \captionof{table}{A table}
  \endgroup
\end{Example}
As you see the command |\caption|\-|setup[figure]{|\ldots|}| only changes
the look of the |figure| caption labels, not touching the other ones.

As \meta{float type} you can usually give one of these two only:
|figure| or |table|.
But as we will see later some \LaTeX\ packages (like the \package{floatrow},
\package{longtable}, and \package{sidecap} package for example) and also
this package offer additional environments with captions and these two
commands can also be used with them.
\See{\Ref{declare} and \Ref{packages}}

\begin{Expert}
\NEWfeature{v3.1}
There is also a starred variant of |\caption|\-|setup|:
\begin{quote}
  |\captionsetup*|\oarg{float type}\marg{options}
\end{quote}
While the non-starred variant can give you warnings---for example if the
\meta{options} are actually not used throughout the document
(e.g.~a |\caption|\-|setup[table]{font=|\x|sf}| without a |table|)---the
starred variant will not.
\iffalse
This can get quite useful when setting generic options which are independent
of a specific document.
\fi
\end{Expert}

\medskip

\pagebreak[3]
\DescribeMacro{\clearcaptionsetup}
If you want to get rid of these parameters marked for an automatic use within
a particular environment you can use the command
\begin{quote}
  |\clearcaptionsetup|\oarg{option}\marg{float type}\quad.
\end{quote}

For example |\clearcaptionsetup{figure}| would clear all the extra handling
for figures in the example above:
\begin{quote}
  |\captionsetup{font=small}|\\
  |\captionsetup[figure]{labelfont=bf,textfont=it}|\\
  \ldots\\
  |  \caption{A figure}|\\
  \ldots\\
  |\clearcaptionsetup{figure}|\\
  \ldots\\
  |  \caption{A figure}|\\
  \ldots
\end{quote}
\begin{Example}
  \captionsetup{font=small}
  \captionsetup[figure]{labelfont=bf,textfont=it}
  \begingroup
    \captionof{figure}{A figure}
  \endgroup
  \captionsetup{position=top}
  \clearcaptionsetup{figure}
  \begingroup
    \captionof{figure}{A figure}
  \endgroup
\end{Example}

\pagebreak[3]
\NEWfeature{v3.1}
If an optional argument \meta{option} is given, only the settings regarding
this particular \meta{option} are cleared.\footnote{You can only specify
\emph{one} option here, not a list of options.
If you want to clean more than one option, you need to use more than one
\cs{clear}\-\texttt{caption}\-\texttt{setup}.}
While the example above not only clears the options
|labelfont=bf,|\x|textfont=it|
for figures (but all options for figures instead),
this one would only clear the |labelfont=bf| setting, leaving all other
settings for figures intact:
\begin{quote}
  |\captionsetup{font=small}|\\
  |\captionsetup[figure]{labelfont=bf,textfont=it}|\\
  \ldots\\
  |  \caption{A figure}|\\
  \ldots\\
  |\clearcaptionsetup[labelfont]{figure}|\\
  \ldots\\
  |  \caption{A figure}|\\
  \ldots
\end{quote}
\begin{Example}
  \captionsetup{font=small}
  \captionsetup[figure]{labelfont=bf,textfont=it}
  \begingroup
    \captionof{figure}{A figure}
  \endgroup
  \captionsetup{position=top}
  \clearcaptionsetup[labelfont]{figure}
  \begingroup
    \captionof{figure}{A figure}
  \endgroup
\end{Example}
\begin{Expert}
Analogous to |\caption|\-|setup*| there is also a starred form
|\clear|\-|caption|\-|setup*| which suppresses warnings if the given
\meta{option} was not setup for the specified \meta{float type}.
\end{Expert}

\medskip

\pagebreak[3]
\DescribeMacro{\showcaptionsetup}
For debugging purposes the command
\nopagebreak[3]
\begin{quote}
  |\showcaptionsetup|\marg{float type}
\end{quote}
\nopagebreak[3]
is offered. It generates a log file entry, showing the given options
for the specified \meta{float type}. For example
\begin{quote}
  |\captionsetup[figure]{labelfont=bf,textfont=it}|\\
  |\showcaptionsetup{figure}|
\end{quote}
gives the info:
\begin{quote}\small
  |Caption Info: Option list on `figure'|\\
  |Caption Data: {labelfont=bf,textfont=it} on input line 5.|
\end{quote}

\pagebreak[3]
\subsection{Continued floats}
\label{ContinuedFloat}

\DescribeMacro{\ContinuedFloat}
Sometimes you want to split figures or tables without giving them
their own reference number. This is what the command
\begin{quote}
  |\ContinuedFloat|
\end{quote}
is for; it should be used as first command inside the floating environment.
It prevents the incrementation of the relevant counter (usually done by
|\caption|), so a figure or table containing a |\Continued|\-|Float| inside
gets the same reference number as the figure or table before.
An example:
\begin{quote}
  |\begin{table}|\\
  |  \caption{A table}|\\
  |  |\ldots\\
  |\end{table}|\\
  \ldots\\
  |\begin{table}\ContinuedFloat|\\
  |  \caption{A table (cont.)}|\\
  |  |\ldots\\
  |\end{table}|
\end{quote}
gives the following result:
\begin{Example}
  \captionsetup{type=table,position=b,skip=0pt}
  \caption{A table}
  \centerline{\ldots}
\end{Example}
\begin{Example}
  \captionsetup{type=table,position=b,skip=0pt}
  \ContinuedFloat
  \caption{A table (cont.)}
  \centerline{\ldots}
\end{Example}

\bigskip

\NEWfeature{v3.1}
Furthermore the |\Continued|\-|Float| command executes options associated with
the type name ``|Continued|\-|Float|''. For example this can be used to switch
to a different label format for continued figures or tables, as shown here:
\begin{quote}
  |\DeclareCaptionLabelFormat{continued}{Continued #1~#2}|\\
  |\captionsetup[ContinuedFloat]{labelformat=continued}|\\
  \ldots\\
  |\begin{table}\ContinuedFloat|\\
  |  \caption{A table}|\\
  |  |\ldots\\
  |\end{table}|
\end{quote}
\begin{Example}
  \captionsetup{type=table,position=b,skip=0pt}
  \captionsetup[ContinuedFloat]{labelformat=continued1}
  \ContinuedFloat
  \caption{A table}\label{continued1}
  \centerline{\ldots}
\end{Example}
\smallskip
\See{\Ref{declare} for an explanation of
     \cs{Declare}\-\texttt{Caption}\-\texttt{Label}\-\texttt{Format}.}

\medskip

There is also a \LaTeX\ counter called |Continued|\-|Float| which could be
used for own purposes.
For ordinary (floating) environments it's set to zero, to one for the first
continued float, to two inside the second one, and so on.
So every |\Continued|\-|Float| increments this counter and a floating
environment without |\Continued|\-|Float| command resets this counter to zero.
An example:
\begin{quote}
  |\DeclareCaptionLabelFormat{cont}{#1~#2\alph{ContinuedFloat}}|\\
  |\captionsetup[ContinuedFloat]{labelformat=cont}|\\
  \ldots\\
  |\begin{table}\ContinuedFloat|\\
  |  \caption{A table}|\\
  |  |\ldots\\
  |\end{table}|
\end{quote}
\begin{Example}
  \captionsetup{type=table,position=b,skip=0pt}
  \captionsetup[ContinuedFloat]{labelformat=continued2}
  \ContinuedFloat
  \caption{A table}\label{continued2}
  \centerline{\ldots}
\end{Example}
\medskip
A reference to this table would still result in the output
``\autoref{continued2}'' since only the caption label format was changed.
If you would like to use the |Continued|\-|Float| counter for the references,
too, you could redefine the command |\the|\-|Continued|\-|Float| instead,
which will be appended to the figure or table counter automatically in
continued floats and is preset to be empty.
\begin{quote}
  |\renewcommand\theContinuedFloat{\alph{ContinuedFloat}}|\\
  \ldots\\
  |\begin{table}\ContinuedFloat|\\
  |  \caption{A table}|\\
  |  |\ldots\\
  |\end{table}|
\end{quote}
\begin{Example}
  \captionsetup{type=table,position=b,skip=0pt}
  \renewcommand\theContinuedFloat{\alph{ContinuedFloat}}
  \ContinuedFloat
  \caption{A table}\label{continued3}
  \centerline{\ldots}
\end{Example}
\medskip
A reference to that table would result in the output ``\autoref{continued3}''.

\pagebreak[3]
\DescribeMacro{\ContinuedFloat*}
Suppose you want to start the first figure or table of such a series with a
kind of ``Figure 7a'' and not with ``Figure 7'' (and the second one with
``Figure 7b'' instead of ``Figure 7a'').
This is possible, too, by using the starred variant |\Continued|\-|Float*|
which---just like |\Continued|\-|Float|---executes options associated with the
type name ``|Continued|\-|Float|'' and increments the \LaTeX\ counter
|Continued|\-|Float|, \emph{but} marks the first figure or table of a series
instead:
\begin{quote}
  |\renewcommand\theContinuedFloat{\alph{ContinuedFloat}}|\\
  \ldots\\
  |\begin{figure}\ContinuedFloat*|\\
  |  |\ldots\\
  |  \caption{First figure of a series}|\\
  |\end{figure}|\\
  \ldots\\
  |\begin{figure}\ContinuedFloat|\\
  |  |\ldots\\
  |  \caption{Second figure of a series}|\\
  |\end{figure}|\\
  \ldots\\
  |\begin{figure}\ContinuedFloat|\\
  |  |\ldots\\
  |  \caption{Third figure of a series}|\\
  |\end{figure}|
\end{quote}
\begin{Example}
  \captionsetup{type=figure,position=b,skip=0pt}
  \renewcommand\theContinuedFloat{\alph{ContinuedFloat}}
  \begingroup
    \ContinuedFloat*
    \centerline{\ldots}
    \caption{First figure of a series}
  \endgroup
  \begingroup
    \ContinuedFloat
    \centerline{\ldots}
    \caption{Second figure of a series}
  \endgroup
  \begingroup
    \ContinuedFloat
    \centerline{\ldots}
    \caption{Third figure of a series}
  \endgroup
\end{Example}
\smallskip
\Note{Unfortunately \cs{ContinuedFloat*} is not available if the
      \package{subfig} package\cite{subfig} is loaded.}

\pagebreak[3]
\subsubsection*{A note about longtables}
If you want to have a different caption label in |long|\-|table|s (offered
by the \package{longtable} package\cite{longtable}) after a page break,
this can \emph{not} be archived by using |\Continued|\-|Float|,
but instead you could write something like:
\begin{quote}
  |\DeclareCaptionLabelFormat{continued}{Continued #1~#2}|\\
  \ldots\\
  |\begin{longtable}{|\ldots|}|\\
  |  \caption{A longtable}\\|\\
  |\endfirsthead|\\
  |  \captionsetup{labelformat=continued}|\\
  |  \caption[]{A longtable}\\|\\
  |\endhead|\\
  |  |\ldots\\
  |\end{longtable}|
\end{quote}

% --------------------------------------------------------------------------- %

\clearpage
\section{Own enhancements}
\label{declare}

A family of commands is provided to allow users to define their own formats
etc.
This enables information on separators, justification, fonts, and styles to
be associated with a name and kept in one place.
(These commands need to appear in the document preamble,
this is the part between |\document|\-|class| and |\begin{document}|.)

\pagebreak[3]
\DescribeMacro{\DeclareCaption-\\Format}
You can define your own caption formats using the command
\begin{quote}
  |\DeclareCaptionFormat|\marg{name}\marg{code using \#1, \#2 and \#3}\mbox{\quad.}
\end{quote}
At usage the system replaces \#1 with the caption label, \#2 with the
separator and \#3 with the text. So the standard format |plain| is
pre-defined by \thispackage\ as
\begin{quote}
  |\DeclareCaptionFormat{plain}{#1#2#3\par}|\quad.
\end{quote}
\begin{Expert}
There is also a starred variant, |\Declare|\-|Caption|\-|Format*|,
which causes the code being typeset in \TeX s vertical (instead of
horizontal) mode, but does not support the |indention=| option.
\end{Expert}

\pagebreak[3]
\DescribeMacro{\DeclareCaption-\\LabelFormat}
Likewise you can define your own caption label formats:
\begin{quote}
  |\DeclareCaptionLabelFormat|\marg{name}\marg{code using \#1 and \#2}
\end{quote}
At usage \#1 gets replaced with the name (e.g. ``figure'') and \#2
gets replaced with the reference number (e.g. ``12'').
An example:
\begin{quote}
  |\DeclareCaptionLabelFormat{bf-parens}{(\textbf{#2})}|\\
  |\captionsetup{labelformat=bf-parens,labelsep=quad}|
\end{quote}
\example*{labelformat=bf-parens,labelsep=quad}{\exampletext}

\pagebreak[3]
\DescribeMacro{\bothIfFirst}
\DescribeMacro{\bothIfSecond}
If you define your own caption label formats and use the
\package{subcaption} or \package{subfig}\cite{subfig} package,
you should take care of empty caption label names.
For this purpose the commands
\begin{quote}
  |\bothIfFirst|\marg{first arg}\marg{second arg}\quad and\\
  |\bothIfSecond|\marg{first arg}\marg{second arg}
\end{quote}
are offered. |\bothIfFirst| tests if the first argument exists (means: is
not empty), |\bothIfSecond| tests if the second argument exists. If yes,
both arguments get typeset, otherwise none of them.

\smallskip

\pagebreak[3]
For example the standard label format |simple| is \emph{not} defined as
\begin{quote}
  |\DeclareCaptionLabelFormat{simple}{#1~#2}|\mbox{\quad,}
\end{quote}
because this could cause an extra space if \#1 is empty. Instead |simple|
is defined as
\begin{quote}
  |\DeclareCaptionLabelFormat{simple}%|\\
  |                  {\bothIfFirst{#1}{~}#2}|\mbox{\quad,}
\end{quote}
causing the space to appear only if the label name is present.

\smallskip

\pagebreak[3]
\DescribeMacro{\DeclareCaption-\\TextFormat}\NEWfeature{v3.0l}
Likewise you can define your own caption text formats:
\begin{quote}
  |\DeclareCaptionTextFormat|\marg{name}\marg{code using \#1}
\end{quote}
At usage \#1 gets replaced with the caption text.

\bigskip

\pagebreak[3]
\DescribeMacro{\DeclareCaption-\\LabelSeparator}
You can define your own caption label separators with
\begin{quote}
  |\DeclareCaptionLabelSeparator|\marg{name}\marg{code}\mbox{\quad.}
\end{quote}
Again an easy example taken from \thispackage\ itself:
\begin{quote}
  |\DeclareCaptionLabelSeparator{colon}{: }|
\end{quote}
\begin{Expert}
There is also a starred variant, |\Declare|\-|Caption|\-|Label|\-|Separator*|,
which causes the code being typeset without using the |label|\-|font=| setting.
So for example the label separators |quad|, |new|\-|line|, and |en|\-|dash|
are defined in this way.
\end{Expert}

\bigskip

\pagebreak[3]
\DescribeMacro{\DeclareCaption-\\Justification}
You can define your own caption justifications with
\begin{quote}
  |\DeclareCaptionJustification|\marg{name}\marg{code}\mbox{\quad.}
\end{quote}
The \meta{code} simply gets typeset just before the caption.
E.g.~using the justification |raggedright|, which is defined as
\begin{quote}
  |\DeclareCaptionJustification{raggedright}{\raggedright}|\mbox{\quad,}
\end{quote}
typesets captions with all lines moved to the left margin.

\bigskip

\pagebreak[3]
\DescribeMacro{\DeclareCaption-\\Font}
You can define your own caption fonts with
\begin{quote}
  |\DeclareCaptionFont|\marg{name}\marg{code}\quad.
\end{quote}
For example this package defines the options |small| and |bf| as
\begin{quote}
  |\DeclareCaptionFont{small}{\small}|\quad and\\
  |\DeclareCaptionFont{bf}{\bfseries}|\quad.
\end{quote}
\iffalse
The line spacing could be customized using the \package{setspace} package,
for example:%\NEWdescription{v3.0h}
\begin{quote}
  |\usepackage{setspace}|\\
  |\DeclareCaptionFont{singlespacing}{\setstretch{1}}|~\footnote{%
  \emph{\DefaultNoteText:} Using \cs{singlespacing} does not work here since
  it contains a \cs{vskip} command.}\\
  |\DeclareCaptionFont{onehalfspacing}{\onehalfspacing}|\\
  |\DeclareCaptionFont{doublespacing}{\doublespacing}|\\
  |\captionsetup{font={onehalfspacing,small},labelfont=bf}|
\end{quote}
\example{font={onehalfspacing,small},labelfont=bf,singlelinecheck=off}{\exampletext}
\fi
An example which brings color into life:
\begin{quote}
  |\usepackage{color}|\\
  |\DeclareCaptionFont{red}{\color{red}}|\\
  |\DeclareCaptionFont{green}{\color{green}}|\\
  |\DeclareCaptionFont{blue}{\color{blue}}|\\
  |\captionsetup{labelfont={blue,bf},textfont=green}|
\end{quote}
\example*{labelfont={color=blue,bf},textfont={color=green},singlelinecheck=off}{\exampletext}
But since \thispackage\ already includes the tricky definition
\begin{quote}
  |\DeclareCaptionFont{color}{\color{#1}}|
\end{quote}
you can get the same result simply with
\begin{quote}
  |\usepackage{color}|\\
  |\captionsetup{labelfont={color=blue,bf},|\\
  |               textfont={color=green}}|\quad.
\end{quote}

\medskip

\pagebreak[3]
\DescribeMacro{\DeclareCaption-\\Style}
You can define your own caption styles with
\begin{quote}
  |\DeclareCaptionStyle|\marg{name}\oarg{additional options}\marg{options}\mbox{\quad.}
\end{quote}
Remember, caption styles are just a collection of suitable options, saved
under a given name. You can wake up these options at any time with the
option |style=|\meta{style name}.

All caption styles are based on the |base| set of options. (See \Ref{style}
for a complete list.) So you only need to specify options which are different
to them.

If you specify \meta{additional options} they get used in addition when
the caption fits into a single line and this check was not disabled with
the option |single|\-|line|\-|check=off|.

Again a very easy example taken from the core of this package: The
caption style |base| is pre-defined as
\begin{quote}
  |\DeclareCaptionStyle{base}%|\\
  |        [justification=centering,indention=0pt]{}|\quad.
\end{quote}

Something more exciting:
\begin{quote}
  |\DeclareCaptionStyle{mystyle}%|\\
  |        [margin=5mm,justification=centering]%|\\
  |        {font=footnotesize,labelfont=sc,margin={10mm,0mm}}|\\
  |\captionsetup{style=mystyle}|
\end{quote}
gives you caption like these ones:
\begin{Example}
  \captionsetup{type=figure,style=mystyle,position=b}
  \caption{A short caption.}
  \caption{A long long long long long long long long long
             long long long long long long long long long
             long long long long long long long long long caption.}
\end{Example}

\bigskip

\pagebreak[3]
\DescribeMacro{\DeclareCaption-\\ListFormat}\NEWfeature{v3.1}
You can define your own caption list formats with
\begin{quote}
  |\DeclareCaptionListFormat|\marg{name}\marg{code using \#1 and \#2}\mbox{\quad.}
\end{quote}
At usage \#1 gets replaced with the label prefix (e.g.~|\p@figure|),
and \#2 gets replaced with the reference number (e.g.~|\thefigure|).

% --------------------------------------------------------------------------- %

\subsection{Further examples}
\label{examples}

\subsubsection*{Example 1}

%\NEWdescription{v3.1}
If you would like to have a line break between caption label and text
you could define it this way:
\begin{quote}
  |\DeclareCaptionFormat{myformat}{#1#2\\#3}|
\end{quote}
If you select this format with |\caption|\-|setup{format=|\x|myformat}|
you get captions like this:
%\begin{Example}
%  \captionsetup{skip=0pt}
  \example{format=myformat1,labelfont=bf}{\exampletext}
%\end{Example}
You could even use an indention with this caption format:
\begin{quote}
  |\captionsetup{format=myformat,indention=1cm}|
\end{quote}
This would look like this:
%\begin{Example}
%  \captionsetup{skip=0pt}
  \example{format=myformat1,indention=1cm,labelfont=bf}{\exampletext}
%\end{Example}
Instead, you would like to have an indention only of the first line of
the caption text? No problem, e.g.
\begin{quote}
  |\newlength\myindention|\\
  |\DeclareCaptionFormat{myformat}%|\\
  |               {#1#2\\\hspace*{\myindention}#3}|\\
  \ldots\\
  |\setlength\myindention{1cm}|\\
  |\captionsetup{format=myformat}|
\end{quote}
would give you captions like
%\begin{Example}
%  \captionsetup{skip=0pt}
  \example{format=myformat2,myindention=1cm,labelfont=bf}{\exampletext}
%\end{Example}
But you want to have an caption option for this indention, so you can use
it for example with
|\caption|\-|setup|\x|[figure]|\x|{my|\-|indention=|\x\ldots|}|?
This could be done, too:
\begin{quote}
  |\newlength\myindention|\\
  |\DeclareCaptionOption{myindention}%|\\
  |               {\setlength\myindention{#1}}|\\
  |\DeclareCaptionFormat{myformat}%|\\
  |               {#1#2\\\hspace*{\myindention} #3}|\\
  \ldots\\
  |\captionsetup{format=myformat,myindention=1cm}|
\end{quote}
would give the same result as the example above.

\subsubsection*{Example 2}

You want captions to look like this:
%\begin{Example}
%  \captionsetup{skip=0pt}
  \example{format=reverse,labelformat=fullparens,labelsep=fill,labelfont=it}{\exampletext}
%\end{Example}
\pagebreak[2]
You could do it this way:
\nopagebreak[3]
\begin{quote}
  |\DeclareCaptionFormat{reverse}{#3#2#1}|\\
  |\DeclareCaptionLabelFormat{fullparens}%|\\
  |               {(\bothIfFirst{#1}{~}#2)}|\\
  |\DeclareCaptionLabelSeparator{fill}{\hfill}|\\
  |\captionsetup{format=reverse,labelformat=fullparens,|\\
  |              labelsep=fill,font=small,labelfont=it}|
\end{quote}

\subsubsection*{Example 3}

The caption text should go into the left margin? A possible solution would be:
\begin{quote}
  |\DeclareCaptionFormat{llap}{\llap{#1#2}#3\par}|\\
  |\captionsetup{format=llap,labelsep=quad,singlelinecheck=no}|
\end{quote}
As a result you would get captions like this:
\example{format=llap,labelsep=quad,singlelinecheck=no,margin=0pt}{\exampletext}

\medskip

If the indention into the margin shall be fixed, you could use a combination
of |\llap| and |\makebox|, for example:
\begin{quote}
  |\DeclareCaptionFormat{llapx}%|\\
  |               {\llap{\makebox[5em][l]{#1}}#3\par}|\\
  |\captionsetup{format=llapx,singlelinecheck=off}|
\end{quote}
\example{format=llapx,singlelinecheck=off,skip=0pt,margin=0pt}{\exampletext}

\subsubsection*{Example 4}

This example puts a figure aside a table, but uses only one common caption for both.
This will be achieved by a combination of
|\Declare|\-|Caption|\-|Label|\-|Format| and |\caption|\-|list|\-|entry|:
\begin{quote}
  |\DeclareCaptionLabelFormat{andtable}%|\\
  |               {#1~#2 \& \tablename~\thetable}|\\
  \ldots\\
  |\begin{figure}|\\
  |  \centering|\\
  |  \includegraphics{|\ldots|}%|\\
  |  \qquad|\\
  |  \begin{tabular}[b]{\ldots}|\\
  |    |\ldots\\
  |  \end{tabular}|\\
  |  \captionlistentry[table]{|\ldots|}|\\
  |  \captionsetup{labelformat=andtable}|\\
  |  \caption{|\ldots|}|\\
  |\end{figure}|
\end{quote}
\DeleteShortVerb{\|}%
\noindent\begin{minipage}{\linewidth}
  \captionsetup{type=figure}
  \centering
  \includegraphics[width=30pt]{cat}%
  \qquad
  \begin{tabular}[b]{r|rr}
      & x & y \\\hline
    a & 1 & 2 \\
    b & 3 & 4 \\
  \end{tabular}
  \captionlistentry[table]{A figure and a table with a common caption}
  \captionsetup{labelformat=andtable}
  \caption{A figure and a table with a common caption~\footnotemark}
\end{minipage}
\footnotetext{The picture was taken with permission from the
  \LaTeX\ Companion\cite{TLC2} examples.}
\MakeShortVerb{\|}

(Please remember that |\caption|\-|list|\-|entry| increments the figure
 resp.~table counter.)

% --------------------------------------------------------------------------- %

\clearpage
\section{Document classes \& Babel support}
\label{classes}

%\NEWdescription{v3.1}
This section will give you an overview of the document classes the
\package{caption} supports, what do they already offer regarding captions,
what side effects will occur when using the \package{caption} package with
them, and what are the `default' settings for them.

\NEWfeature{v3.1}
The `default' settings depend on the document class you use; they represent
how the class author wanted the captions to look like. So for example
setting |format=|\x|default| can give you different visual designs when used
with different document classes.

\begingroup\setlength\leftmargini{0.3em}% default = 2.5em
\INFO
If you don't find the document class you use in this section, you usually
don't have to worry: Many document classes (e.g.~the \class{octavo} class)
are derived from one of the standard document classes \class{article},
\class{report}, or \class{book}, and behave the same regarding captions.
\Thispackage\ automatically does a compatibility check against the document
class used and will give you the clear warning
\begin{quote}\footnotesize
  |Package caption Warning: Unsupported document class (or package) detected,|\\
  |(caption)                usage of the caption package is not recommended.|\\
  |See the caption package documentation for explanation.|
\end{quote}
if such an incompatibility was detected.
If you don't get such warning everything is fine, but if you get it the
usage of \thispackage\ is not recommended and especially not supported.
\endgroup

\begin{Expert}
If you get such a compatibility warning but decide to use \thispackage{}
anyway, you should watch carefully what side-effects occur, usually the
look and feel of your captions will change by just including \thispackage{}
without options, meaning they do not look like as intended by the author
of the document class.
If this is fine for you, you should first specify the option |style=|\x|base|
via |\use|\-|package[style=|\x|base]{caption}| or
|\caption|\-|setup{style=|\x|base}| to set \thispackage\ into a well-defined
state.
Afterwards you can start setting your own options additionally and
keep your fingers crossed.
\end{Expert}

\newcommand*\Option{Option}
\newcommand*\defaultvalue{\texttt{default} value}
\newcommand*\uses{\textit{uses}}
\newcommand*\settings{\textit{settings}}
\newcommand*\nofont{\textit{none}}

\subsection{Standard \LaTeX: article, report, and book}

\begin{tabular}{ll}
\Option          & \defaultvalue \\\hline
|format=|        & |plain| \\
|labelformat=|   & |simple| \\
|labelsep=|      & |colon| \\
|justification=| & |justified| \\
|font=|          & \nofont \\
|labelfont=|     & \nofont \\
|textfont=|      & \nofont \\
\end{tabular}

\begin{Annotation}
This also applies to document classes derived from them.
\end{Annotation}

\subsection{\AmS: amsart, amsproc, and amsbook}
\label{AMS}

\begin{tabular}{ll}
\Option          & \defaultvalue \\\hline
|format=|        & |plain| \\
|labelformat=|   & |simple| \\
|labelsep=|      & |.\enspace| \\
|justification=| & |justified| \\
|font=|          & |\@captionfont| \\
|labelfont=|     & |\@captionheadfont| \\
|textfont=|      & |\@captionfont\upshape| \\
\end{tabular}

\begin{Annotation*}
|\@caption|\-|font| will be set to |\normal|\-|font|, and
|\@caption|\-|head|\-|font| to |\sc|\-|shape| by the \AmS\ document classes.
\end{Annotation*}

Furthermore the margin will be set to |\caption|\-|indent| for
more-than-one-line captions (which will be set to |3pc| by the \AmS\ classes),
the margin for single-line captions will be set to the half of it instead.
If you want to use a common margin for both, insert
|\clear|\-|caption|\-|setup[margin*]{single|\-|line}|
into the preamble of your document, after loading \thispackage.

Additionally the options |figure|\-|position=b,|\x|table|\-|position=t|
will be set. You can override these settings by specifying other values for
|figure|\-|position=| or |table|\-|position=| in the option list while
loading \thispackage.

\subsection{beamer}
\label{beamer}

\begin{tabular}{ll}
\Option          & \defaultvalue \\\hline
|format=|        & |plain| \\
|labelformat=|   & \textit{not numbered} \\
|labelsep=|      & |colon| \\
|justification=| & |raggedright| \\
|font=|          & \class{beamer} ``|caption|'' \settings \\
|labelfont=|     & \class{beamer} ``|caption name|'' \settings \\
|textfont=|      & \nofont \\
\end{tabular}

\subsubsection*{Build-in features, and side-effects}
You can setup font and color settings with
|\set|\-|beamer|\-|font{caption}|\marg{options} and
|\set|\-|beamer|\-|font{caption name}|\marg{options}.
This will still work, unless you set a different font with
|\caption|\-|setup{font=|\x\meta{options}|}| or
|\caption|\-|setup{label|\-|font=|\x\meta{options}|}|.\par
Furthermore the \class{beamer} classes offer different caption templates
which can be chosen with
|\set|\-|beamer|\-|template|\x|{caption}|\x|[|\meta{template}|]|.
Since \thispackage\ replaces this caption template mechanism,
|\def|\-|beamer|\-|template*|\x|{caption}|\x\marg{template code}
and
|\set|\-|beamer|\-|template|\x|{caption}|\x|[|\meta{template}|]|
will have no effect when \thispackage\ is used.
(Exception: Selecting the template |default|, |num|\-|bered|, or
|caption| |name| |own| |line| will be recognized by \thispackage and be mapped
to corresponding options.)

\subsection{\KOMAScript: scrartcl, scrreprt, and scrbook}
\label{KOMA}

\begin{tabular}{ll}
\Option          & \defaultvalue \\\hline
|format=|        & \uses\ |\setcapindent| \textit{\&} |\setcaphanging| \settings \\
|labelformat=|   & \textit{like \purett{simple}, but with support of ``autodot''}\\
|labelsep=|      & |\captionformat| \\
|justification=| & |justified| \\
|font=|          & |\setkomafont{caption}| \settings \\
|labelfont=|     & |\setkomafont{captionlabel}| \settings \\
|textfont=|      & \nofont \\
\end{tabular}

\subsubsection*{Build-in features}
The \KOMAScript\ document classes offer many ways to customize the look and
feel of the captions. For an overview and a full description please take a
look at the \KOMAScript\ documentation, section `Tables and Figures'.

\subsubsection*{Side effects}
The optional argument of |\set|\-|cap|\-|width| is not supported and
will be ignored if used in conjunction with the \package{caption} package.
Furthermore the \KOMAScript\ options |table|\-|caption|\-|above| \&
|table|\-|caption|\-|below| and the commands |\caption|\-|above| \&
|\caption|\-|below| are stronger than the |position=| setting offered by
the \package{caption} package.

\subsection{\NTG: artikel, rapport, and boek}
\label{NTG}

\begin{tabular}{ll}
\Option          & \defaultvalue \\\hline
|format=|        & |plain| \\
|labelformat=|   & |simple| \\
|labelsep=|      & |colon| \\
|justification=| & |justified| \\
|font=|          & \nofont \\
|labelfont=|     & |\CaptionLabelFont| \\
|textfont=|      & |\CaptionTextFont| \\
\end{tabular}

\subsubsection*{Build-in features, and side-effects}
|\Caption|\-|Label|\-|Font| and |\Caption|\-|Text|\-|Font| can be set either
directly or by using |\Caption|\-|Fonts|.
Both is still supported unless you use one of the two options
|label|\-|font=| or |text|\-|font=| offered by \thispackage.

\subsection{\SmF: smfart and smfbook}
\label{SMF}

Since the \SmF\ document classes are derived from the \AmS\ document classes
the same default values are valid here.

Additionally the margin is limited up to the tenth of the |\line|\-|width|.
If you don't like this limitation, you can switch it off with the option
|max|\-|margin=|\x|off| or |max|\-|margin=|\x|false|
(which both means the same).

\subsection{thesis}
\label{thesis}

\begin{tabular}{ll}
\Option          & \defaultvalue \\\hline
|format=|        & |hang| \\
|labelformat=|   & \textit{like \purett{simple}, but with short name}\\
|labelsep=|      & |colon| \\
|justification=| & |justified| \\
|font=|          & \nofont \\
|labelfont=|     & |\captionheaderfont| \\
|textfont=|      & |\captionbodyfont| \\
\end{tabular}

\subsubsection*{Build-in features, and side-effects}
The caption label font can be set with |\caption|\-|header|\-|font|,
the caption text font with |\caption|\-|body|\-|font|.
Both is still supported unless you use one of the two options
|label|\-|font=| or |text|\-|font=| offered by \thispackage.

\subsection{frenchb babel option}
\label{frenchb}

If you use the \package{frenchb} option of the \package{babel} package
with one of the three standard \LaTeX{} classes (or a one derived from them)
the default |label|\-|sep=| will be set to |\Caption|\-|Sep|\-|a|\-|ra|\-|tor|
(offered by \package{frenchb}), overriding the default value set by the
document class.
So redefining |\Caption|\-|Sep|\-|a|\-|ra|\-|tor| will still work, unless you
don't select a different |label|\-|sep=| than the default one.

\INFO*
Please load \thispackage\ \emph{after} the \package{babel} package.

\subsection{frenchle and frenchpro packages}
\label{frenchpro}

If you use the \package{frenchle} or \package{frenchpro} package,
the default |label|\-|sep=| will be set to |\caption|\-|sep|\-|a|\-|ra|\-|tor|
(offered by \package{frenchle/pro}) plus \cs{space}, overriding the default
value set by the document class.
So redefining |\caption|\-|sep|\-|a|\-|ra|\-|tor| will still work, unless you
don't select a different |label|\-|sep=| than the default one.

Furthermore the default |text|\-|font=| will be set to |text|\-|font=|\x|it|,
since this emulates the default setting of |\caption|\-|font| defined by the
\package{frenchle} or \package{frenchpro} package.
%If you redefine |\captionfont| after loading the \package{frenchle} or
%\package{frenchpro} package, this redefinition gets lost.
Please note that the command |\caption|\-|font| is used by the
\package{caption} package internally for a different purpose,
so you should not change it (anymore).

The command |\un|\-|numbered|\-|captions|\marg{figure \emph{or} table} will
still work, but only unless you don't select a different |label|\-|format=|
than the default one.

\INFO*
Please load \thispackage\ \emph{after} the \package{frenchle}
or \package{frenchpro} package.

% --------------------------------------------------------------------------- %

\clearpage
\section{Package support}
\label{packages}
\label{compatibility}

%\NEWdescription{v3.1}
\Thispackage\ was adapted to the following packages which deals with
captions, too:
\begin{quote}
  \package{float}, \package{floatflt}, \package{fltpage},
  \package{hyperref}, \package{hypcap}, \package{listings},
  \package{longtable}, \package{picinpar}, \package{picins},
  \package{rotating}, \package{setspace}, \package{sidecap},
  \package{subfigure}, \package{supertabular}, \package{threeparttable},
  \package{wrapfig}, and \package{xtab}
\end{quote}

Furthermore the \package{floatrow} package\cite{floatrow}, the
\package{subcaption} package (which is part of the \package{caption} package
bundle), and the \package{subfig} package\cite{subfig} support \thispackage{}
and use its |\caption|\-|setup| interface.

\bigskip

\begingroup\setlength\leftmargini{0.3em}% default = 2.5em
\INFO
If a package (or document class) unknown to \thispackage\ redefines
the |\caption| command as well, this redefinition will be preferred
over the one this package offers, providing maximum compatibility and
avoiding conflicts.
If such a potential incompatibility is detected, you will see this
warning message:~\footnote{%
  You can suppress this warning by specifying the option
  \texttt{compatibility=true} when loading \thispackage.}
\begin{quote}\footnotesize
  |Package caption Warning: \caption will not be redefined since it's already|\\
  |(caption)                redefined by a document class or package which is|\\
  |(caption)                unknown to the caption package.|\\
  |See the caption package documentation for explanation.|
\end{quote}
\endgroup

As a result, the following features offered by \thispackage\ will not be available:
\begin{itemize}
  \item the options |labelformat=|, |position=auto|, |list=|, and |listformat=|
  \item |\caption*| (to produce a caption without label)
  \item |\caption[]{|\ldots|}| (to produce no entry in the List of Figures or Tables)
  \item |\caption{}| (to produce an empty caption without label separator)
  \item |\ContinuedFloat|
  \item correctly justified captions in environments like \texttt{wide} and
        \texttt{addmargin} which add extra margins
  \item the \textsf{hypcap} feature \See{\Ref{hyperref}\,}
  \item the sub-caption feature \See{\package{subcaption} package documentation}
\end{itemize}

\DescribeMacro{compatibility=}
You can override this compatibility mode by specifying the option
\begin{quote}
  |compatibility=false|
\end{quote}
when loading \thispackage.
But please note that using this option is neither recommended nor supported
since unwanted side-effects or even errors could occur afterwards.
(For that reason you will get a warning about this.)

\newcommand\packagedescription[1]{%
  \ifvmode\else\par\fi
  \nopagebreak
  \parbox[b]{\linewidth}{\footnotesize\leftskip=10pt\rightskip=10pt\relax#1}\par
  \nopagebreak\smallskip\nopagebreak}

\PageBreak
\subsection{algorithms}
\label{algorithms}
\packagedescription{%
The \package{algorithms} package bundle\cite{algorithms} provides two
environments: The \texttt{algorithmic} environment provides a possibility
for describing algorithms, and the \texttt{algorithm} environment provides
a ``float'' wrapper for algorithms.}

Since the \texttt{algorithm} environment is implemented via |\new|\-|float|
provided by the \package{float} package\cite{float}, please see \Ref{float}.

\iffalse
\pagebreak[3]
\subsection{algorithm2e}
\label{algorithm2e}
\packagedescription{%
The \package{algorithm2e} package\cite{algorithm2e} offers an environment for
writing algorithms in LaTeX2e.}

Since the \package{algorithm2e} package does not use some kind of standard
interface for defining its floating environment (e.g.~|\newfloat| of the
\package{float} package) and typesetting its caption,
the \package{algorithm2e} package is \emph{not} supported by \thispackage.

So if you want to customize the captions using \thispackage,
you have to build a new environment which uses a supported interface regarding
floats (e.g. using the \package{float} or \package{floatrow} package),
and combine this environment with the internal \package{algorithm2e} code.

As first step you could define a non-floating environment \texttt{algorithmic},
for example:
\begin{quote}
  |\usepackage{algorithm2e}|\\
  |% save the "algorithm" environment from the algorithm2e package|\\
  |\let\ORIGalgorithm\algorithm|\\
  |\let\ORIGendalgorithm\endalgorithm|\\
  |% define the algorithmic environment|\\ % , based on the saved environment
  |\newenvironment{algorithmic}%|\\
  |  {\renewenvironment{algocf}[1][h]{}{}% pass over floating stuff|\\
  |   \ORIGalgorithm}%|\\
  |  {\ORIGendalgorithm}|\\
  \ldots
\end{quote}
Having defined this non-floating environment, you could define your own new
floating environment with |\newfloat| of the \package{float} package
(or |\DeclareNewFloatType| of the \package{floatrow} package),
and use the combination of this floating environment and \texttt{algorithmic}
in its body, just like you would do when you use the
\package{algorithm}/\package{algorithmic} package tandem.

You could even use the \package{algorithm} package for this purpose, for example:
\begin{quote}
  \ldots\\
  |% load the algorithm package to re-define the|\\
  |% floating environment "algorithm" and \listofalgorithms|\\
  |\let\listofalgorithms\undefined|\\
  |\usepackage{algorithm}|\\
  \ldots\\
  |% Example usage:|\\
  |\begin{algorithm}|\\
  |\caption{An algorithm}|\\
  |\begin{algorithmic}|\\
  |  \SetVline|\\
  |  \eIf{cond1}{|\\
  |    a line\;|\\
  |  }{|\\
  |    another line\;|\\
  |  }|\\
  |\end{algorithmic}|
\end{quote}
Now you are finally able to customize the float and caption layout like every
other floating environment defined with the \package{float} package\cite{float},
please see \Ref{float}.

\begin{Annotation}
An alternative would be using the \package{algorithmicx} package.
\end{Annotation}
\fi

\pagebreak[3]
\subsection{float}
\label{float}
\packagedescription{%
The \package{float} package\cite{float} introduces the commands
\cs{restylefloat} to give existing floating environments a new look \& feel
and \cs{newfloat} to define new floating environments.
It also provides the ``\texttt{H}'' float placement option which places
the environments ``here'' instead of letting them floating around.}

For floating environments defined with \cs{newfloat} or \cs{restylefloat} the
|po|\-|si|\-|tion| option has no effect on the main caption anymore, since its
placement and spacing will be controlled by the selected float style instead.

A caption style and options defined with the name of the float style will be
executed additionally to the regular ones.
Using this mechanism \thispackage\ emulates the default look \& feel of the
|ruled| captions: It defines the caption style
\begin{quote}
  |\DeclareCaptionStyle{ruled}%|\\
  |       {labelfont=bf,labelsep=space,strut=off}|\mbox{\quad.}
\end{quote}
So to change this you need either define your own caption style called
|ruled| or use |\caption|\-|setup[ruled]|\marg{options} to specify additional
options.%,
%e.g.~|\caption|\-|setup[ruled]{labelsep=|\x|colon}|.

Also by using this mechanism the skip between a |boxed| float and its caption
is specified, overriding its global value:
\begin{quote}
  |\captionsetup[boxed]{skip=2pt}|
\end{quote}
For changing this, just use |\caption|\-|setup[boxed]{skip=|\x\meta{value}|}|
with an appropriate value. Or if you want to use the global |skip| setting
instead, you can remove the usage of the local setting for these floats with
|\clear|\-|caption|\-|setup[skip]{boxed}|.

\begin{Note}
Only one single caption can be typeset inside environments defined with
|\new|\-|float| or |\re|\-|style|\-|float|, furthermore these environments
are not behaving exactly like the pre-defined floats |figure| and |table|.
As a consequence many packages do not cooperate well with these.
Furthermore the \package{float} package has some caveats \& limitations,
so if you just want to define a new simple floating environment---behaving
like |figure| or |table|---I recommend using |\Declare|\-|Floating|\-|Environment|
offered by the \package{newfloat} package instead.
And for defining non-simple floating environments and customization I
recommend using |\Declare|\-|New|\-|Float|\-|Type| offered by the
\package{floatrow} package\cite{floatrow}.
\end{Note}

\iffalse
Please also note that you \emph{don't} need a |\restyle|\-|float| for using
the ``|H|'' float placement specifier. Some docs say so, but they are
wrong.
And |\restyle|\-|float{table}| is a very good method to shoot yourself
in the foot, since many packages using |table| internally are not working
correctly afterwards.
\fi

\pagebreak[3]
\subsection{floatflt}
\label{floatflt}
\packagedescription{%
The \package{floatflt} package\cite{floatflt} offers figures and tables
which do not span the full width of a page and are float around by text.}

\NEWfeature{v3.1}
If you want to setup special options for the |float|\-|ing|\-|figure| and
|float|\-|ing|\-|table| environments you can use
\begin{quote}
  |\captionsetup[floatingfigure]|\marg{options}\quad\emph{and}\\
  |\captionsetup[floatingtable]|\marg{options}\quad.
\end{quote}
These options will be executed additionally to the regular ones for
|figure| resp.~|table|.

\begin{Note}
The |margin| resp.~|width| setting will not be used for these figures resp.~tables,
unless you set it explicit with |\caption|\-|setup[floating|\-|figure]{|\ldots|}|
resp.~|\caption|\-|setup[floating|\-|table]{|\ldots|}|.
\end{Note}

\pagebreak[3]
\subsection{fltpage}
\label{fltpage}
\packagedescription{%
The \package{fltpage} package\cite{fltpage} offers the outhouse of the
caption for |figure|s or |table|s which needs the whole page
for its contents. This will be done by placing the caption on the bottom
of the previous or next page.}

\iffalse
\NEWfeature{v3.1}
Two options control the links to the environments |FPfigure| and |FPtable|:
\begin{description}
\item{\texttt{FPlist=caption} or \texttt{FPlist=figure}}\\[\smallskipamount]
If set to |caption|, the list entry will link to the caption; if set to
|figure|, it will link to the figure contents.
(The default setting is |FP|\-|list=|\x|caption|.)
\item{\texttt{FPref=caption} or \texttt{FPref=figure}}\\[\smallskipamount]
If set to |caption|, a \cs{ref}, \cs{pageref}, \cs{autoref}, or \cs{nameref}
will link to the caption; if set to |figure|, it will link to the figure contents.
(The default setting is |FP|\-|ref=|\x|figure|.)
\end{description}
\fi

If you want to setup special options for the |FPfigure| and |FPtable|
environments you can use
\begin{quote}
  |\captionsetup[FPfigure]|\marg{options}\quad\emph{and}\\
  |\captionsetup[FPtable]|\marg{options}\quad.
\end{quote}
These options will be executed additionally to the regular ones for
|figure| or |table|.

\pagebreak[3]
\subsection{hyperref}
\label{hyperref}
\packagedescription{%
The \package{hyperref} package\cite{hyperref} is used to handle cross
referencing commands in LaTeX to produce hypertext links in the document.}

\NEWfeature{v3.1}
There are two options available to control the placement of hyperlink
anchors:~\footnote{These options are named after the \package{hypcap} package
which they supersede.}
\begin{description}
\item{\texttt{hypcap=true} or \texttt{hypcap=false}}\\[\smallskipamount]
If set to |true| all hyperlink anchors -- where entries in the
List of Figures, |\ref|, and |\auto|\-|ref| will link to -- are placed at
the beginning of the (floating) environment.\par
If set to |false| the hyperlink anchor is (usually) placed at the caption.\par
(The default setting is |hypcap=|\x|true|.)
\item{\texttt{hypcapspace=}\meta{amount}}\\[\smallskipamount]
Because it looks poor if the hyperlink points exactly at the top of
the figure or table, you can specify a vertical distance between
the hyperlink anchor and the (floating) environment itself,
e.g.~|hyp|\-|cap|\-|space=|\x|0pt| removes this distance.\par
(The default setting is |hyp|\-|cap|\-|space=|\x|0.5\base|\-|line|\-|skip|.)
\end{description}

Both settings have no effect in |lst|\-|listing|s (provided by the
\package{listings} package), |long|\-|table|s (provided by the
\package{longtable} package), |super|\-|tabular|s (provided by the
\package{supertabular} package), and |x|\-|tabular|s (provided by the
\package{xtab} package), within these environments hyperlink
anchors will always be placed as if |hyp|\-|cap=|\x|true| and
|hyp|\-|cap|\-|space=|\x|0pt| would be set.

\pagebreak[3]Please note:\nopagebreak
\begin{description}
\item{\cs{captionof}\marg{type}\csmarg{\purerm\ldots}
  vs.~\cs{captionsetup}\csmarg{type=\textrm{\meta{type}}}$+$\cs{caption}\csmarg{\purerm\ldots}}%
\\[\smallskipamount]
Without \package{hyperref} loaded, both will give you identical results.
But with \package{hyperref} loaded, and with |hyp|\-|cap=|\x|true| requested,
the hyperlink anchor will be placed different.
For example:
\begin{quote}
|\begin{minipage}{\linewidth}|\\
|  |\ldots\\
|  \captionof{figure}{A figure}|\\
|\end{minipage}|
\end{quote}
will place the hyperlink anchor at the caption.
(And if |hyp|\-|cap=|\x|true| is set, you will get a warning about this.)
But
\begin{quote}
|\begin{minipage}{\linewidth}|\\
|  \captionsetup{type=figure}|\\
|  |\ldots\\
|  \caption{A figure}|\\
|\end{minipage}|
\end{quote}
will place the hyperlink anchor at the beginning of the |mini|\-|page|,
since |\caption|\-|setup{type=|\x|figure}| does not only set the caption type
to ``figure'' but does place a hyperlink anchor, too.

\item{\cs{caption}\csoarg{}\csmarg{\purerm\ldots}
  vs.~\cs{captionsetup}\csmarg{list=false}$+$\cs{caption}\csmarg{\purerm\ldots}}\\[\smallskipamount]
Again, without \package{hyperref} loaded, both will give you identical results.
But with \package{hyperref} loaded, the difference is in the nuances.
So for example the optional argument of |\caption| will also be written to
the |aux| file, and will be used by the |\name|\-|ref| command.
So if you choose to use |\caption| with empty optional argument,
|\name|\-|ref| will also give you an empty result.
So it's better to use |\caption|\-|setup{list=|\x|false}| if you don't
want an entry in the List of Figures or List of Tables.
\end{description}

\pagebreak[3]
\subsection{hypcap}
\label{hypcap}
\packagedescription{%
The \package{hyp\-cap} package\cite{hypcap} offers a solution to the problem
that links to a float using \package{hyper\-ref} may anchors to the caption
rather than the beginning of the float.
Since \thispackage\ \version{3.1} already solves this problem for itself,
the \package{hypcap} package is usually not needed.}
%, at least when you use \thispackage.}

If the \package{hyp\-cap} package is loaded additionally to the
\package{hyper\-ref} package, it takes over the control of the hyperlink
anchor placement from \thispackage, overriding the options |hyp|\-|cap=|
and |hyp|\-|cap|\-|space=|.

So for a manual placement of hyperlink anchors
|\caption|\-|setup{type=|\x\meta{type}|}| is not sufficient anymore,
instead you need to use |\cap|\-|start| (provided by the \package{hyp\-cap}
package) for this.

Regarding the automatically placement the \package{hypcap} package offers good
placement of hyperlink anchors for the floating environments |figure| and
|table| only. In contrast the |hypcap=|\x|true| option of \thispackage\ also
offers good placements of hyperlink anchors for
|floating|\-|figure|s (provided by the \package{float\-flt} package),
|FP|\-|figure|s \& |FPtable|s (provided by the \package{flt\-page} package),
|fig|\-|window|s (provided by the \package{pic\-in\-par} package),
|par|\-|pic|s (provided by the \package{pic\-ins} package),
|SC|\-|figure|s (provided by the \package{side\-cap} package),
|three|\-|part|\-|table|s (provided by the \package{three\-part\-table} package),
and |wrapfigure|s (provided by the \package{wrap\-fig} package).%---%
%so by loading the \package{hypcap} package additionally you will
%loose that.

\pagebreak[3]
\subsection{listings}
\label{listings}
\packagedescription{%
The \package{listings} package\cite{listings} typesets programming code.}

If you want to setup special options for the |lst|\-|listing| environment
you can use
\begin{quote}
  |\captionsetup[lstlisting]|\marg{options}\quad.
\end{quote}

Please note that the \package{listings} package has its very own options
for controlling the position and the skips of the caption:
|caption|\-|pos=|, |above|\-|caption|\-|skip=|, and |below|\-|caption|\-|skip=|.
\See{\package{listings} documentation for details.}
These \package{listings} options override the \package{caption}'s ones, but
can be again overwritten by |\caption|\-|setup[lst|\-|listing]{|\ldots|}|,
e.g.%~
\begin{quote}|\caption|\-|setup[lst|\-|listing]{skip=|\x|10pt}|\quad.\end{quote}

\pagebreak[3]
\subsection{longtable}
\label{longtable}
\packagedescription{%
The \package{longtable} package\cite{longtable} offers an environment
which behaves similar to the \texttt{tabular} environment, but the table
itself can span multiple pages.}

If you want to setup special options for the |long|\-|table| environment
you can use
\begin{quote}
  |\captionsetup[longtable]|\marg{options}\quad.
\end{quote}
These options will be executed additionally to the regular ones for
|table|.

The |margin| and |width| settings usually override |\LT|\-|cap|\-|width|,
so you get an equal look \& feel of the captions in |table|s and
|long|\-|table|s.
But if you set |\LT|\-|cap|\-|width| to a value different than its default
$=$ |4in|, \thispackage\ will follow that.
(But |\LT|\-|cap|\-|width| will be overwritten by
|\caption|\-|setup[long|\-|table]{width=|\x\meta{value}|}|,
even if it is set to a value different than |4in|.)

\begin{Note}
|\caption|\-|of| and |\Continued|\-|Float| do \emph{not} work for |long|\-|table|s.
Furthermore neither |\caption| nor |\caption|\-|list|\-|entry| will increment
the |table| counter here; it's incremented by the |long|\-|table| environment
instead.
\NEWfeature{v3.1}
If you need a |long|\-|table| which does not increment the |table| counter
please use the |long|\-|table*| environment (offered by the
\package{ltcaption} package which is part of \thispackage\ bundle and will be
loaded automatically).
\end{Note}

\pagebreak[3]
\subsection{picinpar}
\label{picinpar}
\packagedescription{%
Similar to the \package{floatflt} package the \package{picinpar}
package\cite{picinpar} offers figures and tables which do not span the full
width of a page and are float around by text.
For a detailed discussion about the differences between these packages
please take a look at The \LaTeX\ Companion\cite{TLC2}.}

\NEWfeature{v3.1}
If you want to setup special options for the |fig|\-|window| and
|tab|\-|window| environments you can use
\begin{quote}
  |\captionsetup[figwindow]|\marg{options}\quad\emph{and}\\
  |\captionsetup[tabwindow]|\marg{options}\quad.
\end{quote}
These options will be executed additionally to the regular ones for
|figure| or |table|.

\begin{Note}
The |margin| resp.~|width| setting will not be used for these figures and tables,
unless you set it explicit with |\caption|\-|setup[fig|\-|window]{|\ldots|}| or
|\caption|\-|setup[fig|\-|table]{|\ldots|}|.
\end{Note}

\pagebreak[3]
\subsection{picins}
\label{picins}
\packagedescription{%
Similar to the \package{floatflt} and \package{picinpar} package the
\package{picins} package\cite{picins} offers figures and tables which do not
span the full width of a page and are float around by text.
For a detailed discussion about the differences between these packages
please take a look at The \LaTeX\ Companion\cite{TLC2}.}

\NEWfeature{v3.1}
If you want to setup special options for the |par|\-|pic| environment you can use
\begin{quote}
  |\captionsetup[parpic]|\marg{options}\quad.
\end{quote}
These options will be executed additionally to the regular ones for
|figure| or |table|.

Furthermore |\pic|\-|caption[]{|\ldots|}| produce no entry in the List of Figures,
and |\pic|\-|caption*{|\ldots|}| gives an unlabeled \& unnumbered caption.

\begin{Note}
The |margin| resp.~|width| setting will not be used for these figures and tables.
\iffalse
if |pic|\-|caption|\-|side| or |pic|\-|caption|\-|top|\-|side| is used.
The |width| setting will not be used (but the |margin| setting will) if
|pic|\-|caption|\-|out|\-|side| or |pic|\-|caption|\-|in|\-|side| is used.
\fi
But you can override this by setting it explicit with
|\caption|\-|setup[par|\-|pic]{|\ldots|}|.
\end{Note}

If you want to have a |\pic|\-|caption| of another type as |figure|, please
do \emph{not} re-define |\@cap|\-|type| as suggested by the \package{picins}
documentation. Please use the |\pic|\-|caption|\-|type|\marg{type} command
which is offered by \thispackage\ instead, e.g.:
\begin{quote}
  |\piccaptiontype{table}|\\
  |\piccaption{An example table}|\\
  |\parpic(50mm,10mm)[s]{|\ldots|}|
\end{quote}

\pagebreak[3]
\subsection{rotating}
\label{rotating}
\packagedescription{%
The \package{rotating} package\cite{rotating} offers the floating
environments \texttt{side\-ways\-figure} and \texttt{side\-ways\-table} which
are just like normal figures and tables but rotated by 90 resp.~270 degree.
Furthermore a command \cs{rot\-caption} is offered for rotated captions only.}

\iffalse % No, we don't do that
If you want to setup special options for the |side|\-|ways|\-|figure| and
|side|\-|ways|\-|table| environments you can use
\begin{quote}
  |\captionsetup[sidewaysfigure]|\marg{options}\\
  |\captionsetup[sidewaystable]|\marg{options}\quad.
\end{quote}
These options will be executed additionally to the regular ones for
|figure| or |table|.
\fi

The command |\rot|\-|caption| will be extended by \thispackage, so
|\rot|\-|caption*| and |\rot|\-|caption|\-|of| can be used analogous to
|\caption*| and |\caption|\-|of|.

\pagebreak[3]
\subsection{setspace}
\label{setspace}
\packagedescription{%
The \package{setspace} package\cite{setspace} offers options and commands
to change the spacing,
e.g.\ \cs{usepackage}\x\csoarg{one\-half\-spacing}\x\csmarg{set\-space}
causes the document to be typeset in one-and-a-half spacing.}

If the \package{setspace} package is used in conjunction with \thispackage,
the caption will be typeset with single spacing as default. This can be changed
by specifying either |font=|\x|one|\-|half|\-|spacing|,
|font=|\x|double|\-|spacing|, or |font={stretch=|\x\meta{amount}|}|.
\See{also \Ref{fonts}}

\pagebreak[3]
\subsection{sidecap}
\label{sidecap}
\packagedescription{%
The \package{sidecap} package\cite{sidecap} offers the floating environments
\texttt{SC\-figure} and \texttt{SC\-table} which are like usual figures
and tables but the caption will be put \emph{beside} the contents.}

If you want to setup special options for the |SCfigure| and |SCtable|
environments you can use
\begin{quote}
  |\captionsetup[SCfigure]|\marg{options}\quad\emph{and}\\
  |\captionsetup[SCtable]|\marg{options}\quad.
\end{quote}
These options will be executed additionally to the regular ones for
|figure| or |table|.

\pagebreak[3]
\begin{Note}
The \package{sidecap} package offers its own options for justification.
If set, they will override the one specified with the caption option
|jus|\-|ti|\-|fi|\-|ca|\-|tion=| for captions beside their contents.
\end{Note}

\begin{Note}
The |margin| resp.~|width| setting will not be used for these figures and tables,
unless you set it explicit with |\caption|\-|setup[SC|\-|figure]{|\ldots|}|
or |\caption|\-|setup[SC|\-|table]{|\ldots|}|.
\end{Note}

\medskip

\begin{Annotation*}
Instead of using the \package{sidecap} package you can also use the more
powerful and flexible \package{floatrow} package\cite{floatrow} for typesetting
captions beside the contents.
\end{Annotation*}

\pagebreak[3]
\subsubsection*{Undocumented features}
The \package{sidecap} package \version{1.6} has some undocumented package
options and commands which allow further customization of the side-captions:

\pagebreak[3]
\DescribeMacro{margincaption}
The package option
\begin{quote}
  |margincaption|\qquad{\small(e.g.~|\usepackage[margincaption]{sidecap}|)}
\end{quote}
causes all side-captions to be placed into the margin.

\pagebreak[3]
\DescribeMacro{\sidecaptionvpos}
The command
\begin{quote}
  |\sidecaptionvpos|\marg{float type}\marg{pos}
\end{quote}
sets the vertical position of the side-caption. \meta{pos} can be either `|t|'
(for top alignment), `|b|' (for bottom alignment), or `|c|' (for center alignment).
The default setting for |table| is `|t|', for |figure| and all other ones
defined with |\Declare|\-|Floating|\-|Environment| it's `|b|'.

\pagebreak[3]
\subsection{subfigure}
\label{subfigure}
\packagedescription{%
The \package{subfigure} package\cite{subfigure} provides support for small or
`sub' figures and tables within a single figure or table environment.
This package is obsolete, new users should use \package{subfig} instead.}

Since the \package{subfigure} package is obsolete it will only be supported in
a way that old documents (which have used \thispackage\ \version{1.x} during
development) should still compile as expected.

Please use the \package{subfig} or \package{subcaption} package instead which both supports \thispackage\
\version{3.x}.

\See{the \package{subfig} package\cite{subfig} documentation for details.}

\pagebreak[3]
\subsection{supertabular and xtab}
\label{supertabular}
\packagedescription{%
The \package{supertabular}\cite{supertabular} and \package{xtab}\cite{xtab}
packages offer an environment which can span multiple pages and is quite
similar to the \texttt{long\-table} environment provided by the
\package{longtable} package\cite{longtable}.
For a detailed discussion about the differences between these powerful
packages please take a look at The \LaTeX\ Companion\cite{TLC2}.}

If you want to setup special options for the |super|\-|tabular|
resp.~|x|\-|tabular| environment you can use
\begin{quote}
  |\captionsetup[supertabular]|\marg{options}\quad\emph{resp.}\\
  |\captionsetup[xtabular]|\marg{options}\quad.
\end{quote}
These options will be executed additionally to the regular ones for
|table|.

\begin{Note}
|\Continued|\-|Float| does not work for |super|\-|tabular|s and |x|\-|tabular|s.
\end{Note}

\pagebreak[3]
\subsection{threeparttable}
\label{threeparttable}
\packagedescription{%
The \package{threeparttable} package\cite{threeparttable} provides a scheme
for tables that have a structured note section after the table contents and
the caption.
This scheme provides an answer to the old problem of putting footnotes in
tables---by making footnotes entirely unnecessary.}

\NEWfeature{v3.1}
If you want to setup special options for the |three|\-|part|\-|table| and
|measured|\-|figure| environments you can use
\begin{quote}
  |\captionsetup[threeparttable]|\marg{options}\quad\emph{and}\\
  |\captionsetup[measuredfigure]|\marg{options}\quad.
\end{quote}
These options will be executed additionally to the regular ones for
|table| or |figure|.

\begin{Note}
Since the caption will have the same width as the contents here,
the |margin| resp.~|width| setting will not be used for these figures and tables,
at least unless you set it explicit with
|\caption|\-|setup[three|\-|part|\-|table]{|\ldots|}| or
|\caption|\-|setup[measured|\-|figure]{|\ldots|}|.
\end{Note}

\medskip

\begin{Annotation*}
Instead of using the \package{threeparttable} package you can
also use the \package{floatrow} package\cite{floatrow} for this purpose.
\end{Annotation*}

\pagebreak[3]
\subsection{wrapfig}
\label{wrapfig}
\packagedescription{%
Similar to the \package{floatflt}, the \package{picinpar}, and the
\package{picins} package the \package{wrapfig} package\cite{wrapfig} offers
figures and tables which do not span the full width of a page and are float
around by text.
For a detailed discussion about the differences between these packages
please take a look at The \LaTeX\ Companion\cite{TLC2}.}

\NEWfeature{v3.1}
If you want to setup special options for the |wrap|\-|figure| and
|wrap|\-|table| environments you can use
\begin{quote}
  |\captionsetup[wrapfigure]|\marg{options}\quad\emph{and}\\
  |\captionsetup[wraptable]|\marg{options}\quad.
\end{quote}
These options will be executed additionally to the regular ones for
|figure| or |table|.

\begin{Note}
The |margin| resp.~|width| setting will not be used for these figures and
tables, unless you set it explicit with
|\caption|\-|setup[wrap|\-|figure]{|\ldots|}|
or |\caption|\-|setup[wrap|\-|table]{|\ldots|}|.
\end{Note}

% --------------------------------------------------------------------------- %

\clearpage
\section{Further reading}

I recommend the following documents for further reading:

\begin{itemize}
\item
  The \TeX\ FAQ -- Frequently asked questions about \TeX\ and \LaTeX:
  \begin{quote}\url{http://faq.tug.org/}\end{quote}

\item
  A French FAQ can be found at
  \begin{quote}\url{http://www.grappa.univ-lille3.fr/FAQ-LaTeX/}\end{quote}

\item
  ``What is a minimal working example?'' from Christian Faulhammer and Ulrich Schwarz:
  \begin{quote}\url{http://www.minimalbeispiel.de/mini-en.html}\end{quote}

\item
  \textsf{epslatex} from Keith Reckdahl contains many tips around
  including graphics in \LaTeXe\ documents.
  You will find this document in the directory
  \begin{quote}\url{http://mirror.ctan.org/info/epslatex/}\end{quote}
\end{itemize}

% --------------------------------------------------------------------------- %

\pagebreak[3]
\section{Thanks}

I would like to thank Katja Melzner,
Steven D. Cochran, Frank Mittelbach, Olga Lapko,
David Carlisle, Carsten Heinz, Keith Reckdahl, Markus Kohm,
Heiko Oberdiek, and Herbert Vo\ss.
Thanks a lot for all your help, ideas, patience, spirit, and support!

Also I would like to thank
Harald Harders,
Peter L\"offler,
Peng Yu,
Alexander Zimmermann,
Matthias Pospiech,
J\"urgen Wieferink,
Christoph Bartoschek,
Uwe St\"ohr,
Ralf Stubner,
Geoff Vallis,
Florian Keiler,
J\"urgen G\"obel,
Uwe Siart,
Sang-Heon Shim,
Henrik Lundell,
David Byers,
William Asquith,
Prof.~Dr.~Dirk Hoffmann,
Frank Martini,
Danie Els,
Philipp Woock,
Fadi Semmo,
Matthias Stevens, and
Leo Liu
who all helped to make this package a better one.

% --------------------------------------------------------------------------- %

\clearpage\appendix
\section{Alphabetical Reference}

%\newenvironment{Warnings}%
% {\begin{trivlist}%
%   \def\Message{\pagebreak[3]\leftskip=0pt\relax\item[]\color{blue}}%
%   \def\Description{\nopagebreak\par\nopagebreak\color{black}\leftskip=1.5em\nopagebreak}%
% }%
% {\color{black}\end{trivlist}}
\makeatletter
\newenvironment{Warnings}{%
  \newif\ifMessage
  \def\Message{%
    \ifMessage\end{minipage}\par\vspace\itemsep\pagebreak[3]\fi
    \begin{minipage}{\linewidth}%
    \setlength\parindent{0pt}%
    \setlength\parskip{\smallskipamount}%
    \setlength\leftskip{0pt}%
    \color{blue}%
    \Messagetrue}%
  \def\Description{%
    \par\color{black}\leftskip=1.5em}%
  \def\Or{\@testopt\@Or{or}}%
  \def\@Or[##1]{%
    \\{\small\color{black}\hspace{1.5em}\textit{-- ##1 --}}\\}%
}{%
  \ifMessage\end{minipage}\par\fi
}
\makeatother

\subsection{Options}
%\enlargethispage{10ex}
\newcommand\preambleoption{~%
  \begingroup\def\thefootnote{\fnsymbol{footnote}}%
    \footnotemark[2]%
  \endgroup}
\newcommand\preambleoptiontext{%
  \begingroup\def\thefootnote{\fnsymbol{footnote}}%
    \footnotetext[2]{These options are only available in the preamble of the document.}%
  \endgroup}

\begin{longtable*}{lll}
Option           & Short description & Section \\
\hline
\endhead
|aboveskip|       & sets the skip above caption & \ref{skips} \\
|belowskip|       & sets the skip below caption & \ref{skips} \\
|compatibility|\preambleoption & force (non-)compatibility & \ref{compatibility} \\
%|config|         &  & \ref{x} \\
|figureposition|\preambleoption & gives a hint about the figure caption position & \ref{skips} \\
|font|(|+|)       & sets the font & \ref{fonts} \\
|format|          & sets the format & \ref{formats} \\
\iffalse
|FPlist|          & To where the list entry of a FPfigure should link? & \ref{fltpage} \\
|FPref|           & To where a |\ref| to a FPfigure should link? & \ref{fltpage} \\
\fi
|hangindent|      & sets the hang indention & \ref{margins} \\
|hypcap|          & selects `hypcap' feature & \ref{hyperref} \\
|hypcapspace|     & sets the distance between hyperlink and contents & \ref{hyperref} \\
|indention|       & sets the indention & \ref{margins} \\
|justification|   & sets the justification & \ref{justification} \\
|labelfont|(|+|)  & sets the font of the caption label & \ref{fonts} \\
|labelformat|     & sets the format of the caption label & \ref{formats} \\
|labelsep|        & sets the label separator & \ref{formats} \\
|labelseparator|  & --same as |labelsep|-- & \ref{formats} \\
|list|            & switches the entries in the List on or off & \ref{lists} \\
|listformat|      & sets the `List of Figure/Table' entry format & \ref{lists} \\
|margin|          & sets the margin & \ref{margins} \\
|margin*|         & sets the margin, but only if no width is set & \ref{margins} \\
|maxmargin|       & sets the max. margin & \ref{margins} \\
|minmargin|       & sets the min. margin & \ref{margins} \\
|name|            & sets the name of the current environment & \ref{names} \\
|oneside|         & selects the one-side mode & \ref{margins} \\
|options|         & executes the given option list & \\
|parindent|       & sets the paragraph indention & \ref{margins} \\
|parskip|         & sets the skip between paragraphs & \ref{margins} \\
|position|        & gives a hint about the caption position & \ref{skips} \\
|singlelinecheck| & switches the single-line-check on or off & \ref{justification} \\
%|size|           & sets the size of font & \ref{x} \\
|skip|            & sets the skip between content and caption & \ref{skips} \\
|strut|           & switches the usage of |\strut|s on or off & \ref{formats} \\
|style|           & sets the caption style & \ref{style} \\
%|style*|         & sets the caption style & \ref{style} \\
|subtype|         & sets the sub-caption type & \phantom{t}--~\footnotemark \\
|tableposition|\preambleoption & gives a hint about the table caption position & \ref{skips} \\
|textfont|(|+|)   & sets the font of the caption text & \ref{fonts} \\
|textformat|      & sets the format of the caption text & \ref{formats} \\
|twoside|         & selects the two-side mode & \ref{margins} \\
|type|            & sets the caption type \& places a hyperlink anchor & \ref{types} \\
|type*|           & sets the caption type only & \ref{types} \\
|width|           & sets a fixed caption width & \ref{margins} \\
\end{longtable*}
\preambleoptiontext
\footnotetext{The option \texttt{subtype} is explained in the
  \package{subcaption} package documentation.}

\nopagebreak\parbox[t]{\linewidth}{% prevent from page break
\begin{Note*}
  Obsolete options are not listed here. See
  \Ref{caption1} and
  \Ref{caption2} for a list of these options.
\end{Note*}}

\pagebreak[3]
\subsection{Commands}

\begin{longtable*}{lll}
Command                          & & Section \\
\hline
\endhead
|\abovecaptionskip|              & & \ref{skips} \\
%|\AtBeginCaption|               & & \ref{hooks} \\
%|\AtEndCaption|                 & & \ref{hooks} \\
|\belowcaptionskip|              & & \ref{skips} \\
|\caption|                       & & \ref{caption} \\
|\caption*|                      & & \ref{caption} \\
|\captionlistentry|              & & \ref{captionlistentry} \\
|\captionof|                     & & \ref{caption} \\
|\captionof*|                    & & \ref{caption} \\
|\captionsetup|                  & & \ref{captionsetup} \\
|\captionsetup*|                 & & \ref{captionsetup} \\
|\centerfirst|                   & & \ref{justification} \\
|\centerlast|                    & & \ref{justification} \\
|\clearcaptionsetup|             & & \ref{captionsetup} \\
|\clearcaptionsetup*|            & & \ref{captionsetup} \\
|\ContinuedFloat|                & & \ref{ContinuedFloat} \\
|\DeclareCaptionFont|            & & \ref{declare} \\
|\DeclareCaptionFormat|          & & \ref{declare} \\
|\DeclareCaptionFormat*|         & & \ref{declare} \\
|\DeclareCaptionJustification|   & & \ref{declare} \\
|\DeclareCaptionLabelFormat|     & & \ref{declare} \\
|\DeclareCaptionLabelSeparator|  & & \ref{declare} \\
|\DeclareCaptionLabelSeparator*| & & \ref{declare} \\
|\DeclareCaptionListFormat|      & & \ref{declare} \\
|\DeclareCaptionOption|          & & \ref{declare} \\
|\DeclareCaptionStyle|           & & \ref{declare} \\
|\DeclareCaptionSubType|         & & --~\footnotemark \\
|\DeclareCaptionTextFormat|      & & \ref{declare} \\
|\showcaptionsetup|              & & \ref{captionsetup} \\
\end{longtable*}
\footnotetext{\cs{DeclareCaptionSubType} is explained in the
  \package{subcaption} package documentation.}

\clearpage
\subsection{Warnings}

\begin{Warnings}

\Message
  |\caption outside box or environment.|
\Or
  |\captionsetup{type=...} outside box or environment.|
\Or
  |\captionsetup{type*=...} or \captionof outside box|\\
  | or environment.|
\Description
  You have placed a |\caption|, |\caption|\-|of|, or
  |\caption|\-|setup{type=|\x\meta{type}|}| command outside an box, group,
  or environment.
  You should not do this since it could cause some bad side-effects.
  \par\See{\Ref{types} and \Ref{caption}}

\Message
  |\caption will not be redefined since it's already|\\
  |redefined by a document class or package which is|\\
  |unknown to the caption package.|
\Description
  If \thispackage\ detects that some (unknown) document class or
  package has enhanced |\caption|, it will not redefine
  |\caption|, too, since this would simply kill the enhancement.
  As a result some features, like |\caption*|, |\Continued|\-|Float|,
  using the optional argument of |\caption|\-|setup|,
  or the options |list=| and |hypcap=| will not be available.\par
  If you don't care about the original enhancements but would
  like to use the full range of features of \thispackage\ instead
  you can give the unsupported(!) option |compatibility=|\x|false|
  a try and keep your fingers crossed.
  (But you will get the next warning instead.)
  \par\See{\Ref{classes} and \Ref{compatibility}}

\Message
  |Forced redefinition of \caption since the|\\
  |unsupported(!) package option `compatibility=false'|\\
  |was given.|
\Description
  Since you were so keen to specify the option |compatibility=|\x|false|
  \thispackage\ will try to do its best to fulfill your wishes.
  But depending on the document class or other packages you use that
  can end in non-functional features or even errors. So keep your
  fingers crossed!
  \par\See{\Ref{compatibility}}

\Message
  |Hyperref support is turned off because hyperref has|\\
  |stopped early.|
\Description
  If the \package{hyperref} package stops early during loading
  (because of what-ever reason), the \package{hyperref} support of
  \thispackage\ will not be available. As a result you could get
  \package{hyperref} warnings and non-functional hyperlinks to figures
  or tables.
  \par\See{\Ref{hyperref}\,}

\Message
  |Ignoring optional argument [|\meta{pos}|] of \setcapwidth.|
\Description
  \Thispackage\ tries to emulate the \KOMAScript\ commands regarding
  captions as best as it can. But the optional argument of the
  \KOMAScript\ command |\setcapwidth| is not (yet) working if you use
  this package, so if you try to use it anyway, you will get this warning.
  \par\See{\Ref{KOMA}}

\Message
  |Internal Warning: |\meta{warning message}|.|
\Description
  You should never see this warning, either you use a package which redefines
  |figure| or |table| and which is unknown to \thispackage, or this is a bug
  in \thispackage.
  Please send me an e-mail reporting this issue.

\Message
  |\label without proper \caption|
\Description
  Regarding |\label| the floating environments behave differently than its
  non-floating counterparts: The internal reference will not be generated
  at the beginning of the environment, but at |\caption| instead.
  So you have to place the |\label| command either just \emph{after} or
  \emph{inside} the caption text (mandatory argument of |\caption|).

\Message
  |Option `|\meta{option}|' was not in list `|\meta{option list}|'.|
\Description
%\showhyphens{position}
  If you try to remove a specific option of an option list,
  for example with |\clear|\-|caption|\-|setup[po|\-|si|\-|tion]{table}|,
  and this option can not be found inside the option list, you will
  get this warning. If this is not because of a typo and you would like
  to suppress this warning, use |\clear|\-|caption|\-|setup*| instead of
  |\clear|\-|caption|\-|setup|.
  \par\See{\Ref{captionsetup}}

\Message
  |Option list `|\meta{option list}|' undefined.|
\Description
%\showhyphens{format}
  If you try to remove a specific option of an option list,
  for example with |\clear|\-|caption|\-|setup[for|\-|mat]{figure}|,
  and this option list is not defined (yet), you will
  get this warning. If this is not because of a typo and you would like
  to suppress this warning, use |\clear|\-|caption|\-|setup*| instead of
  |\clear|\-|caption|\-|setup|.
  \par\See{\Ref{captionsetup}}

\Message
  |Obsolete option `ignoreLTcapwidth' ignored.|
\Description
  The \package{caption2} package option |ignoreLTcapwidth| will not be
  emulated by this version of \thispackage, usually you can simply wipe it
  away.
  \par\See{\Ref{caption2} and \Ref{longtable}}

\Message
  |`ragged2e' support has been changed. Rerun to get|\\
  |captions right.|
\Description
  The \package{ragged2e} package will only be loaded by \thispackage\ if it
  is actually needed.
  At least two \LaTeX\ runs are needed for that, so on the first run you could
  get this warning.
  Just compile your document again and this warning should go away.
  \par\See{\Ref{justification}}

\Message
  |Reference on page |\meta{page no.}| undefined.|
\Description
  If you use a |twoside| page layout, \thispackage\ needs to track the
  page numbers to get the margins right.
  At least two \LaTeX\ runs are needed for that, on the first run you could get
  this warning.
  Just compile your document again and this warning should go away.
  \par\See{\Ref{margins}}

\Message
  |The caption type was already set to `|\meta{type}|'.|
\Description
  This warning informs you about mixed caption options. For example if
  you use a |\caption|\-|setup{type=|\x|table}| or
  |\caption|\-|of{table}{|\ldots|}| inside a |figure| environment,
  this would result in using both option sets for that specific caption,
  the one for |figure| (specified with |\caption|\-|setup[figure]{|\ldots|}|)
  and the one for |table| (specified with |\caption|\-|setup[table]{|\ldots|}|)
  as well.\par
  {\small(You can suppress this warning by using the starred form
  |\caption|\-|setup*{type=|\x\ldots|}|.)\par}
  \par\See{\Ref{captionsetup}}

\Message
  |The option `hypcap=true' will be ignored for this|\\
  |particular \caption.|
\Description
  \Thispackage\ hasn't found a proper hyperlink anchor for this particular
  caption, so it decides to ignore the setting |hypcap=|\x|true| (which is set by
  default). As a result a link to this caption (e.g.~in the List of Figures,
  or set by |\ref| or |\auto|\-|ref|) will link you to the caption of the figure
  or table, not to the figure or table itself.\par
  This can happen if you use |\caption|\-|of| inside a non-floating environment,
  but also if you use some package which redefines |figure| or |table|, and
  which is unknown to \thispackage.\par
  If this is ok for you but you want to suppress this warning, simply place a
  |\caption|\-|setup{hyp|\-|cap=|\x|false}| just before the |\caption| or
  |\caption|\-|of| command which is causing the warning.
  If this is not ok for you, you can set a hyperlink anchor with
  |\caption|\-|setup{type=|\x\meta{float type}|}| for yourself.
  \par\See{\Ref{hyperref}\,}

\Message
  |Unsupported document class (or package) detected,|\\
  |usage of the caption package is not recommended.|
\Description
  Either the document class you use is unknown to \thispackage, or you
  have included a package in your document which redefines |\@make|\-|caption|
  (which is responsible for typesetting the caption internally) as well.
  This means \thispackage\ will either change the design of captions in an
  unwanted way, or it even refuses to work correctly at all.
  \par\See{\Ref{classes} and \Ref{compatibility}}

\Message
  |Unused \captionsetup[|\meta{type}|].|
\Description
  You have specified options with |\caption|\-|setup[|\meta{type}|]| which are
  not used later on.
  This can be because of a typo in \meta{type},
  or because you use a package which redefines |figure| or |table| and
  which is unknown to \thispackage, or simply because you don't have any usage
  of the environment \meta{type} after this line.
  (If you want to suppress this warning you can use |\caption|\-|setup*|
   instead of |\caption|\-|setup|.)
  \par\See{\Ref{captionsetup}}

\Message
  |Usage of the |\meta{package}| package together with the|\\
  |caption package is strongly not recommended.|\\
  |Instead of loading the |\meta{package}| package you should|\\
  |use the caption package option `tableposition=top'.|
\Description
  The package \meta{package} is dealing with caption skips as well.
  Please decide which one do you actually want to use, \thispackage\ or the
  other one, using both can lead to wrong skips above or below the caption.
  \par\See{\Ref{skips}}

\end{Warnings}

\pagebreak[3]
\subsection{Errors}

\begin{Warnings}

\Message
  |Argument of \@caption has an extra }.|
\Or
  |Paragraph ended before \@caption was complete.|
\Description
  If you want to typeset something special (like a tabular) as caption,
  you need to give an optional argument to |\caption| resp.~|\caption|\-|of|
  for the List of Figures resp.~List of Tables, too, even if you don't
  use such list.
  \par\See{\Ref{caption}, \Ref{lists}, and \Ref{hyperref}\,}

\Message
  |\caption outside float.|
\Description
  You have placed a |\caption| command outside a floating environment, or
  a |long|\-|table| or |wrap|\-|figure|.
  If this is what you want to, please use either
  |\caption|\-|setup{type=|\x\meta{type}|}| $+$ |\caption| or
  |\caption|\-|of|.
  \par\See{\Ref{caption}}

\Message
  |\ContinuedFloat outside float.|
\Description
  You have placed a |\Continued|\-|Float| command outside a floating
  environment. % or |long|\-|table|.
  If this is what you want to, please use the combination
  |\caption|\-|setup{type=|\x\meta{type}|}| $+$ |\Continued|\-|Float|.\par
  Please note that |\Continued|\-|Float| inside a |long|\-|table| is not
  working, but maybe using the |long|\-|table*| environment, which
  typesets a |long|\-|table| without incrementing the |table| counter,
  if sufficient for you.
  \par\See{\Ref{ContinuedFloat} and \Ref{longtable}}

\Message
  |Continued `|\meta{type}|' after `|\meta{type}|'.|
\Description
  Continued figures or tables are not allowed to be interrupted by a floating
  environment (or |long|\-|table|) of another type,
  e.g.~a table between a figure and a continued figure.
  \par\See{\Ref{ContinuedFloat}}

\Message
  |For a successful cooperation we need at least version|\\
  |`|\meta{date}|' of package |\meta{package}|, but only version|\\
  |`|\meta{old-date}|' is available.|
\Description
  \Thispackage\ does not work with such an outdated package, please update it
  to a more recent version, at least to the one requested.

\Message
  |Internal Error: |\meta{error message}|.|
\Description
  You should never see this error.
  Please send me an e-mail reporting this issue.

\Message
  |No float type '|\meta{type}|' defined.|
\Description
  The \meta{type} you have specified in |\caption|\-|setup{type=|\x\meta{type}|}|,
  |\caption|\-|of|\marg{type}, or |\Declare|\-|Caption|\-|Sub|\-|Type| is
  not defined.
  \meta{type} should be either `\texttt{figure}' or `\texttt{table}', or
  any other floating environment defined with
  |\Declare|\-|Floating|\-|Environment| offered by the \package{newfloat} package,
  |\new|\-|float| offered by the \package{float} package\cite{float}, or
  |\Declare|\-|New|\-|Float|\-|Type| offered by the \package{floatrow} package\cite{floatrow}.

\Message
  |Not allowed in longtable* environment.|
\Description
  The usage of |\caption| is not allowed inside the |long|\-|table*| environment.
  Please use either |\caption*| for a caption without label or use the regular
  |long|\-|table| environment instead.

\Message
  |Not available in compatibility mode.|
\Description
  The feature required is not supported in compatibility mode.
  `compatibility mode' means that \thispackage\ has detected either an
  incompatible document class or an incompatible package which also
  extents the |\caption| command. Leaving the original extension intact,
  some features of \thispackage\ are not supported.
  \par\See{\Ref{compatibility}}

\Message
  |Only one \caption can be placed in this environment.|
\Description
  Inside the environments offered by the \package{fltpage} and \package{sidecap}
  package only \emph{one} caption can be placed. (This is due implementation design.)

\Message
  |Option clash for package caption.|
\Or[but sometimes also]
  |Missing \begin{document}.|
\Description
  \Thispackage\ has already been loaded by some other \LaTeX\ package,
  so you can't do that again specifying different options.
  A candidate causing this could be the \package{subfig} package; if this is
  the case, please load \thispackage\ \emph{before} the \package{subfig}
  package or specify the option |caption=|\x|false| while loading the
  \package{subfig} package.
  \par\See{the \package{subfig} package\cite{subfig} documentation}

\Message
  |Paragraph ended before \caption@makecurrent was complete.|
\Or
  |Paragraph ended before \caption@prepareanchor was complete.|
\Description
  If you want to typeset multi-paragraph captions, you need to give an
  optional argument to |\caption| resp.~|\caption|\-|of| for the List of
  Figures resp.~List of Tables, too, even if you don't use such list.
  \par\See{\Ref{caption}, \Ref{lists}, and \Ref{hyperref}\,}

\Message
  |Something's wrong--perhaps a missing \caption|\\
  |in the last figure or table.|
\Description
  It seems that you have used a |\sub|\-|caption| command (or a different one
  which has typeset a sub-caption) without a corresponding |\caption| command.
  This is not supported.

\Message
  |The option `labelsep=|\meta{name}|' does not work|\\
  |with `format=hang'.|
\Or
  |The option `labelsep=|\meta{name}|' does not work|\\
  |with \setcaphanging (which is set by default).|
\Description
  A caption label separator which contains a |\\| command (like
  |labelsep=|\x|newline|) can not be combined with a hanging caption format
  (like |format=|\x|hang|).
  Please select either another caption label separator
  (e.g.~|labelsep=|\x|colon|) or another caption format
  (e.g.~|format=|\x|plain|).
  \par\See{\Ref{formats} resp.~\Ref{KOMA}}

\Message
  |The package option `caption=false' is obsolete.|\\
  |Please pass this option to the subfig package instead|\\
  |and do *not* load the caption package anymore.|
\Description
  You have specified the option |caption=|\x|false|. This used to be a workaround
  for not using the whole \package{caption} package (leaving the caption stuff
  offered by the document class or other packages intact), but keeping the
  \package{subfig} package working. This mechanism is obsolete and not offered
  anymore, please pass this option to the \package{subfig} package instead and
  do not load \thispackage\ anymore.
  \par\See{the \package{subfig} package\cite{subfig} documentation}

\Message
  |Undefined boolean value `|\meta{value}|'.|
\Description
  You tried to set a boolean option (like |singlelinecheck=| or |hypcap=|)
  with an improper value. Only |false|, |no|, |off|, |0| or |true|, |yes|,
  |on|, |1| is allowed here.

\Message
  |Undefined format `|\meta{name}|'.|
\Description
  You tried to set a caption format which does not exists.
  Maybe a typo!?
  \par\See{\Ref{formats}}

\Message
  |Undefined label format `|\meta{name}|'.|
\Description
  You tried to set a caption label format which does not exists.
  Maybe a typo!?
  \par\See{\Ref{formats}}

\Message
  |Undefined label separator `|\meta{name}|'.|
\Description
  You tried to set a caption label separator which does not exists.
  Maybe a typo!?
  \par\See{\Ref{formats}}

\Message
  |Undefined list format `|\meta{name}|'.|
\Description
  You tried to set a caption list-of format which does not exists.
  Maybe a typo!?
  \par\See{\Ref{lists}}

\Message
  |Undefined position `|\meta{name}|'.|
\Description
  You tried to set a caption position with an improper value.
  Maybe a typo!?
  \par\See{\Ref{skips}}

\Message
  |Undefined style `|\meta{name}|'.|
\Description
  You tried to set a caption style which does not exists.
  Maybe a typo!?
  \par\See{\Ref{style}}

\Message
  |Usage of the `position' option is incompatible|\\
  |to the `|\meta{package}|' package.|
\Description
% (ftcap,nonfloat,topcapt)
  The given package is dealing with caption skips as well.
  Please decide which one do you actually want to use: The |position=| option
  of \thispackage\ or the mechanism of the other one; using both this way
  would lead to wrong skips above or below the caption and is therefore
  not supported.
  \par\See{\Ref{skips}}

\Message
  |Undefined text format `|\meta{name}|'.|
\Description
  You tried to set a caption text format which does not exists.
  Maybe a typo!?
  \par\See{\Ref{formats}}

\Message
  |You can't use both, the (obsolete) caption2 *and*|\\
  |the (current) caption package.|
\Description
  This error message says it all, you simply can't do that.
  Please use only \thispackage.
  \par\See{\Ref{caption2}}

\end{Warnings}

% --------------------------------------------------------------------------- %

\clearpage
\section{Version history}
\label{history}

The version $1.0$ was written in 1994 and was offering a handful of options
to customize the look \& feel of the captions.
Furthermore this version already supported the \package{rotating} and
\package{subfigure} packages.
Version $1.1$ introduced the |center|\-|last| option;
version $1.2$ added the support of the \package{float} package.
Version $1,3$ offered a better support of the \package{subfigure} package,
while version $1.4$ came with the new option |nooneline|.

The \package{caption2} package $2.0$ was an experimental side-version of
the regular \package{caption} package. It was made public as beta test version
without proper documentation in 1995 because of the strong demand for new
features and adaptations to additional packages like the \package{longtable}
package.
Furthermore it offered a revised support of the \package{subfigure} package.
(An improved version $2.1$ was offered in 2002.)

In 2003 I finally found some time, so a new regular release $3.0$
of the \package{caption} package could be build in cooperation with
Frank Mittelbach and Steven Cochran.
It was released in December 2003 and superseded the neglected
\package{caption2} package.
Main parts were re-written and it provided a complete re-work of
the user interface.
Furthermore it supported the \package{hyperref}, \package{hypcap},
\package{listings}, \package{sidecap}, and \package{supertabular} packages
additionally.

While all the previous versions were designed for usage with the standard
\LaTeX\ document classes \class{article}, \class{report}, and \class{book},
the current version $3.1$ released in 2007 also supports the \AmS,
\KOMAScript, \NTG, and \SmF\ document classes, and the \class{beamer} class
as well.
Furthermore it was adapted to the \package{frenchb} Babel option,
the \package{frenchle} \& \package{frenchpro} packages,
and the \package{floatflt}, \package{fltpage}, \package{picinpar},
\package{picins}, \package{setspace}, \package{threeparttable},
and \package{wrapfig} packages.
New options and commands were introduced as well, among other things
|font+|, |figure|\-|within| \& |table|\-|within|,
|list| \& |list|\-|format|, |max|\-|margin| \& |min|\-|margin|,
|\caption|\-|list|\-|entry|, |\Declare|\-|Caption|\-|List|\-|Format|.
Further benefits are the new compatibility check \see*{\Ref{compatibility}},
the new ``hypcap'' feature \see*{\Ref{hyperref}\,}, and the sub-caption
feature \see*{\package{subcaption} package documentation}.

% --------------------------------------------------------------------------- %

\clearpage
\section{Compatibility to older versions}

\subsection{caption v\texorpdfstring{$1.x$}{1.x}}
\label{caption1}

This version of \thispackage\ still supports the old options
and commands provided by the version $1.0$ to $1.4$ of this package.
So there shouldn't occur any problems compiling old documents, but please
don't mix old options and commands with the new ones. This isn't supported
and can cause ugly side effects.

Here comes a short oversight of the obsolete options and how
they have been replaced within this version of \thispackage:

\begin{center}\small
\begin{tabular}{ll}
\package{caption} \version{1.x} & \package{caption} \version{3.x}\\
\hline
%\endhead
|normal|        & |format=plain|\\
|hang|          & |format=hang|\\
|isu|           & |format=hang|\\
|center|        & |justification=centering|\\
|centerlast|    & |justification=centerlast|\\
|nooneline|     & |singlelinecheck=off|\\
|scriptsize|    & |font=scriptsize|\\
|footnotesize|  & |font=footnotesize|\\
|small|         & |font=small|\\
|normalsize|    & |font=normalsize|\\
|large|         & |font=large|\\
|Large|         & |font=Large|\\
|up|            & |labelfont=up|\\
|it|            & |labelfont=it|\\
|sl|            & |labelfont=sl|\\
|sc|            & |labelfont=sc|\\
|md|            & |labelfont=md|\\
|bf|            & |labelfont=bf|\\
|rm|            & |labelfont=rm|\\
|sf|            & |labelfont=sf|\\
|tt|            & |labelfont=tt|\\
\end{tabular}
\end{center}

Beside the options for setting up the desired font there were also
the commands |\caption|\-|size| resp.~|\caption|\-|font| and
|\caption|\-|label|\-|font| who could be redefined with
|\re|\-|new|\-|command| and allowed an alternate and more flexible way to
change the font used for captions.
This mechanism was replaced by the commands
\begin{quote}
  |\DeclareCaptionFont{|\ldots|}{|\ldots|}|\qquad and\\
  |\captionsetup{font=|\ldots|,labelfont=|\ldots|}|\qquad.
\end{quote}
\SeeUserDefined

Setting the margin for captions was done in \version{1.x} with
\begin{quote}
  |\setlength{\captionmargin}{|\ldots|}|\quad.
\end{quote}
This was replaced by
\begin{quote}
  |\captionsetup{margin=|\ldots|}|\qquad.
\end{quote}
\See{\Ref{margins}}

For example the old-style code
\begin{quote}
  |\usepackage[hang,bf]{caption}|\\
  |\renewcommand\captionfont{\small\sffamily}|\\
  |\setlength\captionmargin{10pt}|
\end{quote}
will still work fine, but should be written today as
\begin{quote}
  |\usepackage[format=hang,labelfont=bf,font={small,sf},|\\
  |            margin=10pt]{caption}|
\end{quote}
or
\begin{quote}
  |\usepackage{caption}|\\
  |\captionsetup{format=hang,labelfont=bf,font={small,sf},|\\
  |              margin=10pt}|\qquad.
\end{quote}

The quite exotic option |ruled| which allowed a partial usage of
the caption settings for |ruled| floats defined with the
\package{float} package will be emulated by this version of the
caption package, too.
But using this option is not recommended anymore since this
version of \thispackage\ offers a more flexible way
for changing the captions of these floating environments:
\begin{quote}
  |\DeclareCaptionStyle{ruled}{|\ldots|}|
\end{quote}
resp.
\begin{quote}
  |\captionsetup[ruled]{|\ldots|}|\qquad.
\end{quote}
\SeeUserDefined[, \Ref{captionsetup}, and \Ref{float}]

\subsection{caption2 v\texorpdfstring{$2.x$}{2.x}}
\label{caption2}

Although they do very similar stuff, the packages \package{caption} and
its experimental and now obsolete variant \package{caption2} have a
very different implementation design.
Therefore a full compatibility could not be offered.
For that reason you will still find a file called |caption2.sty| in
this package distribution, so old documents using the \package{caption2}
package will still compile fine.

Newly created documents should use the actual version of
\thispackage\ instead. In most cases it's sufficient to replace the command
\begin{quote}
  |\usepackage[...]{caption2}|
\end{quote}
by
\begin{quote}
  |\usepackage[...]{caption}|\qquad.
\end{quote}
But some options and commands will not be emulated, so you can get
error messages afterwards. This section will hopefully help you removing
these errors. If you have problems migrating from \package{caption2}
to \package{caption} please don't hesitate to send me an e-mail asking
for help.

In addition to the obsolete options shown in the last section
these ones will be emulated, too:

\begin{center}\small
\begin{tabular}{ll}
\package{caption2} \version{2.x} & \package{caption} \version{3.x}\\
\hline
%\endhead
|flushleft|   & |justification=raggedright|\\
|flushright|  & |justification=raggedleft|\\
|oneline|     & |singlelinecheck=on|\\
\end{tabular}
\end{center}

Setting the margin for captions was done in \version{2.x} with
\begin{quote}\leavevmode\hbox{%
  |\setcaptionmargin{|\ldots|}| resp.
  |\setcaptionwidth{|\ldots|}|\quad.
}\end{quote}
This was replaced by
\begin{quote}\leavevmode\hbox{%
  |\captionsetup{margin=|\ldots|}| resp.
  |\captionsetup{width=|\ldots|}|\quad.
}\end{quote}
\See{\Ref{margins}}

Setting an indention was done in \version{2.x} with
\begin{quote}
  |\captionstyle{indent}|\\
  |\setlength\captionindent{|\ldots|}|\quad.
\end{quote}
\pagebreak[3]
This is now done with
\nopagebreak[3]
\begin{quote}
  |\captionsetup{format=plain,indention=|\ldots|}|\quad.
\end{quote}

The so-called single-line-check was controlled by the commands
|\oneline|\-|captions|\-|false| (for switching the check off) and
|\oneline|\-|captions|\-|true| (for switching the check on).
This was replaced by
|\caption|\-|setup{single|\-|line|\-|check=|\x|off}|
resp.
|\caption|\-|setup{single|\-|line|\-|check=|\x|on}|\quad.
\See{\Ref{justification}}

The commands
\begin{quote}
  |\captionlabeldelim|, |\captionlabelsep|, |\captionstyle|,\\
  |\defcaptionstyle|, |\newcaptionstyle|, and |\renewcaptionstyle|
\end{quote}
do not have a simple replacement and therefore will not be emulated
by this version of \thispackage. (So using them will
cause error messages.) Rewriting such code is not always easy and
straight-ahead, but by conscientious reading of this manual you should
be able to find appropriate options and commands instead.

\iffalse
... (some examples)
\fi

The \version{2.x} option |ignore|\-|LT|\-|cap|\-|width| do not have a
replacement, too.
But in most cases you can simply drop using that option because
in this version of \thispackage\ the value of |\LT|\-|cap|\-|width|
will be ignored anyway (unless you set it to a different value than the
default one which is |4in|).
\See{\Ref{longtable}}

\subsection{caption v\texorpdfstring{$3.0$}{3.0}}
\label{caption3}

%\NEWdescription{v3.1}
\Thispackage\ \version{3.0} did not support any document classes
other than the standard \LaTeX\ ones: \class{article}, \class{report},
and \class{book}.
Therefore the default settings used to be fixed, but now \thispackage{}
supports more classes, therefore they are now set in dependence on the
document class used.

For example: While in \version{3.0} the default caption
|jus|\-|ti|\-|fi|\-|ca|\-|tion| used to be always |jus|\-|ti|\-|fied|,
it's now still |jus|\-|ti|\-|fied| when using one of the standard document
classes, but |ragged|\-|right| will be used as default format when used
with the \class{beamer} document class.

%An easy way to select the `old' defaults---if desired---is using the option
An easy way to select the `old' defaults is using the option
|style=base| when loading \thispackage\ (or later on using |\caption|\-|setup|).

\medskip

Another new feature is the automatic check for compatibility; if an
incompatibility will be found, a warning message will be issued and if a strong
incompatibility was found some features of \thispackage\ will be disabled.
Please note that only this check was added to \thispackage, so if you get
such warning message, the previous versions of \thispackage\ were incompatible
as well, but did not issued such warning, they were ``only'' having
side-effects or causing problems.
So these warnings only say that some bad side effects or problems could
happen, but not that they actually will happen.
Anyway, if you use \thispackage\ in such circumstances,
you should use it with care.

\bigskip

\begin{Note*}
\Thispackage\ \version{3.0} offered the option |caption=|\x|false|,
which used to be a workaround for not using the whole \package{caption} package
(leaving the caption stuff offered by the document class or other packages
intact), but keeping the \package{subfig} package\cite{subfig} working.
This mechanism is obsolete and not offered anymore, please pass this option
to the \package{subfig} package instead and do not load \thispackage\ anymore.
%\par\See{the \package{subfig} package documentation}
\end{Note*}

% --------------------------------------------------------------------------- %

\clearpage
\begin{thebibliography}{99}

  \bibitem{TLC2}
  Frank Mittelbach and Michel Goossens:\\
  \newblock {\em The {\LaTeX} Companion (2nd.~Ed.)},\\
  \newblock Addison-Wesley, 2004.

  \bibitem{beamer}
  Till Tantau:\\
  \href{http://www.ctan.org/pkg/beamer}%
       {\emph{User Guide to the Beamer Class, Version 3.07}},\\
  March 11, 2007

  \bibitem{KOMAScript}
  Markus Kohm \& Jens-Uwe-Morawski:\\
  \href{http://www.ctan.org/pkg/koma-script}%
       {\emph{KOMA-Script -- a versatile \LaTeXe\ bundle}},\\
  2007-01-09

  \bibitem{NTG}
  Victor Eijkhout:\\
  \href{http://www.ctan.org/pkg/ntgclass}%
       {\emph{An introduction to the Dutch \LaTeX\ document classes}},\\
  3 September 1989

  \bibitem{algorithms}
  Rog\'erio Brito:\\
  \href{http://www.ctan.org/pkg/algorithms}%
       {\emph{Algorithms}},\\
  June 2, 2006

\iffalse
  \bibitem{algorithm2e}
  Christophe Fiorio:\\
  \href{http://www.ctan.org/pkg/algorithm2e}%
       {\emph{algorithm2e.sty -- package for algorithms}},\\
  March 11, 2007
\fi

  \bibitem{float}
  Anselm Lingnau:\\
  \href{http://www.ctan.org/pkg/float}%
       {\emph{An Improved Environment for Floats}},\\
  2001/11/08

  \bibitem{floatflt}
  Mats Dahlgren:\\
  \href{http://www.ctan.org/pkg/floatflt}%
       {\emph{Welcome to the floatflt package}},\\
  1998/06/05

  \bibitem{floatrow}
  Olga Lapko:\\
  \href{http://www.ctan.org/pkg/floatrow}%
       {\emph{The floatrow package documentation}},\\
  2007/12/24

  \bibitem{fltpage}
  Sebastian Gross:\\
  \href{http://www.ctan.org/pkg/fltpage}%
       {\emph{Welcome to the beta test of fltpage package!}},\\
  1998/11/13

  \bibitem{hyperref}
  Sebastian Rahtz \& Heiko Oberdiek:\\
  \href{http://www.ctan.org/pkg/hyperref}%
       {\emph{Hypertext marks in \LaTeX}},\\
  November 12, 2007

  \bibitem{hypcap}
  Heiko Oberdiek:\\
  \href{http://www.ctan.org/pkg/hypcap}%
       {\emph{The hypcap package -- Adjusting anchors of captions}},\\
  2007/04/09

  \bibitem{listings}
  Carsten Heinz \& Brooks Moses:\\
  \href{http://www.ctan.org/pkg/listings}%
       {\emph{The Listings Package}},\\
  2007/02/22

  \bibitem{longtable}
  David Carlisle:\\
  \href{http://www.ctan.org/pkg/longtable}%
       {\emph{The longtable package}},\\
  2004/02/01

  \bibitem{picinpar}
  Friedhelm Sowa:\\
  \href{http://www.ctan.org/pkg/picinpar}%
       {\emph{Pictures in Paragraphs}},\\
  July 13, 1993

  \bibitem{picins}
  Joachim Bleser and Edmund Lang:\\
  \href{http://www.ctan.org/pkg/picins}%
       {\emph{PicIns-Benutzerhandbuch Version 3.0}},\\
  September~1992

  \bibitem{rotating}
  Sebastian Rahtz and Leonor Barroca:\\
  \href{http://www.ctan.org/pkg/rotating}%
       {\emph{A style option for rotated objects in \LaTeX}},\\
  1997/09/26

  \bibitem{setspace}
  Erica M. S. Harris \& Geoffrey Tobin:\\
  \href{http://www.ctan.org/pkg/setspace}%
       {\emph{LaTeX Document Package ``setspace''}},\\
  1 December 2000

  \bibitem{sidecap}
  Rolf Niepraschk \& Hubert G\"a\ss lein:\\
  \href{http://www.ctan.org/pkg/sidecap}%
       {\emph{The sidecap package}},\\
  2003/06/06

  \bibitem{subfigure}
  Steven D. Cochran:\\
  \href{http://www.ctan.org/pkg/subfigure}%
       {\emph{The subfigure package}},\\
  2002/07/02

  \bibitem{subfig}
  Steven D. Cochran:\\
  \href{http://www.ctan.org/pkg/subfig}%
       {\emph{The subfig package}},\\
  2005/07/05

  \bibitem{supertabular}
  Johannes Braams and Theo Jurriens:\\
  \href{http://www.ctan.org/pkg/supertabular}%
       {\emph{The supertabular environment}},\\
  2002/07/19

  \bibitem{threeparttable}
  Donald Arseneau:\\
  \href{http://www.ctan.org/pkg/threeparttable}%
       {\emph{Three part tables: title, tabular environment, notes}},\\
  2003/06/13

  \bibitem{wrapfig}
  Donald Arseneau:\\
  \href{http://www.ctan.org/pkg/wrapfig}%
       {\emph{WRAPFIG.STY ver 3.6}},\\
  2003/01/31

  \bibitem{xtab}
  Peter Wilson:\\
  \href{http://www.ctan.org/pkg/xtab}%
       {\emph{The xtab package}},\\
  2004/05/24

\end{thebibliography}

% --------------------------------------------------------------------------- %

\end{document}
