% \iffalse meta-comment
%
% Copyright (C) 2012 by Edorta Ibarra
%
% This file may be distributed and/or modified under the
% conditions of the LaTeX Project Public License, either
% version 1.2 of this license or (at your option) any later
% version. The latest version of this license is in:
%
%     http://www.latex-project.org/lppl.txt
%
% and version 1.2 or later is part of all the distributions of
% LaTeX version 1999/12/01 or later.
%
% \fi
%
% \iffalse
%
%    \begin{macrocode}
%<package>\NeedsTeXFormat{LaTeX2e}[1999/12/01]
%<package>\ProvidesPackage{basque-date}
%<package>   [2012/05/15 v1.05 basque-date Package]
%    \end{macrocode}
%<*driver>
\documentclass{ltxdoc}
\usepackage{basque-date}
\EnableCrossrefs
\CodelineIndex
\RecordChanges
\begin{document}
  \DocInput{basque-date.dtx}
\end{document}
%</driver>
% \fi
% \CheckSum{0}
%% \CharacterTable
%%  {Upper-case    \A\B\C\D\E\F\G\H\I\J\K\L\M\N\O\P\Q\R\S\T\U\V\W\X\Y\Z
%%   Lower-case    \a\b\c\d\e\f\g\h\i\j\k\l\m\n\o\p\q\r\s\t\u\v\w\x\y\z
%%   Digits        \0\1\2\3\4\5\6\7\8\9
%%   Exclamation   \!     Double quote  \"     Hash (number) \#
%%   Dollar        \$     Percent       \%     Ampersand     \&
%%   Acute accent  \'     Left paren    \(     Right paren   \)
%%   Asterisk      \*     Plus          \+     Comma         \,
%%   Minus         \-     Point         \.     Solidus       \/
%%   Colon         \:     Semicolon     \;     Less than     \<
%%   Equals        \=     Greater than  \>     Question mark \?
%%   Commercial at \@     Left bracket  \[     Backslash     \\
%%   Right bracket \]     Circumflex    \^     Underscore    \_
%%   Grave accent  \`     Left brace    \{     Vertical bar  \|
%%   Right brace   \}     Tilde         \~}
%%
% \changes{v1.00}{12/05/7}{First version}
% \changes{v1.02}{12/05/15}{Some bugs corrected}
% \changes{v1.05}{12/05/16}{First public version}
%
% \GetFileInfo{basque-date.sty}
%
% \title{The \textsf{basque-date} package\thanks{This file has version 
% \fileversion\ last revised \filedate.}}
% \author{Edorta Ibarra\\\texttt{gautegiz@yahoo.es}}
%
% \maketitle

% \begin{abstract}
% \noindent This package provides two \LaTeX\ commands to print the current 
% date in Basque according to the correct forms ruled by The Basque Language 
% Academy (\textit{Eus\-kaltzaindia}). These commands automatically solve the 
% complex decli\-nation issues of numbers in Basque.
% \end{abstract}
%
% \tableofcontents
%
% \section{Commands}
% 
% According to The Basque Language Academy (\textit{Euskaltzaindia}), there 
% are various correct forms to display the date in Basque\footnote{\textit{37. 
% araua. Data nola adierazi}. http://euskaltzaindia.net/dok/arauak/Araua\_0037.pdf}. 
% The most used form, which includes the name of the location where the document or 
% text has been written, is the following:
%
% \begin{center}
% \textit{Durango(n), 1983ko martxoaren 7a(n)},
% \end{center}
% where the inessive case \textit{``n"} is optional. It must be taken into account 
% that the declination of numbers in Basque is not straightforward. In this sense, 
% the following two commands \verb|\eusdata| and \verb|\eusdatainesibo| are provided by 
% the package \texttt{basque-date}.

% \DescribeMacro{\eusdata} The command \verb|\eusdata| prints the current date without 
% the inessive case, e.g.
%
% \begin{quote}
% \verb|Durango, \eusdata|
% \end{quote}
%
% \pagebreak
%
% \begin{quote}
% \verb|\date{Gernika-Lumo, \eusdata.}|
% \end{quote}
%
% \DescribeMacro{\eusdatainesibo} Similarly, the command \verb|\eusdatainesibo| prints 
% the current date including the inessive case, e.g.
%
% \begin{quote}
% \verb|Durangon, \eusdatainesibo|
%
% \verb|\date{Gernika-Lumon, \eusdatainesibo.}|
% \end{quote}
%
% It is important to note that another way to print the date
% in Basque is using \verb|\usepackage[basque]{babel}|. However, the aforementioned
% solution uses a form that does not follow the recommendations of the Basque
% Language Academy.
%
% \section{Calling the package}
%
% The package \verb|basque-date| is called using the \verb|\usepackage|
% command:\\ \verb|\usepackage{basque-date}|. 
%
% No package options are provided in the current \verb|basque-date| version.
%
% \appendix
%
% \section{License}
%
% Copyright 2012 Edorta Ibarra. 
%
% This program can be redistributed and/or modified under the terms of the 
% \LaTeX\ Project Public License Distributed from CTAN archives in directory
% macros/latex/basee/lppl.txe; either version 1.2 of the License, or any later 
% version.
%
% \section{Implementation}
%
% At the beginning of the code, the first two digits of the year are
% extracted\footnote{As it can be seen from the code, this algorithm is 
% intended for the current century. For other centuries, the modification 
% to be performed is straightforward.}. This information will be used to 
% select the correct declination for the year.
%    \begin{macrocode}
\newcounter{urtea}
\setcounter{urtea}{\year}
\addtocounter{urtea}{-2000} 
%    \end{macrocode}
% \DescribeMacro{\eusdata} In the next step, the \verb|\eusdata| command is defined 
% taking into account the Basque declination rules for year, month and day:
%    \begin{macrocode}
\newcommand\eusdata{{\number\year}\ifcase\arabic{urtea}
  %Declination for the year
  %00-20
		  ko \or eko \or ko \or ko \or ko \or eko \or ko
		  \or ko \or ko \or ko \or eko \or ko \or ko 
		  \or ko \or ko \or eko \or ko \or ko \or ko  
		  \or ko \or ko \or 
		  %21-40
		  eko \or ko \or ko \or ko \or eko \or ko
		  \or ko \or ko \or ko \or eko \or ko \or ko 
		  \or ko \or ko \or eko \or ko \or ko \or ko  
		  \or ko \or ko \or
		  %41-60
		  eko \or ko \or ko \or ko \or eko \or ko
		  \or ko \or ko \or ko \or eko \or ko \or ko 
		  \or ko \or ko \or eko \or ko \or ko \or ko  
		  \or ko \or ko \or
		  %61-80
		  eko \or ko \or ko \or ko \or eko \or ko
		  \or ko \or ko \or ko \or eko \or ko \or ko 
		  \or ko \or ko \or eko \or ko \or ko \or ko  
		  \or ko \or ko \or
		  %81-100
		  eko \or ko \or ko \or ko \or eko \or ko
		  \or ko \or ko \or ko \or eko \or ko \or ko 
		  \or ko \or ko \or eko \or ko \or ko \or ko  
		  \or ko \or eko								
		\fi
\ifcase\month\or %months in Basque (declined)
  urtarrilaren\or otsailaren\or martxoaren\or apirilaren\or
  maiatzaren\or ekainaren\or uztailaren\or abuztuaren\or
  irailaren\or urriaren\or azaroaren\or
  abenduaren\fi~\number\day 
\ifcase\number\day %declination for the day
  \or a\or a\or a\or a\or a\or a\or a\or a\or a\or a\or \or
  a\or a\or a\or a\or a\or a\or a\or a\or a\or a\or a\or a\or
  a\or a\or a\or a\or a\or a\or a\or\fi} 
%    \end{macrocode}
% \DescribeMacro{\eusdatainesibo} Finally, the \verb|\eusdatainesibo| 
% command is defined in a similar way: 
%    \begin{macrocode}        
\newcommand\eusdatainesibo{{\number\year}\ifcase\arabic{urtea}
  %Declination for the year
  %00-20
		  ko \or eko \or ko \or ko \or ko \or eko \or ko
		  \or ko \or ko \or ko \or eko \or ko \or ko 
		  \or ko \or ko \or eko \or ko \or ko \or ko  
		  \or ko \or ko \or
		  %21-40
		  eko \or ko \or ko \or ko \or eko \or ko
		  \or ko \or ko \or ko \or eko \or ko \or ko 
		  \or ko \or ko \or eko \or ko \or ko \or ko  
		  \or ko \or ko \or
		  %41-60
		  eko \or ko \or ko \or ko \or eko \or ko
		  \or ko \or ko \or ko \or eko \or ko \or ko 
		  \or ko \or ko \or eko \or ko \or ko \or ko  
		  \or ko \or ko \or
		  %61-80
		  eko \or ko \or ko \or ko \or eko \or ko
		  \or ko \or ko \or ko \or eko \or ko \or ko 
		  \or ko \or ko \or eko \or ko \or ko \or ko  
		  \or ko \or ko \or
		  %81-100
		  eko \or ko \or ko \or ko \or eko \or ko
		  \or ko \or ko \or ko \or eko \or ko \or ko 
		  \or ko \or ko \or eko \or ko \or ko \or ko  
		  \or ko \or eko 								
		  \fi
\ifcase\month\or %months in Basque (declined)
  urtarrilaren\or otsailaren\or martxoaren\or apirilaren\or
  maiatzaren\or ekainaren\or uztailaren\or abuztuaren\or
  irailaren\or urriaren\or azaroaren\or
  abenduaren\fi~\number\day 
  \ifcase\number\day %declination for the day (inessive case)
  \or ean\or an\or an\or an\or ean\or an\or an\or an\or an\or ean\or
  n\or an\or an\or an\or ean\or an\or an\or an\or an\or an\or ean\or
  an\or an\or an\or ean\or an\or an\or an\or an\or ean\or n\fi}        
%    \end{macrocode}
%
% \Finale
\endinput