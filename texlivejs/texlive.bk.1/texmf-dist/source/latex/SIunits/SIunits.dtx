% \iffalse meta-comment
% Note: This package is now maintained by Joseph Wright
% (joseph.wright@morningstar2.co.uk)
% Copyright 2007 Joseph Wright
% ----------------------------------------------------------------------
%<*ID>
%%% ====================================================================
%%%  @LaTeX-package-file{
%%% author    = "Heldoorn M.",
%%% version   = "$Revision: 1.33 $",
%%% date      = "$Date: 2002/08/01 11:20:00 $",
%%% filename  = "SIunits.dtx",
%%% address   = "Marcel Heldoorn
%%%              Kennedylaan 24
%%%              NL-3844 BC  HARDERWIJK
%%%              The Netherlands",
%%% telephone = "+31 341 427983",
%%% FAX       = "+31 71 5276782",
%%% email     = "SIunits@webschool.nl",
%%% codetable = "ISO/ASCII",
%%% keywords  = "SI, units, International System of Units,
%%%              physical unit",
%%% supported = "yes",
%%% abstract  = "This package defines commands for using
%%%              standard SI units in all your texts. The
%%%              package provides different options for the
%%%              uniform typesetting and spacing of the units.",
%%% copyright = "Copyright (C) 2001 Marcel Heldoorn.
%%%
%%% This program may be distributed and/or modified under the
%%% conditions of the LaTeX Project Public License, either version 1.3
%%% of this license or (at your option) any later version.
%%% The latest version of this license is in
%%%   http://www.latex-project.org/lppl.txt
%%% and version 1.3 or later is part of all distributions of LaTeX
%%% version 2003/12/01 or later.
%%% }
%%% ====================================================================
%</ID>
%   \fi
% \CheckSum{1807}
%   \iffalse
%<+package|binary>\def\SIunits@RCS$#1: #2 #3${#2}
%<+package|binary>\def\filename{SIunits.dtx}%Source File Name
%<+package|binary>\xdef\fileversion{\SIunits@RCS$Revision: 1.36 $}%Revision generated by CS-RCS
%<+package|binary>\xdef\filedate{\SIunits@RCS$Date: 2007/12/02 12:00:00 $}%Date generated by CS-RCS
%<+package|binary>\let\docversion=\fileversion
%<+package|binary>\let\docdate=\filedate
%<+package|binary>\NeedsTeXFormat{LaTeX2e}[1997/12/01]
%<+package>\ProvidesPackage{SIunits}
%<+binary>\ProvidesPackage{binary}
%<+package|binary>  [\filedate\space v\fileversion\space
%<+package>     Support for the International System of units (MH)]
%<+binary>      Support for binary prefixes and units (MH)]
%<+binary>\RequirePackageWithOptions{SIunits}
%<+package|binary> \def\packagemessage{}
%   \fi
% \CharacterTable
%  {Upper-case        \A\B\C\D\E\F\G\H\I\J\K\L\M\N\O\P\Q\R\S\T\U\V\W\X\Y\Z
%   Lower-case        \a\b\c\d\e\f\g\h\i\j\k\l\m\n\o\p\q\r\s\t\u\v\w\x\y\z
%   Digits            \0\1\2\3\4\5\6\7\8\9
%   Exclamation       \!  Double quote  \"   Hash (number)      \#
%   Dollar            \$  Percent       \%   Ampersand          \&
%   Acute accent      \'  Left paren    \(   Right paren        \)
%   Asterisk          \*  Plus          \+   Comma              \,
%   Minus             \-  Point         \.   Solidus            \/
%   Colon             \:  Semicolon     \;   Less than          \<
%   Equals            \=  Greater than  \>   Question mark      \?
%   Commercial at     \@  Left bracket  \[   Backslash          \\
%   Right bracket     \]  Circumflex    \^   Underscore         \_
%   Grave accent      \`  Left brace    \{   Vertical bar       \|
%   Right brace       \}  Tilde         \~}
%   \iffalse
%<*config>
%% SIunits configuration file: SIunits.cfg
%%
%% You can uncomment one or more of the lines below to set default options.
%%
%%\ExecuteOptions{binary}  % Load binary.sty (binary units and prefixes)
%%
%% Spacing options (multiplying units):
%%
%%\ExecuteOptions{cdot}         % \cdot
%%\ExecuteOptions{thickspace}   % thick space \;
%%\ExecuteOptions{mediumspace}  % medium space \:
%%\ExecuteOptions{thinspace}    % thin space \,
%%
%% Spacing options (between quantity and unit):
%%
%%\ExecuteOptions{thickqspace}  % thick space \;
%%\ExecuteOptions{mediumqspace} % medium space \:
%%\ExecuteOptions{thinqspace}   % thin space \,
%%
%% Compatibility options:
%%
%%\ExecuteOptions{noams}   % When you don't have the AMS font, eurm10, use this option.
%%
%%\ExecuteOptions{amssymb} % prevent amssymb package from defining \square
%%\ExecuteOptions{squaren} % define \squaren for use with amssymb package
%%
%%\ExecuteOptions{pstricks}% prevent pstricks package from defining \gray
%%\ExecuteOptions{Gray}    % define \Gray for use with pstricks package
%%
%%\ExecuteOptions{italian} % define \unita to prevent interference with babel package.
%%
%% Misc options:
%%
%%\ExecuteOptions{textstyle}% Typeset units in font of context.
%</config>
%<*batchfile>
%
%
% (1) run SIunits.ins through TeX to get the package
% (you get: SIunits.sty, binary.sty, SIunits.cfg and SIunits.cfg)
%
% (2) run SIunits.drv through TeX (LaTeX) to get the documentation
% (you get: SIunits.dvi)
%
% [or use DOCSTRIP and extract SIunits.sty from SIunits.dtx
% using option `package']
%
%
\begin{filecontents}{SIunits.ins}
%% Note: This package is now maintained by Joseph Wright
%% (joseph.wright@morningstar2.co.uk)
%% Copyright 2007 Joseph Wright
%% ---------------------------------------------------------------------
%%
%% SIunits.ins will generate the fast loadable files SIunits.sty and
%% binary.sty, the documentation driver SIunits.drv and the configura-
%% tion file SIunits.cfg from the package file SIunits.dtx when run
%% through LaTeX or TeX.
%%
%% Copyright (C) 2001          Marcel Heldoorn
%%
%% This program may be distributed and/or modified under the
%% conditions of the LaTeX Project Public License, either version 1.3
%% of this license or (at your option) any later version.
%% The latest version of this license is in
%%   http://www.latex-project.org/lppl.txt
%% and version 1.3 or later is part of all distributions of LaTeX
%% version 2003/12/01 or later.
%%
%% NOTE:
%% The LPPL about .ins files:
%%    - Files with extension `.ins' (installation files): these files may
%%  not be modified at all because they contain the legal notices
%%  that are placed in the generated files.
%%
%% --------------- start of docstrip commands ------------------
%%
\def\batchfile{SIunits.ins}
\input docstrip.tex
\keepsilent
\preamble

Copyright (c) 1998-2002 Marcel Heldoorn <m.heldoorn@webschool.nl>.

This program may be distributed and/or modified under the conditions of the LaTeX Project
Public License, either version 1.3 of this license or (at your option) any later version.
The latest version of this license is in http://www.latex-project.org/lppl.txt and
version 1.3a or later is part of all distributions of LaTeX version 2003/12/01 or later.

For error reports in case of UNCHANGED versions see the readme.txt file.

Please do not request updates from me directly. Distribution is done through the
Comprehensive TeX Archive Network (CTAN).

\endpreamble
\postamble

 Source: $Id: SIunits.dtx,v 1.33 2002/08/01 11:20:00 root Exp root $
\endpostamble
\generate{\file{SIunits.sty}{\from{SIunits.dtx}{package}}}
\generate{\file{binary.sty}{\from{SIunits.dtx}{binary}}}
\nopreamble
\generate{\file{SIunits.cfg}{\from{SIunits.dtx}{config}}}
\generate{\file{SIunits.drv}{\from{SIunits.dtx}{driver}}}
 \Msg{************************************************************}
 \Msg{}
 \Msg{ To finish the installation you have to move the file}
 \Msg{ `SIunits.sty' (and optionally the file `SIunits.cfg')}
 \Msg{ into a directory searched by LaTeX.}
 \Msg{}
 \Msg{ To type-set the documentation, run the file `SIunits.drv'}
 \Msg{ through LaTeX.}
 \Msg{ Process SIunits.idx file by:}
 \Msg{ \space\space\space\space makeindex -s gind.ist SIunits}
 \Msg{ Process SIunits.glo file by:}
 \Msg{ \space\space\space\space makeindex -s gglo.ist -o SIunits.gls SIunits.glo}
 \Msg{}
 \Msg{************************************************************}
\endbatchfile
\end{filecontents}
%</batchfile>
%<*driver>
\immediate\write18{makeindex -s gind.ist -o \jobname.ind  \jobname.idx}
\immediate\write18{makeindex -s gglo.ist -o \jobname.gls  \jobname.glo}
\documentclass[a4paper]{ltxdoc}
\IfFileExists{hyperref.sty} {\usepackage{hyperref}}{\relax}
 \IfFileExists{SIunits.sty}
 {\usepackage[derivedinbase,derived,binary]{SIunits}}
 {\GenericWarning{SIunits.DTX}
 {Package file SIunits.STY not found!^^J
 Generate SIunits.STY by (La)TeXing SIunits.INS,
 process SIunits.DTX again.^^J}\stop}
%%
\GetFileInfo{SIunits.dtx}
\CodelineIndex%
\PageIndex% Comment out to get code line numbers in index.
\CodelineNumbered%
\EnableCrossrefs%
%%\DisableCrossrefs% UnComment if the index is ready
\OnlyDescription% Comment out to print Section "The Magic Code" as well.
%%
\begin{document}
\DocInput{SIunits.dtx}
 {\newpage\PrintIndex}
 {\newpage\PrintChanges}
\end{document}
%</driver>
%    \fi
%    \setlength\hfuzz{1000pt}
%    \setlength\overfullrule{1000pt}
%
%    \newcommand{\pkgname}[1]{\texttt{#1}}
%    \newcommand{\opt}[1]{\textsf{#1}}
%
%    \newenvironment{labeling}[1]
%     {\list{}{\settowidth{\labelwidth}{\textbf{#1}}
%     \leftmargin\labelwidth\advance\leftmargin\labelsep
%     \def\makelabel##1{\textbf{##1}\hfil}}}{\endlist}
%
%    \newcommand{\Fe}{\textit{For example:\ }}
%    \newcommand{\fe}{\textit{for example:\ }}
%    \newcommand{\dimn}[1]{\ensuremath{\mathsf{#1}}}
%
%    \def\digitsep{\,}
%    \def\onedigit#1{#1}
%    \def\twodigit#1#2{#1#2}
%    \def\threedigit#1#2#3{#1#2#3}
%    \def\fourdigit#1#2#3#4{#1\digitsep#2#3#4}
%    \def\fivedigit#1#2#3#4#5{#1#2\digitsep#3#4#5}
%    \def\sixdigit#1#2#3#4#5#6{#1#2#3\digitsep#4#5#6}
%    \def\sevendigit#1#2#3#4#5#6#7{#1\digitsep#2#3#4\digitsep#5#6#7}
%    \def\eightdigit#1#2#3#4#5#6#7#8{#1#2\digitsep#3#4#5\digitsep#6#7#8}
%    \def\ninedigit#1#2#3#4#5#6#7#8#9{#1#2#3\digitsep#4#5#6\digitsep#7#8#9}
%    \def\ifourdigit#1#2#3#4{#1#2#3\digitsep#4}
%    \def\ifivedigit#1#2#3#4#5{#1#2#3\digitsep#4#5}
%    \def\isixdigit#1#2#3#4#5#6{#1#2#3\digitsep#4#5#6}
%    \def\isevendigit#1#2#3#4#5#6#7{#1#2#3\digitsep#4#5#6\digitsep#7}
%    \def\ieightdigit#1#2#3#4#5#6#7#8{#1#2#3\digitsep#4#5#6\digitsep#7#8}
%    \def\ininedigit#1#2#3#4#5#6#7#8#9{#1#2#3\digitsep#4#5#6\digitsep#7#8#9}
%
%    \def\getlength#1{\ifx#1\end \let\next=\relax
%      \else\advance\count0 by 1 \let\next=\getlength\fi \next}
%
%    \def\length#1{\count0=0 \getlength#1\end}
%
%    \def\digitspart#1{\length{#1}\ifcase\count0\or
%                        \let\next=\onedigit \or
%                        \let\next=\twodigit \or
%                        \let\next=\threedigit \or
%                        \let\next=\ifourdigit \or
%                        \let\next=\ifivedigit \or
%                        \let\next=\isixdigit \or
%                        \let\next=\isevendigit \or
%                        \let\next=\ieightdigit \or
%                        \let\next=\ininedigit \or
%                        \else \let\next=\relax \fi \next#1}
%
%    \def\integerpart#1{{\length{#1}\ifcase\count0\or
%                      \let\next=\onedigit \or
%                      \let\next=\twodigit \or
%                      \let\next=\threedigit \or
%                      \let\next=\fourdigit \or
%                      \let\next=\fivedigit \or
%                      \let\next=\sixdigit \or
%                      \let\next=\sevendigit \or
%                      \let\next=\eightdigit \or
%                      \let\next=\ninedigit \or
%                      \else \let\next=\relax \fi \next#1}}
%
%    \def\split#1.#2\end{\ensuremath{\integerpart{#1}.\digitspart{#2}}}
%
%    \newcommand{\nmfmt}[1]{{\split #1\end}}
%    \newcommand{\intfmt}[1]{\ensuremath{\integerpart{#1}}}
%
%    \def\scientificnotation#1e#2\end{\ensuremath{{#1}\times 10^{#2}}}
%    \newcommand{\sci}[1]{{\scientificnotation #1\end}}
%    \newcommand{\joenit}[1]{\begingroup\mathcode`\.="8000 \ensuremath{#1}\endgroup}
%
%    \changes{v0.05 Final Release}{1999/03/30}{Release v0.05 under LaTeX Project Public License}
%    \changes{v0.05 Beta 2}{1999/03/30}{First package release under LaTeX Project Public License}
%    \changes{v0.06 Beta  2}{1998/04/07}{Problem with amssymb package solved thanks to Timothy C. Burt}
%    \changes{v0.03 Final Release}{1999/01/15}{Release v0.03}
%    \changes{v0.03 Beta  6}{1998/12/15}{Stable version, before releasing v0.03}
%    \changes{v0.03 Beta  4}{1998/12/10}{Stable version, before adding configuration file parameters}
%    \changes{v0.02 Beta  5}{1998/09/23}{Generated for Timothy C. Burt (\texttt{tcburt@comp.uark.edu})}
%    \changes{v0.02 Beta  4}{1998/09/15}{Typos corrected (thanks to Rafael Rodriguez Pappalardo)}
%    \changes{v0.02 Beta  3}{1998/09/11}{Code documentation checked and corrected}
%    \changes{v0.01}{1998/09/07}{Small documentation errors (thanks to Juergen von Haegen)}
%    \changes{v0.00 Beta 4}{1998/08/14}{Inconsistencies removed/changed}
%    \changes{v0.00 Beta 3}{1998/08/11}{Options implemented; better documentation}
%    \changes{v0.00 Beta 2}{1998/07/30}{Various small improvements}
%    \changes{v0.00 Beta 1}{1998/07/24}{Initial working version}
%    \changes{v1.00}{2000/03/21}{released as SIunits v1.00}
%    \changes{v1.13}{2000/08/29}{implementation of \textit{SI-brochure Supplement 2000}}
%    \changes{v1.15}{2001/02/01}{E-mail address change: \texttt{marcel.heldoorn@webschool.nl}}
%    \changes{v1.30}{2002/08/01}{E-mail address change: \texttt{SIunits@webschool.nl}}
%    \changes{v1.32}{2002/08/01}{\texttt{hyperref} package used in documentation driver}
%
%  \DoNotIndex{\,,\:,\;}
%  \DoNotIndex{\@defGrayfalse,\@defGraytrue,\@defsquarenfalse,\@defsquarentrue}
%  \DoNotIndex{\@elt,\@for}
%  \DoNotIndex{\@ifundefined,\@nameuse,\@ne,\@optionNoAMSfalse,\@optionNoAMStrue  }
%  \DoNotIndex{\@optionNoAMSfalse\@optionNoAMStrue\@optionbinaryfalse\@optionbinarytrue}
%  \DoNotIndex{\@qsk,\@qskwidth,\@redefGrayfalse,\@redefGraytrue}
%  \DoNotIndex{\@redefsquarefalse,\@redefsquaretrue,\@scientificfalse,\@scientifictrue}
%  \DoNotIndex{\@stpelt,\@text,\@text@,\@textstylefalse,\@textstyletrue}
%  \DoNotIndex{\AA,\addtocounter,\def,\edef,\gdef}
%  \DoNotIndex{\amperemetresecond,\amperepermetre,\amperepermetrenp,\amperepersquaremetre,\amperepersquaremetrenp}
%  \DoNotIndex{\arad}
%  \DoNotIndex{\AtBeginDocument,\AtEndOfPackage}
%  \DoNotIndex{\begingroup}
%  \DoNotIndex{\candelapersquaremetre,\candelapersquaremetrenp,\cdot}
%  \DoNotIndex{\check@mathfonts,\circ}
%  \DoNotIndex{\coulombpercubicmetre,\coulombpercubicmetrenp,\coulombperkilogram,\coulombperkilogramnp}
%  \DoNotIndex{\coulombpermol,\coulombpermolnp,\coulombpersquaremetre,\coulombpersquaremetrenp,\csname}
%  \DoNotIndex{\cubicmetreperkilogram,\cubicmetrepersecond,\CurrentOption}
%  \DoNotIndex{\decimals,\DeclareFontEncoding,\DeclareFontFamily}
%  \DoNotIndex{\DeclareFontShape,\DeclareFontSubstitution,\DeclareMathSymbol}
%  \DoNotIndex{\DeclareOption,\DeclareRobustCommand,\DeclareSymbolFont,\DeclareTextSymbol}
%  \DoNotIndex{\DeclareTextSymbolDefault}
%  \DoNotIndex{\def,\displaystyle,\do}
%  \DoNotIndex{\else,\csname,\endcsname,\begingroup,\endgroup,\endinput,\ensuremath,\advancemath,\bgroup}
%  \DoNotIndex{\egroup,\everymath}
%  \DoNotIndex{\ExecuteOptions,\expandafter,\f@size}
%  \DoNotIndex{\faradpermetre,\faradpermetrenp,\fi,\filedate,\fileversion,\firstchoice@false,\firstchoice@true,\font}
%  \DoNotIndex{\graypersecond,\graypersecondnp,\hbox,\henrypermetre,\henrypermetrenp,\hspace}
%  \DoNotIndex{\if@defGray,\if@defsquaren,\if@optionbinary,\if@optionNoAMS,\if@redefGray,\if@redefsquare}
%  \DoNotIndex{\if@scientific,\if@textstyle,\iffirstchoice@,\ifmmode,\ifundefined,\InputIfFileExists}
%  \DoNotIndex{\joulepercubicmetre,\joulepercubicmetrenp,\jouleperkelvin,\jouleperkelvinnp}
%  \DoNotIndex{\jouleperkilogram,\jouleperkilogramkelvin,\jouleperkilogramkelvinnp,\jouleperkilogramnp}
%  \DoNotIndex{\joulepermole,\joulepermolekelvin,\joulepermolekelvinnp,\joulepermolenp,\joulepersquaremetre}
%  \DoNotIndex{\joulepersquaremetrenp,\joulepertesla,\jouleperteslanp}
%  \DoNotIndex{\kilogrammetrepersecond,\kilogrammetrepersecondnp,\kilogrammetrepersquaresecond}
%  \DoNotIndex{\kilogrammetrepersquaresecondnp,\kilogrampercubicmetre,\kilogrampercubicmetrecoulomb}
%  \DoNotIndex{\kilogrampercubicmetrecoulombnp,\kilogrampercubicmetrenp,\kilogramperkilomole}
%  \DoNotIndex{\kilogramperkilomolenp,\kilogrampermetre,\kilogrampermetrenp,\kilogrampersecond}
%  \DoNotIndex{\kilogrampersecondcubicmetre,\kilogrampersecondcubicmetrenp,\kilogrampersecondnp}
%  \DoNotIndex{\kilogrampersquaremetre,\kilogrampersquaremetrenp,\kilogrampersquaremetresecond}
%  \DoNotIndex{\kilogrampersquaremetresecondnp,\kilogramsquaremetre,\kilogramsquaremetrenp}
%  \DoNotIndex{\kilogramsquaremetrepersecond,\kilogramsquaremetrepersecondnp,\let}
%  \DoNotIndex{\mathcode,\mathchoice,\mathord,\mathrm,\mbox}
%  \DoNotIndex{\MessageBreak,\metrepersecond,\metrepersecondnp,\metrepersquaresecond,\metrepersquaresecondnp}
%  \DoNotIndex{\molepercubicmetre,\molepercubicmetrenp,\NeedsTeXFormat,\newcommand}
%  \DoNotIndex{\newif,\newlength}
%  \DoNotIndex{\newtonmetrenp,\newtonpercubicmetre,\newtonpercubicmetrenp}
%  \DoNotIndex{\newtonperkilogram,\newtonperkilogramnp,\newtonpermetre}
%  \DoNotIndex{\newtonpermetrenp,\newtonpersquaremetre,\newtonpersquaremetrenp}
%  \DoNotIndex{\nfss@text,\no@qsk,\ohmmetre,\Omega}
%  \DoNotIndex{\PackageError,\packagemessage,\packagename}
%  \DoNotIndex{\PackageWarning,\PackageWarningNoLine,\persquaremetresecond,\period@active}
%  \DoNotIndex{\persquaremetresecondnp,\Prefixes,\ProcessOptions,\protect,\providecommand}
%  \DoNotIndex{\ProvidesPackage,\quantityskip,\radianpersecond,\radianpersecondnp}
%  \DoNotIndex{\radianpersquaresecond,\radianpersquaresecondnp}
%  \DoNotIndex{\relax,\renewcommand,\RequirePackage,\RequirePackageWithOptions}
%  \DoNotIndex{\rpcubicmetreperkilogram,\rpcubicmetrepersecond,\rperminute,\rpfourth}
%  \DoNotIndex{\rpsquare,\rpsquared,\rpsquaremetreperkilogram}
%  \DoNotIndex{\selectfont,\settowidth,\sf@size,\SI@fstyle}
%  \DoNotIndex{\SIunits@@,\SIunits@execopt,\SIunits@opt@cdot,\SIunits@opt@derived}
%  \DoNotIndex{\SIunits@opt@derivedinbase,\SIunits@opt@mediumqspace,\SIunits@opt@mediumspace}
%  \DoNotIndex{\SIunits@opt@thickqspace,\SIunits@opt@thickspace,\SIunits@opt@thinqspace}
%  \DoNotIndex{\SIunits@opt@thinspace,\skewchar,\space}
%  \DoNotIndex{\squaremetrepercubicmetre,\squaremetrepercubicmetrenp,\squaremetrepercubicsecond}
%  \DoNotIndex{\squaremetrepercubicsecondnp,\squaremetreperkilogram,\squaremetrepernewtonsecond}
%  \DoNotIndex{\squaremetrepernewtonsecondnp,\squaremetrepersecond,\squaremetrepersecondnp}
%  \DoNotIndex{\squaremetrepersquaresecond,\squaremetrepersquaresecondnp,,\ssf@size}
%  \DoNotIndex{\stepcounter,\symbols,\textdef@,\textstyle,\tf@size,\typeout}
%  \DoNotIndex{\voltpermetre,\voltpermetrenp,\wattpercubicmetre,\wattpercubicmetrenp}
%  \DoNotIndex{\wattperkilogram,\wattperkilogramnp,\wattpermetrekelvin,\wattpermetrekelvinnp}
%  \DoNotIndex{\wattpersquaremetre,\wattpersquaremetrenp,\wattpersquaremetresteradian}
%  \DoNotIndex{\wattpersquaremetresteradiannp}
%
%\GetFileInfo{SIunits.sty}
%    \changes{\fileversion}{\filedate}{Current version submitted to CTAN}
%\title{The \pkgname{SIunits}\ package\thanks{This file has version number \fileversion{}, last revised
%\filedate{} \packagemessage}\\{Consistent application of SI units}}
%\author{Marcel Heldoorn\thanks{Now maintained by Joseph Wright, e-mail: %
%joseph.wright@morningstar2.co.uk}}
%\date{File date \filedate}
%\maketitle
%
%\begin{abstract}
%\noindent This article describes the \pkgname{SIunits}\ package that provides support for
%the \textsf{Syst\`{e}me International d'Unit\'{e}s} (SI).
%
%The Syst\`{e}me International d'Unit\'{e}s (SI), the modern form of the metric system, is the
%most widely used system of units and measures around the world. But despite this there is
%widespread misuse of the system with incorrect names and symbols used as a matter a
%course - even by well educated and trained people who should know better. For example how
%often do we see: \milli\hertz, \mega hz or \milli hz written when referring to computer
%clock rates? The correct form is actually \mega\hertz. Note that the capitalisation does
%matter.
%
%Hence, a clear system for the use of units is needed, satisfying the next principles:
%\begin{enumerate}
%\item the system should consist of measuring units based on unvariable quantities in nature;
%\item all units other than the base units should be derived from these base units; and
%\item multiples and submultiples of the units should be decimal.
%\end{enumerate}
%The name Syst\`{e}me International d'Unit\'{e}s (International System of Units) with the
%international abbreviation SI was adopted by the Conf\'{e}rence G\'{e}n\'{e}rale des Poids et Mesures
%(CGPM) in 1960. It is a coherent system based on seven base units (CGPM 1960 and 1971).
%
%
%The \pkgname{SIunits}\ package can be used to standardise the use of units in your
%writings. Most macros are easily adaptable to personal preferences. However, you are
%welcome (and strongly invited\footnote{There is an enormous \LaTeX\ Knowledge Base out
%there.}) to suggest any improvements.
%\end{abstract}
%\newpage
%\section*{What's new?}
%\subsubsection*{New in version~1.36}
%\begin{enumerate}
%  \item Real minus sign in text mode 
%\end{enumerate}
%\subsubsection*{New in version~1.35}
%\begin{enumerate}
%  \item Improved \cs{elecronvolt} appearance\footnote{All changes for this version suggested by Philip Ratcliffe}
%  \item Added \cs{dalton}, \cs{atomicmassunit} units (both formally non-SI)
%  \item Minor improvements to the documentation
%\end{enumerate}
%\subsubsection*{New in version~1.34}
%\begin{enumerate}
%  \item Maintainer is now Joseph Wright (\href{mailto:joseph.wright@morningstar2.co.uk}%
%    {\texttt{joseph.wright@morningstar2.co.uk}})
%  \item Bug fix for negative signs in textstyle mode\footnote{Thanks to Stefan Pinnow}
%  \item License changed to LPPL 1.3 or later
%\end{enumerate}
%\newpage\tableofcontents\newpage
%\section{Introduction}
%
%\subsection{Historical notes}
%In 1948 the 9th General Conference on Weights and Measures (CGPM\footnote{See section
%\ref{acronyms} for acronyms}), by its Resolution 6, instructed the International
%Committee for Weights and Measures (CIPM\footnotemark[\value{footnote}]):
%\begin{verse}
%`to study the establishment of a complete set of rules for units of measurement';
%
%`to find out for this purpose, by official inquiry, the opinion prevailing in scientific,
%technical, and educational circles in all countries'; and
%
%`to make recommendations on the establishment of a \textit{practical system of units of
%measurement} suitable for adoption by all signatories to the Meter Convention.'
%\end{verse}
%
%The same General Conference also laid down, by its Resolution 7, general principles for
%unit symbols and also gave a list of units with special names.
%
%
%The 10th CGPM (1954), by its Resolution 6, and the 14th CGPM (1971), by its Resolution 3,
%adopted as base units of this `practical system of units,' the units of the following
%seven quantities: length, mass, time, electric current, thermodynamic temperature, amount
%of substance, and luminous intensity.
%
%
%The 11th CGPM (1960), by its Resolution 12, adopted the name \textit{Syst\`{e}me
%International d'Unit\'{e}s (International System of Units)}, with the international
%abbreviation \textit{SI}, for this practical system of units of measurement, and laid
%down rules for the prefixes, the derived and supplementary units, and other matters, thus
%establishing a comprehensive specification for units of measurement.
%
%\subsection{The classes of SI units}
%The General Conference decided to base the International
%System on a choice of seven well-defined units which by convention are regarded as
%dimensionally independent: the metre, the kilogram, the second, the ampere, the kelvin,
%the mole, and the candela. These units are called \textit{base units}.
%
%
%The second class of SI units contain \textit{derived units}, i.\,e., units that can be
%formed by combining base units according to the algebraic relations linking the
%corresponding quantities. The names and symbols of some units thus formed in terms of
%base units can be replaced by special names and symbols which can themselves be used to
%form expressions and symbols of other derived units (see section \ref{derived}, p.
%\pageref{derived}).
%
%
%The 11th CGPM (1960) admitted a third class of SI units, called \textit{supplementary
%units} and containing the SI units of plane and solid angle.
%
%
%The 20th CGPM (1995) decided to eliminate the class of supplementary units as a separate
%class in the SI. Thus the SI now consists of only two classes of units: base units and
%derived units, with the radian and the steradian, which are the two supplementary units,
%subsumed into the class of derived SI units.
%
%
%\subsection{The SI prefixes}
%The General Conference has adopted a series of prefixes to be used in forming the decimal
%multiples and submultiples of SI units. Following CIPM Recommendation~1 (1969), the set
%of prefixes is designated by the name \textit{SI prefixes}.
%
%The multiples and submultiples of SI units, which are formed by using the SI prefixes,
%should be designated by their complete name, \textit{multiples and submultiples of SI
%units}, in order to make a distinction between them and the coherent set of SI units
%proper.
%
%\subsection{Acronyms}
%\label{acronyms} The SI was established in 1960 by the CGPM. The CGPM is an
%intergovernmental treaty organisation created by a diplomatic treaty called the Meter
%Convention (\textit{Convention du M\`{e}tre}, often called the Treaty of the Meter in the
%United States). The Meter Convention was signed in Paris in 1875 by representatives of
%seventeen nations, including the United States. There are now forty-eight Member States,
%including all the major industrialised countries. The Convention, modified slightly in
%1921, remains the basis of all international agreement on units of measurement.
%
%The Meter Convention also created the International Bureau of Weights and Measures (BIPM,
%Bureau International des Poids et Mesures) and the International Committee for Weights
%and Measures (CIPM, Comit\'{e} International des Poids et Mesures). The BIPM, which is
%located in S\`{e}vres, a suburb of Paris, France, and which has the task of ensuring
%worldwide unification of physical measurements, operates under the exclusive supervision
%of the CIPM, which itself comes under the authority of the CGPM.
%
%
%
%\begin{description}
%\item[CGPM] General Conference on Weights and Measures (\textit{Conf\'{e}rence G\'{e}n\'{e}rale des Poids
%et Mesures}). The CGPM is the primary intergovernmental treaty organisation responsible
%for the SI, representing nearly 50 countries. It has the responsibility of ensuring that
%the SI is widely disseminated and modifying it as necessary so that it reflects the
%latest advances in science and technology.
%
%\item[CIPM] International Committee for Weights and Measures (\textit{Comit\'{e} International des
%Poids et Mesures}). The CIPM comes under the authority of the CGPM. It suggests
%modifications to the SI to the CGPM for formal adoption. The CIPM may also on its own
%authority pass clarifying resolutions and recommendations regarding the SI.
%
%\item[BIPM] International Bureau of Weights and Measures (\textit{Bureau International des
%Poids et Mesures}). The BIPM, located outside Paris, has the task of ensuring worldwide
%unification of physical measurements. It is the ``international" metrology institute, and
%operates under the exclusive supervision of the CIPM.
%\end{description}
%
%\subsection{Some useful definitions}
%\begin{description}
%\item [quantity in the general sense] A quantity in the general sense is a property ascribed to phenomena, bodies, or
%substances that can be quantified for, or assigned to, a particular phenomenon, body, or
%substance. Examples are mass and electric charge.
%\item [quantity in the particular sense] A quantity in the particular sense is a quantifiable or assignable property ascribed
%to a particular phenomenon, body, or substance. Examples are the mass of the moon and the
%electric charge of the proton.
%\item [physical quantity] A physical quantity is a quantity that can be used in the mathematical equations of
%science and technology.
%\item [unit] A unit is a particular physical quantity, defined and adopted by convention, with
%which other particular quantities of the same kind are compared to express their value.
%\end{description}
%
%
%\noindent The \textbf{value of a physical quantity} is the quantitative expression of a
%particular physical quantity as the product of a number and a unit, the number being its
%numerical value. Thus, the numerical value of a particular physical quantity depends on
%the unit in which it is expressed.
%
%\label{celsiusexample}
%More formally, the value of quantity \(A\) can be written as \(A=\lbrace A\rbrace [A]\),
%where \(\{A\}\) is the numerical value of \(A\) when \(A\) is expressed in the unit
%\([A]\). The numerical value can therefore be written as \(\lbrace A\rbrace=A/[A]\),
%which is a convenient form for use in figures and tables. Thus to eliminate the
%possibility of misunderstanding, an axis of a graph or the heading of a column of a table
%can be labelled `\(t\)\per\celsius' instead of `\(t\)(\celsius)' or `Temperature
%(\celsius)'. Similarly, another example: `\(E/\)(\voltpermetre)' instead of
%`\(E\)(\voltpermetre)' or `Electric field strength (\voltpermetre)'.
%
%\addunit{\foot}{ft}
%\Fe the value of the height \(h_{W}\) of the Washington Monument is \(h_{W} = \unit{169}{\metre}=
%\unit{555}{\foot}\)\footnote{foot (\foot) is not part of the SI units}. Here \(h_{W}\) is the physical quantity, its value expressed in the
%unit metre, unit symbol \metre, is \unit{169}{\metre}, and its numerical value when
%expressed in metres is \(169\).
%
%\section{SI units}
%\subsection{SI base units}
%\subsubsection{Definitions}
%The SI is founded on seven SI base units for seven base quantities assumed to be mutually
%independent. The primary definitions of the SI base units are in French. Their current
%definitions, along with an English translation, are given below:
%\paragraph*{metre; \textit{m\`{e}tre}}
%\begin{description}
%\item[]
%\frenchspacing\textit{Le m\`{e}tre est la longueur du trajet parcouru dans le vide par la lumi\`{e}re pendant
%une dur\'{e}e de \(1/\intfmt{299792458}\) de seconde.}\\ {(17th CGPM (1983), Resolution 1).}\nonfrenchspacing
%\item[]
%{The metre is the length of the path travelled by light in vacuum during a time interval
%of \(1/\intfmt{299792458}\) of a second.}
%\end{description}
%\paragraph*{kilogram; \textit{kilogramme}}
%\begin{description}
%\item[]
%\frenchspacing\textit{Le kilogramme est l'unit\'{e} de masse; il est \'{e}gal \`{a} la masse du prototype
%international du kilogramme.}\\ (1st CGPM (1889) and 3rd CGPM (1901)).\nonfrenchspacing
%\item[]
%{The kilogram is the unit of mass; it is equal to the mass of the international prototype
%of the kilogram.} \\{\textit{Note}: This international prototype is made of
%platinum-iridium and is kept at the International Bureau of Weights and Measures, S\`{e}vres,
%France.}
%\end{description}
%\paragraph*{second; \textit{seconde}}
%\begin{description}
%\item[]
%\frenchspacing\textit{La seconde est la dur\'{e}e de \(9\digitsep192\digitsep631\digitsep770\) p\'{e}riodes de
%la radiation correspondant \`{a} la transition entre les deux niveaux hyperfins de l'\'{e}tat
%fondamental de l'atome de cesium 133.} \\(13th CGPM (1967), Resolution 1).\nonfrenchspacing
%\item[]
%{The second is the duration of \(9\digitsep192\digitsep631\digitsep770\) periods of the
%radiation corresponding to the transition between the two hyperfine levels of the ground
%state of the cesium-\(133\) atom.}
%\end{description}
%\textit{Note}: This definition refers to a caesium atom at rest at a temperature of \unit{0}{\kelvin}.
%\paragraph*{ampere; \textit{amp\`{e}re}}
%\begin{description}
%\item[]
%\frenchspacing\textit{L'amp\`{e}re est l'intensit\'{e} d'un courant constant qui, maintenu dans deux
%conducteurs parall\`{e}les, rectilignes, de longueur infinie, de section circulaire
%n\'{e}gligeable, et plac\'{e}s \`{a} une distance de 1 m\`{e}tre l'un de l'autre dans le vide, produirait
%entre ces conducteurs une force \'{e}gale \`{a} \(\sci{2e-7}\) newton par m\`{e}tre de longueur.}\\
%(9th CGPM (1948), Resolutions 2 and 7).\nonfrenchspacing
%\item[]
%{The ampere is that constant current which, if maintained in two straight parallel
%conductors of infinite length, of negligible circular cross-section, and placed 1 metre
%apart in vacuum, would produce between these conductors a force equal to \(\sci{2e-7}\)
%newton per metre of length.}
%\end{description}
%\paragraph*{kelvin; \textit{kelvin}}
%\begin{description}
%\item[]
%\frenchspacing\textit{Le kelvin, unit\'{e} de temp\'{e}rature thermodynamique, est la fraction \(1/273.16\) de
%la temp\'{e}rature thermodynamique du point triple de l'eau.}\\(13th CGPM (1967), Resolution
%4).\nonfrenchspacing
%\item[]
%{The kelvin, unit of thermodynamic temperature, is the fraction \(1/273.16\) of the
%thermodynamic temperature of the triple point of water.}
%\end{description}
%\textit{Note}: The 13th CGPM (1967, Resolution 3) also decided that the unit kelvin
%and its symbol \kelvin\ should be used to express
%both thermodynamic temperature and an interval or a difference of temperature, instead of
%`degree Kelvin' with symbol \degree\kelvin.
%
%
%In addition to the thermodynamic temperature (symbol \(T\)) there is also the Celsius
%(symbol \(t\)) defined by the equation \(t= T-T_{0}\) where \(T_{0}=\)
%\unit{273.15}{\kelvin}. Celsius temperature is expressed in degree Celsius; \textit{degr\'{e}
%Celsius} (symbol \celsius). The unit `degree Celsius' is equal to the unit
%`kelvin'; in this case, `degree Celsius' is a special name used in place of `kelvin'. A
%temperature interval or difference of Celsius temperature can, however, be expressed in
%kelvins as well as in degrees Celsius.
%
%\paragraph*{mole; \textit{mole}}
%\begin{description}
%\item[]
%\frenchspacing\textit{1�. La mole est la quantit\'{e} de mati\`{e}re d'un syst\`{e}me contenant autant d'entit\'{e}s
%\'{e}l\'{e}mentaires qu'il y a d'atomes dans \(0,012\) kilogramme de carbone \(12\).}
%\item[]
%\textit{2�. Lorsqu'on emploie la mole, les entit\'{e}s \'{e}l\'{e}mentaires doivent \^{e}tre sp\'{e}cifi\'{e}es
%et peuvent \^{e}tre des atomes, des mol\'{e}cules, des ions, des \'{e}lectrons, d'autres particules
%ou des groupements sp\'{e}cifi\'{e}s de telles particules.}\\ (14th CGPM (1971), Resolution 3).\nonfrenchspacing
%\item[]
%{1. The mole is the amount of substance of a system which contains as many elementary
%entities as there are atoms in \(0.012\) kilogram of carbon \(12\).}
%\item[]
%{2. When the mole is used, the elementary entities must be specified and may be atoms,
%molecules, ions, electrons, other particles or specified groups of such
%particle.}
%\end{description}
%\textit{Note}: In this definition, it is understood that the carbon 12 atoms
%are unbound, at rest and in their ground state.
%\paragraph*{candela; \textit{candela}}
%\begin{description}
%\item[]
%\frenchspacing\textit{La candela est l'intensit\'{e} lumineuse, dans une direction donn\'{e}e, d'une source qui
%\'{e}met une radiation monochromatique de fr\'{e}quence \(\sci{540e12}\) hertz et dont
%l'intensit\'{e} \'{e}nerg\'{e}tique dans cette direction est \(1/683\) watt par st\'{e}radian.}\\ (16th
%CGPM (1979), Resolution 3).\nonfrenchspacing
%\item[]
%{The candela is the luminous intensity, in a given direction, of a source that emits
%monochromatic radiation of a frequency \(\sci{540e12}\) hertz and has a radiant intensity
%in that direction of \(1/683\) watt per steradian.}
%\end{description}
%\newpage
%\subsubsection{Symbols}
%The base units of the International System are collected in table~\ref{table:base} with
%their names and their symbols (10th CGPM (1954), Resolution 6; 11th CGPM (1960),
%Resolution 12; 13th CGPM (1967), Resolution 3; 14th CGPM (1971), Resolution 3).\\
% \begin{table}[btp]
%   \caption{--- SI base units ---}\label{table:base}
%   \centering
% \begin{tabular}{llc}\hline
%   \textbf{Quantity}           & \textbf{Name} & \textbf{Symbol} \\ \hline
%   length                      & metre         & \metre      \\
%   mass                        & kilogram      & \kilogram      \\
%   time                        & second        & \second     \\
%   electric current            & ampere        & \ampere     \\
%   thermodynamic temperature   & kelvin        & \kelvin         \\
%   amount of substance         & mole          & \mole      \\
%   luminous intensity          & candela       & \candela    \\ \hline
% \end{tabular}
% \end{table}
%
%\subsection{SI derived units}\label{derived}
%Derived units are units which may be expressed in terms of base units by means of the
%mathematical symbols of multiplication and division. Certain derived units have been
%given special names and symbols, and these special names and symbols may themselves be
%used in combination with those for base and other derived units to express the units of
%other quantities.
%\subsubsection{Units expressed in terms of base units}
%Table~\ref{table:derived1} lists some examples of derived units expressed directly in
%terms of base units. The derived units are obtained by multiplication and division of
%base units.
%   \begin{table}[btp]
%     \caption{--- Examples of SI derived units ---}\label{table:derived1}
%   \begin{minipage}[]{\textwidth}
%     \centering
%   \begin{tabular}{llc}\hline
%     \textbf{Derived quantity}       & \textbf{Name}         &   \textbf{Symbol}\\ \hline
%   area       &  square metre    &   \squaremetre   \\
%   volume     &  cubic metre     &   \cubicmetre     \\
%   speed, velocity    &  metre per second        &   \metrepersecond\\
%   acceleration       &  metre per second squared    &   \metrepersquaresecond  \\
%   wave number &  reciprocal metre   &   \reciprocal\metre   \\
%   mass density       &  kilogram per cubic metre    &   \kilogrampercubicmetre     \\
%   specific volume    &  cubic metre per kilogram    &   \cubicmetreperkilogram     \\
%   current density    &  ampere per square metre     &   \amperepersquaremetre  \\
%   magnetic field strength    &  ampere per metre    &   \amperepermetre       \\
%   amount-of-substance concentration      &  mole per cubic metre    &   \mole\per\cubicmetre    \\
%   luminance          &  candela per square metre    &   \candelapersquaremetre     \\
%   mass fraction      &  kilogram per kilogram   &   \kilogram\per\kilogram\footnote{the symbol 1
%   for quantities of dimension 1 such as mass fraction is generally omitted.} \\
%   \end{tabular}
%   \end{minipage}
%   \end{table}
%
%\subsubsection{SI derived units with special names and symbols}For ease of understanding and
%convenience, 21 SI derived units have been given special names and symbols, as shown in
%table~\ref{table:derived2}. They may themselves be used to express other derived units.
%   \begin{table}[btp]
%     \caption{--- SI derived units with special names and symbols ---}\label{table:derived2}
%   \begin{minipage}[]{\textwidth}\renewcommand{\thefootnote}{\textit{\alph{footnote}}}
%     \centering
%   \begin{tabular}{llll}\hline
%   \textbf{Name} &  \textbf{Expression in} & \textbf{Symbol} & \textbf{Expression in} \\
%     &  \textbf{SI base units} &      & \textbf{SI derived units}\\ \hline
%   radian\footnotemark[1]  &       \radianbase\(\,=1\)\footnotemark[2]  &  \radian  &  \derradian  \\
%   steradian\footnotemark[1]       &  \steradianbase \(\,=1\)\footnotemark[2] &  \steradian\footnotemark[3]  &  \dersteradian  \\
%   hertz  &  \hertzbase  &  \hertz  &       \derhertz  \\
%   newton  &       \newtonbase  &  \newton  &  \dernewton  \\
%   pascal  &       \pascalbase  &  \pascal  &  \derpascal  \\
%   joule  &  \joulebase  &  \joule  &       \derjoule  \\
%   watt  &  \wattbase       &  \watt  &  \derwatt  \\
%   coulomb  &       \coulombbase  &  \coulomb  &  \dercoulomb  \\
%   volt  &  \voltbase       &  \volt  &  \dervolt  \\
%   farad  &  \faradbase  &  \farad  &       \derfarad \\
%   ohm  &  \ohmbase  &  \ohm       &  \derohm  \\
%   siemens  &       \siemensbase  &  \siemens  &  \dersiemens  \\
%   weber  &  \weberbase  &  \weber  &       \derweber  \\
%   tesla  &  \teslabase  &  \tesla  &       \dertesla  \\
%   henry  &  \henrybase  &  \henry  &       \derhenry  \\
%   celsius  &       \celsiusbase  &  \celsius  &  \dercelsius  \\
%   lumen  &  \lumenbase\footnotemark[3]  &  \lumen  &       \derlumen  \\
%   lux  &  \luxbase  &  \lux       &  \derlux  \\
%   becquerel       &  \becquerelbase  &  \becquerel  &  \derbecquerel  \\
%   gray  &  \graybase       &  \gray  &  \dergray  \\
%   sievert\footnotemark[4] &       \sievertbase  &  \sievert  &  \dersievert  \\
%   katal\footnotemark[5] & \katalbase & \katal & \derkatal\\
%   \end{tabular}
%    \footnotetext[1]
%   {The radian and steradian may be used advantageously in expressions for derived units to
%   distinguish between quantities of a different nature but of the same dimension; some
%   examples are given in table~\ref{table:derived3}.}
%    \footnotetext[2]
%   {In practice, the symbols \radian\ and \steradian\ are used where appropriate, but the
%   derived unit `\(1\)' is generally omitted.}
%    \footnotetext[3]
%   {In photometry, the unit name steradian and the unit symbol \steradian\ are usually
%   retained in expressions for derived units.}
%    \footnotetext[4]
%   {Other quantities expressed in sieverts are ambient dose equivalent, directional dose
%   equivalent, personal dose equivalent, and organ equivalent dose.}
%    \footnotetext[5]
%   {The 21st Conf\'{e}rence G\'{e}n\'{e}rale des Poids et Mesures decides to adopt the special name katal,
%   symbol \katal, for the SI unit mole per second to express catalytic activity, especially
%   in the fields of medicine and biochemistry, ... (21th CGPM (1999), Resolution 12).}
%   \end{minipage}
%   \end{table}
%\subsubsection{Use of SI derived units with special names and symbols}
%Examples of SI derived units that can be expressed with the aid of SI derived units
%having special names and symbols (including the radian and steradian) are given in
%table~\ref{table:derived2}. The advantages of using the special names and symbols of SI
%derived units are apparent in table~\ref{table:derived3}. Consider, for example, the
%quantity molar entropy: the unit \joenit{\joule\per\mole.\kelvin}\ is obviously more easily
%understood than its SI base-unit equivalent,
%\joenit{\squaremetre.\kilogram.\second\rpsquared.\reciprocal\kelvin\reciprocal\mole}.
%Nevertheless, it should always be recognised that the special names and symbols exist for
%convenience. Tables~\ref{table:derived2}~\&~\ref{table:derived3} also show that the
%values of several different quantities are expressed in the same SI unit. For example,
%the joule per kelvin (\joule\per\kelvin) is the SI unit for heat capacity as well as for
%entropy. Thus the name of the unit is not sufficient to define the quantity measured. A
%derived unit can often be expressed in several different ways through the use of base
%units and derived units with special names. In practice, with certain quantities,
%preference is given to using certain units with special names, or combinations of units,
%to facilitate the distinction between quantities whose values have identical expressions
%in terms of SI base units. For example, the SI unit of frequency is specified as the
%hertz (Hz) rather than the reciprocal second (\reciprocal\second), and the SI unit of
%moment of force is specified as the newton metre (\newtonmetre) rather than the joule
%(\joule).
%
%   \begin{table}[btp]
%     \caption{--- Examples of SI derived units expressed with the aid of SI
%     derived units having special names and symbols ---}\label{table:derived3}
%   \begin{minipage}[]{\textwidth}\renewcommand{\thefootnote}{\textit{\alph{footnote}}}
%     \centering
%   \begin{tabular}{lll}\hline
%   \textbf{Derived quantity} & \textbf{Name} & \textbf{Symbol}\\ \hline
%     angular velocity        & radian per second &    \radianpersecond \\
%     angular acceleration & radian per second squared & \radianpersquaresecond  \\
%     dynamic viscosity & pascal second & \pascalsecond \\
%     moment of force          & newton metre & \newtonmetre \\
%     surface tension &  newton per metre & \newtonpermetre \\
%     heat flux density,\\ irradiance & watt per square metre & \wattpersquaremetre \\
%     radiant intensity & watt per steradian & \watt\per\steradian \\
%     radiance & watt per square metre steradian & \wattpersquaremetresteradian  \\
%     heat capacity,\\ entropy & joule per kelvin & \jouleperkelvin \\
%     specific heat capacity,\\  specific entropy & joule per kilogram kelvin & \jouleperkilogramkelvin \\
%     specific energy          & joule per kilogram & \jouleperkilogram  \\
%     thermal conductivity & watt per metre kelvin & \wattpermetrekelvin \\
%     energy density & joule per cubic metre & \joulepercubicmetre \\
%     electric field strength & volt per metre & \voltpermetre \\
%     electric charge density & coulomb per cubic metre & \coulombpercubicmetre \\
%     electric flux density & coulomb per square metre & \coulombpersquaremetre \\
%     permittivity & farad per metre & \faradpermetre \\
%     permeability & henry per metre & \henrypermetre \\
%     molar energy & joule per mole  & \joulepermole \\
%     molar entropy, molar\\ heat capacity & joule per mole kelvin &  \joulepermolekelvin \\
%     exposure (x and \(\gamma\) rays)& coulomb per kilogram & \coulombperkilogram \\
%     absorbed dose rate &  gray per second &          \graypersecond \\
%     catalytic (activity)\\ concentration & katal per cubic metre  & \katalpercubicmetre \\
%   \end{tabular}
%   \end{minipage}
%   \end{table}
% \subsection{Dimension of a quantity}
% Any SI derived quantity \(Q\) can be expressed in terms of the SI base quantities length
% (\(l\)), mass (\(m\)), time (\(t\)), electric current (\(I\)), thermodynamic temperature
% (\(T\)), amount of substance (\(n\)), and luminous intensity (\(I_{\mathrm{v}}\)) by an
% equation of the form
% \[
% Q = l^{\alpha}m^{\beta}t^{\gamma}I^{\delta}T^{\varepsilon}n^{\zeta}I_{\mathrm{v}}^{\eta}
% \sum\limits_{k = 1}^K {a_k} ,
% \]
% where the exponents \(\alpha\), \(\beta\), \(\gamma\), \(\ldots\) are numbers and the
% factors \(a_{k}\) are also numbers. The dimension of \(Q\) is defined to be
% \[
% \dim Q = \dimn{L}^{\alpha} \dimn{M}^{\beta} \dimn{T}^{\gamma} \dimn{I}^{\delta}
% \dimn{\Theta}^{\varepsilon} \dimn{N}^{\zeta} \dimn{J}^{\eta} ,
% \]
% where \dimn{L}, \dimn{M}, \dimn{T}, \dimn{I}, \dimn{\Theta}, \dimn{N} and \dimn{J}\ are
% the dimensions of the SI base quantities length, mass, time, electric current,
% thermodynamic temperature, amount of substance, and luminous intensity, respectively. The
% exponents \(\alpha\), \(\beta\), \(\gamma\), \(\ldots\) are called ``dimensional
% exponents''. The SI derived unit of \(Q \) is \(\joenit{\metre^{\alpha}.\kilogram^{\beta}
%.\second^{\gamma}.\ampere^{\delta}.\kelvin^{\varepsilon}.\mole^{\zeta}
%.\candela^{\eta}}\), which is obtained by replacing the dimensions of the SI base
% quantities in the dimension of \(Q\) with the symbols for the corresponding base units.
% \begin{quote}
% \Fe\ Consider a nonrelativistic particle of mass \(m\) in uniform motion which travels a
% distance \(l\) in a time \(t\). Its velocity is \(\upsilon=l / t\) and its kinetic energy
% is \(E_{\mathrm{k}} =m\upsilon^{2} /2 = l^{2}mt^{-2} / 2\). The dimension of
% \(E_{\mathrm{k}}\) is \(\dim E_{\mathrm{k}} = \dimn{L}^{2}\dimn{M}\dimn{T}^{-2}\) and the
% dimensional exponents are \(2\), \(1\), and~\(-2\). The SI derived unit of \(E_{\mathrm{k}}\) is then
% \(\joenit{\metre\squared.\kilogram.\second\rpsquared}\), which is given the special name
% ``joule'' and special symbol \joule.
% \end{quote}
%
% \subsubsection{Units for dimensionless quantities, quantities of dimension one}
% A derived quantity of dimension one, which is sometimes called a ``dimensionless
% quantity'', is one for which all of the dimensional exponents are zero: \(\dim Q = 1\). It
% therefore follows that the derived unit for such a quantity is also the number one,
% symbol 1, which is sometimes called a ``dimensionless derived unit''.
% Thus the SI unit of all quantities having the dimensional product one is the
% number one. Examples of such quantities are refractive index, relative permeability, and
% friction factor. All of these quantities are described as being dimensionless, or of
% dimension one, and have the coherent SI unit 1. Their values are simply expressed as
% numbers and, in general, the unit 1 is not explicitly shown.
% \begin{quote}
% \Fe\ The mass fraction \(w_{\mathrm{B}}\) of a substance B in a mixture is given by
% \(w_{\mathrm{B}} = m_{\mathrm{B}}/m\), where \(w_{\mathrm{B}}\) is the mass of B and
% \(m\) is the mass of the mixture. The dimension of \(w_{\mathrm{B}}\) is \(\dim
% w_{\mathrm{B}} = \dimn{M}^{1} \dimn{M}^{-1} = 1\); all of the dimensional exponents of
% \(w_{\mathrm{B}}\) are zero, and its derived unit is \(\joenit{\kilogram^{1}.\kilogram^{-1}}=
% 1\) also.
% \end{quote}
% In a few cases, however, a special name is given to this unit, mainly to avoid
% confusion between some compound derived units. This is the case for the radian,
% steradian and neper.
%
%\subsection{Rules and style conventions for writing and using SI unit symbols}
%The general principles concerning writing the unit symbols were adopted by the 9th CPGM
%(1948), by its Resolution 7:
%\begin{enumerate}
%\item Roman (upright) type, in general lower case\footnote{The recommended symbol for the
%litre (`liter') in the United States is \liter.}, is used for the unit symbols. If,
%however, the name of the unit is derived from a proper name, the first letter of the symbol
%is in upper case.
% \item Unit symbols are unaltered in the plural.
% \item Unit symbols are not followed by a period\footnote{Unless at the end of a sentence.}.
%\end{enumerate}
%To ensure uniformity in the use of the SI unit symbols, ISO International Standards give
%certain recommendations. Following these recommendations:
%\begin{list}{a)}{}
%\item The product of two or more units are indicated by means of either a half-high (that
%is, centred) dot or a space\footnote{ISO suggests that if a space is used to indicate units
%formed by multiplication, the space may be omitted if it does not cause confusion. This
%possibility is reflected in the common practice of using the symbol \kilo\watt\hour\ rather
%than \kilo\watt\(\cdot\)\hour\ or \joenit{\kilo\watt.\hour}\ for the kilowatt hour.}. The
%half-high dot is preferred, because it is less likely to lead to confusion,
% \begin{list}{}{}
% \item \fe
% \item \newton\(\cdot\)\metre\ or \(\newton\;\metre\).
% \end{list}
%\end{list}
%\begin{list}{b)}{}
% \item A solidus (oblique stroke,/), a horizontal line, or negative exponents may be used to
%express a derived unit formed from two others by division,
% \begin{list}{}{}
% \item \fe
% \item \metre\per\second,\ \(\frac{\metre}{\second}\),\ or \joenit{\metre.\reciprocal\second}
% \end{list}
%\end{list}
%\begin{list}{c)}{}
% \item The solidus must not be repeated on the same line unless ambiguity is avoided by
% parentheses. In complicated cases negative exponents or parentheses should be used,
% \begin{list}{}{}
% \item \fe
% \item \metre\per\second\squared\ or\ \joenit{\metre.\second\rpsquared}\ \textit{but not:\ }\metre\per\second\per\second\
% \item \joenit{\metre.\kilogram\per(\cubic\second.\ampere)}\ or
% \joenit{\metre.\kilogram.\rpcubic\second.\reciprocal\ampere}\ \textit{but not:}\ \joenit{\metre.\kilogram\per\cubic\second\per\ampere}
% \end{list}
%\end{list}
%\subsubsection{Space between numerical value and unit symbol}
%In the expression for the value of a quantity, the unit symbol is placed after the numerical value
%and a space is left between the numerical value and the unit symbol.
%The only exceptions to this rule are for the unit symbols for degree, minute, and second for plane
%angle: \degree, \paminute, and \pasecond, respectively (see Table~\ref{table:accepted1}), in which case
%no space is left between the numerical value and the unit symbol.
% \begin{list}{}{}
% \item \fe
% \item \(\alpha = \unit{30}{\degree}\unit{22}{\paminute}\unit{8}{\pasecond}\) Note: \(\alpha\) is a quantity symbol for plane angle.
% \end{list}
% This rule means that the symbol \celsius\ for the degree Celsius is preceded by a space when one expresses the values of
% Celsius temperatures.
% \begin{list}{}{}
% \item \fe
% \item \(t = \unit{30.2}{\celsius}\) \textit{but not} \(t = 30.2\celsius\)
% \end{list}
%\section{SI Prefixes}
%\subsection{Decimal multiples and submultiples of SI units}
%The 11th CGPM (1960), by its Resolution 12, adopted a first series of prefixes and symbols
%of prefixes to form the names and symbols of the decimal multiples and submultiples of SI
%units. Prefixes for \power {10}{-15} and \power {10}{-18} were added by the 12th CGPM
%(1964), by its Resolution 8, those for \power {10}{15} and \power {10}{18} by the CGPM
%(1975), by its Resolution 10, and those for \power {10}{21}, \power {10}{24}, \power {10}
%{-21}, and \power {10}{-24} were proposed by the CIPM for approval by the 19th CGPM (1991),
%and adopted. The prefixes are as shown in tabel~\ref{table:prefixes}.
%   \begin{table}[btp]
%     \caption{--- SI prefixes ---}\label{table:prefixes}
%   \begin{minipage}[]{\textwidth}\renewcommand{\thefootnote}{\textit{\alph{footnote}}}
%     \centering
%   \begin{tabular}{llllll}\hline
%   \textbf{Name}& \textbf{Symbol}& \textbf{Factor} & \textbf{Name}& \textbf{Symbol}& \textbf{Factor}\\ \hline
%   yocto  &  \yocto  &  \yoctod = \power{(\kilod)}{-8}& yotta      &  \yotta  &  \yottad = \power{(\kilod)}{8}\\
%   zepto  &  \zepto  &  \zeptod = \power{(\kilod)}{-7}& zetta      &  \zetta  &  \zettad = \power{(\kilod)}{7}\\
%   atto  &  \atto  &      \attod = \power{(\kilod)}{-6}   & exa   &  \exa   &  \exad = \power{(\kilod)}{6}\\
%   femto  &  \femto  &  \femtod = \power{(\kilod)}{-5}& peta      &  \peta  &  \petad = \power{(\kilod)}{5}\\
%   pico  &  \pico  &      \picod = \power{(\kilod)}{-4}   & tera  &  \tera  &  \terad = \power{(\kilod)}{4}\\
%   nano  &  \nano  &      \nanod = \power{(\kilod)}{-3}   & giga  &  \giga  &  \gigad = \power{(\kilod)}{3}\\
%   micro  &  \micro  &  \microd = \power{(\kilod)}{-2}& mega      &  \mega  &  \megad = \power{(\kilod)}{2}\\
%   milli  &  \milli  &  \millid = \power{(\kilod)}{-1}& kilo      &  \kilo  &  \kilod = \power{(\kilod)}{1}\\
%   centi  &  \centi  &  \centid           & hecto  &  \hecto  &  \hectod \\
%   deci  &  \deci  &      \decid          & deca\footnotemark[1]  &  \deca  &  \decad \\
%
%   \end{tabular}
%    \footnotetext[1]
%   {In the USA, the spelling `deka' is extensively used.}
%   \end{minipage}
%   \end{table}
%   \subsection{Rules for using SI prefixes}
%   In accord with the general principles adopted by the ISO\footnote{ISO 31, in `Units of
%   measurement,' ISO Standards Handbook 2, 2nd Edition, ISO, Geneva, 1982, pp. 17--238}, the
%   CIPM recommends that the following rules for using the SI prefixes be observed:
%   \begin{enumerate}
%   \item Prefix symbols are printed in roman (upright) type without spacing between the prefix
%   symbol and the unit symbol. \item The grouping formed by the prefix symbol attached to the
%   unit symbol constitutes a new inseparable symbol (of a multiple of the unit concerned) which
%   can be raised to a positive or negative power and which can be combined with other unit
%   symbols to form compound unit symbols,
%   \begin{list}{}{}
%   \item \fe
%    \item \(\unit{1}{\centi\cubic\metre} = \cubic{(\unit{\power{10}{-2}}{\metre})} = \unit{\power{10}{-6}}{\cubic\metre}\)
%    \item \(\unit{1}{\reciprocal{\centi\metre}} = \reciprocal{(\unit{\centid}{\metre})} = \unit{\hectod}{\reciprocal\metre}\)
%    \item \(\unit{1}{\volt\per\centi\metre} = (\unit{1}{\volt})\per(\unit{\centid}{\metre}) = \unit{\hectod}{\volt\per\metre}\)
%   \end{list}
%   \item Compound prefixes, i.\,e., prefixes formed by juxtaposition of two or more SI prefixes
%   are not to be used,
%   \begin{list}{}{}
%   \item \fe \item \unit{1}{\pico\gram}\ (one picogram), \textit{but not} \unit{1}
%   {\milli\nano\gram}\ (one millinanogram)
%   \end{list}
%   \item A prefixes should never be used alone,
%   \begin{list}{}{}
%   \item \fe \item \megad\per\cubic\metre, \textit{but not}
%   \mega\per\cubic\metre
%   \end{list}
%   \end{enumerate}
%   \subsubsection{The kilogram}
%   It is important to note that the kilogram is the only SI unit with a prefix as part of its
%   name and symbol. Because multiple prefixes may not be used, in the case of the kilogram the
%   prefix names are used with the unit name `gram' and the prefix symbols are used with the
%   unit symbol \gram, \fe\\ \unit{\microd}{\kilogram}\ = \unit{1}{\milli\gram}\ (one
%   milligram), \textit{but not} \unit{\microd}{\kilogram}\ = \unit{1}{\micro\kilogram} ~(one
%   microkilogram).
%   \subsubsection{The `degree Celsius'} Except for the kilogram, any SI prefix may be used with any SI unit,
%   including the `degree Celsius' and its symbol \celsius, \fe \\
%   \unit{\millid}{\celsius}\ = \unit{1}{\milli\celsius}\ (one millidegree Celsius), \textit{or}
%   \unit{\megad}{\celsius}\ = \unit{1}{\mega\celsius}.
%    \section{Prefixes for binary multiples} In December 1998 the International
%    Electrotechnical Commission (IEC), the leading international organization for worldwide
%    standardization in electrotechnology, approved as an IEC International Standard names and
%    symbols for prefixes for binary multiples for use in the fields of data processing and
%    data transmission. The prefixes are as shown in table~\ref{table:bipre}.
%   \begin{table}[btp]
%     \caption{--- Prefixes for binary multiples ---}\label{table:bipre}
%   \begin{minipage}[]{\textwidth}\renewcommand{\thefootnote}{\textit{\alph{footnote}}}
%     \centering
%   \begin{tabular}{lllllll}\hline
%    \textbf{Factor} & \textbf{Name} &      \textbf{Symbol} &  \textbf{Origin}& &  \multicolumn{2}{l}{\textbf{Derivation}}\\
%    \hline
%    \kibid   &  kibi         &   \kibi     &  kilobinary: & \power{(\power{2}{10})}{1}  &  kilo:& \power{(\kilod)}{1}\\
%    \mebid   &  mebi         &   \mebi     &  megabinary: & \power{(\power{2}{10})}{2}  &  mega:& \power{(\kilod)}{2}\\
%    \gibid   &  gibi         &   \gibi     &  gigabinary: & \power{(\power{2}{10})}{3}  &  giga:& \power{(\kilod)}{3}\\
%    \tebid   &  tebi         &   \tebi     &  terabinary: & \power{(\power{2}{10})}{4}  &  tera:& \power{(\kilod)}{4}\\
%    \pebid   &  pebi         &   \pebi     &  petabinary: & \power{(\power{2}{10})}{5}  &  peta:& \power{(\kilod)}{5}\\
%    \exbid   &  exbi         &   \exbi     &  exabinary:  & \power{(\power{2}{10})}{6}  &  exa: & \power{(\kilod)}{6}\\
%   \end{tabular}
%   \end{minipage}
%   \end{table}
%   \begin{table}[btp]
%     \caption{--- Examples and comparisons with SI prefixes ---}\label{table:bipre2}
%   \begin{minipage}[]{\textwidth}\renewcommand{\thefootnote}{\textit{\alph{footnote}}}
%     \centering
%   \begin{tabular}{lllllll}\hline
%    one kibibit  & \unit{1}{\kibi\bit } & = & \unit{\kibid}{\bit } & = & \unit{1\,024      }{\bit }\\
%    one kilobit  & \unit{1}{\kilo\bit } & = & \unit{\kilod}{\bit } & = & \unit{1\,000      }{\bit }\\
%    one mebibyte & \unit{1}{\mebi\byte} & = & \unit{\mebid}{\byte} & = & \unit{1\,048\,576}{\byte}\\
%    one megabyte & \unit{1}{\mega\byte} & = & \unit{\megad}{\byte} & = & \unit{1\,000\,000}{\byte}\\
%    one gibibyte & \unit{1}{\gibi\byte} & = & \unit{\gibid}{\byte} & = & \unit{1\,073\,741\,824}{\byte}\\
%    one gigabyte & \unit{1}{\giga\byte} & = & \unit{\gigad}{\byte} & = & \unit{1\,000\,000\,000}{\byte}\\
%   \end{tabular}
%   \end{minipage}
%   \end{table}%
%    It is suggested that in English, the first syllable of the name of the binary-multiple
%    prefix should be pronounced in the same way as the first syllable of the name of the
%    corresponding SI prefix, and that the second syllable should be pronounced as ``bee".
%
%\subsubsection*{Note}
%It is important to recognize that the new prefixes for binary multiples are \textit{not}
%part of the International System of Units (SI), the modern metric system. However, for
%ease of understanding and recall, they were derived from the SI prefixes for positive
%powers of ten. As can be seen from the above table, the name of each new prefix is
%derived from the name of the corresponding SI prefix by retaining the first two letters
%of the name of the SI prefix and adding the letters ``\textrm{bi}", which recalls the word
%``binary". Similarly, the symbol of each new prefix is derived from the symbol of the
%corresponding SI prefix by adding the letter ``\textrm{i}", which again recalls the word
%``binary". (For consistency with the other prefixes for binary multiples, the symbol
%\kibi\ is used for \kibid\ rather than \textrm{ki}.)
%\subsection{Official publication}
%These prefixes for binary multiples, which were developed by IEC Technical Committee (TC)
%25, Quantities and units, and their letter symbols, with the strong support of the
%International Committee for Weights and Measures (CIPM) and the Institute of Electrical
%and Electronics Engineers (IEEE), were adopted by the IEC as \textit{Amendment 2 to IEC
%International Standard IEC 60027-2: Letter symbols to be used in electrical technology -
%Part 2: Telecommunications and electronics}. The full content of \textit{Amendment 2},
%which has a publication date of 1999-01, is reflected in the tables above and the
%suggestion regarding pronunciation.
%\changes{v0.99}{1999/07/28}{New binary prefixes support/documentation section added}
%\subsection{The \texttt{binary.sty} style for binary prefixes and (non-SI) units}
%   The \texttt{binary.sty} style for binary prefixes and (non-SI) units can be loaded by using the option \opt{binary}, as in |\usepackage[binary]{SIunits}|. This unit should always
%   be used in conjunction with the \pkgname{SIunits}\ package.
%   \section{Units outside the SI}
%   Units that are outside the SI may be divided into three categories:
%   \begin{enumerate}
%   \item those units that are accepted for use with the SI;
%   \item those units that are temporarily accepted for use with the SI; and
%   \item those units that are not accepted for use with the SI and thus must strictly be avoided.
%   \end{enumerate}
%   \subsection{Units accepted for use with the SI}
%   The CIPM (1969) recognised that users of SI will also wish to employ with it certain units not part of it, but
%   which are important and are widely used. These units are given in table~\ref{table:accepted1}. The combination of
%   units of this table with SI units to form compound units should be restricted to special cases in order not
%   to lose the advantage of the coherence of SI units.\\
%    \begin{table}[btp]
%      \caption{--- Units accepted for use with the SI ---}\label{table:accepted1}
%    \begin{minipage}[]{\textwidth}\renewcommand{\thefootnote}{\textit{\alph{footnote}}}
%      \centering
%    \begin{tabular}{lll}\hline
%    \textbf{Name} & \textbf{Symbol} &       \textbf{Value in SI units}\\ \hline
%     minute (time)    &     \minute    &   \unit{1}{\minute} = \unit{60}{\second}   \\
%     hour     &      \hour     &   \unit{1}{\hour} = \unit{60}{\minute} = \unit{\intfmt{3600}}{\second} \\
%     day    &  \dday &      \unit{1}{\dday} = \unit{24}{\hour} = \unit{\intfmt{86400}}{\second} \\
%     degree\footnotemark[1]  &     \degree    &   \unit{1}{\degree} = \unit{(\pi/180)}{\radian}  \\
%     minute (plane angle)    & \paminute &   \unit{1}{\paminute} = \unit{(1/60)}{\degree} = \unit{(\pi/\intfmt{10800})}{\radian}      \\
%     second (plane angle)      & \pasecond     &  \unit{1}{\pasecond} = \unit{(1/60)}{\paminute} = \unit{(\pi/\intfmt{648000})}{\radian}  \\
%     litre    &      \litre, \liter\footnotemark[2]    &   \unit{1}{\litre} = \unit{1}{\liter} = \unit{1}{\cubic{\deci\metre}} = \unit{\millid}{\cubic\metre} \\
%     tonne\footnotemark[3]    &      \tonne &   \unit{1}{\ton} = \unit{\kilod}{\kilogram}   \-\\
%     neper\footnotemark[4]\footnotemark[5]    &     \neper    &   \unit{1}{\neper} = \unit{1}{\one}   \\
%     bel\footnotemark[6]\footnotemark[5]     &      \bel     &   \unit{1}{\bel} = \unit{(1/2)\ln 10}{(\neper)}\footnotemark[7]\\
%    \end{tabular}
%     \footnotetext[1]
%    {ISO 31 recommends that the degree be subdivided decimally rather than using the minute
%    and second.}
%     \footnotetext[2]
%    {The alternative symbol for the litre, \liter, was adopted by the CGPM in order to avoid
%    the risk of confusion between the letter l and the number \(1\). Thus, although both
%    \litre\ and \liter\ are internationally accepted symbols for the litre, to avoid this
%    risk the symbol to be used in the United States is \liter.}
%     \footnotetext[3]
%    {In some English-speaking countries this unit is called `metric ton'.}
%     \footnotetext[4]
%    {The neper is used to express values of such logarithmic quantities as field level, power
%    level, sound pressure level, and logarithmic decrement. Natural logarithms are used to
%    obtain the numerical values of quantities expressed in nepers. The neper is coherent with
%    the SI, but not yet adopted by the CGPM as an SI unit. For further information see
%    International Standard ISO 31.}
%     \footnotetext[5]
%    {The bel is used to express values of such logarithmic quantities as field level, power
%    level, sound pressure level, and attenuation. Logarithms to base ten are used to obtain
%    the numerical values of quantities expressed in bels. The submultiple decibel, \deci\bel,
%    is commonly used. For further information see International Standard ISO 31.}
%     \footnotetext[6]
%    {In using these units it is particularly important that the quantity be specified. The
%    unit must not be used to imply the quantity.}
%     \footnotetext[7]
%    {\neper\ is enclosed in parentheses because, although the neper is coherent with the SI,
%    it has not yet been adopted by the CGPM.}
%
%
%    \end{minipage}
%    \end{table}
%   It is likewise necessary to recognise, outside the International System, some other units that are useful in
%   specialised fields, because their values expressed in SI units must be obtained by experiment, and are therefore
%   not known exactly (table~\ref{table:accepted2}).
%   \begin{table}[btp]
%     \caption{--- Units accepted for use with the SI whose values in SI units are obtained experimentally ---}\label{table:accepted2}
%   \begin{minipage}[]{\textwidth}\renewcommand{\thefootnote}{\textit{\alph{footnote}}}
%     \centering
%   \begin{tabular}{llc}\hline
%    \textbf{Name}            & \textbf{Symbol}  & \textbf{Definition}\\ \hline
%    electronvolt             &  \electronvolt   & \footnotemark[1]   \\
%    unified atomic mass unit &  \atomicmass     & \footnotemark[2]   \\
%    \end{tabular}
%    \footnotetext[1]
%   {The electronvolt is the kinetic energy acquired by an electron in passing through a
%   potential difference of \(1\) \volt\ in vacuum; \unit{1}{\electronvolt} =
%   \(\unit{\sci{\nmfmt{1.60217733}e-19}}{\joule}\) with a combined standard
%   uncertainty of \(\unit{\sci{\nmfmt{0.00000049}e-19}}{\joule}\).}
%    \footnotetext[2]
%   {The unified atomic mass unit is equal to \(1/12\) of the mass of an atom of the nuclide
%   \(^{12}\mathrm{C}\); \unit{1}{\atomicmass} =
%   \unit{\sci{\nmfmt{1.6605402}e-27}}{\kilogram} with a combined standard
%   uncertainty of \unit{\sci{\nmfmt{0.0000010}e-27}}{\kilogram}.}
%   \end{minipage}
%   \end{table}
%   \subsection{Units temporarily accepted for use with the SI}
%   Because of existing practice in certain fields or countries, in 1978 the CIPM considered that it was permissible
%   for the units given in table~\ref{table:temporarily} to continue to be used with the SI until the CIPM considers that
%   their use is no longer necessary. However, these units must not be introduced where they are not presently used.
%   \begin{table}[btp]
%     \caption{--- Units in use temporarily with the SI ---}\label{table:temporarily}
%   \begin{minipage}[]{\textwidth}\renewcommand{\thefootnote}{\textit{\alph{footnote}}}
%     \centering
%   \begin{tabular}{lll}\hline
%   \textbf{Name} & \textbf{Symbol} & \textbf{Value in SI units}\\ \hline
%   nautical mile\footnotemark[1]    &      \     &   \unit{1}{nautical\ mile} = \unit{\intfmt{1852}}{\metre}   \\
%   knot     &      \     &   \unit{1}{nautical\ mile\ per\ hour} = \unit{(\intfmt{1852}/\intfmt{3600})}{\metre\per\second} \\
%   \aa ngstr\"{o}m &     \angstrom    &   \unit{1}{\angstrom} = \unit{0.1}{\nano\metre} = \unit{\power{10}{-10}}{\metre} \\
%   are\footnotemark[2]  &  \are    &     \unit{1}{\are} = \unit{1}{\square{\deca\metre}} = \unit{\power{10}{2}}{\square\metre} \\
%   hectare\footnotemark[2]   &  \hectare    &     \unit{1}{\hectare} = \unit{1}{\square{\hecto\metre}} = \unit{\power{10}{4}}{\square\metre} \\
%   barn\footnotemark[3]   &  \barn   &  \unit{1}{\barn} = \unit{100}{\square{\femto\metre}} = \unit{\power{10}{-28}}{\square\metre} \\
%   bar\footnotemark[4]   &  \bbar   &      \unit{1}{\bbar} = \unit{0.1}{\mega\pascal} = \unit{\power{10}{5}}{\pascal} \\
%   gal\footnotemark[5]   &  \gal    &     \unit{1}{\gal} = \unit{1}{\centi\metre\per\second\squared} = \unit{\power{10}{-2}}{\metre\per\second\squared} \\
%   curie\footnotemark[6]   &      \curie   &  \unit{1}{\curie} = \(\unit{\sci{3.7e10}}{\becquerel}\) \\
%   roentgen\footnotemark[7]       &  \roentgen   & \unit{1}{\roentgen} = \(\unit{\sci{2.58e-4}}{\coulomb\per\second}\)  \\
%   rad\footnotemark[8]   &  \rad    &     \unit{1}{\rad} = \unit{1}{\centi\gray} = \unit{\power{10}{-2}}{\gray} \\
%   rem\footnotemark[9]   &  \rem    &     \unit{1}{\rem} = \unit{1}{\centi\sievert} = \unit{\power{10}{-2}}{\sievert} \\
%   \end{tabular}
%    \footnotetext[1]
%   {The nautical mile is a special unit employed for marine and aerial navigation to express
%   distances. The conventional value given above was adopted by the First International
%   Extraordinary Hydrographic Conference, Monaco, 1929, under the name ``International
%   nautical mile''.}
%    \footnotetext[2]
%   {This unit and its symbol were adopted by the CIPM in 1879 (BIPM Proc. Verb. Com. Int.
%   Poids et Mesures, 1879, p. 41) and are used to express agrarian areas.}
%    \footnotetext[3]
%   {The barn is a special unit employed in nuclear physics to express effective cross
%   sections.}
%    \footnotetext[4]
%   {This unit and its symbol are included in Resolution 7 of the 9th CGPM (1948).}
%    \footnotetext[5]
%   {The gal is a special unit employed in geodesy and geophysics to express the acceleration
%   due to gravity.}
%    \footnotetext[6]
%   {The curie is a special unit employed in nuclear physics to express activity of
%   radionuclides (12th CGPM (1964), Resolution 7).}
%    \footnotetext[7]
%   {The roentgen is a special unit employed to express exposure of x or \(\gamma\)
%   radiations.}
%    \footnotetext[8]
%   {The rad is a special unit employed to express absorbed dose of ionising radiations. When
%   there is risk of confusion with the symbol for radian, rd may be used as the symbol for
%   rad.}
%    \footnotetext[9]
%   {The rem is a special unit used in radioprotection to express dose equivalent.}
%
%   \end{minipage}
%   \end{table}
%
%   \section{Last notes about correct usage of the SI}
%   The following points underline some of the important aspects about using SI units and their
%   symbols, and also mention some of the common errors that are made. The SI differs from some
%   of the older systems in that it has \textit{definite} rules governing the way the units and
%   symbols are used.
%   \begin{itemize}
%   \item The unit of measure is the \textit{`metre'}, not \textit{`meter'}. The latter is a device used
%   for measuring things. (Unless you live in the USA - in which case you will just have to live
%   with the ambiguity.) \item Using a comma to separate groups of three digits is not
%   recommended - a (thin) space is preferable, since many countries use the comma as the decimal point
%   marker. Both the USA and UK use the `dot on the line' (full stop). So the following would be
%   correct: \nmfmt{1234555.678990}. \item The term \textbf{billion} should be avoided since in
%   most countries outside the USA (including the UK) it means a million-million (prefix tera),
%   whereas in the USA it means a thousand million (prefix giga). Likewise the term \textbf{trillion}
%   means million-million-million (prefix exa) in most countries outside the USA.
%   \item The `litre' (`liter' in the US) is one of those units which is approved by the CGPM for
%   use with the metric system. The official unit of volume in the SI is the cubic metre.
%   However, since this is not convenient for much day-to-day use the CGPM has approved the use
%   of the `other unit', the litre. The litre represents a cubic decimetre and you may use
%   either the symbol `\litre' or `\liter'\footnote{Recommended symbol for the `liter' in the USA} (small
%   or capital `ell') to represent it. They do not approve using any prefixes other than milli
%   or micro with it. It was originally defined as the volume occupied by \unit{1}{\kilogram}\ of water.
%   Subsequently it was found that this was not precisely 1 cubic decimetre, so the term litre
%   was withdrawn. Later it was re-introduced officially as 1 cubic decimetre exactly. So,
%   \unit{1}{\litre} = \unit{1}{\cubic{\deci\metre}} = \unit{1}{\liter}.
%   \end{itemize}
%
%    \section{How to use the package}
%
%    \subsection{Loading}
%    Most features are controlled by package options that can be selected
%    when the package is loaded (e.\,g |\usepackage|\oarg{options}|{SIunits}|) or at
%    `runtime' as an optional argument(list) to the |\SIunits| command
%    (e.\,g. \cmd{\SIunits}\oarg{options}).
%
%    \begin{verbatim}
%    \documentclass[]{article}
%
%    \usepackage[options]{SIunits}
%
%    \begin{document}
%    ...
%    \SIunits[options]
%    ...
%    \end{document}
%    \end{verbatim}
%
%
%    \subsection{The package options}\label{sec:options}
%    \changes{v0.02 Beta  2}{1998/09/10}{Code documentation corrections}
%    The options can be grouped in the following categories:
%    \begin{enumerate}
%    \item unit spacing;
%    \item quantity-unit spacing;
%    \item conflicts;
%    \item textstyle;
%    \item miscellaneous.
%    \end{enumerate}
%    \subsubsection{Unit spacing options}
%    \begin{labeling}{\hspace{12mm}}
%    \item[\opt{cdot}]
%    This mode provides the use of \cmd{\cdot} as spacing in units.
%    \item[\opt{thickspace}]
%    This mode provides the use of \cmd{\;} (thick math space) as spacing in units.
%    \item[\opt{mediumspace}]
%    This mode provides the use of \cmd{\:} (medium math space) as spacing in units.
%    \item[\opt{thinspace}]
%    This mode provides the use of \cmd{\,} (thin math space) as spacing in units.
%    \end{labeling}
%    \subsubsection{Quantity-unit spacing options}
%    \begin{labeling}{\hspace{12mm}}
%    \item[\opt{thickqspace}]
%    This mode provides the use of \cmd{\;} (thick math space) as spacing between numerical quantities and units.
%    \item[\opt{mediumqspace}]
%    This mode provides the use of \cmd{\:} (medium math space) as spacing between numerical quantities and units.
%    \item[\opt{thinqspace}]
%    This mode provides the use of \cmd{\,} (thin math space) as spacing between numerical quantities and units.
%    \end{labeling}
%    \subsubsection{Options to prevent conflicts}
%    \subsubsection*{Conflicts with the \texttt{amssymb} package}
%    In the |amssymb| package the command |\square| is defined. This will cause error messages
%    when the |amssymb| package is used in combination with the \pkgname{SIunits}\ package. To prevent
%    errors one can choose two different options:
%    \begin{labeling}{\hspace{15mm}}
%    \item[\opt{amssymb}]This option redefines the \texttt{amssymb} command |\square| to get the desired
%    \pkgname{SIunits}\ definition of the command.\\ \textbf{Note: }When using this option, the \texttt{amssymb} command
%    |\square| can \textbf{not} be used.
%    \item[\opt{squaren}]This option defines a new command |\squaren| that can be used instead of the
%    \pkgname{SIunits}\ command |\square|.\\ \textbf{Note: }When using this option, the \texttt{amssymb} definition for
%    |\square| is used.
%    \end{labeling}
%    \changes{v0.06 Beta  2}{1998/04/07}{amssymb conflicts section added to documentation}
%    \subsubsection*{Conflicts with the \texttt{pstricks} package}
%    In the |pstricks| package the command |\gray| is defined. This will cause error messages
%    when the |pstricks| package is used in combination with the \pkgname{SIunits}\ package. To prevent
%    errors one can choose two different options:
%    \changes{v0.99}{1999/10/29}{pstricks conflicts section added to documentation}
%    \begin{labeling}{\hspace{15mm}}
%    \item[\opt{pstricks}]This option redefines the \texttt{pstricks} command |\gray| to get the desired
%    \pkgname{SIunits}\ definition of the command.\\ \textbf{Note: }When using this option, the \texttt{pstricks} command
%    |\gray| can \textbf{not} be used.
%    \item[\opt{Gray}]This option defines a new command |\Gray| that can be used instead of the
%    \pkgname{SIunits}\ command |\gray|.\\ \textbf{Note: }When using this option, the \texttt{pstricks} definition for
%    |\gray| is used.
%    \end{labeling}
%    \subsubsection*{Conflicts with the \texttt{babel} package in combination with the \opt{italian} language}
%    In the |babel| package, when using the italian language, the command |\unit| is defined. This will prevent \pkgname{SIunits}\ from functioning.
%    To prevent this, choose the option:
%    \begin{labeling}{\hspace{15mm}}
%    \item[\opt{italian}]This option defines a new command |\unita| (italian for unit) that can be used instead of the
%    \pkgname{SIunits}\ command |\unit|.\\ \textbf{Note: }When using this option, the \texttt{babel} definition for
%    |\unit| is used.
%    \end{labeling}
%    \subsubsection{textstyle}
%    \begin{labeling}{\hspace{12mm}}
%    \item[\opt{textstyle}]
%    When using the option \opt{textstyle} units are printed in the typeface of the
%    enclosing text, automatically.
%    \end{labeling}
%    \subsubsection{miscellaneous}
%    \begin{labeling}{\hspace{12mm}}
%    \item[\opt{binary}]
%    This option loads the file \texttt{binary.sty}, which defines prefixes for binary multiples.
%    \item[\opt{noams}]
%    This option redefines the \cmd{\micro} command; use it when you don't have the AMS font, eurm10.
%    \item[\opt{derivedinbase}]
%    This mode provides the ready-to-use expressions of SI derived units in SI base units,
%    e.\,g. \verb=\pascalbase= to get `\pascalbase'.
%    \item[\opt{derived}]
%    This mode provides the ready-to-use expressions of SI derived units in SI derived units,
%    e.\,g. \verb=\derpascal= to get `\derpascal'.
%    \end{labeling}
%    \changes{v0.02 Beta  2}{1998/09/10}{Spacing examples table added}
%    \changes{v0.03 Beta  2}{1998/10/30}{Spacing examples table update}
%    See table~\ref{table:spacing} for examples of the spacing options.
%   \begin{table}[btp]
%     \caption{--- Spacing options ---}\label{table:spacing}
%   \begin{minipage}[]{\textwidth}\renewcommand{\thefootnote}{\textit{\alph{footnote}}}
%   \centering
%   \begin{tabular}{lll}\hline
%   \textbf{Option} &      \textbf{Example}\\ \hline
%   \opt{cdot} &\(\newton\cdot\metre\)\\
%   \opt{thickspace}& \(\newton\;\metre\)\\
%   \opt{mediumspace}&\(\newton\:\metre\)\\
%   \opt{thinspace}&\(\newton\,\metre\)\\
%   \opt{thickqspace}& \(10\;\newton\;\metre\)\\
%   \opt{mediumqspace}&\(10\:\newton\;\metre\)\\
%   \opt{thinqspace}&\(10\,\newton\;\metre\)\\
%   \end{tabular}
%   \end{minipage}
%   \end{table}
%    \changes{v0.02 Beta  6}{1998/09/24}{Inconsistencies removed in documentation}
%    \normalsize
%    \section*{Command Reference}
%    \subsection{How to compose units in your text.}
%    The purpose of the \pkgname{SIunits}\ package is: to give an author an intuitive system for writing
%    units.
%    Just type (in \LaTeX-kind commands) what you would say: |\kilogram| or |\kelvin| to get
%    `\kilogram' or `\kelvin'.
%
%
%    To use the prefixes with SI units simply place them before the unit, e.\,g. |\milli\ampere|,
%    |\deca\metre| (or |\deka\meter|) or |\mega\ohm| to get: `\milli\ampere', `\deca\metre' or `\mega\ohm'.
%    Decimal values of the prefixes can be made by adding |d| behind the prefix command. See command reference on page~\pageref{commandreference}.
%     \begin{table}[p]
%     \label{commandreference}
%     \centering
%    \begin{tabular}{lr@{\ \vline\ }lr@{\ \vline\ }lr}
%    \multicolumn{6}{c}{\textbf{SI base units}}\\ \hline
%     |\metre|     & \metre      & |\second| & \second  & |\mole|    & \mole     \\
%     |\meter|     & \meter      & |\ampere| & \ampere  & |\candela| & \candela  \\
%     |\kilogram|  & \kilogram   & |\kelvin| & \kelvin                           \\ \hline \\
%    \multicolumn{6}{c}{\textbf{SI derived units}}\\ \hline
%     |\hertz|     & \hertz    & |\farad|    & \farad    & |\degreecelsius|     & \degreecelsius \\
%     |\newton|    & \newton   & |\ohm|      & \ohm      & |\lumen|             & \lumen        \\
%     |\pascal|    & \pascal   & |\siemens|  & \siemens  & |\lux|               & \lux          \\
%     |\joule|     & \joule    & |\weber|    & \weber    & |\becquerel|         & \becquerel    \\
%     |\watt|      & \watt     & |\tesla|    & \tesla    & |\gray|              & \gray         \\
%     |\coulomb|   & \coulomb  & |\henry|    & \henry    & |\sievert|           & \sievert      \\
%     |\volt|      & \volt     & |\celsius|  & \celsius                             \\ \hline   \\
%    \multicolumn{6}{c}{\textbf{Units outside of SI}}\\ \hline
%     |\angstrom|   & \angstrom    & |\dday|          &  \dday          & |\minute|       & \minute      \\
%     |\arcminute|  & \arcminute   & |\degree|        &  \degree        & |\neper|        & \neper       \\
%     |\arcsecond|  & \arcsecond   & |\electronvolt|  &  \electronvolt  & |\rad|          & \rad         \\
%     |\are|        & \are         & |\gal|           &  \gal           & |\rem|          & \rem         \\
%     |\atomicmass| & \atomicmass  & |\gram|          &  \gram          & |\roentgen|     & \roentgen    \\
%     |\barn|       & \barn        & |\hectare|       &  \hectare       & |\rperminute|   & \rperminute  \\
%     |\bbar|       & \bbar        & |\hour|          &  \hour          & |\tonne|        & \tonne       \\
%     |\bel|        & \bel         & |\liter|         &  \liter         & |\ton|          & \ton         \\
%     |\curie|      & \curie       & |\litre|         &  \litre         \\ \hline \\
%    \multicolumn{6}{c}{\textbf{SI Prefixes}}\\ \hline
%     |\yocto|  &  \yocto   & |\milli|  &  \milli   & |\mega|   &  \mega   \\
%     |\zepto|  &  \zepto   & |\centi|  &  \centi   & |\giga|   &  \giga   \\
%     |\atto|   &  \atto    & |\deci|   &  \deci    & |\tera|   &  \tera   \\
%     |\femto|  &  \femto   & |\deca|   &  \deca    & |\peta|   &  \peta   \\
%     |\pico|   &  \pico    & |\deka|   &  \deca    & |\exa|    &  \exa    \\
%     |\nano|   &  \nano    & |\hecto|  &  \hecto   & |\zetta|  &  \zetta  \\
%     |\micro|  &  \micro   & |\kilo|   &  \kilo    & |\yotta|  &  \yotta  \\ \hline \\
%    \multicolumn{6}{c}{\textbf{Decimal values of SI Prefixes}}\\ \hline
%     |\yoctod|  &  \yoctod   & |\millid|  &  \millid   & |\megad|   &  \megad   \\
%     |\zeptod|  &  \zeptod   & |\centid|  &  \centid   & |\gigad|   &  \gigad   \\
%     |\attod|   &  \attod    & |\decid|   &  \decid    & |\terad|   &  \terad   \\
%     |\femtod|  &  \femtod   & |\decad|   &  \decad    & |\petad|   &  \petad   \\
%     |\picod|   &  \picod    & |\dekad|   &  \dekad    & |\exad|    &  \exad    \\
%     |\nanod|   &  \nanod    & |\hectod|  &  \hectod   & |\zettad|  &  \zettad  \\
%     |\microd|  &  \microd   & |\kilod|   &  \kilod    & |\yottad|  &  \yottad  \\
%    \end{tabular}
%     \end{table}
%   \subsubsection{Division or multiplication of SI units}
%   The next step is the formation of units based on division and/or multiplication of SI units.
%   \paragraph{Division}
%   How to get the unit of speed?
%   \begin{enumerate}
%   \item Write down the unit in words: |metre per second|
%   \item Replace the spaces with backlashes to get the command: |\metre\per\second|
%   \item The result is: `\metre\per\second'.
%   \end{enumerate}
%   Simple! Ready!
%   \paragraph{Multiplication}
%    Now an example of multiplication of units, the unit of torque (newton metre):
%   \begin{enumerate}
%   \item Write down the unit in words: |newton metre|
%   \item To get an separation character between the two units use the command \cmd{\usk} (unitskip): |\newton\usk\metre|
%   \item The result is: `\joenit{\newton.\metre}'. The spacing between the units depends on the spacing
%   options (see: page \pageref{sec:options}).
%   \end{enumerate}
%   \paragraph{Mixed case}
%   The mixed case should be simple now; the unit of thermal conductivity (watt per metre kelvin):
%   \begin{enumerate}
%   \item Use your just-learned-knowledge:|\watt\per\metre\usk\kelvin|
%   \item The result is: `\wattpermetrekelvin'.
%   \end{enumerate}
%   Now, you can do it all in one step! Intuitive \& simple.
%
%   \subsubsection{Raising SI units to a power}
%   The \pkgname{SIunits}\ package provides a set of functions to get units raised to a particular power.
%   \paragraph{Squaring and cubing}
%   How to get the units of area (square metre) and volume (cubic metre)?
%   \begin{enumerate}
%   \item Write down the unit in words: |square metre| and |cubic metre|
%   \item Replace the spaces with backlashes to get the commands: |\square\metre| and |\cubic\metre|
%   \item The result is: ` \square\metre' and `\cubic\metre'.
%   \end{enumerate}
%   I can hear you say: ``We only use the word `square' before the unit metre, normally we place
%   the word `squared' behind the unit name.''. OK, lets try: |\second\squared| and
%   |\second\cubed| gives: `\second\squared' and `\second\cubed'.
%   Thus, no problem.
%   \paragraph{The reciprocal, reciprocal squaring and - cubing}
%   How to get negative powers?
%   \begin{enumerate}
%   \item Use |\rpsquare| or |\rpsquared|, and |\rpcubic| and |\rpcubed|
%   \item \Fe |\rpsquare\metre| and |\second\rpcubed|
%   \item The result is: ` \rpsquare\metre' and `\second\rpcubed'.
%   \end{enumerate}
%   Normally, we leave out the exponent \(1\), but sometimes we want to use the exponent \(-1\).
%   How to form the unit of frequency (reciprocal second = \hertz)
%   \begin{enumerate}
%   \item Write down the unit in words: |reciprocal second|,
%   \item Replace the spaces with backlashes to get the commands: |\reciprocal\second|,
%   \item The result is: `\reciprocal\second'.
%   \end{enumerate}
%   \paragraph{The power function}
%   The \cmd{\power} macro has been added to be able to form the wildest types of power raising:
%   |\power{10}{35}| gives: \power{10}{35}.
%
%    \subsection{Quantities and units}
%    Use the command \cmd{\unit} to get consistent spacing between numerical quantities and units. Usage:\\
%    |\unit{120}{\kilo\meter\per\hour}| gives: \unit{120}{\kilo\meter\per\hour}.
%    \changes{v0.03 Beta  2}{1998/10/30}{documentation update}
%
%    \subsubsection{Ready-to-use units}
%    {\begin{tabbing} \label{commands:prefab}
%    \changes{v0.05 Beta 1}{1999/02/01}{Inconsistencies in Ready-to-use units corrected}
%    \hspace{0.5cm}\= \textbf{xxxxxxxxxxxxxxxxxxxxxxxxxxxxxx}\hspace{1cm} \= \textbf{xxxxx}\kill \+
%    \\ |\amperemetresecond| \>  \amperemetresecond
%    \\ |\amperepermetre | \> \amperepermetre
%    \\ |\amperepermetrenp  | \> \amperepermetrenp
%    \\ |\amperepersquaremetre| \> \amperepersquaremetre
%    \\ |\amperepersquaremetrenp| \> \amperepersquaremetrenp
%    \\ |\candelapersquaremetre | \> \candelapersquaremetre
%    \\ |\candelapersquaremetrenp      | \> \candelapersquaremetrenp
%    \\ |\coulombpercubicmetre| \> \coulombpercubicmetre
%    \\ |\coulombpercubicmetrenp| \>  \coulombpercubicmetrenp
%    \\ |\coulombperkilogram  | \> \coulombperkilogram
%    \\ |\coulombperkilogramnp | \> \coulombperkilogramnp
%    \\ |\coulombpermol | \> \coulombpermol
%    \\ |\coulombpermolnp| \> \coulombpermolnp
%    \\ |\coulombpersquaremetre| \> \coulombpersquaremetre
%    \\ |\coulombpersquaremetrenp| \>  \coulombpersquaremetrenp
%    \\ |\cubicmetre| \> \cubicmetre
%    \\ |\faradpermetre | \> \faradpermetre
%    \\ |\faradpermetrenp      | \> \faradpermetrenp
%    \\ |\graypersecond| \> \graypersecond
%    \\ |\graypersecondnp| \> \graypersecondnp
%    \\ |\henrypermetre | \> \henrypermetre
%    \\ |\henrypermetrenp | \> \henrypermetrenp
%    \\ |\jouleperkelvin          | \> \jouleperkelvin
%    \\ |\jouleperkelvinnp  | \> \jouleperkelvinnp
%    \\ |\jouleperkilogram | \> \jouleperkilogram
%    \\ |\jouleperkilogramnp | \> \jouleperkilogramnp
%    \\ |\joulepermole|\> \joulepermole
%    \\ |\joulepermolenp|\> \joulepermolenp
%    \\ |\joulepermolekelvin | \> \joulepermolekelvin
%    \\ |\joulepermolekelvinnp | \> \joulepermolekelvinnp
%    \\ |\joulepersquaremetre | \> \joulepersquaremetre
%    \\ |\joulepersquaremetrenp | \> \joulepersquaremetrenp
%    \\ |\joulepertesla | \> \joulepertesla
%    \\ |\jouleperteslanp | \> \jouleperteslanp
%    \\ |\kilogrammetrepersecond          | \> \kilogrammetrepersecond
%    \\ |\kilogrammetrepersecondnp | \> \kilogrammetrepersecondnp
%    \\ |\kilogrammetrepersquaresecond | \> \kilogrammetrepersquaresecond
%    \\ |\kilogrammetrepersquaresecondnp| \> \kilogrammetrepersquaresecondnp
%    \\ |\kilogrampercubicmetre| \> \kilogrampercubicmetre
%    \\ |\kilogrampercubicmetrenp| \> \kilogrampercubicmetrenp
%    \\ |\kilogramperkilomole | \> \kilogramperkilomole
%    \\ |\kilogramperkilomolenp| \> \kilogramperkilomolenp
%    \\ |\kilogrampermetre | \> \kilogrampermetre
%    \\ |\kilogrampermetrenp  | \> \kilogrampermetrenp
%    \\ |\kilogrampersecond | \> \kilogrampersecond
%    \\ |\kilogrampersecondnp| \> \kilogrampersecondnp
%    \\ |\kilogrampersquaremetre | \> \kilogrampersquaremetre
%    \\ |\kilogrampersquaremetrenp| \> \kilogrampersquaremetrenp
%    \\ |\kilogrampersquaremetresecond | \> \kilogrampersquaremetresecond
%    \\ |\kilogrampersquaremetresecondnp| \> \kilogrampersquaremetresecondnp
%    \\ |\kilogramsquaremetre | \> \kilogramsquaremetre
%    \\ |\kilogramsquaremetrenp| \> \kilogramsquaremetrenp
%    \\ |\kilogramsquaremetrepersecond| \> \kilogramsquaremetrepersecond
%    \\ |\kilogramsquaremetrepersecondnp| \>  \kilogramsquaremetrepersecondnp
%    \\ |\kilowatthour| \>  \kilowatthour
%    \\ |\metrepersquaresecond| \>\metrepersquaresecond
%    \\ |\metrepersquaresecondnp| \>\metrepersquaresecondnp
%    \\ |\molepercubicmetre| \> \molepercubicmetre
%    \\ |\molepercubicmetrenp| \> \molepercubicmetrenp
%    \\ |\newtonpercubicmetre | \> \newtonpercubicmetre
%    \\ |\newtonpercubicmetrenp | \> \newtonpercubicmetrenp
%    \\ |\newtonperkilogram | \> \newtonperkilogram
%    \\ |\newtonperkilogramnp  | \> \newtonperkilogramnp
%    \\ |\newtonpersquaremetre | \> \newtonpersquaremetre
%    \\ |\newtonpersquaremetrenp| \> \newtonpersquaremetrenp
%    \\ |\ohmmetre | \> \ohmmetre
%    \\ |\pascalsecond| \>  \pascalsecond
%    \\ |\persquaremetresecond | \> \persquaremetresecond
%    \\ |\persquaremetresecondnp| \> \persquaremetresecondnp
%    \\ |\radianpersecond| \>  \radianpersecond
%    \\ |\radianpersecondnp| \> \radianpersecondnp
%    \\ |\radianpersquaresecond| \> \radianpersquaresecond
%    \\ |\radianpersquaresecondnp| \> \radianpersquaresecondnp
%    \\ |\squaremetre| \> \squaremetre
%    \\ |\squaremetrepercubicmetre| \> \squaremetrepercubicmetre
%    \\ |\squaremetrepercubicmetrenp| \> \squaremetrepercubicmetrenp
%    \\ |\squaremetrepernewtonsecond| \>          \squaremetrepernewtonsecond
%    \\ |\squaremetrepernewtonsecondnp| \> \squaremetrepernewtonsecondnp
%    \\ |\squaremetrepersecond | \> \squaremetrepersecond
%    \\ |\squaremetrepersecondnp          | \> \squaremetrepersecondnp
%    \\ |\squaremetrepersquaresecond  | \> \squaremetrepersquaresecond
%    \\ |\squaremetrepersquaresecondnp | \> \squaremetrepersquaresecondnp
%    \\ |\voltpermetre| \> \voltpermetre
%    \\ |\voltpermetrenp| \>  \voltpermetrenp
%    \\ |\wattpercubicmetre | \> \wattpercubicmetre
%    \\ |\wattpercubicmetrenp| \> \wattpercubicmetrenp
%    \\ |\wattperkilogram| \> \wattperkilogram
%    \\ |\wattperkilogramnp| \> \wattperkilogramnp
%    \\ |\wattpersquaremetre  | \> \wattpersquaremetre
%    \\ |\wattpersquaremetrenp | \> \wattpersquaremetrenp
%    \\ |\wattpersquaremetresteradian | \> \wattpersquaremetresteradian
%    \\ |\wattpersquaremetresteradiannp | \> \wattpersquaremetresteradiannp
%    \end{tabbing}}
%
%    \section{How the package works}
%    \subsection{Compatibility}
%    The package has been tested using:
%    \begin{enumerate}
%    \item MiK\TeX\ 1.10\textit{b}, including \LaTeXe\ standard classes (\LaTeXe{} [1997/12/01] patch level 1)
%    and \TeX\ 3.14159, both under Microsoft Windows 95 and MS Windows NT 4.0.
%    \item MiK\TeX\ 1.11, including \LaTeXe\ standard classes (\LaTeXe{} [1998/06/01])
%    and \TeX\ 3.14159, both under Microsoft Windows 95 and MS Windows NT 4.0.
%    \item MiK\TeX\ 2 UP 1, including \LaTeXe\ standard classes (\LaTeXe{} [2000/11/28])
%    and \TeX\ 3.14159, under Microsoft Windows 2000 professional.
%    \end{enumerate}
%    \subsection{Known problems and limitations}
%    \begin{enumerate}
%    \item When you don't have the AMS font |eurm10| use the option \opt{noams}.
%    \item The amssymb package defines the |\square| command. Two possible solutions to avoid
%    conflicts:
%    \begin{itemize}
%    \item Use option \opt{amssymb}: |\usepackage[amssymb]{SIunits}|.
%    When using this option the amssymb command |\square| is redefined to the SIunits command.
%    \item Use option \opt{squaren}: |\usepackage[squaren]{SIunits}|.
%    When using this option the amssymb command |\square| is not redefined. Use the newly defined SIunits command
%    |\squaren| instead of |\square| to get the desired behaviour.
%    \end{itemize}
%    \textbf{Note: }Load \pkgname{SIunits}\ package after \texttt{amssymb} package.
%    \item The pstricks package defines the |\gray| command. Two possible solutions to avoid
%    conflicts:
%    \begin{itemize}
%    \item Use option \opt{pstricks}: |\usepackage[pstricks]{SIunits}|.
%    When using this option the pstricks command |\gray| is redefined to the SIunits command.
%    \item Use option \opt{Gray}: |\usepackage[Gray]{SIunits}|.
%    When using this option the pstricks command |\gray| is not redefined. Use the newly defined SIunits command
%    |\Gray| instead of |\gray| to get the desired behaviour.
%    \end{itemize}
%    \textbf{Note: }Load \pkgname{SIunits}\ package after \texttt{pstricks} package.
%    \end{enumerate}
%    No further known problems or limitations. That doesn't mean this package is bug free, but it
%    indicates the lack of testing that's been done on the package.
%
%    \subsection{Sending a bug report}
%    Reports of new bugs in the package are most welcome.
%    However, I do \textbf{not} consider this to be a `supported'
%    package. This means that there is no guarantee I (or
%    anyone else) will put any effort into fixing the bug (of course I will try to find some time).
%    But, on the other hand, someone may try debugging, so filing a
%    bug report is always a good thing to do! (If nothing else,
%    your discoveries may end up in future releases of this
%    document.) Before filing a bug report, please take the
%    following actions:
%    \begin{enumerate}
%    \item Ensure your problem is not due to your input file;
%    \item Ensure your problem is not due to your own
%    package(s) or class(es);
%    \item Ensure your problem is not covered in the section ``Known
%    problems and limitations'' above;
%    \item Try to locate the problem by writing a minimal
%    \LaTeXe\ input file which reproduces the problem. Include the command\\
%    |\setcounter{errorcontextlines}{999}|\\ in your input;
%    \item Run your file through \LaTeXe;
%    \item Send a description of your problem, the input file and the
%    log file via e-mail to: \texttt{SIunits@webschool.nl}.
%    \end{enumerate}
%
%    \section{In conclusion}
%    \subsection{Acknowledgements}
%    I want to thank \textsf{Werenfried Spit} (|w.spit@WITBO.NL|)
%    answering my question to |TEX-NL@NIC.SURFNET.NL| about the ``power functie'',
%    as well as \textsf{Hans Hagen} (|pragma@WXS.NL|) for
%    the kind reaction to that question.
%    \changes{v0.02 Final Release}{1998/10/09}{Acknowledgements updated}
%    \begin{labeling}{\hspace{12mm}}
%    \item[\opt{v0.01}: Typos]
%    J\"{u}rgen von Haegen (\texttt{vonHagen@engr.psu.edu})
%    \item[\opt{v0.02 Beta  1}: \cmd{\addunit} macro added]
%    Hint: Hans Bessem\\ (\texttt{j.m.bessem@wbmt.tudelft.nl})
%    \item[\opt{v0.02 Beta  4}: Typos]
%    Rafael Rodriguez Pappalardo (\texttt{rafapa@cica.es})
%    \item[\opt{v0.02 Beta  5}: Tips/non-SI units]
%    Timothy C. Burt\\ (\texttt{tcburt@comp.uark.edu})
%    \item[\opt{v0.02 Beta  7}: \cmd{\angstrom} definition changed]
%    Hint: Lutz Schwalowsky\\ (\texttt{schalow@mineralogie.uni-hamburg.de});
%    Solution: Piet van Oostrum (\texttt{piet@cs.uu.nl})
%    \item[\opt{v0.04}: \cmd{\ohm} definition corrected]
%    J\"{u}rgen von Haegen (\texttt{vonHagen@engr.psu.edu})
%    \item[\opt{v0.06}: Conflict with \texttt{amssymb} solved] thanks to
%    Timothy C. Burt\\ (\texttt{tcburt@comp.uark.edu})
%    \end{labeling}
%
%    \subsection{References}
%    \begin{enumerate}
%    \item \textsf{National Institute of Standards and Technology Special Publication 330},\ \textit{The International System of Units (SI), 1991 Edition}, by Barry~N.~Taylor, 62 p.:
%    |http://physics.nist.gov/Document/sp330.pdf|
%    \item \textsf{National Institute of Standards and Technology Special Publication 811},\ \textit{Guide for the Use of the International System of Units (SI), 1995 Edition}, by Barry~N.~Taylor, 84 p.:
%    |http://physics.nist.gov/Document/sp811.pdf|
%    \item \textsf{National Institute of Standards and Technology},\ \textit{Diagram of SI unit relationships:}
%    |http://physics.nist.gov/cuu/Units/SIdiagram2.html|
%    \item \textsf{International Bureau of Weights and Measures (\textit{Bureau International des
%   Poids et Mesures})},\ \textit{SI brochure:}
%    |http://www.bipm.fr/pdf/si-brochure.pdf| and \textit{Supplement 2000:}
%    |http://www.bipm.fr/pdf/si-supplement2000.pdf|
%    \item \textsf{National Physical Laboratory},
%    \textit{The International System of Units:} \\ |http://www.npl.co.uk/npl/reference/si_units.html|
%    \item \textsf{National Institute of Standards and Technology}, \\
%    \textit{The NIST reference on Constants, Units and Uncertainty:} \\
%    |http://physics.nist.gov/cuu/Units/introduction.html|
%    \item \textsf{David Barlett},
%    \textit{The Metric System: a concise reference guide:} \\ |http://subnet.virtual-pc.com/ba424872/|
%    \end{enumerate}
%    \StopEventually{}
%
%    \section{The Magic Code}
%    \subsection{Hello world}
%    \iffalse
%<*package>
%    \fi
%    First, we show the package message.
%    \begin{macrocode}
\typeout{\packagemessage}
%    \end{macrocode}
%    \subsubsection{Declare globals}
%    Declare global |\newif|(s) and |\newlength|(s):\\
%    boolean for redefinition of |\square|
%    \begin{macrocode}
\newif\if@redefsquare\@redefsquarefalse
%    \end{macrocode}
%    boolean for definition of |\squaren|
%    \begin{macrocode}
\newif\if@defsquaren\@defsquarenfalse
%    \end{macrocode}
%    boolean for redefinition of |\gray|
%    \begin{macrocode}
\newif\if@redefGray\@redefGrayfalse
%    \end{macrocode}
%    boolean for definition of |\Gray|
%    \begin{macrocode}
\newif\if@defGray\@defGrayfalse
%    \end{macrocode}
%    boolean for detection of textstyle option
%    \begin{macrocode}
\newif\if@textstyle\@textstylefalse
%    \end{macrocode}
%    boolean for detection of binary option
%    \begin{macrocode}
\newif\if@optionbinary\@optionbinaryfalse
%    \end{macrocode}
%    boolean for detection of NoAMS option
%    \begin{macrocode}
\newif\if@optionNoAMS\@optionNoAMSfalse
%    \end{macrocode}
%    boolean for detection of |\unit| command
%    \begin{macrocode}
\newif\if@inunitcommand\@inunitcommandfalse
\newlength{\@qskwidth}
%    \end{macrocode}
%    boolean for detection of italian option
%    \begin{macrocode}
\newif\if@defitalian\@defitalianfalse
%    \end{macrocode}
%    \subsubsection{Font handling}
%    \changes{v0.99}{1999/07/23}{Font handling enhanced.}
%    When using the option \opt{textstyle} units are printed in the typeface of the
%    enclosing text, automatically.
%    \begin{macrocode}
\DeclareRobustCommand\SI@fstyle[1]{\mathrm{#1}}
%    \end{macrocode}
%    \subsubsection{The text sensitive \protect\SImu}
%    Ripped form the |textcomp| package: the text sensitive --- but ugly --- \SImu\textbf{ \SImu}\textsf{\SImu}\textit{\SImu}.
%    \begin{macrocode}
\DeclareTextSymbolDefault{\SImu}{TS1}
\DeclareTextSymbol{\SImu}{TS1}{181} % micro sign
\DeclareFontEncoding{TS1}{}{}
\DeclareFontSubstitution{TS1}{cmr}{m}{n}
%    \end{macrocode}
%
%    \subsubsection{The upright (roman) \ensuremath{\upmu}}
%    The next lines of code are necessary to get an beautifull upright (roman) \(\upmu\) (Greek `em').
%    \begin{macrocode}
\DeclareFontFamily{OML}{eur}{\skewchar\font127} \DeclareFontShape{OML}{eur}{m}{n}{<5> <6>
                <7> <8> <9> gen * eurm <10><10.95><12><14.4><17.28><20.74><24.88>eurm10}{}
\DeclareSymbolFont{greek}{OML}{eur}{m}{n}
%    \end{macrocode}
%    \begin{macro}{\upmu}
%    \begin{macrocode}
\DeclareMathSymbol{\upmu}{\mathord}{greek}{"16}
%    \end{macrocode}
%    \end{macro}
%
%    \begin{macro}{\SIunits}
%    The |\SIunits| macro allows runtime option requests. Every argument of
%    the optional argument list is passed to the macro |\SIunits@execopt|.
%    The options \opt{thickspace \& thickqspace} is selected by default.
%
%    \begin{macrocode}
\newcommand*\SIunits[1][thickspace,thickqspace]{\@for\SIunits@@:=#1%
  \do{\SIunits@execopt\SIunits@@}}
%    \end{macrocode}
%    \end{macro}
%
%    \begin{macro}{\SIunits@execopt}
%    Every execution of this macro with an argument \(n\) leads to the
%    execution of a macro |\SIunits@opt@|\(n\) or a warning if no such exists.
%
%    \begin{macrocode}
\newcommand*\SIunits@execopt[1]{\@ifundefined{SIunits@opt@#1}%
  {\PackageWarning{SIunits}{Requested option `#1' not provided}}%
  {\@nameuse{SIunits@opt@#1}}}
%    \end{macrocode}
%    \end{macro}
%
%    \subsection{Runtime options to use with the \cs{SIunits} command}
%    \subsubsection{\opt{thickspace}}
%    \begin{macro}{\SIunits@opt@thickspace}
%    This macro provides a thick math space (|\;|) between units.
%    \begin{macrocode}
\newcommand*\SIunits@opt@thickspace{%
  \@thickspace{runtime option `thickspace' provided!}}
%    \end{macrocode}
%    \end{macro}
%    \subsubsection{\opt{mediumspace}}
%    \begin{macro}{\SIunits@opt@mediumspace}
%    This macro provides a medium math space (|\:|) between units.
%    \begin{macrocode}
\newcommand*\SIunits@opt@mediumspace{%
  \@mediumspace{runtime option `mediumspace' provided!}}
%    \end{macrocode}
%    \end{macro}
%    \subsubsection{\opt{thinspace}}
%    \begin{macro}{\SIunits@opt@thinspace}
%    This macro provides a thin math space (|\,|) between units.
%    \begin{macrocode}
\newcommand*\SIunits@opt@thinspace{%
  \@thinspace{runtime option `thinspace' provided!}}
%    \end{macrocode}
%    \end{macro}
%    \subsubsection{\opt{cdot}}
%    \begin{macro}{\SIunits@opt@cdot}
%    This macro provides a |\cdot| (\(\cdot\)) between units.
%    \begin{macrocode}
\newcommand*\SIunits@opt@cdot{%
  \@cdot{runtime option `cdot' provided!}}
%    \end{macrocode}
%    \end{macro}
%    \subsubsection{\opt{thickqspace}}
%    \begin{macro}{\SIunits@opt@thickqspace}
%    This macro provides a thick math space (|\;|) between quantities and units.
%    \begin{macrocode}
\newcommand*\SIunits@opt@thickqspace{%
  \@thickqspace{runtime option `thickqspace' provided!}}
%    \end{macrocode}
%    \end{macro}
%    \subsubsection{\opt{mediumqspace}}
%    \begin{macro}{\SIunits@opt@mediumqspace}
%    This macro provides a medium math space (|\:|) between quantities and units.
%   \changes{v0.07 Beta 1}{1999/04/09}{mediumqspace option corrected}
%    \begin{macrocode}
\newcommand*\SIunits@opt@mediumqspace{%
  \@mediumqspace{runtime option `mediumqspace' provided!}}
%    \end{macrocode}
%    \end{macro}
%    \subsubsection{\opt{thinqspace}}
%    \begin{macro}{\SIunits@opt@thinqspace}
%    This macro provides a thin math space (|\;|) between quantities and units.
%    \begin{macrocode}
\newcommand*\SIunits@opt@thinqspace{%
  \@thinqspace{runtime option `thinqspace' provided!}}
%    \end{macrocode}
%    \end{macro}
%    \subsection{text}
%    \begin{macrocode}
\DeclareRobustCommand{\@text}{%
  \ifmmode\expandafter\@text@\else\expandafter\mbox\fi}
\let\nfss@text\@text%
\def\@text@#1{\mathchoice%
  {\textdef@\displaystyle\f@size{#1}}%
  {\textdef@\textstyle\tf@size{\firstchoice@false #1}}%
  {\textdef@\textstyle\sf@size{\firstchoice@false #1}}%
  {\textdef@\textstyle\ssf@size{\firstchoice@false #1}}%
  \check@mathfonts}%
\def\textdef@#1#2#3{\hbox{{%
                    \everymath{#1}%
                    \let\f@size#2\selectfont%
                    #3}}}%
\newif\iffirstchoice@%
\firstchoice@true%
\def\stepcounter#1{%
  \iffirstchoice@%
     \addtocounter{#1}\@ne%
     \begingroup \let\@elt\@stpelt \csname cl@#1\endcsname \endgroup
  \fi%
}%
%    \end{macrocode}
%
%    \subsection{International needs}
%    To prevent international problems, one can use both |\meter| and |\metre| for the SI length
%    unit, and |\deka| and |\dekad| for the SI prefix commands |\deca| and |\decad|.
%    \begin{macro}{\meter}
%    \begin{macrocode}
\DeclareRobustCommand*{\meter}{\metre}
%    \end{macrocode}
%    \end{macro}
%    \begin{macro}{\deka}
%    \begin{macrocode}
\DeclareRobustCommand*{\deka}{\deca}
%    \end{macrocode}
%    \end{macro}
%    \begin{macro}{\dekad}
%    \begin{macrocode}
\DeclareRobustCommand*{\dekad}{\decad}
%    \end{macrocode}
%    \end{macro}
%    \subsection{Personal needs}
%    \begin{macro}{\NoAMS}
%    The |\NoAMS| macro has to be added in the preamble, when you don't have the AMS font |eurm10|
%    \begin{macrocode}
\DeclareRobustCommand*{\NoAMS}{\addprefix{\micro}{\mbox{\SImu}}}
%    \end{macrocode}
%    \end{macro}
%    \changes{v0.03 Beta  2}{1998/10/30}{\cs{NoAMS} command added}
%    \begin{macro}{\addunit}
%    The |\addunit| and |\addprefix| macros give one the possibility to add units and prefixes. This possibility was added after a lot of
%    questions for support of non SI units, that can not be added to this package (it's called \textit{SI}units!). \\Usage:
%    |\addunit{\foot}{ft}|; then the unit can be used: |\unit{1}{\foot}| gives \unit{1}{ft}.
%    \changes{v0.02 Beta  1}{1998/09/09}{\cs{addunit} command added}
%    \begin{macrocode}
\DeclareRobustCommand{\addunit}[2]{\newcommand{#1}{\ensuremath{\SI@fstyle{#2}}}}
\DeclareRobustCommand{\addprefix}[2]{\newcommand{#1}{\ensuremath{\SI@fstyle{#2}}}}
%    \end{macrocode}
%    \end{macro}
%    \begin{macro}{\unitskip}
%    The |\unitskip| macro gives one the possibility to choose spacing characters that
%    are not already defined, by the spacing options (page \pageref{sec:options}). It also
%    gives the possibility to use various spacing character in your documents.
%    \begin{macrocode}
\DeclareRobustCommand*{\unitskip}[1]{\renewcommand{\usk}{\ensuremath{#1}}}
%    \end{macrocode}
%    \end{macro}
%    \begin{macro}{\quantityskip}
%    The |\quantityskip| macro gives one the possibility to choose spacing characters that
%    are not already defined, by the spacing options (page \pageref{sec:options}). It also
%    gives the possibility to use various spacing character in your documents.
%    \begin{macrocode}
\DeclareRobustCommand*{\quantityskip}[1]{\renewcommand{\@qsk}{\ensuremath{#1}}}
%    \end{macrocode}
%    \end{macro}
%    \changes{v0.03 Beta  2}{1998/10/30}{\cs{quantityskip} command added}
%    \subsection{Spacing units}
%    \changes{v0.03 Beta  2}{1998/10/30}{\cs{qsk} command added}
%    In version \fileversion\ of the \pkgname{SIunits}\ package, one has to do the spacing of units by hand. I have plans to get
%    some things automated in a future version. The |\per| macro gives |/| to be used in a quotient of two units; |\usk|
%    (\textbf{u}nit\textbf{sk}ip) makes a thick math space by default, but can be changed by the spacing options
%    (page \pageref{sec:options}) or the |\unitskip| command. Usage:\\ |\metre\per\second| (unit of speed) gives:
%    \(\mathrm{m/s}\) \\ |\newton\usk\metre| (unit of torque) gives: \(\joenit{\mathrm{N}.\mathrm{m}}\).\\
%    |\@qsk| (\textbf{q}uantity \textbf{sk}ip) makes a thick math space
%    by default, but can be changed by the options for spacing between quantity and unit
%    (page \pageref{sec:options}) or the |\quantityskip| command. |\@qsk| is used in the |\unit| macro.
%    \begin{macro}{\per}
%    \begin{macrocode}
\DeclareRobustCommand*{\per}{\ensuremath{\SI@fstyle{/}}}
%    \end{macrocode}
%    \end{macro}
%    \begin{macro}{\usk}
%    \begin{macrocode}
\DeclareRobustCommand*{\usk}{\ensuremath{\;}}
%    \end{macrocode}
%    \end{macro}
%    \begin{macro}{\@qsk}
%    \begin{macrocode}
\DeclareRobustCommand*{\@qsk}{\ensuremath{\;}}
%    \end{macrocode}
%    \end{macro}
%    \subsubsection{(Re)define the spacing commands.}
%    \changes{v1.20}{2001/06/15}{Solved bug: Defining units using \cs{addunit} in combination with the \opt{cdot} and \opt{textstyle} options, by redefining the \cs{cdot} command. Thanks to Michael M\"{u}ller.}
%    \begin{macrocode}
\renewcommand{\cdot}{\,\mbox{\textperiodcentered}\,}
\newcommand{\@cdot}[1]{\DeclareRobustCommand*{\usk}{\ensuremath{\cdot}}\typeout{#1}}
\newcommand{\@thickspace}[1]{\DeclareRobustCommand*{\usk}{\ensuremath{\;}}\typeout{#1}}
\newcommand{\@mediumspace}[1]{\DeclareRobustCommand*{\usk}{\ensuremath{\:}}\typeout{#1}}
\newcommand{\@thinspace}[1]{\DeclareRobustCommand*{\usk}{\ensuremath{\,}}\typeout{#1}}
\newcommand{\@thickqspace}[1]{\DeclareRobustCommand*{\@qsk}{\ensuremath{\;}}\typeout{#1}}
\newcommand{\@mediumqspace}[1]{\DeclareRobustCommand*{\@qsk}{\ensuremath{\:}}\typeout{#1}}
\newcommand{\@thinqspace}[1]{\DeclareRobustCommand*{\@qsk}{\ensuremath{\,}}\typeout{#1}}
%    \end{macrocode}
%
%    \subsection{Spacing between numerical quantities and unit}
%    \begin{macro}{\unit}
%    \changes{v0.99}{2000/02/21}{period in second argument of \cs{unit} automatically spaces the unit using \cs{usk}}
%    The \cmd{\unit} macro is used to typeset conjunction of a numerical quantity and a unit. Usage:
%    |\unit{120}{\kilo\meter\per\hour} = \unit{33.3}{\meter\per\second}| to get:
%    \unit{120}{\kilo\meter\per\hour} = \unit{33.3}{\meter\per\second}.
%    \changes{v0.05 Beta 1}{1999/02/02}{\cs{unit} command changed, thanks to Nancy Winfree}
%    \changes{v0.06 Beta 1}{1999/04/01}{\cs{unit} command changed back to v0.04 version, thanks to J\"{u}rgen von Hagen}
%    \changes{v1.23}{2001/07/21}{\cs{unit} command: parameter \#1 made math by \cs{ensuremath}}
%    \begin{macrocode}
{\catcode`\.=13\gdef.{\usk}}
\newcommand{\period@active}[1]{\begingroup\mathcode`\.="8000\ensuremath{#1}\endgroup}
\DeclareRobustCommand{\unit}[2]{\@inunitcommandtrue%
 \ensuremath{\SI@fstyle{#1\@qsk\period@active{#2}}}%
 \@inunitcommandfalse}
%    \end{macrocode}
%    \end{macro}
%    \changes{v0.03 Beta  3}{1998/11/04}{\cs{unit} command changed}
%    \changes{v0.03 Beta  2}{1998/10/30}{\cs{unit} command added}
%    The |\one| macro is defined to be used for quantities of dimension 1 such as mass fraction. Usage:
%    |\unit{10}{\kilo\gram\per\kilo\gram} = \unit{10}{\one}| to get:
%    \unit{10}{\kilo\gram\per\kilo\gram} = \unit{10}{\one}.
%    \begin{macro}{\one}
%    \begin{macrocode}
\DeclareRobustCommand{\one}{\settowidth{\@qskwidth}{\@qsk}\hspace*{-\@qskwidth}}
%    \end{macrocode}
%    \end{macro}
%    |\no@qsk| is a negative |\hspace| of length |\@qskwidth| if |\if@inunitcommand| is true, else it does nothing.
%    \begin{macro}{\no@qsk}
%    \changes{v1.26}{2001/07/24}{\cs{no@qsk} command changed to get right behaviour with degree, minute, second}
%    \begin{macrocode}
\DeclareRobustCommand{\no@qsk}{%
 \if@inunitcommand%
  \one%
 \else%
  \relax%
 \fi%
}
%    \end{macrocode}
%    \end{macro}
%    \changes{v0.99}{1999/11/08}{\cs{one} command added}
%    \changes{v1.01}{2000/03/16}{exponent of \cs{power} command made textstyle sensitive}
%    \subsection{Power(full) macros}
%    \begin{macro}{\power}
%    The |\power|\footnotemark[1]\footnotetext{Thanks to Werenfried Spit  --- (|w.spit@WITBO.NL|)} macro is used to typeset a superscript. Usage:
%    |\power{\metre}{2}| to get: m\(^{2}\)
%    \begin{macrocode}
\DeclareRobustCommand{\power}[2]{\ensuremath{\SI@fstyle{#1}^{\SI@fstyle{#2}}}}
%    \end{macrocode}
%    \end{macro}
%   \changes{v0.06 Beta 2}{1999/04/06}{amssymb compatibility}
%   |\square| and |\squaren| are defined |\AtBeginDocument| to detect and prevent conflicts with
%   packages defining |\square|.
%    \begin{macro}{\square}
%    \begin{macrocode}
\AtBeginDocument{%
 \if@redefsquare
  \providecommand{\square}[1]{\power{#1}{2}}
  \renewcommand{\square}[1]{\power{#1}{2}}
  \typeout{Option `amssymb' provided! ^^J
  Command \protect\square\space redefined by SIunits package!}
  \typeout{}
 \else
%    \end{macrocode}
%    \end{macro}
%    \begin{macro}{\squaren}
%    \begin{macrocode}
   \if@defsquaren
    \providecommand{\squaren}[1]{\power{#1}{2}}
    \renewcommand{\squaren}[1]{\power{#1}{2}}
    \typeout{Option `squaren' provided! ^^J
    Command \protect\squaren\space defined by SIunits package!}
    \typeout{}
   \else
    \@ifundefined{square}{%
     \newcommand*{\square}[1]{\power{#1}{2}}
     }{%
     \PackageError{SIunits}{%
     The command \protect\square\space was already defined.\MessageBreak
     Possibly due to the amssymb package}%
     {Hint: use option `amssymb' or `squaren' with SIunits package.\MessageBreak
     See SIunits.dvi or readme.txt section: Known problems and limitations.}
    }  %\ifundefined{square}
   \fi %\if@defsquaren
 \fi   %\if@redefsquare
%    \end{macrocode}
%    \end{macro}
%    \begin{macro}{\unita}
%    \begin{macrocode}
 \if@defitalian
 \PackageWarning{SIunits}{Option `italian' provided.\MessageBreak
                          Command \protect\unit\space defined by babel.\MessageBreak
                          Mind to use \protect\unita\space instead.}%
 \DeclareRobustCommand{\unita}[2]{%
           \@inunitcommandtrue%
           \ensuremath{\SI@fstyle{#1\@qsk\period@active{#2}}}%
           \@inunitcommandfalse%
           }%
 \fi%\if@defitalian
}     %\AtBeginDocument
%    \end{macrocode}
%    \changes{v1.30}{2002/08/01}{\cs{unita} added to resolve conflict with \texttt{babel} with \opt{italian} language option}
%    \end{macro}
%   \changes{v1.08}{2000/05/17}{amssymb compatibility improved}
%    \begin{macro}{\SI@square}
%    \changes{v1.29}{2002/07/12}{unwanted space removed (thanks to Svend Tollak Munkejord)}
%    This internal macro is used in the definitions of the ready-to-use units.
%    \begin{macrocode}
\DeclareRobustCommand{\SI@square}[1]
    {\if@defsquaren%
      \squaren{#1}%
        \else
      \square{#1}%
     \fi %\if@defsquaren
    }
%    \end{macrocode}
%    \end{macro}
%    \begin{macro}{\squared}
%\changes{v1.35}{2007/11/24}{Bug fix for textstyle mode}
%    The above example can be realised in a more intuitive way: |\square\metre|: m\(^{2}\).
%    The same goes for |\cubic| \& |\fourth|: m\(^{3}\) \& m\(^{4}\).
%    \begin{macrocode}
\DeclareRobustCommand*{\squared}{\ensuremath{^{\SI@fstyle{2}}}}
%    \end{macrocode}
%    \end{macro}
%    \begin{macro}{\cubic}
%    \begin{macrocode}
\DeclareRobustCommand*{\cubic}[1]{\power{#1}{3}}
%    \end{macrocode}
%    \end{macro}
%    \begin{macro}{\cubed}
%\changes{v1.35}{2007/11/24}{Bug fix for textstyle mode}
%    \begin{macrocode}
\DeclareRobustCommand*{\cubed}{\ensuremath{^{\SI@fstyle{3}}}}
%    \end{macrocode}
%    \end{macro}
%    \begin{macro}{\fourth}
%    \begin{macrocode}
\DeclareRobustCommand*{\fourth}[1]{\power{#1}{4}}
%    \end{macrocode}
%    \end{macro}
%
%\begin{macro}{\SIminus}
%\changes{v1.34}{2007/11/22}{New macro}
%\changes{v1.35}{2007/12/02}{Use real minus sign in text mode}
%\changes{v1.35}{2007/12/02}{Require \texttt{amstext} package}
% To allow text-mode and maths-mode use, the appearance of the minus
% sign needs to be controlled by a macro.
%    \begin{macrocode}
\RequirePackage{amstext}
\DeclareRobustCommand*{\SIminus}{%
  \let\SI@tempa\relax
  \ifmmode
    \edef\SI@tempb{bold}%
    \ifx\math@version\SI@tempb
      \let\SI@tempa\boldmath%
    \fi
  \else
    \if b\expandafter\@car\f@series\@nil
      \let\SI@tempa\bfseries%
    \fi
  \fi
  \text{\SI@tempa$-$}%
}
%    \end{macrocode}
%\end{macro}
%    The macros |\reciprocal|, |\rpsquare|, |\rpsquared|, |\rpcubic|, |\rpcubed| and |\rpfourth| provide the reciprocal (negative power): e.\,g.
%    m\(^{-1}\), m\(^{-2}\), m\(^{-3}\) and m\(^{-4}\). |\rp| is a short form for |\reciprocal|.
%    \begin{macro}{\reciprocal}
%    \begin{macrocode}
\DeclareRobustCommand*{\reciprocal}[1]{\power{#1}{\SIminus1}}
%    \end{macrocode}
%    \end{macro}
%    \begin{macro}{\rp}
%    \begin{macrocode}
\DeclareRobustCommand*{\rp}{\reciprocal}
%    \end{macrocode}
%    \end{macro}
%    \begin{macro}{\rpsquare}
%    \begin{macrocode}
\DeclareRobustCommand*{\rpsquare}[1]{\power{#1}{\SIminus2}}
%    \end{macrocode}
%    \end{macro}
%    \begin{macro}{\rpsquared}
%\changes{v1.35}{2007/11/24}{Bug fix for textstyle mode}
%    \begin{macrocode}
\DeclareRobustCommand*{\rpsquared}{\ensuremath{^{\SI@fstyle{\SIminus2}}}}
%    \end{macrocode}
%    \end%    \begin{macro}{\rpcubic}
%    \begin{macrocode}
\DeclareRobustCommand*{\rpcubic}[1]{\power{#1}{\SIminus3}}
%    \end{macrocode}
%    \end{macro}
%    \begin{macro}{\rpcubed}
%\changes{v1.35}{2007/11/24}{Bug fix for textstyle mode}
%    \begin{macrocode}
\DeclareRobustCommand*{\rpcubed}{\ensuremath{^{\SI@fstyle{\SIminus3}}}}
%    \end{macrocode}
%    \end{macro}
%    \begin{macro}{\rpfourth}
%    \begin{macrocode}
\DeclareRobustCommand*{\rpfourth}[1]{\power{#1}{\SIminus4}}
%    \end{macrocode}
%    \end{macro}
%    \subsection{SI decimal \textit{prefixes}}
%    These prefixes may be used to construct decimal fractions or multiples of units. Two
%    different forms are provided, e.\,g. |\milli| and |\millid|.
%    \subsubsection{Symbols}
%    The first form gives the symbol of the prefix: |\milli\second|: ms;
%    \begin{macrocode}
\addprefix{\yocto}{y}
\addprefix{\zepto}{z}
\addprefix{\atto}{a}
\addprefix{\femto}{f}
\addprefix{\pico}{p}
\addprefix{\nano}{n}
\AtBeginDocument{%
\if@optionNoAMS%
 \addprefix{\micro}{\mbox{\SImu}}%
\else%
  \addprefix{\micro}{\upmu}%
 \fi%
\if@textstyle%
  \DeclareRobustCommand{\micro}{{\ensuremath{\@text{\SImu}}}}%
\fi}
\addprefix{\milli}{m}
\addprefix{\centi}{c}
\addprefix{\deci}{d}
\addprefix{\deca}{da}
\addprefix{\hecto}{h}
\addprefix{\kilo}{k}
\addprefix{\mega}{M}
\addprefix{\giga}{G}
\addprefix{\tera}{T}
\addprefix{\peta}{P}
\addprefix{\exa}{E}
\addprefix{\zetta}{Z}
\addprefix{\yotta}{Y}
%    \end{macrocode}
%    \subsubsection{Decimal form}
%    \begin{macro}{decimals}
%    The other form gives the \textbf{d}ecimal factor: |\kilod\usk\hertz|:
%    \(\joenit{\kilod.\hertz}\)
%    \begin{macrocode}
\DeclareRobustCommand*{\yoctod}{\power{10}{-24}}
\DeclareRobustCommand*{\zeptod}{\power{10}{-21}}
\DeclareRobustCommand*{\attod}{\power{10}{-18}}
\DeclareRobustCommand*{\femtod}{\power{10}{-15}}
\DeclareRobustCommand*{\picod}{\power{10}{-12}}
\DeclareRobustCommand*{\nanod}{\power{10}{-9}}
\DeclareRobustCommand*{\microd}{\power{10}{-6}}
\DeclareRobustCommand*{\millid}{\power{10}{-3}}
\DeclareRobustCommand*{\centid}{\power{10}{-2}}
\DeclareRobustCommand*{\decid}{\power{10}{-1}}
\DeclareRobustCommand*{\decad}{\power{10}{1}}
\DeclareRobustCommand*{\hectod}{\power{10}{2}}
\DeclareRobustCommand*{\kilod}{\power{10}{3}}
\DeclareRobustCommand*{\megad}{\power{10}{6}}
\DeclareRobustCommand*{\gigad}{\power{10}{9}}
\DeclareRobustCommand*{\terad}{\power{10}{12}}
\DeclareRobustCommand*{\petad}{\power{10}{15}}
\DeclareRobustCommand*{\exad}{\power{10}{18}}
\DeclareRobustCommand*{\zettad}{\power{10}{21}}
\DeclareRobustCommand*{\yottad}{\power{10}{24}}
%    \end{macrocode}
%    \subsubsection*{The SI exception}
%    In the SI, \textit{Base} units and \textit{Derived} units do not have
%    prefixes, except for the \textit{base} unit of mass: \textit{kilo}gram, not: gram. However,
%    the macro |\gram| provides the symbol of gram: \gram.
%    \begin{macro}{\gram}
%    \begin{macrocode}
\addunit{\gram}{g}
%    \end{macrocode}
%    \end{macro}
%    \subsection{SI \textit{base} units}
%    \paragraph{length} metre --- \metre \\
%    Both |\metre| and |\meter| can be used.
%    \begin{macro}{\metre}
%    \begin{macrocode}
\addunit{\metre}{m}
%    \end{macrocode}
%    \end{macro}
%    \paragraph{mass} kilogram --- \kilogram
%    \begin{macro}{\kilogram}
%    \begin{macrocode}
\addunit{\kilogram}{\kilo\gram}
%    \end{macrocode}
%    \end{macro}
%    \paragraph{time}  second --- \second
%    \begin{macro}{\second}
%    \begin{macrocode}
\addunit{\second}{s}
%    \end{macrocode}
%    \end{macro}
%    \paragraph{electric current} ampere --- \ampere
%    \begin{macro}{\ampere}
%    \begin{macrocode}
\addunit{\ampere}{A}
%    \end{macrocode}
%    \end{macro}
%    \paragraph{thermodynamic temperature} kelvin --- \kelvin
%    \begin{macro}{\kelvin}
%    \begin{macrocode}
\addunit{\kelvin}{K}
%    \end{macrocode}
%    \end{macro}
%    \paragraph{amount of substance} mole --- \mole
%    \begin{macro}{\mole}
%    \begin{macrocode}
\addunit{\mole}{mol}
%    \end{macrocode}
%    \end{macro}
%    \paragraph{luminous intensity} candela --- \candela
%    \begin{macro}{\candela}
%    \begin{macrocode}
\addunit{\candela}{cd}
%    \end{macrocode}
%    \end{macro}
%
%    \subsection{SI \textit{derived} units}
%    \paragraph{plane angle} radian --- \radian
%    \begin{macro}{\radian}
%    \begin{macrocode}
\addunit{\radian}{rad}
%    \end{macrocode}
%    \end{macro}
%    \paragraph{solid angle} steradian --- \steradian
%    \begin{macro}{\steradian}
%    \begin{macrocode}
\addunit{\steradian}{sr}
%    \end{macrocode}
%    \end{macro}
%    \paragraph{frequency} hertz --- \hertz
%    \begin{macro}{\hertz}
%    \begin{macrocode}
\addunit{\hertz}{Hz}
%    \end{macrocode}
%    \end{macro}
%    \paragraph{force} newton --- \newton
%    \begin{macro}{\newton}
%    \begin{macrocode}
\addunit{\newton}{N}
%    \end{macrocode}
%    \end{macro}
%    \paragraph{pressure} pascal --- \pascal
%    \begin{macro}{\pascal}
%    \begin{macrocode}
\addunit{\pascal}{Pa}
%    \end{macrocode}
%    \end{macro}
%    \paragraph{energy, work, quantity of heat} joule --- \joule
%    \begin{macro}{\joule}
%    \begin{macrocode}
\addunit{\joule}{J}
%    \end{macrocode}
%    \end{macro}
%    \paragraph{power, radiant flux} watt --- \watt
%    \begin{macro}{\watt}
%    \begin{macrocode}
\addunit{\watt}{W}
%    \end{macrocode}
%    \end{macro}
%    \paragraph{electric charge, quantity of electricity} coulomb -- \coulomb
%    \begin{macro}{\coulomb}
%    \begin{macrocode}
\addunit{\coulomb}{C}
%    \end{macrocode}
%    \end{macro}
%    \paragraph{electrical potential, potential difference, electromotive force} volt --- \volt
%    \begin{macro}{\volt}
%    \begin{macrocode}
\addunit{\volt}{V}
%    \end{macrocode}
%    \end{macro}
%    \paragraph{capacitance} farad --- \farad
%    \begin{macro}{\farad}
%    \begin{macrocode}
\addunit{\farad}{F}
%    \end{macrocode}
%    \end{macro}
%    \paragraph{electrical resistance} ohm --- \ohm
%    \begin{macro}{\ohm}
%    \changes{v0.04}{1999/01/22}{\cs{ohm} definition changed (thanks to Juergen von Haegen)}
%    \begin{macrocode}
\addunit{\ohm}{\ensuremath{\Omega}}
%    \end{macrocode}
%    \end{macro}
%    \paragraph{electrical conductance} siemens --- \siemens
%    \begin{macro}{\siemens}
%    \begin{macrocode}
\addunit{\siemens}{S}
%    \end{macrocode}
%    \end{macro}
%    \paragraph{magnetic flux, magnetic field strength} weber --- \weber
%    \begin{macro}{\weber}
%    \begin{macrocode}
\addunit{\weber}{Wb}
%    \end{macrocode}
%    \end{macro}
%    \paragraph{magnetic flux density} tesla --- \tesla
%    \begin{macro}{\tesla}
%    \begin{macrocode}
\addunit{\tesla}{T}
%    \end{macrocode}
%    \end{macro}
%    \paragraph{inductance} henry --- \henry
%    \begin{macro}{\henry}
%    \begin{macrocode}
\addunit{\henry}{H}
%    \end{macrocode}
%    \end{macro}
%    \paragraph{Celsius temperature} degree Celsius --- \degreecelsius \\ both
%    |\degreecelsius| and |\celsius| can be used.
%    \begin{macro}{\degreecelsius}
%    \begin{macrocode}
\newcommand{\degreecelsius}{\protect\@inunitcommandfalse\ensuremath{\SI@fstyle{\degree\Celsius}}}
%    \end{macrocode}
%    \end{macro}
%    \begin{macro}{\celsius}
%    \begin{macrocode}
\addunit{\celsius}{\degreecelsius}
%    \end{macrocode}
%    \end{macro}
%    \paragraph{luminous flux} lumen --- \lumen
%    \begin{macro}{\lumen}
%    \begin{macrocode}
\addunit{\lumen}{lm}
%    \end{macrocode}
%    \end{macro}
%    \paragraph{illuminance} lux --- \lux
%    \begin{macro}{\lux}
%    \changes{v0.03 Beta  1}{1998/10/23}{\cs{lux} unit corrected: lx }
%    \begin{macrocode}
\addunit{\lux}{lx}
%    \end{macrocode}
%    \end{macro}
%    \paragraph{activity of a radionuclide} becquerel --- \becquerel
%    \begin{macro}{\becquerel}
%    \begin{macrocode}
\addunit{\becquerel}{Bq}
%    \end{macrocode}
%    \end{macro}
%    \paragraph{absorbed dose, specific energy imparted, kerma} gray --- \gray \\
%    |\gray| is defined |\AtBeginDocument|.
%    \begin{macro}{\gray}
%    \begin{macrocode}
\AtBeginDocument{%
\if@redefGray
  \providecommand{\gray}{\ensuremath{\SI@fstyle{Gy}}}
  \renewcommand{\gray}{\ensuremath{\SI@fstyle{Gy}}}
  \typeout{Option `pstricks' provided! ^^J
           Command \protect\gray\space redefined by SIunits package!}
  \typeout{}
 \else
   \if@defGray
    \providecommand{\Gray}{\ensuremath{\SI@fstyle{Gy}}}
    \renewcommand{\Gray}{\ensuremath{\SI@fstyle{Gy}}}
    \typeout{Option `Gray' provided! ^^J
             Command \protect\Gray\space defined by SIunits package!}
    \typeout{}
   \else
    \@ifundefined{gray}{%
     \newcommand*{\gray}{\ensuremath{\SI@fstyle{Gy}}}
     }{%
     \PackageWarningNoLine{SIunits}{%
     The command \protect\gray\space was already defined.\MessageBreak
     Possibly due to the pstricks package}
     \typeout{Hint: use option `pstricks' or `Gray' with SIunits package.}
     \typeout{See SIunits.dvi or readme.txt section: Known problems and limitations.}
     \typeout{}
    }          %\ifundefined{gray}
   \fi %\if@defGray
 \fi   %\if@redefGray
}      %\AtBeginDocument
%    \changes{v0.99}{1999/09/06}{Conflict between pstricks and \cs{gray} solved}
%    \end{macrocode}
%    \end{macro}
%    \paragraph{dose equivalent} sievert --- \sievert
%    \begin{macro}{\sievert}
%    \begin{macrocode}
\addunit{\sievert}{Sv}
%    \end{macrocode}
%    \end{macro}
%    \paragraph{catalytic activity} katal --- \katal
%    \begin{macro}{\katal}
%    \changes{v1.13}{2000/08/29}{unit katal with symbol \katal\ implemented}
%    \begin{macrocode}
\addunit{\katal}{kat}
%    \end{macrocode}
%    \end{macro}

%    \subsubsection{The \opt{derivedinbase} mode}
%    \subsection*{Expression of derived SI units in SI base units}
%    \begin{macro}{\SIunits@opt@derivedinbase}
%    This macro provides the expression of derived SI units in SI base units.
%    These macros can be accessed by putting `base' behind the SI derived unit command,
%    e.\,g. (|\pascalbase|) to get `\pascalbase'.
%
%    \begin{macrocode}
\newcommand*\SIunits@opt@derivedinbase{%
\typeout{Option 'derivedinbase' provided!^^J}
\addunit{\radianbase}%
        {\metre\usk\reciprocal\metre}
\addunit{\steradianbase}%
        {\squaremetre\usk\rpsquare\metre}
\addunit{\hertzbase}%
        {\reciprocal\second}
\addunit{\newtonbase}%
        {\metre\usk\kilogram\usk\second\rpsquared}
\addunit{\pascalbase}%
        {\reciprocal\metre\usk\kilogram\usk\second\rpsquared}
\addunit{\joulebase}%
        {\squaremetre\usk\kilogram\usk\second\rpsquared}
\addunit{\wattbase}%
        {\squaremetre\usk\kilogram\usk\rpcubic\second}
\addunit{\coulombbase}%
        {\ampere\usk\second}
\addunit{\voltbase}%
        {\squaremetre\usk\kilogram\usk\rpcubic\second\usk\reciprocal\ampere}
\addunit{\faradbase}%
        {\rpsquare\metre\usk\reciprocal\kilogram\usk\fourth\second\usk\ampere\squared}
\addunit{\ohmbase}%
        {\squaremetre\usk\kilogram\usk\rpcubic\second\usk\rpsquare\ampere}
\addunit{\siemensbase}%
        {\rpsquare\metre\usk\reciprocal\kilogram\usk\cubic\second\usk\ampere\squared}
\addunit{\weberbase}%
        {\squaremetre\usk\kilogram\usk\second\rpsquared\usk\reciprocal\ampere}
\addunit{\teslabase}%
        {\kilogram\usk\second\rpsquared\usk\reciprocal\ampere}
\addunit{\henrybase}%
        {\squaremetre\usk\kilogram\usk\second\rpsquared\usk\rpsquare\ampere}
\addunit{\celsiusbase}%
        {\kelvin}
\addunit{\lumenbase}%
        {\candela\usk\squaremetre\usk\rpsquare\metre}
\addunit{\luxbase}%
        {\candela\usk\squaremetre\usk\rpfourth\metre}
\addunit{\becquerelbase}%
        {\hertzbase}
\addunit{\graybase}%
        {\squaremetre\usk\second\rpsquared}
\addunit{\sievertbase}%
        {\graybase}
\addunit{\katalbase}%
        {\rp\second\usk\mole }
}
%    \end{macrocode}
%    \end{macro}
%    \subsubsection{The \opt{derived} mode}
%    \subsection*{Expression of derived SI units in other derived SI units}
%    \begin{macro}{\SIunits@opt@derived}
%    This macro provides the expression of derived SI units in other SI derived units (if possible).
%    These macros can be accessed by putting `der' in front of the SI derived unit command,
%    e.\,g. (|\derpascal|) to get `\derpascal'.
%
%    \begin{macrocode}
\newcommand*\SIunits@opt@derived{%
\addunit{\derradian}%
        {\metre\usk\reciprocal\metre}
\addunit{\dersteradian}%
        {\squaremetre\usk\rpsquare\metre}
\addunit{\derhertz}%
        {\reciprocal\second}
\addunit{\dernewton}%
        {\metre\usk\kilogram\usk\second\rpsquared}
\addunit{\derpascal}%
        {\newton\usk\rpsquare\metre}
\addunit{\derjoule}%
        {\newton\usk\metre}
\addunit{\derwatt}%
        {\joule\usk\reciprocal\second}
\addunit{\dercoulomb}%
        {\ampere\usk\second}
\addunit{\dervolt}%
        {\watt\usk\reciprocal\ampere}
\addunit{\derfarad}%
        {\coulomb\usk\reciprocal\volt}
\addunit{\derohm}%
        {\volt\usk\reciprocal\ampere}
\addunit{\dersiemens}%
        {\ampere\usk\reciprocal\volt}
\addunit{\derweber}%
        {\squaremetre\usk\kilogram\usk\second\rpsquared\usk\reciprocal\ampere}
\addunit{\dertesla}%
        {\weber\usk\rpsquare\metre}
\addunit{\derhenry}%
        {\weber\usk\reciprocal\ampere}
\addunit{\dercelsius}%
        {\kelvin}
\addunit{\derlumen}%
        {\candela\usk\steradian}
\addunit{\derlux}%
        {\lumen\usk\rpsquare\metre}
\addunit{\derbecquerel}%
        {\derhertz}
\addunit{\dergray}%
        {\joule\usk\reciprocal\kilogram}
\addunit{\dersievert}%
        {\dergray}
\addunit{\derkatal}%
        {\katalbase}
     \typeout{Option `derived' provided!}}
%    \end{macrocode}
%    \end{macro}
%    \subsection{Units that are used with the SI}
%    \paragraph{Time} minute --- \minute; hour --- \hour; day --- \dday \\ |\day| was already defined, so use |\dday|.
%    \begin{macro}{\minute}
%    \begin{macrocode}
\addunit{\minute}{min}
%    \end{macrocode}
%    \end{macro}
%    \begin{macro}{\hour}
%    \begin{macrocode}
\addunit{\hour}{h}
%    \end{macrocode}
%    \end{macro}
%    \begin{macro}{\dday}
%    \begin{macrocode}
\addunit{\dday}{d}
%    \end{macrocode}
%    \end{macro}
%    \paragraph{Plane angle} degree --- \degree; minute --- \paminute; second --- \arcsecond \\ |\minute| and |\second| were already defined.
%    \begin{macro}{\degree}
%    \changes{v0.99}{1999/11/05}{\cs{arcsecond} and \cs{arcminute} added.}
%    \begin{macrocode}
\addunit{\degree}{\no@qsk\ensuremath{^{\circ}}}
%    \end{macrocode}
%    \end{macro}
%    \begin{macro}{\paminute}
%    \begin{macrocode}
\addunit{\paminute}{\no@qsk\ensuremath{'}}
%    \end{macrocode}
%    \end{macro}
%    \begin{macro}{\arcminute}
%    \begin{macrocode}
\addunit{\arcminute}{\no@qsk\ensuremath{'}}
%    \end{macrocode}
%    \end{macro}
%    \begin{macro}{\pasecond}
%    \begin{macrocode}
\addunit{\pasecond}{\no@qsk\ensuremath{''}}
%    \end{macrocode}
%    \end{macro}
%    \begin{macro}{\arcsecond}
%    \begin{macrocode}
\addunit{\arcsecond}{\no@qsk\ensuremath{''}}
%    \end{macrocode}
%    \end{macro}
%    \paragraph{Mass} metric ton or tonne --- \ton
%    \begin{macro}{\ton}
%    \begin{macrocode}
\addunit{\ton}{t}
%    \end{macrocode}
%    \end{macro}
%    \begin{macro}{\tonne}
%    \begin{macrocode}
\addunit{\tonne}{t}
%    \end{macrocode}
%    \end{macro}
%    \paragraph{Volume} litre --- \litre; liter --- \liter
%    \begin{macro}{\liter}
%    \begin{macrocode}
\addunit{\liter}{L}
%    \end{macrocode}
%    \end{macro}
%    \begin{macro}{\litre}
%    \begin{macrocode}
\addunit{\litre}{l}
%    \end{macrocode}
%    \end{macro}
%    \begin{macro}{\neper}
%    \begin{macrocode}
\addunit{\neper}{Np}
%    \end{macrocode}
%    \end{macro}
%    \begin{macro}{\bel}
%    \begin{macrocode}
\addunit{\bel}{B}
%    \end{macrocode}
%    \end{macro}
%    \paragraph{Radioactivity} curie --- \curie
%    \begin{macro}{\curie}
%    \begin{macrocode}
\addunit{\curie}{Ci}
%    \end{macrocode}
%    \end{macro}
%    \paragraph{Absorbed dose} rad --- \rad \\ When there is risk of confusion with the symbol for radian (\radian), \arad\ may be used as the symbol for rad.
%    \begin{macro}{\rad}
%    \begin{macrocode}
\addunit{\rad}{rad}
%    \end{macrocode}
%    \end{macro}
%    \begin{macro}{\arad}
%    \begin{macrocode}
\addunit{\arad}{rd}
%    \end{macrocode}
%    \end{macro}
%    \paragraph{Dose equivalent} rem --- \rem
%    \begin{macro}{\rem}
%    \begin{macrocode}
\addunit{\rem}{rem}
%    \end{macrocode}
%    \end{macro}
%    \paragraph{Exposure roentgen} roentgen--- \roentgen
%    \begin{macro}{\roentgen}
%    \begin{macrocode}
\addunit{\roentgen}{R}
%    \end{macrocode}
%    \end{macro}
%    \paragraph{Energy} electronvolt --- \electronvolt
%    \begin{macro}{\electronvolt}
%\changes{v1.35}{2007/11/23}{Improved appearance}
%    \begin{macrocode}
\addunit{\electronvolt}{e\kern-0.05ex\volt}
%    \end{macrocode}
%    \end{macro}
%    \paragraph{Unified atomic mass unit} atomic mass --- \atomicmass
%    \begin{macro}{\atomicmass}
%    \begin{macro}{\atomicmassunit}
%\changes{v1.35}{2007/11/23}{New unit}
%    \begin{macro}{\dalton}
%\changes{v1.35}{2007/11/23}{New unit}
%    \begin{macrocode}
\addunit{\atomicmass}{u}
\addunit{\atomicmassunit}{u}
\addunit{\dalton}{Da}
%    \end{macrocode}
%    \end{macro}
%    \end{macro}
%    \end{macro}
%    \paragraph{Area} are --- \are; hectare --- \hectare; barn --- \barn
%    \begin{macro}{\are}
%    \begin{macrocode}
\addunit{\are}{a}
%    \end{macrocode}
%    \end{macro}
%    \begin{macro}{\hectare}
%    \begin{macrocode}
\addunit{\hectare}{\hecto\are}
%    \end{macrocode}
%    \end{macro}
%    \begin{macro}{\barn}
%    \begin{macrocode}
\addunit{\barn}{b}
%    \end{macrocode}
%    \end{macro}
%    \paragraph{Pressure} bar --- \bbar
%    \begin{macro}{\bbar}
%    \begin{macrocode}
\addunit{\bbar}{bar}
%    \end{macrocode}
%    \end{macro}
%    \paragraph{Acceleration} gal --- \gal
%    \begin{macro}{\gal}
%    \begin{macrocode}
\addunit{\gal}{Gal}
%    \end{macrocode}
%    \end{macro}
%    \paragraph{Length} \aa ngstr\"{o}m --- \angstrom
%    \changes{v0.02 Beta  7}{1998/10/01}{\cs{angstrom} changed; thanks to Lutz Schwalowsky}
%    \begin{macro}{\angstrom}
%    \begin{macrocode}
\addunit{\angstrom}{\mbox{{\AA}}}
%    \end{macrocode}
%    \end{macro}
%    \paragraph{Rotational frequency} revolutions per minute --- \rperminute; revolutions per second --- \rpersecond
%    \begin{macro}{\rperminute}
%    \begin{macrocode}
\addunit{\rperminute}{r\per\minute}
%    \end{macrocode}
%    \end{macro}
%    \begin{macro}{\rpersecond}
%    \begin{macrocode}
\addunit{\rpersecond}{r\per\second}
%    \end{macrocode}
%    \end{macro}
%
%    \subsection{SI units with compound names}
%    \paragraph{Area} square metre --- \squaremetre
%    \begin{macro}{\squaremetre}
%    \begin{macrocode}
\addunit{\squaremetre}{\SI@square\metre}
%    \end{macrocode}
%    \end{macro}
%    \paragraph{Volume} cubic metre --- \cubicmetre
%    \begin{macro}{\cubicmetre}
%    \begin{macrocode}
\addunit{\cubicmetre}{\cubic\metre}
%    \end{macrocode}
%    \end{macro}
%    \subsection{Various ready-to-use units}
%    These units are provided for the ease of the users of the \pkgname{SIunits}\ package. Normally, two
%    forms of the units are provided, e.\,g. |\graypersecond| and |\graypersecondnp|. The
%    |command|\textsf{np} form uses negative powers instead of /: `\graypersecond' and
%    `\graypersecondnp'.\\
%    \changes{v0.02 Beta  3}{1998/09/11}{\cs{pH} command removed (not in SI)}
%    \paragraph{absorbed dose rate}
%    \begin{macrocode}
\addunit{\graypersecond}{\gray\per\second}
\addunit{\graypersecondnp}{\gray\usk\reciprocal\second}
%    \end{macrocode}
%    \end{macro}
%    \paragraph{acceleration}
%    \begin{macrocode}
\addunit{\metrepersquaresecond}{\metre\per\second\squared}
\addunit{\metrepersquaresecondnp}{\metre\usk\second\rpsquared}
%    \end{macrocode}
%    \paragraph{activation energy, molar energy}
%    \begin{macrocode}
\addunit{\joulepermole}{\joule\per\mole}
\addunit{\joulepermolenp}{\joule\usk\reciprocal\mole}
%    \end{macrocode}
%    \paragraph{amount-of-substance concentration}
%    \begin{macrocode}
\addunit{\molepercubicmetre}{\mole\per\cubic\metre}
\addunit{\molepercubicmetrenp}{\mole\usk\rpcubic\metre}
%    \end{macrocode}
%    \paragraph{angular acceleration}
%    \begin{macrocode}
\addunit{\radianpersquaresecond}{\radian\per\second\squared}
\addunit{\radianpersquaresecondnp}{\radian\usk\second\rpsquared}
%    \end{macrocode}
%    \paragraph{angular momentum}
%    \begin{macrocode}
\addunit{\kilogramsquaremetrepersecond}{\kilogram\usk\squaremetre\per\second}
\addunit{\kilogramsquaremetrepersecondnp}{\kilogram\usk\squaremetre\usk\reciprocal\second}
%    \end{macrocode}
%    \paragraph{angular velocity}
%    \begin{macrocode}
\addunit{\radianpersecond}{\radian\per\second}
\addunit{\radianpersecondnp}{\radian\usk\reciprocal\second}
%    \end{macrocode}
%    \paragraph{area per unit volume}
%    \begin{macrocode}
\addunit{\squaremetrepercubicmetre}{\squaremetre\per\cubic\metre}
\addunit{\squaremetrepercubicmetrenp}{\squaremetre\usk\rpcubic\metre}
%    \end{macrocode}
%    \paragraph{catalytic (activity) concentration}
%    \changes{v1.13}{2000/08/29}{catalytic concentration added}
%    \begin{macrocode}
\addunit{\katalpercubicmetre}{\katal\per\cubic\metre}
\addunit{\katalpercubicmetrenp}{\katal\usk\rpcubic\metre}
%    \end{macrocode}
%    \paragraph{charge per mole}
%    \begin{macrocode}
\addunit{\coulombpermol}{\coulomb\per\mole}
\addunit{\coulombpermolnp}{\coulomb\usk\reciprocal\mole}
%    \end{macrocode}
%    \paragraph{current density}
%    \begin{macrocode}
\addunit{\amperepersquaremetre}{\ampere\per\squaremetre}
\addunit{\amperepersquaremetrenp}{\ampere\usk\rpsquare\metre}
%    \end{macrocode}
%    \paragraph{density}
%    \begin{macrocode}
\addunit{\kilogrampercubicmetre}{\kilogram\per\cubic\metre}
\addunit{\kilogrampercubicmetrenp}{\kilogram\usk\rpcubic\metre}
%    \end{macrocode}
%    \paragraph{dynamic fluidity (1/viscosity)}
%    \begin{macrocode}
\addunit{\squaremetrepernewtonsecond}{\squaremetre\per\newton\usk\second}
\addunit{\squaremetrepernewtonsecondnp}{\squaremetre\usk\reciprocal\newton\usk\reciprocal\second}
%    \end{macrocode}
%    \paragraph{dynamic viscosity}
%    \begin{macrocode}
\addunit{\pascalsecond}{\pascal\usk\second}
%    \end{macrocode}
%    \paragraph{electric charge density}
%    \begin{macrocode}
\addunit{\coulombpercubicmetre}{\coulomb\per\cubic\metre}
\addunit{\coulombpercubicmetrenp}{\coulomb\usk\rpcubic\metre}
%    \end{macrocode}
%    \paragraph{electric dipole moment}
%    \begin{macrocode}
\addunit{\amperemetresecond}{\ampere\usk\metre\usk\second}
%    \end{macrocode}
%    \paragraph{electric field strength}
%    \begin{macrocode}
\addunit{\voltpermetre}{\volt\per\metre}
\addunit{\voltpermetrenp}{\volt\usk\reciprocal\metre}
%    \end{macrocode}
%    \paragraph{electric flux density}
%    \begin{macrocode}
\addunit{\coulombpersquaremetre}{\coulomb\per\squaremetre}
\addunit{\coulombpersquaremetrenp}{\coulomb\usk\rpsquare\metre}
%    \end{macrocode}
%    \paragraph{electrical permittivity}
%    \begin{macrocode}
\addunit{\faradpermetre}{\farad\per\metre}
\addunit{\faradpermetrenp}{\farad\usk\reciprocal\metre}
%    \end{macrocode}
%    \paragraph{electrical resistivity}
%    \begin{macrocode}
\addunit{\ohmmetre}{\ohm\usk\metre}
%    \end{macrocode}
%    \paragraph{energy}
%    \begin{macrocode}
\addunit{\kilowatthour}{\kilo\watt\hour}
%    \end{macrocode}
%    \paragraph{energy flux}
%    \begin{macrocode}
\addunit{\wattpersquaremetre}{\watt\per\squaremetre}
\addunit{\wattpersquaremetrenp}{\watt\usk\rpsquare\metre}
%    \end{macrocode}
%    \paragraph{energy per unit area}
%    \begin{macrocode}
\addunit{\joulepersquaremetre}{\joule\per\squaremetre}
\addunit{\joulepersquaremetrenp}{\joule\usk\rpsquare\metre}
%    \end{macrocode}
%    \paragraph{force (body)}
%    \begin{macrocode}
\addunit{\newtonpercubicmetre}{\newton\per\cubic\metre}
\addunit{\newtonpercubicmetrenp}{\newton\usk\rpcubic\metre}
%    \end{macrocode}
%    \paragraph{force per unit mass}
%    \begin{macrocode}
\addunit{\newtonperkilogram}{\newton\per\kilogram}
\addunit{\newtonperkilogramnp}{\newton\usk\reciprocal\kilogram}
%    \end{macrocode}
%    \paragraph{heat capacity, entropy}
%    \begin{macrocode}
\addunit{\jouleperkelvin}{\joule\per\kelvin}
\addunit{\jouleperkelvinnp}{\joule\usk\reciprocal\kelvin}
%    \end{macrocode}
%    \paragraph{heat of combustion, fusion or vaporisation}
%    \begin{macrocode}
\addunit{\jouleperkilogram}{\joule\per\kilogram}
\addunit{\jouleperkilogramnp}{\joule\usk\reciprocal\kilogram}
%    \end{macrocode}
%    \paragraph{intensity of ionising radiation}
%    \begin{macrocode}
\addunit{\coulombperkilogram}{\coulomb\per\kilogram}
\addunit{\coulombperkilogramnp}{\coulomb\usk\reciprocal\kilogram}
%    \end{macrocode}
%    \paragraph{kinematic viscosity}
%    \begin{macrocode}
\addunit{\squaremetrepersecond}{\squaremetre\per\second}
\addunit{\squaremetrepersecondnp}{\squaremetre\usk\reciprocal\second}
%    \end{macrocode}
%    \paragraph{kinematic energy of turbulence}
%    \begin{macrocode}
\addunit{\squaremetrepersquaresecond}{\squaremetre\per\second\squared}
\addunit{\squaremetrepersquaresecondnp}{\squaremetre\usk\second\rpsquared}
%    \end{macrocode}
%    \paragraph{linear momentum}
%    \begin{macrocode}
\addunit{\kilogrammetrepersecond}{\kilogram\usk\metre\per\second}
\addunit{\kilogrammetrepersecondnp}{\kilogram\usk\metre\usk\reciprocal\second}
%    \end{macrocode}
%    \paragraph{luminance}
%    \begin{macrocode}
\addunit{\candelapersquaremetre}{\candela\per\squaremetre}
\addunit{\candelapersquaremetrenp}{\candela\usk\rpsquare\metre}
%    \end{macrocode}
%    \paragraph{magnetic field strength}
%    \begin{macrocode}
\addunit{\amperepermetre}{\ampere\per\metre}
\addunit{\amperepermetrenp}{\ampere\usk\reciprocal\metre}
%    \end{macrocode}
%    \paragraph{magnetic moment}
%    \begin{macrocode}
\addunit{\joulepertesla}{\joule\per\tesla}
\addunit{\jouleperteslanp}{\joule\usk\reciprocal\tesla}
%    \end{macrocode}
%    \paragraph{magnetic permeability}
%    \begin{macrocode}
\addunit{\henrypermetre}{\henry\per\metre}
\addunit{\henrypermetrenp}{\henry\usk\reciprocal\metre}
%    \end{macrocode}
%    \paragraph{mass flow rate}
%    \begin{macrocode}
\addunit{\kilogrampersecond}{\kilogram\per\second}
\addunit{\kilogrampersecondnp}{\kilogram\usk\reciprocal\second}
%    \end{macrocode}
%    \paragraph{mass flux}
%    \begin{macrocode}
\addunit{\kilogrampersquaremetresecond}{\kilogram\per\squaremetre\usk\second}
\addunit{\kilogrampersquaremetresecondnp}{\kilogram\usk\rpsquare\metre\usk\reciprocal\second}
%    \end{macrocode}
%    \paragraph{mass per unit area}
%    \begin{macrocode}
\addunit{\kilogrampersquaremetre}{\kilogram\per\squaremetre}
\addunit{\kilogrampersquaremetrenp}{\kilogram\usk\rpsquare\metre}
%    \end{macrocode}
%    \paragraph{mass per unit length}
%    \begin{macrocode}
\addunit{\kilogrampermetre}{\kilogram\per\metre}
\addunit{\kilogrampermetrenp}{\kilogram\usk\reciprocal\metre}
%    \end{macrocode}
%    \paragraph{molar heat capacity, molar entropy}
%    \begin{macrocode}
\addunit{\joulepermolekelvin}{\joule\per\mole\usk\kelvin}
\addunit{\joulepermolekelvinnp}{\joule\usk\reciprocal\mole\usk\reciprocal\kelvin}
%    \end{macrocode}
%    \paragraph{molecular weight}
%    \begin{macrocode}
\addunit{\kilogramperkilomole}{\kilogram\per\kilo\mole}
\addunit{\kilogramperkilomolenp}{\kilogram\usk\kilo\reciprocal\mole}
%    \end{macrocode}
%    \paragraph{moment of inertia}
%    \begin{macrocode}
\addunit{\kilogramsquaremetre}{\kilogram\usk\squaremetre}
\addunit{\kilogramsquaremetrenp}{\kilogramsquaremetre}
%    \end{macrocode}
%    \paragraph{momentum flow rate}
%    \begin{macrocode}
\addunit{\kilogrammetrepersquaresecond}{\kilogram\usk\metre\per\second\squared}
\addunit{\kilogrammetrepersquaresecondnp}{\kilogram\usk\metre\usk\second\rpsquared}
%    \end{macrocode}
%    \paragraph{momentum flux}
%    \begin{macrocode}
\addunit{\newtonpersquaremetre}{\newton\per\squaremetre}
\addunit{\newtonpersquaremetrenp}{\newton\usk\rpsquare\metre}
%    \end{macrocode}
%    \paragraph{photon emission rate}
%    \begin{macrocode}
\addunit{\persquaremetresecond}{1\per\squaremetre\usk\second}
\addunit{\persquaremetresecondnp}{\rpsquare\metre\usk\reciprocal\second}
%    \end{macrocode}
%    \paragraph{power per unit mass}
%    \begin{macrocode}
\addunit{\wattperkilogram}{\watt\per\kilogram}
\addunit{\wattperkilogramnp}{\watt\usk\reciprocal\kilogram}
%    \end{macrocode}
%    \paragraph{power per unit volume}
%    \begin{macrocode}
\addunit{\wattpercubicmetre}{\watt\per\cubic\metre}
\addunit{\wattpercubicmetrenp}{\watt\usk\rpcubic\metre}
%    \end{macrocode}
%    \paragraph{radiance}
%    \begin{macrocode}
\addunit{\wattpersquaremetresteradian}{\watt\per\squaremetre\usk\steradian}
\addunit{\wattpersquaremetresteradiannp}{\watt\usk\rpsquare\metre\usk\rp\steradian}
%    \end{macrocode}
%    \paragraph{specific heat capacity}
%    \begin{macrocode}
\addunit{\jouleperkilogramkelvin}{\joule\per\kilogram\usk\kelvin}
\addunit{\jouleperkilogramkelvinnp}{\joule\usk\reciprocal\kilogram\usk\reciprocal\kelvin}
%    \end{macrocode}
%    \paragraph{specific surface}
%    \begin{macrocode}
\addunit{\squaremetreperkilogram}{\squaremetre\per\kilogram}
\addunit{\rpsquaremetreperkilogram}{\squaremetre\usk\reciprocal\kilogram}
%    \end{macrocode}
%    \paragraph{specific volume}
%    \begin{macrocode}
\addunit{\cubicmetreperkilogram}{\cubic\metre\per\kilogram}
\addunit{\rpcubicmetreperkilogram}{\cubic\metre\usk\reciprocal\kilogram}
%    \end{macrocode}
%    \paragraph{surface tension}
%    \begin{macrocode}
\addunit{\newtonpermetre}{\newton\per\metre}
\addunit{\newtonpermetrenp}{\newton\usk\reciprocal\metre}
%    \end{macrocode}
%    \paragraph{derived SI unit: \degreecelsius}
%    \begin{macrocode}
\addunit{\Celsius}{\ensuremath{\SI@fstyle{C}}}
%    \end{macrocode}
%    \paragraph{thermal conductivity}
%    \begin{macrocode}
\addunit{\wattpermetrekelvin}{\watt\per\metre\usk\kelvin}
\addunit{\wattpermetrekelvinnp}{\watt\usk\reciprocal\metre\usk\reciprocal\kelvin}
%    \end{macrocode}
%    \paragraph{torque}
%    \begin{macrocode}
\addunit{\newtonmetre}{\newton\usk\metre} \addunit{\newtonmetrenp}{\newtonmetre}
%    \end{macrocode}
%    \paragraph{turbulence energy dissipation rate}
%    \begin{macrocode}
\addunit{\squaremetrepercubicsecond}{\squaremetre\per\cubic\second}
\addunit{\squaremetrepercubicsecondnp}{\squaremetre\usk\rpcubic\second}
%    \end{macrocode}
%    \paragraph{velocity}
%    \begin{macrocode}
\addunit{\metrepersecond}{\metre\per\second}
\addunit{\metrepersecondnp}{\metre\usk\reciprocal\second}
%    \end{macrocode}
%    \paragraph{volumetric calorific value}
%    \begin{macrocode}
\addunit{\joulepercubicmetre}{\joule\per\cubicmetre}
\addunit{\joulepercubicmetrenp}{\joule\usk\rpcubic\metre}
%    \end{macrocode}
%    \paragraph{volumetric coefficient of expansion}
%    \begin{macrocode}
\addunit{\kilogrampercubicmetrecoulomb}{\kilogram\per\cubic\metre\usk\coulomb}
\addunit{\kilogrampercubicmetrecoulombnp}{\kilogram\usk\rpcubic\metre\usk\reciprocal\coulomb}
%    \end{macrocode}
%    \paragraph{volumetric flow rate}
%    \begin{macrocode}
\addunit{\cubicmetrepersecond}{\cubicmetre\per\second}
\addunit{\rpcubicmetrepersecond}{\cubicmetre\usk\reciprocal\second}
%    \end{macrocode}
%    \paragraph{volumetric mass flow rate}
%    \begin{macrocode}
\addunit{\kilogrampersecondcubicmetre}{\kilogram\per\second\usk\cubicmetre}
\addunit{\kilogrampersecondcubicmetrenp}{\kilogram\usk\reciprocal\second\usk\rpcubic\metre}
%    \end{macrocode}

%  \subsection{Option handling}
%  \changes{v0.99}{1999/07/23}{LaTeX2e option handling implemented.}
%    \DescribeMacro{Options}
%  \subsubsection{\opt{cdot} option}
%    \begin{macrocode}
\DeclareOption{cdot}{\@cdot{Option `cdot' provided!}}
%    \end{macrocode}
%  \subsubsection{\opt{thickspace} option}
%    \begin{macrocode}
\DeclareOption{thickspace}{\@thickspace{Option `thickspace' provided!}}
%    \end{macrocode}
%  \subsubsection{\opt{mediumspace} option}
%    \begin{macrocode}
\DeclareOption{mediumspace}{\@mediumspace{Option `mediumspace' provided!}}
%    \end{macrocode}
%  \subsubsection{\opt{thinspace} option}
%    \begin{macrocode}
\DeclareOption{thinspace}{\@thinspace{Option `thinspace' provided!}}
%    \end{macrocode}
%  \subsubsection{\opt{thickqspace} option}
%    \begin{macrocode}
\DeclareOption{thickqspace}{\@thickqspace{Option `thickqspace' provided!}}
%    \end{macrocode}
%  \subsubsection{\opt{mediumqspace} option}
%    \begin{macrocode}
\DeclareOption{mediumqspace}{\@mediumqspace{Option `mediumqspace' provided!}}
%    \end{macrocode}
%  \subsubsection{\opt{thinqspace} option}
%    \begin{macrocode}
\DeclareOption{thinqspace}{\@thinqspace{Option `thinqspace' provided!}}
%    \end{macrocode}
%   \subsubsection{\opt{textstyle} option}
%   Typeset units in text style.
%    \begin{macrocode}
\DeclareOption{textstyle}{\renewcommand\SI@fstyle[1]{\@text{\protect#1}}%
\@textstyletrue%
\typeout{Option `textstyle' provided!}}
%    \end{macrocode}
%    \subsection{compatibility options}
%    \begin{macrocode}
\DeclareOption{amssymb}{\@redefsquaretrue%
\typeout{Option `amssymb' provided!}}
\DeclareOption{squaren}{\@defsquarentrue%
\typeout{Option `squaren' provided!}}
\DeclareOption{pstricks}{\@redefGraytrue%
\typeout{Option `pstricks' provided!}}
\DeclareOption{Gray}{\@defGraytrue%
\typeout{Option `Gray' provided!}}
\DeclareOption{italian}{\@defitaliantrue%
\typeout{Option `italian' provided!}}
%    \end{macrocode}
%    \changes{v1.30}{2002/08/01}{option \opt{italian} added}
%    \subsection{Miscellaneous options}
%    \begin{macrocode}
\DeclareOption{binary}{\@optionbinarytrue }
\AtEndOfPackage{\if@optionbinary\RequirePackage{binary}\fi}
\DeclareOption{derivedinbase}{\SIunits@opt@derivedinbase}
\DeclareOption{derived}{\SIunits@opt@derived}
\DeclareOption{noams}{\@optionNoAMStrue%
\typeout{Option `noams' provided!}}
%    \end{macrocode}
%    \subsection{Unknown options}
%    \begin{macrocode}
\DeclareOption*{\PackageWarningNoLine{SIunits}{What is `\CurrentOption'?}}
%    \end{macrocode}
%  \subsection{The \pkgname{SIunits}\texttt{.cfg} file}
%  Load the |SIunits.cfg| file.
%  \changes{v0.03 Beta 5}{1998/12/11}{Load optional configuration file `SIunits.cfg'}
%    \begin{macrocode}
\InputIfFileExists{SIunits.cfg}{}%
{\PackageWarningNoLine{SIunits}{You have no `SIunits.cfg' file installed.
\MessageBreak I will assume you are using `thickspace' and `thickqspace'}
\ExecuteOptions{thickspace,thickqspace}}
\ProcessOptions\relax
%</package>
%    \end{macrocode}
%    \section{The \texttt{binary.sty} style for binary prefixes and (non-SI) units}
%    \begin{macrocode}
%<*binary>
\AtBeginDocument{%
\addprefix{\kibi}{Ki} \newcommand{\kibid}{\power{2}{10}}
\addprefix{\mebi}{Mi} \newcommand{\mebid}{\power{2}{20}}
\addprefix{\gibi}{Gi} \newcommand{\gibid}{\power{2}{30}}
\addprefix{\tebi}{Ti} \newcommand{\tebid}{\power{2}{40}}
\addprefix{\pebi}{Pi} \newcommand{\pebid}{\power{2}{50}}
\addprefix{\exbi}{Ei} \newcommand{\exbid}{\power{2}{60}}

\addunit{\bit}{bit}
\addunit{\byte}{B}%
} %\AtBeginDocument
%</binary>
%    \end{macrocode}
%    \changes{v1.13}{2000/08/29}{Index and change history generation error fixed}
%    \changes{v1.29}{2002/08/01}{Index and change history generation errors fixed}
%
%    \Finale
\endinput
