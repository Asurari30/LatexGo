% \iffalse meta-comment
%
% File: hobsub.dtx
% Version: 2016/05/16 v1.14
% Info: Construct package bundles
%
% Copyright (C) 2011, 2012 by
%    Heiko Oberdiek <heiko.oberdiek at googlemail.com>
%    2016
%    https://github.com/ho-tex/oberdiek/issues
%
% This work may be distributed and/or modified under the
% conditions of the LaTeX Project Public License, either
% version 1.3c of this license or (at your option) any later
% version. This version of this license is in
%    http://www.latex-project.org/lppl/lppl-1-3c.txt
% and the latest version of this license is in
%    http://www.latex-project.org/lppl.txt
% and version 1.3 or later is part of all distributions of
% LaTeX version 2005/12/01 or later.
%
% This work has the LPPL maintenance status "maintained".
%
% This Current Maintainer of this work is Heiko Oberdiek.
%
% The Base Interpreter refers to any `TeX-Format',
% because some files are installed in TDS:tex/generic//.
%
% This work consists of the main source file hobsub.dtx
% and the derived files
%    hobsub.sty, hobsub.pdf, hobsub.ins, hobsub.drv, hobsub-generic.sty,
%    hobsub-hyperref.sty, hobsub-test1.tex.
%
% Distribution:
%    CTAN:macros/latex/contrib/oberdiek/hobsub.dtx
%    CTAN:macros/latex/contrib/oberdiek/hobsub.pdf
%
% Unpacking:
%    (a) If hobsub.ins is present:
%           tex hobsub.ins
%    (b) Without hobsub.ins:
%           tex hobsub.dtx
%    (c) If you insist on using LaTeX
%           latex \let\install=y% \iffalse meta-comment
%
% File: hobsub.dtx
% Version: 2016/05/16 v1.14
% Info: Construct package bundles
%
% Copyright (C) 2011, 2012 by
%    Heiko Oberdiek <heiko.oberdiek at googlemail.com>
%    2016
%    https://github.com/ho-tex/oberdiek/issues
%
% This work may be distributed and/or modified under the
% conditions of the LaTeX Project Public License, either
% version 1.3c of this license or (at your option) any later
% version. This version of this license is in
%    http://www.latex-project.org/lppl/lppl-1-3c.txt
% and the latest version of this license is in
%    http://www.latex-project.org/lppl.txt
% and version 1.3 or later is part of all distributions of
% LaTeX version 2005/12/01 or later.
%
% This work has the LPPL maintenance status "maintained".
%
% This Current Maintainer of this work is Heiko Oberdiek.
%
% The Base Interpreter refers to any `TeX-Format',
% because some files are installed in TDS:tex/generic//.
%
% This work consists of the main source file hobsub.dtx
% and the derived files
%    hobsub.sty, hobsub.pdf, hobsub.ins, hobsub.drv, hobsub-generic.sty,
%    hobsub-hyperref.sty, hobsub-test1.tex.
%
% Distribution:
%    CTAN:macros/latex/contrib/oberdiek/hobsub.dtx
%    CTAN:macros/latex/contrib/oberdiek/hobsub.pdf
%
% Unpacking:
%    (a) If hobsub.ins is present:
%           tex hobsub.ins
%    (b) Without hobsub.ins:
%           tex hobsub.dtx
%    (c) If you insist on using LaTeX
%           latex \let\install=y% \iffalse meta-comment
%
% File: hobsub.dtx
% Version: 2016/05/16 v1.14
% Info: Construct package bundles
%
% Copyright (C) 2011, 2012 by
%    Heiko Oberdiek <heiko.oberdiek at googlemail.com>
%    2016
%    https://github.com/ho-tex/oberdiek/issues
%
% This work may be distributed and/or modified under the
% conditions of the LaTeX Project Public License, either
% version 1.3c of this license or (at your option) any later
% version. This version of this license is in
%    http://www.latex-project.org/lppl/lppl-1-3c.txt
% and the latest version of this license is in
%    http://www.latex-project.org/lppl.txt
% and version 1.3 or later is part of all distributions of
% LaTeX version 2005/12/01 or later.
%
% This work has the LPPL maintenance status "maintained".
%
% This Current Maintainer of this work is Heiko Oberdiek.
%
% The Base Interpreter refers to any `TeX-Format',
% because some files are installed in TDS:tex/generic//.
%
% This work consists of the main source file hobsub.dtx
% and the derived files
%    hobsub.sty, hobsub.pdf, hobsub.ins, hobsub.drv, hobsub-generic.sty,
%    hobsub-hyperref.sty, hobsub-test1.tex.
%
% Distribution:
%    CTAN:macros/latex/contrib/oberdiek/hobsub.dtx
%    CTAN:macros/latex/contrib/oberdiek/hobsub.pdf
%
% Unpacking:
%    (a) If hobsub.ins is present:
%           tex hobsub.ins
%    (b) Without hobsub.ins:
%           tex hobsub.dtx
%    (c) If you insist on using LaTeX
%           latex \let\install=y% \iffalse meta-comment
%
% File: hobsub.dtx
% Version: 2016/05/16 v1.14
% Info: Construct package bundles
%
% Copyright (C) 2011, 2012 by
%    Heiko Oberdiek <heiko.oberdiek at googlemail.com>
%    2016
%    https://github.com/ho-tex/oberdiek/issues
%
% This work may be distributed and/or modified under the
% conditions of the LaTeX Project Public License, either
% version 1.3c of this license or (at your option) any later
% version. This version of this license is in
%    http://www.latex-project.org/lppl/lppl-1-3c.txt
% and the latest version of this license is in
%    http://www.latex-project.org/lppl.txt
% and version 1.3 or later is part of all distributions of
% LaTeX version 2005/12/01 or later.
%
% This work has the LPPL maintenance status "maintained".
%
% This Current Maintainer of this work is Heiko Oberdiek.
%
% The Base Interpreter refers to any `TeX-Format',
% because some files are installed in TDS:tex/generic//.
%
% This work consists of the main source file hobsub.dtx
% and the derived files
%    hobsub.sty, hobsub.pdf, hobsub.ins, hobsub.drv, hobsub-generic.sty,
%    hobsub-hyperref.sty, hobsub-test1.tex.
%
% Distribution:
%    CTAN:macros/latex/contrib/oberdiek/hobsub.dtx
%    CTAN:macros/latex/contrib/oberdiek/hobsub.pdf
%
% Unpacking:
%    (a) If hobsub.ins is present:
%           tex hobsub.ins
%    (b) Without hobsub.ins:
%           tex hobsub.dtx
%    (c) If you insist on using LaTeX
%           latex \let\install=y\input{hobsub.dtx}
%        (quote the arguments according to the demands of your shell)
%
% Documentation:
%    (a) If hobsub.drv is present:
%           latex hobsub.drv
%    (b) Without hobsub.drv:
%           latex hobsub.dtx; ...
%    The class ltxdoc loads the configuration file ltxdoc.cfg
%    if available. Here you can specify further options, e.g.
%    use A4 as paper format:
%       \PassOptionsToClass{a4paper}{article}
%
%    Programm calls to get the documentation (example):
%       pdflatex hobsub.dtx
%       makeindex -s gind.ist hobsub.idx
%       pdflatex hobsub.dtx
%       makeindex -s gind.ist hobsub.idx
%       pdflatex hobsub.dtx
%
% Installation:
%    TDS:tex/generic/oberdiek/hobsub.sty
%    TDS:tex/generic/oberdiek/hobsub-generic.sty
%    TDS:tex/generic/oberdiek/hobsub-hyperref.sty
%    TDS:doc/latex/oberdiek/hobsub.pdf
%    TDS:doc/latex/oberdiek/test/hobsub-test1.tex
%    TDS:source/latex/oberdiek/hobsub.dtx
%
%<*ignore>
\begingroup
  \catcode123=1 %
  \catcode125=2 %
  \def\x{LaTeX2e}%
\expandafter\endgroup
\ifcase 0\ifx\install y1\fi\expandafter
         \ifx\csname processbatchFile\endcsname\relax\else1\fi
         \ifx\fmtname\x\else 1\fi\relax
\else\csname fi\endcsname
%</ignore>
%<*install>
\input docstrip.tex
\Msg{************************************************************************}
\Msg{* Installation}
\Msg{* Package: hobsub 2016/05/16 v1.14 Construct package bundles (HO)}
\Msg{************************************************************************}

\keepsilent
\askforoverwritefalse

\let\MetaPrefix\relax
\preamble

This is a generated file.

Project: hobsub
Version: 2016/05/16 v1.14

Copyright (C) 2011, 2012 by
   Heiko Oberdiek <heiko.oberdiek at googlemail.com>

This work may be distributed and/or modified under the
conditions of the LaTeX Project Public License, either
version 1.3c of this license or (at your option) any later
version. This version of this license is in
   http://www.latex-project.org/lppl/lppl-1-3c.txt
and the latest version of this license is in
   http://www.latex-project.org/lppl.txt
and version 1.3 or later is part of all distributions of
LaTeX version 2005/12/01 or later.

This work has the LPPL maintenance status "maintained".

This Current Maintainer of this work is Heiko Oberdiek.

The Base Interpreter refers to any `TeX-Format',
because some files are installed in TDS:tex/generic//.

This work consists of the main source file hobsub.dtx
and the derived files
   hobsub.sty, hobsub.pdf, hobsub.ins, hobsub.drv, hobsub-generic.sty,
   hobsub-hyperref.sty, hobsub-test1.tex.

\endpreamble
\let\MetaPrefix\DoubleperCent

\generate{%
  \file{hobsub.ins}{\from{hobsub.dtx}{install}}%
  \file{hobsub.drv}{\from{hobsub.dtx}{driver}}%
  \usedir{tex/generic/oberdiek}%
  \file{hobsub.sty}{\from{hobsub.dtx}{package}}%
  \usedir{doc/latex/oberdiek/test}%
  \file{hobsub-test1.tex}{\from{hobsub.dtx}{test1}}%
  \nopreamble
  \nopostamble
  \usedir{source/latex/oberdiek/catalogue}%
  \file{hobsub.xml}{\from{hobsub.dtx}{catalogue}}%
}
\let\hobsubSavedNormalLine\normalLine
\def\hobsubGobble#1{}
\begingroup
  \lccode`\0=32 % space
  \lccode`\1=35 % hash
  \lccode`\2=37 % percent
  \lccode`\3=92 % backslash
  \lccode`\4=123 % {
  \lccode`\5=125 % }
\lowercase{\endgroup
  \let\hobsubTokenSpaceOther= 0%
  \let\hobsubTokenHash= 1%
  \let\hobsubTokenPercent= 2%
  \let\hobsubTokenBackslash= 3%
  \let\hobsubTokenBraceLeft= 4%
  \let\hobsubTokenBraceRight= 5%
  \def\hobsubNormalLineWithSpace{%
    \hobsubSavedNormalLine0%
  }%
}
\def\normalLine{%
  \futurelet\hobsubToken\hobsubFirst
}
\def\hobsubFirst{%
  \ifx\hobsubToken\hobsubTokenSpaceOther
    \expandafter\expandafter\expandafter\futurelet
    \expandafter\expandafter\expandafter\hobsubToken
    \expandafter\expandafter\expandafter\hobsubNext
    \expandafter\hobsubGobble
  \else
    \expandafter\hobsubSavedNormalLine
  \fi
}
\def\hobsubNext{%
  \ifx\hobsubToken\hobsubTokenSpaceOther
    \expandafter\expandafter\expandafter\futurelet
    \expandafter\expandafter\expandafter\hobsubToken
    \expandafter\expandafter\expandafter\hobsubNext 
    \expandafter\hobsubGobble
  \else
    \let\hobsubWithSpace=Y%
    \ifx\hobsubToken\hobsubTokenBackslash
      \let\hobsubWithSpace=N%
    \else
      \ifx\hobsubToken\hobsubTokenBraceRight
        \let\hobsubWithSpace=N%
      \else
        \ifx\hobsubToken\hobsubTokenBraceLeft
          \let\hobsubWithSpace=N%
        \else
          \ifx\hobsubToken\hobsubTokenPercent
            \let\hobsubWithSpace=N%
          \else
            \ifx\hobsubToken\hobsubTokenHash
              \let\hobsubWithSpace=N%
            \fi
          \fi
        \fi
      \fi
    \fi
    \ifx\hobsubWithSpace Y%
      \expandafter\expandafter\expandafter\hobsubNormalLineWithSpace
    \else
      \expandafter\expandafter\expandafter\hobsubSavedNormalLine
    \fi
  \fi
}
\generate{%
  \usedir{tex/generic/oberdiek}%
  \file{hobsub-generic.sty}{%
    \from{hobsub.dtx}{generic,preamble,infwarerr}%
    \from{infwarerr.dtx}{package}%
    \from{hobsub.dtx}{post,ltxcmds}%
    \from{ltxcmds.dtx}{package}%
    \from{hobsub.dtx}{post,ifluatex}%
    \from{ifluatex.dtx}{package}%
    \from{hobsub.dtx}{post,ifvtex}%
    \from{ifvtex.dtx}{package}%
    \from{hobsub.dtx}{post,intcalc}%
    \from{intcalc.dtx}{package}%
    \from{hobsub.dtx}{post,ifpdf}%
    \from{ifpdf.dtx}{package}%
    \from{hobsub.dtx}{post,etexcmds}%
    \from{etexcmds.dtx}{package}%
    \from{hobsub.dtx}{post,kvsetkeys}%
    \from{kvsetkeys.dtx}{package}%
    \from{hobsub.dtx}{post,kvdefinekeys}%
    \from{kvdefinekeys.dtx}{package}%
    \from{hobsub.dtx}{post,luatex-loader}%
    \from{luatex.dtx}{loader}%
    \from{hobsub.dtx}{post,pdftexcmds}%
    \from{pdftexcmds.dtx}{package}%
    \from{hobsub.dtx}{post,pdfescape}%
    \from{pdfescape.dtx}{package}%
    \from{hobsub.dtx}{post,bigintcalc}%
    \from{bigintcalc.dtx}{package}%
    \from{hobsub.dtx}{post,bitset}%
    \from{bitset.dtx}{package}%
    \from{hobsub.dtx}{post,uniquecounter}%
    \from{uniquecounter.dtx}{package}%
    \from{hobsub.dtx}{post,atend}%
  }%
}
\generate{%
  \usedir{tex/generic/oberdiek}%
  \file{hobsub-hyperref.sty}{%
    \from{hobsub.dtx}{hyperref,preamble,,letltxmacro}%
    \from{letltxmacro.dtx}{package}%
    \from{hobsub.dtx}{post,hopatch}%
    \from{hopatch.dtx}{package}%
    \from{hobsub.dtx}{post,xcolor-patch}%
    \from{hycolor.dtx}{xcolor}%
    \from{hobsub.dtx}{post,atveryend}%
    \from{atveryend.dtx}{package}%
    \from{hobsub.dtx}{post,atbegshi}%
    \from{atbegshi.dtx}{package}%
    \from{hobsub.dtx}{post,refcount}%
    \from{refcount.dtx}{package}%
    \from{hobsub.dtx}{post,hycolor}%
    \from{hycolor.dtx}{package}%
    \from{hobsub.dtx}{post}%
  }%
}
\let\normalLine\hobsubSavedNormalLine

\catcode32=13\relax% active space
\let =\space%
\Msg{************************************************************************}
\Msg{*}
\Msg{* To finish the installation you have to move the following}
\Msg{* files into a directory searched by TeX:}
\Msg{*}
\Msg{*     hobsub.sty, hobsub-generic.sty, hobsub-hyperref.sty}
\Msg{*}
\Msg{* To produce the documentation run the file `hobsub.drv'}
\Msg{* through LaTeX.}
\Msg{*}
\Msg{* Happy TeXing!}
\Msg{*}
\Msg{************************************************************************}

\endbatchfile
%</install>
%<*ignore>
\fi
%</ignore>
%<*driver>
\NeedsTeXFormat{LaTeX2e}
\ProvidesFile{hobsub.drv}%
  [2016/05/16 v1.14 Construct package bundles (HO)]%
\documentclass{ltxdoc}
\usepackage{holtxdoc}[2011/11/22]
\usepackage{array}
\usepackage{caption}
\renewcommand*{\bottomfraction}{.5}
\renewcommand*{\topfraction}{.5}
\begin{document}
  \DocInput{hobsub.dtx}%
\end{document}
%</driver>
% \fi
%
%
% \CharacterTable
%  {Upper-case    \A\B\C\D\E\F\G\H\I\J\K\L\M\N\O\P\Q\R\S\T\U\V\W\X\Y\Z
%   Lower-case    \a\b\c\d\e\f\g\h\i\j\k\l\m\n\o\p\q\r\s\t\u\v\w\x\y\z
%   Digits        \0\1\2\3\4\5\6\7\8\9
%   Exclamation   \!     Double quote  \"     Hash (number) \#
%   Dollar        \$     Percent       \%     Ampersand     \&
%   Acute accent  \'     Left paren    \(     Right paren   \)
%   Asterisk      \*     Plus          \+     Comma         \,
%   Minus         \-     Point         \.     Solidus       \/
%   Colon         \:     Semicolon     \;     Less than     \<
%   Equals        \=     Greater than  \>     Question mark \?
%   Commercial at \@     Left bracket  \[     Backslash     \\
%   Right bracket \]     Circumflex    \^     Underscore    \_
%   Grave accent  \`     Left brace    \{     Vertical bar  \|
%   Right brace   \}     Tilde         \~}
%
% \GetFileInfo{hobsub.drv}
%
% \title{The \xpackage{hobsub} package}
% \date{2016/05/16 v1.14}
% \author{Heiko Oberdiek\thanks
% {Please report any issues at https://github.com/ho-tex/oberdiek/issues}\\
% \xemail{heiko.oberdiek at googlemail.com}}
%
% \maketitle
%
% \begin{abstract}
% Package \xpackage{hobsub} implements the idea of loading
% several packages in one big collection package.
% \end{abstract}
%
% \tableofcontents
%
% \def\TableGeneric{^^A
%   \TableEntry{\xpackage{infwarerr}}{}{2010/04/08 v1.3}^^A
%   \TableEntry{\xpackage{ltxcmds}}{}{2011/11/09 v1.22}^^A
%   \TableEntry{\xpackage{ifluatex}}{}{2010/03/01 v1.3}^^A
%   \TableEntry{\xpackage{ifvtex}}{}{2010/03/01 v1.5}^^A
%   \TableEntry{\xpackage{intcalc}}{}{2007/09/27 v1.1}^^A
%   \TableEntry{\xpackage{ifpdf}}{\xpackage{ifluatex}}{2011/01/30 v2.3}^^A
%   \TableEntry{\xpackage{etexcmds}}{}{2011/02/16 v1.5}^^A
%   \TableEntry{\xpackage{kvsetkeys}}{}{2012/04/25 v1.16}^^A
%   \TableEntry{\xpackage{kvdefinekeys}}{}{2011/04/07 v1.3}^^A
%   \TableEntry{\xpackage{luatex-loader}}{}{2010/03/09 v0.4}^^A
%   \TableEntry{\xpackage{pdftexcmds}}{\xpackage{ifluatex}, \xpackage{ifpdf}, \xpackage{infwarerr}, \xpackage{ltxcmds}, \xpackage{luatex-loader}}{2011/11/29 v0.20}^^A
%   \TableEntry{\xpackage{pdfescape}}{\xpackage{ltxcmds}, \xpackage{pdftexcmds}}{2011/11/25 v1.13}^^A
%   \TableEntry{\xpackage{bigintcalc}}{\xpackage{pdftexcmds}}{2012/04/08 v1.3}^^A
%   \TableEntry{\xpackage{bitset}}{\xpackage{bigintcalc}, \xpackage{infwarerr}, \xpackage{intcalc}}{2011/01/30 v1.1}^^A
%   \TableEntry{\xpackage{uniquecounter}}{\xpackage{bigintcalc}, \xpackage{infwarerr}}{2011/01/30 v1.2}^^A
% }
% \def\TableHyperref{^^A
%   \TableEntry{\xpackage{letltxmacro}}{}{2010/09/02 v1.4}^^A
%   \TableEntry{\xpackage{hopatch}}{}{2016/05/16 v1.2}^^A
%   \TableEntry{\xpackage{xcolor-patch}}{}{2011/01/30 v1.7}^^A
%   \TableEntry{\xpackage{atveryend}}{}{2011/06/30 v1.8}^^A
%   \TableEntry{\xpackage{atbegshi}}{\xpackage{ifpdf}, \xpackage{infwarerr}, \xpackage{ltxcmds}}{2011/10/05 v1.16}^^A
%   \TableEntry{\xpackage{refcount}}{\xpackage{infwarerr}, \xpackage{ltxcmds}}{2011/10/16 v3.4}^^A
%   \TableEntry{\xpackage{hycolor}}{\xpackage{xcolor-patch}}{2011/01/30 v1.7}^^A
% }
%
% \newenvironment{TableList}[2][htbp]{^^A
%   \begin{table}[{#1}]^^A
%   \centering
%   \caption{Overview \xpackage{hobsub-#2}}^^A
%   \label{tab:#2}^^A
%   \begin{tabular}{ll>{\raggedright}p{.4\linewidth}}^^A
%   \bfseries Package & \bfseries Release &
%   \bfseries Dependencies\tabularnewline
%   \hline
% }{^^A
%   \end{tabular}^^A
%   \end{table}^^A
% }
% \newcommand*{\TableEntry}[3]{%
%   #1&#3&#2\tabularnewline
% }
%
% \renewcommand*{\arraystretch}{1.1}
% \setlength{\extrarowheight}{2pt}
%
% \section{Documentation}
%
% In January 2011 the mailing list lualatex-dev discussed
% ``quietening lualatex console output'' \cite{lualatex:quiet}.
% Inspired from this, I implemented this package \xpackage{hobsub}
% to load several packages in one file. The package is applied for
% two packages collections that are subsets of my package bundle
% \xpackage{oberdiek}.
% \begin{description}
% \item[\xpackage{hobsub-generic}] is a collection of packages
%   that can also be used with \hologo{plainTeX}.
% \item[\xpackage{hobsub-hyperref}] is a collection of packages
%   that is used by package \xpackage{hyperref}.
% \end{description}
%
% Package \xpackage{hobsub} provides the macros that
% are used in the collection packages and might be used
% in other projects as well. But there are many caveats,
% some of them are discussed below.
%
% The most serious problem is consistency. A package
% that is available as standalone package and is part
% of a collection might have different versions.
% Because my packages are organized in a bundle and should
% be installed and updated as bundle, the risk seems low
% to me. However, if a single package is updated manually, for
% example for testing, then it should be loaded before the
% collection that contains the package with an older version.
% In a collection, the package will not be loaded again if it
% is already loaded.
%
% \subsection{Usage}
%
% The collection packages \xpackage{hobsub-generic} and
% \xpackage{hobsub-hyperref} are loaded as usual:
% \begin{quote}
%  \begin{tabular}{@{}ll@{}}
%    |\usepackage{hobsub-generic}| & \hologo{LaTeX}\\
%    |\input hobsub-generic.sty| & \hologo{plainTeX}
%  \end{tabular}
% \end{quote}
% or
% \begin{quote}|\usepackage{hobsub-hyperref}|\end{quote}
% Of course, both \cs{usepackage} or \cs{RequirePackage}
% can be used.
%
% If you need requirements on the version date of a package
% inside a collection, then specify them afterwards, for
% example:
% \begin{quote}
%   |\usepackage{hobsub-generic}|\\
%   |\usepackage{ltxcmds}[2010/12/04]% |^^A
%   {\itshape |\ltx@ifblank is needed|}
% \end{quote}
%
% Also it is not garanteed that a package will be always
% part of a collection. Therefore it does not harm,
% to make the requirements explicite \emph{after}
% a collection is loaded, see the previous example.
% \hologo{LaTeX} knows that a
% package is already loaded and does not load it again.
%
% \subsection{Supported features}
%
% \begin{description}
% \item[No reloads:] If a package is already loaded then it
%  is not reloaded again if it is part of a collection package.
% \item[\cs{listfiles}:]
%   Each package that is loaded in a collection is also
%   added to the output of \cs{listfiles}.
% \item[\cs{@currname}, \cs{@currext}]
%   are updated, they are used by \hologo{LaTeX}'s
%  \cs{ProvidesPackage}.
% \item[\cs{AtEndOfPackage}:]
%   \hologo{LaTeX}'s hook mechanism that is used for
%   \cs{AtEndOfPackage} is supported. However none of the
%   packages in the collections are using \cs{AtEndOfPackage},
%   therefore this feature is completely untested.
% \end{description}
%
% \subsection{Common limitations}
%
% \begin{description}
% \item[\cs{endinput}:]
%   Often a package is using \cs{endinput}, either added by
%   \xpackage{docstrip}, or to stop package loading at an earlier
%   time. Therefore \cs{endinput} must be catched during package
%   loading to prevent that the collection package is closed
%   too early before reading all packages it contains.
%
%   This redefinition of \cs{endinput} fails if the package loads
%   other files that contain \cs{endinput}. To some degree it could
%   be catched by redefining macros that load files. But this will
%   not work for the primitive \cs{input}.
% \item[Options:]
%   \hologo{LaTeX}'s package options are not supported.
%   Therefore the packages that are put in a collection must
%   not support options.
% \item[Hook packages:]
%   Some packages and classes provides hooks for packages, see
%   packages \xpackage{scrlfile}, \xpackage{filehook}
%   or class \xclass{memoir}. These hooks are not supported
%   for embedded packages inside a collection package.
%   Because of the problem with \cs{endinput}, these hooks
%   are not supported. The hooks might contain code that
%   loads other files that execute \cs{endinput}.
% \end{description}
%
% \subsection{Package \xpackage{hobsub-generic}}
%
% \begin{TableList}[bp]{generic}\TableGeneric\end{TableList}
%
% Collection `generic' contains some of the \hologo{plainTeX} compatible
% packages of my bundle, see table \ref{tab:generic}.
% It adds the following restrictions to any package
% that is part of this collection:
% \begin{description}
% \item[Generic formats:] The following formats must be supported:
%   \hologo{LaTeX}, \hologo{plainTeX} and even \hologo{iniTeX}.
%   This applies to the collection package \xpackage{hobsub-generic}
%   the same way.
% \item[Prefixed macros:]
%   Any package of this collection should only define macros with
%   prefixed names to avoid name clashes. Resources like registers
%   must not be allocated and other global states should not be
%   changed except for added macro definitions.
% \end{description}
% Thus the package \xpackage{hobsub-generic} can also preloaded
% in the format file with \hologo{iniTeX}.
%
% \subsection{Package \xpackage{hobsub-hyperref}}
%
% \begin{TableList}[tp]{hyperref}\TableHyperref\end{TableList}
%
% Collection `hyperref' contains additional packages that are
% used by package \xpackage{hyperref} and that does not fit
% in collection `generic', see table{tab:hyperref}.
% The latter collection package
% \xpackage{hobsub-generic} is loaded via a conventional
% \cs{RequirePackage}.
%
% Because most of the packages in the collection `hyperref'
% needs \hologo{LaTeX}, the collection requires \hologo{LaTeX}.
%
% \subsection{Package \xpackage{hobsub}}
%
% This section is for advanced users that might want to
% build a collection package. Then it can be done by
% using \xpackage{hobsub}:
% \begin{quote}
%   \dots\\
%   |\RequirePackage{hobsub}[2016/05/16]|\\
%   \dots\\
%   |\hobsub@StartPackage{foo1}|\\
%   \textit{\dots\unskip\ package code of \xfile{foo1.sty} \dots}\\
%   |\endinput|\\
%   |<hobsub>|\\
%   |\hobsub@StopPackage|\\
%   |\hobsub@StartPackage{bar2}|\\
%   \textit{\dots\unskip\ package code of \xfile{bar2.sty} \dots}\\
%   |\endinput|\\
%   |<hobsub>|\\
%   |\hobsub@StopPackage|
%   \dots
% \end{quote}
% But be aware of the caveats and limitations, see above.
% Only quite a small portion of the packages can be embedded
% like this.
%
% \clearpage
%
% \StopEventually{
% }
%
% \section{Implementation}
%
% \subsection{Reload check and package identification}
%
% \subsubsection{Package \xpackage{hobsub}}
%
%    \begin{macrocode}
%<*package>
%    \end{macrocode}
%    Reload check, especially if the package is not used with \LaTeX.
%    \begin{macrocode}
\begingroup\catcode61\catcode48\catcode32=10\relax%
  \catcode13=5 % ^^M
  \endlinechar=13 %
  \catcode35=6 % #
  \catcode39=12 % '
  \catcode44=12 % ,
  \catcode45=12 % -
  \catcode46=12 % .
  \catcode58=12 % :
  \catcode64=11 % @
  \catcode123=1 % {
  \catcode125=2 % }
  \expandafter\let\expandafter\x\csname ver@hobsub.sty\endcsname
  \ifx\x\relax % plain-TeX, first loading
  \else
    \def\empty{}%
    \ifx\x\empty % LaTeX, first loading,
      % variable is initialized, but \ProvidesPackage not yet seen
    \else
      \expandafter\ifx\csname PackageInfo\endcsname\relax
        \def\x#1#2{%
          \immediate\write-1{Package #1 Info: #2.}%
        }%
      \else
        \def\x#1#2{\PackageInfo{#1}{#2, stopped}}%
      \fi
      \x{hobsub}{The package is already loaded}%
      \aftergroup\endinput
    \fi
  \fi
\endgroup%
%    \end{macrocode}
%    Package identification:
%    \begin{macrocode}
\begingroup\catcode61\catcode48\catcode32=10\relax%
  \catcode13=5 % ^^M
  \endlinechar=13 %
  \catcode35=6 % #
  \catcode39=12 % '
  \catcode40=12 % (
  \catcode41=12 % )
  \catcode44=12 % ,
  \catcode45=12 % -
  \catcode46=12 % .
  \catcode47=12 % /
  \catcode58=12 % :
  \catcode64=11 % @
  \catcode91=12 % [
  \catcode93=12 % ]
  \catcode123=1 % {
  \catcode125=2 % }
  \expandafter\ifx\csname ProvidesPackage\endcsname\relax
    \def\x#1#2#3[#4]{\endgroup
      \immediate\write-1{Package: #3 #4}%
      \xdef#1{#4}%
    }%
  \else
    \def\x#1#2[#3]{\endgroup
      #2[{#3}]%
      \ifx#1\@undefined
        \xdef#1{#3}%
      \fi
      \ifx#1\relax
        \xdef#1{#3}%
      \fi
    }%
  \fi
\expandafter\x\csname ver@hobsub.sty\endcsname
\ProvidesPackage{hobsub}%
  [2016/05/16 v1.14 Construct package bundles (HO)]%
%    \end{macrocode}
%    \begin{macrocode}
%</package>
%    \end{macrocode}
%
% \subsubsection{Package \xpackage{hobsub-generic}}
%
%    \begin{macrocode}
%<*generic>
%    \end{macrocode}
%    Reload check, especially if the package is not used with \LaTeX.
%    \begin{macrocode}
\begingroup\catcode61\catcode48\catcode32=10\relax%
  \catcode13=5 % ^^M
  \endlinechar=13 %
  \catcode35=6 % #
  \catcode39=12 % '
  \catcode44=12 % ,
  \catcode45=12 % -
  \catcode46=12 % .
  \catcode58=12 % :
  \catcode64=11 % @
  \catcode123=1 % {
  \catcode125=2 % }
  \expandafter\let\expandafter\x\csname ver@hobsub-generic.sty\endcsname
  \ifx\x\relax % plain-TeX, first loading
  \else
    \def\empty{}%
    \ifx\x\empty % LaTeX, first loading,
      % variable is initialized, but \ProvidesPackage not yet seen
    \else
      \expandafter\ifx\csname PackageInfo\endcsname\relax
        \def\x#1#2{%
          \immediate\write-1{Package #1 Info: #2.}%
        }%
      \else
        \def\x#1#2{\PackageInfo{#1}{#2, stopped}}%
      \fi
      \x{hobsub-generic}{The package is already loaded}%
      \aftergroup\endinput
    \fi
  \fi
\endgroup%
%    \end{macrocode}
%    Package identification:
%    \begin{macrocode}
\begingroup\catcode61\catcode48\catcode32=10\relax%
  \catcode13=5 % ^^M
  \endlinechar=13 %
  \catcode35=6 % #
  \catcode39=12 % '
  \catcode40=12 % (
  \catcode41=12 % )
  \catcode44=12 % ,
  \catcode45=12 % -
  \catcode46=12 % .
  \catcode47=12 % /
  \catcode58=12 % :
  \catcode64=11 % @
  \catcode91=12 % [
  \catcode93=12 % ]
  \catcode123=1 % {
  \catcode125=2 % }
  \expandafter\ifx\csname ProvidesPackage\endcsname\relax
    \def\x#1#2#3[#4]{\endgroup
      \immediate\write-1{Package: #3 #4}%
      \xdef#1{#4}%
    }%
  \else
    \def\x#1#2[#3]{\endgroup
      #2[{#3}]%
      \ifx#1\@undefined
        \xdef#1{#3}%
      \fi
      \ifx#1\relax
        \xdef#1{#3}%
      \fi
    }%
  \fi
\expandafter\x\csname ver@hobsub-generic.sty\endcsname
\ProvidesPackage{hobsub-generic}%
  [2016/05/16 v1.14 Bundle oberdiek, subset generic (HO)]%
%    \end{macrocode}
%    \begin{macrocode}
%</generic>
%    \end{macrocode}
%
% \subsubsection{Package \xpackage{hobsub-hyperref}}
%
%    \begin{macrocode}
%<*hyperref>
%    \end{macrocode}
%    Reload check, especially if the package is not used with \LaTeX.
%    \begin{macrocode}
\begingroup\catcode61\catcode48\catcode32=10\relax%
  \catcode13=5 % ^^M
  \endlinechar=13 %
  \catcode35=6 % #
  \catcode39=12 % '
  \catcode44=12 % ,
  \catcode45=12 % -
  \catcode46=12 % .
  \catcode58=12 % :
  \catcode64=11 % @
  \catcode123=1 % {
  \catcode125=2 % }
  \expandafter\let\expandafter\x\csname ver@hobsub-hyperref.sty\endcsname
  \ifx\x\relax % plain-TeX, first loading
  \else
    \def\empty{}%
    \ifx\x\empty % LaTeX, first loading,
      % variable is initialized, but \ProvidesPackage not yet seen
    \else
      \expandafter\ifx\csname PackageInfo\endcsname\relax
        \def\x#1#2{%
          \immediate\write-1{Package #1 Info: #2.}%
        }%
      \else
        \def\x#1#2{\PackageInfo{#1}{#2, stopped}}%
      \fi
      \x{hobsub-hyperref}{The package is already loaded}%
      \aftergroup\endinput
    \fi
  \fi
\endgroup%
%    \end{macrocode}
%    Package identification:
%    \begin{macrocode}
\begingroup\catcode61\catcode48\catcode32=10\relax%
  \catcode13=5 % ^^M
  \endlinechar=13 %
  \catcode35=6 % #
  \catcode39=12 % '
  \catcode40=12 % (
  \catcode41=12 % )
  \catcode44=12 % ,
  \catcode45=12 % -
  \catcode46=12 % .
  \catcode47=12 % /
  \catcode58=12 % :
  \catcode64=11 % @
  \catcode91=12 % [
  \catcode93=12 % ]
  \catcode123=1 % {
  \catcode125=2 % }
  \expandafter\ifx\csname ProvidesPackage\endcsname\relax
    \def\x#1#2#3[#4]{\endgroup
      \immediate\write-1{Package: #3 #4}%
      \xdef#1{#4}%
    }%
  \else
    \def\x#1#2[#3]{\endgroup
      #2[{#3}]%
      \ifx#1\@undefined
        \xdef#1{#3}%
      \fi
      \ifx#1\relax
        \xdef#1{#3}%
      \fi
    }%
  \fi
\expandafter\x\csname ver@hobsub-hyperref.sty\endcsname
\ProvidesPackage{hobsub-hyperref}%
  [2016/05/16 v1.14 Bundle oberdiek, subset hyperref (HO)]%
%    \end{macrocode}
%    \begin{macrocode}
%</hyperref>
%    \end{macrocode}
%
% \subsection{Catcodes}
%
%    \begin{macrocode}
%<*package|generic|hyperref>
\begingroup\catcode61\catcode48\catcode32=10\relax%
  \catcode13=5 % ^^M
  \endlinechar=13 %
  \catcode123=1 % {
  \catcode125=2 % }
  \catcode64=11 % @
  \def\x{\endgroup
    \expandafter\edef\csname HOBsub@AtEnd\endcsname{%
      \endlinechar=\the\endlinechar\relax
      \catcode13=\the\catcode13\relax
      \catcode32=\the\catcode32\relax
      \catcode35=\the\catcode35\relax
      \catcode61=\the\catcode61\relax
      \catcode64=\the\catcode64\relax
      \catcode123=\the\catcode123\relax
      \catcode125=\the\catcode125\relax
    }%
  }%
\x\catcode61\catcode48\catcode32=10\relax%
\catcode13=5 % ^^M
\endlinechar=13 %
\catcode35=6 % #
\catcode64=11 % @
\catcode123=1 % {
\catcode125=2 % }
\def\TMP@EnsureCode#1#2{%
  \edef\HOBsub@AtEnd{%
    \HOBsub@AtEnd
    \catcode#1=\the\catcode#1\relax
  }%
  \catcode#1=#2\relax
}
\TMP@EnsureCode{39}{12}% '
\TMP@EnsureCode{40}{12}% (
\TMP@EnsureCode{41}{12}% )
\TMP@EnsureCode{45}{12}% -
\TMP@EnsureCode{46}{12}% .
\TMP@EnsureCode{47}{12}% /
\TMP@EnsureCode{58}{12}% :
\TMP@EnsureCode{60}{12}% <
\TMP@EnsureCode{62}{12}% >
\TMP@EnsureCode{96}{12}% `
\edef\HOBsub@AtEnd{\HOBsub@AtEnd\noexpand\endinput}
%    \end{macrocode}
%
%    \begin{macrocode}
%</package|generic|hyperref>
%    \end{macrocode}
%
% \subsection{Package \xpackage{hobsub-hyperref} loads \xpackage{hobsub-generic}}
%
%    \begin{macrocode}
%<*hyperref>
\NeedsTeXFormat{LaTeX2e}
\RequirePackage{hobsub-generic}[2016/05/16]
%</hyperref>
%    \end{macrocode}
%
% \subsection{Preamble}
%
%    \begin{macrocode}
%<*preamble>
\begingroup\expandafter\expandafter\expandafter\endgroup
\expandafter\ifx\csname ver@hobsub.sty\endcsname\relax
\else
  \expandafter\hobsub@GobbleRemainingPackage
\fi
%</preamble>
%    \end{macrocode}
%
%
%    \begin{macrocode}
%<*package|preamble>
%    \end{macrocode}
%
%    \begin{macro}{\HOBsub@OrgEndinput}
%    \begin{macrocode}
\begingroup\expandafter\expandafter\expandafter\endgroup
\expandafter\ifx\csname HOBsub@OrgEndinput\endcsname\relax
  \let\HOBsub@OrgEndinput\endinput
\fi
%    \end{macrocode}
%    \end{macro}
%    \begin{macro}{\hobsub@GobbleRemainingPackage}
%    \begin{macrocode}
\def\hobsub@GobbleRemainingPackage{%
  \begingroup
  \catcode92=14 % backslash: comment
  \catcode32=14 % space: comment
  \catcode35=14 % hash: comment
  \catcode123=14 % left brace: comment
  \catcode125=14 % right brace: comment
  \catcode60=3 % less: math
  \catcode62=4 % greater: align
  \endlinechar=-1 %
  \HOBsub@GobbleRemainingPackage
}
%    \end{macrocode}
%    \end{macro}
%    \begin{macro}{\HOBsub@GobbleRemainingPackage}
%    \begin{macrocode}
\catcode60=3 % less: dollar
\catcode62=4 % greater: align
\long\def\HOBsub@GobbleRemainingPackage#1<hobsub>{%
  \endgroup
}
\catcode60=12 % less: other
\catcode62=12 % greater: other
%    \end{macrocode}
%    \end{macro}
%
%    \begin{macro}{\hobsub@StartPackage}
%    \begin{macrocode}
\def\hobsub@StartPackage#1{%
  \begingroup\expandafter\expandafter\expandafter\endgroup
  \expandafter\ifx\csname ver@#1.sty\endcsname\relax
    \let\HOBsub@OrgCurrName\@currname
    \let\HOBsub@OrgCurrExt\@currext
    \csname @pushfilename\endcsname
    \def\@currname{#1}%
    \def\@currext{sty}%
    \expandafter\def\csname\@currname.\@currext-h@@k\endcsname{}%
    \let\endinput\hobsub@GobbleRemainingPackage
    \def\hobsub@StopPackage{%
      \let\hobsub@StopPackage\relax
      \HOBsub@StopPackage
%     \hobsub@Info{hobsub}{Package `#1' loaded}%
    }%
    \hobsub@AddToFileList{#1.sty}%
  \else
    \hobsub@Info{hobsub}{Skipping package `#1' (already loaded)}%
    \let\hobsub@StopPackage\relax
    \expandafter\hobsub@GobbleRemainingPackage
  \fi
}
%    \end{macrocode}
%    \end{macro}
%
%    \begin{macro}{\hobsub@StopPackage}
%    \begin{macrocode}
\let\hobsub@StopPackage\relax
%    \end{macrocode}
%    \end{macro}
%
%    \begin{macro}{\hobsub@Info}
%    \begin{macrocode}
\def\hobsub@Info#1#2{%
  \begingroup\expandafter\expandafter\expandafter\endgroup
  \expandafter\ifx\csname @PackageInfoNoLine\endcsname\relax
    \immediate\write-1{Package #1 Info: #2.}%
  \else
    \let\hobsub@Info\@PackageInfoNoLine
    \hobsub@Info{#1}{#2}%
  \fi
}
%    \end{macrocode}
%    \end{macro}
%
%    \begin{macro}{\HOBsub@StopPackage}
%    \begin{macrocode}
\def\HOBsub@StopPackage{%
  \csname\@currname.\@currext-h@@k\endcsname
  \let\endinput\HOBsub@OrgEndinput
  \csname @popfilename\endcsname
  \let\@currname\HOBsub@OrgCurrName
  \let\@currext\HOBsub@OrgCurrExt
}
%    \end{macrocode}
%    \end{macro}
%
%    \begin{macro}{\hobsub@AddToFileList}
%    \begin{macrocode}
\begingroup\expandafter\expandafter\expandafter\endgroup
\expandafter\ifx\csname @addtofilelist\endcsname\relax
  \def\hobsub@AddToFileList#1{}%
\else
  \def\hobsub@AddToFileList#1{%
    \@addtofilelist{#1}%
  }%
\fi
%    \end{macrocode}
%    \end{macro}
%
%    \begin{macrocode}
%</package|preamble>
%    \end{macrocode}
%
% \subsection{End of preamble for subset packages}
%
%    Add the end marker for possible skipping of previous
%    macros, if package \xpackage{hobsub} was already loaded.
%    \begin{macrocode}
%<*preamble>
\hobsub@GobbleRemainingPackage
<hobsub>
%    \end{macrocode}
%
%    All macros of package \xpackage{hobsub} are now available.
%    Now we officially load the package \xpackage{hobsub} inline
%    for proper package identification.
%    \begin{macrocode}
\hobsub@StartPackage{hobsub}%
\begingroup\expandafter\expandafter\expandafter\endgroup
\expandafter\ifx\csname ProvidesPackage\endcsname\relax
  \immediate\write-1{%
    Package: hobsub 2016/05/16 v1.14 Construct package bundles (HO)%
  }%
\else
  \ProvidesPackage{hobsub}%
  [2016/05/16 v1.14 Construct package bundles (HO)]%
\fi
%</preamble>
%    \end{macrocode}
%
%
%    \begin{macrocode}
%<*post|preamble>
 \endinput
<hobsub>
\hobsub@StopPackage
%</post|preamble>
%    \end{macrocode}
%
% \subsection{Package list}
%
%    \begin{macrocode}
%<infwarerr>\hobsub@StartPackage{infwarerr}
%<ltxcmds>\hobsub@StartPackage{ltxcmds}
%<ifluatex>\hobsub@StartPackage{ifluatex}
%<ifvtex>\hobsub@StartPackage{ifvtex}
%<intcalc>\hobsub@StartPackage{intcalc}
%<ifpdf>\hobsub@StartPackage{ifpdf}
%<etexcmds>\hobsub@StartPackage{etexcmds}
%<kvsetkeys>\hobsub@StartPackage{kvsetkeys}
%<kvdefinekeys>\hobsub@StartPackage{kvdefinekeys}
%<*luatex-loader>
\ifluatex
\else
  \expandafter\hobsub@GobbleRemainingPackage
\fi
\hobsub@StartPackage{luatex-loader}
%</luatex-loader>
%<pdftexcmds>\hobsub@StartPackage{pdftexcmds}
%<pdfescape>\hobsub@StartPackage{pdfescape}
%<bigintcalc>\hobsub@StartPackage{bigintcalc}
%<bitset>\hobsub@StartPackage{bitset}
%<uniquecounter>\hobsub@StartPackage{uniquecounter}
%<letltxmacro>\hobsub@StartPackage{letltxmacro}
%<hopatch>\hobsub@StartPackage{hopatch}
%<xcolor-patch>\hobsub@StartPackage{xcolor-patch}
%<atveryend>\hobsub@StartPackage{atveryend}
%<atbegshi>\hobsub@StartPackage{atbegshi}
%<refcount>\hobsub@StartPackage{refcount}
%<hycolor>\hobsub@StartPackage{hycolor}
%    \end{macrocode}
%
% \subsection{End of packages}
%
%    \begin{macrocode}
%<package|atend>\HOBsub@AtEnd%
%    \end{macrocode}
%
% \section{Test}
%
% \subsection{Catcode checks for loading}
%
%    \begin{macrocode}
%<*test1>
%    \end{macrocode}
%    \begin{macrocode}
\catcode`\{=1 %
\catcode`\}=2 %
\catcode`\#=6 %
\catcode`\@=11 %
\expandafter\ifx\csname count@\endcsname\relax
  \countdef\count@=255 %
\fi
\expandafter\ifx\csname @gobble\endcsname\relax
  \long\def\@gobble#1{}%
\fi
\expandafter\ifx\csname @firstofone\endcsname\relax
  \long\def\@firstofone#1{#1}%
\fi
\expandafter\ifx\csname loop\endcsname\relax
  \expandafter\@firstofone
\else
  \expandafter\@gobble
\fi
{%
  \def\loop#1\repeat{%
    \def\body{#1}%
    \iterate
  }%
  \def\iterate{%
    \body
      \let\next\iterate
    \else
      \let\next\relax
    \fi
    \next
  }%
  \let\repeat=\fi
}%
\def\RestoreCatcodes{}
\count@=0 %
\loop
  \edef\RestoreCatcodes{%
    \RestoreCatcodes
    \catcode\the\count@=\the\catcode\count@\relax
  }%
\ifnum\count@<255 %
  \advance\count@ 1 %
\repeat

\def\RangeCatcodeInvalid#1#2{%
  \count@=#1\relax
  \loop
    \catcode\count@=15 %
  \ifnum\count@<#2\relax
    \advance\count@ 1 %
  \repeat
}
\def\RangeCatcodeCheck#1#2#3{%
  \count@=#1\relax
  \loop
    \ifnum#3=\catcode\count@
    \else
      \errmessage{%
        Character \the\count@\space
        with wrong catcode \the\catcode\count@\space
        instead of \number#3%
      }%
    \fi
  \ifnum\count@<#2\relax
    \advance\count@ 1 %
  \repeat
}
\def\space{ }
\expandafter\ifx\csname LoadCommand\endcsname\relax
  \def\LoadCommand{\input hobsub.sty\relax}%
\fi
\def\Test{%
  \RangeCatcodeInvalid{0}{47}%
  \RangeCatcodeInvalid{58}{64}%
  \RangeCatcodeInvalid{91}{96}%
  \RangeCatcodeInvalid{123}{255}%
  \catcode`\@=12 %
  \catcode`\\=0 %
  \catcode`\%=14 %
  \LoadCommand
  \RangeCatcodeCheck{0}{36}{15}%
  \RangeCatcodeCheck{37}{37}{14}%
  \RangeCatcodeCheck{38}{47}{15}%
  \RangeCatcodeCheck{48}{57}{12}%
  \RangeCatcodeCheck{58}{63}{15}%
  \RangeCatcodeCheck{64}{64}{12}%
  \RangeCatcodeCheck{65}{90}{11}%
  \RangeCatcodeCheck{91}{91}{15}%
  \RangeCatcodeCheck{92}{92}{0}%
  \RangeCatcodeCheck{93}{96}{15}%
  \RangeCatcodeCheck{97}{122}{11}%
  \RangeCatcodeCheck{123}{255}{15}%
  \RestoreCatcodes
}
\Test
\csname @@end\endcsname
\end
%    \end{macrocode}
%    \begin{macrocode}
%</test1>
%    \end{macrocode}
%
% \section{Installation}
%
% \subsection{Download}
%
% \paragraph{Package.} This package is available on
% CTAN\footnote{\url{http://ctan.org/pkg/hobsub}}:
% \begin{description}
% \item[\CTAN{macros/latex/contrib/oberdiek/hobsub.dtx}] The source file.
% \item[\CTAN{macros/latex/contrib/oberdiek/hobsub.pdf}] Documentation.
% \end{description}
%
%
% \paragraph{Bundle.} All the packages of the bundle `oberdiek'
% are also available in a TDS compliant ZIP archive. There
% the packages are already unpacked and the documentation files
% are generated. The files and directories obey the TDS standard.
% \begin{description}
% \item[\CTAN{install/macros/latex/contrib/oberdiek.tds.zip}]
% \end{description}
% \emph{TDS} refers to the standard ``A Directory Structure
% for \TeX\ Files'' (\CTAN{tds/tds.pdf}). Directories
% with \xfile{texmf} in their name are usually organized this way.
%
% \subsection{Bundle installation}
%
% \paragraph{Unpacking.} Unpack the \xfile{oberdiek.tds.zip} in the
% TDS tree (also known as \xfile{texmf} tree) of your choice.
% Example (linux):
% \begin{quote}
%   |unzip oberdiek.tds.zip -d ~/texmf|
% \end{quote}
%
% \paragraph{Script installation.}
% Check the directory \xfile{TDS:scripts/oberdiek/} for
% scripts that need further installation steps.
% Package \xpackage{attachfile2} comes with the Perl script
% \xfile{pdfatfi.pl} that should be installed in such a way
% that it can be called as \texttt{pdfatfi}.
% Example (linux):
% \begin{quote}
%   |chmod +x scripts/oberdiek/pdfatfi.pl|\\
%   |cp scripts/oberdiek/pdfatfi.pl /usr/local/bin/|
% \end{quote}
%
% \subsection{Package installation}
%
% \paragraph{Unpacking.} The \xfile{.dtx} file is a self-extracting
% \docstrip\ archive. The files are extracted by running the
% \xfile{.dtx} through \plainTeX:
% \begin{quote}
%   \verb|tex hobsub.dtx|
% \end{quote}
%
% \paragraph{TDS.} Now the different files must be moved into
% the different directories in your installation TDS tree
% (also known as \xfile{texmf} tree):
% \begin{quote}
% \def\t{^^A
% \begin{tabular}{@{}>{\ttfamily}l@{ $\rightarrow$ }>{\ttfamily}l@{}}
%   hobsub.sty & tex/generic/oberdiek/hobsub.sty\\
%   hobsub-generic.sty & tex/generic/oberdiek/hobsub-generic.sty\\
%   hobsub-hyperref.sty & tex/generic/oberdiek/hobsub-hyperref.sty\\
%   hobsub.pdf & doc/latex/oberdiek/hobsub.pdf\\
%   test/hobsub-test1.tex & doc/latex/oberdiek/test/hobsub-test1.tex\\
%   hobsub.dtx & source/latex/oberdiek/hobsub.dtx\\
% \end{tabular}^^A
% }^^A
% \sbox0{\t}^^A
% \ifdim\wd0>\linewidth
%   \begingroup
%     \advance\linewidth by\leftmargin
%     \advance\linewidth by\rightmargin
%   \edef\x{\endgroup
%     \def\noexpand\lw{\the\linewidth}^^A
%   }\x
%   \def\lwbox{^^A
%     \leavevmode
%     \hbox to \linewidth{^^A
%       \kern-\leftmargin\relax
%       \hss
%       \usebox0
%       \hss
%       \kern-\rightmargin\relax
%     }^^A
%   }^^A
%   \ifdim\wd0>\lw
%     \sbox0{\small\t}^^A
%     \ifdim\wd0>\linewidth
%       \ifdim\wd0>\lw
%         \sbox0{\footnotesize\t}^^A
%         \ifdim\wd0>\linewidth
%           \ifdim\wd0>\lw
%             \sbox0{\scriptsize\t}^^A
%             \ifdim\wd0>\linewidth
%               \ifdim\wd0>\lw
%                 \sbox0{\tiny\t}^^A
%                 \ifdim\wd0>\linewidth
%                   \lwbox
%                 \else
%                   \usebox0
%                 \fi
%               \else
%                 \lwbox
%               \fi
%             \else
%               \usebox0
%             \fi
%           \else
%             \lwbox
%           \fi
%         \else
%           \usebox0
%         \fi
%       \else
%         \lwbox
%       \fi
%     \else
%       \usebox0
%     \fi
%   \else
%     \lwbox
%   \fi
% \else
%   \usebox0
% \fi
% \end{quote}
% If you have a \xfile{docstrip.cfg} that configures and enables \docstrip's
% TDS installing feature, then some files can already be in the right
% place, see the documentation of \docstrip.
%
% \subsection{Refresh file name databases}
%
% If your \TeX~distribution
% (\teTeX, \mikTeX, \dots) relies on file name databases, you must refresh
% these. For example, \teTeX\ users run \verb|texhash| or
% \verb|mktexlsr|.
%
% \subsection{Some details for the interested}
%
% \paragraph{Attached source.}
%
% The PDF documentation on CTAN also includes the
% \xfile{.dtx} source file. It can be extracted by
% AcrobatReader 6 or higher. Another option is \textsf{pdftk},
% e.g. unpack the file into the current directory:
% \begin{quote}
%   \verb|pdftk hobsub.pdf unpack_files output .|
% \end{quote}
%
% \paragraph{Unpacking with \LaTeX.}
% The \xfile{.dtx} chooses its action depending on the format:
% \begin{description}
% \item[\plainTeX:] Run \docstrip\ and extract the files.
% \item[\LaTeX:] Generate the documentation.
% \end{description}
% If you insist on using \LaTeX\ for \docstrip\ (really,
% \docstrip\ does not need \LaTeX), then inform the autodetect routine
% about your intention:
% \begin{quote}
%   \verb|latex \let\install=y\input{hobsub.dtx}|
% \end{quote}
% Do not forget to quote the argument according to the demands
% of your shell.
%
% \paragraph{Generating the documentation.}
% You can use both the \xfile{.dtx} or the \xfile{.drv} to generate
% the documentation. The process can be configured by the
% configuration file \xfile{ltxdoc.cfg}. For instance, put this
% line into this file, if you want to have A4 as paper format:
% \begin{quote}
%   \verb|\PassOptionsToClass{a4paper}{article}|
% \end{quote}
% An example follows how to generate the
% documentation with pdf\LaTeX:
% \begin{quote}
%\begin{verbatim}
%pdflatex hobsub.dtx
%makeindex -s gind.ist hobsub.idx
%pdflatex hobsub.dtx
%makeindex -s gind.ist hobsub.idx
%pdflatex hobsub.dtx
%\end{verbatim}
% \end{quote}
%
% \section{Catalogue}
%
% The following XML file can be used as source for the
% \href{http://mirror.ctan.org/help/Catalogue/catalogue.html}{\TeX\ Catalogue}.
% The elements \texttt{caption} and \texttt{description} are imported
% from the original XML file from the Catalogue.
% The name of the XML file in the Catalogue is \xfile{hobsub.xml}.
%    \begin{macrocode}
%<*catalogue>
<?xml version='1.0' encoding='us-ascii'?>
<!DOCTYPE entry SYSTEM 'catalogue.dtd'>
<entry datestamp='$Date$' modifier='$Author$' id='hobsub'>
  <name>hobsub</name>
  <caption>Construct package bundles</caption>
  <authorref id='auth:oberdiek'/>
  <copyright owner='Heiko Oberdiek' year='2011,2012'/>
  <license type='lppl1.3'/>
  <version number='1.14'/>
  <description>
    The package offers a means of loading a bundle of (other) packages
    &#x201C;at once&#x201D;, offering a means of reducing the amount of log output
    as the packages are loaded, and an aide-memoire for complicated
    requirements.  In some ways, loading a package via hobsub behaves
    much like loading the package in the normal way; for example,
    reloads are suppressed, as in LaTeX proper, and correct
    <tt>\listfiles</tt> output is produced.
    <p/>
    Examples provided are a &#x2018;generic&#x2019; bundle (comprising the author&#x2019;s
    packages that can be used with Plain TeX) and a &#x2018;hyperref&#x2019; bundle
    (comprising packages useful when generating hypertext output with
    LaTeX).
    <p/>
    The package is part of the <xref refid='oberdiek'>oberdiek</xref> bundle.
  </description>
  <documentation details='Package documentation'
      href='ctan:/macros/latex/contrib/oberdiek/hobsub.pdf'/>
  <ctan file='true' path='/macros/latex/contrib/oberdiek/hobsub.dtx'/>
  <miktex location='oberdiek'/>
  <texlive location='oberdiek'/>
  <install path='/macros/latex/contrib/oberdiek/oberdiek.tds.zip'/>
</entry>
%</catalogue>
%    \end{macrocode}
%
% \begin{thebibliography}{9}
% \raggedright
% \bibitem{lualatex:quiet}
% Will Robertson,
% \textit{[lltx] quietening lualatex console output};
% mailing list \textsf{lualatex-dev at tug.org},
% 2011-01-29;
% \url{http://tug.org/pipermail/lualatex-dev/2011-January/001031.html}.
%
% \end{thebibliography}
%
% \begin{History}
%   \begin{Version}{2011/01/30 v1.0}
%   \item
%     First public version.
%   \end{Version}
%   \begin{Version}{2011/04/16 v1.1}
%   \item
%     Package updates.
%   \end{Version}
%   \begin{Version}{2011/04/17 v1.2}
%   \item
%     White spaces at line begins are removed or reduced
%     in the generated collection packages.
%   \end{Version}
%   \begin{Version}{2011/04/18 v1.3}
%   \item
%     Package updates.
%   \end{Version}
%   \begin{Version}{2011/04/23 v1.4}
%   \item
%     Package updates.
%   \end{Version}
%   \begin{Version}{2011/06/24 v1.5}
%   \item
%     Package updates.
%   \end{Version}
%   \begin{Version}{2011/06/30 v1.6}
%   \item
%     Package updates.
%   \end{Version}
%   \begin{Version}{2011/07/01 v1.7}
%   \item
%     Package updates.
%   \end{Version}
%   \begin{Version}{2011/07/28 v1.8}
%   \item
%     Package updates.
%   \end{Version}
%   \begin{Version}{2011/08/22 v1.9}
%   \item
%     Package updates.
%   \end{Version}
%   \begin{Version}{2011/10/16 v1.10}
%   \item
%     Package updates.
%   \end{Version}
%   \begin{Version}{2011/11/29 v1.11}
%   \item
%     Package updates.
%   \end{Version}
%   \begin{Version}{2012/04/25 v1.12}
%   \item
%     Package updates.
%   \end{Version}
%   \begin{Version}{2012/05/28 v1.13}
%   \item
%     Package updates.
%   \end{Version}
%   \begin{Version}{2016/05/16 v1.14}
%   \item
%     Documentation updates.
%   \end{Version}
% \end{History}
%
% \PrintIndex
%
% \Finale
\endinput

%        (quote the arguments according to the demands of your shell)
%
% Documentation:
%    (a) If hobsub.drv is present:
%           latex hobsub.drv
%    (b) Without hobsub.drv:
%           latex hobsub.dtx; ...
%    The class ltxdoc loads the configuration file ltxdoc.cfg
%    if available. Here you can specify further options, e.g.
%    use A4 as paper format:
%       \PassOptionsToClass{a4paper}{article}
%
%    Programm calls to get the documentation (example):
%       pdflatex hobsub.dtx
%       makeindex -s gind.ist hobsub.idx
%       pdflatex hobsub.dtx
%       makeindex -s gind.ist hobsub.idx
%       pdflatex hobsub.dtx
%
% Installation:
%    TDS:tex/generic/oberdiek/hobsub.sty
%    TDS:tex/generic/oberdiek/hobsub-generic.sty
%    TDS:tex/generic/oberdiek/hobsub-hyperref.sty
%    TDS:doc/latex/oberdiek/hobsub.pdf
%    TDS:doc/latex/oberdiek/test/hobsub-test1.tex
%    TDS:source/latex/oberdiek/hobsub.dtx
%
%<*ignore>
\begingroup
  \catcode123=1 %
  \catcode125=2 %
  \def\x{LaTeX2e}%
\expandafter\endgroup
\ifcase 0\ifx\install y1\fi\expandafter
         \ifx\csname processbatchFile\endcsname\relax\else1\fi
         \ifx\fmtname\x\else 1\fi\relax
\else\csname fi\endcsname
%</ignore>
%<*install>
\input docstrip.tex
\Msg{************************************************************************}
\Msg{* Installation}
\Msg{* Package: hobsub 2016/05/16 v1.14 Construct package bundles (HO)}
\Msg{************************************************************************}

\keepsilent
\askforoverwritefalse

\let\MetaPrefix\relax
\preamble

This is a generated file.

Project: hobsub
Version: 2016/05/16 v1.14

Copyright (C) 2011, 2012 by
   Heiko Oberdiek <heiko.oberdiek at googlemail.com>

This work may be distributed and/or modified under the
conditions of the LaTeX Project Public License, either
version 1.3c of this license or (at your option) any later
version. This version of this license is in
   http://www.latex-project.org/lppl/lppl-1-3c.txt
and the latest version of this license is in
   http://www.latex-project.org/lppl.txt
and version 1.3 or later is part of all distributions of
LaTeX version 2005/12/01 or later.

This work has the LPPL maintenance status "maintained".

This Current Maintainer of this work is Heiko Oberdiek.

The Base Interpreter refers to any `TeX-Format',
because some files are installed in TDS:tex/generic//.

This work consists of the main source file hobsub.dtx
and the derived files
   hobsub.sty, hobsub.pdf, hobsub.ins, hobsub.drv, hobsub-generic.sty,
   hobsub-hyperref.sty, hobsub-test1.tex.

\endpreamble
\let\MetaPrefix\DoubleperCent

\generate{%
  \file{hobsub.ins}{\from{hobsub.dtx}{install}}%
  \file{hobsub.drv}{\from{hobsub.dtx}{driver}}%
  \usedir{tex/generic/oberdiek}%
  \file{hobsub.sty}{\from{hobsub.dtx}{package}}%
  \usedir{doc/latex/oberdiek/test}%
  \file{hobsub-test1.tex}{\from{hobsub.dtx}{test1}}%
  \nopreamble
  \nopostamble
  \usedir{source/latex/oberdiek/catalogue}%
  \file{hobsub.xml}{\from{hobsub.dtx}{catalogue}}%
}
\let\hobsubSavedNormalLine\normalLine
\def\hobsubGobble#1{}
\begingroup
  \lccode`\0=32 % space
  \lccode`\1=35 % hash
  \lccode`\2=37 % percent
  \lccode`\3=92 % backslash
  \lccode`\4=123 % {
  \lccode`\5=125 % }
\lowercase{\endgroup
  \let\hobsubTokenSpaceOther= 0%
  \let\hobsubTokenHash= 1%
  \let\hobsubTokenPercent= 2%
  \let\hobsubTokenBackslash= 3%
  \let\hobsubTokenBraceLeft= 4%
  \let\hobsubTokenBraceRight= 5%
  \def\hobsubNormalLineWithSpace{%
    \hobsubSavedNormalLine0%
  }%
}
\def\normalLine{%
  \futurelet\hobsubToken\hobsubFirst
}
\def\hobsubFirst{%
  \ifx\hobsubToken\hobsubTokenSpaceOther
    \expandafter\expandafter\expandafter\futurelet
    \expandafter\expandafter\expandafter\hobsubToken
    \expandafter\expandafter\expandafter\hobsubNext
    \expandafter\hobsubGobble
  \else
    \expandafter\hobsubSavedNormalLine
  \fi
}
\def\hobsubNext{%
  \ifx\hobsubToken\hobsubTokenSpaceOther
    \expandafter\expandafter\expandafter\futurelet
    \expandafter\expandafter\expandafter\hobsubToken
    \expandafter\expandafter\expandafter\hobsubNext 
    \expandafter\hobsubGobble
  \else
    \let\hobsubWithSpace=Y%
    \ifx\hobsubToken\hobsubTokenBackslash
      \let\hobsubWithSpace=N%
    \else
      \ifx\hobsubToken\hobsubTokenBraceRight
        \let\hobsubWithSpace=N%
      \else
        \ifx\hobsubToken\hobsubTokenBraceLeft
          \let\hobsubWithSpace=N%
        \else
          \ifx\hobsubToken\hobsubTokenPercent
            \let\hobsubWithSpace=N%
          \else
            \ifx\hobsubToken\hobsubTokenHash
              \let\hobsubWithSpace=N%
            \fi
          \fi
        \fi
      \fi
    \fi
    \ifx\hobsubWithSpace Y%
      \expandafter\expandafter\expandafter\hobsubNormalLineWithSpace
    \else
      \expandafter\expandafter\expandafter\hobsubSavedNormalLine
    \fi
  \fi
}
\generate{%
  \usedir{tex/generic/oberdiek}%
  \file{hobsub-generic.sty}{%
    \from{hobsub.dtx}{generic,preamble,infwarerr}%
    \from{infwarerr.dtx}{package}%
    \from{hobsub.dtx}{post,ltxcmds}%
    \from{ltxcmds.dtx}{package}%
    \from{hobsub.dtx}{post,ifluatex}%
    \from{ifluatex.dtx}{package}%
    \from{hobsub.dtx}{post,ifvtex}%
    \from{ifvtex.dtx}{package}%
    \from{hobsub.dtx}{post,intcalc}%
    \from{intcalc.dtx}{package}%
    \from{hobsub.dtx}{post,ifpdf}%
    \from{ifpdf.dtx}{package}%
    \from{hobsub.dtx}{post,etexcmds}%
    \from{etexcmds.dtx}{package}%
    \from{hobsub.dtx}{post,kvsetkeys}%
    \from{kvsetkeys.dtx}{package}%
    \from{hobsub.dtx}{post,kvdefinekeys}%
    \from{kvdefinekeys.dtx}{package}%
    \from{hobsub.dtx}{post,luatex-loader}%
    \from{luatex.dtx}{loader}%
    \from{hobsub.dtx}{post,pdftexcmds}%
    \from{pdftexcmds.dtx}{package}%
    \from{hobsub.dtx}{post,pdfescape}%
    \from{pdfescape.dtx}{package}%
    \from{hobsub.dtx}{post,bigintcalc}%
    \from{bigintcalc.dtx}{package}%
    \from{hobsub.dtx}{post,bitset}%
    \from{bitset.dtx}{package}%
    \from{hobsub.dtx}{post,uniquecounter}%
    \from{uniquecounter.dtx}{package}%
    \from{hobsub.dtx}{post,atend}%
  }%
}
\generate{%
  \usedir{tex/generic/oberdiek}%
  \file{hobsub-hyperref.sty}{%
    \from{hobsub.dtx}{hyperref,preamble,,letltxmacro}%
    \from{letltxmacro.dtx}{package}%
    \from{hobsub.dtx}{post,hopatch}%
    \from{hopatch.dtx}{package}%
    \from{hobsub.dtx}{post,xcolor-patch}%
    \from{hycolor.dtx}{xcolor}%
    \from{hobsub.dtx}{post,atveryend}%
    \from{atveryend.dtx}{package}%
    \from{hobsub.dtx}{post,atbegshi}%
    \from{atbegshi.dtx}{package}%
    \from{hobsub.dtx}{post,refcount}%
    \from{refcount.dtx}{package}%
    \from{hobsub.dtx}{post,hycolor}%
    \from{hycolor.dtx}{package}%
    \from{hobsub.dtx}{post}%
  }%
}
\let\normalLine\hobsubSavedNormalLine

\catcode32=13\relax% active space
\let =\space%
\Msg{************************************************************************}
\Msg{*}
\Msg{* To finish the installation you have to move the following}
\Msg{* files into a directory searched by TeX:}
\Msg{*}
\Msg{*     hobsub.sty, hobsub-generic.sty, hobsub-hyperref.sty}
\Msg{*}
\Msg{* To produce the documentation run the file `hobsub.drv'}
\Msg{* through LaTeX.}
\Msg{*}
\Msg{* Happy TeXing!}
\Msg{*}
\Msg{************************************************************************}

\endbatchfile
%</install>
%<*ignore>
\fi
%</ignore>
%<*driver>
\NeedsTeXFormat{LaTeX2e}
\ProvidesFile{hobsub.drv}%
  [2016/05/16 v1.14 Construct package bundles (HO)]%
\documentclass{ltxdoc}
\usepackage{holtxdoc}[2011/11/22]
\usepackage{array}
\usepackage{caption}
\renewcommand*{\bottomfraction}{.5}
\renewcommand*{\topfraction}{.5}
\begin{document}
  \DocInput{hobsub.dtx}%
\end{document}
%</driver>
% \fi
%
%
% \CharacterTable
%  {Upper-case    \A\B\C\D\E\F\G\H\I\J\K\L\M\N\O\P\Q\R\S\T\U\V\W\X\Y\Z
%   Lower-case    \a\b\c\d\e\f\g\h\i\j\k\l\m\n\o\p\q\r\s\t\u\v\w\x\y\z
%   Digits        \0\1\2\3\4\5\6\7\8\9
%   Exclamation   \!     Double quote  \"     Hash (number) \#
%   Dollar        \$     Percent       \%     Ampersand     \&
%   Acute accent  \'     Left paren    \(     Right paren   \)
%   Asterisk      \*     Plus          \+     Comma         \,
%   Minus         \-     Point         \.     Solidus       \/
%   Colon         \:     Semicolon     \;     Less than     \<
%   Equals        \=     Greater than  \>     Question mark \?
%   Commercial at \@     Left bracket  \[     Backslash     \\
%   Right bracket \]     Circumflex    \^     Underscore    \_
%   Grave accent  \`     Left brace    \{     Vertical bar  \|
%   Right brace   \}     Tilde         \~}
%
% \GetFileInfo{hobsub.drv}
%
% \title{The \xpackage{hobsub} package}
% \date{2016/05/16 v1.14}
% \author{Heiko Oberdiek\thanks
% {Please report any issues at https://github.com/ho-tex/oberdiek/issues}\\
% \xemail{heiko.oberdiek at googlemail.com}}
%
% \maketitle
%
% \begin{abstract}
% Package \xpackage{hobsub} implements the idea of loading
% several packages in one big collection package.
% \end{abstract}
%
% \tableofcontents
%
% \def\TableGeneric{^^A
%   \TableEntry{\xpackage{infwarerr}}{}{2010/04/08 v1.3}^^A
%   \TableEntry{\xpackage{ltxcmds}}{}{2011/11/09 v1.22}^^A
%   \TableEntry{\xpackage{ifluatex}}{}{2010/03/01 v1.3}^^A
%   \TableEntry{\xpackage{ifvtex}}{}{2010/03/01 v1.5}^^A
%   \TableEntry{\xpackage{intcalc}}{}{2007/09/27 v1.1}^^A
%   \TableEntry{\xpackage{ifpdf}}{\xpackage{ifluatex}}{2011/01/30 v2.3}^^A
%   \TableEntry{\xpackage{etexcmds}}{}{2011/02/16 v1.5}^^A
%   \TableEntry{\xpackage{kvsetkeys}}{}{2012/04/25 v1.16}^^A
%   \TableEntry{\xpackage{kvdefinekeys}}{}{2011/04/07 v1.3}^^A
%   \TableEntry{\xpackage{luatex-loader}}{}{2010/03/09 v0.4}^^A
%   \TableEntry{\xpackage{pdftexcmds}}{\xpackage{ifluatex}, \xpackage{ifpdf}, \xpackage{infwarerr}, \xpackage{ltxcmds}, \xpackage{luatex-loader}}{2011/11/29 v0.20}^^A
%   \TableEntry{\xpackage{pdfescape}}{\xpackage{ltxcmds}, \xpackage{pdftexcmds}}{2011/11/25 v1.13}^^A
%   \TableEntry{\xpackage{bigintcalc}}{\xpackage{pdftexcmds}}{2012/04/08 v1.3}^^A
%   \TableEntry{\xpackage{bitset}}{\xpackage{bigintcalc}, \xpackage{infwarerr}, \xpackage{intcalc}}{2011/01/30 v1.1}^^A
%   \TableEntry{\xpackage{uniquecounter}}{\xpackage{bigintcalc}, \xpackage{infwarerr}}{2011/01/30 v1.2}^^A
% }
% \def\TableHyperref{^^A
%   \TableEntry{\xpackage{letltxmacro}}{}{2010/09/02 v1.4}^^A
%   \TableEntry{\xpackage{hopatch}}{}{2016/05/16 v1.2}^^A
%   \TableEntry{\xpackage{xcolor-patch}}{}{2011/01/30 v1.7}^^A
%   \TableEntry{\xpackage{atveryend}}{}{2011/06/30 v1.8}^^A
%   \TableEntry{\xpackage{atbegshi}}{\xpackage{ifpdf}, \xpackage{infwarerr}, \xpackage{ltxcmds}}{2011/10/05 v1.16}^^A
%   \TableEntry{\xpackage{refcount}}{\xpackage{infwarerr}, \xpackage{ltxcmds}}{2011/10/16 v3.4}^^A
%   \TableEntry{\xpackage{hycolor}}{\xpackage{xcolor-patch}}{2011/01/30 v1.7}^^A
% }
%
% \newenvironment{TableList}[2][htbp]{^^A
%   \begin{table}[{#1}]^^A
%   \centering
%   \caption{Overview \xpackage{hobsub-#2}}^^A
%   \label{tab:#2}^^A
%   \begin{tabular}{ll>{\raggedright}p{.4\linewidth}}^^A
%   \bfseries Package & \bfseries Release &
%   \bfseries Dependencies\tabularnewline
%   \hline
% }{^^A
%   \end{tabular}^^A
%   \end{table}^^A
% }
% \newcommand*{\TableEntry}[3]{%
%   #1&#3&#2\tabularnewline
% }
%
% \renewcommand*{\arraystretch}{1.1}
% \setlength{\extrarowheight}{2pt}
%
% \section{Documentation}
%
% In January 2011 the mailing list lualatex-dev discussed
% ``quietening lualatex console output'' \cite{lualatex:quiet}.
% Inspired from this, I implemented this package \xpackage{hobsub}
% to load several packages in one file. The package is applied for
% two packages collections that are subsets of my package bundle
% \xpackage{oberdiek}.
% \begin{description}
% \item[\xpackage{hobsub-generic}] is a collection of packages
%   that can also be used with \hologo{plainTeX}.
% \item[\xpackage{hobsub-hyperref}] is a collection of packages
%   that is used by package \xpackage{hyperref}.
% \end{description}
%
% Package \xpackage{hobsub} provides the macros that
% are used in the collection packages and might be used
% in other projects as well. But there are many caveats,
% some of them are discussed below.
%
% The most serious problem is consistency. A package
% that is available as standalone package and is part
% of a collection might have different versions.
% Because my packages are organized in a bundle and should
% be installed and updated as bundle, the risk seems low
% to me. However, if a single package is updated manually, for
% example for testing, then it should be loaded before the
% collection that contains the package with an older version.
% In a collection, the package will not be loaded again if it
% is already loaded.
%
% \subsection{Usage}
%
% The collection packages \xpackage{hobsub-generic} and
% \xpackage{hobsub-hyperref} are loaded as usual:
% \begin{quote}
%  \begin{tabular}{@{}ll@{}}
%    |\usepackage{hobsub-generic}| & \hologo{LaTeX}\\
%    |\input hobsub-generic.sty| & \hologo{plainTeX}
%  \end{tabular}
% \end{quote}
% or
% \begin{quote}|\usepackage{hobsub-hyperref}|\end{quote}
% Of course, both \cs{usepackage} or \cs{RequirePackage}
% can be used.
%
% If you need requirements on the version date of a package
% inside a collection, then specify them afterwards, for
% example:
% \begin{quote}
%   |\usepackage{hobsub-generic}|\\
%   |\usepackage{ltxcmds}[2010/12/04]% |^^A
%   {\itshape |\ltx@ifblank is needed|}
% \end{quote}
%
% Also it is not garanteed that a package will be always
% part of a collection. Therefore it does not harm,
% to make the requirements explicite \emph{after}
% a collection is loaded, see the previous example.
% \hologo{LaTeX} knows that a
% package is already loaded and does not load it again.
%
% \subsection{Supported features}
%
% \begin{description}
% \item[No reloads:] If a package is already loaded then it
%  is not reloaded again if it is part of a collection package.
% \item[\cs{listfiles}:]
%   Each package that is loaded in a collection is also
%   added to the output of \cs{listfiles}.
% \item[\cs{@currname}, \cs{@currext}]
%   are updated, they are used by \hologo{LaTeX}'s
%  \cs{ProvidesPackage}.
% \item[\cs{AtEndOfPackage}:]
%   \hologo{LaTeX}'s hook mechanism that is used for
%   \cs{AtEndOfPackage} is supported. However none of the
%   packages in the collections are using \cs{AtEndOfPackage},
%   therefore this feature is completely untested.
% \end{description}
%
% \subsection{Common limitations}
%
% \begin{description}
% \item[\cs{endinput}:]
%   Often a package is using \cs{endinput}, either added by
%   \xpackage{docstrip}, or to stop package loading at an earlier
%   time. Therefore \cs{endinput} must be catched during package
%   loading to prevent that the collection package is closed
%   too early before reading all packages it contains.
%
%   This redefinition of \cs{endinput} fails if the package loads
%   other files that contain \cs{endinput}. To some degree it could
%   be catched by redefining macros that load files. But this will
%   not work for the primitive \cs{input}.
% \item[Options:]
%   \hologo{LaTeX}'s package options are not supported.
%   Therefore the packages that are put in a collection must
%   not support options.
% \item[Hook packages:]
%   Some packages and classes provides hooks for packages, see
%   packages \xpackage{scrlfile}, \xpackage{filehook}
%   or class \xclass{memoir}. These hooks are not supported
%   for embedded packages inside a collection package.
%   Because of the problem with \cs{endinput}, these hooks
%   are not supported. The hooks might contain code that
%   loads other files that execute \cs{endinput}.
% \end{description}
%
% \subsection{Package \xpackage{hobsub-generic}}
%
% \begin{TableList}[bp]{generic}\TableGeneric\end{TableList}
%
% Collection `generic' contains some of the \hologo{plainTeX} compatible
% packages of my bundle, see table \ref{tab:generic}.
% It adds the following restrictions to any package
% that is part of this collection:
% \begin{description}
% \item[Generic formats:] The following formats must be supported:
%   \hologo{LaTeX}, \hologo{plainTeX} and even \hologo{iniTeX}.
%   This applies to the collection package \xpackage{hobsub-generic}
%   the same way.
% \item[Prefixed macros:]
%   Any package of this collection should only define macros with
%   prefixed names to avoid name clashes. Resources like registers
%   must not be allocated and other global states should not be
%   changed except for added macro definitions.
% \end{description}
% Thus the package \xpackage{hobsub-generic} can also preloaded
% in the format file with \hologo{iniTeX}.
%
% \subsection{Package \xpackage{hobsub-hyperref}}
%
% \begin{TableList}[tp]{hyperref}\TableHyperref\end{TableList}
%
% Collection `hyperref' contains additional packages that are
% used by package \xpackage{hyperref} and that does not fit
% in collection `generic', see table{tab:hyperref}.
% The latter collection package
% \xpackage{hobsub-generic} is loaded via a conventional
% \cs{RequirePackage}.
%
% Because most of the packages in the collection `hyperref'
% needs \hologo{LaTeX}, the collection requires \hologo{LaTeX}.
%
% \subsection{Package \xpackage{hobsub}}
%
% This section is for advanced users that might want to
% build a collection package. Then it can be done by
% using \xpackage{hobsub}:
% \begin{quote}
%   \dots\\
%   |\RequirePackage{hobsub}[2016/05/16]|\\
%   \dots\\
%   |\hobsub@StartPackage{foo1}|\\
%   \textit{\dots\unskip\ package code of \xfile{foo1.sty} \dots}\\
%   |\endinput|\\
%   |<hobsub>|\\
%   |\hobsub@StopPackage|\\
%   |\hobsub@StartPackage{bar2}|\\
%   \textit{\dots\unskip\ package code of \xfile{bar2.sty} \dots}\\
%   |\endinput|\\
%   |<hobsub>|\\
%   |\hobsub@StopPackage|
%   \dots
% \end{quote}
% But be aware of the caveats and limitations, see above.
% Only quite a small portion of the packages can be embedded
% like this.
%
% \clearpage
%
% \StopEventually{
% }
%
% \section{Implementation}
%
% \subsection{Reload check and package identification}
%
% \subsubsection{Package \xpackage{hobsub}}
%
%    \begin{macrocode}
%<*package>
%    \end{macrocode}
%    Reload check, especially if the package is not used with \LaTeX.
%    \begin{macrocode}
\begingroup\catcode61\catcode48\catcode32=10\relax%
  \catcode13=5 % ^^M
  \endlinechar=13 %
  \catcode35=6 % #
  \catcode39=12 % '
  \catcode44=12 % ,
  \catcode45=12 % -
  \catcode46=12 % .
  \catcode58=12 % :
  \catcode64=11 % @
  \catcode123=1 % {
  \catcode125=2 % }
  \expandafter\let\expandafter\x\csname ver@hobsub.sty\endcsname
  \ifx\x\relax % plain-TeX, first loading
  \else
    \def\empty{}%
    \ifx\x\empty % LaTeX, first loading,
      % variable is initialized, but \ProvidesPackage not yet seen
    \else
      \expandafter\ifx\csname PackageInfo\endcsname\relax
        \def\x#1#2{%
          \immediate\write-1{Package #1 Info: #2.}%
        }%
      \else
        \def\x#1#2{\PackageInfo{#1}{#2, stopped}}%
      \fi
      \x{hobsub}{The package is already loaded}%
      \aftergroup\endinput
    \fi
  \fi
\endgroup%
%    \end{macrocode}
%    Package identification:
%    \begin{macrocode}
\begingroup\catcode61\catcode48\catcode32=10\relax%
  \catcode13=5 % ^^M
  \endlinechar=13 %
  \catcode35=6 % #
  \catcode39=12 % '
  \catcode40=12 % (
  \catcode41=12 % )
  \catcode44=12 % ,
  \catcode45=12 % -
  \catcode46=12 % .
  \catcode47=12 % /
  \catcode58=12 % :
  \catcode64=11 % @
  \catcode91=12 % [
  \catcode93=12 % ]
  \catcode123=1 % {
  \catcode125=2 % }
  \expandafter\ifx\csname ProvidesPackage\endcsname\relax
    \def\x#1#2#3[#4]{\endgroup
      \immediate\write-1{Package: #3 #4}%
      \xdef#1{#4}%
    }%
  \else
    \def\x#1#2[#3]{\endgroup
      #2[{#3}]%
      \ifx#1\@undefined
        \xdef#1{#3}%
      \fi
      \ifx#1\relax
        \xdef#1{#3}%
      \fi
    }%
  \fi
\expandafter\x\csname ver@hobsub.sty\endcsname
\ProvidesPackage{hobsub}%
  [2016/05/16 v1.14 Construct package bundles (HO)]%
%    \end{macrocode}
%    \begin{macrocode}
%</package>
%    \end{macrocode}
%
% \subsubsection{Package \xpackage{hobsub-generic}}
%
%    \begin{macrocode}
%<*generic>
%    \end{macrocode}
%    Reload check, especially if the package is not used with \LaTeX.
%    \begin{macrocode}
\begingroup\catcode61\catcode48\catcode32=10\relax%
  \catcode13=5 % ^^M
  \endlinechar=13 %
  \catcode35=6 % #
  \catcode39=12 % '
  \catcode44=12 % ,
  \catcode45=12 % -
  \catcode46=12 % .
  \catcode58=12 % :
  \catcode64=11 % @
  \catcode123=1 % {
  \catcode125=2 % }
  \expandafter\let\expandafter\x\csname ver@hobsub-generic.sty\endcsname
  \ifx\x\relax % plain-TeX, first loading
  \else
    \def\empty{}%
    \ifx\x\empty % LaTeX, first loading,
      % variable is initialized, but \ProvidesPackage not yet seen
    \else
      \expandafter\ifx\csname PackageInfo\endcsname\relax
        \def\x#1#2{%
          \immediate\write-1{Package #1 Info: #2.}%
        }%
      \else
        \def\x#1#2{\PackageInfo{#1}{#2, stopped}}%
      \fi
      \x{hobsub-generic}{The package is already loaded}%
      \aftergroup\endinput
    \fi
  \fi
\endgroup%
%    \end{macrocode}
%    Package identification:
%    \begin{macrocode}
\begingroup\catcode61\catcode48\catcode32=10\relax%
  \catcode13=5 % ^^M
  \endlinechar=13 %
  \catcode35=6 % #
  \catcode39=12 % '
  \catcode40=12 % (
  \catcode41=12 % )
  \catcode44=12 % ,
  \catcode45=12 % -
  \catcode46=12 % .
  \catcode47=12 % /
  \catcode58=12 % :
  \catcode64=11 % @
  \catcode91=12 % [
  \catcode93=12 % ]
  \catcode123=1 % {
  \catcode125=2 % }
  \expandafter\ifx\csname ProvidesPackage\endcsname\relax
    \def\x#1#2#3[#4]{\endgroup
      \immediate\write-1{Package: #3 #4}%
      \xdef#1{#4}%
    }%
  \else
    \def\x#1#2[#3]{\endgroup
      #2[{#3}]%
      \ifx#1\@undefined
        \xdef#1{#3}%
      \fi
      \ifx#1\relax
        \xdef#1{#3}%
      \fi
    }%
  \fi
\expandafter\x\csname ver@hobsub-generic.sty\endcsname
\ProvidesPackage{hobsub-generic}%
  [2016/05/16 v1.14 Bundle oberdiek, subset generic (HO)]%
%    \end{macrocode}
%    \begin{macrocode}
%</generic>
%    \end{macrocode}
%
% \subsubsection{Package \xpackage{hobsub-hyperref}}
%
%    \begin{macrocode}
%<*hyperref>
%    \end{macrocode}
%    Reload check, especially if the package is not used with \LaTeX.
%    \begin{macrocode}
\begingroup\catcode61\catcode48\catcode32=10\relax%
  \catcode13=5 % ^^M
  \endlinechar=13 %
  \catcode35=6 % #
  \catcode39=12 % '
  \catcode44=12 % ,
  \catcode45=12 % -
  \catcode46=12 % .
  \catcode58=12 % :
  \catcode64=11 % @
  \catcode123=1 % {
  \catcode125=2 % }
  \expandafter\let\expandafter\x\csname ver@hobsub-hyperref.sty\endcsname
  \ifx\x\relax % plain-TeX, first loading
  \else
    \def\empty{}%
    \ifx\x\empty % LaTeX, first loading,
      % variable is initialized, but \ProvidesPackage not yet seen
    \else
      \expandafter\ifx\csname PackageInfo\endcsname\relax
        \def\x#1#2{%
          \immediate\write-1{Package #1 Info: #2.}%
        }%
      \else
        \def\x#1#2{\PackageInfo{#1}{#2, stopped}}%
      \fi
      \x{hobsub-hyperref}{The package is already loaded}%
      \aftergroup\endinput
    \fi
  \fi
\endgroup%
%    \end{macrocode}
%    Package identification:
%    \begin{macrocode}
\begingroup\catcode61\catcode48\catcode32=10\relax%
  \catcode13=5 % ^^M
  \endlinechar=13 %
  \catcode35=6 % #
  \catcode39=12 % '
  \catcode40=12 % (
  \catcode41=12 % )
  \catcode44=12 % ,
  \catcode45=12 % -
  \catcode46=12 % .
  \catcode47=12 % /
  \catcode58=12 % :
  \catcode64=11 % @
  \catcode91=12 % [
  \catcode93=12 % ]
  \catcode123=1 % {
  \catcode125=2 % }
  \expandafter\ifx\csname ProvidesPackage\endcsname\relax
    \def\x#1#2#3[#4]{\endgroup
      \immediate\write-1{Package: #3 #4}%
      \xdef#1{#4}%
    }%
  \else
    \def\x#1#2[#3]{\endgroup
      #2[{#3}]%
      \ifx#1\@undefined
        \xdef#1{#3}%
      \fi
      \ifx#1\relax
        \xdef#1{#3}%
      \fi
    }%
  \fi
\expandafter\x\csname ver@hobsub-hyperref.sty\endcsname
\ProvidesPackage{hobsub-hyperref}%
  [2016/05/16 v1.14 Bundle oberdiek, subset hyperref (HO)]%
%    \end{macrocode}
%    \begin{macrocode}
%</hyperref>
%    \end{macrocode}
%
% \subsection{Catcodes}
%
%    \begin{macrocode}
%<*package|generic|hyperref>
\begingroup\catcode61\catcode48\catcode32=10\relax%
  \catcode13=5 % ^^M
  \endlinechar=13 %
  \catcode123=1 % {
  \catcode125=2 % }
  \catcode64=11 % @
  \def\x{\endgroup
    \expandafter\edef\csname HOBsub@AtEnd\endcsname{%
      \endlinechar=\the\endlinechar\relax
      \catcode13=\the\catcode13\relax
      \catcode32=\the\catcode32\relax
      \catcode35=\the\catcode35\relax
      \catcode61=\the\catcode61\relax
      \catcode64=\the\catcode64\relax
      \catcode123=\the\catcode123\relax
      \catcode125=\the\catcode125\relax
    }%
  }%
\x\catcode61\catcode48\catcode32=10\relax%
\catcode13=5 % ^^M
\endlinechar=13 %
\catcode35=6 % #
\catcode64=11 % @
\catcode123=1 % {
\catcode125=2 % }
\def\TMP@EnsureCode#1#2{%
  \edef\HOBsub@AtEnd{%
    \HOBsub@AtEnd
    \catcode#1=\the\catcode#1\relax
  }%
  \catcode#1=#2\relax
}
\TMP@EnsureCode{39}{12}% '
\TMP@EnsureCode{40}{12}% (
\TMP@EnsureCode{41}{12}% )
\TMP@EnsureCode{45}{12}% -
\TMP@EnsureCode{46}{12}% .
\TMP@EnsureCode{47}{12}% /
\TMP@EnsureCode{58}{12}% :
\TMP@EnsureCode{60}{12}% <
\TMP@EnsureCode{62}{12}% >
\TMP@EnsureCode{96}{12}% `
\edef\HOBsub@AtEnd{\HOBsub@AtEnd\noexpand\endinput}
%    \end{macrocode}
%
%    \begin{macrocode}
%</package|generic|hyperref>
%    \end{macrocode}
%
% \subsection{Package \xpackage{hobsub-hyperref} loads \xpackage{hobsub-generic}}
%
%    \begin{macrocode}
%<*hyperref>
\NeedsTeXFormat{LaTeX2e}
\RequirePackage{hobsub-generic}[2016/05/16]
%</hyperref>
%    \end{macrocode}
%
% \subsection{Preamble}
%
%    \begin{macrocode}
%<*preamble>
\begingroup\expandafter\expandafter\expandafter\endgroup
\expandafter\ifx\csname ver@hobsub.sty\endcsname\relax
\else
  \expandafter\hobsub@GobbleRemainingPackage
\fi
%</preamble>
%    \end{macrocode}
%
%
%    \begin{macrocode}
%<*package|preamble>
%    \end{macrocode}
%
%    \begin{macro}{\HOBsub@OrgEndinput}
%    \begin{macrocode}
\begingroup\expandafter\expandafter\expandafter\endgroup
\expandafter\ifx\csname HOBsub@OrgEndinput\endcsname\relax
  \let\HOBsub@OrgEndinput\endinput
\fi
%    \end{macrocode}
%    \end{macro}
%    \begin{macro}{\hobsub@GobbleRemainingPackage}
%    \begin{macrocode}
\def\hobsub@GobbleRemainingPackage{%
  \begingroup
  \catcode92=14 % backslash: comment
  \catcode32=14 % space: comment
  \catcode35=14 % hash: comment
  \catcode123=14 % left brace: comment
  \catcode125=14 % right brace: comment
  \catcode60=3 % less: math
  \catcode62=4 % greater: align
  \endlinechar=-1 %
  \HOBsub@GobbleRemainingPackage
}
%    \end{macrocode}
%    \end{macro}
%    \begin{macro}{\HOBsub@GobbleRemainingPackage}
%    \begin{macrocode}
\catcode60=3 % less: dollar
\catcode62=4 % greater: align
\long\def\HOBsub@GobbleRemainingPackage#1<hobsub>{%
  \endgroup
}
\catcode60=12 % less: other
\catcode62=12 % greater: other
%    \end{macrocode}
%    \end{macro}
%
%    \begin{macro}{\hobsub@StartPackage}
%    \begin{macrocode}
\def\hobsub@StartPackage#1{%
  \begingroup\expandafter\expandafter\expandafter\endgroup
  \expandafter\ifx\csname ver@#1.sty\endcsname\relax
    \let\HOBsub@OrgCurrName\@currname
    \let\HOBsub@OrgCurrExt\@currext
    \csname @pushfilename\endcsname
    \def\@currname{#1}%
    \def\@currext{sty}%
    \expandafter\def\csname\@currname.\@currext-h@@k\endcsname{}%
    \let\endinput\hobsub@GobbleRemainingPackage
    \def\hobsub@StopPackage{%
      \let\hobsub@StopPackage\relax
      \HOBsub@StopPackage
%     \hobsub@Info{hobsub}{Package `#1' loaded}%
    }%
    \hobsub@AddToFileList{#1.sty}%
  \else
    \hobsub@Info{hobsub}{Skipping package `#1' (already loaded)}%
    \let\hobsub@StopPackage\relax
    \expandafter\hobsub@GobbleRemainingPackage
  \fi
}
%    \end{macrocode}
%    \end{macro}
%
%    \begin{macro}{\hobsub@StopPackage}
%    \begin{macrocode}
\let\hobsub@StopPackage\relax
%    \end{macrocode}
%    \end{macro}
%
%    \begin{macro}{\hobsub@Info}
%    \begin{macrocode}
\def\hobsub@Info#1#2{%
  \begingroup\expandafter\expandafter\expandafter\endgroup
  \expandafter\ifx\csname @PackageInfoNoLine\endcsname\relax
    \immediate\write-1{Package #1 Info: #2.}%
  \else
    \let\hobsub@Info\@PackageInfoNoLine
    \hobsub@Info{#1}{#2}%
  \fi
}
%    \end{macrocode}
%    \end{macro}
%
%    \begin{macro}{\HOBsub@StopPackage}
%    \begin{macrocode}
\def\HOBsub@StopPackage{%
  \csname\@currname.\@currext-h@@k\endcsname
  \let\endinput\HOBsub@OrgEndinput
  \csname @popfilename\endcsname
  \let\@currname\HOBsub@OrgCurrName
  \let\@currext\HOBsub@OrgCurrExt
}
%    \end{macrocode}
%    \end{macro}
%
%    \begin{macro}{\hobsub@AddToFileList}
%    \begin{macrocode}
\begingroup\expandafter\expandafter\expandafter\endgroup
\expandafter\ifx\csname @addtofilelist\endcsname\relax
  \def\hobsub@AddToFileList#1{}%
\else
  \def\hobsub@AddToFileList#1{%
    \@addtofilelist{#1}%
  }%
\fi
%    \end{macrocode}
%    \end{macro}
%
%    \begin{macrocode}
%</package|preamble>
%    \end{macrocode}
%
% \subsection{End of preamble for subset packages}
%
%    Add the end marker for possible skipping of previous
%    macros, if package \xpackage{hobsub} was already loaded.
%    \begin{macrocode}
%<*preamble>
\hobsub@GobbleRemainingPackage
<hobsub>
%    \end{macrocode}
%
%    All macros of package \xpackage{hobsub} are now available.
%    Now we officially load the package \xpackage{hobsub} inline
%    for proper package identification.
%    \begin{macrocode}
\hobsub@StartPackage{hobsub}%
\begingroup\expandafter\expandafter\expandafter\endgroup
\expandafter\ifx\csname ProvidesPackage\endcsname\relax
  \immediate\write-1{%
    Package: hobsub 2016/05/16 v1.14 Construct package bundles (HO)%
  }%
\else
  \ProvidesPackage{hobsub}%
  [2016/05/16 v1.14 Construct package bundles (HO)]%
\fi
%</preamble>
%    \end{macrocode}
%
%
%    \begin{macrocode}
%<*post|preamble>
 \endinput
<hobsub>
\hobsub@StopPackage
%</post|preamble>
%    \end{macrocode}
%
% \subsection{Package list}
%
%    \begin{macrocode}
%<infwarerr>\hobsub@StartPackage{infwarerr}
%<ltxcmds>\hobsub@StartPackage{ltxcmds}
%<ifluatex>\hobsub@StartPackage{ifluatex}
%<ifvtex>\hobsub@StartPackage{ifvtex}
%<intcalc>\hobsub@StartPackage{intcalc}
%<ifpdf>\hobsub@StartPackage{ifpdf}
%<etexcmds>\hobsub@StartPackage{etexcmds}
%<kvsetkeys>\hobsub@StartPackage{kvsetkeys}
%<kvdefinekeys>\hobsub@StartPackage{kvdefinekeys}
%<*luatex-loader>
\ifluatex
\else
  \expandafter\hobsub@GobbleRemainingPackage
\fi
\hobsub@StartPackage{luatex-loader}
%</luatex-loader>
%<pdftexcmds>\hobsub@StartPackage{pdftexcmds}
%<pdfescape>\hobsub@StartPackage{pdfescape}
%<bigintcalc>\hobsub@StartPackage{bigintcalc}
%<bitset>\hobsub@StartPackage{bitset}
%<uniquecounter>\hobsub@StartPackage{uniquecounter}
%<letltxmacro>\hobsub@StartPackage{letltxmacro}
%<hopatch>\hobsub@StartPackage{hopatch}
%<xcolor-patch>\hobsub@StartPackage{xcolor-patch}
%<atveryend>\hobsub@StartPackage{atveryend}
%<atbegshi>\hobsub@StartPackage{atbegshi}
%<refcount>\hobsub@StartPackage{refcount}
%<hycolor>\hobsub@StartPackage{hycolor}
%    \end{macrocode}
%
% \subsection{End of packages}
%
%    \begin{macrocode}
%<package|atend>\HOBsub@AtEnd%
%    \end{macrocode}
%
% \section{Test}
%
% \subsection{Catcode checks for loading}
%
%    \begin{macrocode}
%<*test1>
%    \end{macrocode}
%    \begin{macrocode}
\catcode`\{=1 %
\catcode`\}=2 %
\catcode`\#=6 %
\catcode`\@=11 %
\expandafter\ifx\csname count@\endcsname\relax
  \countdef\count@=255 %
\fi
\expandafter\ifx\csname @gobble\endcsname\relax
  \long\def\@gobble#1{}%
\fi
\expandafter\ifx\csname @firstofone\endcsname\relax
  \long\def\@firstofone#1{#1}%
\fi
\expandafter\ifx\csname loop\endcsname\relax
  \expandafter\@firstofone
\else
  \expandafter\@gobble
\fi
{%
  \def\loop#1\repeat{%
    \def\body{#1}%
    \iterate
  }%
  \def\iterate{%
    \body
      \let\next\iterate
    \else
      \let\next\relax
    \fi
    \next
  }%
  \let\repeat=\fi
}%
\def\RestoreCatcodes{}
\count@=0 %
\loop
  \edef\RestoreCatcodes{%
    \RestoreCatcodes
    \catcode\the\count@=\the\catcode\count@\relax
  }%
\ifnum\count@<255 %
  \advance\count@ 1 %
\repeat

\def\RangeCatcodeInvalid#1#2{%
  \count@=#1\relax
  \loop
    \catcode\count@=15 %
  \ifnum\count@<#2\relax
    \advance\count@ 1 %
  \repeat
}
\def\RangeCatcodeCheck#1#2#3{%
  \count@=#1\relax
  \loop
    \ifnum#3=\catcode\count@
    \else
      \errmessage{%
        Character \the\count@\space
        with wrong catcode \the\catcode\count@\space
        instead of \number#3%
      }%
    \fi
  \ifnum\count@<#2\relax
    \advance\count@ 1 %
  \repeat
}
\def\space{ }
\expandafter\ifx\csname LoadCommand\endcsname\relax
  \def\LoadCommand{\input hobsub.sty\relax}%
\fi
\def\Test{%
  \RangeCatcodeInvalid{0}{47}%
  \RangeCatcodeInvalid{58}{64}%
  \RangeCatcodeInvalid{91}{96}%
  \RangeCatcodeInvalid{123}{255}%
  \catcode`\@=12 %
  \catcode`\\=0 %
  \catcode`\%=14 %
  \LoadCommand
  \RangeCatcodeCheck{0}{36}{15}%
  \RangeCatcodeCheck{37}{37}{14}%
  \RangeCatcodeCheck{38}{47}{15}%
  \RangeCatcodeCheck{48}{57}{12}%
  \RangeCatcodeCheck{58}{63}{15}%
  \RangeCatcodeCheck{64}{64}{12}%
  \RangeCatcodeCheck{65}{90}{11}%
  \RangeCatcodeCheck{91}{91}{15}%
  \RangeCatcodeCheck{92}{92}{0}%
  \RangeCatcodeCheck{93}{96}{15}%
  \RangeCatcodeCheck{97}{122}{11}%
  \RangeCatcodeCheck{123}{255}{15}%
  \RestoreCatcodes
}
\Test
\csname @@end\endcsname
\end
%    \end{macrocode}
%    \begin{macrocode}
%</test1>
%    \end{macrocode}
%
% \section{Installation}
%
% \subsection{Download}
%
% \paragraph{Package.} This package is available on
% CTAN\footnote{\url{http://ctan.org/pkg/hobsub}}:
% \begin{description}
% \item[\CTAN{macros/latex/contrib/oberdiek/hobsub.dtx}] The source file.
% \item[\CTAN{macros/latex/contrib/oberdiek/hobsub.pdf}] Documentation.
% \end{description}
%
%
% \paragraph{Bundle.} All the packages of the bundle `oberdiek'
% are also available in a TDS compliant ZIP archive. There
% the packages are already unpacked and the documentation files
% are generated. The files and directories obey the TDS standard.
% \begin{description}
% \item[\CTAN{install/macros/latex/contrib/oberdiek.tds.zip}]
% \end{description}
% \emph{TDS} refers to the standard ``A Directory Structure
% for \TeX\ Files'' (\CTAN{tds/tds.pdf}). Directories
% with \xfile{texmf} in their name are usually organized this way.
%
% \subsection{Bundle installation}
%
% \paragraph{Unpacking.} Unpack the \xfile{oberdiek.tds.zip} in the
% TDS tree (also known as \xfile{texmf} tree) of your choice.
% Example (linux):
% \begin{quote}
%   |unzip oberdiek.tds.zip -d ~/texmf|
% \end{quote}
%
% \paragraph{Script installation.}
% Check the directory \xfile{TDS:scripts/oberdiek/} for
% scripts that need further installation steps.
% Package \xpackage{attachfile2} comes with the Perl script
% \xfile{pdfatfi.pl} that should be installed in such a way
% that it can be called as \texttt{pdfatfi}.
% Example (linux):
% \begin{quote}
%   |chmod +x scripts/oberdiek/pdfatfi.pl|\\
%   |cp scripts/oberdiek/pdfatfi.pl /usr/local/bin/|
% \end{quote}
%
% \subsection{Package installation}
%
% \paragraph{Unpacking.} The \xfile{.dtx} file is a self-extracting
% \docstrip\ archive. The files are extracted by running the
% \xfile{.dtx} through \plainTeX:
% \begin{quote}
%   \verb|tex hobsub.dtx|
% \end{quote}
%
% \paragraph{TDS.} Now the different files must be moved into
% the different directories in your installation TDS tree
% (also known as \xfile{texmf} tree):
% \begin{quote}
% \def\t{^^A
% \begin{tabular}{@{}>{\ttfamily}l@{ $\rightarrow$ }>{\ttfamily}l@{}}
%   hobsub.sty & tex/generic/oberdiek/hobsub.sty\\
%   hobsub-generic.sty & tex/generic/oberdiek/hobsub-generic.sty\\
%   hobsub-hyperref.sty & tex/generic/oberdiek/hobsub-hyperref.sty\\
%   hobsub.pdf & doc/latex/oberdiek/hobsub.pdf\\
%   test/hobsub-test1.tex & doc/latex/oberdiek/test/hobsub-test1.tex\\
%   hobsub.dtx & source/latex/oberdiek/hobsub.dtx\\
% \end{tabular}^^A
% }^^A
% \sbox0{\t}^^A
% \ifdim\wd0>\linewidth
%   \begingroup
%     \advance\linewidth by\leftmargin
%     \advance\linewidth by\rightmargin
%   \edef\x{\endgroup
%     \def\noexpand\lw{\the\linewidth}^^A
%   }\x
%   \def\lwbox{^^A
%     \leavevmode
%     \hbox to \linewidth{^^A
%       \kern-\leftmargin\relax
%       \hss
%       \usebox0
%       \hss
%       \kern-\rightmargin\relax
%     }^^A
%   }^^A
%   \ifdim\wd0>\lw
%     \sbox0{\small\t}^^A
%     \ifdim\wd0>\linewidth
%       \ifdim\wd0>\lw
%         \sbox0{\footnotesize\t}^^A
%         \ifdim\wd0>\linewidth
%           \ifdim\wd0>\lw
%             \sbox0{\scriptsize\t}^^A
%             \ifdim\wd0>\linewidth
%               \ifdim\wd0>\lw
%                 \sbox0{\tiny\t}^^A
%                 \ifdim\wd0>\linewidth
%                   \lwbox
%                 \else
%                   \usebox0
%                 \fi
%               \else
%                 \lwbox
%               \fi
%             \else
%               \usebox0
%             \fi
%           \else
%             \lwbox
%           \fi
%         \else
%           \usebox0
%         \fi
%       \else
%         \lwbox
%       \fi
%     \else
%       \usebox0
%     \fi
%   \else
%     \lwbox
%   \fi
% \else
%   \usebox0
% \fi
% \end{quote}
% If you have a \xfile{docstrip.cfg} that configures and enables \docstrip's
% TDS installing feature, then some files can already be in the right
% place, see the documentation of \docstrip.
%
% \subsection{Refresh file name databases}
%
% If your \TeX~distribution
% (\teTeX, \mikTeX, \dots) relies on file name databases, you must refresh
% these. For example, \teTeX\ users run \verb|texhash| or
% \verb|mktexlsr|.
%
% \subsection{Some details for the interested}
%
% \paragraph{Attached source.}
%
% The PDF documentation on CTAN also includes the
% \xfile{.dtx} source file. It can be extracted by
% AcrobatReader 6 or higher. Another option is \textsf{pdftk},
% e.g. unpack the file into the current directory:
% \begin{quote}
%   \verb|pdftk hobsub.pdf unpack_files output .|
% \end{quote}
%
% \paragraph{Unpacking with \LaTeX.}
% The \xfile{.dtx} chooses its action depending on the format:
% \begin{description}
% \item[\plainTeX:] Run \docstrip\ and extract the files.
% \item[\LaTeX:] Generate the documentation.
% \end{description}
% If you insist on using \LaTeX\ for \docstrip\ (really,
% \docstrip\ does not need \LaTeX), then inform the autodetect routine
% about your intention:
% \begin{quote}
%   \verb|latex \let\install=y% \iffalse meta-comment
%
% File: hobsub.dtx
% Version: 2016/05/16 v1.14
% Info: Construct package bundles
%
% Copyright (C) 2011, 2012 by
%    Heiko Oberdiek <heiko.oberdiek at googlemail.com>
%    2016
%    https://github.com/ho-tex/oberdiek/issues
%
% This work may be distributed and/or modified under the
% conditions of the LaTeX Project Public License, either
% version 1.3c of this license or (at your option) any later
% version. This version of this license is in
%    http://www.latex-project.org/lppl/lppl-1-3c.txt
% and the latest version of this license is in
%    http://www.latex-project.org/lppl.txt
% and version 1.3 or later is part of all distributions of
% LaTeX version 2005/12/01 or later.
%
% This work has the LPPL maintenance status "maintained".
%
% This Current Maintainer of this work is Heiko Oberdiek.
%
% The Base Interpreter refers to any `TeX-Format',
% because some files are installed in TDS:tex/generic//.
%
% This work consists of the main source file hobsub.dtx
% and the derived files
%    hobsub.sty, hobsub.pdf, hobsub.ins, hobsub.drv, hobsub-generic.sty,
%    hobsub-hyperref.sty, hobsub-test1.tex.
%
% Distribution:
%    CTAN:macros/latex/contrib/oberdiek/hobsub.dtx
%    CTAN:macros/latex/contrib/oberdiek/hobsub.pdf
%
% Unpacking:
%    (a) If hobsub.ins is present:
%           tex hobsub.ins
%    (b) Without hobsub.ins:
%           tex hobsub.dtx
%    (c) If you insist on using LaTeX
%           latex \let\install=y\input{hobsub.dtx}
%        (quote the arguments according to the demands of your shell)
%
% Documentation:
%    (a) If hobsub.drv is present:
%           latex hobsub.drv
%    (b) Without hobsub.drv:
%           latex hobsub.dtx; ...
%    The class ltxdoc loads the configuration file ltxdoc.cfg
%    if available. Here you can specify further options, e.g.
%    use A4 as paper format:
%       \PassOptionsToClass{a4paper}{article}
%
%    Programm calls to get the documentation (example):
%       pdflatex hobsub.dtx
%       makeindex -s gind.ist hobsub.idx
%       pdflatex hobsub.dtx
%       makeindex -s gind.ist hobsub.idx
%       pdflatex hobsub.dtx
%
% Installation:
%    TDS:tex/generic/oberdiek/hobsub.sty
%    TDS:tex/generic/oberdiek/hobsub-generic.sty
%    TDS:tex/generic/oberdiek/hobsub-hyperref.sty
%    TDS:doc/latex/oberdiek/hobsub.pdf
%    TDS:doc/latex/oberdiek/test/hobsub-test1.tex
%    TDS:source/latex/oberdiek/hobsub.dtx
%
%<*ignore>
\begingroup
  \catcode123=1 %
  \catcode125=2 %
  \def\x{LaTeX2e}%
\expandafter\endgroup
\ifcase 0\ifx\install y1\fi\expandafter
         \ifx\csname processbatchFile\endcsname\relax\else1\fi
         \ifx\fmtname\x\else 1\fi\relax
\else\csname fi\endcsname
%</ignore>
%<*install>
\input docstrip.tex
\Msg{************************************************************************}
\Msg{* Installation}
\Msg{* Package: hobsub 2016/05/16 v1.14 Construct package bundles (HO)}
\Msg{************************************************************************}

\keepsilent
\askforoverwritefalse

\let\MetaPrefix\relax
\preamble

This is a generated file.

Project: hobsub
Version: 2016/05/16 v1.14

Copyright (C) 2011, 2012 by
   Heiko Oberdiek <heiko.oberdiek at googlemail.com>

This work may be distributed and/or modified under the
conditions of the LaTeX Project Public License, either
version 1.3c of this license or (at your option) any later
version. This version of this license is in
   http://www.latex-project.org/lppl/lppl-1-3c.txt
and the latest version of this license is in
   http://www.latex-project.org/lppl.txt
and version 1.3 or later is part of all distributions of
LaTeX version 2005/12/01 or later.

This work has the LPPL maintenance status "maintained".

This Current Maintainer of this work is Heiko Oberdiek.

The Base Interpreter refers to any `TeX-Format',
because some files are installed in TDS:tex/generic//.

This work consists of the main source file hobsub.dtx
and the derived files
   hobsub.sty, hobsub.pdf, hobsub.ins, hobsub.drv, hobsub-generic.sty,
   hobsub-hyperref.sty, hobsub-test1.tex.

\endpreamble
\let\MetaPrefix\DoubleperCent

\generate{%
  \file{hobsub.ins}{\from{hobsub.dtx}{install}}%
  \file{hobsub.drv}{\from{hobsub.dtx}{driver}}%
  \usedir{tex/generic/oberdiek}%
  \file{hobsub.sty}{\from{hobsub.dtx}{package}}%
  \usedir{doc/latex/oberdiek/test}%
  \file{hobsub-test1.tex}{\from{hobsub.dtx}{test1}}%
  \nopreamble
  \nopostamble
  \usedir{source/latex/oberdiek/catalogue}%
  \file{hobsub.xml}{\from{hobsub.dtx}{catalogue}}%
}
\let\hobsubSavedNormalLine\normalLine
\def\hobsubGobble#1{}
\begingroup
  \lccode`\0=32 % space
  \lccode`\1=35 % hash
  \lccode`\2=37 % percent
  \lccode`\3=92 % backslash
  \lccode`\4=123 % {
  \lccode`\5=125 % }
\lowercase{\endgroup
  \let\hobsubTokenSpaceOther= 0%
  \let\hobsubTokenHash= 1%
  \let\hobsubTokenPercent= 2%
  \let\hobsubTokenBackslash= 3%
  \let\hobsubTokenBraceLeft= 4%
  \let\hobsubTokenBraceRight= 5%
  \def\hobsubNormalLineWithSpace{%
    \hobsubSavedNormalLine0%
  }%
}
\def\normalLine{%
  \futurelet\hobsubToken\hobsubFirst
}
\def\hobsubFirst{%
  \ifx\hobsubToken\hobsubTokenSpaceOther
    \expandafter\expandafter\expandafter\futurelet
    \expandafter\expandafter\expandafter\hobsubToken
    \expandafter\expandafter\expandafter\hobsubNext
    \expandafter\hobsubGobble
  \else
    \expandafter\hobsubSavedNormalLine
  \fi
}
\def\hobsubNext{%
  \ifx\hobsubToken\hobsubTokenSpaceOther
    \expandafter\expandafter\expandafter\futurelet
    \expandafter\expandafter\expandafter\hobsubToken
    \expandafter\expandafter\expandafter\hobsubNext 
    \expandafter\hobsubGobble
  \else
    \let\hobsubWithSpace=Y%
    \ifx\hobsubToken\hobsubTokenBackslash
      \let\hobsubWithSpace=N%
    \else
      \ifx\hobsubToken\hobsubTokenBraceRight
        \let\hobsubWithSpace=N%
      \else
        \ifx\hobsubToken\hobsubTokenBraceLeft
          \let\hobsubWithSpace=N%
        \else
          \ifx\hobsubToken\hobsubTokenPercent
            \let\hobsubWithSpace=N%
          \else
            \ifx\hobsubToken\hobsubTokenHash
              \let\hobsubWithSpace=N%
            \fi
          \fi
        \fi
      \fi
    \fi
    \ifx\hobsubWithSpace Y%
      \expandafter\expandafter\expandafter\hobsubNormalLineWithSpace
    \else
      \expandafter\expandafter\expandafter\hobsubSavedNormalLine
    \fi
  \fi
}
\generate{%
  \usedir{tex/generic/oberdiek}%
  \file{hobsub-generic.sty}{%
    \from{hobsub.dtx}{generic,preamble,infwarerr}%
    \from{infwarerr.dtx}{package}%
    \from{hobsub.dtx}{post,ltxcmds}%
    \from{ltxcmds.dtx}{package}%
    \from{hobsub.dtx}{post,ifluatex}%
    \from{ifluatex.dtx}{package}%
    \from{hobsub.dtx}{post,ifvtex}%
    \from{ifvtex.dtx}{package}%
    \from{hobsub.dtx}{post,intcalc}%
    \from{intcalc.dtx}{package}%
    \from{hobsub.dtx}{post,ifpdf}%
    \from{ifpdf.dtx}{package}%
    \from{hobsub.dtx}{post,etexcmds}%
    \from{etexcmds.dtx}{package}%
    \from{hobsub.dtx}{post,kvsetkeys}%
    \from{kvsetkeys.dtx}{package}%
    \from{hobsub.dtx}{post,kvdefinekeys}%
    \from{kvdefinekeys.dtx}{package}%
    \from{hobsub.dtx}{post,luatex-loader}%
    \from{luatex.dtx}{loader}%
    \from{hobsub.dtx}{post,pdftexcmds}%
    \from{pdftexcmds.dtx}{package}%
    \from{hobsub.dtx}{post,pdfescape}%
    \from{pdfescape.dtx}{package}%
    \from{hobsub.dtx}{post,bigintcalc}%
    \from{bigintcalc.dtx}{package}%
    \from{hobsub.dtx}{post,bitset}%
    \from{bitset.dtx}{package}%
    \from{hobsub.dtx}{post,uniquecounter}%
    \from{uniquecounter.dtx}{package}%
    \from{hobsub.dtx}{post,atend}%
  }%
}
\generate{%
  \usedir{tex/generic/oberdiek}%
  \file{hobsub-hyperref.sty}{%
    \from{hobsub.dtx}{hyperref,preamble,,letltxmacro}%
    \from{letltxmacro.dtx}{package}%
    \from{hobsub.dtx}{post,hopatch}%
    \from{hopatch.dtx}{package}%
    \from{hobsub.dtx}{post,xcolor-patch}%
    \from{hycolor.dtx}{xcolor}%
    \from{hobsub.dtx}{post,atveryend}%
    \from{atveryend.dtx}{package}%
    \from{hobsub.dtx}{post,atbegshi}%
    \from{atbegshi.dtx}{package}%
    \from{hobsub.dtx}{post,refcount}%
    \from{refcount.dtx}{package}%
    \from{hobsub.dtx}{post,hycolor}%
    \from{hycolor.dtx}{package}%
    \from{hobsub.dtx}{post}%
  }%
}
\let\normalLine\hobsubSavedNormalLine

\catcode32=13\relax% active space
\let =\space%
\Msg{************************************************************************}
\Msg{*}
\Msg{* To finish the installation you have to move the following}
\Msg{* files into a directory searched by TeX:}
\Msg{*}
\Msg{*     hobsub.sty, hobsub-generic.sty, hobsub-hyperref.sty}
\Msg{*}
\Msg{* To produce the documentation run the file `hobsub.drv'}
\Msg{* through LaTeX.}
\Msg{*}
\Msg{* Happy TeXing!}
\Msg{*}
\Msg{************************************************************************}

\endbatchfile
%</install>
%<*ignore>
\fi
%</ignore>
%<*driver>
\NeedsTeXFormat{LaTeX2e}
\ProvidesFile{hobsub.drv}%
  [2016/05/16 v1.14 Construct package bundles (HO)]%
\documentclass{ltxdoc}
\usepackage{holtxdoc}[2011/11/22]
\usepackage{array}
\usepackage{caption}
\renewcommand*{\bottomfraction}{.5}
\renewcommand*{\topfraction}{.5}
\begin{document}
  \DocInput{hobsub.dtx}%
\end{document}
%</driver>
% \fi
%
%
% \CharacterTable
%  {Upper-case    \A\B\C\D\E\F\G\H\I\J\K\L\M\N\O\P\Q\R\S\T\U\V\W\X\Y\Z
%   Lower-case    \a\b\c\d\e\f\g\h\i\j\k\l\m\n\o\p\q\r\s\t\u\v\w\x\y\z
%   Digits        \0\1\2\3\4\5\6\7\8\9
%   Exclamation   \!     Double quote  \"     Hash (number) \#
%   Dollar        \$     Percent       \%     Ampersand     \&
%   Acute accent  \'     Left paren    \(     Right paren   \)
%   Asterisk      \*     Plus          \+     Comma         \,
%   Minus         \-     Point         \.     Solidus       \/
%   Colon         \:     Semicolon     \;     Less than     \<
%   Equals        \=     Greater than  \>     Question mark \?
%   Commercial at \@     Left bracket  \[     Backslash     \\
%   Right bracket \]     Circumflex    \^     Underscore    \_
%   Grave accent  \`     Left brace    \{     Vertical bar  \|
%   Right brace   \}     Tilde         \~}
%
% \GetFileInfo{hobsub.drv}
%
% \title{The \xpackage{hobsub} package}
% \date{2016/05/16 v1.14}
% \author{Heiko Oberdiek\thanks
% {Please report any issues at https://github.com/ho-tex/oberdiek/issues}\\
% \xemail{heiko.oberdiek at googlemail.com}}
%
% \maketitle
%
% \begin{abstract}
% Package \xpackage{hobsub} implements the idea of loading
% several packages in one big collection package.
% \end{abstract}
%
% \tableofcontents
%
% \def\TableGeneric{^^A
%   \TableEntry{\xpackage{infwarerr}}{}{2010/04/08 v1.3}^^A
%   \TableEntry{\xpackage{ltxcmds}}{}{2011/11/09 v1.22}^^A
%   \TableEntry{\xpackage{ifluatex}}{}{2010/03/01 v1.3}^^A
%   \TableEntry{\xpackage{ifvtex}}{}{2010/03/01 v1.5}^^A
%   \TableEntry{\xpackage{intcalc}}{}{2007/09/27 v1.1}^^A
%   \TableEntry{\xpackage{ifpdf}}{\xpackage{ifluatex}}{2011/01/30 v2.3}^^A
%   \TableEntry{\xpackage{etexcmds}}{}{2011/02/16 v1.5}^^A
%   \TableEntry{\xpackage{kvsetkeys}}{}{2012/04/25 v1.16}^^A
%   \TableEntry{\xpackage{kvdefinekeys}}{}{2011/04/07 v1.3}^^A
%   \TableEntry{\xpackage{luatex-loader}}{}{2010/03/09 v0.4}^^A
%   \TableEntry{\xpackage{pdftexcmds}}{\xpackage{ifluatex}, \xpackage{ifpdf}, \xpackage{infwarerr}, \xpackage{ltxcmds}, \xpackage{luatex-loader}}{2011/11/29 v0.20}^^A
%   \TableEntry{\xpackage{pdfescape}}{\xpackage{ltxcmds}, \xpackage{pdftexcmds}}{2011/11/25 v1.13}^^A
%   \TableEntry{\xpackage{bigintcalc}}{\xpackage{pdftexcmds}}{2012/04/08 v1.3}^^A
%   \TableEntry{\xpackage{bitset}}{\xpackage{bigintcalc}, \xpackage{infwarerr}, \xpackage{intcalc}}{2011/01/30 v1.1}^^A
%   \TableEntry{\xpackage{uniquecounter}}{\xpackage{bigintcalc}, \xpackage{infwarerr}}{2011/01/30 v1.2}^^A
% }
% \def\TableHyperref{^^A
%   \TableEntry{\xpackage{letltxmacro}}{}{2010/09/02 v1.4}^^A
%   \TableEntry{\xpackage{hopatch}}{}{2016/05/16 v1.2}^^A
%   \TableEntry{\xpackage{xcolor-patch}}{}{2011/01/30 v1.7}^^A
%   \TableEntry{\xpackage{atveryend}}{}{2011/06/30 v1.8}^^A
%   \TableEntry{\xpackage{atbegshi}}{\xpackage{ifpdf}, \xpackage{infwarerr}, \xpackage{ltxcmds}}{2011/10/05 v1.16}^^A
%   \TableEntry{\xpackage{refcount}}{\xpackage{infwarerr}, \xpackage{ltxcmds}}{2011/10/16 v3.4}^^A
%   \TableEntry{\xpackage{hycolor}}{\xpackage{xcolor-patch}}{2011/01/30 v1.7}^^A
% }
%
% \newenvironment{TableList}[2][htbp]{^^A
%   \begin{table}[{#1}]^^A
%   \centering
%   \caption{Overview \xpackage{hobsub-#2}}^^A
%   \label{tab:#2}^^A
%   \begin{tabular}{ll>{\raggedright}p{.4\linewidth}}^^A
%   \bfseries Package & \bfseries Release &
%   \bfseries Dependencies\tabularnewline
%   \hline
% }{^^A
%   \end{tabular}^^A
%   \end{table}^^A
% }
% \newcommand*{\TableEntry}[3]{%
%   #1&#3&#2\tabularnewline
% }
%
% \renewcommand*{\arraystretch}{1.1}
% \setlength{\extrarowheight}{2pt}
%
% \section{Documentation}
%
% In January 2011 the mailing list lualatex-dev discussed
% ``quietening lualatex console output'' \cite{lualatex:quiet}.
% Inspired from this, I implemented this package \xpackage{hobsub}
% to load several packages in one file. The package is applied for
% two packages collections that are subsets of my package bundle
% \xpackage{oberdiek}.
% \begin{description}
% \item[\xpackage{hobsub-generic}] is a collection of packages
%   that can also be used with \hologo{plainTeX}.
% \item[\xpackage{hobsub-hyperref}] is a collection of packages
%   that is used by package \xpackage{hyperref}.
% \end{description}
%
% Package \xpackage{hobsub} provides the macros that
% are used in the collection packages and might be used
% in other projects as well. But there are many caveats,
% some of them are discussed below.
%
% The most serious problem is consistency. A package
% that is available as standalone package and is part
% of a collection might have different versions.
% Because my packages are organized in a bundle and should
% be installed and updated as bundle, the risk seems low
% to me. However, if a single package is updated manually, for
% example for testing, then it should be loaded before the
% collection that contains the package with an older version.
% In a collection, the package will not be loaded again if it
% is already loaded.
%
% \subsection{Usage}
%
% The collection packages \xpackage{hobsub-generic} and
% \xpackage{hobsub-hyperref} are loaded as usual:
% \begin{quote}
%  \begin{tabular}{@{}ll@{}}
%    |\usepackage{hobsub-generic}| & \hologo{LaTeX}\\
%    |\input hobsub-generic.sty| & \hologo{plainTeX}
%  \end{tabular}
% \end{quote}
% or
% \begin{quote}|\usepackage{hobsub-hyperref}|\end{quote}
% Of course, both \cs{usepackage} or \cs{RequirePackage}
% can be used.
%
% If you need requirements on the version date of a package
% inside a collection, then specify them afterwards, for
% example:
% \begin{quote}
%   |\usepackage{hobsub-generic}|\\
%   |\usepackage{ltxcmds}[2010/12/04]% |^^A
%   {\itshape |\ltx@ifblank is needed|}
% \end{quote}
%
% Also it is not garanteed that a package will be always
% part of a collection. Therefore it does not harm,
% to make the requirements explicite \emph{after}
% a collection is loaded, see the previous example.
% \hologo{LaTeX} knows that a
% package is already loaded and does not load it again.
%
% \subsection{Supported features}
%
% \begin{description}
% \item[No reloads:] If a package is already loaded then it
%  is not reloaded again if it is part of a collection package.
% \item[\cs{listfiles}:]
%   Each package that is loaded in a collection is also
%   added to the output of \cs{listfiles}.
% \item[\cs{@currname}, \cs{@currext}]
%   are updated, they are used by \hologo{LaTeX}'s
%  \cs{ProvidesPackage}.
% \item[\cs{AtEndOfPackage}:]
%   \hologo{LaTeX}'s hook mechanism that is used for
%   \cs{AtEndOfPackage} is supported. However none of the
%   packages in the collections are using \cs{AtEndOfPackage},
%   therefore this feature is completely untested.
% \end{description}
%
% \subsection{Common limitations}
%
% \begin{description}
% \item[\cs{endinput}:]
%   Often a package is using \cs{endinput}, either added by
%   \xpackage{docstrip}, or to stop package loading at an earlier
%   time. Therefore \cs{endinput} must be catched during package
%   loading to prevent that the collection package is closed
%   too early before reading all packages it contains.
%
%   This redefinition of \cs{endinput} fails if the package loads
%   other files that contain \cs{endinput}. To some degree it could
%   be catched by redefining macros that load files. But this will
%   not work for the primitive \cs{input}.
% \item[Options:]
%   \hologo{LaTeX}'s package options are not supported.
%   Therefore the packages that are put in a collection must
%   not support options.
% \item[Hook packages:]
%   Some packages and classes provides hooks for packages, see
%   packages \xpackage{scrlfile}, \xpackage{filehook}
%   or class \xclass{memoir}. These hooks are not supported
%   for embedded packages inside a collection package.
%   Because of the problem with \cs{endinput}, these hooks
%   are not supported. The hooks might contain code that
%   loads other files that execute \cs{endinput}.
% \end{description}
%
% \subsection{Package \xpackage{hobsub-generic}}
%
% \begin{TableList}[bp]{generic}\TableGeneric\end{TableList}
%
% Collection `generic' contains some of the \hologo{plainTeX} compatible
% packages of my bundle, see table \ref{tab:generic}.
% It adds the following restrictions to any package
% that is part of this collection:
% \begin{description}
% \item[Generic formats:] The following formats must be supported:
%   \hologo{LaTeX}, \hologo{plainTeX} and even \hologo{iniTeX}.
%   This applies to the collection package \xpackage{hobsub-generic}
%   the same way.
% \item[Prefixed macros:]
%   Any package of this collection should only define macros with
%   prefixed names to avoid name clashes. Resources like registers
%   must not be allocated and other global states should not be
%   changed except for added macro definitions.
% \end{description}
% Thus the package \xpackage{hobsub-generic} can also preloaded
% in the format file with \hologo{iniTeX}.
%
% \subsection{Package \xpackage{hobsub-hyperref}}
%
% \begin{TableList}[tp]{hyperref}\TableHyperref\end{TableList}
%
% Collection `hyperref' contains additional packages that are
% used by package \xpackage{hyperref} and that does not fit
% in collection `generic', see table{tab:hyperref}.
% The latter collection package
% \xpackage{hobsub-generic} is loaded via a conventional
% \cs{RequirePackage}.
%
% Because most of the packages in the collection `hyperref'
% needs \hologo{LaTeX}, the collection requires \hologo{LaTeX}.
%
% \subsection{Package \xpackage{hobsub}}
%
% This section is for advanced users that might want to
% build a collection package. Then it can be done by
% using \xpackage{hobsub}:
% \begin{quote}
%   \dots\\
%   |\RequirePackage{hobsub}[2016/05/16]|\\
%   \dots\\
%   |\hobsub@StartPackage{foo1}|\\
%   \textit{\dots\unskip\ package code of \xfile{foo1.sty} \dots}\\
%   |\endinput|\\
%   |<hobsub>|\\
%   |\hobsub@StopPackage|\\
%   |\hobsub@StartPackage{bar2}|\\
%   \textit{\dots\unskip\ package code of \xfile{bar2.sty} \dots}\\
%   |\endinput|\\
%   |<hobsub>|\\
%   |\hobsub@StopPackage|
%   \dots
% \end{quote}
% But be aware of the caveats and limitations, see above.
% Only quite a small portion of the packages can be embedded
% like this.
%
% \clearpage
%
% \StopEventually{
% }
%
% \section{Implementation}
%
% \subsection{Reload check and package identification}
%
% \subsubsection{Package \xpackage{hobsub}}
%
%    \begin{macrocode}
%<*package>
%    \end{macrocode}
%    Reload check, especially if the package is not used with \LaTeX.
%    \begin{macrocode}
\begingroup\catcode61\catcode48\catcode32=10\relax%
  \catcode13=5 % ^^M
  \endlinechar=13 %
  \catcode35=6 % #
  \catcode39=12 % '
  \catcode44=12 % ,
  \catcode45=12 % -
  \catcode46=12 % .
  \catcode58=12 % :
  \catcode64=11 % @
  \catcode123=1 % {
  \catcode125=2 % }
  \expandafter\let\expandafter\x\csname ver@hobsub.sty\endcsname
  \ifx\x\relax % plain-TeX, first loading
  \else
    \def\empty{}%
    \ifx\x\empty % LaTeX, first loading,
      % variable is initialized, but \ProvidesPackage not yet seen
    \else
      \expandafter\ifx\csname PackageInfo\endcsname\relax
        \def\x#1#2{%
          \immediate\write-1{Package #1 Info: #2.}%
        }%
      \else
        \def\x#1#2{\PackageInfo{#1}{#2, stopped}}%
      \fi
      \x{hobsub}{The package is already loaded}%
      \aftergroup\endinput
    \fi
  \fi
\endgroup%
%    \end{macrocode}
%    Package identification:
%    \begin{macrocode}
\begingroup\catcode61\catcode48\catcode32=10\relax%
  \catcode13=5 % ^^M
  \endlinechar=13 %
  \catcode35=6 % #
  \catcode39=12 % '
  \catcode40=12 % (
  \catcode41=12 % )
  \catcode44=12 % ,
  \catcode45=12 % -
  \catcode46=12 % .
  \catcode47=12 % /
  \catcode58=12 % :
  \catcode64=11 % @
  \catcode91=12 % [
  \catcode93=12 % ]
  \catcode123=1 % {
  \catcode125=2 % }
  \expandafter\ifx\csname ProvidesPackage\endcsname\relax
    \def\x#1#2#3[#4]{\endgroup
      \immediate\write-1{Package: #3 #4}%
      \xdef#1{#4}%
    }%
  \else
    \def\x#1#2[#3]{\endgroup
      #2[{#3}]%
      \ifx#1\@undefined
        \xdef#1{#3}%
      \fi
      \ifx#1\relax
        \xdef#1{#3}%
      \fi
    }%
  \fi
\expandafter\x\csname ver@hobsub.sty\endcsname
\ProvidesPackage{hobsub}%
  [2016/05/16 v1.14 Construct package bundles (HO)]%
%    \end{macrocode}
%    \begin{macrocode}
%</package>
%    \end{macrocode}
%
% \subsubsection{Package \xpackage{hobsub-generic}}
%
%    \begin{macrocode}
%<*generic>
%    \end{macrocode}
%    Reload check, especially if the package is not used with \LaTeX.
%    \begin{macrocode}
\begingroup\catcode61\catcode48\catcode32=10\relax%
  \catcode13=5 % ^^M
  \endlinechar=13 %
  \catcode35=6 % #
  \catcode39=12 % '
  \catcode44=12 % ,
  \catcode45=12 % -
  \catcode46=12 % .
  \catcode58=12 % :
  \catcode64=11 % @
  \catcode123=1 % {
  \catcode125=2 % }
  \expandafter\let\expandafter\x\csname ver@hobsub-generic.sty\endcsname
  \ifx\x\relax % plain-TeX, first loading
  \else
    \def\empty{}%
    \ifx\x\empty % LaTeX, first loading,
      % variable is initialized, but \ProvidesPackage not yet seen
    \else
      \expandafter\ifx\csname PackageInfo\endcsname\relax
        \def\x#1#2{%
          \immediate\write-1{Package #1 Info: #2.}%
        }%
      \else
        \def\x#1#2{\PackageInfo{#1}{#2, stopped}}%
      \fi
      \x{hobsub-generic}{The package is already loaded}%
      \aftergroup\endinput
    \fi
  \fi
\endgroup%
%    \end{macrocode}
%    Package identification:
%    \begin{macrocode}
\begingroup\catcode61\catcode48\catcode32=10\relax%
  \catcode13=5 % ^^M
  \endlinechar=13 %
  \catcode35=6 % #
  \catcode39=12 % '
  \catcode40=12 % (
  \catcode41=12 % )
  \catcode44=12 % ,
  \catcode45=12 % -
  \catcode46=12 % .
  \catcode47=12 % /
  \catcode58=12 % :
  \catcode64=11 % @
  \catcode91=12 % [
  \catcode93=12 % ]
  \catcode123=1 % {
  \catcode125=2 % }
  \expandafter\ifx\csname ProvidesPackage\endcsname\relax
    \def\x#1#2#3[#4]{\endgroup
      \immediate\write-1{Package: #3 #4}%
      \xdef#1{#4}%
    }%
  \else
    \def\x#1#2[#3]{\endgroup
      #2[{#3}]%
      \ifx#1\@undefined
        \xdef#1{#3}%
      \fi
      \ifx#1\relax
        \xdef#1{#3}%
      \fi
    }%
  \fi
\expandafter\x\csname ver@hobsub-generic.sty\endcsname
\ProvidesPackage{hobsub-generic}%
  [2016/05/16 v1.14 Bundle oberdiek, subset generic (HO)]%
%    \end{macrocode}
%    \begin{macrocode}
%</generic>
%    \end{macrocode}
%
% \subsubsection{Package \xpackage{hobsub-hyperref}}
%
%    \begin{macrocode}
%<*hyperref>
%    \end{macrocode}
%    Reload check, especially if the package is not used with \LaTeX.
%    \begin{macrocode}
\begingroup\catcode61\catcode48\catcode32=10\relax%
  \catcode13=5 % ^^M
  \endlinechar=13 %
  \catcode35=6 % #
  \catcode39=12 % '
  \catcode44=12 % ,
  \catcode45=12 % -
  \catcode46=12 % .
  \catcode58=12 % :
  \catcode64=11 % @
  \catcode123=1 % {
  \catcode125=2 % }
  \expandafter\let\expandafter\x\csname ver@hobsub-hyperref.sty\endcsname
  \ifx\x\relax % plain-TeX, first loading
  \else
    \def\empty{}%
    \ifx\x\empty % LaTeX, first loading,
      % variable is initialized, but \ProvidesPackage not yet seen
    \else
      \expandafter\ifx\csname PackageInfo\endcsname\relax
        \def\x#1#2{%
          \immediate\write-1{Package #1 Info: #2.}%
        }%
      \else
        \def\x#1#2{\PackageInfo{#1}{#2, stopped}}%
      \fi
      \x{hobsub-hyperref}{The package is already loaded}%
      \aftergroup\endinput
    \fi
  \fi
\endgroup%
%    \end{macrocode}
%    Package identification:
%    \begin{macrocode}
\begingroup\catcode61\catcode48\catcode32=10\relax%
  \catcode13=5 % ^^M
  \endlinechar=13 %
  \catcode35=6 % #
  \catcode39=12 % '
  \catcode40=12 % (
  \catcode41=12 % )
  \catcode44=12 % ,
  \catcode45=12 % -
  \catcode46=12 % .
  \catcode47=12 % /
  \catcode58=12 % :
  \catcode64=11 % @
  \catcode91=12 % [
  \catcode93=12 % ]
  \catcode123=1 % {
  \catcode125=2 % }
  \expandafter\ifx\csname ProvidesPackage\endcsname\relax
    \def\x#1#2#3[#4]{\endgroup
      \immediate\write-1{Package: #3 #4}%
      \xdef#1{#4}%
    }%
  \else
    \def\x#1#2[#3]{\endgroup
      #2[{#3}]%
      \ifx#1\@undefined
        \xdef#1{#3}%
      \fi
      \ifx#1\relax
        \xdef#1{#3}%
      \fi
    }%
  \fi
\expandafter\x\csname ver@hobsub-hyperref.sty\endcsname
\ProvidesPackage{hobsub-hyperref}%
  [2016/05/16 v1.14 Bundle oberdiek, subset hyperref (HO)]%
%    \end{macrocode}
%    \begin{macrocode}
%</hyperref>
%    \end{macrocode}
%
% \subsection{Catcodes}
%
%    \begin{macrocode}
%<*package|generic|hyperref>
\begingroup\catcode61\catcode48\catcode32=10\relax%
  \catcode13=5 % ^^M
  \endlinechar=13 %
  \catcode123=1 % {
  \catcode125=2 % }
  \catcode64=11 % @
  \def\x{\endgroup
    \expandafter\edef\csname HOBsub@AtEnd\endcsname{%
      \endlinechar=\the\endlinechar\relax
      \catcode13=\the\catcode13\relax
      \catcode32=\the\catcode32\relax
      \catcode35=\the\catcode35\relax
      \catcode61=\the\catcode61\relax
      \catcode64=\the\catcode64\relax
      \catcode123=\the\catcode123\relax
      \catcode125=\the\catcode125\relax
    }%
  }%
\x\catcode61\catcode48\catcode32=10\relax%
\catcode13=5 % ^^M
\endlinechar=13 %
\catcode35=6 % #
\catcode64=11 % @
\catcode123=1 % {
\catcode125=2 % }
\def\TMP@EnsureCode#1#2{%
  \edef\HOBsub@AtEnd{%
    \HOBsub@AtEnd
    \catcode#1=\the\catcode#1\relax
  }%
  \catcode#1=#2\relax
}
\TMP@EnsureCode{39}{12}% '
\TMP@EnsureCode{40}{12}% (
\TMP@EnsureCode{41}{12}% )
\TMP@EnsureCode{45}{12}% -
\TMP@EnsureCode{46}{12}% .
\TMP@EnsureCode{47}{12}% /
\TMP@EnsureCode{58}{12}% :
\TMP@EnsureCode{60}{12}% <
\TMP@EnsureCode{62}{12}% >
\TMP@EnsureCode{96}{12}% `
\edef\HOBsub@AtEnd{\HOBsub@AtEnd\noexpand\endinput}
%    \end{macrocode}
%
%    \begin{macrocode}
%</package|generic|hyperref>
%    \end{macrocode}
%
% \subsection{Package \xpackage{hobsub-hyperref} loads \xpackage{hobsub-generic}}
%
%    \begin{macrocode}
%<*hyperref>
\NeedsTeXFormat{LaTeX2e}
\RequirePackage{hobsub-generic}[2016/05/16]
%</hyperref>
%    \end{macrocode}
%
% \subsection{Preamble}
%
%    \begin{macrocode}
%<*preamble>
\begingroup\expandafter\expandafter\expandafter\endgroup
\expandafter\ifx\csname ver@hobsub.sty\endcsname\relax
\else
  \expandafter\hobsub@GobbleRemainingPackage
\fi
%</preamble>
%    \end{macrocode}
%
%
%    \begin{macrocode}
%<*package|preamble>
%    \end{macrocode}
%
%    \begin{macro}{\HOBsub@OrgEndinput}
%    \begin{macrocode}
\begingroup\expandafter\expandafter\expandafter\endgroup
\expandafter\ifx\csname HOBsub@OrgEndinput\endcsname\relax
  \let\HOBsub@OrgEndinput\endinput
\fi
%    \end{macrocode}
%    \end{macro}
%    \begin{macro}{\hobsub@GobbleRemainingPackage}
%    \begin{macrocode}
\def\hobsub@GobbleRemainingPackage{%
  \begingroup
  \catcode92=14 % backslash: comment
  \catcode32=14 % space: comment
  \catcode35=14 % hash: comment
  \catcode123=14 % left brace: comment
  \catcode125=14 % right brace: comment
  \catcode60=3 % less: math
  \catcode62=4 % greater: align
  \endlinechar=-1 %
  \HOBsub@GobbleRemainingPackage
}
%    \end{macrocode}
%    \end{macro}
%    \begin{macro}{\HOBsub@GobbleRemainingPackage}
%    \begin{macrocode}
\catcode60=3 % less: dollar
\catcode62=4 % greater: align
\long\def\HOBsub@GobbleRemainingPackage#1<hobsub>{%
  \endgroup
}
\catcode60=12 % less: other
\catcode62=12 % greater: other
%    \end{macrocode}
%    \end{macro}
%
%    \begin{macro}{\hobsub@StartPackage}
%    \begin{macrocode}
\def\hobsub@StartPackage#1{%
  \begingroup\expandafter\expandafter\expandafter\endgroup
  \expandafter\ifx\csname ver@#1.sty\endcsname\relax
    \let\HOBsub@OrgCurrName\@currname
    \let\HOBsub@OrgCurrExt\@currext
    \csname @pushfilename\endcsname
    \def\@currname{#1}%
    \def\@currext{sty}%
    \expandafter\def\csname\@currname.\@currext-h@@k\endcsname{}%
    \let\endinput\hobsub@GobbleRemainingPackage
    \def\hobsub@StopPackage{%
      \let\hobsub@StopPackage\relax
      \HOBsub@StopPackage
%     \hobsub@Info{hobsub}{Package `#1' loaded}%
    }%
    \hobsub@AddToFileList{#1.sty}%
  \else
    \hobsub@Info{hobsub}{Skipping package `#1' (already loaded)}%
    \let\hobsub@StopPackage\relax
    \expandafter\hobsub@GobbleRemainingPackage
  \fi
}
%    \end{macrocode}
%    \end{macro}
%
%    \begin{macro}{\hobsub@StopPackage}
%    \begin{macrocode}
\let\hobsub@StopPackage\relax
%    \end{macrocode}
%    \end{macro}
%
%    \begin{macro}{\hobsub@Info}
%    \begin{macrocode}
\def\hobsub@Info#1#2{%
  \begingroup\expandafter\expandafter\expandafter\endgroup
  \expandafter\ifx\csname @PackageInfoNoLine\endcsname\relax
    \immediate\write-1{Package #1 Info: #2.}%
  \else
    \let\hobsub@Info\@PackageInfoNoLine
    \hobsub@Info{#1}{#2}%
  \fi
}
%    \end{macrocode}
%    \end{macro}
%
%    \begin{macro}{\HOBsub@StopPackage}
%    \begin{macrocode}
\def\HOBsub@StopPackage{%
  \csname\@currname.\@currext-h@@k\endcsname
  \let\endinput\HOBsub@OrgEndinput
  \csname @popfilename\endcsname
  \let\@currname\HOBsub@OrgCurrName
  \let\@currext\HOBsub@OrgCurrExt
}
%    \end{macrocode}
%    \end{macro}
%
%    \begin{macro}{\hobsub@AddToFileList}
%    \begin{macrocode}
\begingroup\expandafter\expandafter\expandafter\endgroup
\expandafter\ifx\csname @addtofilelist\endcsname\relax
  \def\hobsub@AddToFileList#1{}%
\else
  \def\hobsub@AddToFileList#1{%
    \@addtofilelist{#1}%
  }%
\fi
%    \end{macrocode}
%    \end{macro}
%
%    \begin{macrocode}
%</package|preamble>
%    \end{macrocode}
%
% \subsection{End of preamble for subset packages}
%
%    Add the end marker for possible skipping of previous
%    macros, if package \xpackage{hobsub} was already loaded.
%    \begin{macrocode}
%<*preamble>
\hobsub@GobbleRemainingPackage
<hobsub>
%    \end{macrocode}
%
%    All macros of package \xpackage{hobsub} are now available.
%    Now we officially load the package \xpackage{hobsub} inline
%    for proper package identification.
%    \begin{macrocode}
\hobsub@StartPackage{hobsub}%
\begingroup\expandafter\expandafter\expandafter\endgroup
\expandafter\ifx\csname ProvidesPackage\endcsname\relax
  \immediate\write-1{%
    Package: hobsub 2016/05/16 v1.14 Construct package bundles (HO)%
  }%
\else
  \ProvidesPackage{hobsub}%
  [2016/05/16 v1.14 Construct package bundles (HO)]%
\fi
%</preamble>
%    \end{macrocode}
%
%
%    \begin{macrocode}
%<*post|preamble>
 \endinput
<hobsub>
\hobsub@StopPackage
%</post|preamble>
%    \end{macrocode}
%
% \subsection{Package list}
%
%    \begin{macrocode}
%<infwarerr>\hobsub@StartPackage{infwarerr}
%<ltxcmds>\hobsub@StartPackage{ltxcmds}
%<ifluatex>\hobsub@StartPackage{ifluatex}
%<ifvtex>\hobsub@StartPackage{ifvtex}
%<intcalc>\hobsub@StartPackage{intcalc}
%<ifpdf>\hobsub@StartPackage{ifpdf}
%<etexcmds>\hobsub@StartPackage{etexcmds}
%<kvsetkeys>\hobsub@StartPackage{kvsetkeys}
%<kvdefinekeys>\hobsub@StartPackage{kvdefinekeys}
%<*luatex-loader>
\ifluatex
\else
  \expandafter\hobsub@GobbleRemainingPackage
\fi
\hobsub@StartPackage{luatex-loader}
%</luatex-loader>
%<pdftexcmds>\hobsub@StartPackage{pdftexcmds}
%<pdfescape>\hobsub@StartPackage{pdfescape}
%<bigintcalc>\hobsub@StartPackage{bigintcalc}
%<bitset>\hobsub@StartPackage{bitset}
%<uniquecounter>\hobsub@StartPackage{uniquecounter}
%<letltxmacro>\hobsub@StartPackage{letltxmacro}
%<hopatch>\hobsub@StartPackage{hopatch}
%<xcolor-patch>\hobsub@StartPackage{xcolor-patch}
%<atveryend>\hobsub@StartPackage{atveryend}
%<atbegshi>\hobsub@StartPackage{atbegshi}
%<refcount>\hobsub@StartPackage{refcount}
%<hycolor>\hobsub@StartPackage{hycolor}
%    \end{macrocode}
%
% \subsection{End of packages}
%
%    \begin{macrocode}
%<package|atend>\HOBsub@AtEnd%
%    \end{macrocode}
%
% \section{Test}
%
% \subsection{Catcode checks for loading}
%
%    \begin{macrocode}
%<*test1>
%    \end{macrocode}
%    \begin{macrocode}
\catcode`\{=1 %
\catcode`\}=2 %
\catcode`\#=6 %
\catcode`\@=11 %
\expandafter\ifx\csname count@\endcsname\relax
  \countdef\count@=255 %
\fi
\expandafter\ifx\csname @gobble\endcsname\relax
  \long\def\@gobble#1{}%
\fi
\expandafter\ifx\csname @firstofone\endcsname\relax
  \long\def\@firstofone#1{#1}%
\fi
\expandafter\ifx\csname loop\endcsname\relax
  \expandafter\@firstofone
\else
  \expandafter\@gobble
\fi
{%
  \def\loop#1\repeat{%
    \def\body{#1}%
    \iterate
  }%
  \def\iterate{%
    \body
      \let\next\iterate
    \else
      \let\next\relax
    \fi
    \next
  }%
  \let\repeat=\fi
}%
\def\RestoreCatcodes{}
\count@=0 %
\loop
  \edef\RestoreCatcodes{%
    \RestoreCatcodes
    \catcode\the\count@=\the\catcode\count@\relax
  }%
\ifnum\count@<255 %
  \advance\count@ 1 %
\repeat

\def\RangeCatcodeInvalid#1#2{%
  \count@=#1\relax
  \loop
    \catcode\count@=15 %
  \ifnum\count@<#2\relax
    \advance\count@ 1 %
  \repeat
}
\def\RangeCatcodeCheck#1#2#3{%
  \count@=#1\relax
  \loop
    \ifnum#3=\catcode\count@
    \else
      \errmessage{%
        Character \the\count@\space
        with wrong catcode \the\catcode\count@\space
        instead of \number#3%
      }%
    \fi
  \ifnum\count@<#2\relax
    \advance\count@ 1 %
  \repeat
}
\def\space{ }
\expandafter\ifx\csname LoadCommand\endcsname\relax
  \def\LoadCommand{\input hobsub.sty\relax}%
\fi
\def\Test{%
  \RangeCatcodeInvalid{0}{47}%
  \RangeCatcodeInvalid{58}{64}%
  \RangeCatcodeInvalid{91}{96}%
  \RangeCatcodeInvalid{123}{255}%
  \catcode`\@=12 %
  \catcode`\\=0 %
  \catcode`\%=14 %
  \LoadCommand
  \RangeCatcodeCheck{0}{36}{15}%
  \RangeCatcodeCheck{37}{37}{14}%
  \RangeCatcodeCheck{38}{47}{15}%
  \RangeCatcodeCheck{48}{57}{12}%
  \RangeCatcodeCheck{58}{63}{15}%
  \RangeCatcodeCheck{64}{64}{12}%
  \RangeCatcodeCheck{65}{90}{11}%
  \RangeCatcodeCheck{91}{91}{15}%
  \RangeCatcodeCheck{92}{92}{0}%
  \RangeCatcodeCheck{93}{96}{15}%
  \RangeCatcodeCheck{97}{122}{11}%
  \RangeCatcodeCheck{123}{255}{15}%
  \RestoreCatcodes
}
\Test
\csname @@end\endcsname
\end
%    \end{macrocode}
%    \begin{macrocode}
%</test1>
%    \end{macrocode}
%
% \section{Installation}
%
% \subsection{Download}
%
% \paragraph{Package.} This package is available on
% CTAN\footnote{\url{http://ctan.org/pkg/hobsub}}:
% \begin{description}
% \item[\CTAN{macros/latex/contrib/oberdiek/hobsub.dtx}] The source file.
% \item[\CTAN{macros/latex/contrib/oberdiek/hobsub.pdf}] Documentation.
% \end{description}
%
%
% \paragraph{Bundle.} All the packages of the bundle `oberdiek'
% are also available in a TDS compliant ZIP archive. There
% the packages are already unpacked and the documentation files
% are generated. The files and directories obey the TDS standard.
% \begin{description}
% \item[\CTAN{install/macros/latex/contrib/oberdiek.tds.zip}]
% \end{description}
% \emph{TDS} refers to the standard ``A Directory Structure
% for \TeX\ Files'' (\CTAN{tds/tds.pdf}). Directories
% with \xfile{texmf} in their name are usually organized this way.
%
% \subsection{Bundle installation}
%
% \paragraph{Unpacking.} Unpack the \xfile{oberdiek.tds.zip} in the
% TDS tree (also known as \xfile{texmf} tree) of your choice.
% Example (linux):
% \begin{quote}
%   |unzip oberdiek.tds.zip -d ~/texmf|
% \end{quote}
%
% \paragraph{Script installation.}
% Check the directory \xfile{TDS:scripts/oberdiek/} for
% scripts that need further installation steps.
% Package \xpackage{attachfile2} comes with the Perl script
% \xfile{pdfatfi.pl} that should be installed in such a way
% that it can be called as \texttt{pdfatfi}.
% Example (linux):
% \begin{quote}
%   |chmod +x scripts/oberdiek/pdfatfi.pl|\\
%   |cp scripts/oberdiek/pdfatfi.pl /usr/local/bin/|
% \end{quote}
%
% \subsection{Package installation}
%
% \paragraph{Unpacking.} The \xfile{.dtx} file is a self-extracting
% \docstrip\ archive. The files are extracted by running the
% \xfile{.dtx} through \plainTeX:
% \begin{quote}
%   \verb|tex hobsub.dtx|
% \end{quote}
%
% \paragraph{TDS.} Now the different files must be moved into
% the different directories in your installation TDS tree
% (also known as \xfile{texmf} tree):
% \begin{quote}
% \def\t{^^A
% \begin{tabular}{@{}>{\ttfamily}l@{ $\rightarrow$ }>{\ttfamily}l@{}}
%   hobsub.sty & tex/generic/oberdiek/hobsub.sty\\
%   hobsub-generic.sty & tex/generic/oberdiek/hobsub-generic.sty\\
%   hobsub-hyperref.sty & tex/generic/oberdiek/hobsub-hyperref.sty\\
%   hobsub.pdf & doc/latex/oberdiek/hobsub.pdf\\
%   test/hobsub-test1.tex & doc/latex/oberdiek/test/hobsub-test1.tex\\
%   hobsub.dtx & source/latex/oberdiek/hobsub.dtx\\
% \end{tabular}^^A
% }^^A
% \sbox0{\t}^^A
% \ifdim\wd0>\linewidth
%   \begingroup
%     \advance\linewidth by\leftmargin
%     \advance\linewidth by\rightmargin
%   \edef\x{\endgroup
%     \def\noexpand\lw{\the\linewidth}^^A
%   }\x
%   \def\lwbox{^^A
%     \leavevmode
%     \hbox to \linewidth{^^A
%       \kern-\leftmargin\relax
%       \hss
%       \usebox0
%       \hss
%       \kern-\rightmargin\relax
%     }^^A
%   }^^A
%   \ifdim\wd0>\lw
%     \sbox0{\small\t}^^A
%     \ifdim\wd0>\linewidth
%       \ifdim\wd0>\lw
%         \sbox0{\footnotesize\t}^^A
%         \ifdim\wd0>\linewidth
%           \ifdim\wd0>\lw
%             \sbox0{\scriptsize\t}^^A
%             \ifdim\wd0>\linewidth
%               \ifdim\wd0>\lw
%                 \sbox0{\tiny\t}^^A
%                 \ifdim\wd0>\linewidth
%                   \lwbox
%                 \else
%                   \usebox0
%                 \fi
%               \else
%                 \lwbox
%               \fi
%             \else
%               \usebox0
%             \fi
%           \else
%             \lwbox
%           \fi
%         \else
%           \usebox0
%         \fi
%       \else
%         \lwbox
%       \fi
%     \else
%       \usebox0
%     \fi
%   \else
%     \lwbox
%   \fi
% \else
%   \usebox0
% \fi
% \end{quote}
% If you have a \xfile{docstrip.cfg} that configures and enables \docstrip's
% TDS installing feature, then some files can already be in the right
% place, see the documentation of \docstrip.
%
% \subsection{Refresh file name databases}
%
% If your \TeX~distribution
% (\teTeX, \mikTeX, \dots) relies on file name databases, you must refresh
% these. For example, \teTeX\ users run \verb|texhash| or
% \verb|mktexlsr|.
%
% \subsection{Some details for the interested}
%
% \paragraph{Attached source.}
%
% The PDF documentation on CTAN also includes the
% \xfile{.dtx} source file. It can be extracted by
% AcrobatReader 6 or higher. Another option is \textsf{pdftk},
% e.g. unpack the file into the current directory:
% \begin{quote}
%   \verb|pdftk hobsub.pdf unpack_files output .|
% \end{quote}
%
% \paragraph{Unpacking with \LaTeX.}
% The \xfile{.dtx} chooses its action depending on the format:
% \begin{description}
% \item[\plainTeX:] Run \docstrip\ and extract the files.
% \item[\LaTeX:] Generate the documentation.
% \end{description}
% If you insist on using \LaTeX\ for \docstrip\ (really,
% \docstrip\ does not need \LaTeX), then inform the autodetect routine
% about your intention:
% \begin{quote}
%   \verb|latex \let\install=y\input{hobsub.dtx}|
% \end{quote}
% Do not forget to quote the argument according to the demands
% of your shell.
%
% \paragraph{Generating the documentation.}
% You can use both the \xfile{.dtx} or the \xfile{.drv} to generate
% the documentation. The process can be configured by the
% configuration file \xfile{ltxdoc.cfg}. For instance, put this
% line into this file, if you want to have A4 as paper format:
% \begin{quote}
%   \verb|\PassOptionsToClass{a4paper}{article}|
% \end{quote}
% An example follows how to generate the
% documentation with pdf\LaTeX:
% \begin{quote}
%\begin{verbatim}
%pdflatex hobsub.dtx
%makeindex -s gind.ist hobsub.idx
%pdflatex hobsub.dtx
%makeindex -s gind.ist hobsub.idx
%pdflatex hobsub.dtx
%\end{verbatim}
% \end{quote}
%
% \section{Catalogue}
%
% The following XML file can be used as source for the
% \href{http://mirror.ctan.org/help/Catalogue/catalogue.html}{\TeX\ Catalogue}.
% The elements \texttt{caption} and \texttt{description} are imported
% from the original XML file from the Catalogue.
% The name of the XML file in the Catalogue is \xfile{hobsub.xml}.
%    \begin{macrocode}
%<*catalogue>
<?xml version='1.0' encoding='us-ascii'?>
<!DOCTYPE entry SYSTEM 'catalogue.dtd'>
<entry datestamp='$Date$' modifier='$Author$' id='hobsub'>
  <name>hobsub</name>
  <caption>Construct package bundles</caption>
  <authorref id='auth:oberdiek'/>
  <copyright owner='Heiko Oberdiek' year='2011,2012'/>
  <license type='lppl1.3'/>
  <version number='1.14'/>
  <description>
    The package offers a means of loading a bundle of (other) packages
    &#x201C;at once&#x201D;, offering a means of reducing the amount of log output
    as the packages are loaded, and an aide-memoire for complicated
    requirements.  In some ways, loading a package via hobsub behaves
    much like loading the package in the normal way; for example,
    reloads are suppressed, as in LaTeX proper, and correct
    <tt>\listfiles</tt> output is produced.
    <p/>
    Examples provided are a &#x2018;generic&#x2019; bundle (comprising the author&#x2019;s
    packages that can be used with Plain TeX) and a &#x2018;hyperref&#x2019; bundle
    (comprising packages useful when generating hypertext output with
    LaTeX).
    <p/>
    The package is part of the <xref refid='oberdiek'>oberdiek</xref> bundle.
  </description>
  <documentation details='Package documentation'
      href='ctan:/macros/latex/contrib/oberdiek/hobsub.pdf'/>
  <ctan file='true' path='/macros/latex/contrib/oberdiek/hobsub.dtx'/>
  <miktex location='oberdiek'/>
  <texlive location='oberdiek'/>
  <install path='/macros/latex/contrib/oberdiek/oberdiek.tds.zip'/>
</entry>
%</catalogue>
%    \end{macrocode}
%
% \begin{thebibliography}{9}
% \raggedright
% \bibitem{lualatex:quiet}
% Will Robertson,
% \textit{[lltx] quietening lualatex console output};
% mailing list \textsf{lualatex-dev at tug.org},
% 2011-01-29;
% \url{http://tug.org/pipermail/lualatex-dev/2011-January/001031.html}.
%
% \end{thebibliography}
%
% \begin{History}
%   \begin{Version}{2011/01/30 v1.0}
%   \item
%     First public version.
%   \end{Version}
%   \begin{Version}{2011/04/16 v1.1}
%   \item
%     Package updates.
%   \end{Version}
%   \begin{Version}{2011/04/17 v1.2}
%   \item
%     White spaces at line begins are removed or reduced
%     in the generated collection packages.
%   \end{Version}
%   \begin{Version}{2011/04/18 v1.3}
%   \item
%     Package updates.
%   \end{Version}
%   \begin{Version}{2011/04/23 v1.4}
%   \item
%     Package updates.
%   \end{Version}
%   \begin{Version}{2011/06/24 v1.5}
%   \item
%     Package updates.
%   \end{Version}
%   \begin{Version}{2011/06/30 v1.6}
%   \item
%     Package updates.
%   \end{Version}
%   \begin{Version}{2011/07/01 v1.7}
%   \item
%     Package updates.
%   \end{Version}
%   \begin{Version}{2011/07/28 v1.8}
%   \item
%     Package updates.
%   \end{Version}
%   \begin{Version}{2011/08/22 v1.9}
%   \item
%     Package updates.
%   \end{Version}
%   \begin{Version}{2011/10/16 v1.10}
%   \item
%     Package updates.
%   \end{Version}
%   \begin{Version}{2011/11/29 v1.11}
%   \item
%     Package updates.
%   \end{Version}
%   \begin{Version}{2012/04/25 v1.12}
%   \item
%     Package updates.
%   \end{Version}
%   \begin{Version}{2012/05/28 v1.13}
%   \item
%     Package updates.
%   \end{Version}
%   \begin{Version}{2016/05/16 v1.14}
%   \item
%     Documentation updates.
%   \end{Version}
% \end{History}
%
% \PrintIndex
%
% \Finale
\endinput
|
% \end{quote}
% Do not forget to quote the argument according to the demands
% of your shell.
%
% \paragraph{Generating the documentation.}
% You can use both the \xfile{.dtx} or the \xfile{.drv} to generate
% the documentation. The process can be configured by the
% configuration file \xfile{ltxdoc.cfg}. For instance, put this
% line into this file, if you want to have A4 as paper format:
% \begin{quote}
%   \verb|\PassOptionsToClass{a4paper}{article}|
% \end{quote}
% An example follows how to generate the
% documentation with pdf\LaTeX:
% \begin{quote}
%\begin{verbatim}
%pdflatex hobsub.dtx
%makeindex -s gind.ist hobsub.idx
%pdflatex hobsub.dtx
%makeindex -s gind.ist hobsub.idx
%pdflatex hobsub.dtx
%\end{verbatim}
% \end{quote}
%
% \section{Catalogue}
%
% The following XML file can be used as source for the
% \href{http://mirror.ctan.org/help/Catalogue/catalogue.html}{\TeX\ Catalogue}.
% The elements \texttt{caption} and \texttt{description} are imported
% from the original XML file from the Catalogue.
% The name of the XML file in the Catalogue is \xfile{hobsub.xml}.
%    \begin{macrocode}
%<*catalogue>
<?xml version='1.0' encoding='us-ascii'?>
<!DOCTYPE entry SYSTEM 'catalogue.dtd'>
<entry datestamp='$Date$' modifier='$Author$' id='hobsub'>
  <name>hobsub</name>
  <caption>Construct package bundles</caption>
  <authorref id='auth:oberdiek'/>
  <copyright owner='Heiko Oberdiek' year='2011,2012'/>
  <license type='lppl1.3'/>
  <version number='1.14'/>
  <description>
    The package offers a means of loading a bundle of (other) packages
    &#x201C;at once&#x201D;, offering a means of reducing the amount of log output
    as the packages are loaded, and an aide-memoire for complicated
    requirements.  In some ways, loading a package via hobsub behaves
    much like loading the package in the normal way; for example,
    reloads are suppressed, as in LaTeX proper, and correct
    <tt>\listfiles</tt> output is produced.
    <p/>
    Examples provided are a &#x2018;generic&#x2019; bundle (comprising the author&#x2019;s
    packages that can be used with Plain TeX) and a &#x2018;hyperref&#x2019; bundle
    (comprising packages useful when generating hypertext output with
    LaTeX).
    <p/>
    The package is part of the <xref refid='oberdiek'>oberdiek</xref> bundle.
  </description>
  <documentation details='Package documentation'
      href='ctan:/macros/latex/contrib/oberdiek/hobsub.pdf'/>
  <ctan file='true' path='/macros/latex/contrib/oberdiek/hobsub.dtx'/>
  <miktex location='oberdiek'/>
  <texlive location='oberdiek'/>
  <install path='/macros/latex/contrib/oberdiek/oberdiek.tds.zip'/>
</entry>
%</catalogue>
%    \end{macrocode}
%
% \begin{thebibliography}{9}
% \raggedright
% \bibitem{lualatex:quiet}
% Will Robertson,
% \textit{[lltx] quietening lualatex console output};
% mailing list \textsf{lualatex-dev at tug.org},
% 2011-01-29;
% \url{http://tug.org/pipermail/lualatex-dev/2011-January/001031.html}.
%
% \end{thebibliography}
%
% \begin{History}
%   \begin{Version}{2011/01/30 v1.0}
%   \item
%     First public version.
%   \end{Version}
%   \begin{Version}{2011/04/16 v1.1}
%   \item
%     Package updates.
%   \end{Version}
%   \begin{Version}{2011/04/17 v1.2}
%   \item
%     White spaces at line begins are removed or reduced
%     in the generated collection packages.
%   \end{Version}
%   \begin{Version}{2011/04/18 v1.3}
%   \item
%     Package updates.
%   \end{Version}
%   \begin{Version}{2011/04/23 v1.4}
%   \item
%     Package updates.
%   \end{Version}
%   \begin{Version}{2011/06/24 v1.5}
%   \item
%     Package updates.
%   \end{Version}
%   \begin{Version}{2011/06/30 v1.6}
%   \item
%     Package updates.
%   \end{Version}
%   \begin{Version}{2011/07/01 v1.7}
%   \item
%     Package updates.
%   \end{Version}
%   \begin{Version}{2011/07/28 v1.8}
%   \item
%     Package updates.
%   \end{Version}
%   \begin{Version}{2011/08/22 v1.9}
%   \item
%     Package updates.
%   \end{Version}
%   \begin{Version}{2011/10/16 v1.10}
%   \item
%     Package updates.
%   \end{Version}
%   \begin{Version}{2011/11/29 v1.11}
%   \item
%     Package updates.
%   \end{Version}
%   \begin{Version}{2012/04/25 v1.12}
%   \item
%     Package updates.
%   \end{Version}
%   \begin{Version}{2012/05/28 v1.13}
%   \item
%     Package updates.
%   \end{Version}
%   \begin{Version}{2016/05/16 v1.14}
%   \item
%     Documentation updates.
%   \end{Version}
% \end{History}
%
% \PrintIndex
%
% \Finale
\endinput

%        (quote the arguments according to the demands of your shell)
%
% Documentation:
%    (a) If hobsub.drv is present:
%           latex hobsub.drv
%    (b) Without hobsub.drv:
%           latex hobsub.dtx; ...
%    The class ltxdoc loads the configuration file ltxdoc.cfg
%    if available. Here you can specify further options, e.g.
%    use A4 as paper format:
%       \PassOptionsToClass{a4paper}{article}
%
%    Programm calls to get the documentation (example):
%       pdflatex hobsub.dtx
%       makeindex -s gind.ist hobsub.idx
%       pdflatex hobsub.dtx
%       makeindex -s gind.ist hobsub.idx
%       pdflatex hobsub.dtx
%
% Installation:
%    TDS:tex/generic/oberdiek/hobsub.sty
%    TDS:tex/generic/oberdiek/hobsub-generic.sty
%    TDS:tex/generic/oberdiek/hobsub-hyperref.sty
%    TDS:doc/latex/oberdiek/hobsub.pdf
%    TDS:doc/latex/oberdiek/test/hobsub-test1.tex
%    TDS:source/latex/oberdiek/hobsub.dtx
%
%<*ignore>
\begingroup
  \catcode123=1 %
  \catcode125=2 %
  \def\x{LaTeX2e}%
\expandafter\endgroup
\ifcase 0\ifx\install y1\fi\expandafter
         \ifx\csname processbatchFile\endcsname\relax\else1\fi
         \ifx\fmtname\x\else 1\fi\relax
\else\csname fi\endcsname
%</ignore>
%<*install>
\input docstrip.tex
\Msg{************************************************************************}
\Msg{* Installation}
\Msg{* Package: hobsub 2016/05/16 v1.14 Construct package bundles (HO)}
\Msg{************************************************************************}

\keepsilent
\askforoverwritefalse

\let\MetaPrefix\relax
\preamble

This is a generated file.

Project: hobsub
Version: 2016/05/16 v1.14

Copyright (C) 2011, 2012 by
   Heiko Oberdiek <heiko.oberdiek at googlemail.com>

This work may be distributed and/or modified under the
conditions of the LaTeX Project Public License, either
version 1.3c of this license or (at your option) any later
version. This version of this license is in
   http://www.latex-project.org/lppl/lppl-1-3c.txt
and the latest version of this license is in
   http://www.latex-project.org/lppl.txt
and version 1.3 or later is part of all distributions of
LaTeX version 2005/12/01 or later.

This work has the LPPL maintenance status "maintained".

This Current Maintainer of this work is Heiko Oberdiek.

The Base Interpreter refers to any `TeX-Format',
because some files are installed in TDS:tex/generic//.

This work consists of the main source file hobsub.dtx
and the derived files
   hobsub.sty, hobsub.pdf, hobsub.ins, hobsub.drv, hobsub-generic.sty,
   hobsub-hyperref.sty, hobsub-test1.tex.

\endpreamble
\let\MetaPrefix\DoubleperCent

\generate{%
  \file{hobsub.ins}{\from{hobsub.dtx}{install}}%
  \file{hobsub.drv}{\from{hobsub.dtx}{driver}}%
  \usedir{tex/generic/oberdiek}%
  \file{hobsub.sty}{\from{hobsub.dtx}{package}}%
  \usedir{doc/latex/oberdiek/test}%
  \file{hobsub-test1.tex}{\from{hobsub.dtx}{test1}}%
  \nopreamble
  \nopostamble
  \usedir{source/latex/oberdiek/catalogue}%
  \file{hobsub.xml}{\from{hobsub.dtx}{catalogue}}%
}
\let\hobsubSavedNormalLine\normalLine
\def\hobsubGobble#1{}
\begingroup
  \lccode`\0=32 % space
  \lccode`\1=35 % hash
  \lccode`\2=37 % percent
  \lccode`\3=92 % backslash
  \lccode`\4=123 % {
  \lccode`\5=125 % }
\lowercase{\endgroup
  \let\hobsubTokenSpaceOther= 0%
  \let\hobsubTokenHash= 1%
  \let\hobsubTokenPercent= 2%
  \let\hobsubTokenBackslash= 3%
  \let\hobsubTokenBraceLeft= 4%
  \let\hobsubTokenBraceRight= 5%
  \def\hobsubNormalLineWithSpace{%
    \hobsubSavedNormalLine0%
  }%
}
\def\normalLine{%
  \futurelet\hobsubToken\hobsubFirst
}
\def\hobsubFirst{%
  \ifx\hobsubToken\hobsubTokenSpaceOther
    \expandafter\expandafter\expandafter\futurelet
    \expandafter\expandafter\expandafter\hobsubToken
    \expandafter\expandafter\expandafter\hobsubNext
    \expandafter\hobsubGobble
  \else
    \expandafter\hobsubSavedNormalLine
  \fi
}
\def\hobsubNext{%
  \ifx\hobsubToken\hobsubTokenSpaceOther
    \expandafter\expandafter\expandafter\futurelet
    \expandafter\expandafter\expandafter\hobsubToken
    \expandafter\expandafter\expandafter\hobsubNext 
    \expandafter\hobsubGobble
  \else
    \let\hobsubWithSpace=Y%
    \ifx\hobsubToken\hobsubTokenBackslash
      \let\hobsubWithSpace=N%
    \else
      \ifx\hobsubToken\hobsubTokenBraceRight
        \let\hobsubWithSpace=N%
      \else
        \ifx\hobsubToken\hobsubTokenBraceLeft
          \let\hobsubWithSpace=N%
        \else
          \ifx\hobsubToken\hobsubTokenPercent
            \let\hobsubWithSpace=N%
          \else
            \ifx\hobsubToken\hobsubTokenHash
              \let\hobsubWithSpace=N%
            \fi
          \fi
        \fi
      \fi
    \fi
    \ifx\hobsubWithSpace Y%
      \expandafter\expandafter\expandafter\hobsubNormalLineWithSpace
    \else
      \expandafter\expandafter\expandafter\hobsubSavedNormalLine
    \fi
  \fi
}
\generate{%
  \usedir{tex/generic/oberdiek}%
  \file{hobsub-generic.sty}{%
    \from{hobsub.dtx}{generic,preamble,infwarerr}%
    \from{infwarerr.dtx}{package}%
    \from{hobsub.dtx}{post,ltxcmds}%
    \from{ltxcmds.dtx}{package}%
    \from{hobsub.dtx}{post,ifluatex}%
    \from{ifluatex.dtx}{package}%
    \from{hobsub.dtx}{post,ifvtex}%
    \from{ifvtex.dtx}{package}%
    \from{hobsub.dtx}{post,intcalc}%
    \from{intcalc.dtx}{package}%
    \from{hobsub.dtx}{post,ifpdf}%
    \from{ifpdf.dtx}{package}%
    \from{hobsub.dtx}{post,etexcmds}%
    \from{etexcmds.dtx}{package}%
    \from{hobsub.dtx}{post,kvsetkeys}%
    \from{kvsetkeys.dtx}{package}%
    \from{hobsub.dtx}{post,kvdefinekeys}%
    \from{kvdefinekeys.dtx}{package}%
    \from{hobsub.dtx}{post,luatex-loader}%
    \from{luatex.dtx}{loader}%
    \from{hobsub.dtx}{post,pdftexcmds}%
    \from{pdftexcmds.dtx}{package}%
    \from{hobsub.dtx}{post,pdfescape}%
    \from{pdfescape.dtx}{package}%
    \from{hobsub.dtx}{post,bigintcalc}%
    \from{bigintcalc.dtx}{package}%
    \from{hobsub.dtx}{post,bitset}%
    \from{bitset.dtx}{package}%
    \from{hobsub.dtx}{post,uniquecounter}%
    \from{uniquecounter.dtx}{package}%
    \from{hobsub.dtx}{post,atend}%
  }%
}
\generate{%
  \usedir{tex/generic/oberdiek}%
  \file{hobsub-hyperref.sty}{%
    \from{hobsub.dtx}{hyperref,preamble,,letltxmacro}%
    \from{letltxmacro.dtx}{package}%
    \from{hobsub.dtx}{post,hopatch}%
    \from{hopatch.dtx}{package}%
    \from{hobsub.dtx}{post,xcolor-patch}%
    \from{hycolor.dtx}{xcolor}%
    \from{hobsub.dtx}{post,atveryend}%
    \from{atveryend.dtx}{package}%
    \from{hobsub.dtx}{post,atbegshi}%
    \from{atbegshi.dtx}{package}%
    \from{hobsub.dtx}{post,refcount}%
    \from{refcount.dtx}{package}%
    \from{hobsub.dtx}{post,hycolor}%
    \from{hycolor.dtx}{package}%
    \from{hobsub.dtx}{post}%
  }%
}
\let\normalLine\hobsubSavedNormalLine

\catcode32=13\relax% active space
\let =\space%
\Msg{************************************************************************}
\Msg{*}
\Msg{* To finish the installation you have to move the following}
\Msg{* files into a directory searched by TeX:}
\Msg{*}
\Msg{*     hobsub.sty, hobsub-generic.sty, hobsub-hyperref.sty}
\Msg{*}
\Msg{* To produce the documentation run the file `hobsub.drv'}
\Msg{* through LaTeX.}
\Msg{*}
\Msg{* Happy TeXing!}
\Msg{*}
\Msg{************************************************************************}

\endbatchfile
%</install>
%<*ignore>
\fi
%</ignore>
%<*driver>
\NeedsTeXFormat{LaTeX2e}
\ProvidesFile{hobsub.drv}%
  [2016/05/16 v1.14 Construct package bundles (HO)]%
\documentclass{ltxdoc}
\usepackage{holtxdoc}[2011/11/22]
\usepackage{array}
\usepackage{caption}
\renewcommand*{\bottomfraction}{.5}
\renewcommand*{\topfraction}{.5}
\begin{document}
  \DocInput{hobsub.dtx}%
\end{document}
%</driver>
% \fi
%
%
% \CharacterTable
%  {Upper-case    \A\B\C\D\E\F\G\H\I\J\K\L\M\N\O\P\Q\R\S\T\U\V\W\X\Y\Z
%   Lower-case    \a\b\c\d\e\f\g\h\i\j\k\l\m\n\o\p\q\r\s\t\u\v\w\x\y\z
%   Digits        \0\1\2\3\4\5\6\7\8\9
%   Exclamation   \!     Double quote  \"     Hash (number) \#
%   Dollar        \$     Percent       \%     Ampersand     \&
%   Acute accent  \'     Left paren    \(     Right paren   \)
%   Asterisk      \*     Plus          \+     Comma         \,
%   Minus         \-     Point         \.     Solidus       \/
%   Colon         \:     Semicolon     \;     Less than     \<
%   Equals        \=     Greater than  \>     Question mark \?
%   Commercial at \@     Left bracket  \[     Backslash     \\
%   Right bracket \]     Circumflex    \^     Underscore    \_
%   Grave accent  \`     Left brace    \{     Vertical bar  \|
%   Right brace   \}     Tilde         \~}
%
% \GetFileInfo{hobsub.drv}
%
% \title{The \xpackage{hobsub} package}
% \date{2016/05/16 v1.14}
% \author{Heiko Oberdiek\thanks
% {Please report any issues at https://github.com/ho-tex/oberdiek/issues}\\
% \xemail{heiko.oberdiek at googlemail.com}}
%
% \maketitle
%
% \begin{abstract}
% Package \xpackage{hobsub} implements the idea of loading
% several packages in one big collection package.
% \end{abstract}
%
% \tableofcontents
%
% \def\TableGeneric{^^A
%   \TableEntry{\xpackage{infwarerr}}{}{2010/04/08 v1.3}^^A
%   \TableEntry{\xpackage{ltxcmds}}{}{2011/11/09 v1.22}^^A
%   \TableEntry{\xpackage{ifluatex}}{}{2010/03/01 v1.3}^^A
%   \TableEntry{\xpackage{ifvtex}}{}{2010/03/01 v1.5}^^A
%   \TableEntry{\xpackage{intcalc}}{}{2007/09/27 v1.1}^^A
%   \TableEntry{\xpackage{ifpdf}}{\xpackage{ifluatex}}{2011/01/30 v2.3}^^A
%   \TableEntry{\xpackage{etexcmds}}{}{2011/02/16 v1.5}^^A
%   \TableEntry{\xpackage{kvsetkeys}}{}{2012/04/25 v1.16}^^A
%   \TableEntry{\xpackage{kvdefinekeys}}{}{2011/04/07 v1.3}^^A
%   \TableEntry{\xpackage{luatex-loader}}{}{2010/03/09 v0.4}^^A
%   \TableEntry{\xpackage{pdftexcmds}}{\xpackage{ifluatex}, \xpackage{ifpdf}, \xpackage{infwarerr}, \xpackage{ltxcmds}, \xpackage{luatex-loader}}{2011/11/29 v0.20}^^A
%   \TableEntry{\xpackage{pdfescape}}{\xpackage{ltxcmds}, \xpackage{pdftexcmds}}{2011/11/25 v1.13}^^A
%   \TableEntry{\xpackage{bigintcalc}}{\xpackage{pdftexcmds}}{2012/04/08 v1.3}^^A
%   \TableEntry{\xpackage{bitset}}{\xpackage{bigintcalc}, \xpackage{infwarerr}, \xpackage{intcalc}}{2011/01/30 v1.1}^^A
%   \TableEntry{\xpackage{uniquecounter}}{\xpackage{bigintcalc}, \xpackage{infwarerr}}{2011/01/30 v1.2}^^A
% }
% \def\TableHyperref{^^A
%   \TableEntry{\xpackage{letltxmacro}}{}{2010/09/02 v1.4}^^A
%   \TableEntry{\xpackage{hopatch}}{}{2016/05/16 v1.2}^^A
%   \TableEntry{\xpackage{xcolor-patch}}{}{2011/01/30 v1.7}^^A
%   \TableEntry{\xpackage{atveryend}}{}{2011/06/30 v1.8}^^A
%   \TableEntry{\xpackage{atbegshi}}{\xpackage{ifpdf}, \xpackage{infwarerr}, \xpackage{ltxcmds}}{2011/10/05 v1.16}^^A
%   \TableEntry{\xpackage{refcount}}{\xpackage{infwarerr}, \xpackage{ltxcmds}}{2011/10/16 v3.4}^^A
%   \TableEntry{\xpackage{hycolor}}{\xpackage{xcolor-patch}}{2011/01/30 v1.7}^^A
% }
%
% \newenvironment{TableList}[2][htbp]{^^A
%   \begin{table}[{#1}]^^A
%   \centering
%   \caption{Overview \xpackage{hobsub-#2}}^^A
%   \label{tab:#2}^^A
%   \begin{tabular}{ll>{\raggedright}p{.4\linewidth}}^^A
%   \bfseries Package & \bfseries Release &
%   \bfseries Dependencies\tabularnewline
%   \hline
% }{^^A
%   \end{tabular}^^A
%   \end{table}^^A
% }
% \newcommand*{\TableEntry}[3]{%
%   #1&#3&#2\tabularnewline
% }
%
% \renewcommand*{\arraystretch}{1.1}
% \setlength{\extrarowheight}{2pt}
%
% \section{Documentation}
%
% In January 2011 the mailing list lualatex-dev discussed
% ``quietening lualatex console output'' \cite{lualatex:quiet}.
% Inspired from this, I implemented this package \xpackage{hobsub}
% to load several packages in one file. The package is applied for
% two packages collections that are subsets of my package bundle
% \xpackage{oberdiek}.
% \begin{description}
% \item[\xpackage{hobsub-generic}] is a collection of packages
%   that can also be used with \hologo{plainTeX}.
% \item[\xpackage{hobsub-hyperref}] is a collection of packages
%   that is used by package \xpackage{hyperref}.
% \end{description}
%
% Package \xpackage{hobsub} provides the macros that
% are used in the collection packages and might be used
% in other projects as well. But there are many caveats,
% some of them are discussed below.
%
% The most serious problem is consistency. A package
% that is available as standalone package and is part
% of a collection might have different versions.
% Because my packages are organized in a bundle and should
% be installed and updated as bundle, the risk seems low
% to me. However, if a single package is updated manually, for
% example for testing, then it should be loaded before the
% collection that contains the package with an older version.
% In a collection, the package will not be loaded again if it
% is already loaded.
%
% \subsection{Usage}
%
% The collection packages \xpackage{hobsub-generic} and
% \xpackage{hobsub-hyperref} are loaded as usual:
% \begin{quote}
%  \begin{tabular}{@{}ll@{}}
%    |\usepackage{hobsub-generic}| & \hologo{LaTeX}\\
%    |\input hobsub-generic.sty| & \hologo{plainTeX}
%  \end{tabular}
% \end{quote}
% or
% \begin{quote}|\usepackage{hobsub-hyperref}|\end{quote}
% Of course, both \cs{usepackage} or \cs{RequirePackage}
% can be used.
%
% If you need requirements on the version date of a package
% inside a collection, then specify them afterwards, for
% example:
% \begin{quote}
%   |\usepackage{hobsub-generic}|\\
%   |\usepackage{ltxcmds}[2010/12/04]% |^^A
%   {\itshape |\ltx@ifblank is needed|}
% \end{quote}
%
% Also it is not garanteed that a package will be always
% part of a collection. Therefore it does not harm,
% to make the requirements explicite \emph{after}
% a collection is loaded, see the previous example.
% \hologo{LaTeX} knows that a
% package is already loaded and does not load it again.
%
% \subsection{Supported features}
%
% \begin{description}
% \item[No reloads:] If a package is already loaded then it
%  is not reloaded again if it is part of a collection package.
% \item[\cs{listfiles}:]
%   Each package that is loaded in a collection is also
%   added to the output of \cs{listfiles}.
% \item[\cs{@currname}, \cs{@currext}]
%   are updated, they are used by \hologo{LaTeX}'s
%  \cs{ProvidesPackage}.
% \item[\cs{AtEndOfPackage}:]
%   \hologo{LaTeX}'s hook mechanism that is used for
%   \cs{AtEndOfPackage} is supported. However none of the
%   packages in the collections are using \cs{AtEndOfPackage},
%   therefore this feature is completely untested.
% \end{description}
%
% \subsection{Common limitations}
%
% \begin{description}
% \item[\cs{endinput}:]
%   Often a package is using \cs{endinput}, either added by
%   \xpackage{docstrip}, or to stop package loading at an earlier
%   time. Therefore \cs{endinput} must be catched during package
%   loading to prevent that the collection package is closed
%   too early before reading all packages it contains.
%
%   This redefinition of \cs{endinput} fails if the package loads
%   other files that contain \cs{endinput}. To some degree it could
%   be catched by redefining macros that load files. But this will
%   not work for the primitive \cs{input}.
% \item[Options:]
%   \hologo{LaTeX}'s package options are not supported.
%   Therefore the packages that are put in a collection must
%   not support options.
% \item[Hook packages:]
%   Some packages and classes provides hooks for packages, see
%   packages \xpackage{scrlfile}, \xpackage{filehook}
%   or class \xclass{memoir}. These hooks are not supported
%   for embedded packages inside a collection package.
%   Because of the problem with \cs{endinput}, these hooks
%   are not supported. The hooks might contain code that
%   loads other files that execute \cs{endinput}.
% \end{description}
%
% \subsection{Package \xpackage{hobsub-generic}}
%
% \begin{TableList}[bp]{generic}\TableGeneric\end{TableList}
%
% Collection `generic' contains some of the \hologo{plainTeX} compatible
% packages of my bundle, see table \ref{tab:generic}.
% It adds the following restrictions to any package
% that is part of this collection:
% \begin{description}
% \item[Generic formats:] The following formats must be supported:
%   \hologo{LaTeX}, \hologo{plainTeX} and even \hologo{iniTeX}.
%   This applies to the collection package \xpackage{hobsub-generic}
%   the same way.
% \item[Prefixed macros:]
%   Any package of this collection should only define macros with
%   prefixed names to avoid name clashes. Resources like registers
%   must not be allocated and other global states should not be
%   changed except for added macro definitions.
% \end{description}
% Thus the package \xpackage{hobsub-generic} can also preloaded
% in the format file with \hologo{iniTeX}.
%
% \subsection{Package \xpackage{hobsub-hyperref}}
%
% \begin{TableList}[tp]{hyperref}\TableHyperref\end{TableList}
%
% Collection `hyperref' contains additional packages that are
% used by package \xpackage{hyperref} and that does not fit
% in collection `generic', see table{tab:hyperref}.
% The latter collection package
% \xpackage{hobsub-generic} is loaded via a conventional
% \cs{RequirePackage}.
%
% Because most of the packages in the collection `hyperref'
% needs \hologo{LaTeX}, the collection requires \hologo{LaTeX}.
%
% \subsection{Package \xpackage{hobsub}}
%
% This section is for advanced users that might want to
% build a collection package. Then it can be done by
% using \xpackage{hobsub}:
% \begin{quote}
%   \dots\\
%   |\RequirePackage{hobsub}[2016/05/16]|\\
%   \dots\\
%   |\hobsub@StartPackage{foo1}|\\
%   \textit{\dots\unskip\ package code of \xfile{foo1.sty} \dots}\\
%   |\endinput|\\
%   |<hobsub>|\\
%   |\hobsub@StopPackage|\\
%   |\hobsub@StartPackage{bar2}|\\
%   \textit{\dots\unskip\ package code of \xfile{bar2.sty} \dots}\\
%   |\endinput|\\
%   |<hobsub>|\\
%   |\hobsub@StopPackage|
%   \dots
% \end{quote}
% But be aware of the caveats and limitations, see above.
% Only quite a small portion of the packages can be embedded
% like this.
%
% \clearpage
%
% \StopEventually{
% }
%
% \section{Implementation}
%
% \subsection{Reload check and package identification}
%
% \subsubsection{Package \xpackage{hobsub}}
%
%    \begin{macrocode}
%<*package>
%    \end{macrocode}
%    Reload check, especially if the package is not used with \LaTeX.
%    \begin{macrocode}
\begingroup\catcode61\catcode48\catcode32=10\relax%
  \catcode13=5 % ^^M
  \endlinechar=13 %
  \catcode35=6 % #
  \catcode39=12 % '
  \catcode44=12 % ,
  \catcode45=12 % -
  \catcode46=12 % .
  \catcode58=12 % :
  \catcode64=11 % @
  \catcode123=1 % {
  \catcode125=2 % }
  \expandafter\let\expandafter\x\csname ver@hobsub.sty\endcsname
  \ifx\x\relax % plain-TeX, first loading
  \else
    \def\empty{}%
    \ifx\x\empty % LaTeX, first loading,
      % variable is initialized, but \ProvidesPackage not yet seen
    \else
      \expandafter\ifx\csname PackageInfo\endcsname\relax
        \def\x#1#2{%
          \immediate\write-1{Package #1 Info: #2.}%
        }%
      \else
        \def\x#1#2{\PackageInfo{#1}{#2, stopped}}%
      \fi
      \x{hobsub}{The package is already loaded}%
      \aftergroup\endinput
    \fi
  \fi
\endgroup%
%    \end{macrocode}
%    Package identification:
%    \begin{macrocode}
\begingroup\catcode61\catcode48\catcode32=10\relax%
  \catcode13=5 % ^^M
  \endlinechar=13 %
  \catcode35=6 % #
  \catcode39=12 % '
  \catcode40=12 % (
  \catcode41=12 % )
  \catcode44=12 % ,
  \catcode45=12 % -
  \catcode46=12 % .
  \catcode47=12 % /
  \catcode58=12 % :
  \catcode64=11 % @
  \catcode91=12 % [
  \catcode93=12 % ]
  \catcode123=1 % {
  \catcode125=2 % }
  \expandafter\ifx\csname ProvidesPackage\endcsname\relax
    \def\x#1#2#3[#4]{\endgroup
      \immediate\write-1{Package: #3 #4}%
      \xdef#1{#4}%
    }%
  \else
    \def\x#1#2[#3]{\endgroup
      #2[{#3}]%
      \ifx#1\@undefined
        \xdef#1{#3}%
      \fi
      \ifx#1\relax
        \xdef#1{#3}%
      \fi
    }%
  \fi
\expandafter\x\csname ver@hobsub.sty\endcsname
\ProvidesPackage{hobsub}%
  [2016/05/16 v1.14 Construct package bundles (HO)]%
%    \end{macrocode}
%    \begin{macrocode}
%</package>
%    \end{macrocode}
%
% \subsubsection{Package \xpackage{hobsub-generic}}
%
%    \begin{macrocode}
%<*generic>
%    \end{macrocode}
%    Reload check, especially if the package is not used with \LaTeX.
%    \begin{macrocode}
\begingroup\catcode61\catcode48\catcode32=10\relax%
  \catcode13=5 % ^^M
  \endlinechar=13 %
  \catcode35=6 % #
  \catcode39=12 % '
  \catcode44=12 % ,
  \catcode45=12 % -
  \catcode46=12 % .
  \catcode58=12 % :
  \catcode64=11 % @
  \catcode123=1 % {
  \catcode125=2 % }
  \expandafter\let\expandafter\x\csname ver@hobsub-generic.sty\endcsname
  \ifx\x\relax % plain-TeX, first loading
  \else
    \def\empty{}%
    \ifx\x\empty % LaTeX, first loading,
      % variable is initialized, but \ProvidesPackage not yet seen
    \else
      \expandafter\ifx\csname PackageInfo\endcsname\relax
        \def\x#1#2{%
          \immediate\write-1{Package #1 Info: #2.}%
        }%
      \else
        \def\x#1#2{\PackageInfo{#1}{#2, stopped}}%
      \fi
      \x{hobsub-generic}{The package is already loaded}%
      \aftergroup\endinput
    \fi
  \fi
\endgroup%
%    \end{macrocode}
%    Package identification:
%    \begin{macrocode}
\begingroup\catcode61\catcode48\catcode32=10\relax%
  \catcode13=5 % ^^M
  \endlinechar=13 %
  \catcode35=6 % #
  \catcode39=12 % '
  \catcode40=12 % (
  \catcode41=12 % )
  \catcode44=12 % ,
  \catcode45=12 % -
  \catcode46=12 % .
  \catcode47=12 % /
  \catcode58=12 % :
  \catcode64=11 % @
  \catcode91=12 % [
  \catcode93=12 % ]
  \catcode123=1 % {
  \catcode125=2 % }
  \expandafter\ifx\csname ProvidesPackage\endcsname\relax
    \def\x#1#2#3[#4]{\endgroup
      \immediate\write-1{Package: #3 #4}%
      \xdef#1{#4}%
    }%
  \else
    \def\x#1#2[#3]{\endgroup
      #2[{#3}]%
      \ifx#1\@undefined
        \xdef#1{#3}%
      \fi
      \ifx#1\relax
        \xdef#1{#3}%
      \fi
    }%
  \fi
\expandafter\x\csname ver@hobsub-generic.sty\endcsname
\ProvidesPackage{hobsub-generic}%
  [2016/05/16 v1.14 Bundle oberdiek, subset generic (HO)]%
%    \end{macrocode}
%    \begin{macrocode}
%</generic>
%    \end{macrocode}
%
% \subsubsection{Package \xpackage{hobsub-hyperref}}
%
%    \begin{macrocode}
%<*hyperref>
%    \end{macrocode}
%    Reload check, especially if the package is not used with \LaTeX.
%    \begin{macrocode}
\begingroup\catcode61\catcode48\catcode32=10\relax%
  \catcode13=5 % ^^M
  \endlinechar=13 %
  \catcode35=6 % #
  \catcode39=12 % '
  \catcode44=12 % ,
  \catcode45=12 % -
  \catcode46=12 % .
  \catcode58=12 % :
  \catcode64=11 % @
  \catcode123=1 % {
  \catcode125=2 % }
  \expandafter\let\expandafter\x\csname ver@hobsub-hyperref.sty\endcsname
  \ifx\x\relax % plain-TeX, first loading
  \else
    \def\empty{}%
    \ifx\x\empty % LaTeX, first loading,
      % variable is initialized, but \ProvidesPackage not yet seen
    \else
      \expandafter\ifx\csname PackageInfo\endcsname\relax
        \def\x#1#2{%
          \immediate\write-1{Package #1 Info: #2.}%
        }%
      \else
        \def\x#1#2{\PackageInfo{#1}{#2, stopped}}%
      \fi
      \x{hobsub-hyperref}{The package is already loaded}%
      \aftergroup\endinput
    \fi
  \fi
\endgroup%
%    \end{macrocode}
%    Package identification:
%    \begin{macrocode}
\begingroup\catcode61\catcode48\catcode32=10\relax%
  \catcode13=5 % ^^M
  \endlinechar=13 %
  \catcode35=6 % #
  \catcode39=12 % '
  \catcode40=12 % (
  \catcode41=12 % )
  \catcode44=12 % ,
  \catcode45=12 % -
  \catcode46=12 % .
  \catcode47=12 % /
  \catcode58=12 % :
  \catcode64=11 % @
  \catcode91=12 % [
  \catcode93=12 % ]
  \catcode123=1 % {
  \catcode125=2 % }
  \expandafter\ifx\csname ProvidesPackage\endcsname\relax
    \def\x#1#2#3[#4]{\endgroup
      \immediate\write-1{Package: #3 #4}%
      \xdef#1{#4}%
    }%
  \else
    \def\x#1#2[#3]{\endgroup
      #2[{#3}]%
      \ifx#1\@undefined
        \xdef#1{#3}%
      \fi
      \ifx#1\relax
        \xdef#1{#3}%
      \fi
    }%
  \fi
\expandafter\x\csname ver@hobsub-hyperref.sty\endcsname
\ProvidesPackage{hobsub-hyperref}%
  [2016/05/16 v1.14 Bundle oberdiek, subset hyperref (HO)]%
%    \end{macrocode}
%    \begin{macrocode}
%</hyperref>
%    \end{macrocode}
%
% \subsection{Catcodes}
%
%    \begin{macrocode}
%<*package|generic|hyperref>
\begingroup\catcode61\catcode48\catcode32=10\relax%
  \catcode13=5 % ^^M
  \endlinechar=13 %
  \catcode123=1 % {
  \catcode125=2 % }
  \catcode64=11 % @
  \def\x{\endgroup
    \expandafter\edef\csname HOBsub@AtEnd\endcsname{%
      \endlinechar=\the\endlinechar\relax
      \catcode13=\the\catcode13\relax
      \catcode32=\the\catcode32\relax
      \catcode35=\the\catcode35\relax
      \catcode61=\the\catcode61\relax
      \catcode64=\the\catcode64\relax
      \catcode123=\the\catcode123\relax
      \catcode125=\the\catcode125\relax
    }%
  }%
\x\catcode61\catcode48\catcode32=10\relax%
\catcode13=5 % ^^M
\endlinechar=13 %
\catcode35=6 % #
\catcode64=11 % @
\catcode123=1 % {
\catcode125=2 % }
\def\TMP@EnsureCode#1#2{%
  \edef\HOBsub@AtEnd{%
    \HOBsub@AtEnd
    \catcode#1=\the\catcode#1\relax
  }%
  \catcode#1=#2\relax
}
\TMP@EnsureCode{39}{12}% '
\TMP@EnsureCode{40}{12}% (
\TMP@EnsureCode{41}{12}% )
\TMP@EnsureCode{45}{12}% -
\TMP@EnsureCode{46}{12}% .
\TMP@EnsureCode{47}{12}% /
\TMP@EnsureCode{58}{12}% :
\TMP@EnsureCode{60}{12}% <
\TMP@EnsureCode{62}{12}% >
\TMP@EnsureCode{96}{12}% `
\edef\HOBsub@AtEnd{\HOBsub@AtEnd\noexpand\endinput}
%    \end{macrocode}
%
%    \begin{macrocode}
%</package|generic|hyperref>
%    \end{macrocode}
%
% \subsection{Package \xpackage{hobsub-hyperref} loads \xpackage{hobsub-generic}}
%
%    \begin{macrocode}
%<*hyperref>
\NeedsTeXFormat{LaTeX2e}
\RequirePackage{hobsub-generic}[2016/05/16]
%</hyperref>
%    \end{macrocode}
%
% \subsection{Preamble}
%
%    \begin{macrocode}
%<*preamble>
\begingroup\expandafter\expandafter\expandafter\endgroup
\expandafter\ifx\csname ver@hobsub.sty\endcsname\relax
\else
  \expandafter\hobsub@GobbleRemainingPackage
\fi
%</preamble>
%    \end{macrocode}
%
%
%    \begin{macrocode}
%<*package|preamble>
%    \end{macrocode}
%
%    \begin{macro}{\HOBsub@OrgEndinput}
%    \begin{macrocode}
\begingroup\expandafter\expandafter\expandafter\endgroup
\expandafter\ifx\csname HOBsub@OrgEndinput\endcsname\relax
  \let\HOBsub@OrgEndinput\endinput
\fi
%    \end{macrocode}
%    \end{macro}
%    \begin{macro}{\hobsub@GobbleRemainingPackage}
%    \begin{macrocode}
\def\hobsub@GobbleRemainingPackage{%
  \begingroup
  \catcode92=14 % backslash: comment
  \catcode32=14 % space: comment
  \catcode35=14 % hash: comment
  \catcode123=14 % left brace: comment
  \catcode125=14 % right brace: comment
  \catcode60=3 % less: math
  \catcode62=4 % greater: align
  \endlinechar=-1 %
  \HOBsub@GobbleRemainingPackage
}
%    \end{macrocode}
%    \end{macro}
%    \begin{macro}{\HOBsub@GobbleRemainingPackage}
%    \begin{macrocode}
\catcode60=3 % less: dollar
\catcode62=4 % greater: align
\long\def\HOBsub@GobbleRemainingPackage#1<hobsub>{%
  \endgroup
}
\catcode60=12 % less: other
\catcode62=12 % greater: other
%    \end{macrocode}
%    \end{macro}
%
%    \begin{macro}{\hobsub@StartPackage}
%    \begin{macrocode}
\def\hobsub@StartPackage#1{%
  \begingroup\expandafter\expandafter\expandafter\endgroup
  \expandafter\ifx\csname ver@#1.sty\endcsname\relax
    \let\HOBsub@OrgCurrName\@currname
    \let\HOBsub@OrgCurrExt\@currext
    \csname @pushfilename\endcsname
    \def\@currname{#1}%
    \def\@currext{sty}%
    \expandafter\def\csname\@currname.\@currext-h@@k\endcsname{}%
    \let\endinput\hobsub@GobbleRemainingPackage
    \def\hobsub@StopPackage{%
      \let\hobsub@StopPackage\relax
      \HOBsub@StopPackage
%     \hobsub@Info{hobsub}{Package `#1' loaded}%
    }%
    \hobsub@AddToFileList{#1.sty}%
  \else
    \hobsub@Info{hobsub}{Skipping package `#1' (already loaded)}%
    \let\hobsub@StopPackage\relax
    \expandafter\hobsub@GobbleRemainingPackage
  \fi
}
%    \end{macrocode}
%    \end{macro}
%
%    \begin{macro}{\hobsub@StopPackage}
%    \begin{macrocode}
\let\hobsub@StopPackage\relax
%    \end{macrocode}
%    \end{macro}
%
%    \begin{macro}{\hobsub@Info}
%    \begin{macrocode}
\def\hobsub@Info#1#2{%
  \begingroup\expandafter\expandafter\expandafter\endgroup
  \expandafter\ifx\csname @PackageInfoNoLine\endcsname\relax
    \immediate\write-1{Package #1 Info: #2.}%
  \else
    \let\hobsub@Info\@PackageInfoNoLine
    \hobsub@Info{#1}{#2}%
  \fi
}
%    \end{macrocode}
%    \end{macro}
%
%    \begin{macro}{\HOBsub@StopPackage}
%    \begin{macrocode}
\def\HOBsub@StopPackage{%
  \csname\@currname.\@currext-h@@k\endcsname
  \let\endinput\HOBsub@OrgEndinput
  \csname @popfilename\endcsname
  \let\@currname\HOBsub@OrgCurrName
  \let\@currext\HOBsub@OrgCurrExt
}
%    \end{macrocode}
%    \end{macro}
%
%    \begin{macro}{\hobsub@AddToFileList}
%    \begin{macrocode}
\begingroup\expandafter\expandafter\expandafter\endgroup
\expandafter\ifx\csname @addtofilelist\endcsname\relax
  \def\hobsub@AddToFileList#1{}%
\else
  \def\hobsub@AddToFileList#1{%
    \@addtofilelist{#1}%
  }%
\fi
%    \end{macrocode}
%    \end{macro}
%
%    \begin{macrocode}
%</package|preamble>
%    \end{macrocode}
%
% \subsection{End of preamble for subset packages}
%
%    Add the end marker for possible skipping of previous
%    macros, if package \xpackage{hobsub} was already loaded.
%    \begin{macrocode}
%<*preamble>
\hobsub@GobbleRemainingPackage
<hobsub>
%    \end{macrocode}
%
%    All macros of package \xpackage{hobsub} are now available.
%    Now we officially load the package \xpackage{hobsub} inline
%    for proper package identification.
%    \begin{macrocode}
\hobsub@StartPackage{hobsub}%
\begingroup\expandafter\expandafter\expandafter\endgroup
\expandafter\ifx\csname ProvidesPackage\endcsname\relax
  \immediate\write-1{%
    Package: hobsub 2016/05/16 v1.14 Construct package bundles (HO)%
  }%
\else
  \ProvidesPackage{hobsub}%
  [2016/05/16 v1.14 Construct package bundles (HO)]%
\fi
%</preamble>
%    \end{macrocode}
%
%
%    \begin{macrocode}
%<*post|preamble>
 \endinput
<hobsub>
\hobsub@StopPackage
%</post|preamble>
%    \end{macrocode}
%
% \subsection{Package list}
%
%    \begin{macrocode}
%<infwarerr>\hobsub@StartPackage{infwarerr}
%<ltxcmds>\hobsub@StartPackage{ltxcmds}
%<ifluatex>\hobsub@StartPackage{ifluatex}
%<ifvtex>\hobsub@StartPackage{ifvtex}
%<intcalc>\hobsub@StartPackage{intcalc}
%<ifpdf>\hobsub@StartPackage{ifpdf}
%<etexcmds>\hobsub@StartPackage{etexcmds}
%<kvsetkeys>\hobsub@StartPackage{kvsetkeys}
%<kvdefinekeys>\hobsub@StartPackage{kvdefinekeys}
%<*luatex-loader>
\ifluatex
\else
  \expandafter\hobsub@GobbleRemainingPackage
\fi
\hobsub@StartPackage{luatex-loader}
%</luatex-loader>
%<pdftexcmds>\hobsub@StartPackage{pdftexcmds}
%<pdfescape>\hobsub@StartPackage{pdfescape}
%<bigintcalc>\hobsub@StartPackage{bigintcalc}
%<bitset>\hobsub@StartPackage{bitset}
%<uniquecounter>\hobsub@StartPackage{uniquecounter}
%<letltxmacro>\hobsub@StartPackage{letltxmacro}
%<hopatch>\hobsub@StartPackage{hopatch}
%<xcolor-patch>\hobsub@StartPackage{xcolor-patch}
%<atveryend>\hobsub@StartPackage{atveryend}
%<atbegshi>\hobsub@StartPackage{atbegshi}
%<refcount>\hobsub@StartPackage{refcount}
%<hycolor>\hobsub@StartPackage{hycolor}
%    \end{macrocode}
%
% \subsection{End of packages}
%
%    \begin{macrocode}
%<package|atend>\HOBsub@AtEnd%
%    \end{macrocode}
%
% \section{Test}
%
% \subsection{Catcode checks for loading}
%
%    \begin{macrocode}
%<*test1>
%    \end{macrocode}
%    \begin{macrocode}
\catcode`\{=1 %
\catcode`\}=2 %
\catcode`\#=6 %
\catcode`\@=11 %
\expandafter\ifx\csname count@\endcsname\relax
  \countdef\count@=255 %
\fi
\expandafter\ifx\csname @gobble\endcsname\relax
  \long\def\@gobble#1{}%
\fi
\expandafter\ifx\csname @firstofone\endcsname\relax
  \long\def\@firstofone#1{#1}%
\fi
\expandafter\ifx\csname loop\endcsname\relax
  \expandafter\@firstofone
\else
  \expandafter\@gobble
\fi
{%
  \def\loop#1\repeat{%
    \def\body{#1}%
    \iterate
  }%
  \def\iterate{%
    \body
      \let\next\iterate
    \else
      \let\next\relax
    \fi
    \next
  }%
  \let\repeat=\fi
}%
\def\RestoreCatcodes{}
\count@=0 %
\loop
  \edef\RestoreCatcodes{%
    \RestoreCatcodes
    \catcode\the\count@=\the\catcode\count@\relax
  }%
\ifnum\count@<255 %
  \advance\count@ 1 %
\repeat

\def\RangeCatcodeInvalid#1#2{%
  \count@=#1\relax
  \loop
    \catcode\count@=15 %
  \ifnum\count@<#2\relax
    \advance\count@ 1 %
  \repeat
}
\def\RangeCatcodeCheck#1#2#3{%
  \count@=#1\relax
  \loop
    \ifnum#3=\catcode\count@
    \else
      \errmessage{%
        Character \the\count@\space
        with wrong catcode \the\catcode\count@\space
        instead of \number#3%
      }%
    \fi
  \ifnum\count@<#2\relax
    \advance\count@ 1 %
  \repeat
}
\def\space{ }
\expandafter\ifx\csname LoadCommand\endcsname\relax
  \def\LoadCommand{\input hobsub.sty\relax}%
\fi
\def\Test{%
  \RangeCatcodeInvalid{0}{47}%
  \RangeCatcodeInvalid{58}{64}%
  \RangeCatcodeInvalid{91}{96}%
  \RangeCatcodeInvalid{123}{255}%
  \catcode`\@=12 %
  \catcode`\\=0 %
  \catcode`\%=14 %
  \LoadCommand
  \RangeCatcodeCheck{0}{36}{15}%
  \RangeCatcodeCheck{37}{37}{14}%
  \RangeCatcodeCheck{38}{47}{15}%
  \RangeCatcodeCheck{48}{57}{12}%
  \RangeCatcodeCheck{58}{63}{15}%
  \RangeCatcodeCheck{64}{64}{12}%
  \RangeCatcodeCheck{65}{90}{11}%
  \RangeCatcodeCheck{91}{91}{15}%
  \RangeCatcodeCheck{92}{92}{0}%
  \RangeCatcodeCheck{93}{96}{15}%
  \RangeCatcodeCheck{97}{122}{11}%
  \RangeCatcodeCheck{123}{255}{15}%
  \RestoreCatcodes
}
\Test
\csname @@end\endcsname
\end
%    \end{macrocode}
%    \begin{macrocode}
%</test1>
%    \end{macrocode}
%
% \section{Installation}
%
% \subsection{Download}
%
% \paragraph{Package.} This package is available on
% CTAN\footnote{\url{http://ctan.org/pkg/hobsub}}:
% \begin{description}
% \item[\CTAN{macros/latex/contrib/oberdiek/hobsub.dtx}] The source file.
% \item[\CTAN{macros/latex/contrib/oberdiek/hobsub.pdf}] Documentation.
% \end{description}
%
%
% \paragraph{Bundle.} All the packages of the bundle `oberdiek'
% are also available in a TDS compliant ZIP archive. There
% the packages are already unpacked and the documentation files
% are generated. The files and directories obey the TDS standard.
% \begin{description}
% \item[\CTAN{install/macros/latex/contrib/oberdiek.tds.zip}]
% \end{description}
% \emph{TDS} refers to the standard ``A Directory Structure
% for \TeX\ Files'' (\CTAN{tds/tds.pdf}). Directories
% with \xfile{texmf} in their name are usually organized this way.
%
% \subsection{Bundle installation}
%
% \paragraph{Unpacking.} Unpack the \xfile{oberdiek.tds.zip} in the
% TDS tree (also known as \xfile{texmf} tree) of your choice.
% Example (linux):
% \begin{quote}
%   |unzip oberdiek.tds.zip -d ~/texmf|
% \end{quote}
%
% \paragraph{Script installation.}
% Check the directory \xfile{TDS:scripts/oberdiek/} for
% scripts that need further installation steps.
% Package \xpackage{attachfile2} comes with the Perl script
% \xfile{pdfatfi.pl} that should be installed in such a way
% that it can be called as \texttt{pdfatfi}.
% Example (linux):
% \begin{quote}
%   |chmod +x scripts/oberdiek/pdfatfi.pl|\\
%   |cp scripts/oberdiek/pdfatfi.pl /usr/local/bin/|
% \end{quote}
%
% \subsection{Package installation}
%
% \paragraph{Unpacking.} The \xfile{.dtx} file is a self-extracting
% \docstrip\ archive. The files are extracted by running the
% \xfile{.dtx} through \plainTeX:
% \begin{quote}
%   \verb|tex hobsub.dtx|
% \end{quote}
%
% \paragraph{TDS.} Now the different files must be moved into
% the different directories in your installation TDS tree
% (also known as \xfile{texmf} tree):
% \begin{quote}
% \def\t{^^A
% \begin{tabular}{@{}>{\ttfamily}l@{ $\rightarrow$ }>{\ttfamily}l@{}}
%   hobsub.sty & tex/generic/oberdiek/hobsub.sty\\
%   hobsub-generic.sty & tex/generic/oberdiek/hobsub-generic.sty\\
%   hobsub-hyperref.sty & tex/generic/oberdiek/hobsub-hyperref.sty\\
%   hobsub.pdf & doc/latex/oberdiek/hobsub.pdf\\
%   test/hobsub-test1.tex & doc/latex/oberdiek/test/hobsub-test1.tex\\
%   hobsub.dtx & source/latex/oberdiek/hobsub.dtx\\
% \end{tabular}^^A
% }^^A
% \sbox0{\t}^^A
% \ifdim\wd0>\linewidth
%   \begingroup
%     \advance\linewidth by\leftmargin
%     \advance\linewidth by\rightmargin
%   \edef\x{\endgroup
%     \def\noexpand\lw{\the\linewidth}^^A
%   }\x
%   \def\lwbox{^^A
%     \leavevmode
%     \hbox to \linewidth{^^A
%       \kern-\leftmargin\relax
%       \hss
%       \usebox0
%       \hss
%       \kern-\rightmargin\relax
%     }^^A
%   }^^A
%   \ifdim\wd0>\lw
%     \sbox0{\small\t}^^A
%     \ifdim\wd0>\linewidth
%       \ifdim\wd0>\lw
%         \sbox0{\footnotesize\t}^^A
%         \ifdim\wd0>\linewidth
%           \ifdim\wd0>\lw
%             \sbox0{\scriptsize\t}^^A
%             \ifdim\wd0>\linewidth
%               \ifdim\wd0>\lw
%                 \sbox0{\tiny\t}^^A
%                 \ifdim\wd0>\linewidth
%                   \lwbox
%                 \else
%                   \usebox0
%                 \fi
%               \else
%                 \lwbox
%               \fi
%             \else
%               \usebox0
%             \fi
%           \else
%             \lwbox
%           \fi
%         \else
%           \usebox0
%         \fi
%       \else
%         \lwbox
%       \fi
%     \else
%       \usebox0
%     \fi
%   \else
%     \lwbox
%   \fi
% \else
%   \usebox0
% \fi
% \end{quote}
% If you have a \xfile{docstrip.cfg} that configures and enables \docstrip's
% TDS installing feature, then some files can already be in the right
% place, see the documentation of \docstrip.
%
% \subsection{Refresh file name databases}
%
% If your \TeX~distribution
% (\teTeX, \mikTeX, \dots) relies on file name databases, you must refresh
% these. For example, \teTeX\ users run \verb|texhash| or
% \verb|mktexlsr|.
%
% \subsection{Some details for the interested}
%
% \paragraph{Attached source.}
%
% The PDF documentation on CTAN also includes the
% \xfile{.dtx} source file. It can be extracted by
% AcrobatReader 6 or higher. Another option is \textsf{pdftk},
% e.g. unpack the file into the current directory:
% \begin{quote}
%   \verb|pdftk hobsub.pdf unpack_files output .|
% \end{quote}
%
% \paragraph{Unpacking with \LaTeX.}
% The \xfile{.dtx} chooses its action depending on the format:
% \begin{description}
% \item[\plainTeX:] Run \docstrip\ and extract the files.
% \item[\LaTeX:] Generate the documentation.
% \end{description}
% If you insist on using \LaTeX\ for \docstrip\ (really,
% \docstrip\ does not need \LaTeX), then inform the autodetect routine
% about your intention:
% \begin{quote}
%   \verb|latex \let\install=y% \iffalse meta-comment
%
% File: hobsub.dtx
% Version: 2016/05/16 v1.14
% Info: Construct package bundles
%
% Copyright (C) 2011, 2012 by
%    Heiko Oberdiek <heiko.oberdiek at googlemail.com>
%    2016
%    https://github.com/ho-tex/oberdiek/issues
%
% This work may be distributed and/or modified under the
% conditions of the LaTeX Project Public License, either
% version 1.3c of this license or (at your option) any later
% version. This version of this license is in
%    http://www.latex-project.org/lppl/lppl-1-3c.txt
% and the latest version of this license is in
%    http://www.latex-project.org/lppl.txt
% and version 1.3 or later is part of all distributions of
% LaTeX version 2005/12/01 or later.
%
% This work has the LPPL maintenance status "maintained".
%
% This Current Maintainer of this work is Heiko Oberdiek.
%
% The Base Interpreter refers to any `TeX-Format',
% because some files are installed in TDS:tex/generic//.
%
% This work consists of the main source file hobsub.dtx
% and the derived files
%    hobsub.sty, hobsub.pdf, hobsub.ins, hobsub.drv, hobsub-generic.sty,
%    hobsub-hyperref.sty, hobsub-test1.tex.
%
% Distribution:
%    CTAN:macros/latex/contrib/oberdiek/hobsub.dtx
%    CTAN:macros/latex/contrib/oberdiek/hobsub.pdf
%
% Unpacking:
%    (a) If hobsub.ins is present:
%           tex hobsub.ins
%    (b) Without hobsub.ins:
%           tex hobsub.dtx
%    (c) If you insist on using LaTeX
%           latex \let\install=y% \iffalse meta-comment
%
% File: hobsub.dtx
% Version: 2016/05/16 v1.14
% Info: Construct package bundles
%
% Copyright (C) 2011, 2012 by
%    Heiko Oberdiek <heiko.oberdiek at googlemail.com>
%    2016
%    https://github.com/ho-tex/oberdiek/issues
%
% This work may be distributed and/or modified under the
% conditions of the LaTeX Project Public License, either
% version 1.3c of this license or (at your option) any later
% version. This version of this license is in
%    http://www.latex-project.org/lppl/lppl-1-3c.txt
% and the latest version of this license is in
%    http://www.latex-project.org/lppl.txt
% and version 1.3 or later is part of all distributions of
% LaTeX version 2005/12/01 or later.
%
% This work has the LPPL maintenance status "maintained".
%
% This Current Maintainer of this work is Heiko Oberdiek.
%
% The Base Interpreter refers to any `TeX-Format',
% because some files are installed in TDS:tex/generic//.
%
% This work consists of the main source file hobsub.dtx
% and the derived files
%    hobsub.sty, hobsub.pdf, hobsub.ins, hobsub.drv, hobsub-generic.sty,
%    hobsub-hyperref.sty, hobsub-test1.tex.
%
% Distribution:
%    CTAN:macros/latex/contrib/oberdiek/hobsub.dtx
%    CTAN:macros/latex/contrib/oberdiek/hobsub.pdf
%
% Unpacking:
%    (a) If hobsub.ins is present:
%           tex hobsub.ins
%    (b) Without hobsub.ins:
%           tex hobsub.dtx
%    (c) If you insist on using LaTeX
%           latex \let\install=y\input{hobsub.dtx}
%        (quote the arguments according to the demands of your shell)
%
% Documentation:
%    (a) If hobsub.drv is present:
%           latex hobsub.drv
%    (b) Without hobsub.drv:
%           latex hobsub.dtx; ...
%    The class ltxdoc loads the configuration file ltxdoc.cfg
%    if available. Here you can specify further options, e.g.
%    use A4 as paper format:
%       \PassOptionsToClass{a4paper}{article}
%
%    Programm calls to get the documentation (example):
%       pdflatex hobsub.dtx
%       makeindex -s gind.ist hobsub.idx
%       pdflatex hobsub.dtx
%       makeindex -s gind.ist hobsub.idx
%       pdflatex hobsub.dtx
%
% Installation:
%    TDS:tex/generic/oberdiek/hobsub.sty
%    TDS:tex/generic/oberdiek/hobsub-generic.sty
%    TDS:tex/generic/oberdiek/hobsub-hyperref.sty
%    TDS:doc/latex/oberdiek/hobsub.pdf
%    TDS:doc/latex/oberdiek/test/hobsub-test1.tex
%    TDS:source/latex/oberdiek/hobsub.dtx
%
%<*ignore>
\begingroup
  \catcode123=1 %
  \catcode125=2 %
  \def\x{LaTeX2e}%
\expandafter\endgroup
\ifcase 0\ifx\install y1\fi\expandafter
         \ifx\csname processbatchFile\endcsname\relax\else1\fi
         \ifx\fmtname\x\else 1\fi\relax
\else\csname fi\endcsname
%</ignore>
%<*install>
\input docstrip.tex
\Msg{************************************************************************}
\Msg{* Installation}
\Msg{* Package: hobsub 2016/05/16 v1.14 Construct package bundles (HO)}
\Msg{************************************************************************}

\keepsilent
\askforoverwritefalse

\let\MetaPrefix\relax
\preamble

This is a generated file.

Project: hobsub
Version: 2016/05/16 v1.14

Copyright (C) 2011, 2012 by
   Heiko Oberdiek <heiko.oberdiek at googlemail.com>

This work may be distributed and/or modified under the
conditions of the LaTeX Project Public License, either
version 1.3c of this license or (at your option) any later
version. This version of this license is in
   http://www.latex-project.org/lppl/lppl-1-3c.txt
and the latest version of this license is in
   http://www.latex-project.org/lppl.txt
and version 1.3 or later is part of all distributions of
LaTeX version 2005/12/01 or later.

This work has the LPPL maintenance status "maintained".

This Current Maintainer of this work is Heiko Oberdiek.

The Base Interpreter refers to any `TeX-Format',
because some files are installed in TDS:tex/generic//.

This work consists of the main source file hobsub.dtx
and the derived files
   hobsub.sty, hobsub.pdf, hobsub.ins, hobsub.drv, hobsub-generic.sty,
   hobsub-hyperref.sty, hobsub-test1.tex.

\endpreamble
\let\MetaPrefix\DoubleperCent

\generate{%
  \file{hobsub.ins}{\from{hobsub.dtx}{install}}%
  \file{hobsub.drv}{\from{hobsub.dtx}{driver}}%
  \usedir{tex/generic/oberdiek}%
  \file{hobsub.sty}{\from{hobsub.dtx}{package}}%
  \usedir{doc/latex/oberdiek/test}%
  \file{hobsub-test1.tex}{\from{hobsub.dtx}{test1}}%
  \nopreamble
  \nopostamble
  \usedir{source/latex/oberdiek/catalogue}%
  \file{hobsub.xml}{\from{hobsub.dtx}{catalogue}}%
}
\let\hobsubSavedNormalLine\normalLine
\def\hobsubGobble#1{}
\begingroup
  \lccode`\0=32 % space
  \lccode`\1=35 % hash
  \lccode`\2=37 % percent
  \lccode`\3=92 % backslash
  \lccode`\4=123 % {
  \lccode`\5=125 % }
\lowercase{\endgroup
  \let\hobsubTokenSpaceOther= 0%
  \let\hobsubTokenHash= 1%
  \let\hobsubTokenPercent= 2%
  \let\hobsubTokenBackslash= 3%
  \let\hobsubTokenBraceLeft= 4%
  \let\hobsubTokenBraceRight= 5%
  \def\hobsubNormalLineWithSpace{%
    \hobsubSavedNormalLine0%
  }%
}
\def\normalLine{%
  \futurelet\hobsubToken\hobsubFirst
}
\def\hobsubFirst{%
  \ifx\hobsubToken\hobsubTokenSpaceOther
    \expandafter\expandafter\expandafter\futurelet
    \expandafter\expandafter\expandafter\hobsubToken
    \expandafter\expandafter\expandafter\hobsubNext
    \expandafter\hobsubGobble
  \else
    \expandafter\hobsubSavedNormalLine
  \fi
}
\def\hobsubNext{%
  \ifx\hobsubToken\hobsubTokenSpaceOther
    \expandafter\expandafter\expandafter\futurelet
    \expandafter\expandafter\expandafter\hobsubToken
    \expandafter\expandafter\expandafter\hobsubNext 
    \expandafter\hobsubGobble
  \else
    \let\hobsubWithSpace=Y%
    \ifx\hobsubToken\hobsubTokenBackslash
      \let\hobsubWithSpace=N%
    \else
      \ifx\hobsubToken\hobsubTokenBraceRight
        \let\hobsubWithSpace=N%
      \else
        \ifx\hobsubToken\hobsubTokenBraceLeft
          \let\hobsubWithSpace=N%
        \else
          \ifx\hobsubToken\hobsubTokenPercent
            \let\hobsubWithSpace=N%
          \else
            \ifx\hobsubToken\hobsubTokenHash
              \let\hobsubWithSpace=N%
            \fi
          \fi
        \fi
      \fi
    \fi
    \ifx\hobsubWithSpace Y%
      \expandafter\expandafter\expandafter\hobsubNormalLineWithSpace
    \else
      \expandafter\expandafter\expandafter\hobsubSavedNormalLine
    \fi
  \fi
}
\generate{%
  \usedir{tex/generic/oberdiek}%
  \file{hobsub-generic.sty}{%
    \from{hobsub.dtx}{generic,preamble,infwarerr}%
    \from{infwarerr.dtx}{package}%
    \from{hobsub.dtx}{post,ltxcmds}%
    \from{ltxcmds.dtx}{package}%
    \from{hobsub.dtx}{post,ifluatex}%
    \from{ifluatex.dtx}{package}%
    \from{hobsub.dtx}{post,ifvtex}%
    \from{ifvtex.dtx}{package}%
    \from{hobsub.dtx}{post,intcalc}%
    \from{intcalc.dtx}{package}%
    \from{hobsub.dtx}{post,ifpdf}%
    \from{ifpdf.dtx}{package}%
    \from{hobsub.dtx}{post,etexcmds}%
    \from{etexcmds.dtx}{package}%
    \from{hobsub.dtx}{post,kvsetkeys}%
    \from{kvsetkeys.dtx}{package}%
    \from{hobsub.dtx}{post,kvdefinekeys}%
    \from{kvdefinekeys.dtx}{package}%
    \from{hobsub.dtx}{post,luatex-loader}%
    \from{luatex.dtx}{loader}%
    \from{hobsub.dtx}{post,pdftexcmds}%
    \from{pdftexcmds.dtx}{package}%
    \from{hobsub.dtx}{post,pdfescape}%
    \from{pdfescape.dtx}{package}%
    \from{hobsub.dtx}{post,bigintcalc}%
    \from{bigintcalc.dtx}{package}%
    \from{hobsub.dtx}{post,bitset}%
    \from{bitset.dtx}{package}%
    \from{hobsub.dtx}{post,uniquecounter}%
    \from{uniquecounter.dtx}{package}%
    \from{hobsub.dtx}{post,atend}%
  }%
}
\generate{%
  \usedir{tex/generic/oberdiek}%
  \file{hobsub-hyperref.sty}{%
    \from{hobsub.dtx}{hyperref,preamble,,letltxmacro}%
    \from{letltxmacro.dtx}{package}%
    \from{hobsub.dtx}{post,hopatch}%
    \from{hopatch.dtx}{package}%
    \from{hobsub.dtx}{post,xcolor-patch}%
    \from{hycolor.dtx}{xcolor}%
    \from{hobsub.dtx}{post,atveryend}%
    \from{atveryend.dtx}{package}%
    \from{hobsub.dtx}{post,atbegshi}%
    \from{atbegshi.dtx}{package}%
    \from{hobsub.dtx}{post,refcount}%
    \from{refcount.dtx}{package}%
    \from{hobsub.dtx}{post,hycolor}%
    \from{hycolor.dtx}{package}%
    \from{hobsub.dtx}{post}%
  }%
}
\let\normalLine\hobsubSavedNormalLine

\catcode32=13\relax% active space
\let =\space%
\Msg{************************************************************************}
\Msg{*}
\Msg{* To finish the installation you have to move the following}
\Msg{* files into a directory searched by TeX:}
\Msg{*}
\Msg{*     hobsub.sty, hobsub-generic.sty, hobsub-hyperref.sty}
\Msg{*}
\Msg{* To produce the documentation run the file `hobsub.drv'}
\Msg{* through LaTeX.}
\Msg{*}
\Msg{* Happy TeXing!}
\Msg{*}
\Msg{************************************************************************}

\endbatchfile
%</install>
%<*ignore>
\fi
%</ignore>
%<*driver>
\NeedsTeXFormat{LaTeX2e}
\ProvidesFile{hobsub.drv}%
  [2016/05/16 v1.14 Construct package bundles (HO)]%
\documentclass{ltxdoc}
\usepackage{holtxdoc}[2011/11/22]
\usepackage{array}
\usepackage{caption}
\renewcommand*{\bottomfraction}{.5}
\renewcommand*{\topfraction}{.5}
\begin{document}
  \DocInput{hobsub.dtx}%
\end{document}
%</driver>
% \fi
%
%
% \CharacterTable
%  {Upper-case    \A\B\C\D\E\F\G\H\I\J\K\L\M\N\O\P\Q\R\S\T\U\V\W\X\Y\Z
%   Lower-case    \a\b\c\d\e\f\g\h\i\j\k\l\m\n\o\p\q\r\s\t\u\v\w\x\y\z
%   Digits        \0\1\2\3\4\5\6\7\8\9
%   Exclamation   \!     Double quote  \"     Hash (number) \#
%   Dollar        \$     Percent       \%     Ampersand     \&
%   Acute accent  \'     Left paren    \(     Right paren   \)
%   Asterisk      \*     Plus          \+     Comma         \,
%   Minus         \-     Point         \.     Solidus       \/
%   Colon         \:     Semicolon     \;     Less than     \<
%   Equals        \=     Greater than  \>     Question mark \?
%   Commercial at \@     Left bracket  \[     Backslash     \\
%   Right bracket \]     Circumflex    \^     Underscore    \_
%   Grave accent  \`     Left brace    \{     Vertical bar  \|
%   Right brace   \}     Tilde         \~}
%
% \GetFileInfo{hobsub.drv}
%
% \title{The \xpackage{hobsub} package}
% \date{2016/05/16 v1.14}
% \author{Heiko Oberdiek\thanks
% {Please report any issues at https://github.com/ho-tex/oberdiek/issues}\\
% \xemail{heiko.oberdiek at googlemail.com}}
%
% \maketitle
%
% \begin{abstract}
% Package \xpackage{hobsub} implements the idea of loading
% several packages in one big collection package.
% \end{abstract}
%
% \tableofcontents
%
% \def\TableGeneric{^^A
%   \TableEntry{\xpackage{infwarerr}}{}{2010/04/08 v1.3}^^A
%   \TableEntry{\xpackage{ltxcmds}}{}{2011/11/09 v1.22}^^A
%   \TableEntry{\xpackage{ifluatex}}{}{2010/03/01 v1.3}^^A
%   \TableEntry{\xpackage{ifvtex}}{}{2010/03/01 v1.5}^^A
%   \TableEntry{\xpackage{intcalc}}{}{2007/09/27 v1.1}^^A
%   \TableEntry{\xpackage{ifpdf}}{\xpackage{ifluatex}}{2011/01/30 v2.3}^^A
%   \TableEntry{\xpackage{etexcmds}}{}{2011/02/16 v1.5}^^A
%   \TableEntry{\xpackage{kvsetkeys}}{}{2012/04/25 v1.16}^^A
%   \TableEntry{\xpackage{kvdefinekeys}}{}{2011/04/07 v1.3}^^A
%   \TableEntry{\xpackage{luatex-loader}}{}{2010/03/09 v0.4}^^A
%   \TableEntry{\xpackage{pdftexcmds}}{\xpackage{ifluatex}, \xpackage{ifpdf}, \xpackage{infwarerr}, \xpackage{ltxcmds}, \xpackage{luatex-loader}}{2011/11/29 v0.20}^^A
%   \TableEntry{\xpackage{pdfescape}}{\xpackage{ltxcmds}, \xpackage{pdftexcmds}}{2011/11/25 v1.13}^^A
%   \TableEntry{\xpackage{bigintcalc}}{\xpackage{pdftexcmds}}{2012/04/08 v1.3}^^A
%   \TableEntry{\xpackage{bitset}}{\xpackage{bigintcalc}, \xpackage{infwarerr}, \xpackage{intcalc}}{2011/01/30 v1.1}^^A
%   \TableEntry{\xpackage{uniquecounter}}{\xpackage{bigintcalc}, \xpackage{infwarerr}}{2011/01/30 v1.2}^^A
% }
% \def\TableHyperref{^^A
%   \TableEntry{\xpackage{letltxmacro}}{}{2010/09/02 v1.4}^^A
%   \TableEntry{\xpackage{hopatch}}{}{2016/05/16 v1.2}^^A
%   \TableEntry{\xpackage{xcolor-patch}}{}{2011/01/30 v1.7}^^A
%   \TableEntry{\xpackage{atveryend}}{}{2011/06/30 v1.8}^^A
%   \TableEntry{\xpackage{atbegshi}}{\xpackage{ifpdf}, \xpackage{infwarerr}, \xpackage{ltxcmds}}{2011/10/05 v1.16}^^A
%   \TableEntry{\xpackage{refcount}}{\xpackage{infwarerr}, \xpackage{ltxcmds}}{2011/10/16 v3.4}^^A
%   \TableEntry{\xpackage{hycolor}}{\xpackage{xcolor-patch}}{2011/01/30 v1.7}^^A
% }
%
% \newenvironment{TableList}[2][htbp]{^^A
%   \begin{table}[{#1}]^^A
%   \centering
%   \caption{Overview \xpackage{hobsub-#2}}^^A
%   \label{tab:#2}^^A
%   \begin{tabular}{ll>{\raggedright}p{.4\linewidth}}^^A
%   \bfseries Package & \bfseries Release &
%   \bfseries Dependencies\tabularnewline
%   \hline
% }{^^A
%   \end{tabular}^^A
%   \end{table}^^A
% }
% \newcommand*{\TableEntry}[3]{%
%   #1&#3&#2\tabularnewline
% }
%
% \renewcommand*{\arraystretch}{1.1}
% \setlength{\extrarowheight}{2pt}
%
% \section{Documentation}
%
% In January 2011 the mailing list lualatex-dev discussed
% ``quietening lualatex console output'' \cite{lualatex:quiet}.
% Inspired from this, I implemented this package \xpackage{hobsub}
% to load several packages in one file. The package is applied for
% two packages collections that are subsets of my package bundle
% \xpackage{oberdiek}.
% \begin{description}
% \item[\xpackage{hobsub-generic}] is a collection of packages
%   that can also be used with \hologo{plainTeX}.
% \item[\xpackage{hobsub-hyperref}] is a collection of packages
%   that is used by package \xpackage{hyperref}.
% \end{description}
%
% Package \xpackage{hobsub} provides the macros that
% are used in the collection packages and might be used
% in other projects as well. But there are many caveats,
% some of them are discussed below.
%
% The most serious problem is consistency. A package
% that is available as standalone package and is part
% of a collection might have different versions.
% Because my packages are organized in a bundle and should
% be installed and updated as bundle, the risk seems low
% to me. However, if a single package is updated manually, for
% example for testing, then it should be loaded before the
% collection that contains the package with an older version.
% In a collection, the package will not be loaded again if it
% is already loaded.
%
% \subsection{Usage}
%
% The collection packages \xpackage{hobsub-generic} and
% \xpackage{hobsub-hyperref} are loaded as usual:
% \begin{quote}
%  \begin{tabular}{@{}ll@{}}
%    |\usepackage{hobsub-generic}| & \hologo{LaTeX}\\
%    |\input hobsub-generic.sty| & \hologo{plainTeX}
%  \end{tabular}
% \end{quote}
% or
% \begin{quote}|\usepackage{hobsub-hyperref}|\end{quote}
% Of course, both \cs{usepackage} or \cs{RequirePackage}
% can be used.
%
% If you need requirements on the version date of a package
% inside a collection, then specify them afterwards, for
% example:
% \begin{quote}
%   |\usepackage{hobsub-generic}|\\
%   |\usepackage{ltxcmds}[2010/12/04]% |^^A
%   {\itshape |\ltx@ifblank is needed|}
% \end{quote}
%
% Also it is not garanteed that a package will be always
% part of a collection. Therefore it does not harm,
% to make the requirements explicite \emph{after}
% a collection is loaded, see the previous example.
% \hologo{LaTeX} knows that a
% package is already loaded and does not load it again.
%
% \subsection{Supported features}
%
% \begin{description}
% \item[No reloads:] If a package is already loaded then it
%  is not reloaded again if it is part of a collection package.
% \item[\cs{listfiles}:]
%   Each package that is loaded in a collection is also
%   added to the output of \cs{listfiles}.
% \item[\cs{@currname}, \cs{@currext}]
%   are updated, they are used by \hologo{LaTeX}'s
%  \cs{ProvidesPackage}.
% \item[\cs{AtEndOfPackage}:]
%   \hologo{LaTeX}'s hook mechanism that is used for
%   \cs{AtEndOfPackage} is supported. However none of the
%   packages in the collections are using \cs{AtEndOfPackage},
%   therefore this feature is completely untested.
% \end{description}
%
% \subsection{Common limitations}
%
% \begin{description}
% \item[\cs{endinput}:]
%   Often a package is using \cs{endinput}, either added by
%   \xpackage{docstrip}, or to stop package loading at an earlier
%   time. Therefore \cs{endinput} must be catched during package
%   loading to prevent that the collection package is closed
%   too early before reading all packages it contains.
%
%   This redefinition of \cs{endinput} fails if the package loads
%   other files that contain \cs{endinput}. To some degree it could
%   be catched by redefining macros that load files. But this will
%   not work for the primitive \cs{input}.
% \item[Options:]
%   \hologo{LaTeX}'s package options are not supported.
%   Therefore the packages that are put in a collection must
%   not support options.
% \item[Hook packages:]
%   Some packages and classes provides hooks for packages, see
%   packages \xpackage{scrlfile}, \xpackage{filehook}
%   or class \xclass{memoir}. These hooks are not supported
%   for embedded packages inside a collection package.
%   Because of the problem with \cs{endinput}, these hooks
%   are not supported. The hooks might contain code that
%   loads other files that execute \cs{endinput}.
% \end{description}
%
% \subsection{Package \xpackage{hobsub-generic}}
%
% \begin{TableList}[bp]{generic}\TableGeneric\end{TableList}
%
% Collection `generic' contains some of the \hologo{plainTeX} compatible
% packages of my bundle, see table \ref{tab:generic}.
% It adds the following restrictions to any package
% that is part of this collection:
% \begin{description}
% \item[Generic formats:] The following formats must be supported:
%   \hologo{LaTeX}, \hologo{plainTeX} and even \hologo{iniTeX}.
%   This applies to the collection package \xpackage{hobsub-generic}
%   the same way.
% \item[Prefixed macros:]
%   Any package of this collection should only define macros with
%   prefixed names to avoid name clashes. Resources like registers
%   must not be allocated and other global states should not be
%   changed except for added macro definitions.
% \end{description}
% Thus the package \xpackage{hobsub-generic} can also preloaded
% in the format file with \hologo{iniTeX}.
%
% \subsection{Package \xpackage{hobsub-hyperref}}
%
% \begin{TableList}[tp]{hyperref}\TableHyperref\end{TableList}
%
% Collection `hyperref' contains additional packages that are
% used by package \xpackage{hyperref} and that does not fit
% in collection `generic', see table{tab:hyperref}.
% The latter collection package
% \xpackage{hobsub-generic} is loaded via a conventional
% \cs{RequirePackage}.
%
% Because most of the packages in the collection `hyperref'
% needs \hologo{LaTeX}, the collection requires \hologo{LaTeX}.
%
% \subsection{Package \xpackage{hobsub}}
%
% This section is for advanced users that might want to
% build a collection package. Then it can be done by
% using \xpackage{hobsub}:
% \begin{quote}
%   \dots\\
%   |\RequirePackage{hobsub}[2016/05/16]|\\
%   \dots\\
%   |\hobsub@StartPackage{foo1}|\\
%   \textit{\dots\unskip\ package code of \xfile{foo1.sty} \dots}\\
%   |\endinput|\\
%   |<hobsub>|\\
%   |\hobsub@StopPackage|\\
%   |\hobsub@StartPackage{bar2}|\\
%   \textit{\dots\unskip\ package code of \xfile{bar2.sty} \dots}\\
%   |\endinput|\\
%   |<hobsub>|\\
%   |\hobsub@StopPackage|
%   \dots
% \end{quote}
% But be aware of the caveats and limitations, see above.
% Only quite a small portion of the packages can be embedded
% like this.
%
% \clearpage
%
% \StopEventually{
% }
%
% \section{Implementation}
%
% \subsection{Reload check and package identification}
%
% \subsubsection{Package \xpackage{hobsub}}
%
%    \begin{macrocode}
%<*package>
%    \end{macrocode}
%    Reload check, especially if the package is not used with \LaTeX.
%    \begin{macrocode}
\begingroup\catcode61\catcode48\catcode32=10\relax%
  \catcode13=5 % ^^M
  \endlinechar=13 %
  \catcode35=6 % #
  \catcode39=12 % '
  \catcode44=12 % ,
  \catcode45=12 % -
  \catcode46=12 % .
  \catcode58=12 % :
  \catcode64=11 % @
  \catcode123=1 % {
  \catcode125=2 % }
  \expandafter\let\expandafter\x\csname ver@hobsub.sty\endcsname
  \ifx\x\relax % plain-TeX, first loading
  \else
    \def\empty{}%
    \ifx\x\empty % LaTeX, first loading,
      % variable is initialized, but \ProvidesPackage not yet seen
    \else
      \expandafter\ifx\csname PackageInfo\endcsname\relax
        \def\x#1#2{%
          \immediate\write-1{Package #1 Info: #2.}%
        }%
      \else
        \def\x#1#2{\PackageInfo{#1}{#2, stopped}}%
      \fi
      \x{hobsub}{The package is already loaded}%
      \aftergroup\endinput
    \fi
  \fi
\endgroup%
%    \end{macrocode}
%    Package identification:
%    \begin{macrocode}
\begingroup\catcode61\catcode48\catcode32=10\relax%
  \catcode13=5 % ^^M
  \endlinechar=13 %
  \catcode35=6 % #
  \catcode39=12 % '
  \catcode40=12 % (
  \catcode41=12 % )
  \catcode44=12 % ,
  \catcode45=12 % -
  \catcode46=12 % .
  \catcode47=12 % /
  \catcode58=12 % :
  \catcode64=11 % @
  \catcode91=12 % [
  \catcode93=12 % ]
  \catcode123=1 % {
  \catcode125=2 % }
  \expandafter\ifx\csname ProvidesPackage\endcsname\relax
    \def\x#1#2#3[#4]{\endgroup
      \immediate\write-1{Package: #3 #4}%
      \xdef#1{#4}%
    }%
  \else
    \def\x#1#2[#3]{\endgroup
      #2[{#3}]%
      \ifx#1\@undefined
        \xdef#1{#3}%
      \fi
      \ifx#1\relax
        \xdef#1{#3}%
      \fi
    }%
  \fi
\expandafter\x\csname ver@hobsub.sty\endcsname
\ProvidesPackage{hobsub}%
  [2016/05/16 v1.14 Construct package bundles (HO)]%
%    \end{macrocode}
%    \begin{macrocode}
%</package>
%    \end{macrocode}
%
% \subsubsection{Package \xpackage{hobsub-generic}}
%
%    \begin{macrocode}
%<*generic>
%    \end{macrocode}
%    Reload check, especially if the package is not used with \LaTeX.
%    \begin{macrocode}
\begingroup\catcode61\catcode48\catcode32=10\relax%
  \catcode13=5 % ^^M
  \endlinechar=13 %
  \catcode35=6 % #
  \catcode39=12 % '
  \catcode44=12 % ,
  \catcode45=12 % -
  \catcode46=12 % .
  \catcode58=12 % :
  \catcode64=11 % @
  \catcode123=1 % {
  \catcode125=2 % }
  \expandafter\let\expandafter\x\csname ver@hobsub-generic.sty\endcsname
  \ifx\x\relax % plain-TeX, first loading
  \else
    \def\empty{}%
    \ifx\x\empty % LaTeX, first loading,
      % variable is initialized, but \ProvidesPackage not yet seen
    \else
      \expandafter\ifx\csname PackageInfo\endcsname\relax
        \def\x#1#2{%
          \immediate\write-1{Package #1 Info: #2.}%
        }%
      \else
        \def\x#1#2{\PackageInfo{#1}{#2, stopped}}%
      \fi
      \x{hobsub-generic}{The package is already loaded}%
      \aftergroup\endinput
    \fi
  \fi
\endgroup%
%    \end{macrocode}
%    Package identification:
%    \begin{macrocode}
\begingroup\catcode61\catcode48\catcode32=10\relax%
  \catcode13=5 % ^^M
  \endlinechar=13 %
  \catcode35=6 % #
  \catcode39=12 % '
  \catcode40=12 % (
  \catcode41=12 % )
  \catcode44=12 % ,
  \catcode45=12 % -
  \catcode46=12 % .
  \catcode47=12 % /
  \catcode58=12 % :
  \catcode64=11 % @
  \catcode91=12 % [
  \catcode93=12 % ]
  \catcode123=1 % {
  \catcode125=2 % }
  \expandafter\ifx\csname ProvidesPackage\endcsname\relax
    \def\x#1#2#3[#4]{\endgroup
      \immediate\write-1{Package: #3 #4}%
      \xdef#1{#4}%
    }%
  \else
    \def\x#1#2[#3]{\endgroup
      #2[{#3}]%
      \ifx#1\@undefined
        \xdef#1{#3}%
      \fi
      \ifx#1\relax
        \xdef#1{#3}%
      \fi
    }%
  \fi
\expandafter\x\csname ver@hobsub-generic.sty\endcsname
\ProvidesPackage{hobsub-generic}%
  [2016/05/16 v1.14 Bundle oberdiek, subset generic (HO)]%
%    \end{macrocode}
%    \begin{macrocode}
%</generic>
%    \end{macrocode}
%
% \subsubsection{Package \xpackage{hobsub-hyperref}}
%
%    \begin{macrocode}
%<*hyperref>
%    \end{macrocode}
%    Reload check, especially if the package is not used with \LaTeX.
%    \begin{macrocode}
\begingroup\catcode61\catcode48\catcode32=10\relax%
  \catcode13=5 % ^^M
  \endlinechar=13 %
  \catcode35=6 % #
  \catcode39=12 % '
  \catcode44=12 % ,
  \catcode45=12 % -
  \catcode46=12 % .
  \catcode58=12 % :
  \catcode64=11 % @
  \catcode123=1 % {
  \catcode125=2 % }
  \expandafter\let\expandafter\x\csname ver@hobsub-hyperref.sty\endcsname
  \ifx\x\relax % plain-TeX, first loading
  \else
    \def\empty{}%
    \ifx\x\empty % LaTeX, first loading,
      % variable is initialized, but \ProvidesPackage not yet seen
    \else
      \expandafter\ifx\csname PackageInfo\endcsname\relax
        \def\x#1#2{%
          \immediate\write-1{Package #1 Info: #2.}%
        }%
      \else
        \def\x#1#2{\PackageInfo{#1}{#2, stopped}}%
      \fi
      \x{hobsub-hyperref}{The package is already loaded}%
      \aftergroup\endinput
    \fi
  \fi
\endgroup%
%    \end{macrocode}
%    Package identification:
%    \begin{macrocode}
\begingroup\catcode61\catcode48\catcode32=10\relax%
  \catcode13=5 % ^^M
  \endlinechar=13 %
  \catcode35=6 % #
  \catcode39=12 % '
  \catcode40=12 % (
  \catcode41=12 % )
  \catcode44=12 % ,
  \catcode45=12 % -
  \catcode46=12 % .
  \catcode47=12 % /
  \catcode58=12 % :
  \catcode64=11 % @
  \catcode91=12 % [
  \catcode93=12 % ]
  \catcode123=1 % {
  \catcode125=2 % }
  \expandafter\ifx\csname ProvidesPackage\endcsname\relax
    \def\x#1#2#3[#4]{\endgroup
      \immediate\write-1{Package: #3 #4}%
      \xdef#1{#4}%
    }%
  \else
    \def\x#1#2[#3]{\endgroup
      #2[{#3}]%
      \ifx#1\@undefined
        \xdef#1{#3}%
      \fi
      \ifx#1\relax
        \xdef#1{#3}%
      \fi
    }%
  \fi
\expandafter\x\csname ver@hobsub-hyperref.sty\endcsname
\ProvidesPackage{hobsub-hyperref}%
  [2016/05/16 v1.14 Bundle oberdiek, subset hyperref (HO)]%
%    \end{macrocode}
%    \begin{macrocode}
%</hyperref>
%    \end{macrocode}
%
% \subsection{Catcodes}
%
%    \begin{macrocode}
%<*package|generic|hyperref>
\begingroup\catcode61\catcode48\catcode32=10\relax%
  \catcode13=5 % ^^M
  \endlinechar=13 %
  \catcode123=1 % {
  \catcode125=2 % }
  \catcode64=11 % @
  \def\x{\endgroup
    \expandafter\edef\csname HOBsub@AtEnd\endcsname{%
      \endlinechar=\the\endlinechar\relax
      \catcode13=\the\catcode13\relax
      \catcode32=\the\catcode32\relax
      \catcode35=\the\catcode35\relax
      \catcode61=\the\catcode61\relax
      \catcode64=\the\catcode64\relax
      \catcode123=\the\catcode123\relax
      \catcode125=\the\catcode125\relax
    }%
  }%
\x\catcode61\catcode48\catcode32=10\relax%
\catcode13=5 % ^^M
\endlinechar=13 %
\catcode35=6 % #
\catcode64=11 % @
\catcode123=1 % {
\catcode125=2 % }
\def\TMP@EnsureCode#1#2{%
  \edef\HOBsub@AtEnd{%
    \HOBsub@AtEnd
    \catcode#1=\the\catcode#1\relax
  }%
  \catcode#1=#2\relax
}
\TMP@EnsureCode{39}{12}% '
\TMP@EnsureCode{40}{12}% (
\TMP@EnsureCode{41}{12}% )
\TMP@EnsureCode{45}{12}% -
\TMP@EnsureCode{46}{12}% .
\TMP@EnsureCode{47}{12}% /
\TMP@EnsureCode{58}{12}% :
\TMP@EnsureCode{60}{12}% <
\TMP@EnsureCode{62}{12}% >
\TMP@EnsureCode{96}{12}% `
\edef\HOBsub@AtEnd{\HOBsub@AtEnd\noexpand\endinput}
%    \end{macrocode}
%
%    \begin{macrocode}
%</package|generic|hyperref>
%    \end{macrocode}
%
% \subsection{Package \xpackage{hobsub-hyperref} loads \xpackage{hobsub-generic}}
%
%    \begin{macrocode}
%<*hyperref>
\NeedsTeXFormat{LaTeX2e}
\RequirePackage{hobsub-generic}[2016/05/16]
%</hyperref>
%    \end{macrocode}
%
% \subsection{Preamble}
%
%    \begin{macrocode}
%<*preamble>
\begingroup\expandafter\expandafter\expandafter\endgroup
\expandafter\ifx\csname ver@hobsub.sty\endcsname\relax
\else
  \expandafter\hobsub@GobbleRemainingPackage
\fi
%</preamble>
%    \end{macrocode}
%
%
%    \begin{macrocode}
%<*package|preamble>
%    \end{macrocode}
%
%    \begin{macro}{\HOBsub@OrgEndinput}
%    \begin{macrocode}
\begingroup\expandafter\expandafter\expandafter\endgroup
\expandafter\ifx\csname HOBsub@OrgEndinput\endcsname\relax
  \let\HOBsub@OrgEndinput\endinput
\fi
%    \end{macrocode}
%    \end{macro}
%    \begin{macro}{\hobsub@GobbleRemainingPackage}
%    \begin{macrocode}
\def\hobsub@GobbleRemainingPackage{%
  \begingroup
  \catcode92=14 % backslash: comment
  \catcode32=14 % space: comment
  \catcode35=14 % hash: comment
  \catcode123=14 % left brace: comment
  \catcode125=14 % right brace: comment
  \catcode60=3 % less: math
  \catcode62=4 % greater: align
  \endlinechar=-1 %
  \HOBsub@GobbleRemainingPackage
}
%    \end{macrocode}
%    \end{macro}
%    \begin{macro}{\HOBsub@GobbleRemainingPackage}
%    \begin{macrocode}
\catcode60=3 % less: dollar
\catcode62=4 % greater: align
\long\def\HOBsub@GobbleRemainingPackage#1<hobsub>{%
  \endgroup
}
\catcode60=12 % less: other
\catcode62=12 % greater: other
%    \end{macrocode}
%    \end{macro}
%
%    \begin{macro}{\hobsub@StartPackage}
%    \begin{macrocode}
\def\hobsub@StartPackage#1{%
  \begingroup\expandafter\expandafter\expandafter\endgroup
  \expandafter\ifx\csname ver@#1.sty\endcsname\relax
    \let\HOBsub@OrgCurrName\@currname
    \let\HOBsub@OrgCurrExt\@currext
    \csname @pushfilename\endcsname
    \def\@currname{#1}%
    \def\@currext{sty}%
    \expandafter\def\csname\@currname.\@currext-h@@k\endcsname{}%
    \let\endinput\hobsub@GobbleRemainingPackage
    \def\hobsub@StopPackage{%
      \let\hobsub@StopPackage\relax
      \HOBsub@StopPackage
%     \hobsub@Info{hobsub}{Package `#1' loaded}%
    }%
    \hobsub@AddToFileList{#1.sty}%
  \else
    \hobsub@Info{hobsub}{Skipping package `#1' (already loaded)}%
    \let\hobsub@StopPackage\relax
    \expandafter\hobsub@GobbleRemainingPackage
  \fi
}
%    \end{macrocode}
%    \end{macro}
%
%    \begin{macro}{\hobsub@StopPackage}
%    \begin{macrocode}
\let\hobsub@StopPackage\relax
%    \end{macrocode}
%    \end{macro}
%
%    \begin{macro}{\hobsub@Info}
%    \begin{macrocode}
\def\hobsub@Info#1#2{%
  \begingroup\expandafter\expandafter\expandafter\endgroup
  \expandafter\ifx\csname @PackageInfoNoLine\endcsname\relax
    \immediate\write-1{Package #1 Info: #2.}%
  \else
    \let\hobsub@Info\@PackageInfoNoLine
    \hobsub@Info{#1}{#2}%
  \fi
}
%    \end{macrocode}
%    \end{macro}
%
%    \begin{macro}{\HOBsub@StopPackage}
%    \begin{macrocode}
\def\HOBsub@StopPackage{%
  \csname\@currname.\@currext-h@@k\endcsname
  \let\endinput\HOBsub@OrgEndinput
  \csname @popfilename\endcsname
  \let\@currname\HOBsub@OrgCurrName
  \let\@currext\HOBsub@OrgCurrExt
}
%    \end{macrocode}
%    \end{macro}
%
%    \begin{macro}{\hobsub@AddToFileList}
%    \begin{macrocode}
\begingroup\expandafter\expandafter\expandafter\endgroup
\expandafter\ifx\csname @addtofilelist\endcsname\relax
  \def\hobsub@AddToFileList#1{}%
\else
  \def\hobsub@AddToFileList#1{%
    \@addtofilelist{#1}%
  }%
\fi
%    \end{macrocode}
%    \end{macro}
%
%    \begin{macrocode}
%</package|preamble>
%    \end{macrocode}
%
% \subsection{End of preamble for subset packages}
%
%    Add the end marker for possible skipping of previous
%    macros, if package \xpackage{hobsub} was already loaded.
%    \begin{macrocode}
%<*preamble>
\hobsub@GobbleRemainingPackage
<hobsub>
%    \end{macrocode}
%
%    All macros of package \xpackage{hobsub} are now available.
%    Now we officially load the package \xpackage{hobsub} inline
%    for proper package identification.
%    \begin{macrocode}
\hobsub@StartPackage{hobsub}%
\begingroup\expandafter\expandafter\expandafter\endgroup
\expandafter\ifx\csname ProvidesPackage\endcsname\relax
  \immediate\write-1{%
    Package: hobsub 2016/05/16 v1.14 Construct package bundles (HO)%
  }%
\else
  \ProvidesPackage{hobsub}%
  [2016/05/16 v1.14 Construct package bundles (HO)]%
\fi
%</preamble>
%    \end{macrocode}
%
%
%    \begin{macrocode}
%<*post|preamble>
 \endinput
<hobsub>
\hobsub@StopPackage
%</post|preamble>
%    \end{macrocode}
%
% \subsection{Package list}
%
%    \begin{macrocode}
%<infwarerr>\hobsub@StartPackage{infwarerr}
%<ltxcmds>\hobsub@StartPackage{ltxcmds}
%<ifluatex>\hobsub@StartPackage{ifluatex}
%<ifvtex>\hobsub@StartPackage{ifvtex}
%<intcalc>\hobsub@StartPackage{intcalc}
%<ifpdf>\hobsub@StartPackage{ifpdf}
%<etexcmds>\hobsub@StartPackage{etexcmds}
%<kvsetkeys>\hobsub@StartPackage{kvsetkeys}
%<kvdefinekeys>\hobsub@StartPackage{kvdefinekeys}
%<*luatex-loader>
\ifluatex
\else
  \expandafter\hobsub@GobbleRemainingPackage
\fi
\hobsub@StartPackage{luatex-loader}
%</luatex-loader>
%<pdftexcmds>\hobsub@StartPackage{pdftexcmds}
%<pdfescape>\hobsub@StartPackage{pdfescape}
%<bigintcalc>\hobsub@StartPackage{bigintcalc}
%<bitset>\hobsub@StartPackage{bitset}
%<uniquecounter>\hobsub@StartPackage{uniquecounter}
%<letltxmacro>\hobsub@StartPackage{letltxmacro}
%<hopatch>\hobsub@StartPackage{hopatch}
%<xcolor-patch>\hobsub@StartPackage{xcolor-patch}
%<atveryend>\hobsub@StartPackage{atveryend}
%<atbegshi>\hobsub@StartPackage{atbegshi}
%<refcount>\hobsub@StartPackage{refcount}
%<hycolor>\hobsub@StartPackage{hycolor}
%    \end{macrocode}
%
% \subsection{End of packages}
%
%    \begin{macrocode}
%<package|atend>\HOBsub@AtEnd%
%    \end{macrocode}
%
% \section{Test}
%
% \subsection{Catcode checks for loading}
%
%    \begin{macrocode}
%<*test1>
%    \end{macrocode}
%    \begin{macrocode}
\catcode`\{=1 %
\catcode`\}=2 %
\catcode`\#=6 %
\catcode`\@=11 %
\expandafter\ifx\csname count@\endcsname\relax
  \countdef\count@=255 %
\fi
\expandafter\ifx\csname @gobble\endcsname\relax
  \long\def\@gobble#1{}%
\fi
\expandafter\ifx\csname @firstofone\endcsname\relax
  \long\def\@firstofone#1{#1}%
\fi
\expandafter\ifx\csname loop\endcsname\relax
  \expandafter\@firstofone
\else
  \expandafter\@gobble
\fi
{%
  \def\loop#1\repeat{%
    \def\body{#1}%
    \iterate
  }%
  \def\iterate{%
    \body
      \let\next\iterate
    \else
      \let\next\relax
    \fi
    \next
  }%
  \let\repeat=\fi
}%
\def\RestoreCatcodes{}
\count@=0 %
\loop
  \edef\RestoreCatcodes{%
    \RestoreCatcodes
    \catcode\the\count@=\the\catcode\count@\relax
  }%
\ifnum\count@<255 %
  \advance\count@ 1 %
\repeat

\def\RangeCatcodeInvalid#1#2{%
  \count@=#1\relax
  \loop
    \catcode\count@=15 %
  \ifnum\count@<#2\relax
    \advance\count@ 1 %
  \repeat
}
\def\RangeCatcodeCheck#1#2#3{%
  \count@=#1\relax
  \loop
    \ifnum#3=\catcode\count@
    \else
      \errmessage{%
        Character \the\count@\space
        with wrong catcode \the\catcode\count@\space
        instead of \number#3%
      }%
    \fi
  \ifnum\count@<#2\relax
    \advance\count@ 1 %
  \repeat
}
\def\space{ }
\expandafter\ifx\csname LoadCommand\endcsname\relax
  \def\LoadCommand{\input hobsub.sty\relax}%
\fi
\def\Test{%
  \RangeCatcodeInvalid{0}{47}%
  \RangeCatcodeInvalid{58}{64}%
  \RangeCatcodeInvalid{91}{96}%
  \RangeCatcodeInvalid{123}{255}%
  \catcode`\@=12 %
  \catcode`\\=0 %
  \catcode`\%=14 %
  \LoadCommand
  \RangeCatcodeCheck{0}{36}{15}%
  \RangeCatcodeCheck{37}{37}{14}%
  \RangeCatcodeCheck{38}{47}{15}%
  \RangeCatcodeCheck{48}{57}{12}%
  \RangeCatcodeCheck{58}{63}{15}%
  \RangeCatcodeCheck{64}{64}{12}%
  \RangeCatcodeCheck{65}{90}{11}%
  \RangeCatcodeCheck{91}{91}{15}%
  \RangeCatcodeCheck{92}{92}{0}%
  \RangeCatcodeCheck{93}{96}{15}%
  \RangeCatcodeCheck{97}{122}{11}%
  \RangeCatcodeCheck{123}{255}{15}%
  \RestoreCatcodes
}
\Test
\csname @@end\endcsname
\end
%    \end{macrocode}
%    \begin{macrocode}
%</test1>
%    \end{macrocode}
%
% \section{Installation}
%
% \subsection{Download}
%
% \paragraph{Package.} This package is available on
% CTAN\footnote{\url{http://ctan.org/pkg/hobsub}}:
% \begin{description}
% \item[\CTAN{macros/latex/contrib/oberdiek/hobsub.dtx}] The source file.
% \item[\CTAN{macros/latex/contrib/oberdiek/hobsub.pdf}] Documentation.
% \end{description}
%
%
% \paragraph{Bundle.} All the packages of the bundle `oberdiek'
% are also available in a TDS compliant ZIP archive. There
% the packages are already unpacked and the documentation files
% are generated. The files and directories obey the TDS standard.
% \begin{description}
% \item[\CTAN{install/macros/latex/contrib/oberdiek.tds.zip}]
% \end{description}
% \emph{TDS} refers to the standard ``A Directory Structure
% for \TeX\ Files'' (\CTAN{tds/tds.pdf}). Directories
% with \xfile{texmf} in their name are usually organized this way.
%
% \subsection{Bundle installation}
%
% \paragraph{Unpacking.} Unpack the \xfile{oberdiek.tds.zip} in the
% TDS tree (also known as \xfile{texmf} tree) of your choice.
% Example (linux):
% \begin{quote}
%   |unzip oberdiek.tds.zip -d ~/texmf|
% \end{quote}
%
% \paragraph{Script installation.}
% Check the directory \xfile{TDS:scripts/oberdiek/} for
% scripts that need further installation steps.
% Package \xpackage{attachfile2} comes with the Perl script
% \xfile{pdfatfi.pl} that should be installed in such a way
% that it can be called as \texttt{pdfatfi}.
% Example (linux):
% \begin{quote}
%   |chmod +x scripts/oberdiek/pdfatfi.pl|\\
%   |cp scripts/oberdiek/pdfatfi.pl /usr/local/bin/|
% \end{quote}
%
% \subsection{Package installation}
%
% \paragraph{Unpacking.} The \xfile{.dtx} file is a self-extracting
% \docstrip\ archive. The files are extracted by running the
% \xfile{.dtx} through \plainTeX:
% \begin{quote}
%   \verb|tex hobsub.dtx|
% \end{quote}
%
% \paragraph{TDS.} Now the different files must be moved into
% the different directories in your installation TDS tree
% (also known as \xfile{texmf} tree):
% \begin{quote}
% \def\t{^^A
% \begin{tabular}{@{}>{\ttfamily}l@{ $\rightarrow$ }>{\ttfamily}l@{}}
%   hobsub.sty & tex/generic/oberdiek/hobsub.sty\\
%   hobsub-generic.sty & tex/generic/oberdiek/hobsub-generic.sty\\
%   hobsub-hyperref.sty & tex/generic/oberdiek/hobsub-hyperref.sty\\
%   hobsub.pdf & doc/latex/oberdiek/hobsub.pdf\\
%   test/hobsub-test1.tex & doc/latex/oberdiek/test/hobsub-test1.tex\\
%   hobsub.dtx & source/latex/oberdiek/hobsub.dtx\\
% \end{tabular}^^A
% }^^A
% \sbox0{\t}^^A
% \ifdim\wd0>\linewidth
%   \begingroup
%     \advance\linewidth by\leftmargin
%     \advance\linewidth by\rightmargin
%   \edef\x{\endgroup
%     \def\noexpand\lw{\the\linewidth}^^A
%   }\x
%   \def\lwbox{^^A
%     \leavevmode
%     \hbox to \linewidth{^^A
%       \kern-\leftmargin\relax
%       \hss
%       \usebox0
%       \hss
%       \kern-\rightmargin\relax
%     }^^A
%   }^^A
%   \ifdim\wd0>\lw
%     \sbox0{\small\t}^^A
%     \ifdim\wd0>\linewidth
%       \ifdim\wd0>\lw
%         \sbox0{\footnotesize\t}^^A
%         \ifdim\wd0>\linewidth
%           \ifdim\wd0>\lw
%             \sbox0{\scriptsize\t}^^A
%             \ifdim\wd0>\linewidth
%               \ifdim\wd0>\lw
%                 \sbox0{\tiny\t}^^A
%                 \ifdim\wd0>\linewidth
%                   \lwbox
%                 \else
%                   \usebox0
%                 \fi
%               \else
%                 \lwbox
%               \fi
%             \else
%               \usebox0
%             \fi
%           \else
%             \lwbox
%           \fi
%         \else
%           \usebox0
%         \fi
%       \else
%         \lwbox
%       \fi
%     \else
%       \usebox0
%     \fi
%   \else
%     \lwbox
%   \fi
% \else
%   \usebox0
% \fi
% \end{quote}
% If you have a \xfile{docstrip.cfg} that configures and enables \docstrip's
% TDS installing feature, then some files can already be in the right
% place, see the documentation of \docstrip.
%
% \subsection{Refresh file name databases}
%
% If your \TeX~distribution
% (\teTeX, \mikTeX, \dots) relies on file name databases, you must refresh
% these. For example, \teTeX\ users run \verb|texhash| or
% \verb|mktexlsr|.
%
% \subsection{Some details for the interested}
%
% \paragraph{Attached source.}
%
% The PDF documentation on CTAN also includes the
% \xfile{.dtx} source file. It can be extracted by
% AcrobatReader 6 or higher. Another option is \textsf{pdftk},
% e.g. unpack the file into the current directory:
% \begin{quote}
%   \verb|pdftk hobsub.pdf unpack_files output .|
% \end{quote}
%
% \paragraph{Unpacking with \LaTeX.}
% The \xfile{.dtx} chooses its action depending on the format:
% \begin{description}
% \item[\plainTeX:] Run \docstrip\ and extract the files.
% \item[\LaTeX:] Generate the documentation.
% \end{description}
% If you insist on using \LaTeX\ for \docstrip\ (really,
% \docstrip\ does not need \LaTeX), then inform the autodetect routine
% about your intention:
% \begin{quote}
%   \verb|latex \let\install=y\input{hobsub.dtx}|
% \end{quote}
% Do not forget to quote the argument according to the demands
% of your shell.
%
% \paragraph{Generating the documentation.}
% You can use both the \xfile{.dtx} or the \xfile{.drv} to generate
% the documentation. The process can be configured by the
% configuration file \xfile{ltxdoc.cfg}. For instance, put this
% line into this file, if you want to have A4 as paper format:
% \begin{quote}
%   \verb|\PassOptionsToClass{a4paper}{article}|
% \end{quote}
% An example follows how to generate the
% documentation with pdf\LaTeX:
% \begin{quote}
%\begin{verbatim}
%pdflatex hobsub.dtx
%makeindex -s gind.ist hobsub.idx
%pdflatex hobsub.dtx
%makeindex -s gind.ist hobsub.idx
%pdflatex hobsub.dtx
%\end{verbatim}
% \end{quote}
%
% \section{Catalogue}
%
% The following XML file can be used as source for the
% \href{http://mirror.ctan.org/help/Catalogue/catalogue.html}{\TeX\ Catalogue}.
% The elements \texttt{caption} and \texttt{description} are imported
% from the original XML file from the Catalogue.
% The name of the XML file in the Catalogue is \xfile{hobsub.xml}.
%    \begin{macrocode}
%<*catalogue>
<?xml version='1.0' encoding='us-ascii'?>
<!DOCTYPE entry SYSTEM 'catalogue.dtd'>
<entry datestamp='$Date$' modifier='$Author$' id='hobsub'>
  <name>hobsub</name>
  <caption>Construct package bundles</caption>
  <authorref id='auth:oberdiek'/>
  <copyright owner='Heiko Oberdiek' year='2011,2012'/>
  <license type='lppl1.3'/>
  <version number='1.14'/>
  <description>
    The package offers a means of loading a bundle of (other) packages
    &#x201C;at once&#x201D;, offering a means of reducing the amount of log output
    as the packages are loaded, and an aide-memoire for complicated
    requirements.  In some ways, loading a package via hobsub behaves
    much like loading the package in the normal way; for example,
    reloads are suppressed, as in LaTeX proper, and correct
    <tt>\listfiles</tt> output is produced.
    <p/>
    Examples provided are a &#x2018;generic&#x2019; bundle (comprising the author&#x2019;s
    packages that can be used with Plain TeX) and a &#x2018;hyperref&#x2019; bundle
    (comprising packages useful when generating hypertext output with
    LaTeX).
    <p/>
    The package is part of the <xref refid='oberdiek'>oberdiek</xref> bundle.
  </description>
  <documentation details='Package documentation'
      href='ctan:/macros/latex/contrib/oberdiek/hobsub.pdf'/>
  <ctan file='true' path='/macros/latex/contrib/oberdiek/hobsub.dtx'/>
  <miktex location='oberdiek'/>
  <texlive location='oberdiek'/>
  <install path='/macros/latex/contrib/oberdiek/oberdiek.tds.zip'/>
</entry>
%</catalogue>
%    \end{macrocode}
%
% \begin{thebibliography}{9}
% \raggedright
% \bibitem{lualatex:quiet}
% Will Robertson,
% \textit{[lltx] quietening lualatex console output};
% mailing list \textsf{lualatex-dev at tug.org},
% 2011-01-29;
% \url{http://tug.org/pipermail/lualatex-dev/2011-January/001031.html}.
%
% \end{thebibliography}
%
% \begin{History}
%   \begin{Version}{2011/01/30 v1.0}
%   \item
%     First public version.
%   \end{Version}
%   \begin{Version}{2011/04/16 v1.1}
%   \item
%     Package updates.
%   \end{Version}
%   \begin{Version}{2011/04/17 v1.2}
%   \item
%     White spaces at line begins are removed or reduced
%     in the generated collection packages.
%   \end{Version}
%   \begin{Version}{2011/04/18 v1.3}
%   \item
%     Package updates.
%   \end{Version}
%   \begin{Version}{2011/04/23 v1.4}
%   \item
%     Package updates.
%   \end{Version}
%   \begin{Version}{2011/06/24 v1.5}
%   \item
%     Package updates.
%   \end{Version}
%   \begin{Version}{2011/06/30 v1.6}
%   \item
%     Package updates.
%   \end{Version}
%   \begin{Version}{2011/07/01 v1.7}
%   \item
%     Package updates.
%   \end{Version}
%   \begin{Version}{2011/07/28 v1.8}
%   \item
%     Package updates.
%   \end{Version}
%   \begin{Version}{2011/08/22 v1.9}
%   \item
%     Package updates.
%   \end{Version}
%   \begin{Version}{2011/10/16 v1.10}
%   \item
%     Package updates.
%   \end{Version}
%   \begin{Version}{2011/11/29 v1.11}
%   \item
%     Package updates.
%   \end{Version}
%   \begin{Version}{2012/04/25 v1.12}
%   \item
%     Package updates.
%   \end{Version}
%   \begin{Version}{2012/05/28 v1.13}
%   \item
%     Package updates.
%   \end{Version}
%   \begin{Version}{2016/05/16 v1.14}
%   \item
%     Documentation updates.
%   \end{Version}
% \end{History}
%
% \PrintIndex
%
% \Finale
\endinput

%        (quote the arguments according to the demands of your shell)
%
% Documentation:
%    (a) If hobsub.drv is present:
%           latex hobsub.drv
%    (b) Without hobsub.drv:
%           latex hobsub.dtx; ...
%    The class ltxdoc loads the configuration file ltxdoc.cfg
%    if available. Here you can specify further options, e.g.
%    use A4 as paper format:
%       \PassOptionsToClass{a4paper}{article}
%
%    Programm calls to get the documentation (example):
%       pdflatex hobsub.dtx
%       makeindex -s gind.ist hobsub.idx
%       pdflatex hobsub.dtx
%       makeindex -s gind.ist hobsub.idx
%       pdflatex hobsub.dtx
%
% Installation:
%    TDS:tex/generic/oberdiek/hobsub.sty
%    TDS:tex/generic/oberdiek/hobsub-generic.sty
%    TDS:tex/generic/oberdiek/hobsub-hyperref.sty
%    TDS:doc/latex/oberdiek/hobsub.pdf
%    TDS:doc/latex/oberdiek/test/hobsub-test1.tex
%    TDS:source/latex/oberdiek/hobsub.dtx
%
%<*ignore>
\begingroup
  \catcode123=1 %
  \catcode125=2 %
  \def\x{LaTeX2e}%
\expandafter\endgroup
\ifcase 0\ifx\install y1\fi\expandafter
         \ifx\csname processbatchFile\endcsname\relax\else1\fi
         \ifx\fmtname\x\else 1\fi\relax
\else\csname fi\endcsname
%</ignore>
%<*install>
\input docstrip.tex
\Msg{************************************************************************}
\Msg{* Installation}
\Msg{* Package: hobsub 2016/05/16 v1.14 Construct package bundles (HO)}
\Msg{************************************************************************}

\keepsilent
\askforoverwritefalse

\let\MetaPrefix\relax
\preamble

This is a generated file.

Project: hobsub
Version: 2016/05/16 v1.14

Copyright (C) 2011, 2012 by
   Heiko Oberdiek <heiko.oberdiek at googlemail.com>

This work may be distributed and/or modified under the
conditions of the LaTeX Project Public License, either
version 1.3c of this license or (at your option) any later
version. This version of this license is in
   http://www.latex-project.org/lppl/lppl-1-3c.txt
and the latest version of this license is in
   http://www.latex-project.org/lppl.txt
and version 1.3 or later is part of all distributions of
LaTeX version 2005/12/01 or later.

This work has the LPPL maintenance status "maintained".

This Current Maintainer of this work is Heiko Oberdiek.

The Base Interpreter refers to any `TeX-Format',
because some files are installed in TDS:tex/generic//.

This work consists of the main source file hobsub.dtx
and the derived files
   hobsub.sty, hobsub.pdf, hobsub.ins, hobsub.drv, hobsub-generic.sty,
   hobsub-hyperref.sty, hobsub-test1.tex.

\endpreamble
\let\MetaPrefix\DoubleperCent

\generate{%
  \file{hobsub.ins}{\from{hobsub.dtx}{install}}%
  \file{hobsub.drv}{\from{hobsub.dtx}{driver}}%
  \usedir{tex/generic/oberdiek}%
  \file{hobsub.sty}{\from{hobsub.dtx}{package}}%
  \usedir{doc/latex/oberdiek/test}%
  \file{hobsub-test1.tex}{\from{hobsub.dtx}{test1}}%
  \nopreamble
  \nopostamble
  \usedir{source/latex/oberdiek/catalogue}%
  \file{hobsub.xml}{\from{hobsub.dtx}{catalogue}}%
}
\let\hobsubSavedNormalLine\normalLine
\def\hobsubGobble#1{}
\begingroup
  \lccode`\0=32 % space
  \lccode`\1=35 % hash
  \lccode`\2=37 % percent
  \lccode`\3=92 % backslash
  \lccode`\4=123 % {
  \lccode`\5=125 % }
\lowercase{\endgroup
  \let\hobsubTokenSpaceOther= 0%
  \let\hobsubTokenHash= 1%
  \let\hobsubTokenPercent= 2%
  \let\hobsubTokenBackslash= 3%
  \let\hobsubTokenBraceLeft= 4%
  \let\hobsubTokenBraceRight= 5%
  \def\hobsubNormalLineWithSpace{%
    \hobsubSavedNormalLine0%
  }%
}
\def\normalLine{%
  \futurelet\hobsubToken\hobsubFirst
}
\def\hobsubFirst{%
  \ifx\hobsubToken\hobsubTokenSpaceOther
    \expandafter\expandafter\expandafter\futurelet
    \expandafter\expandafter\expandafter\hobsubToken
    \expandafter\expandafter\expandafter\hobsubNext
    \expandafter\hobsubGobble
  \else
    \expandafter\hobsubSavedNormalLine
  \fi
}
\def\hobsubNext{%
  \ifx\hobsubToken\hobsubTokenSpaceOther
    \expandafter\expandafter\expandafter\futurelet
    \expandafter\expandafter\expandafter\hobsubToken
    \expandafter\expandafter\expandafter\hobsubNext 
    \expandafter\hobsubGobble
  \else
    \let\hobsubWithSpace=Y%
    \ifx\hobsubToken\hobsubTokenBackslash
      \let\hobsubWithSpace=N%
    \else
      \ifx\hobsubToken\hobsubTokenBraceRight
        \let\hobsubWithSpace=N%
      \else
        \ifx\hobsubToken\hobsubTokenBraceLeft
          \let\hobsubWithSpace=N%
        \else
          \ifx\hobsubToken\hobsubTokenPercent
            \let\hobsubWithSpace=N%
          \else
            \ifx\hobsubToken\hobsubTokenHash
              \let\hobsubWithSpace=N%
            \fi
          \fi
        \fi
      \fi
    \fi
    \ifx\hobsubWithSpace Y%
      \expandafter\expandafter\expandafter\hobsubNormalLineWithSpace
    \else
      \expandafter\expandafter\expandafter\hobsubSavedNormalLine
    \fi
  \fi
}
\generate{%
  \usedir{tex/generic/oberdiek}%
  \file{hobsub-generic.sty}{%
    \from{hobsub.dtx}{generic,preamble,infwarerr}%
    \from{infwarerr.dtx}{package}%
    \from{hobsub.dtx}{post,ltxcmds}%
    \from{ltxcmds.dtx}{package}%
    \from{hobsub.dtx}{post,ifluatex}%
    \from{ifluatex.dtx}{package}%
    \from{hobsub.dtx}{post,ifvtex}%
    \from{ifvtex.dtx}{package}%
    \from{hobsub.dtx}{post,intcalc}%
    \from{intcalc.dtx}{package}%
    \from{hobsub.dtx}{post,ifpdf}%
    \from{ifpdf.dtx}{package}%
    \from{hobsub.dtx}{post,etexcmds}%
    \from{etexcmds.dtx}{package}%
    \from{hobsub.dtx}{post,kvsetkeys}%
    \from{kvsetkeys.dtx}{package}%
    \from{hobsub.dtx}{post,kvdefinekeys}%
    \from{kvdefinekeys.dtx}{package}%
    \from{hobsub.dtx}{post,luatex-loader}%
    \from{luatex.dtx}{loader}%
    \from{hobsub.dtx}{post,pdftexcmds}%
    \from{pdftexcmds.dtx}{package}%
    \from{hobsub.dtx}{post,pdfescape}%
    \from{pdfescape.dtx}{package}%
    \from{hobsub.dtx}{post,bigintcalc}%
    \from{bigintcalc.dtx}{package}%
    \from{hobsub.dtx}{post,bitset}%
    \from{bitset.dtx}{package}%
    \from{hobsub.dtx}{post,uniquecounter}%
    \from{uniquecounter.dtx}{package}%
    \from{hobsub.dtx}{post,atend}%
  }%
}
\generate{%
  \usedir{tex/generic/oberdiek}%
  \file{hobsub-hyperref.sty}{%
    \from{hobsub.dtx}{hyperref,preamble,,letltxmacro}%
    \from{letltxmacro.dtx}{package}%
    \from{hobsub.dtx}{post,hopatch}%
    \from{hopatch.dtx}{package}%
    \from{hobsub.dtx}{post,xcolor-patch}%
    \from{hycolor.dtx}{xcolor}%
    \from{hobsub.dtx}{post,atveryend}%
    \from{atveryend.dtx}{package}%
    \from{hobsub.dtx}{post,atbegshi}%
    \from{atbegshi.dtx}{package}%
    \from{hobsub.dtx}{post,refcount}%
    \from{refcount.dtx}{package}%
    \from{hobsub.dtx}{post,hycolor}%
    \from{hycolor.dtx}{package}%
    \from{hobsub.dtx}{post}%
  }%
}
\let\normalLine\hobsubSavedNormalLine

\catcode32=13\relax% active space
\let =\space%
\Msg{************************************************************************}
\Msg{*}
\Msg{* To finish the installation you have to move the following}
\Msg{* files into a directory searched by TeX:}
\Msg{*}
\Msg{*     hobsub.sty, hobsub-generic.sty, hobsub-hyperref.sty}
\Msg{*}
\Msg{* To produce the documentation run the file `hobsub.drv'}
\Msg{* through LaTeX.}
\Msg{*}
\Msg{* Happy TeXing!}
\Msg{*}
\Msg{************************************************************************}

\endbatchfile
%</install>
%<*ignore>
\fi
%</ignore>
%<*driver>
\NeedsTeXFormat{LaTeX2e}
\ProvidesFile{hobsub.drv}%
  [2016/05/16 v1.14 Construct package bundles (HO)]%
\documentclass{ltxdoc}
\usepackage{holtxdoc}[2011/11/22]
\usepackage{array}
\usepackage{caption}
\renewcommand*{\bottomfraction}{.5}
\renewcommand*{\topfraction}{.5}
\begin{document}
  \DocInput{hobsub.dtx}%
\end{document}
%</driver>
% \fi
%
%
% \CharacterTable
%  {Upper-case    \A\B\C\D\E\F\G\H\I\J\K\L\M\N\O\P\Q\R\S\T\U\V\W\X\Y\Z
%   Lower-case    \a\b\c\d\e\f\g\h\i\j\k\l\m\n\o\p\q\r\s\t\u\v\w\x\y\z
%   Digits        \0\1\2\3\4\5\6\7\8\9
%   Exclamation   \!     Double quote  \"     Hash (number) \#
%   Dollar        \$     Percent       \%     Ampersand     \&
%   Acute accent  \'     Left paren    \(     Right paren   \)
%   Asterisk      \*     Plus          \+     Comma         \,
%   Minus         \-     Point         \.     Solidus       \/
%   Colon         \:     Semicolon     \;     Less than     \<
%   Equals        \=     Greater than  \>     Question mark \?
%   Commercial at \@     Left bracket  \[     Backslash     \\
%   Right bracket \]     Circumflex    \^     Underscore    \_
%   Grave accent  \`     Left brace    \{     Vertical bar  \|
%   Right brace   \}     Tilde         \~}
%
% \GetFileInfo{hobsub.drv}
%
% \title{The \xpackage{hobsub} package}
% \date{2016/05/16 v1.14}
% \author{Heiko Oberdiek\thanks
% {Please report any issues at https://github.com/ho-tex/oberdiek/issues}\\
% \xemail{heiko.oberdiek at googlemail.com}}
%
% \maketitle
%
% \begin{abstract}
% Package \xpackage{hobsub} implements the idea of loading
% several packages in one big collection package.
% \end{abstract}
%
% \tableofcontents
%
% \def\TableGeneric{^^A
%   \TableEntry{\xpackage{infwarerr}}{}{2010/04/08 v1.3}^^A
%   \TableEntry{\xpackage{ltxcmds}}{}{2011/11/09 v1.22}^^A
%   \TableEntry{\xpackage{ifluatex}}{}{2010/03/01 v1.3}^^A
%   \TableEntry{\xpackage{ifvtex}}{}{2010/03/01 v1.5}^^A
%   \TableEntry{\xpackage{intcalc}}{}{2007/09/27 v1.1}^^A
%   \TableEntry{\xpackage{ifpdf}}{\xpackage{ifluatex}}{2011/01/30 v2.3}^^A
%   \TableEntry{\xpackage{etexcmds}}{}{2011/02/16 v1.5}^^A
%   \TableEntry{\xpackage{kvsetkeys}}{}{2012/04/25 v1.16}^^A
%   \TableEntry{\xpackage{kvdefinekeys}}{}{2011/04/07 v1.3}^^A
%   \TableEntry{\xpackage{luatex-loader}}{}{2010/03/09 v0.4}^^A
%   \TableEntry{\xpackage{pdftexcmds}}{\xpackage{ifluatex}, \xpackage{ifpdf}, \xpackage{infwarerr}, \xpackage{ltxcmds}, \xpackage{luatex-loader}}{2011/11/29 v0.20}^^A
%   \TableEntry{\xpackage{pdfescape}}{\xpackage{ltxcmds}, \xpackage{pdftexcmds}}{2011/11/25 v1.13}^^A
%   \TableEntry{\xpackage{bigintcalc}}{\xpackage{pdftexcmds}}{2012/04/08 v1.3}^^A
%   \TableEntry{\xpackage{bitset}}{\xpackage{bigintcalc}, \xpackage{infwarerr}, \xpackage{intcalc}}{2011/01/30 v1.1}^^A
%   \TableEntry{\xpackage{uniquecounter}}{\xpackage{bigintcalc}, \xpackage{infwarerr}}{2011/01/30 v1.2}^^A
% }
% \def\TableHyperref{^^A
%   \TableEntry{\xpackage{letltxmacro}}{}{2010/09/02 v1.4}^^A
%   \TableEntry{\xpackage{hopatch}}{}{2016/05/16 v1.2}^^A
%   \TableEntry{\xpackage{xcolor-patch}}{}{2011/01/30 v1.7}^^A
%   \TableEntry{\xpackage{atveryend}}{}{2011/06/30 v1.8}^^A
%   \TableEntry{\xpackage{atbegshi}}{\xpackage{ifpdf}, \xpackage{infwarerr}, \xpackage{ltxcmds}}{2011/10/05 v1.16}^^A
%   \TableEntry{\xpackage{refcount}}{\xpackage{infwarerr}, \xpackage{ltxcmds}}{2011/10/16 v3.4}^^A
%   \TableEntry{\xpackage{hycolor}}{\xpackage{xcolor-patch}}{2011/01/30 v1.7}^^A
% }
%
% \newenvironment{TableList}[2][htbp]{^^A
%   \begin{table}[{#1}]^^A
%   \centering
%   \caption{Overview \xpackage{hobsub-#2}}^^A
%   \label{tab:#2}^^A
%   \begin{tabular}{ll>{\raggedright}p{.4\linewidth}}^^A
%   \bfseries Package & \bfseries Release &
%   \bfseries Dependencies\tabularnewline
%   \hline
% }{^^A
%   \end{tabular}^^A
%   \end{table}^^A
% }
% \newcommand*{\TableEntry}[3]{%
%   #1&#3&#2\tabularnewline
% }
%
% \renewcommand*{\arraystretch}{1.1}
% \setlength{\extrarowheight}{2pt}
%
% \section{Documentation}
%
% In January 2011 the mailing list lualatex-dev discussed
% ``quietening lualatex console output'' \cite{lualatex:quiet}.
% Inspired from this, I implemented this package \xpackage{hobsub}
% to load several packages in one file. The package is applied for
% two packages collections that are subsets of my package bundle
% \xpackage{oberdiek}.
% \begin{description}
% \item[\xpackage{hobsub-generic}] is a collection of packages
%   that can also be used with \hologo{plainTeX}.
% \item[\xpackage{hobsub-hyperref}] is a collection of packages
%   that is used by package \xpackage{hyperref}.
% \end{description}
%
% Package \xpackage{hobsub} provides the macros that
% are used in the collection packages and might be used
% in other projects as well. But there are many caveats,
% some of them are discussed below.
%
% The most serious problem is consistency. A package
% that is available as standalone package and is part
% of a collection might have different versions.
% Because my packages are organized in a bundle and should
% be installed and updated as bundle, the risk seems low
% to me. However, if a single package is updated manually, for
% example for testing, then it should be loaded before the
% collection that contains the package with an older version.
% In a collection, the package will not be loaded again if it
% is already loaded.
%
% \subsection{Usage}
%
% The collection packages \xpackage{hobsub-generic} and
% \xpackage{hobsub-hyperref} are loaded as usual:
% \begin{quote}
%  \begin{tabular}{@{}ll@{}}
%    |\usepackage{hobsub-generic}| & \hologo{LaTeX}\\
%    |\input hobsub-generic.sty| & \hologo{plainTeX}
%  \end{tabular}
% \end{quote}
% or
% \begin{quote}|\usepackage{hobsub-hyperref}|\end{quote}
% Of course, both \cs{usepackage} or \cs{RequirePackage}
% can be used.
%
% If you need requirements on the version date of a package
% inside a collection, then specify them afterwards, for
% example:
% \begin{quote}
%   |\usepackage{hobsub-generic}|\\
%   |\usepackage{ltxcmds}[2010/12/04]% |^^A
%   {\itshape |\ltx@ifblank is needed|}
% \end{quote}
%
% Also it is not garanteed that a package will be always
% part of a collection. Therefore it does not harm,
% to make the requirements explicite \emph{after}
% a collection is loaded, see the previous example.
% \hologo{LaTeX} knows that a
% package is already loaded and does not load it again.
%
% \subsection{Supported features}
%
% \begin{description}
% \item[No reloads:] If a package is already loaded then it
%  is not reloaded again if it is part of a collection package.
% \item[\cs{listfiles}:]
%   Each package that is loaded in a collection is also
%   added to the output of \cs{listfiles}.
% \item[\cs{@currname}, \cs{@currext}]
%   are updated, they are used by \hologo{LaTeX}'s
%  \cs{ProvidesPackage}.
% \item[\cs{AtEndOfPackage}:]
%   \hologo{LaTeX}'s hook mechanism that is used for
%   \cs{AtEndOfPackage} is supported. However none of the
%   packages in the collections are using \cs{AtEndOfPackage},
%   therefore this feature is completely untested.
% \end{description}
%
% \subsection{Common limitations}
%
% \begin{description}
% \item[\cs{endinput}:]
%   Often a package is using \cs{endinput}, either added by
%   \xpackage{docstrip}, or to stop package loading at an earlier
%   time. Therefore \cs{endinput} must be catched during package
%   loading to prevent that the collection package is closed
%   too early before reading all packages it contains.
%
%   This redefinition of \cs{endinput} fails if the package loads
%   other files that contain \cs{endinput}. To some degree it could
%   be catched by redefining macros that load files. But this will
%   not work for the primitive \cs{input}.
% \item[Options:]
%   \hologo{LaTeX}'s package options are not supported.
%   Therefore the packages that are put in a collection must
%   not support options.
% \item[Hook packages:]
%   Some packages and classes provides hooks for packages, see
%   packages \xpackage{scrlfile}, \xpackage{filehook}
%   or class \xclass{memoir}. These hooks are not supported
%   for embedded packages inside a collection package.
%   Because of the problem with \cs{endinput}, these hooks
%   are not supported. The hooks might contain code that
%   loads other files that execute \cs{endinput}.
% \end{description}
%
% \subsection{Package \xpackage{hobsub-generic}}
%
% \begin{TableList}[bp]{generic}\TableGeneric\end{TableList}
%
% Collection `generic' contains some of the \hologo{plainTeX} compatible
% packages of my bundle, see table \ref{tab:generic}.
% It adds the following restrictions to any package
% that is part of this collection:
% \begin{description}
% \item[Generic formats:] The following formats must be supported:
%   \hologo{LaTeX}, \hologo{plainTeX} and even \hologo{iniTeX}.
%   This applies to the collection package \xpackage{hobsub-generic}
%   the same way.
% \item[Prefixed macros:]
%   Any package of this collection should only define macros with
%   prefixed names to avoid name clashes. Resources like registers
%   must not be allocated and other global states should not be
%   changed except for added macro definitions.
% \end{description}
% Thus the package \xpackage{hobsub-generic} can also preloaded
% in the format file with \hologo{iniTeX}.
%
% \subsection{Package \xpackage{hobsub-hyperref}}
%
% \begin{TableList}[tp]{hyperref}\TableHyperref\end{TableList}
%
% Collection `hyperref' contains additional packages that are
% used by package \xpackage{hyperref} and that does not fit
% in collection `generic', see table{tab:hyperref}.
% The latter collection package
% \xpackage{hobsub-generic} is loaded via a conventional
% \cs{RequirePackage}.
%
% Because most of the packages in the collection `hyperref'
% needs \hologo{LaTeX}, the collection requires \hologo{LaTeX}.
%
% \subsection{Package \xpackage{hobsub}}
%
% This section is for advanced users that might want to
% build a collection package. Then it can be done by
% using \xpackage{hobsub}:
% \begin{quote}
%   \dots\\
%   |\RequirePackage{hobsub}[2016/05/16]|\\
%   \dots\\
%   |\hobsub@StartPackage{foo1}|\\
%   \textit{\dots\unskip\ package code of \xfile{foo1.sty} \dots}\\
%   |\endinput|\\
%   |<hobsub>|\\
%   |\hobsub@StopPackage|\\
%   |\hobsub@StartPackage{bar2}|\\
%   \textit{\dots\unskip\ package code of \xfile{bar2.sty} \dots}\\
%   |\endinput|\\
%   |<hobsub>|\\
%   |\hobsub@StopPackage|
%   \dots
% \end{quote}
% But be aware of the caveats and limitations, see above.
% Only quite a small portion of the packages can be embedded
% like this.
%
% \clearpage
%
% \StopEventually{
% }
%
% \section{Implementation}
%
% \subsection{Reload check and package identification}
%
% \subsubsection{Package \xpackage{hobsub}}
%
%    \begin{macrocode}
%<*package>
%    \end{macrocode}
%    Reload check, especially if the package is not used with \LaTeX.
%    \begin{macrocode}
\begingroup\catcode61\catcode48\catcode32=10\relax%
  \catcode13=5 % ^^M
  \endlinechar=13 %
  \catcode35=6 % #
  \catcode39=12 % '
  \catcode44=12 % ,
  \catcode45=12 % -
  \catcode46=12 % .
  \catcode58=12 % :
  \catcode64=11 % @
  \catcode123=1 % {
  \catcode125=2 % }
  \expandafter\let\expandafter\x\csname ver@hobsub.sty\endcsname
  \ifx\x\relax % plain-TeX, first loading
  \else
    \def\empty{}%
    \ifx\x\empty % LaTeX, first loading,
      % variable is initialized, but \ProvidesPackage not yet seen
    \else
      \expandafter\ifx\csname PackageInfo\endcsname\relax
        \def\x#1#2{%
          \immediate\write-1{Package #1 Info: #2.}%
        }%
      \else
        \def\x#1#2{\PackageInfo{#1}{#2, stopped}}%
      \fi
      \x{hobsub}{The package is already loaded}%
      \aftergroup\endinput
    \fi
  \fi
\endgroup%
%    \end{macrocode}
%    Package identification:
%    \begin{macrocode}
\begingroup\catcode61\catcode48\catcode32=10\relax%
  \catcode13=5 % ^^M
  \endlinechar=13 %
  \catcode35=6 % #
  \catcode39=12 % '
  \catcode40=12 % (
  \catcode41=12 % )
  \catcode44=12 % ,
  \catcode45=12 % -
  \catcode46=12 % .
  \catcode47=12 % /
  \catcode58=12 % :
  \catcode64=11 % @
  \catcode91=12 % [
  \catcode93=12 % ]
  \catcode123=1 % {
  \catcode125=2 % }
  \expandafter\ifx\csname ProvidesPackage\endcsname\relax
    \def\x#1#2#3[#4]{\endgroup
      \immediate\write-1{Package: #3 #4}%
      \xdef#1{#4}%
    }%
  \else
    \def\x#1#2[#3]{\endgroup
      #2[{#3}]%
      \ifx#1\@undefined
        \xdef#1{#3}%
      \fi
      \ifx#1\relax
        \xdef#1{#3}%
      \fi
    }%
  \fi
\expandafter\x\csname ver@hobsub.sty\endcsname
\ProvidesPackage{hobsub}%
  [2016/05/16 v1.14 Construct package bundles (HO)]%
%    \end{macrocode}
%    \begin{macrocode}
%</package>
%    \end{macrocode}
%
% \subsubsection{Package \xpackage{hobsub-generic}}
%
%    \begin{macrocode}
%<*generic>
%    \end{macrocode}
%    Reload check, especially if the package is not used with \LaTeX.
%    \begin{macrocode}
\begingroup\catcode61\catcode48\catcode32=10\relax%
  \catcode13=5 % ^^M
  \endlinechar=13 %
  \catcode35=6 % #
  \catcode39=12 % '
  \catcode44=12 % ,
  \catcode45=12 % -
  \catcode46=12 % .
  \catcode58=12 % :
  \catcode64=11 % @
  \catcode123=1 % {
  \catcode125=2 % }
  \expandafter\let\expandafter\x\csname ver@hobsub-generic.sty\endcsname
  \ifx\x\relax % plain-TeX, first loading
  \else
    \def\empty{}%
    \ifx\x\empty % LaTeX, first loading,
      % variable is initialized, but \ProvidesPackage not yet seen
    \else
      \expandafter\ifx\csname PackageInfo\endcsname\relax
        \def\x#1#2{%
          \immediate\write-1{Package #1 Info: #2.}%
        }%
      \else
        \def\x#1#2{\PackageInfo{#1}{#2, stopped}}%
      \fi
      \x{hobsub-generic}{The package is already loaded}%
      \aftergroup\endinput
    \fi
  \fi
\endgroup%
%    \end{macrocode}
%    Package identification:
%    \begin{macrocode}
\begingroup\catcode61\catcode48\catcode32=10\relax%
  \catcode13=5 % ^^M
  \endlinechar=13 %
  \catcode35=6 % #
  \catcode39=12 % '
  \catcode40=12 % (
  \catcode41=12 % )
  \catcode44=12 % ,
  \catcode45=12 % -
  \catcode46=12 % .
  \catcode47=12 % /
  \catcode58=12 % :
  \catcode64=11 % @
  \catcode91=12 % [
  \catcode93=12 % ]
  \catcode123=1 % {
  \catcode125=2 % }
  \expandafter\ifx\csname ProvidesPackage\endcsname\relax
    \def\x#1#2#3[#4]{\endgroup
      \immediate\write-1{Package: #3 #4}%
      \xdef#1{#4}%
    }%
  \else
    \def\x#1#2[#3]{\endgroup
      #2[{#3}]%
      \ifx#1\@undefined
        \xdef#1{#3}%
      \fi
      \ifx#1\relax
        \xdef#1{#3}%
      \fi
    }%
  \fi
\expandafter\x\csname ver@hobsub-generic.sty\endcsname
\ProvidesPackage{hobsub-generic}%
  [2016/05/16 v1.14 Bundle oberdiek, subset generic (HO)]%
%    \end{macrocode}
%    \begin{macrocode}
%</generic>
%    \end{macrocode}
%
% \subsubsection{Package \xpackage{hobsub-hyperref}}
%
%    \begin{macrocode}
%<*hyperref>
%    \end{macrocode}
%    Reload check, especially if the package is not used with \LaTeX.
%    \begin{macrocode}
\begingroup\catcode61\catcode48\catcode32=10\relax%
  \catcode13=5 % ^^M
  \endlinechar=13 %
  \catcode35=6 % #
  \catcode39=12 % '
  \catcode44=12 % ,
  \catcode45=12 % -
  \catcode46=12 % .
  \catcode58=12 % :
  \catcode64=11 % @
  \catcode123=1 % {
  \catcode125=2 % }
  \expandafter\let\expandafter\x\csname ver@hobsub-hyperref.sty\endcsname
  \ifx\x\relax % plain-TeX, first loading
  \else
    \def\empty{}%
    \ifx\x\empty % LaTeX, first loading,
      % variable is initialized, but \ProvidesPackage not yet seen
    \else
      \expandafter\ifx\csname PackageInfo\endcsname\relax
        \def\x#1#2{%
          \immediate\write-1{Package #1 Info: #2.}%
        }%
      \else
        \def\x#1#2{\PackageInfo{#1}{#2, stopped}}%
      \fi
      \x{hobsub-hyperref}{The package is already loaded}%
      \aftergroup\endinput
    \fi
  \fi
\endgroup%
%    \end{macrocode}
%    Package identification:
%    \begin{macrocode}
\begingroup\catcode61\catcode48\catcode32=10\relax%
  \catcode13=5 % ^^M
  \endlinechar=13 %
  \catcode35=6 % #
  \catcode39=12 % '
  \catcode40=12 % (
  \catcode41=12 % )
  \catcode44=12 % ,
  \catcode45=12 % -
  \catcode46=12 % .
  \catcode47=12 % /
  \catcode58=12 % :
  \catcode64=11 % @
  \catcode91=12 % [
  \catcode93=12 % ]
  \catcode123=1 % {
  \catcode125=2 % }
  \expandafter\ifx\csname ProvidesPackage\endcsname\relax
    \def\x#1#2#3[#4]{\endgroup
      \immediate\write-1{Package: #3 #4}%
      \xdef#1{#4}%
    }%
  \else
    \def\x#1#2[#3]{\endgroup
      #2[{#3}]%
      \ifx#1\@undefined
        \xdef#1{#3}%
      \fi
      \ifx#1\relax
        \xdef#1{#3}%
      \fi
    }%
  \fi
\expandafter\x\csname ver@hobsub-hyperref.sty\endcsname
\ProvidesPackage{hobsub-hyperref}%
  [2016/05/16 v1.14 Bundle oberdiek, subset hyperref (HO)]%
%    \end{macrocode}
%    \begin{macrocode}
%</hyperref>
%    \end{macrocode}
%
% \subsection{Catcodes}
%
%    \begin{macrocode}
%<*package|generic|hyperref>
\begingroup\catcode61\catcode48\catcode32=10\relax%
  \catcode13=5 % ^^M
  \endlinechar=13 %
  \catcode123=1 % {
  \catcode125=2 % }
  \catcode64=11 % @
  \def\x{\endgroup
    \expandafter\edef\csname HOBsub@AtEnd\endcsname{%
      \endlinechar=\the\endlinechar\relax
      \catcode13=\the\catcode13\relax
      \catcode32=\the\catcode32\relax
      \catcode35=\the\catcode35\relax
      \catcode61=\the\catcode61\relax
      \catcode64=\the\catcode64\relax
      \catcode123=\the\catcode123\relax
      \catcode125=\the\catcode125\relax
    }%
  }%
\x\catcode61\catcode48\catcode32=10\relax%
\catcode13=5 % ^^M
\endlinechar=13 %
\catcode35=6 % #
\catcode64=11 % @
\catcode123=1 % {
\catcode125=2 % }
\def\TMP@EnsureCode#1#2{%
  \edef\HOBsub@AtEnd{%
    \HOBsub@AtEnd
    \catcode#1=\the\catcode#1\relax
  }%
  \catcode#1=#2\relax
}
\TMP@EnsureCode{39}{12}% '
\TMP@EnsureCode{40}{12}% (
\TMP@EnsureCode{41}{12}% )
\TMP@EnsureCode{45}{12}% -
\TMP@EnsureCode{46}{12}% .
\TMP@EnsureCode{47}{12}% /
\TMP@EnsureCode{58}{12}% :
\TMP@EnsureCode{60}{12}% <
\TMP@EnsureCode{62}{12}% >
\TMP@EnsureCode{96}{12}% `
\edef\HOBsub@AtEnd{\HOBsub@AtEnd\noexpand\endinput}
%    \end{macrocode}
%
%    \begin{macrocode}
%</package|generic|hyperref>
%    \end{macrocode}
%
% \subsection{Package \xpackage{hobsub-hyperref} loads \xpackage{hobsub-generic}}
%
%    \begin{macrocode}
%<*hyperref>
\NeedsTeXFormat{LaTeX2e}
\RequirePackage{hobsub-generic}[2016/05/16]
%</hyperref>
%    \end{macrocode}
%
% \subsection{Preamble}
%
%    \begin{macrocode}
%<*preamble>
\begingroup\expandafter\expandafter\expandafter\endgroup
\expandafter\ifx\csname ver@hobsub.sty\endcsname\relax
\else
  \expandafter\hobsub@GobbleRemainingPackage
\fi
%</preamble>
%    \end{macrocode}
%
%
%    \begin{macrocode}
%<*package|preamble>
%    \end{macrocode}
%
%    \begin{macro}{\HOBsub@OrgEndinput}
%    \begin{macrocode}
\begingroup\expandafter\expandafter\expandafter\endgroup
\expandafter\ifx\csname HOBsub@OrgEndinput\endcsname\relax
  \let\HOBsub@OrgEndinput\endinput
\fi
%    \end{macrocode}
%    \end{macro}
%    \begin{macro}{\hobsub@GobbleRemainingPackage}
%    \begin{macrocode}
\def\hobsub@GobbleRemainingPackage{%
  \begingroup
  \catcode92=14 % backslash: comment
  \catcode32=14 % space: comment
  \catcode35=14 % hash: comment
  \catcode123=14 % left brace: comment
  \catcode125=14 % right brace: comment
  \catcode60=3 % less: math
  \catcode62=4 % greater: align
  \endlinechar=-1 %
  \HOBsub@GobbleRemainingPackage
}
%    \end{macrocode}
%    \end{macro}
%    \begin{macro}{\HOBsub@GobbleRemainingPackage}
%    \begin{macrocode}
\catcode60=3 % less: dollar
\catcode62=4 % greater: align
\long\def\HOBsub@GobbleRemainingPackage#1<hobsub>{%
  \endgroup
}
\catcode60=12 % less: other
\catcode62=12 % greater: other
%    \end{macrocode}
%    \end{macro}
%
%    \begin{macro}{\hobsub@StartPackage}
%    \begin{macrocode}
\def\hobsub@StartPackage#1{%
  \begingroup\expandafter\expandafter\expandafter\endgroup
  \expandafter\ifx\csname ver@#1.sty\endcsname\relax
    \let\HOBsub@OrgCurrName\@currname
    \let\HOBsub@OrgCurrExt\@currext
    \csname @pushfilename\endcsname
    \def\@currname{#1}%
    \def\@currext{sty}%
    \expandafter\def\csname\@currname.\@currext-h@@k\endcsname{}%
    \let\endinput\hobsub@GobbleRemainingPackage
    \def\hobsub@StopPackage{%
      \let\hobsub@StopPackage\relax
      \HOBsub@StopPackage
%     \hobsub@Info{hobsub}{Package `#1' loaded}%
    }%
    \hobsub@AddToFileList{#1.sty}%
  \else
    \hobsub@Info{hobsub}{Skipping package `#1' (already loaded)}%
    \let\hobsub@StopPackage\relax
    \expandafter\hobsub@GobbleRemainingPackage
  \fi
}
%    \end{macrocode}
%    \end{macro}
%
%    \begin{macro}{\hobsub@StopPackage}
%    \begin{macrocode}
\let\hobsub@StopPackage\relax
%    \end{macrocode}
%    \end{macro}
%
%    \begin{macro}{\hobsub@Info}
%    \begin{macrocode}
\def\hobsub@Info#1#2{%
  \begingroup\expandafter\expandafter\expandafter\endgroup
  \expandafter\ifx\csname @PackageInfoNoLine\endcsname\relax
    \immediate\write-1{Package #1 Info: #2.}%
  \else
    \let\hobsub@Info\@PackageInfoNoLine
    \hobsub@Info{#1}{#2}%
  \fi
}
%    \end{macrocode}
%    \end{macro}
%
%    \begin{macro}{\HOBsub@StopPackage}
%    \begin{macrocode}
\def\HOBsub@StopPackage{%
  \csname\@currname.\@currext-h@@k\endcsname
  \let\endinput\HOBsub@OrgEndinput
  \csname @popfilename\endcsname
  \let\@currname\HOBsub@OrgCurrName
  \let\@currext\HOBsub@OrgCurrExt
}
%    \end{macrocode}
%    \end{macro}
%
%    \begin{macro}{\hobsub@AddToFileList}
%    \begin{macrocode}
\begingroup\expandafter\expandafter\expandafter\endgroup
\expandafter\ifx\csname @addtofilelist\endcsname\relax
  \def\hobsub@AddToFileList#1{}%
\else
  \def\hobsub@AddToFileList#1{%
    \@addtofilelist{#1}%
  }%
\fi
%    \end{macrocode}
%    \end{macro}
%
%    \begin{macrocode}
%</package|preamble>
%    \end{macrocode}
%
% \subsection{End of preamble for subset packages}
%
%    Add the end marker for possible skipping of previous
%    macros, if package \xpackage{hobsub} was already loaded.
%    \begin{macrocode}
%<*preamble>
\hobsub@GobbleRemainingPackage
<hobsub>
%    \end{macrocode}
%
%    All macros of package \xpackage{hobsub} are now available.
%    Now we officially load the package \xpackage{hobsub} inline
%    for proper package identification.
%    \begin{macrocode}
\hobsub@StartPackage{hobsub}%
\begingroup\expandafter\expandafter\expandafter\endgroup
\expandafter\ifx\csname ProvidesPackage\endcsname\relax
  \immediate\write-1{%
    Package: hobsub 2016/05/16 v1.14 Construct package bundles (HO)%
  }%
\else
  \ProvidesPackage{hobsub}%
  [2016/05/16 v1.14 Construct package bundles (HO)]%
\fi
%</preamble>
%    \end{macrocode}
%
%
%    \begin{macrocode}
%<*post|preamble>
 \endinput
<hobsub>
\hobsub@StopPackage
%</post|preamble>
%    \end{macrocode}
%
% \subsection{Package list}
%
%    \begin{macrocode}
%<infwarerr>\hobsub@StartPackage{infwarerr}
%<ltxcmds>\hobsub@StartPackage{ltxcmds}
%<ifluatex>\hobsub@StartPackage{ifluatex}
%<ifvtex>\hobsub@StartPackage{ifvtex}
%<intcalc>\hobsub@StartPackage{intcalc}
%<ifpdf>\hobsub@StartPackage{ifpdf}
%<etexcmds>\hobsub@StartPackage{etexcmds}
%<kvsetkeys>\hobsub@StartPackage{kvsetkeys}
%<kvdefinekeys>\hobsub@StartPackage{kvdefinekeys}
%<*luatex-loader>
\ifluatex
\else
  \expandafter\hobsub@GobbleRemainingPackage
\fi
\hobsub@StartPackage{luatex-loader}
%</luatex-loader>
%<pdftexcmds>\hobsub@StartPackage{pdftexcmds}
%<pdfescape>\hobsub@StartPackage{pdfescape}
%<bigintcalc>\hobsub@StartPackage{bigintcalc}
%<bitset>\hobsub@StartPackage{bitset}
%<uniquecounter>\hobsub@StartPackage{uniquecounter}
%<letltxmacro>\hobsub@StartPackage{letltxmacro}
%<hopatch>\hobsub@StartPackage{hopatch}
%<xcolor-patch>\hobsub@StartPackage{xcolor-patch}
%<atveryend>\hobsub@StartPackage{atveryend}
%<atbegshi>\hobsub@StartPackage{atbegshi}
%<refcount>\hobsub@StartPackage{refcount}
%<hycolor>\hobsub@StartPackage{hycolor}
%    \end{macrocode}
%
% \subsection{End of packages}
%
%    \begin{macrocode}
%<package|atend>\HOBsub@AtEnd%
%    \end{macrocode}
%
% \section{Test}
%
% \subsection{Catcode checks for loading}
%
%    \begin{macrocode}
%<*test1>
%    \end{macrocode}
%    \begin{macrocode}
\catcode`\{=1 %
\catcode`\}=2 %
\catcode`\#=6 %
\catcode`\@=11 %
\expandafter\ifx\csname count@\endcsname\relax
  \countdef\count@=255 %
\fi
\expandafter\ifx\csname @gobble\endcsname\relax
  \long\def\@gobble#1{}%
\fi
\expandafter\ifx\csname @firstofone\endcsname\relax
  \long\def\@firstofone#1{#1}%
\fi
\expandafter\ifx\csname loop\endcsname\relax
  \expandafter\@firstofone
\else
  \expandafter\@gobble
\fi
{%
  \def\loop#1\repeat{%
    \def\body{#1}%
    \iterate
  }%
  \def\iterate{%
    \body
      \let\next\iterate
    \else
      \let\next\relax
    \fi
    \next
  }%
  \let\repeat=\fi
}%
\def\RestoreCatcodes{}
\count@=0 %
\loop
  \edef\RestoreCatcodes{%
    \RestoreCatcodes
    \catcode\the\count@=\the\catcode\count@\relax
  }%
\ifnum\count@<255 %
  \advance\count@ 1 %
\repeat

\def\RangeCatcodeInvalid#1#2{%
  \count@=#1\relax
  \loop
    \catcode\count@=15 %
  \ifnum\count@<#2\relax
    \advance\count@ 1 %
  \repeat
}
\def\RangeCatcodeCheck#1#2#3{%
  \count@=#1\relax
  \loop
    \ifnum#3=\catcode\count@
    \else
      \errmessage{%
        Character \the\count@\space
        with wrong catcode \the\catcode\count@\space
        instead of \number#3%
      }%
    \fi
  \ifnum\count@<#2\relax
    \advance\count@ 1 %
  \repeat
}
\def\space{ }
\expandafter\ifx\csname LoadCommand\endcsname\relax
  \def\LoadCommand{\input hobsub.sty\relax}%
\fi
\def\Test{%
  \RangeCatcodeInvalid{0}{47}%
  \RangeCatcodeInvalid{58}{64}%
  \RangeCatcodeInvalid{91}{96}%
  \RangeCatcodeInvalid{123}{255}%
  \catcode`\@=12 %
  \catcode`\\=0 %
  \catcode`\%=14 %
  \LoadCommand
  \RangeCatcodeCheck{0}{36}{15}%
  \RangeCatcodeCheck{37}{37}{14}%
  \RangeCatcodeCheck{38}{47}{15}%
  \RangeCatcodeCheck{48}{57}{12}%
  \RangeCatcodeCheck{58}{63}{15}%
  \RangeCatcodeCheck{64}{64}{12}%
  \RangeCatcodeCheck{65}{90}{11}%
  \RangeCatcodeCheck{91}{91}{15}%
  \RangeCatcodeCheck{92}{92}{0}%
  \RangeCatcodeCheck{93}{96}{15}%
  \RangeCatcodeCheck{97}{122}{11}%
  \RangeCatcodeCheck{123}{255}{15}%
  \RestoreCatcodes
}
\Test
\csname @@end\endcsname
\end
%    \end{macrocode}
%    \begin{macrocode}
%</test1>
%    \end{macrocode}
%
% \section{Installation}
%
% \subsection{Download}
%
% \paragraph{Package.} This package is available on
% CTAN\footnote{\url{http://ctan.org/pkg/hobsub}}:
% \begin{description}
% \item[\CTAN{macros/latex/contrib/oberdiek/hobsub.dtx}] The source file.
% \item[\CTAN{macros/latex/contrib/oberdiek/hobsub.pdf}] Documentation.
% \end{description}
%
%
% \paragraph{Bundle.} All the packages of the bundle `oberdiek'
% are also available in a TDS compliant ZIP archive. There
% the packages are already unpacked and the documentation files
% are generated. The files and directories obey the TDS standard.
% \begin{description}
% \item[\CTAN{install/macros/latex/contrib/oberdiek.tds.zip}]
% \end{description}
% \emph{TDS} refers to the standard ``A Directory Structure
% for \TeX\ Files'' (\CTAN{tds/tds.pdf}). Directories
% with \xfile{texmf} in their name are usually organized this way.
%
% \subsection{Bundle installation}
%
% \paragraph{Unpacking.} Unpack the \xfile{oberdiek.tds.zip} in the
% TDS tree (also known as \xfile{texmf} tree) of your choice.
% Example (linux):
% \begin{quote}
%   |unzip oberdiek.tds.zip -d ~/texmf|
% \end{quote}
%
% \paragraph{Script installation.}
% Check the directory \xfile{TDS:scripts/oberdiek/} for
% scripts that need further installation steps.
% Package \xpackage{attachfile2} comes with the Perl script
% \xfile{pdfatfi.pl} that should be installed in such a way
% that it can be called as \texttt{pdfatfi}.
% Example (linux):
% \begin{quote}
%   |chmod +x scripts/oberdiek/pdfatfi.pl|\\
%   |cp scripts/oberdiek/pdfatfi.pl /usr/local/bin/|
% \end{quote}
%
% \subsection{Package installation}
%
% \paragraph{Unpacking.} The \xfile{.dtx} file is a self-extracting
% \docstrip\ archive. The files are extracted by running the
% \xfile{.dtx} through \plainTeX:
% \begin{quote}
%   \verb|tex hobsub.dtx|
% \end{quote}
%
% \paragraph{TDS.} Now the different files must be moved into
% the different directories in your installation TDS tree
% (also known as \xfile{texmf} tree):
% \begin{quote}
% \def\t{^^A
% \begin{tabular}{@{}>{\ttfamily}l@{ $\rightarrow$ }>{\ttfamily}l@{}}
%   hobsub.sty & tex/generic/oberdiek/hobsub.sty\\
%   hobsub-generic.sty & tex/generic/oberdiek/hobsub-generic.sty\\
%   hobsub-hyperref.sty & tex/generic/oberdiek/hobsub-hyperref.sty\\
%   hobsub.pdf & doc/latex/oberdiek/hobsub.pdf\\
%   test/hobsub-test1.tex & doc/latex/oberdiek/test/hobsub-test1.tex\\
%   hobsub.dtx & source/latex/oberdiek/hobsub.dtx\\
% \end{tabular}^^A
% }^^A
% \sbox0{\t}^^A
% \ifdim\wd0>\linewidth
%   \begingroup
%     \advance\linewidth by\leftmargin
%     \advance\linewidth by\rightmargin
%   \edef\x{\endgroup
%     \def\noexpand\lw{\the\linewidth}^^A
%   }\x
%   \def\lwbox{^^A
%     \leavevmode
%     \hbox to \linewidth{^^A
%       \kern-\leftmargin\relax
%       \hss
%       \usebox0
%       \hss
%       \kern-\rightmargin\relax
%     }^^A
%   }^^A
%   \ifdim\wd0>\lw
%     \sbox0{\small\t}^^A
%     \ifdim\wd0>\linewidth
%       \ifdim\wd0>\lw
%         \sbox0{\footnotesize\t}^^A
%         \ifdim\wd0>\linewidth
%           \ifdim\wd0>\lw
%             \sbox0{\scriptsize\t}^^A
%             \ifdim\wd0>\linewidth
%               \ifdim\wd0>\lw
%                 \sbox0{\tiny\t}^^A
%                 \ifdim\wd0>\linewidth
%                   \lwbox
%                 \else
%                   \usebox0
%                 \fi
%               \else
%                 \lwbox
%               \fi
%             \else
%               \usebox0
%             \fi
%           \else
%             \lwbox
%           \fi
%         \else
%           \usebox0
%         \fi
%       \else
%         \lwbox
%       \fi
%     \else
%       \usebox0
%     \fi
%   \else
%     \lwbox
%   \fi
% \else
%   \usebox0
% \fi
% \end{quote}
% If you have a \xfile{docstrip.cfg} that configures and enables \docstrip's
% TDS installing feature, then some files can already be in the right
% place, see the documentation of \docstrip.
%
% \subsection{Refresh file name databases}
%
% If your \TeX~distribution
% (\teTeX, \mikTeX, \dots) relies on file name databases, you must refresh
% these. For example, \teTeX\ users run \verb|texhash| or
% \verb|mktexlsr|.
%
% \subsection{Some details for the interested}
%
% \paragraph{Attached source.}
%
% The PDF documentation on CTAN also includes the
% \xfile{.dtx} source file. It can be extracted by
% AcrobatReader 6 or higher. Another option is \textsf{pdftk},
% e.g. unpack the file into the current directory:
% \begin{quote}
%   \verb|pdftk hobsub.pdf unpack_files output .|
% \end{quote}
%
% \paragraph{Unpacking with \LaTeX.}
% The \xfile{.dtx} chooses its action depending on the format:
% \begin{description}
% \item[\plainTeX:] Run \docstrip\ and extract the files.
% \item[\LaTeX:] Generate the documentation.
% \end{description}
% If you insist on using \LaTeX\ for \docstrip\ (really,
% \docstrip\ does not need \LaTeX), then inform the autodetect routine
% about your intention:
% \begin{quote}
%   \verb|latex \let\install=y% \iffalse meta-comment
%
% File: hobsub.dtx
% Version: 2016/05/16 v1.14
% Info: Construct package bundles
%
% Copyright (C) 2011, 2012 by
%    Heiko Oberdiek <heiko.oberdiek at googlemail.com>
%    2016
%    https://github.com/ho-tex/oberdiek/issues
%
% This work may be distributed and/or modified under the
% conditions of the LaTeX Project Public License, either
% version 1.3c of this license or (at your option) any later
% version. This version of this license is in
%    http://www.latex-project.org/lppl/lppl-1-3c.txt
% and the latest version of this license is in
%    http://www.latex-project.org/lppl.txt
% and version 1.3 or later is part of all distributions of
% LaTeX version 2005/12/01 or later.
%
% This work has the LPPL maintenance status "maintained".
%
% This Current Maintainer of this work is Heiko Oberdiek.
%
% The Base Interpreter refers to any `TeX-Format',
% because some files are installed in TDS:tex/generic//.
%
% This work consists of the main source file hobsub.dtx
% and the derived files
%    hobsub.sty, hobsub.pdf, hobsub.ins, hobsub.drv, hobsub-generic.sty,
%    hobsub-hyperref.sty, hobsub-test1.tex.
%
% Distribution:
%    CTAN:macros/latex/contrib/oberdiek/hobsub.dtx
%    CTAN:macros/latex/contrib/oberdiek/hobsub.pdf
%
% Unpacking:
%    (a) If hobsub.ins is present:
%           tex hobsub.ins
%    (b) Without hobsub.ins:
%           tex hobsub.dtx
%    (c) If you insist on using LaTeX
%           latex \let\install=y\input{hobsub.dtx}
%        (quote the arguments according to the demands of your shell)
%
% Documentation:
%    (a) If hobsub.drv is present:
%           latex hobsub.drv
%    (b) Without hobsub.drv:
%           latex hobsub.dtx; ...
%    The class ltxdoc loads the configuration file ltxdoc.cfg
%    if available. Here you can specify further options, e.g.
%    use A4 as paper format:
%       \PassOptionsToClass{a4paper}{article}
%
%    Programm calls to get the documentation (example):
%       pdflatex hobsub.dtx
%       makeindex -s gind.ist hobsub.idx
%       pdflatex hobsub.dtx
%       makeindex -s gind.ist hobsub.idx
%       pdflatex hobsub.dtx
%
% Installation:
%    TDS:tex/generic/oberdiek/hobsub.sty
%    TDS:tex/generic/oberdiek/hobsub-generic.sty
%    TDS:tex/generic/oberdiek/hobsub-hyperref.sty
%    TDS:doc/latex/oberdiek/hobsub.pdf
%    TDS:doc/latex/oberdiek/test/hobsub-test1.tex
%    TDS:source/latex/oberdiek/hobsub.dtx
%
%<*ignore>
\begingroup
  \catcode123=1 %
  \catcode125=2 %
  \def\x{LaTeX2e}%
\expandafter\endgroup
\ifcase 0\ifx\install y1\fi\expandafter
         \ifx\csname processbatchFile\endcsname\relax\else1\fi
         \ifx\fmtname\x\else 1\fi\relax
\else\csname fi\endcsname
%</ignore>
%<*install>
\input docstrip.tex
\Msg{************************************************************************}
\Msg{* Installation}
\Msg{* Package: hobsub 2016/05/16 v1.14 Construct package bundles (HO)}
\Msg{************************************************************************}

\keepsilent
\askforoverwritefalse

\let\MetaPrefix\relax
\preamble

This is a generated file.

Project: hobsub
Version: 2016/05/16 v1.14

Copyright (C) 2011, 2012 by
   Heiko Oberdiek <heiko.oberdiek at googlemail.com>

This work may be distributed and/or modified under the
conditions of the LaTeX Project Public License, either
version 1.3c of this license or (at your option) any later
version. This version of this license is in
   http://www.latex-project.org/lppl/lppl-1-3c.txt
and the latest version of this license is in
   http://www.latex-project.org/lppl.txt
and version 1.3 or later is part of all distributions of
LaTeX version 2005/12/01 or later.

This work has the LPPL maintenance status "maintained".

This Current Maintainer of this work is Heiko Oberdiek.

The Base Interpreter refers to any `TeX-Format',
because some files are installed in TDS:tex/generic//.

This work consists of the main source file hobsub.dtx
and the derived files
   hobsub.sty, hobsub.pdf, hobsub.ins, hobsub.drv, hobsub-generic.sty,
   hobsub-hyperref.sty, hobsub-test1.tex.

\endpreamble
\let\MetaPrefix\DoubleperCent

\generate{%
  \file{hobsub.ins}{\from{hobsub.dtx}{install}}%
  \file{hobsub.drv}{\from{hobsub.dtx}{driver}}%
  \usedir{tex/generic/oberdiek}%
  \file{hobsub.sty}{\from{hobsub.dtx}{package}}%
  \usedir{doc/latex/oberdiek/test}%
  \file{hobsub-test1.tex}{\from{hobsub.dtx}{test1}}%
  \nopreamble
  \nopostamble
  \usedir{source/latex/oberdiek/catalogue}%
  \file{hobsub.xml}{\from{hobsub.dtx}{catalogue}}%
}
\let\hobsubSavedNormalLine\normalLine
\def\hobsubGobble#1{}
\begingroup
  \lccode`\0=32 % space
  \lccode`\1=35 % hash
  \lccode`\2=37 % percent
  \lccode`\3=92 % backslash
  \lccode`\4=123 % {
  \lccode`\5=125 % }
\lowercase{\endgroup
  \let\hobsubTokenSpaceOther= 0%
  \let\hobsubTokenHash= 1%
  \let\hobsubTokenPercent= 2%
  \let\hobsubTokenBackslash= 3%
  \let\hobsubTokenBraceLeft= 4%
  \let\hobsubTokenBraceRight= 5%
  \def\hobsubNormalLineWithSpace{%
    \hobsubSavedNormalLine0%
  }%
}
\def\normalLine{%
  \futurelet\hobsubToken\hobsubFirst
}
\def\hobsubFirst{%
  \ifx\hobsubToken\hobsubTokenSpaceOther
    \expandafter\expandafter\expandafter\futurelet
    \expandafter\expandafter\expandafter\hobsubToken
    \expandafter\expandafter\expandafter\hobsubNext
    \expandafter\hobsubGobble
  \else
    \expandafter\hobsubSavedNormalLine
  \fi
}
\def\hobsubNext{%
  \ifx\hobsubToken\hobsubTokenSpaceOther
    \expandafter\expandafter\expandafter\futurelet
    \expandafter\expandafter\expandafter\hobsubToken
    \expandafter\expandafter\expandafter\hobsubNext 
    \expandafter\hobsubGobble
  \else
    \let\hobsubWithSpace=Y%
    \ifx\hobsubToken\hobsubTokenBackslash
      \let\hobsubWithSpace=N%
    \else
      \ifx\hobsubToken\hobsubTokenBraceRight
        \let\hobsubWithSpace=N%
      \else
        \ifx\hobsubToken\hobsubTokenBraceLeft
          \let\hobsubWithSpace=N%
        \else
          \ifx\hobsubToken\hobsubTokenPercent
            \let\hobsubWithSpace=N%
          \else
            \ifx\hobsubToken\hobsubTokenHash
              \let\hobsubWithSpace=N%
            \fi
          \fi
        \fi
      \fi
    \fi
    \ifx\hobsubWithSpace Y%
      \expandafter\expandafter\expandafter\hobsubNormalLineWithSpace
    \else
      \expandafter\expandafter\expandafter\hobsubSavedNormalLine
    \fi
  \fi
}
\generate{%
  \usedir{tex/generic/oberdiek}%
  \file{hobsub-generic.sty}{%
    \from{hobsub.dtx}{generic,preamble,infwarerr}%
    \from{infwarerr.dtx}{package}%
    \from{hobsub.dtx}{post,ltxcmds}%
    \from{ltxcmds.dtx}{package}%
    \from{hobsub.dtx}{post,ifluatex}%
    \from{ifluatex.dtx}{package}%
    \from{hobsub.dtx}{post,ifvtex}%
    \from{ifvtex.dtx}{package}%
    \from{hobsub.dtx}{post,intcalc}%
    \from{intcalc.dtx}{package}%
    \from{hobsub.dtx}{post,ifpdf}%
    \from{ifpdf.dtx}{package}%
    \from{hobsub.dtx}{post,etexcmds}%
    \from{etexcmds.dtx}{package}%
    \from{hobsub.dtx}{post,kvsetkeys}%
    \from{kvsetkeys.dtx}{package}%
    \from{hobsub.dtx}{post,kvdefinekeys}%
    \from{kvdefinekeys.dtx}{package}%
    \from{hobsub.dtx}{post,luatex-loader}%
    \from{luatex.dtx}{loader}%
    \from{hobsub.dtx}{post,pdftexcmds}%
    \from{pdftexcmds.dtx}{package}%
    \from{hobsub.dtx}{post,pdfescape}%
    \from{pdfescape.dtx}{package}%
    \from{hobsub.dtx}{post,bigintcalc}%
    \from{bigintcalc.dtx}{package}%
    \from{hobsub.dtx}{post,bitset}%
    \from{bitset.dtx}{package}%
    \from{hobsub.dtx}{post,uniquecounter}%
    \from{uniquecounter.dtx}{package}%
    \from{hobsub.dtx}{post,atend}%
  }%
}
\generate{%
  \usedir{tex/generic/oberdiek}%
  \file{hobsub-hyperref.sty}{%
    \from{hobsub.dtx}{hyperref,preamble,,letltxmacro}%
    \from{letltxmacro.dtx}{package}%
    \from{hobsub.dtx}{post,hopatch}%
    \from{hopatch.dtx}{package}%
    \from{hobsub.dtx}{post,xcolor-patch}%
    \from{hycolor.dtx}{xcolor}%
    \from{hobsub.dtx}{post,atveryend}%
    \from{atveryend.dtx}{package}%
    \from{hobsub.dtx}{post,atbegshi}%
    \from{atbegshi.dtx}{package}%
    \from{hobsub.dtx}{post,refcount}%
    \from{refcount.dtx}{package}%
    \from{hobsub.dtx}{post,hycolor}%
    \from{hycolor.dtx}{package}%
    \from{hobsub.dtx}{post}%
  }%
}
\let\normalLine\hobsubSavedNormalLine

\catcode32=13\relax% active space
\let =\space%
\Msg{************************************************************************}
\Msg{*}
\Msg{* To finish the installation you have to move the following}
\Msg{* files into a directory searched by TeX:}
\Msg{*}
\Msg{*     hobsub.sty, hobsub-generic.sty, hobsub-hyperref.sty}
\Msg{*}
\Msg{* To produce the documentation run the file `hobsub.drv'}
\Msg{* through LaTeX.}
\Msg{*}
\Msg{* Happy TeXing!}
\Msg{*}
\Msg{************************************************************************}

\endbatchfile
%</install>
%<*ignore>
\fi
%</ignore>
%<*driver>
\NeedsTeXFormat{LaTeX2e}
\ProvidesFile{hobsub.drv}%
  [2016/05/16 v1.14 Construct package bundles (HO)]%
\documentclass{ltxdoc}
\usepackage{holtxdoc}[2011/11/22]
\usepackage{array}
\usepackage{caption}
\renewcommand*{\bottomfraction}{.5}
\renewcommand*{\topfraction}{.5}
\begin{document}
  \DocInput{hobsub.dtx}%
\end{document}
%</driver>
% \fi
%
%
% \CharacterTable
%  {Upper-case    \A\B\C\D\E\F\G\H\I\J\K\L\M\N\O\P\Q\R\S\T\U\V\W\X\Y\Z
%   Lower-case    \a\b\c\d\e\f\g\h\i\j\k\l\m\n\o\p\q\r\s\t\u\v\w\x\y\z
%   Digits        \0\1\2\3\4\5\6\7\8\9
%   Exclamation   \!     Double quote  \"     Hash (number) \#
%   Dollar        \$     Percent       \%     Ampersand     \&
%   Acute accent  \'     Left paren    \(     Right paren   \)
%   Asterisk      \*     Plus          \+     Comma         \,
%   Minus         \-     Point         \.     Solidus       \/
%   Colon         \:     Semicolon     \;     Less than     \<
%   Equals        \=     Greater than  \>     Question mark \?
%   Commercial at \@     Left bracket  \[     Backslash     \\
%   Right bracket \]     Circumflex    \^     Underscore    \_
%   Grave accent  \`     Left brace    \{     Vertical bar  \|
%   Right brace   \}     Tilde         \~}
%
% \GetFileInfo{hobsub.drv}
%
% \title{The \xpackage{hobsub} package}
% \date{2016/05/16 v1.14}
% \author{Heiko Oberdiek\thanks
% {Please report any issues at https://github.com/ho-tex/oberdiek/issues}\\
% \xemail{heiko.oberdiek at googlemail.com}}
%
% \maketitle
%
% \begin{abstract}
% Package \xpackage{hobsub} implements the idea of loading
% several packages in one big collection package.
% \end{abstract}
%
% \tableofcontents
%
% \def\TableGeneric{^^A
%   \TableEntry{\xpackage{infwarerr}}{}{2010/04/08 v1.3}^^A
%   \TableEntry{\xpackage{ltxcmds}}{}{2011/11/09 v1.22}^^A
%   \TableEntry{\xpackage{ifluatex}}{}{2010/03/01 v1.3}^^A
%   \TableEntry{\xpackage{ifvtex}}{}{2010/03/01 v1.5}^^A
%   \TableEntry{\xpackage{intcalc}}{}{2007/09/27 v1.1}^^A
%   \TableEntry{\xpackage{ifpdf}}{\xpackage{ifluatex}}{2011/01/30 v2.3}^^A
%   \TableEntry{\xpackage{etexcmds}}{}{2011/02/16 v1.5}^^A
%   \TableEntry{\xpackage{kvsetkeys}}{}{2012/04/25 v1.16}^^A
%   \TableEntry{\xpackage{kvdefinekeys}}{}{2011/04/07 v1.3}^^A
%   \TableEntry{\xpackage{luatex-loader}}{}{2010/03/09 v0.4}^^A
%   \TableEntry{\xpackage{pdftexcmds}}{\xpackage{ifluatex}, \xpackage{ifpdf}, \xpackage{infwarerr}, \xpackage{ltxcmds}, \xpackage{luatex-loader}}{2011/11/29 v0.20}^^A
%   \TableEntry{\xpackage{pdfescape}}{\xpackage{ltxcmds}, \xpackage{pdftexcmds}}{2011/11/25 v1.13}^^A
%   \TableEntry{\xpackage{bigintcalc}}{\xpackage{pdftexcmds}}{2012/04/08 v1.3}^^A
%   \TableEntry{\xpackage{bitset}}{\xpackage{bigintcalc}, \xpackage{infwarerr}, \xpackage{intcalc}}{2011/01/30 v1.1}^^A
%   \TableEntry{\xpackage{uniquecounter}}{\xpackage{bigintcalc}, \xpackage{infwarerr}}{2011/01/30 v1.2}^^A
% }
% \def\TableHyperref{^^A
%   \TableEntry{\xpackage{letltxmacro}}{}{2010/09/02 v1.4}^^A
%   \TableEntry{\xpackage{hopatch}}{}{2016/05/16 v1.2}^^A
%   \TableEntry{\xpackage{xcolor-patch}}{}{2011/01/30 v1.7}^^A
%   \TableEntry{\xpackage{atveryend}}{}{2011/06/30 v1.8}^^A
%   \TableEntry{\xpackage{atbegshi}}{\xpackage{ifpdf}, \xpackage{infwarerr}, \xpackage{ltxcmds}}{2011/10/05 v1.16}^^A
%   \TableEntry{\xpackage{refcount}}{\xpackage{infwarerr}, \xpackage{ltxcmds}}{2011/10/16 v3.4}^^A
%   \TableEntry{\xpackage{hycolor}}{\xpackage{xcolor-patch}}{2011/01/30 v1.7}^^A
% }
%
% \newenvironment{TableList}[2][htbp]{^^A
%   \begin{table}[{#1}]^^A
%   \centering
%   \caption{Overview \xpackage{hobsub-#2}}^^A
%   \label{tab:#2}^^A
%   \begin{tabular}{ll>{\raggedright}p{.4\linewidth}}^^A
%   \bfseries Package & \bfseries Release &
%   \bfseries Dependencies\tabularnewline
%   \hline
% }{^^A
%   \end{tabular}^^A
%   \end{table}^^A
% }
% \newcommand*{\TableEntry}[3]{%
%   #1&#3&#2\tabularnewline
% }
%
% \renewcommand*{\arraystretch}{1.1}
% \setlength{\extrarowheight}{2pt}
%
% \section{Documentation}
%
% In January 2011 the mailing list lualatex-dev discussed
% ``quietening lualatex console output'' \cite{lualatex:quiet}.
% Inspired from this, I implemented this package \xpackage{hobsub}
% to load several packages in one file. The package is applied for
% two packages collections that are subsets of my package bundle
% \xpackage{oberdiek}.
% \begin{description}
% \item[\xpackage{hobsub-generic}] is a collection of packages
%   that can also be used with \hologo{plainTeX}.
% \item[\xpackage{hobsub-hyperref}] is a collection of packages
%   that is used by package \xpackage{hyperref}.
% \end{description}
%
% Package \xpackage{hobsub} provides the macros that
% are used in the collection packages and might be used
% in other projects as well. But there are many caveats,
% some of them are discussed below.
%
% The most serious problem is consistency. A package
% that is available as standalone package and is part
% of a collection might have different versions.
% Because my packages are organized in a bundle and should
% be installed and updated as bundle, the risk seems low
% to me. However, if a single package is updated manually, for
% example for testing, then it should be loaded before the
% collection that contains the package with an older version.
% In a collection, the package will not be loaded again if it
% is already loaded.
%
% \subsection{Usage}
%
% The collection packages \xpackage{hobsub-generic} and
% \xpackage{hobsub-hyperref} are loaded as usual:
% \begin{quote}
%  \begin{tabular}{@{}ll@{}}
%    |\usepackage{hobsub-generic}| & \hologo{LaTeX}\\
%    |\input hobsub-generic.sty| & \hologo{plainTeX}
%  \end{tabular}
% \end{quote}
% or
% \begin{quote}|\usepackage{hobsub-hyperref}|\end{quote}
% Of course, both \cs{usepackage} or \cs{RequirePackage}
% can be used.
%
% If you need requirements on the version date of a package
% inside a collection, then specify them afterwards, for
% example:
% \begin{quote}
%   |\usepackage{hobsub-generic}|\\
%   |\usepackage{ltxcmds}[2010/12/04]% |^^A
%   {\itshape |\ltx@ifblank is needed|}
% \end{quote}
%
% Also it is not garanteed that a package will be always
% part of a collection. Therefore it does not harm,
% to make the requirements explicite \emph{after}
% a collection is loaded, see the previous example.
% \hologo{LaTeX} knows that a
% package is already loaded and does not load it again.
%
% \subsection{Supported features}
%
% \begin{description}
% \item[No reloads:] If a package is already loaded then it
%  is not reloaded again if it is part of a collection package.
% \item[\cs{listfiles}:]
%   Each package that is loaded in a collection is also
%   added to the output of \cs{listfiles}.
% \item[\cs{@currname}, \cs{@currext}]
%   are updated, they are used by \hologo{LaTeX}'s
%  \cs{ProvidesPackage}.
% \item[\cs{AtEndOfPackage}:]
%   \hologo{LaTeX}'s hook mechanism that is used for
%   \cs{AtEndOfPackage} is supported. However none of the
%   packages in the collections are using \cs{AtEndOfPackage},
%   therefore this feature is completely untested.
% \end{description}
%
% \subsection{Common limitations}
%
% \begin{description}
% \item[\cs{endinput}:]
%   Often a package is using \cs{endinput}, either added by
%   \xpackage{docstrip}, or to stop package loading at an earlier
%   time. Therefore \cs{endinput} must be catched during package
%   loading to prevent that the collection package is closed
%   too early before reading all packages it contains.
%
%   This redefinition of \cs{endinput} fails if the package loads
%   other files that contain \cs{endinput}. To some degree it could
%   be catched by redefining macros that load files. But this will
%   not work for the primitive \cs{input}.
% \item[Options:]
%   \hologo{LaTeX}'s package options are not supported.
%   Therefore the packages that are put in a collection must
%   not support options.
% \item[Hook packages:]
%   Some packages and classes provides hooks for packages, see
%   packages \xpackage{scrlfile}, \xpackage{filehook}
%   or class \xclass{memoir}. These hooks are not supported
%   for embedded packages inside a collection package.
%   Because of the problem with \cs{endinput}, these hooks
%   are not supported. The hooks might contain code that
%   loads other files that execute \cs{endinput}.
% \end{description}
%
% \subsection{Package \xpackage{hobsub-generic}}
%
% \begin{TableList}[bp]{generic}\TableGeneric\end{TableList}
%
% Collection `generic' contains some of the \hologo{plainTeX} compatible
% packages of my bundle, see table \ref{tab:generic}.
% It adds the following restrictions to any package
% that is part of this collection:
% \begin{description}
% \item[Generic formats:] The following formats must be supported:
%   \hologo{LaTeX}, \hologo{plainTeX} and even \hologo{iniTeX}.
%   This applies to the collection package \xpackage{hobsub-generic}
%   the same way.
% \item[Prefixed macros:]
%   Any package of this collection should only define macros with
%   prefixed names to avoid name clashes. Resources like registers
%   must not be allocated and other global states should not be
%   changed except for added macro definitions.
% \end{description}
% Thus the package \xpackage{hobsub-generic} can also preloaded
% in the format file with \hologo{iniTeX}.
%
% \subsection{Package \xpackage{hobsub-hyperref}}
%
% \begin{TableList}[tp]{hyperref}\TableHyperref\end{TableList}
%
% Collection `hyperref' contains additional packages that are
% used by package \xpackage{hyperref} and that does not fit
% in collection `generic', see table{tab:hyperref}.
% The latter collection package
% \xpackage{hobsub-generic} is loaded via a conventional
% \cs{RequirePackage}.
%
% Because most of the packages in the collection `hyperref'
% needs \hologo{LaTeX}, the collection requires \hologo{LaTeX}.
%
% \subsection{Package \xpackage{hobsub}}
%
% This section is for advanced users that might want to
% build a collection package. Then it can be done by
% using \xpackage{hobsub}:
% \begin{quote}
%   \dots\\
%   |\RequirePackage{hobsub}[2016/05/16]|\\
%   \dots\\
%   |\hobsub@StartPackage{foo1}|\\
%   \textit{\dots\unskip\ package code of \xfile{foo1.sty} \dots}\\
%   |\endinput|\\
%   |<hobsub>|\\
%   |\hobsub@StopPackage|\\
%   |\hobsub@StartPackage{bar2}|\\
%   \textit{\dots\unskip\ package code of \xfile{bar2.sty} \dots}\\
%   |\endinput|\\
%   |<hobsub>|\\
%   |\hobsub@StopPackage|
%   \dots
% \end{quote}
% But be aware of the caveats and limitations, see above.
% Only quite a small portion of the packages can be embedded
% like this.
%
% \clearpage
%
% \StopEventually{
% }
%
% \section{Implementation}
%
% \subsection{Reload check and package identification}
%
% \subsubsection{Package \xpackage{hobsub}}
%
%    \begin{macrocode}
%<*package>
%    \end{macrocode}
%    Reload check, especially if the package is not used with \LaTeX.
%    \begin{macrocode}
\begingroup\catcode61\catcode48\catcode32=10\relax%
  \catcode13=5 % ^^M
  \endlinechar=13 %
  \catcode35=6 % #
  \catcode39=12 % '
  \catcode44=12 % ,
  \catcode45=12 % -
  \catcode46=12 % .
  \catcode58=12 % :
  \catcode64=11 % @
  \catcode123=1 % {
  \catcode125=2 % }
  \expandafter\let\expandafter\x\csname ver@hobsub.sty\endcsname
  \ifx\x\relax % plain-TeX, first loading
  \else
    \def\empty{}%
    \ifx\x\empty % LaTeX, first loading,
      % variable is initialized, but \ProvidesPackage not yet seen
    \else
      \expandafter\ifx\csname PackageInfo\endcsname\relax
        \def\x#1#2{%
          \immediate\write-1{Package #1 Info: #2.}%
        }%
      \else
        \def\x#1#2{\PackageInfo{#1}{#2, stopped}}%
      \fi
      \x{hobsub}{The package is already loaded}%
      \aftergroup\endinput
    \fi
  \fi
\endgroup%
%    \end{macrocode}
%    Package identification:
%    \begin{macrocode}
\begingroup\catcode61\catcode48\catcode32=10\relax%
  \catcode13=5 % ^^M
  \endlinechar=13 %
  \catcode35=6 % #
  \catcode39=12 % '
  \catcode40=12 % (
  \catcode41=12 % )
  \catcode44=12 % ,
  \catcode45=12 % -
  \catcode46=12 % .
  \catcode47=12 % /
  \catcode58=12 % :
  \catcode64=11 % @
  \catcode91=12 % [
  \catcode93=12 % ]
  \catcode123=1 % {
  \catcode125=2 % }
  \expandafter\ifx\csname ProvidesPackage\endcsname\relax
    \def\x#1#2#3[#4]{\endgroup
      \immediate\write-1{Package: #3 #4}%
      \xdef#1{#4}%
    }%
  \else
    \def\x#1#2[#3]{\endgroup
      #2[{#3}]%
      \ifx#1\@undefined
        \xdef#1{#3}%
      \fi
      \ifx#1\relax
        \xdef#1{#3}%
      \fi
    }%
  \fi
\expandafter\x\csname ver@hobsub.sty\endcsname
\ProvidesPackage{hobsub}%
  [2016/05/16 v1.14 Construct package bundles (HO)]%
%    \end{macrocode}
%    \begin{macrocode}
%</package>
%    \end{macrocode}
%
% \subsubsection{Package \xpackage{hobsub-generic}}
%
%    \begin{macrocode}
%<*generic>
%    \end{macrocode}
%    Reload check, especially if the package is not used with \LaTeX.
%    \begin{macrocode}
\begingroup\catcode61\catcode48\catcode32=10\relax%
  \catcode13=5 % ^^M
  \endlinechar=13 %
  \catcode35=6 % #
  \catcode39=12 % '
  \catcode44=12 % ,
  \catcode45=12 % -
  \catcode46=12 % .
  \catcode58=12 % :
  \catcode64=11 % @
  \catcode123=1 % {
  \catcode125=2 % }
  \expandafter\let\expandafter\x\csname ver@hobsub-generic.sty\endcsname
  \ifx\x\relax % plain-TeX, first loading
  \else
    \def\empty{}%
    \ifx\x\empty % LaTeX, first loading,
      % variable is initialized, but \ProvidesPackage not yet seen
    \else
      \expandafter\ifx\csname PackageInfo\endcsname\relax
        \def\x#1#2{%
          \immediate\write-1{Package #1 Info: #2.}%
        }%
      \else
        \def\x#1#2{\PackageInfo{#1}{#2, stopped}}%
      \fi
      \x{hobsub-generic}{The package is already loaded}%
      \aftergroup\endinput
    \fi
  \fi
\endgroup%
%    \end{macrocode}
%    Package identification:
%    \begin{macrocode}
\begingroup\catcode61\catcode48\catcode32=10\relax%
  \catcode13=5 % ^^M
  \endlinechar=13 %
  \catcode35=6 % #
  \catcode39=12 % '
  \catcode40=12 % (
  \catcode41=12 % )
  \catcode44=12 % ,
  \catcode45=12 % -
  \catcode46=12 % .
  \catcode47=12 % /
  \catcode58=12 % :
  \catcode64=11 % @
  \catcode91=12 % [
  \catcode93=12 % ]
  \catcode123=1 % {
  \catcode125=2 % }
  \expandafter\ifx\csname ProvidesPackage\endcsname\relax
    \def\x#1#2#3[#4]{\endgroup
      \immediate\write-1{Package: #3 #4}%
      \xdef#1{#4}%
    }%
  \else
    \def\x#1#2[#3]{\endgroup
      #2[{#3}]%
      \ifx#1\@undefined
        \xdef#1{#3}%
      \fi
      \ifx#1\relax
        \xdef#1{#3}%
      \fi
    }%
  \fi
\expandafter\x\csname ver@hobsub-generic.sty\endcsname
\ProvidesPackage{hobsub-generic}%
  [2016/05/16 v1.14 Bundle oberdiek, subset generic (HO)]%
%    \end{macrocode}
%    \begin{macrocode}
%</generic>
%    \end{macrocode}
%
% \subsubsection{Package \xpackage{hobsub-hyperref}}
%
%    \begin{macrocode}
%<*hyperref>
%    \end{macrocode}
%    Reload check, especially if the package is not used with \LaTeX.
%    \begin{macrocode}
\begingroup\catcode61\catcode48\catcode32=10\relax%
  \catcode13=5 % ^^M
  \endlinechar=13 %
  \catcode35=6 % #
  \catcode39=12 % '
  \catcode44=12 % ,
  \catcode45=12 % -
  \catcode46=12 % .
  \catcode58=12 % :
  \catcode64=11 % @
  \catcode123=1 % {
  \catcode125=2 % }
  \expandafter\let\expandafter\x\csname ver@hobsub-hyperref.sty\endcsname
  \ifx\x\relax % plain-TeX, first loading
  \else
    \def\empty{}%
    \ifx\x\empty % LaTeX, first loading,
      % variable is initialized, but \ProvidesPackage not yet seen
    \else
      \expandafter\ifx\csname PackageInfo\endcsname\relax
        \def\x#1#2{%
          \immediate\write-1{Package #1 Info: #2.}%
        }%
      \else
        \def\x#1#2{\PackageInfo{#1}{#2, stopped}}%
      \fi
      \x{hobsub-hyperref}{The package is already loaded}%
      \aftergroup\endinput
    \fi
  \fi
\endgroup%
%    \end{macrocode}
%    Package identification:
%    \begin{macrocode}
\begingroup\catcode61\catcode48\catcode32=10\relax%
  \catcode13=5 % ^^M
  \endlinechar=13 %
  \catcode35=6 % #
  \catcode39=12 % '
  \catcode40=12 % (
  \catcode41=12 % )
  \catcode44=12 % ,
  \catcode45=12 % -
  \catcode46=12 % .
  \catcode47=12 % /
  \catcode58=12 % :
  \catcode64=11 % @
  \catcode91=12 % [
  \catcode93=12 % ]
  \catcode123=1 % {
  \catcode125=2 % }
  \expandafter\ifx\csname ProvidesPackage\endcsname\relax
    \def\x#1#2#3[#4]{\endgroup
      \immediate\write-1{Package: #3 #4}%
      \xdef#1{#4}%
    }%
  \else
    \def\x#1#2[#3]{\endgroup
      #2[{#3}]%
      \ifx#1\@undefined
        \xdef#1{#3}%
      \fi
      \ifx#1\relax
        \xdef#1{#3}%
      \fi
    }%
  \fi
\expandafter\x\csname ver@hobsub-hyperref.sty\endcsname
\ProvidesPackage{hobsub-hyperref}%
  [2016/05/16 v1.14 Bundle oberdiek, subset hyperref (HO)]%
%    \end{macrocode}
%    \begin{macrocode}
%</hyperref>
%    \end{macrocode}
%
% \subsection{Catcodes}
%
%    \begin{macrocode}
%<*package|generic|hyperref>
\begingroup\catcode61\catcode48\catcode32=10\relax%
  \catcode13=5 % ^^M
  \endlinechar=13 %
  \catcode123=1 % {
  \catcode125=2 % }
  \catcode64=11 % @
  \def\x{\endgroup
    \expandafter\edef\csname HOBsub@AtEnd\endcsname{%
      \endlinechar=\the\endlinechar\relax
      \catcode13=\the\catcode13\relax
      \catcode32=\the\catcode32\relax
      \catcode35=\the\catcode35\relax
      \catcode61=\the\catcode61\relax
      \catcode64=\the\catcode64\relax
      \catcode123=\the\catcode123\relax
      \catcode125=\the\catcode125\relax
    }%
  }%
\x\catcode61\catcode48\catcode32=10\relax%
\catcode13=5 % ^^M
\endlinechar=13 %
\catcode35=6 % #
\catcode64=11 % @
\catcode123=1 % {
\catcode125=2 % }
\def\TMP@EnsureCode#1#2{%
  \edef\HOBsub@AtEnd{%
    \HOBsub@AtEnd
    \catcode#1=\the\catcode#1\relax
  }%
  \catcode#1=#2\relax
}
\TMP@EnsureCode{39}{12}% '
\TMP@EnsureCode{40}{12}% (
\TMP@EnsureCode{41}{12}% )
\TMP@EnsureCode{45}{12}% -
\TMP@EnsureCode{46}{12}% .
\TMP@EnsureCode{47}{12}% /
\TMP@EnsureCode{58}{12}% :
\TMP@EnsureCode{60}{12}% <
\TMP@EnsureCode{62}{12}% >
\TMP@EnsureCode{96}{12}% `
\edef\HOBsub@AtEnd{\HOBsub@AtEnd\noexpand\endinput}
%    \end{macrocode}
%
%    \begin{macrocode}
%</package|generic|hyperref>
%    \end{macrocode}
%
% \subsection{Package \xpackage{hobsub-hyperref} loads \xpackage{hobsub-generic}}
%
%    \begin{macrocode}
%<*hyperref>
\NeedsTeXFormat{LaTeX2e}
\RequirePackage{hobsub-generic}[2016/05/16]
%</hyperref>
%    \end{macrocode}
%
% \subsection{Preamble}
%
%    \begin{macrocode}
%<*preamble>
\begingroup\expandafter\expandafter\expandafter\endgroup
\expandafter\ifx\csname ver@hobsub.sty\endcsname\relax
\else
  \expandafter\hobsub@GobbleRemainingPackage
\fi
%</preamble>
%    \end{macrocode}
%
%
%    \begin{macrocode}
%<*package|preamble>
%    \end{macrocode}
%
%    \begin{macro}{\HOBsub@OrgEndinput}
%    \begin{macrocode}
\begingroup\expandafter\expandafter\expandafter\endgroup
\expandafter\ifx\csname HOBsub@OrgEndinput\endcsname\relax
  \let\HOBsub@OrgEndinput\endinput
\fi
%    \end{macrocode}
%    \end{macro}
%    \begin{macro}{\hobsub@GobbleRemainingPackage}
%    \begin{macrocode}
\def\hobsub@GobbleRemainingPackage{%
  \begingroup
  \catcode92=14 % backslash: comment
  \catcode32=14 % space: comment
  \catcode35=14 % hash: comment
  \catcode123=14 % left brace: comment
  \catcode125=14 % right brace: comment
  \catcode60=3 % less: math
  \catcode62=4 % greater: align
  \endlinechar=-1 %
  \HOBsub@GobbleRemainingPackage
}
%    \end{macrocode}
%    \end{macro}
%    \begin{macro}{\HOBsub@GobbleRemainingPackage}
%    \begin{macrocode}
\catcode60=3 % less: dollar
\catcode62=4 % greater: align
\long\def\HOBsub@GobbleRemainingPackage#1<hobsub>{%
  \endgroup
}
\catcode60=12 % less: other
\catcode62=12 % greater: other
%    \end{macrocode}
%    \end{macro}
%
%    \begin{macro}{\hobsub@StartPackage}
%    \begin{macrocode}
\def\hobsub@StartPackage#1{%
  \begingroup\expandafter\expandafter\expandafter\endgroup
  \expandafter\ifx\csname ver@#1.sty\endcsname\relax
    \let\HOBsub@OrgCurrName\@currname
    \let\HOBsub@OrgCurrExt\@currext
    \csname @pushfilename\endcsname
    \def\@currname{#1}%
    \def\@currext{sty}%
    \expandafter\def\csname\@currname.\@currext-h@@k\endcsname{}%
    \let\endinput\hobsub@GobbleRemainingPackage
    \def\hobsub@StopPackage{%
      \let\hobsub@StopPackage\relax
      \HOBsub@StopPackage
%     \hobsub@Info{hobsub}{Package `#1' loaded}%
    }%
    \hobsub@AddToFileList{#1.sty}%
  \else
    \hobsub@Info{hobsub}{Skipping package `#1' (already loaded)}%
    \let\hobsub@StopPackage\relax
    \expandafter\hobsub@GobbleRemainingPackage
  \fi
}
%    \end{macrocode}
%    \end{macro}
%
%    \begin{macro}{\hobsub@StopPackage}
%    \begin{macrocode}
\let\hobsub@StopPackage\relax
%    \end{macrocode}
%    \end{macro}
%
%    \begin{macro}{\hobsub@Info}
%    \begin{macrocode}
\def\hobsub@Info#1#2{%
  \begingroup\expandafter\expandafter\expandafter\endgroup
  \expandafter\ifx\csname @PackageInfoNoLine\endcsname\relax
    \immediate\write-1{Package #1 Info: #2.}%
  \else
    \let\hobsub@Info\@PackageInfoNoLine
    \hobsub@Info{#1}{#2}%
  \fi
}
%    \end{macrocode}
%    \end{macro}
%
%    \begin{macro}{\HOBsub@StopPackage}
%    \begin{macrocode}
\def\HOBsub@StopPackage{%
  \csname\@currname.\@currext-h@@k\endcsname
  \let\endinput\HOBsub@OrgEndinput
  \csname @popfilename\endcsname
  \let\@currname\HOBsub@OrgCurrName
  \let\@currext\HOBsub@OrgCurrExt
}
%    \end{macrocode}
%    \end{macro}
%
%    \begin{macro}{\hobsub@AddToFileList}
%    \begin{macrocode}
\begingroup\expandafter\expandafter\expandafter\endgroup
\expandafter\ifx\csname @addtofilelist\endcsname\relax
  \def\hobsub@AddToFileList#1{}%
\else
  \def\hobsub@AddToFileList#1{%
    \@addtofilelist{#1}%
  }%
\fi
%    \end{macrocode}
%    \end{macro}
%
%    \begin{macrocode}
%</package|preamble>
%    \end{macrocode}
%
% \subsection{End of preamble for subset packages}
%
%    Add the end marker for possible skipping of previous
%    macros, if package \xpackage{hobsub} was already loaded.
%    \begin{macrocode}
%<*preamble>
\hobsub@GobbleRemainingPackage
<hobsub>
%    \end{macrocode}
%
%    All macros of package \xpackage{hobsub} are now available.
%    Now we officially load the package \xpackage{hobsub} inline
%    for proper package identification.
%    \begin{macrocode}
\hobsub@StartPackage{hobsub}%
\begingroup\expandafter\expandafter\expandafter\endgroup
\expandafter\ifx\csname ProvidesPackage\endcsname\relax
  \immediate\write-1{%
    Package: hobsub 2016/05/16 v1.14 Construct package bundles (HO)%
  }%
\else
  \ProvidesPackage{hobsub}%
  [2016/05/16 v1.14 Construct package bundles (HO)]%
\fi
%</preamble>
%    \end{macrocode}
%
%
%    \begin{macrocode}
%<*post|preamble>
 \endinput
<hobsub>
\hobsub@StopPackage
%</post|preamble>
%    \end{macrocode}
%
% \subsection{Package list}
%
%    \begin{macrocode}
%<infwarerr>\hobsub@StartPackage{infwarerr}
%<ltxcmds>\hobsub@StartPackage{ltxcmds}
%<ifluatex>\hobsub@StartPackage{ifluatex}
%<ifvtex>\hobsub@StartPackage{ifvtex}
%<intcalc>\hobsub@StartPackage{intcalc}
%<ifpdf>\hobsub@StartPackage{ifpdf}
%<etexcmds>\hobsub@StartPackage{etexcmds}
%<kvsetkeys>\hobsub@StartPackage{kvsetkeys}
%<kvdefinekeys>\hobsub@StartPackage{kvdefinekeys}
%<*luatex-loader>
\ifluatex
\else
  \expandafter\hobsub@GobbleRemainingPackage
\fi
\hobsub@StartPackage{luatex-loader}
%</luatex-loader>
%<pdftexcmds>\hobsub@StartPackage{pdftexcmds}
%<pdfescape>\hobsub@StartPackage{pdfescape}
%<bigintcalc>\hobsub@StartPackage{bigintcalc}
%<bitset>\hobsub@StartPackage{bitset}
%<uniquecounter>\hobsub@StartPackage{uniquecounter}
%<letltxmacro>\hobsub@StartPackage{letltxmacro}
%<hopatch>\hobsub@StartPackage{hopatch}
%<xcolor-patch>\hobsub@StartPackage{xcolor-patch}
%<atveryend>\hobsub@StartPackage{atveryend}
%<atbegshi>\hobsub@StartPackage{atbegshi}
%<refcount>\hobsub@StartPackage{refcount}
%<hycolor>\hobsub@StartPackage{hycolor}
%    \end{macrocode}
%
% \subsection{End of packages}
%
%    \begin{macrocode}
%<package|atend>\HOBsub@AtEnd%
%    \end{macrocode}
%
% \section{Test}
%
% \subsection{Catcode checks for loading}
%
%    \begin{macrocode}
%<*test1>
%    \end{macrocode}
%    \begin{macrocode}
\catcode`\{=1 %
\catcode`\}=2 %
\catcode`\#=6 %
\catcode`\@=11 %
\expandafter\ifx\csname count@\endcsname\relax
  \countdef\count@=255 %
\fi
\expandafter\ifx\csname @gobble\endcsname\relax
  \long\def\@gobble#1{}%
\fi
\expandafter\ifx\csname @firstofone\endcsname\relax
  \long\def\@firstofone#1{#1}%
\fi
\expandafter\ifx\csname loop\endcsname\relax
  \expandafter\@firstofone
\else
  \expandafter\@gobble
\fi
{%
  \def\loop#1\repeat{%
    \def\body{#1}%
    \iterate
  }%
  \def\iterate{%
    \body
      \let\next\iterate
    \else
      \let\next\relax
    \fi
    \next
  }%
  \let\repeat=\fi
}%
\def\RestoreCatcodes{}
\count@=0 %
\loop
  \edef\RestoreCatcodes{%
    \RestoreCatcodes
    \catcode\the\count@=\the\catcode\count@\relax
  }%
\ifnum\count@<255 %
  \advance\count@ 1 %
\repeat

\def\RangeCatcodeInvalid#1#2{%
  \count@=#1\relax
  \loop
    \catcode\count@=15 %
  \ifnum\count@<#2\relax
    \advance\count@ 1 %
  \repeat
}
\def\RangeCatcodeCheck#1#2#3{%
  \count@=#1\relax
  \loop
    \ifnum#3=\catcode\count@
    \else
      \errmessage{%
        Character \the\count@\space
        with wrong catcode \the\catcode\count@\space
        instead of \number#3%
      }%
    \fi
  \ifnum\count@<#2\relax
    \advance\count@ 1 %
  \repeat
}
\def\space{ }
\expandafter\ifx\csname LoadCommand\endcsname\relax
  \def\LoadCommand{\input hobsub.sty\relax}%
\fi
\def\Test{%
  \RangeCatcodeInvalid{0}{47}%
  \RangeCatcodeInvalid{58}{64}%
  \RangeCatcodeInvalid{91}{96}%
  \RangeCatcodeInvalid{123}{255}%
  \catcode`\@=12 %
  \catcode`\\=0 %
  \catcode`\%=14 %
  \LoadCommand
  \RangeCatcodeCheck{0}{36}{15}%
  \RangeCatcodeCheck{37}{37}{14}%
  \RangeCatcodeCheck{38}{47}{15}%
  \RangeCatcodeCheck{48}{57}{12}%
  \RangeCatcodeCheck{58}{63}{15}%
  \RangeCatcodeCheck{64}{64}{12}%
  \RangeCatcodeCheck{65}{90}{11}%
  \RangeCatcodeCheck{91}{91}{15}%
  \RangeCatcodeCheck{92}{92}{0}%
  \RangeCatcodeCheck{93}{96}{15}%
  \RangeCatcodeCheck{97}{122}{11}%
  \RangeCatcodeCheck{123}{255}{15}%
  \RestoreCatcodes
}
\Test
\csname @@end\endcsname
\end
%    \end{macrocode}
%    \begin{macrocode}
%</test1>
%    \end{macrocode}
%
% \section{Installation}
%
% \subsection{Download}
%
% \paragraph{Package.} This package is available on
% CTAN\footnote{\url{http://ctan.org/pkg/hobsub}}:
% \begin{description}
% \item[\CTAN{macros/latex/contrib/oberdiek/hobsub.dtx}] The source file.
% \item[\CTAN{macros/latex/contrib/oberdiek/hobsub.pdf}] Documentation.
% \end{description}
%
%
% \paragraph{Bundle.} All the packages of the bundle `oberdiek'
% are also available in a TDS compliant ZIP archive. There
% the packages are already unpacked and the documentation files
% are generated. The files and directories obey the TDS standard.
% \begin{description}
% \item[\CTAN{install/macros/latex/contrib/oberdiek.tds.zip}]
% \end{description}
% \emph{TDS} refers to the standard ``A Directory Structure
% for \TeX\ Files'' (\CTAN{tds/tds.pdf}). Directories
% with \xfile{texmf} in their name are usually organized this way.
%
% \subsection{Bundle installation}
%
% \paragraph{Unpacking.} Unpack the \xfile{oberdiek.tds.zip} in the
% TDS tree (also known as \xfile{texmf} tree) of your choice.
% Example (linux):
% \begin{quote}
%   |unzip oberdiek.tds.zip -d ~/texmf|
% \end{quote}
%
% \paragraph{Script installation.}
% Check the directory \xfile{TDS:scripts/oberdiek/} for
% scripts that need further installation steps.
% Package \xpackage{attachfile2} comes with the Perl script
% \xfile{pdfatfi.pl} that should be installed in such a way
% that it can be called as \texttt{pdfatfi}.
% Example (linux):
% \begin{quote}
%   |chmod +x scripts/oberdiek/pdfatfi.pl|\\
%   |cp scripts/oberdiek/pdfatfi.pl /usr/local/bin/|
% \end{quote}
%
% \subsection{Package installation}
%
% \paragraph{Unpacking.} The \xfile{.dtx} file is a self-extracting
% \docstrip\ archive. The files are extracted by running the
% \xfile{.dtx} through \plainTeX:
% \begin{quote}
%   \verb|tex hobsub.dtx|
% \end{quote}
%
% \paragraph{TDS.} Now the different files must be moved into
% the different directories in your installation TDS tree
% (also known as \xfile{texmf} tree):
% \begin{quote}
% \def\t{^^A
% \begin{tabular}{@{}>{\ttfamily}l@{ $\rightarrow$ }>{\ttfamily}l@{}}
%   hobsub.sty & tex/generic/oberdiek/hobsub.sty\\
%   hobsub-generic.sty & tex/generic/oberdiek/hobsub-generic.sty\\
%   hobsub-hyperref.sty & tex/generic/oberdiek/hobsub-hyperref.sty\\
%   hobsub.pdf & doc/latex/oberdiek/hobsub.pdf\\
%   test/hobsub-test1.tex & doc/latex/oberdiek/test/hobsub-test1.tex\\
%   hobsub.dtx & source/latex/oberdiek/hobsub.dtx\\
% \end{tabular}^^A
% }^^A
% \sbox0{\t}^^A
% \ifdim\wd0>\linewidth
%   \begingroup
%     \advance\linewidth by\leftmargin
%     \advance\linewidth by\rightmargin
%   \edef\x{\endgroup
%     \def\noexpand\lw{\the\linewidth}^^A
%   }\x
%   \def\lwbox{^^A
%     \leavevmode
%     \hbox to \linewidth{^^A
%       \kern-\leftmargin\relax
%       \hss
%       \usebox0
%       \hss
%       \kern-\rightmargin\relax
%     }^^A
%   }^^A
%   \ifdim\wd0>\lw
%     \sbox0{\small\t}^^A
%     \ifdim\wd0>\linewidth
%       \ifdim\wd0>\lw
%         \sbox0{\footnotesize\t}^^A
%         \ifdim\wd0>\linewidth
%           \ifdim\wd0>\lw
%             \sbox0{\scriptsize\t}^^A
%             \ifdim\wd0>\linewidth
%               \ifdim\wd0>\lw
%                 \sbox0{\tiny\t}^^A
%                 \ifdim\wd0>\linewidth
%                   \lwbox
%                 \else
%                   \usebox0
%                 \fi
%               \else
%                 \lwbox
%               \fi
%             \else
%               \usebox0
%             \fi
%           \else
%             \lwbox
%           \fi
%         \else
%           \usebox0
%         \fi
%       \else
%         \lwbox
%       \fi
%     \else
%       \usebox0
%     \fi
%   \else
%     \lwbox
%   \fi
% \else
%   \usebox0
% \fi
% \end{quote}
% If you have a \xfile{docstrip.cfg} that configures and enables \docstrip's
% TDS installing feature, then some files can already be in the right
% place, see the documentation of \docstrip.
%
% \subsection{Refresh file name databases}
%
% If your \TeX~distribution
% (\teTeX, \mikTeX, \dots) relies on file name databases, you must refresh
% these. For example, \teTeX\ users run \verb|texhash| or
% \verb|mktexlsr|.
%
% \subsection{Some details for the interested}
%
% \paragraph{Attached source.}
%
% The PDF documentation on CTAN also includes the
% \xfile{.dtx} source file. It can be extracted by
% AcrobatReader 6 or higher. Another option is \textsf{pdftk},
% e.g. unpack the file into the current directory:
% \begin{quote}
%   \verb|pdftk hobsub.pdf unpack_files output .|
% \end{quote}
%
% \paragraph{Unpacking with \LaTeX.}
% The \xfile{.dtx} chooses its action depending on the format:
% \begin{description}
% \item[\plainTeX:] Run \docstrip\ and extract the files.
% \item[\LaTeX:] Generate the documentation.
% \end{description}
% If you insist on using \LaTeX\ for \docstrip\ (really,
% \docstrip\ does not need \LaTeX), then inform the autodetect routine
% about your intention:
% \begin{quote}
%   \verb|latex \let\install=y\input{hobsub.dtx}|
% \end{quote}
% Do not forget to quote the argument according to the demands
% of your shell.
%
% \paragraph{Generating the documentation.}
% You can use both the \xfile{.dtx} or the \xfile{.drv} to generate
% the documentation. The process can be configured by the
% configuration file \xfile{ltxdoc.cfg}. For instance, put this
% line into this file, if you want to have A4 as paper format:
% \begin{quote}
%   \verb|\PassOptionsToClass{a4paper}{article}|
% \end{quote}
% An example follows how to generate the
% documentation with pdf\LaTeX:
% \begin{quote}
%\begin{verbatim}
%pdflatex hobsub.dtx
%makeindex -s gind.ist hobsub.idx
%pdflatex hobsub.dtx
%makeindex -s gind.ist hobsub.idx
%pdflatex hobsub.dtx
%\end{verbatim}
% \end{quote}
%
% \section{Catalogue}
%
% The following XML file can be used as source for the
% \href{http://mirror.ctan.org/help/Catalogue/catalogue.html}{\TeX\ Catalogue}.
% The elements \texttt{caption} and \texttt{description} are imported
% from the original XML file from the Catalogue.
% The name of the XML file in the Catalogue is \xfile{hobsub.xml}.
%    \begin{macrocode}
%<*catalogue>
<?xml version='1.0' encoding='us-ascii'?>
<!DOCTYPE entry SYSTEM 'catalogue.dtd'>
<entry datestamp='$Date$' modifier='$Author$' id='hobsub'>
  <name>hobsub</name>
  <caption>Construct package bundles</caption>
  <authorref id='auth:oberdiek'/>
  <copyright owner='Heiko Oberdiek' year='2011,2012'/>
  <license type='lppl1.3'/>
  <version number='1.14'/>
  <description>
    The package offers a means of loading a bundle of (other) packages
    &#x201C;at once&#x201D;, offering a means of reducing the amount of log output
    as the packages are loaded, and an aide-memoire for complicated
    requirements.  In some ways, loading a package via hobsub behaves
    much like loading the package in the normal way; for example,
    reloads are suppressed, as in LaTeX proper, and correct
    <tt>\listfiles</tt> output is produced.
    <p/>
    Examples provided are a &#x2018;generic&#x2019; bundle (comprising the author&#x2019;s
    packages that can be used with Plain TeX) and a &#x2018;hyperref&#x2019; bundle
    (comprising packages useful when generating hypertext output with
    LaTeX).
    <p/>
    The package is part of the <xref refid='oberdiek'>oberdiek</xref> bundle.
  </description>
  <documentation details='Package documentation'
      href='ctan:/macros/latex/contrib/oberdiek/hobsub.pdf'/>
  <ctan file='true' path='/macros/latex/contrib/oberdiek/hobsub.dtx'/>
  <miktex location='oberdiek'/>
  <texlive location='oberdiek'/>
  <install path='/macros/latex/contrib/oberdiek/oberdiek.tds.zip'/>
</entry>
%</catalogue>
%    \end{macrocode}
%
% \begin{thebibliography}{9}
% \raggedright
% \bibitem{lualatex:quiet}
% Will Robertson,
% \textit{[lltx] quietening lualatex console output};
% mailing list \textsf{lualatex-dev at tug.org},
% 2011-01-29;
% \url{http://tug.org/pipermail/lualatex-dev/2011-January/001031.html}.
%
% \end{thebibliography}
%
% \begin{History}
%   \begin{Version}{2011/01/30 v1.0}
%   \item
%     First public version.
%   \end{Version}
%   \begin{Version}{2011/04/16 v1.1}
%   \item
%     Package updates.
%   \end{Version}
%   \begin{Version}{2011/04/17 v1.2}
%   \item
%     White spaces at line begins are removed or reduced
%     in the generated collection packages.
%   \end{Version}
%   \begin{Version}{2011/04/18 v1.3}
%   \item
%     Package updates.
%   \end{Version}
%   \begin{Version}{2011/04/23 v1.4}
%   \item
%     Package updates.
%   \end{Version}
%   \begin{Version}{2011/06/24 v1.5}
%   \item
%     Package updates.
%   \end{Version}
%   \begin{Version}{2011/06/30 v1.6}
%   \item
%     Package updates.
%   \end{Version}
%   \begin{Version}{2011/07/01 v1.7}
%   \item
%     Package updates.
%   \end{Version}
%   \begin{Version}{2011/07/28 v1.8}
%   \item
%     Package updates.
%   \end{Version}
%   \begin{Version}{2011/08/22 v1.9}
%   \item
%     Package updates.
%   \end{Version}
%   \begin{Version}{2011/10/16 v1.10}
%   \item
%     Package updates.
%   \end{Version}
%   \begin{Version}{2011/11/29 v1.11}
%   \item
%     Package updates.
%   \end{Version}
%   \begin{Version}{2012/04/25 v1.12}
%   \item
%     Package updates.
%   \end{Version}
%   \begin{Version}{2012/05/28 v1.13}
%   \item
%     Package updates.
%   \end{Version}
%   \begin{Version}{2016/05/16 v1.14}
%   \item
%     Documentation updates.
%   \end{Version}
% \end{History}
%
% \PrintIndex
%
% \Finale
\endinput
|
% \end{quote}
% Do not forget to quote the argument according to the demands
% of your shell.
%
% \paragraph{Generating the documentation.}
% You can use both the \xfile{.dtx} or the \xfile{.drv} to generate
% the documentation. The process can be configured by the
% configuration file \xfile{ltxdoc.cfg}. For instance, put this
% line into this file, if you want to have A4 as paper format:
% \begin{quote}
%   \verb|\PassOptionsToClass{a4paper}{article}|
% \end{quote}
% An example follows how to generate the
% documentation with pdf\LaTeX:
% \begin{quote}
%\begin{verbatim}
%pdflatex hobsub.dtx
%makeindex -s gind.ist hobsub.idx
%pdflatex hobsub.dtx
%makeindex -s gind.ist hobsub.idx
%pdflatex hobsub.dtx
%\end{verbatim}
% \end{quote}
%
% \section{Catalogue}
%
% The following XML file can be used as source for the
% \href{http://mirror.ctan.org/help/Catalogue/catalogue.html}{\TeX\ Catalogue}.
% The elements \texttt{caption} and \texttt{description} are imported
% from the original XML file from the Catalogue.
% The name of the XML file in the Catalogue is \xfile{hobsub.xml}.
%    \begin{macrocode}
%<*catalogue>
<?xml version='1.0' encoding='us-ascii'?>
<!DOCTYPE entry SYSTEM 'catalogue.dtd'>
<entry datestamp='$Date$' modifier='$Author$' id='hobsub'>
  <name>hobsub</name>
  <caption>Construct package bundles</caption>
  <authorref id='auth:oberdiek'/>
  <copyright owner='Heiko Oberdiek' year='2011,2012'/>
  <license type='lppl1.3'/>
  <version number='1.14'/>
  <description>
    The package offers a means of loading a bundle of (other) packages
    &#x201C;at once&#x201D;, offering a means of reducing the amount of log output
    as the packages are loaded, and an aide-memoire for complicated
    requirements.  In some ways, loading a package via hobsub behaves
    much like loading the package in the normal way; for example,
    reloads are suppressed, as in LaTeX proper, and correct
    <tt>\listfiles</tt> output is produced.
    <p/>
    Examples provided are a &#x2018;generic&#x2019; bundle (comprising the author&#x2019;s
    packages that can be used with Plain TeX) and a &#x2018;hyperref&#x2019; bundle
    (comprising packages useful when generating hypertext output with
    LaTeX).
    <p/>
    The package is part of the <xref refid='oberdiek'>oberdiek</xref> bundle.
  </description>
  <documentation details='Package documentation'
      href='ctan:/macros/latex/contrib/oberdiek/hobsub.pdf'/>
  <ctan file='true' path='/macros/latex/contrib/oberdiek/hobsub.dtx'/>
  <miktex location='oberdiek'/>
  <texlive location='oberdiek'/>
  <install path='/macros/latex/contrib/oberdiek/oberdiek.tds.zip'/>
</entry>
%</catalogue>
%    \end{macrocode}
%
% \begin{thebibliography}{9}
% \raggedright
% \bibitem{lualatex:quiet}
% Will Robertson,
% \textit{[lltx] quietening lualatex console output};
% mailing list \textsf{lualatex-dev at tug.org},
% 2011-01-29;
% \url{http://tug.org/pipermail/lualatex-dev/2011-January/001031.html}.
%
% \end{thebibliography}
%
% \begin{History}
%   \begin{Version}{2011/01/30 v1.0}
%   \item
%     First public version.
%   \end{Version}
%   \begin{Version}{2011/04/16 v1.1}
%   \item
%     Package updates.
%   \end{Version}
%   \begin{Version}{2011/04/17 v1.2}
%   \item
%     White spaces at line begins are removed or reduced
%     in the generated collection packages.
%   \end{Version}
%   \begin{Version}{2011/04/18 v1.3}
%   \item
%     Package updates.
%   \end{Version}
%   \begin{Version}{2011/04/23 v1.4}
%   \item
%     Package updates.
%   \end{Version}
%   \begin{Version}{2011/06/24 v1.5}
%   \item
%     Package updates.
%   \end{Version}
%   \begin{Version}{2011/06/30 v1.6}
%   \item
%     Package updates.
%   \end{Version}
%   \begin{Version}{2011/07/01 v1.7}
%   \item
%     Package updates.
%   \end{Version}
%   \begin{Version}{2011/07/28 v1.8}
%   \item
%     Package updates.
%   \end{Version}
%   \begin{Version}{2011/08/22 v1.9}
%   \item
%     Package updates.
%   \end{Version}
%   \begin{Version}{2011/10/16 v1.10}
%   \item
%     Package updates.
%   \end{Version}
%   \begin{Version}{2011/11/29 v1.11}
%   \item
%     Package updates.
%   \end{Version}
%   \begin{Version}{2012/04/25 v1.12}
%   \item
%     Package updates.
%   \end{Version}
%   \begin{Version}{2012/05/28 v1.13}
%   \item
%     Package updates.
%   \end{Version}
%   \begin{Version}{2016/05/16 v1.14}
%   \item
%     Documentation updates.
%   \end{Version}
% \end{History}
%
% \PrintIndex
%
% \Finale
\endinput
|
% \end{quote}
% Do not forget to quote the argument according to the demands
% of your shell.
%
% \paragraph{Generating the documentation.}
% You can use both the \xfile{.dtx} or the \xfile{.drv} to generate
% the documentation. The process can be configured by the
% configuration file \xfile{ltxdoc.cfg}. For instance, put this
% line into this file, if you want to have A4 as paper format:
% \begin{quote}
%   \verb|\PassOptionsToClass{a4paper}{article}|
% \end{quote}
% An example follows how to generate the
% documentation with pdf\LaTeX:
% \begin{quote}
%\begin{verbatim}
%pdflatex hobsub.dtx
%makeindex -s gind.ist hobsub.idx
%pdflatex hobsub.dtx
%makeindex -s gind.ist hobsub.idx
%pdflatex hobsub.dtx
%\end{verbatim}
% \end{quote}
%
% \section{Catalogue}
%
% The following XML file can be used as source for the
% \href{http://mirror.ctan.org/help/Catalogue/catalogue.html}{\TeX\ Catalogue}.
% The elements \texttt{caption} and \texttt{description} are imported
% from the original XML file from the Catalogue.
% The name of the XML file in the Catalogue is \xfile{hobsub.xml}.
%    \begin{macrocode}
%<*catalogue>
<?xml version='1.0' encoding='us-ascii'?>
<!DOCTYPE entry SYSTEM 'catalogue.dtd'>
<entry datestamp='$Date$' modifier='$Author$' id='hobsub'>
  <name>hobsub</name>
  <caption>Construct package bundles</caption>
  <authorref id='auth:oberdiek'/>
  <copyright owner='Heiko Oberdiek' year='2011,2012'/>
  <license type='lppl1.3'/>
  <version number='1.14'/>
  <description>
    The package offers a means of loading a bundle of (other) packages
    &#x201C;at once&#x201D;, offering a means of reducing the amount of log output
    as the packages are loaded, and an aide-memoire for complicated
    requirements.  In some ways, loading a package via hobsub behaves
    much like loading the package in the normal way; for example,
    reloads are suppressed, as in LaTeX proper, and correct
    <tt>\listfiles</tt> output is produced.
    <p/>
    Examples provided are a &#x2018;generic&#x2019; bundle (comprising the author&#x2019;s
    packages that can be used with Plain TeX) and a &#x2018;hyperref&#x2019; bundle
    (comprising packages useful when generating hypertext output with
    LaTeX).
    <p/>
    The package is part of the <xref refid='oberdiek'>oberdiek</xref> bundle.
  </description>
  <documentation details='Package documentation'
      href='ctan:/macros/latex/contrib/oberdiek/hobsub.pdf'/>
  <ctan file='true' path='/macros/latex/contrib/oberdiek/hobsub.dtx'/>
  <miktex location='oberdiek'/>
  <texlive location='oberdiek'/>
  <install path='/macros/latex/contrib/oberdiek/oberdiek.tds.zip'/>
</entry>
%</catalogue>
%    \end{macrocode}
%
% \begin{thebibliography}{9}
% \raggedright
% \bibitem{lualatex:quiet}
% Will Robertson,
% \textit{[lltx] quietening lualatex console output};
% mailing list \textsf{lualatex-dev at tug.org},
% 2011-01-29;
% \url{http://tug.org/pipermail/lualatex-dev/2011-January/001031.html}.
%
% \end{thebibliography}
%
% \begin{History}
%   \begin{Version}{2011/01/30 v1.0}
%   \item
%     First public version.
%   \end{Version}
%   \begin{Version}{2011/04/16 v1.1}
%   \item
%     Package updates.
%   \end{Version}
%   \begin{Version}{2011/04/17 v1.2}
%   \item
%     White spaces at line begins are removed or reduced
%     in the generated collection packages.
%   \end{Version}
%   \begin{Version}{2011/04/18 v1.3}
%   \item
%     Package updates.
%   \end{Version}
%   \begin{Version}{2011/04/23 v1.4}
%   \item
%     Package updates.
%   \end{Version}
%   \begin{Version}{2011/06/24 v1.5}
%   \item
%     Package updates.
%   \end{Version}
%   \begin{Version}{2011/06/30 v1.6}
%   \item
%     Package updates.
%   \end{Version}
%   \begin{Version}{2011/07/01 v1.7}
%   \item
%     Package updates.
%   \end{Version}
%   \begin{Version}{2011/07/28 v1.8}
%   \item
%     Package updates.
%   \end{Version}
%   \begin{Version}{2011/08/22 v1.9}
%   \item
%     Package updates.
%   \end{Version}
%   \begin{Version}{2011/10/16 v1.10}
%   \item
%     Package updates.
%   \end{Version}
%   \begin{Version}{2011/11/29 v1.11}
%   \item
%     Package updates.
%   \end{Version}
%   \begin{Version}{2012/04/25 v1.12}
%   \item
%     Package updates.
%   \end{Version}
%   \begin{Version}{2012/05/28 v1.13}
%   \item
%     Package updates.
%   \end{Version}
%   \begin{Version}{2016/05/16 v1.14}
%   \item
%     Documentation updates.
%   \end{Version}
% \end{History}
%
% \PrintIndex
%
% \Finale
\endinput

%        (quote the arguments according to the demands of your shell)
%
% Documentation:
%    (a) If hobsub.drv is present:
%           latex hobsub.drv
%    (b) Without hobsub.drv:
%           latex hobsub.dtx; ...
%    The class ltxdoc loads the configuration file ltxdoc.cfg
%    if available. Here you can specify further options, e.g.
%    use A4 as paper format:
%       \PassOptionsToClass{a4paper}{article}
%
%    Programm calls to get the documentation (example):
%       pdflatex hobsub.dtx
%       makeindex -s gind.ist hobsub.idx
%       pdflatex hobsub.dtx
%       makeindex -s gind.ist hobsub.idx
%       pdflatex hobsub.dtx
%
% Installation:
%    TDS:tex/generic/oberdiek/hobsub.sty
%    TDS:tex/generic/oberdiek/hobsub-generic.sty
%    TDS:tex/generic/oberdiek/hobsub-hyperref.sty
%    TDS:doc/latex/oberdiek/hobsub.pdf
%    TDS:doc/latex/oberdiek/test/hobsub-test1.tex
%    TDS:source/latex/oberdiek/hobsub.dtx
%
%<*ignore>
\begingroup
  \catcode123=1 %
  \catcode125=2 %
  \def\x{LaTeX2e}%
\expandafter\endgroup
\ifcase 0\ifx\install y1\fi\expandafter
         \ifx\csname processbatchFile\endcsname\relax\else1\fi
         \ifx\fmtname\x\else 1\fi\relax
\else\csname fi\endcsname
%</ignore>
%<*install>
\input docstrip.tex
\Msg{************************************************************************}
\Msg{* Installation}
\Msg{* Package: hobsub 2016/05/16 v1.14 Construct package bundles (HO)}
\Msg{************************************************************************}

\keepsilent
\askforoverwritefalse

\let\MetaPrefix\relax
\preamble

This is a generated file.

Project: hobsub
Version: 2016/05/16 v1.14

Copyright (C) 2011, 2012 by
   Heiko Oberdiek <heiko.oberdiek at googlemail.com>

This work may be distributed and/or modified under the
conditions of the LaTeX Project Public License, either
version 1.3c of this license or (at your option) any later
version. This version of this license is in
   http://www.latex-project.org/lppl/lppl-1-3c.txt
and the latest version of this license is in
   http://www.latex-project.org/lppl.txt
and version 1.3 or later is part of all distributions of
LaTeX version 2005/12/01 or later.

This work has the LPPL maintenance status "maintained".

This Current Maintainer of this work is Heiko Oberdiek.

The Base Interpreter refers to any `TeX-Format',
because some files are installed in TDS:tex/generic//.

This work consists of the main source file hobsub.dtx
and the derived files
   hobsub.sty, hobsub.pdf, hobsub.ins, hobsub.drv, hobsub-generic.sty,
   hobsub-hyperref.sty, hobsub-test1.tex.

\endpreamble
\let\MetaPrefix\DoubleperCent

\generate{%
  \file{hobsub.ins}{\from{hobsub.dtx}{install}}%
  \file{hobsub.drv}{\from{hobsub.dtx}{driver}}%
  \usedir{tex/generic/oberdiek}%
  \file{hobsub.sty}{\from{hobsub.dtx}{package}}%
%  \usedir{doc/latex/oberdiek/test}%
%  \file{hobsub-test1.tex}{\from{hobsub.dtx}{test1}}%
  \nopreamble
  \nopostamble
%  \usedir{source/latex/oberdiek/catalogue}%
%  \file{hobsub.xml}{\from{hobsub.dtx}{catalogue}}%
}
\let\hobsubSavedNormalLine\normalLine
\def\hobsubGobble#1{}
\begingroup
  \lccode`\0=32 % space
  \lccode`\1=35 % hash
  \lccode`\2=37 % percent
  \lccode`\3=92 % backslash
  \lccode`\4=123 % {
  \lccode`\5=125 % }
\lowercase{\endgroup
  \let\hobsubTokenSpaceOther= 0%
  \let\hobsubTokenHash= 1%
  \let\hobsubTokenPercent= 2%
  \let\hobsubTokenBackslash= 3%
  \let\hobsubTokenBraceLeft= 4%
  \let\hobsubTokenBraceRight= 5%
  \def\hobsubNormalLineWithSpace{%
    \hobsubSavedNormalLine0%
  }%
}
\def\normalLine{%
  \futurelet\hobsubToken\hobsubFirst
}
\def\hobsubFirst{%
  \ifx\hobsubToken\hobsubTokenSpaceOther
    \expandafter\expandafter\expandafter\futurelet
    \expandafter\expandafter\expandafter\hobsubToken
    \expandafter\expandafter\expandafter\hobsubNext
    \expandafter\hobsubGobble
  \else
    \expandafter\hobsubSavedNormalLine
  \fi
}
\def\hobsubNext{%
  \ifx\hobsubToken\hobsubTokenSpaceOther
    \expandafter\expandafter\expandafter\futurelet
    \expandafter\expandafter\expandafter\hobsubToken
    \expandafter\expandafter\expandafter\hobsubNext 
    \expandafter\hobsubGobble
  \else
    \let\hobsubWithSpace=Y%
    \ifx\hobsubToken\hobsubTokenBackslash
      \let\hobsubWithSpace=N%
    \else
      \ifx\hobsubToken\hobsubTokenBraceRight
        \let\hobsubWithSpace=N%
      \else
        \ifx\hobsubToken\hobsubTokenBraceLeft
          \let\hobsubWithSpace=N%
        \else
          \ifx\hobsubToken\hobsubTokenPercent
            \let\hobsubWithSpace=N%
          \else
            \ifx\hobsubToken\hobsubTokenHash
              \let\hobsubWithSpace=N%
            \fi
          \fi
        \fi
      \fi
    \fi
    \ifx\hobsubWithSpace Y%
      \expandafter\expandafter\expandafter\hobsubNormalLineWithSpace
    \else
      \expandafter\expandafter\expandafter\hobsubSavedNormalLine
    \fi
  \fi
}
\generate{%
  \usedir{tex/generic/oberdiek}%
  \file{hobsub-generic.sty}{%
    \from{hobsub.dtx}{generic,preamble,infwarerr}%
    \from{infwarerr.dtx}{package}%
    \from{hobsub.dtx}{post,ltxcmds}%
    \from{ltxcmds.dtx}{package}%
    \from{hobsub.dtx}{post,ifluatex}%
    \from{ifluatex.dtx}{package}%
    \from{hobsub.dtx}{post,ifvtex}%
    \from{ifvtex.dtx}{package}%
    \from{hobsub.dtx}{post,intcalc}%
    \from{intcalc.dtx}{package}%
    \from{hobsub.dtx}{post,ifpdf}%
    \from{ifpdf.dtx}{package}%
    \from{hobsub.dtx}{post,etexcmds}%
    \from{etexcmds.dtx}{package}%
    \from{hobsub.dtx}{post,kvsetkeys}%
    \from{kvsetkeys.dtx}{package}%
    \from{hobsub.dtx}{post,kvdefinekeys}%
    \from{kvdefinekeys.dtx}{package}%
    \from{hobsub.dtx}{post,luatex-loader}%
    \from{luatex.dtx}{loader}%
    \from{hobsub.dtx}{post,pdftexcmds}%
    \from{pdftexcmds.dtx}{package}%
    \from{hobsub.dtx}{post,pdfescape}%
    \from{pdfescape.dtx}{package}%
    \from{hobsub.dtx}{post,bigintcalc}%
    \from{bigintcalc.dtx}{package}%
    \from{hobsub.dtx}{post,bitset}%
    \from{bitset.dtx}{package}%
    \from{hobsub.dtx}{post,uniquecounter}%
    \from{uniquecounter.dtx}{package}%
    \from{hobsub.dtx}{post,atend}%
  }%
}
\generate{%
  \usedir{tex/generic/oberdiek}%
  \file{hobsub-hyperref.sty}{%
    \from{hobsub.dtx}{hyperref,preamble,,letltxmacro}%
    \from{letltxmacro.dtx}{package}%
    \from{hobsub.dtx}{post,hopatch}%
    \from{hopatch.dtx}{package}%
    \from{hobsub.dtx}{post,xcolor-patch}%
    \from{hycolor.dtx}{xcolor}%
    \from{hobsub.dtx}{post,atveryend}%
    \from{atveryend.dtx}{package}%
    \from{hobsub.dtx}{post,atbegshi}%
    \from{atbegshi.dtx}{package}%
    \from{hobsub.dtx}{post,refcount}%
    \from{refcount.dtx}{package}%
    \from{hobsub.dtx}{post,hycolor}%
    \from{hycolor.dtx}{package}%
    \from{hobsub.dtx}{post}%
  }%
}
\let\normalLine\hobsubSavedNormalLine

\catcode32=13\relax% active space
\let =\space%
\Msg{************************************************************************}
\Msg{*}
\Msg{* To finish the installation you have to move the following}
\Msg{* files into a directory searched by TeX:}
\Msg{*}
\Msg{*     hobsub.sty, hobsub-generic.sty, hobsub-hyperref.sty}
\Msg{*}
\Msg{* To produce the documentation run the file `hobsub.drv'}
\Msg{* through LaTeX.}
\Msg{*}
\Msg{* Happy TeXing!}
\Msg{*}
\Msg{************************************************************************}

\endbatchfile
%</install>
%<*ignore>
\fi
%</ignore>
%<*driver>
\NeedsTeXFormat{LaTeX2e}
\ProvidesFile{hobsub.drv}%
  [2016/05/16 v1.14 Construct package bundles (HO)]%
\documentclass{ltxdoc}
\usepackage{holtxdoc}[2011/11/22]
\usepackage{array}
\usepackage{caption}
\renewcommand*{\bottomfraction}{.5}
\renewcommand*{\topfraction}{.5}
\begin{document}
  \DocInput{hobsub.dtx}%
\end{document}
%</driver>
% \fi
%
%
% \CharacterTable
%  {Upper-case    \A\B\C\D\E\F\G\H\I\J\K\L\M\N\O\P\Q\R\S\T\U\V\W\X\Y\Z
%   Lower-case    \a\b\c\d\e\f\g\h\i\j\k\l\m\n\o\p\q\r\s\t\u\v\w\x\y\z
%   Digits        \0\1\2\3\4\5\6\7\8\9
%   Exclamation   \!     Double quote  \"     Hash (number) \#
%   Dollar        \$     Percent       \%     Ampersand     \&
%   Acute accent  \'     Left paren    \(     Right paren   \)
%   Asterisk      \*     Plus          \+     Comma         \,
%   Minus         \-     Point         \.     Solidus       \/
%   Colon         \:     Semicolon     \;     Less than     \<
%   Equals        \=     Greater than  \>     Question mark \?
%   Commercial at \@     Left bracket  \[     Backslash     \\
%   Right bracket \]     Circumflex    \^     Underscore    \_
%   Grave accent  \`     Left brace    \{     Vertical bar  \|
%   Right brace   \}     Tilde         \~}
%
% \GetFileInfo{hobsub.drv}
%
% \title{The \xpackage{hobsub} package}
% \date{2016/05/16 v1.14}
% \author{Heiko Oberdiek\thanks
% {Please report any issues at https://github.com/ho-tex/oberdiek/issues}\\
% \xemail{heiko.oberdiek at googlemail.com}}
%
% \maketitle
%
% \begin{abstract}
% Package \xpackage{hobsub} implements the idea of loading
% several packages in one big collection package.
% \end{abstract}
%
% \tableofcontents
%
% \def\TableGeneric{^^A
%   \TableEntry{\xpackage{infwarerr}}{}{2010/04/08 v1.3}^^A
%   \TableEntry{\xpackage{ltxcmds}}{}{2011/11/09 v1.22}^^A
%   \TableEntry{\xpackage{ifluatex}}{}{2010/03/01 v1.3}^^A
%   \TableEntry{\xpackage{ifvtex}}{}{2010/03/01 v1.5}^^A
%   \TableEntry{\xpackage{intcalc}}{}{2007/09/27 v1.1}^^A
%   \TableEntry{\xpackage{ifpdf}}{\xpackage{ifluatex}}{2011/01/30 v2.3}^^A
%   \TableEntry{\xpackage{etexcmds}}{}{2011/02/16 v1.5}^^A
%   \TableEntry{\xpackage{kvsetkeys}}{}{2012/04/25 v1.16}^^A
%   \TableEntry{\xpackage{kvdefinekeys}}{}{2011/04/07 v1.3}^^A
%   \TableEntry{\xpackage{luatex-loader}}{}{2010/03/09 v0.4}^^A
%   \TableEntry{\xpackage{pdftexcmds}}{\xpackage{ifluatex}, \xpackage{ifpdf}, \xpackage{infwarerr}, \xpackage{ltxcmds}, \xpackage{luatex-loader}}{2011/11/29 v0.20}^^A
%   \TableEntry{\xpackage{pdfescape}}{\xpackage{ltxcmds}, \xpackage{pdftexcmds}}{2011/11/25 v1.13}^^A
%   \TableEntry{\xpackage{bigintcalc}}{\xpackage{pdftexcmds}}{2012/04/08 v1.3}^^A
%   \TableEntry{\xpackage{bitset}}{\xpackage{bigintcalc}, \xpackage{infwarerr}, \xpackage{intcalc}}{2011/01/30 v1.1}^^A
%   \TableEntry{\xpackage{uniquecounter}}{\xpackage{bigintcalc}, \xpackage{infwarerr}}{2011/01/30 v1.2}^^A
% }
% \def\TableHyperref{^^A
%   \TableEntry{\xpackage{letltxmacro}}{}{2010/09/02 v1.4}^^A
%   \TableEntry{\xpackage{hopatch}}{}{2016/05/16 v1.2}^^A
%   \TableEntry{\xpackage{xcolor-patch}}{}{2011/01/30 v1.7}^^A
%   \TableEntry{\xpackage{atveryend}}{}{2011/06/30 v1.8}^^A
%   \TableEntry{\xpackage{atbegshi}}{\xpackage{ifpdf}, \xpackage{infwarerr}, \xpackage{ltxcmds}}{2011/10/05 v1.16}^^A
%   \TableEntry{\xpackage{refcount}}{\xpackage{infwarerr}, \xpackage{ltxcmds}}{2011/10/16 v3.4}^^A
%   \TableEntry{\xpackage{hycolor}}{\xpackage{xcolor-patch}}{2011/01/30 v1.7}^^A
% }
%
% \newenvironment{TableList}[2][htbp]{^^A
%   \begin{table}[{#1}]^^A
%   \centering
%   \caption{Overview \xpackage{hobsub-#2}}^^A
%   \label{tab:#2}^^A
%   \begin{tabular}{ll>{\raggedright}p{.4\linewidth}}^^A
%   \bfseries Package & \bfseries Release &
%   \bfseries Dependencies\tabularnewline
%   \hline
% }{^^A
%   \end{tabular}^^A
%   \end{table}^^A
% }
% \newcommand*{\TableEntry}[3]{%
%   #1&#3&#2\tabularnewline
% }
%
% \renewcommand*{\arraystretch}{1.1}
% \setlength{\extrarowheight}{2pt}
%
% \section{Documentation}
%
% In January 2011 the mailing list lualatex-dev discussed
% ``quietening lualatex console output'' \cite{lualatex:quiet}.
% Inspired from this, I implemented this package \xpackage{hobsub}
% to load several packages in one file. The package is applied for
% two packages collections that are subsets of my package bundle
% \xpackage{oberdiek}.
% \begin{description}
% \item[\xpackage{hobsub-generic}] is a collection of packages
%   that can also be used with \hologo{plainTeX}.
% \item[\xpackage{hobsub-hyperref}] is a collection of packages
%   that is used by package \xpackage{hyperref}.
% \end{description}
%
% Package \xpackage{hobsub} provides the macros that
% are used in the collection packages and might be used
% in other projects as well. But there are many caveats,
% some of them are discussed below.
%
% The most serious problem is consistency. A package
% that is available as standalone package and is part
% of a collection might have different versions.
% Because my packages are organized in a bundle and should
% be installed and updated as bundle, the risk seems low
% to me. However, if a single package is updated manually, for
% example for testing, then it should be loaded before the
% collection that contains the package with an older version.
% In a collection, the package will not be loaded again if it
% is already loaded.
%
% \subsection{Usage}
%
% The collection packages \xpackage{hobsub-generic} and
% \xpackage{hobsub-hyperref} are loaded as usual:
% \begin{quote}
%  \begin{tabular}{@{}ll@{}}
%    |\usepackage{hobsub-generic}| & \hologo{LaTeX}\\
%    |\input hobsub-generic.sty| & \hologo{plainTeX}
%  \end{tabular}
% \end{quote}
% or
% \begin{quote}|\usepackage{hobsub-hyperref}|\end{quote}
% Of course, both \cs{usepackage} or \cs{RequirePackage}
% can be used.
%
% If you need requirements on the version date of a package
% inside a collection, then specify them afterwards, for
% example:
% \begin{quote}
%   |\usepackage{hobsub-generic}|\\
%   |\usepackage{ltxcmds}[2010/12/04]% |^^A
%   {\itshape |\ltx@ifblank is needed|}
% \end{quote}
%
% Also it is not garanteed that a package will be always
% part of a collection. Therefore it does not harm,
% to make the requirements explicite \emph{after}
% a collection is loaded, see the previous example.
% \hologo{LaTeX} knows that a
% package is already loaded and does not load it again.
%
% \subsection{Supported features}
%
% \begin{description}
% \item[No reloads:] If a package is already loaded then it
%  is not reloaded again if it is part of a collection package.
% \item[\cs{listfiles}:]
%   Each package that is loaded in a collection is also
%   added to the output of \cs{listfiles}.
% \item[\cs{@currname}, \cs{@currext}]
%   are updated, they are used by \hologo{LaTeX}'s
%  \cs{ProvidesPackage}.
% \item[\cs{AtEndOfPackage}:]
%   \hologo{LaTeX}'s hook mechanism that is used for
%   \cs{AtEndOfPackage} is supported. However none of the
%   packages in the collections are using \cs{AtEndOfPackage},
%   therefore this feature is completely untested.
% \end{description}
%
% \subsection{Common limitations}
%
% \begin{description}
% \item[\cs{endinput}:]
%   Often a package is using \cs{endinput}, either added by
%   \xpackage{docstrip}, or to stop package loading at an earlier
%   time. Therefore \cs{endinput} must be catched during package
%   loading to prevent that the collection package is closed
%   too early before reading all packages it contains.
%
%   This redefinition of \cs{endinput} fails if the package loads
%   other files that contain \cs{endinput}. To some degree it could
%   be catched by redefining macros that load files. But this will
%   not work for the primitive \cs{input}.
% \item[Options:]
%   \hologo{LaTeX}'s package options are not supported.
%   Therefore the packages that are put in a collection must
%   not support options.
% \item[Hook packages:]
%   Some packages and classes provides hooks for packages, see
%   packages \xpackage{scrlfile}, \xpackage{filehook}
%   or class \xclass{memoir}. These hooks are not supported
%   for embedded packages inside a collection package.
%   Because of the problem with \cs{endinput}, these hooks
%   are not supported. The hooks might contain code that
%   loads other files that execute \cs{endinput}.
% \end{description}
%
% \subsection{Package \xpackage{hobsub-generic}}
%
% \begin{TableList}[bp]{generic}\TableGeneric\end{TableList}
%
% Collection `generic' contains some of the \hologo{plainTeX} compatible
% packages of my bundle, see table \ref{tab:generic}.
% It adds the following restrictions to any package
% that is part of this collection:
% \begin{description}
% \item[Generic formats:] The following formats must be supported:
%   \hologo{LaTeX}, \hologo{plainTeX} and even \hologo{iniTeX}.
%   This applies to the collection package \xpackage{hobsub-generic}
%   the same way.
% \item[Prefixed macros:]
%   Any package of this collection should only define macros with
%   prefixed names to avoid name clashes. Resources like registers
%   must not be allocated and other global states should not be
%   changed except for added macro definitions.
% \end{description}
% Thus the package \xpackage{hobsub-generic} can also preloaded
% in the format file with \hologo{iniTeX}.
%
% \subsection{Package \xpackage{hobsub-hyperref}}
%
% \begin{TableList}[tp]{hyperref}\TableHyperref\end{TableList}
%
% Collection `hyperref' contains additional packages that are
% used by package \xpackage{hyperref} and that does not fit
% in collection `generic', see table{tab:hyperref}.
% The latter collection package
% \xpackage{hobsub-generic} is loaded via a conventional
% \cs{RequirePackage}.
%
% Because most of the packages in the collection `hyperref'
% needs \hologo{LaTeX}, the collection requires \hologo{LaTeX}.
%
% \subsection{Package \xpackage{hobsub}}
%
% This section is for advanced users that might want to
% build a collection package. Then it can be done by
% using \xpackage{hobsub}:
% \begin{quote}
%   \dots\\
%   |\RequirePackage{hobsub}[2016/05/16]|\\
%   \dots\\
%   |\hobsub@StartPackage{foo1}|\\
%   \textit{\dots\unskip\ package code of \xfile{foo1.sty} \dots}\\
%   |\endinput|\\
%   |<hobsub>|\\
%   |\hobsub@StopPackage|\\
%   |\hobsub@StartPackage{bar2}|\\
%   \textit{\dots\unskip\ package code of \xfile{bar2.sty} \dots}\\
%   |\endinput|\\
%   |<hobsub>|\\
%   |\hobsub@StopPackage|
%   \dots
% \end{quote}
% But be aware of the caveats and limitations, see above.
% Only quite a small portion of the packages can be embedded
% like this.
%
% \clearpage
%
% \StopEventually{
% }
%
% \section{Implementation}
%
% \subsection{Reload check and package identification}
%
% \subsubsection{Package \xpackage{hobsub}}
%
%    \begin{macrocode}
%<*package>
%    \end{macrocode}
%    Reload check, especially if the package is not used with \LaTeX.
%    \begin{macrocode}
\begingroup\catcode61\catcode48\catcode32=10\relax%
  \catcode13=5 % ^^M
  \endlinechar=13 %
  \catcode35=6 % #
  \catcode39=12 % '
  \catcode44=12 % ,
  \catcode45=12 % -
  \catcode46=12 % .
  \catcode58=12 % :
  \catcode64=11 % @
  \catcode123=1 % {
  \catcode125=2 % }
  \expandafter\let\expandafter\x\csname ver@hobsub.sty\endcsname
  \ifx\x\relax % plain-TeX, first loading
  \else
    \def\empty{}%
    \ifx\x\empty % LaTeX, first loading,
      % variable is initialized, but \ProvidesPackage not yet seen
    \else
      \expandafter\ifx\csname PackageInfo\endcsname\relax
        \def\x#1#2{%
          \immediate\write-1{Package #1 Info: #2.}%
        }%
      \else
        \def\x#1#2{\PackageInfo{#1}{#2, stopped}}%
      \fi
      \x{hobsub}{The package is already loaded}%
      \aftergroup\endinput
    \fi
  \fi
\endgroup%
%    \end{macrocode}
%    Package identification:
%    \begin{macrocode}
\begingroup\catcode61\catcode48\catcode32=10\relax%
  \catcode13=5 % ^^M
  \endlinechar=13 %
  \catcode35=6 % #
  \catcode39=12 % '
  \catcode40=12 % (
  \catcode41=12 % )
  \catcode44=12 % ,
  \catcode45=12 % -
  \catcode46=12 % .
  \catcode47=12 % /
  \catcode58=12 % :
  \catcode64=11 % @
  \catcode91=12 % [
  \catcode93=12 % ]
  \catcode123=1 % {
  \catcode125=2 % }
  \expandafter\ifx\csname ProvidesPackage\endcsname\relax
    \def\x#1#2#3[#4]{\endgroup
      \immediate\write-1{Package: #3 #4}%
      \xdef#1{#4}%
    }%
  \else
    \def\x#1#2[#3]{\endgroup
      #2[{#3}]%
      \ifx#1\@undefined
        \xdef#1{#3}%
      \fi
      \ifx#1\relax
        \xdef#1{#3}%
      \fi
    }%
  \fi
\expandafter\x\csname ver@hobsub.sty\endcsname
\ProvidesPackage{hobsub}%
  [2016/05/16 v1.14 Construct package bundles (HO)]%
%    \end{macrocode}
%    \begin{macrocode}
%</package>
%    \end{macrocode}
%
% \subsubsection{Package \xpackage{hobsub-generic}}
%
%    \begin{macrocode}
%<*generic>
%    \end{macrocode}
%    Reload check, especially if the package is not used with \LaTeX.
%    \begin{macrocode}
\begingroup\catcode61\catcode48\catcode32=10\relax%
  \catcode13=5 % ^^M
  \endlinechar=13 %
  \catcode35=6 % #
  \catcode39=12 % '
  \catcode44=12 % ,
  \catcode45=12 % -
  \catcode46=12 % .
  \catcode58=12 % :
  \catcode64=11 % @
  \catcode123=1 % {
  \catcode125=2 % }
  \expandafter\let\expandafter\x\csname ver@hobsub-generic.sty\endcsname
  \ifx\x\relax % plain-TeX, first loading
  \else
    \def\empty{}%
    \ifx\x\empty % LaTeX, first loading,
      % variable is initialized, but \ProvidesPackage not yet seen
    \else
      \expandafter\ifx\csname PackageInfo\endcsname\relax
        \def\x#1#2{%
          \immediate\write-1{Package #1 Info: #2.}%
        }%
      \else
        \def\x#1#2{\PackageInfo{#1}{#2, stopped}}%
      \fi
      \x{hobsub-generic}{The package is already loaded}%
      \aftergroup\endinput
    \fi
  \fi
\endgroup%
%    \end{macrocode}
%    Package identification:
%    \begin{macrocode}
\begingroup\catcode61\catcode48\catcode32=10\relax%
  \catcode13=5 % ^^M
  \endlinechar=13 %
  \catcode35=6 % #
  \catcode39=12 % '
  \catcode40=12 % (
  \catcode41=12 % )
  \catcode44=12 % ,
  \catcode45=12 % -
  \catcode46=12 % .
  \catcode47=12 % /
  \catcode58=12 % :
  \catcode64=11 % @
  \catcode91=12 % [
  \catcode93=12 % ]
  \catcode123=1 % {
  \catcode125=2 % }
  \expandafter\ifx\csname ProvidesPackage\endcsname\relax
    \def\x#1#2#3[#4]{\endgroup
      \immediate\write-1{Package: #3 #4}%
      \xdef#1{#4}%
    }%
  \else
    \def\x#1#2[#3]{\endgroup
      #2[{#3}]%
      \ifx#1\@undefined
        \xdef#1{#3}%
      \fi
      \ifx#1\relax
        \xdef#1{#3}%
      \fi
    }%
  \fi
\expandafter\x\csname ver@hobsub-generic.sty\endcsname
\ProvidesPackage{hobsub-generic}%
  [2016/05/16 v1.14 Bundle oberdiek, subset generic (HO)]%
%    \end{macrocode}
%    \begin{macrocode}
%</generic>
%    \end{macrocode}
%
% \subsubsection{Package \xpackage{hobsub-hyperref}}
%
%    \begin{macrocode}
%<*hyperref>
%    \end{macrocode}
%    Reload check, especially if the package is not used with \LaTeX.
%    \begin{macrocode}
\begingroup\catcode61\catcode48\catcode32=10\relax%
  \catcode13=5 % ^^M
  \endlinechar=13 %
  \catcode35=6 % #
  \catcode39=12 % '
  \catcode44=12 % ,
  \catcode45=12 % -
  \catcode46=12 % .
  \catcode58=12 % :
  \catcode64=11 % @
  \catcode123=1 % {
  \catcode125=2 % }
  \expandafter\let\expandafter\x\csname ver@hobsub-hyperref.sty\endcsname
  \ifx\x\relax % plain-TeX, first loading
  \else
    \def\empty{}%
    \ifx\x\empty % LaTeX, first loading,
      % variable is initialized, but \ProvidesPackage not yet seen
    \else
      \expandafter\ifx\csname PackageInfo\endcsname\relax
        \def\x#1#2{%
          \immediate\write-1{Package #1 Info: #2.}%
        }%
      \else
        \def\x#1#2{\PackageInfo{#1}{#2, stopped}}%
      \fi
      \x{hobsub-hyperref}{The package is already loaded}%
      \aftergroup\endinput
    \fi
  \fi
\endgroup%
%    \end{macrocode}
%    Package identification:
%    \begin{macrocode}
\begingroup\catcode61\catcode48\catcode32=10\relax%
  \catcode13=5 % ^^M
  \endlinechar=13 %
  \catcode35=6 % #
  \catcode39=12 % '
  \catcode40=12 % (
  \catcode41=12 % )
  \catcode44=12 % ,
  \catcode45=12 % -
  \catcode46=12 % .
  \catcode47=12 % /
  \catcode58=12 % :
  \catcode64=11 % @
  \catcode91=12 % [
  \catcode93=12 % ]
  \catcode123=1 % {
  \catcode125=2 % }
  \expandafter\ifx\csname ProvidesPackage\endcsname\relax
    \def\x#1#2#3[#4]{\endgroup
      \immediate\write-1{Package: #3 #4}%
      \xdef#1{#4}%
    }%
  \else
    \def\x#1#2[#3]{\endgroup
      #2[{#3}]%
      \ifx#1\@undefined
        \xdef#1{#3}%
      \fi
      \ifx#1\relax
        \xdef#1{#3}%
      \fi
    }%
  \fi
\expandafter\x\csname ver@hobsub-hyperref.sty\endcsname
\ProvidesPackage{hobsub-hyperref}%
  [2016/05/16 v1.14 Bundle oberdiek, subset hyperref (HO)]%
%    \end{macrocode}
%    \begin{macrocode}
%</hyperref>
%    \end{macrocode}
%
% \subsection{Catcodes}
%
%    \begin{macrocode}
%<*package|generic|hyperref>
\begingroup\catcode61\catcode48\catcode32=10\relax%
  \catcode13=5 % ^^M
  \endlinechar=13 %
  \catcode123=1 % {
  \catcode125=2 % }
  \catcode64=11 % @
  \def\x{\endgroup
    \expandafter\edef\csname HOBsub@AtEnd\endcsname{%
      \endlinechar=\the\endlinechar\relax
      \catcode13=\the\catcode13\relax
      \catcode32=\the\catcode32\relax
      \catcode35=\the\catcode35\relax
      \catcode61=\the\catcode61\relax
      \catcode64=\the\catcode64\relax
      \catcode123=\the\catcode123\relax
      \catcode125=\the\catcode125\relax
    }%
  }%
\x\catcode61\catcode48\catcode32=10\relax%
\catcode13=5 % ^^M
\endlinechar=13 %
\catcode35=6 % #
\catcode64=11 % @
\catcode123=1 % {
\catcode125=2 % }
\def\TMP@EnsureCode#1#2{%
  \edef\HOBsub@AtEnd{%
    \HOBsub@AtEnd
    \catcode#1=\the\catcode#1\relax
  }%
  \catcode#1=#2\relax
}
\TMP@EnsureCode{39}{12}% '
\TMP@EnsureCode{40}{12}% (
\TMP@EnsureCode{41}{12}% )
\TMP@EnsureCode{45}{12}% -
\TMP@EnsureCode{46}{12}% .
\TMP@EnsureCode{47}{12}% /
\TMP@EnsureCode{58}{12}% :
\TMP@EnsureCode{60}{12}% <
\TMP@EnsureCode{62}{12}% >
\TMP@EnsureCode{96}{12}% `
\edef\HOBsub@AtEnd{\HOBsub@AtEnd\noexpand\endinput}
%    \end{macrocode}
%
%    \begin{macrocode}
%</package|generic|hyperref>
%    \end{macrocode}
%
% \subsection{Package \xpackage{hobsub-hyperref} loads \xpackage{hobsub-generic}}
%
%    \begin{macrocode}
%<*hyperref>
\NeedsTeXFormat{LaTeX2e}
\RequirePackage{hobsub-generic}[2016/05/16]
%</hyperref>
%    \end{macrocode}
%
% \subsection{Preamble}
%
%    \begin{macrocode}
%<*preamble>
\begingroup\expandafter\expandafter\expandafter\endgroup
\expandafter\ifx\csname ver@hobsub.sty\endcsname\relax
\else
  \expandafter\hobsub@GobbleRemainingPackage
\fi
%</preamble>
%    \end{macrocode}
%
%
%    \begin{macrocode}
%<*package|preamble>
%    \end{macrocode}
%
%    \begin{macro}{\HOBsub@OrgEndinput}
%    \begin{macrocode}
\begingroup\expandafter\expandafter\expandafter\endgroup
\expandafter\ifx\csname HOBsub@OrgEndinput\endcsname\relax
  \let\HOBsub@OrgEndinput\endinput
\fi
%    \end{macrocode}
%    \end{macro}
%    \begin{macro}{\hobsub@GobbleRemainingPackage}
%    \begin{macrocode}
\def\hobsub@GobbleRemainingPackage{%
  \begingroup
  \catcode92=14 % backslash: comment
  \catcode32=14 % space: comment
  \catcode35=14 % hash: comment
  \catcode123=14 % left brace: comment
  \catcode125=14 % right brace: comment
  \catcode60=3 % less: math
  \catcode62=4 % greater: align
  \endlinechar=-1 %
  \HOBsub@GobbleRemainingPackage
}
%    \end{macrocode}
%    \end{macro}
%    \begin{macro}{\HOBsub@GobbleRemainingPackage}
%    \begin{macrocode}
\catcode60=3 % less: dollar
\catcode62=4 % greater: align
\long\def\HOBsub@GobbleRemainingPackage#1<hobsub>{%
  \endgroup
}
\catcode60=12 % less: other
\catcode62=12 % greater: other
%    \end{macrocode}
%    \end{macro}
%
%    \begin{macro}{\hobsub@StartPackage}
%    \begin{macrocode}
\def\hobsub@StartPackage#1{%
  \begingroup\expandafter\expandafter\expandafter\endgroup
  \expandafter\ifx\csname ver@#1.sty\endcsname\relax
    \let\HOBsub@OrgCurrName\@currname
    \let\HOBsub@OrgCurrExt\@currext
    \csname @pushfilename\endcsname
    \def\@currname{#1}%
    \def\@currext{sty}%
    \expandafter\def\csname\@currname.\@currext-h@@k\endcsname{}%
    \let\endinput\hobsub@GobbleRemainingPackage
    \def\hobsub@StopPackage{%
      \let\hobsub@StopPackage\relax
      \HOBsub@StopPackage
%     \hobsub@Info{hobsub}{Package `#1' loaded}%
    }%
    \hobsub@AddToFileList{#1.sty}%
  \else
    \hobsub@Info{hobsub}{Skipping package `#1' (already loaded)}%
    \let\hobsub@StopPackage\relax
    \expandafter\hobsub@GobbleRemainingPackage
  \fi
}
%    \end{macrocode}
%    \end{macro}
%
%    \begin{macro}{\hobsub@StopPackage}
%    \begin{macrocode}
\let\hobsub@StopPackage\relax
%    \end{macrocode}
%    \end{macro}
%
%    \begin{macro}{\hobsub@Info}
%    \begin{macrocode}
\def\hobsub@Info#1#2{%
  \begingroup\expandafter\expandafter\expandafter\endgroup
  \expandafter\ifx\csname @PackageInfoNoLine\endcsname\relax
    \immediate\write-1{Package #1 Info: #2.}%
  \else
    \let\hobsub@Info\@PackageInfoNoLine
    \hobsub@Info{#1}{#2}%
  \fi
}
%    \end{macrocode}
%    \end{macro}
%
%    \begin{macro}{\HOBsub@StopPackage}
%    \begin{macrocode}
\def\HOBsub@StopPackage{%
  \csname\@currname.\@currext-h@@k\endcsname
  \let\endinput\HOBsub@OrgEndinput
  \csname @popfilename\endcsname
  \let\@currname\HOBsub@OrgCurrName
  \let\@currext\HOBsub@OrgCurrExt
}
%    \end{macrocode}
%    \end{macro}
%
%    \begin{macro}{\hobsub@AddToFileList}
%    \begin{macrocode}
\begingroup\expandafter\expandafter\expandafter\endgroup
\expandafter\ifx\csname @addtofilelist\endcsname\relax
  \def\hobsub@AddToFileList#1{}%
\else
  \def\hobsub@AddToFileList#1{%
    \@addtofilelist{#1}%
  }%
\fi
%    \end{macrocode}
%    \end{macro}
%
%    \begin{macrocode}
%</package|preamble>
%    \end{macrocode}
%
% \subsection{End of preamble for subset packages}
%
%    Add the end marker for possible skipping of previous
%    macros, if package \xpackage{hobsub} was already loaded.
%    \begin{macrocode}
%<*preamble>
\hobsub@GobbleRemainingPackage
<hobsub>
%    \end{macrocode}
%
%    All macros of package \xpackage{hobsub} are now available.
%    Now we officially load the package \xpackage{hobsub} inline
%    for proper package identification.
%    \begin{macrocode}
\hobsub@StartPackage{hobsub}%
\begingroup\expandafter\expandafter\expandafter\endgroup
\expandafter\ifx\csname ProvidesPackage\endcsname\relax
  \immediate\write-1{%
    Package: hobsub 2016/05/16 v1.14 Construct package bundles (HO)%
  }%
\else
  \ProvidesPackage{hobsub}%
  [2016/05/16 v1.14 Construct package bundles (HO)]%
\fi
%</preamble>
%    \end{macrocode}
%
%
%    \begin{macrocode}
%<*post|preamble>
 \endinput
<hobsub>
\hobsub@StopPackage
%</post|preamble>
%    \end{macrocode}
%
% \subsection{Package list}
%
%    \begin{macrocode}
%<infwarerr>\hobsub@StartPackage{infwarerr}
%<ltxcmds>\hobsub@StartPackage{ltxcmds}
%<ifluatex>\hobsub@StartPackage{ifluatex}
%<ifvtex>\hobsub@StartPackage{ifvtex}
%<intcalc>\hobsub@StartPackage{intcalc}
%<ifpdf>\hobsub@StartPackage{ifpdf}
%<etexcmds>\hobsub@StartPackage{etexcmds}
%<kvsetkeys>\hobsub@StartPackage{kvsetkeys}
%<kvdefinekeys>\hobsub@StartPackage{kvdefinekeys}
%<*luatex-loader>
\ifluatex
\else
  \expandafter\hobsub@GobbleRemainingPackage
\fi
\hobsub@StartPackage{luatex-loader}
%</luatex-loader>
%<pdftexcmds>\hobsub@StartPackage{pdftexcmds}
%<pdfescape>\hobsub@StartPackage{pdfescape}
%<bigintcalc>\hobsub@StartPackage{bigintcalc}
%<bitset>\hobsub@StartPackage{bitset}
%<uniquecounter>\hobsub@StartPackage{uniquecounter}
%<letltxmacro>\hobsub@StartPackage{letltxmacro}
%<hopatch>\hobsub@StartPackage{hopatch}
%<xcolor-patch>\hobsub@StartPackage{xcolor-patch}
%<atveryend>\hobsub@StartPackage{atveryend}
%<atbegshi>\hobsub@StartPackage{atbegshi}
%<refcount>\hobsub@StartPackage{refcount}
%<hycolor>\hobsub@StartPackage{hycolor}
%    \end{macrocode}
%
% \subsection{End of packages}
%
%    \begin{macrocode}
%<package|atend>\HOBsub@AtEnd%
%    \end{macrocode}
%
% \section{Test}
%
% \subsection{Catcode checks for loading}
%
%    \begin{macrocode}
%<*test1>
%    \end{macrocode}
%    \begin{macrocode}
\catcode`\{=1 %
\catcode`\}=2 %
\catcode`\#=6 %
\catcode`\@=11 %
\expandafter\ifx\csname count@\endcsname\relax
  \countdef\count@=255 %
\fi
\expandafter\ifx\csname @gobble\endcsname\relax
  \long\def\@gobble#1{}%
\fi
\expandafter\ifx\csname @firstofone\endcsname\relax
  \long\def\@firstofone#1{#1}%
\fi
\expandafter\ifx\csname loop\endcsname\relax
  \expandafter\@firstofone
\else
  \expandafter\@gobble
\fi
{%
  \def\loop#1\repeat{%
    \def\body{#1}%
    \iterate
  }%
  \def\iterate{%
    \body
      \let\next\iterate
    \else
      \let\next\relax
    \fi
    \next
  }%
  \let\repeat=\fi
}%
\def\RestoreCatcodes{}
\count@=0 %
\loop
  \edef\RestoreCatcodes{%
    \RestoreCatcodes
    \catcode\the\count@=\the\catcode\count@\relax
  }%
\ifnum\count@<255 %
  \advance\count@ 1 %
\repeat

\def\RangeCatcodeInvalid#1#2{%
  \count@=#1\relax
  \loop
    \catcode\count@=15 %
  \ifnum\count@<#2\relax
    \advance\count@ 1 %
  \repeat
}
\def\RangeCatcodeCheck#1#2#3{%
  \count@=#1\relax
  \loop
    \ifnum#3=\catcode\count@
    \else
      \errmessage{%
        Character \the\count@\space
        with wrong catcode \the\catcode\count@\space
        instead of \number#3%
      }%
    \fi
  \ifnum\count@<#2\relax
    \advance\count@ 1 %
  \repeat
}
\def\space{ }
\expandafter\ifx\csname LoadCommand\endcsname\relax
  \def\LoadCommand{\input hobsub.sty\relax}%
\fi
\def\Test{%
  \RangeCatcodeInvalid{0}{47}%
  \RangeCatcodeInvalid{58}{64}%
  \RangeCatcodeInvalid{91}{96}%
  \RangeCatcodeInvalid{123}{255}%
  \catcode`\@=12 %
  \catcode`\\=0 %
  \catcode`\%=14 %
  \LoadCommand
  \RangeCatcodeCheck{0}{36}{15}%
  \RangeCatcodeCheck{37}{37}{14}%
  \RangeCatcodeCheck{38}{47}{15}%
  \RangeCatcodeCheck{48}{57}{12}%
  \RangeCatcodeCheck{58}{63}{15}%
  \RangeCatcodeCheck{64}{64}{12}%
  \RangeCatcodeCheck{65}{90}{11}%
  \RangeCatcodeCheck{91}{91}{15}%
  \RangeCatcodeCheck{92}{92}{0}%
  \RangeCatcodeCheck{93}{96}{15}%
  \RangeCatcodeCheck{97}{122}{11}%
  \RangeCatcodeCheck{123}{255}{15}%
  \RestoreCatcodes
}
\Test
\csname @@end\endcsname
\end
%    \end{macrocode}
%    \begin{macrocode}
%</test1>
%    \end{macrocode}
%
% \section{Installation}
%
% \subsection{Download}
%
% \paragraph{Package.} This package is available on
% CTAN\footnote{\url{http://ctan.org/pkg/hobsub}}:
% \begin{description}
% \item[\CTAN{macros/latex/contrib/oberdiek/hobsub.dtx}] The source file.
% \item[\CTAN{macros/latex/contrib/oberdiek/hobsub.pdf}] Documentation.
% \end{description}
%
%
% \paragraph{Bundle.} All the packages of the bundle `oberdiek'
% are also available in a TDS compliant ZIP archive. There
% the packages are already unpacked and the documentation files
% are generated. The files and directories obey the TDS standard.
% \begin{description}
% \item[\CTAN{install/macros/latex/contrib/oberdiek.tds.zip}]
% \end{description}
% \emph{TDS} refers to the standard ``A Directory Structure
% for \TeX\ Files'' (\CTAN{tds/tds.pdf}). Directories
% with \xfile{texmf} in their name are usually organized this way.
%
% \subsection{Bundle installation}
%
% \paragraph{Unpacking.} Unpack the \xfile{oberdiek.tds.zip} in the
% TDS tree (also known as \xfile{texmf} tree) of your choice.
% Example (linux):
% \begin{quote}
%   |unzip oberdiek.tds.zip -d ~/texmf|
% \end{quote}
%
% \paragraph{Script installation.}
% Check the directory \xfile{TDS:scripts/oberdiek/} for
% scripts that need further installation steps.
% Package \xpackage{attachfile2} comes with the Perl script
% \xfile{pdfatfi.pl} that should be installed in such a way
% that it can be called as \texttt{pdfatfi}.
% Example (linux):
% \begin{quote}
%   |chmod +x scripts/oberdiek/pdfatfi.pl|\\
%   |cp scripts/oberdiek/pdfatfi.pl /usr/local/bin/|
% \end{quote}
%
% \subsection{Package installation}
%
% \paragraph{Unpacking.} The \xfile{.dtx} file is a self-extracting
% \docstrip\ archive. The files are extracted by running the
% \xfile{.dtx} through \plainTeX:
% \begin{quote}
%   \verb|tex hobsub.dtx|
% \end{quote}
%
% \paragraph{TDS.} Now the different files must be moved into
% the different directories in your installation TDS tree
% (also known as \xfile{texmf} tree):
% \begin{quote}
% \def\t{^^A
% \begin{tabular}{@{}>{\ttfamily}l@{ $\rightarrow$ }>{\ttfamily}l@{}}
%   hobsub.sty & tex/generic/oberdiek/hobsub.sty\\
%   hobsub-generic.sty & tex/generic/oberdiek/hobsub-generic.sty\\
%   hobsub-hyperref.sty & tex/generic/oberdiek/hobsub-hyperref.sty\\
%   hobsub.pdf & doc/latex/oberdiek/hobsub.pdf\\
%   test/hobsub-test1.tex & doc/latex/oberdiek/test/hobsub-test1.tex\\
%   hobsub.dtx & source/latex/oberdiek/hobsub.dtx\\
% \end{tabular}^^A
% }^^A
% \sbox0{\t}^^A
% \ifdim\wd0>\linewidth
%   \begingroup
%     \advance\linewidth by\leftmargin
%     \advance\linewidth by\rightmargin
%   \edef\x{\endgroup
%     \def\noexpand\lw{\the\linewidth}^^A
%   }\x
%   \def\lwbox{^^A
%     \leavevmode
%     \hbox to \linewidth{^^A
%       \kern-\leftmargin\relax
%       \hss
%       \usebox0
%       \hss
%       \kern-\rightmargin\relax
%     }^^A
%   }^^A
%   \ifdim\wd0>\lw
%     \sbox0{\small\t}^^A
%     \ifdim\wd0>\linewidth
%       \ifdim\wd0>\lw
%         \sbox0{\footnotesize\t}^^A
%         \ifdim\wd0>\linewidth
%           \ifdim\wd0>\lw
%             \sbox0{\scriptsize\t}^^A
%             \ifdim\wd0>\linewidth
%               \ifdim\wd0>\lw
%                 \sbox0{\tiny\t}^^A
%                 \ifdim\wd0>\linewidth
%                   \lwbox
%                 \else
%                   \usebox0
%                 \fi
%               \else
%                 \lwbox
%               \fi
%             \else
%               \usebox0
%             \fi
%           \else
%             \lwbox
%           \fi
%         \else
%           \usebox0
%         \fi
%       \else
%         \lwbox
%       \fi
%     \else
%       \usebox0
%     \fi
%   \else
%     \lwbox
%   \fi
% \else
%   \usebox0
% \fi
% \end{quote}
% If you have a \xfile{docstrip.cfg} that configures and enables \docstrip's
% TDS installing feature, then some files can already be in the right
% place, see the documentation of \docstrip.
%
% \subsection{Refresh file name databases}
%
% If your \TeX~distribution
% (\teTeX, \mikTeX, \dots) relies on file name databases, you must refresh
% these. For example, \teTeX\ users run \verb|texhash| or
% \verb|mktexlsr|.
%
% \subsection{Some details for the interested}
%
% \paragraph{Attached source.}
%
% The PDF documentation on CTAN also includes the
% \xfile{.dtx} source file. It can be extracted by
% AcrobatReader 6 or higher. Another option is \textsf{pdftk},
% e.g. unpack the file into the current directory:
% \begin{quote}
%   \verb|pdftk hobsub.pdf unpack_files output .|
% \end{quote}
%
% \paragraph{Unpacking with \LaTeX.}
% The \xfile{.dtx} chooses its action depending on the format:
% \begin{description}
% \item[\plainTeX:] Run \docstrip\ and extract the files.
% \item[\LaTeX:] Generate the documentation.
% \end{description}
% If you insist on using \LaTeX\ for \docstrip\ (really,
% \docstrip\ does not need \LaTeX), then inform the autodetect routine
% about your intention:
% \begin{quote}
%   \verb|latex \let\install=y% \iffalse meta-comment
%
% File: hobsub.dtx
% Version: 2016/05/16 v1.14
% Info: Construct package bundles
%
% Copyright (C) 2011, 2012 by
%    Heiko Oberdiek <heiko.oberdiek at googlemail.com>
%    2016
%    https://github.com/ho-tex/oberdiek/issues
%
% This work may be distributed and/or modified under the
% conditions of the LaTeX Project Public License, either
% version 1.3c of this license or (at your option) any later
% version. This version of this license is in
%    http://www.latex-project.org/lppl/lppl-1-3c.txt
% and the latest version of this license is in
%    http://www.latex-project.org/lppl.txt
% and version 1.3 or later is part of all distributions of
% LaTeX version 2005/12/01 or later.
%
% This work has the LPPL maintenance status "maintained".
%
% This Current Maintainer of this work is Heiko Oberdiek.
%
% The Base Interpreter refers to any `TeX-Format',
% because some files are installed in TDS:tex/generic//.
%
% This work consists of the main source file hobsub.dtx
% and the derived files
%    hobsub.sty, hobsub.pdf, hobsub.ins, hobsub.drv, hobsub-generic.sty,
%    hobsub-hyperref.sty, hobsub-test1.tex.
%
% Distribution:
%    CTAN:macros/latex/contrib/oberdiek/hobsub.dtx
%    CTAN:macros/latex/contrib/oberdiek/hobsub.pdf
%
% Unpacking:
%    (a) If hobsub.ins is present:
%           tex hobsub.ins
%    (b) Without hobsub.ins:
%           tex hobsub.dtx
%    (c) If you insist on using LaTeX
%           latex \let\install=y% \iffalse meta-comment
%
% File: hobsub.dtx
% Version: 2016/05/16 v1.14
% Info: Construct package bundles
%
% Copyright (C) 2011, 2012 by
%    Heiko Oberdiek <heiko.oberdiek at googlemail.com>
%    2016
%    https://github.com/ho-tex/oberdiek/issues
%
% This work may be distributed and/or modified under the
% conditions of the LaTeX Project Public License, either
% version 1.3c of this license or (at your option) any later
% version. This version of this license is in
%    http://www.latex-project.org/lppl/lppl-1-3c.txt
% and the latest version of this license is in
%    http://www.latex-project.org/lppl.txt
% and version 1.3 or later is part of all distributions of
% LaTeX version 2005/12/01 or later.
%
% This work has the LPPL maintenance status "maintained".
%
% This Current Maintainer of this work is Heiko Oberdiek.
%
% The Base Interpreter refers to any `TeX-Format',
% because some files are installed in TDS:tex/generic//.
%
% This work consists of the main source file hobsub.dtx
% and the derived files
%    hobsub.sty, hobsub.pdf, hobsub.ins, hobsub.drv, hobsub-generic.sty,
%    hobsub-hyperref.sty, hobsub-test1.tex.
%
% Distribution:
%    CTAN:macros/latex/contrib/oberdiek/hobsub.dtx
%    CTAN:macros/latex/contrib/oberdiek/hobsub.pdf
%
% Unpacking:
%    (a) If hobsub.ins is present:
%           tex hobsub.ins
%    (b) Without hobsub.ins:
%           tex hobsub.dtx
%    (c) If you insist on using LaTeX
%           latex \let\install=y% \iffalse meta-comment
%
% File: hobsub.dtx
% Version: 2016/05/16 v1.14
% Info: Construct package bundles
%
% Copyright (C) 2011, 2012 by
%    Heiko Oberdiek <heiko.oberdiek at googlemail.com>
%    2016
%    https://github.com/ho-tex/oberdiek/issues
%
% This work may be distributed and/or modified under the
% conditions of the LaTeX Project Public License, either
% version 1.3c of this license or (at your option) any later
% version. This version of this license is in
%    http://www.latex-project.org/lppl/lppl-1-3c.txt
% and the latest version of this license is in
%    http://www.latex-project.org/lppl.txt
% and version 1.3 or later is part of all distributions of
% LaTeX version 2005/12/01 or later.
%
% This work has the LPPL maintenance status "maintained".
%
% This Current Maintainer of this work is Heiko Oberdiek.
%
% The Base Interpreter refers to any `TeX-Format',
% because some files are installed in TDS:tex/generic//.
%
% This work consists of the main source file hobsub.dtx
% and the derived files
%    hobsub.sty, hobsub.pdf, hobsub.ins, hobsub.drv, hobsub-generic.sty,
%    hobsub-hyperref.sty, hobsub-test1.tex.
%
% Distribution:
%    CTAN:macros/latex/contrib/oberdiek/hobsub.dtx
%    CTAN:macros/latex/contrib/oberdiek/hobsub.pdf
%
% Unpacking:
%    (a) If hobsub.ins is present:
%           tex hobsub.ins
%    (b) Without hobsub.ins:
%           tex hobsub.dtx
%    (c) If you insist on using LaTeX
%           latex \let\install=y\input{hobsub.dtx}
%        (quote the arguments according to the demands of your shell)
%
% Documentation:
%    (a) If hobsub.drv is present:
%           latex hobsub.drv
%    (b) Without hobsub.drv:
%           latex hobsub.dtx; ...
%    The class ltxdoc loads the configuration file ltxdoc.cfg
%    if available. Here you can specify further options, e.g.
%    use A4 as paper format:
%       \PassOptionsToClass{a4paper}{article}
%
%    Programm calls to get the documentation (example):
%       pdflatex hobsub.dtx
%       makeindex -s gind.ist hobsub.idx
%       pdflatex hobsub.dtx
%       makeindex -s gind.ist hobsub.idx
%       pdflatex hobsub.dtx
%
% Installation:
%    TDS:tex/generic/oberdiek/hobsub.sty
%    TDS:tex/generic/oberdiek/hobsub-generic.sty
%    TDS:tex/generic/oberdiek/hobsub-hyperref.sty
%    TDS:doc/latex/oberdiek/hobsub.pdf
%    TDS:doc/latex/oberdiek/test/hobsub-test1.tex
%    TDS:source/latex/oberdiek/hobsub.dtx
%
%<*ignore>
\begingroup
  \catcode123=1 %
  \catcode125=2 %
  \def\x{LaTeX2e}%
\expandafter\endgroup
\ifcase 0\ifx\install y1\fi\expandafter
         \ifx\csname processbatchFile\endcsname\relax\else1\fi
         \ifx\fmtname\x\else 1\fi\relax
\else\csname fi\endcsname
%</ignore>
%<*install>
\input docstrip.tex
\Msg{************************************************************************}
\Msg{* Installation}
\Msg{* Package: hobsub 2016/05/16 v1.14 Construct package bundles (HO)}
\Msg{************************************************************************}

\keepsilent
\askforoverwritefalse

\let\MetaPrefix\relax
\preamble

This is a generated file.

Project: hobsub
Version: 2016/05/16 v1.14

Copyright (C) 2011, 2012 by
   Heiko Oberdiek <heiko.oberdiek at googlemail.com>

This work may be distributed and/or modified under the
conditions of the LaTeX Project Public License, either
version 1.3c of this license or (at your option) any later
version. This version of this license is in
   http://www.latex-project.org/lppl/lppl-1-3c.txt
and the latest version of this license is in
   http://www.latex-project.org/lppl.txt
and version 1.3 or later is part of all distributions of
LaTeX version 2005/12/01 or later.

This work has the LPPL maintenance status "maintained".

This Current Maintainer of this work is Heiko Oberdiek.

The Base Interpreter refers to any `TeX-Format',
because some files are installed in TDS:tex/generic//.

This work consists of the main source file hobsub.dtx
and the derived files
   hobsub.sty, hobsub.pdf, hobsub.ins, hobsub.drv, hobsub-generic.sty,
   hobsub-hyperref.sty, hobsub-test1.tex.

\endpreamble
\let\MetaPrefix\DoubleperCent

\generate{%
  \file{hobsub.ins}{\from{hobsub.dtx}{install}}%
  \file{hobsub.drv}{\from{hobsub.dtx}{driver}}%
  \usedir{tex/generic/oberdiek}%
  \file{hobsub.sty}{\from{hobsub.dtx}{package}}%
  \usedir{doc/latex/oberdiek/test}%
  \file{hobsub-test1.tex}{\from{hobsub.dtx}{test1}}%
  \nopreamble
  \nopostamble
  \usedir{source/latex/oberdiek/catalogue}%
  \file{hobsub.xml}{\from{hobsub.dtx}{catalogue}}%
}
\let\hobsubSavedNormalLine\normalLine
\def\hobsubGobble#1{}
\begingroup
  \lccode`\0=32 % space
  \lccode`\1=35 % hash
  \lccode`\2=37 % percent
  \lccode`\3=92 % backslash
  \lccode`\4=123 % {
  \lccode`\5=125 % }
\lowercase{\endgroup
  \let\hobsubTokenSpaceOther= 0%
  \let\hobsubTokenHash= 1%
  \let\hobsubTokenPercent= 2%
  \let\hobsubTokenBackslash= 3%
  \let\hobsubTokenBraceLeft= 4%
  \let\hobsubTokenBraceRight= 5%
  \def\hobsubNormalLineWithSpace{%
    \hobsubSavedNormalLine0%
  }%
}
\def\normalLine{%
  \futurelet\hobsubToken\hobsubFirst
}
\def\hobsubFirst{%
  \ifx\hobsubToken\hobsubTokenSpaceOther
    \expandafter\expandafter\expandafter\futurelet
    \expandafter\expandafter\expandafter\hobsubToken
    \expandafter\expandafter\expandafter\hobsubNext
    \expandafter\hobsubGobble
  \else
    \expandafter\hobsubSavedNormalLine
  \fi
}
\def\hobsubNext{%
  \ifx\hobsubToken\hobsubTokenSpaceOther
    \expandafter\expandafter\expandafter\futurelet
    \expandafter\expandafter\expandafter\hobsubToken
    \expandafter\expandafter\expandafter\hobsubNext 
    \expandafter\hobsubGobble
  \else
    \let\hobsubWithSpace=Y%
    \ifx\hobsubToken\hobsubTokenBackslash
      \let\hobsubWithSpace=N%
    \else
      \ifx\hobsubToken\hobsubTokenBraceRight
        \let\hobsubWithSpace=N%
      \else
        \ifx\hobsubToken\hobsubTokenBraceLeft
          \let\hobsubWithSpace=N%
        \else
          \ifx\hobsubToken\hobsubTokenPercent
            \let\hobsubWithSpace=N%
          \else
            \ifx\hobsubToken\hobsubTokenHash
              \let\hobsubWithSpace=N%
            \fi
          \fi
        \fi
      \fi
    \fi
    \ifx\hobsubWithSpace Y%
      \expandafter\expandafter\expandafter\hobsubNormalLineWithSpace
    \else
      \expandafter\expandafter\expandafter\hobsubSavedNormalLine
    \fi
  \fi
}
\generate{%
  \usedir{tex/generic/oberdiek}%
  \file{hobsub-generic.sty}{%
    \from{hobsub.dtx}{generic,preamble,infwarerr}%
    \from{infwarerr.dtx}{package}%
    \from{hobsub.dtx}{post,ltxcmds}%
    \from{ltxcmds.dtx}{package}%
    \from{hobsub.dtx}{post,ifluatex}%
    \from{ifluatex.dtx}{package}%
    \from{hobsub.dtx}{post,ifvtex}%
    \from{ifvtex.dtx}{package}%
    \from{hobsub.dtx}{post,intcalc}%
    \from{intcalc.dtx}{package}%
    \from{hobsub.dtx}{post,ifpdf}%
    \from{ifpdf.dtx}{package}%
    \from{hobsub.dtx}{post,etexcmds}%
    \from{etexcmds.dtx}{package}%
    \from{hobsub.dtx}{post,kvsetkeys}%
    \from{kvsetkeys.dtx}{package}%
    \from{hobsub.dtx}{post,kvdefinekeys}%
    \from{kvdefinekeys.dtx}{package}%
    \from{hobsub.dtx}{post,luatex-loader}%
    \from{luatex.dtx}{loader}%
    \from{hobsub.dtx}{post,pdftexcmds}%
    \from{pdftexcmds.dtx}{package}%
    \from{hobsub.dtx}{post,pdfescape}%
    \from{pdfescape.dtx}{package}%
    \from{hobsub.dtx}{post,bigintcalc}%
    \from{bigintcalc.dtx}{package}%
    \from{hobsub.dtx}{post,bitset}%
    \from{bitset.dtx}{package}%
    \from{hobsub.dtx}{post,uniquecounter}%
    \from{uniquecounter.dtx}{package}%
    \from{hobsub.dtx}{post,atend}%
  }%
}
\generate{%
  \usedir{tex/generic/oberdiek}%
  \file{hobsub-hyperref.sty}{%
    \from{hobsub.dtx}{hyperref,preamble,,letltxmacro}%
    \from{letltxmacro.dtx}{package}%
    \from{hobsub.dtx}{post,hopatch}%
    \from{hopatch.dtx}{package}%
    \from{hobsub.dtx}{post,xcolor-patch}%
    \from{hycolor.dtx}{xcolor}%
    \from{hobsub.dtx}{post,atveryend}%
    \from{atveryend.dtx}{package}%
    \from{hobsub.dtx}{post,atbegshi}%
    \from{atbegshi.dtx}{package}%
    \from{hobsub.dtx}{post,refcount}%
    \from{refcount.dtx}{package}%
    \from{hobsub.dtx}{post,hycolor}%
    \from{hycolor.dtx}{package}%
    \from{hobsub.dtx}{post}%
  }%
}
\let\normalLine\hobsubSavedNormalLine

\catcode32=13\relax% active space
\let =\space%
\Msg{************************************************************************}
\Msg{*}
\Msg{* To finish the installation you have to move the following}
\Msg{* files into a directory searched by TeX:}
\Msg{*}
\Msg{*     hobsub.sty, hobsub-generic.sty, hobsub-hyperref.sty}
\Msg{*}
\Msg{* To produce the documentation run the file `hobsub.drv'}
\Msg{* through LaTeX.}
\Msg{*}
\Msg{* Happy TeXing!}
\Msg{*}
\Msg{************************************************************************}

\endbatchfile
%</install>
%<*ignore>
\fi
%</ignore>
%<*driver>
\NeedsTeXFormat{LaTeX2e}
\ProvidesFile{hobsub.drv}%
  [2016/05/16 v1.14 Construct package bundles (HO)]%
\documentclass{ltxdoc}
\usepackage{holtxdoc}[2011/11/22]
\usepackage{array}
\usepackage{caption}
\renewcommand*{\bottomfraction}{.5}
\renewcommand*{\topfraction}{.5}
\begin{document}
  \DocInput{hobsub.dtx}%
\end{document}
%</driver>
% \fi
%
%
% \CharacterTable
%  {Upper-case    \A\B\C\D\E\F\G\H\I\J\K\L\M\N\O\P\Q\R\S\T\U\V\W\X\Y\Z
%   Lower-case    \a\b\c\d\e\f\g\h\i\j\k\l\m\n\o\p\q\r\s\t\u\v\w\x\y\z
%   Digits        \0\1\2\3\4\5\6\7\8\9
%   Exclamation   \!     Double quote  \"     Hash (number) \#
%   Dollar        \$     Percent       \%     Ampersand     \&
%   Acute accent  \'     Left paren    \(     Right paren   \)
%   Asterisk      \*     Plus          \+     Comma         \,
%   Minus         \-     Point         \.     Solidus       \/
%   Colon         \:     Semicolon     \;     Less than     \<
%   Equals        \=     Greater than  \>     Question mark \?
%   Commercial at \@     Left bracket  \[     Backslash     \\
%   Right bracket \]     Circumflex    \^     Underscore    \_
%   Grave accent  \`     Left brace    \{     Vertical bar  \|
%   Right brace   \}     Tilde         \~}
%
% \GetFileInfo{hobsub.drv}
%
% \title{The \xpackage{hobsub} package}
% \date{2016/05/16 v1.14}
% \author{Heiko Oberdiek\thanks
% {Please report any issues at https://github.com/ho-tex/oberdiek/issues}\\
% \xemail{heiko.oberdiek at googlemail.com}}
%
% \maketitle
%
% \begin{abstract}
% Package \xpackage{hobsub} implements the idea of loading
% several packages in one big collection package.
% \end{abstract}
%
% \tableofcontents
%
% \def\TableGeneric{^^A
%   \TableEntry{\xpackage{infwarerr}}{}{2010/04/08 v1.3}^^A
%   \TableEntry{\xpackage{ltxcmds}}{}{2011/11/09 v1.22}^^A
%   \TableEntry{\xpackage{ifluatex}}{}{2010/03/01 v1.3}^^A
%   \TableEntry{\xpackage{ifvtex}}{}{2010/03/01 v1.5}^^A
%   \TableEntry{\xpackage{intcalc}}{}{2007/09/27 v1.1}^^A
%   \TableEntry{\xpackage{ifpdf}}{\xpackage{ifluatex}}{2011/01/30 v2.3}^^A
%   \TableEntry{\xpackage{etexcmds}}{}{2011/02/16 v1.5}^^A
%   \TableEntry{\xpackage{kvsetkeys}}{}{2012/04/25 v1.16}^^A
%   \TableEntry{\xpackage{kvdefinekeys}}{}{2011/04/07 v1.3}^^A
%   \TableEntry{\xpackage{luatex-loader}}{}{2010/03/09 v0.4}^^A
%   \TableEntry{\xpackage{pdftexcmds}}{\xpackage{ifluatex}, \xpackage{ifpdf}, \xpackage{infwarerr}, \xpackage{ltxcmds}, \xpackage{luatex-loader}}{2011/11/29 v0.20}^^A
%   \TableEntry{\xpackage{pdfescape}}{\xpackage{ltxcmds}, \xpackage{pdftexcmds}}{2011/11/25 v1.13}^^A
%   \TableEntry{\xpackage{bigintcalc}}{\xpackage{pdftexcmds}}{2012/04/08 v1.3}^^A
%   \TableEntry{\xpackage{bitset}}{\xpackage{bigintcalc}, \xpackage{infwarerr}, \xpackage{intcalc}}{2011/01/30 v1.1}^^A
%   \TableEntry{\xpackage{uniquecounter}}{\xpackage{bigintcalc}, \xpackage{infwarerr}}{2011/01/30 v1.2}^^A
% }
% \def\TableHyperref{^^A
%   \TableEntry{\xpackage{letltxmacro}}{}{2010/09/02 v1.4}^^A
%   \TableEntry{\xpackage{hopatch}}{}{2016/05/16 v1.2}^^A
%   \TableEntry{\xpackage{xcolor-patch}}{}{2011/01/30 v1.7}^^A
%   \TableEntry{\xpackage{atveryend}}{}{2011/06/30 v1.8}^^A
%   \TableEntry{\xpackage{atbegshi}}{\xpackage{ifpdf}, \xpackage{infwarerr}, \xpackage{ltxcmds}}{2011/10/05 v1.16}^^A
%   \TableEntry{\xpackage{refcount}}{\xpackage{infwarerr}, \xpackage{ltxcmds}}{2011/10/16 v3.4}^^A
%   \TableEntry{\xpackage{hycolor}}{\xpackage{xcolor-patch}}{2011/01/30 v1.7}^^A
% }
%
% \newenvironment{TableList}[2][htbp]{^^A
%   \begin{table}[{#1}]^^A
%   \centering
%   \caption{Overview \xpackage{hobsub-#2}}^^A
%   \label{tab:#2}^^A
%   \begin{tabular}{ll>{\raggedright}p{.4\linewidth}}^^A
%   \bfseries Package & \bfseries Release &
%   \bfseries Dependencies\tabularnewline
%   \hline
% }{^^A
%   \end{tabular}^^A
%   \end{table}^^A
% }
% \newcommand*{\TableEntry}[3]{%
%   #1&#3&#2\tabularnewline
% }
%
% \renewcommand*{\arraystretch}{1.1}
% \setlength{\extrarowheight}{2pt}
%
% \section{Documentation}
%
% In January 2011 the mailing list lualatex-dev discussed
% ``quietening lualatex console output'' \cite{lualatex:quiet}.
% Inspired from this, I implemented this package \xpackage{hobsub}
% to load several packages in one file. The package is applied for
% two packages collections that are subsets of my package bundle
% \xpackage{oberdiek}.
% \begin{description}
% \item[\xpackage{hobsub-generic}] is a collection of packages
%   that can also be used with \hologo{plainTeX}.
% \item[\xpackage{hobsub-hyperref}] is a collection of packages
%   that is used by package \xpackage{hyperref}.
% \end{description}
%
% Package \xpackage{hobsub} provides the macros that
% are used in the collection packages and might be used
% in other projects as well. But there are many caveats,
% some of them are discussed below.
%
% The most serious problem is consistency. A package
% that is available as standalone package and is part
% of a collection might have different versions.
% Because my packages are organized in a bundle and should
% be installed and updated as bundle, the risk seems low
% to me. However, if a single package is updated manually, for
% example for testing, then it should be loaded before the
% collection that contains the package with an older version.
% In a collection, the package will not be loaded again if it
% is already loaded.
%
% \subsection{Usage}
%
% The collection packages \xpackage{hobsub-generic} and
% \xpackage{hobsub-hyperref} are loaded as usual:
% \begin{quote}
%  \begin{tabular}{@{}ll@{}}
%    |\usepackage{hobsub-generic}| & \hologo{LaTeX}\\
%    |\input hobsub-generic.sty| & \hologo{plainTeX}
%  \end{tabular}
% \end{quote}
% or
% \begin{quote}|\usepackage{hobsub-hyperref}|\end{quote}
% Of course, both \cs{usepackage} or \cs{RequirePackage}
% can be used.
%
% If you need requirements on the version date of a package
% inside a collection, then specify them afterwards, for
% example:
% \begin{quote}
%   |\usepackage{hobsub-generic}|\\
%   |\usepackage{ltxcmds}[2010/12/04]% |^^A
%   {\itshape |\ltx@ifblank is needed|}
% \end{quote}
%
% Also it is not garanteed that a package will be always
% part of a collection. Therefore it does not harm,
% to make the requirements explicite \emph{after}
% a collection is loaded, see the previous example.
% \hologo{LaTeX} knows that a
% package is already loaded and does not load it again.
%
% \subsection{Supported features}
%
% \begin{description}
% \item[No reloads:] If a package is already loaded then it
%  is not reloaded again if it is part of a collection package.
% \item[\cs{listfiles}:]
%   Each package that is loaded in a collection is also
%   added to the output of \cs{listfiles}.
% \item[\cs{@currname}, \cs{@currext}]
%   are updated, they are used by \hologo{LaTeX}'s
%  \cs{ProvidesPackage}.
% \item[\cs{AtEndOfPackage}:]
%   \hologo{LaTeX}'s hook mechanism that is used for
%   \cs{AtEndOfPackage} is supported. However none of the
%   packages in the collections are using \cs{AtEndOfPackage},
%   therefore this feature is completely untested.
% \end{description}
%
% \subsection{Common limitations}
%
% \begin{description}
% \item[\cs{endinput}:]
%   Often a package is using \cs{endinput}, either added by
%   \xpackage{docstrip}, or to stop package loading at an earlier
%   time. Therefore \cs{endinput} must be catched during package
%   loading to prevent that the collection package is closed
%   too early before reading all packages it contains.
%
%   This redefinition of \cs{endinput} fails if the package loads
%   other files that contain \cs{endinput}. To some degree it could
%   be catched by redefining macros that load files. But this will
%   not work for the primitive \cs{input}.
% \item[Options:]
%   \hologo{LaTeX}'s package options are not supported.
%   Therefore the packages that are put in a collection must
%   not support options.
% \item[Hook packages:]
%   Some packages and classes provides hooks for packages, see
%   packages \xpackage{scrlfile}, \xpackage{filehook}
%   or class \xclass{memoir}. These hooks are not supported
%   for embedded packages inside a collection package.
%   Because of the problem with \cs{endinput}, these hooks
%   are not supported. The hooks might contain code that
%   loads other files that execute \cs{endinput}.
% \end{description}
%
% \subsection{Package \xpackage{hobsub-generic}}
%
% \begin{TableList}[bp]{generic}\TableGeneric\end{TableList}
%
% Collection `generic' contains some of the \hologo{plainTeX} compatible
% packages of my bundle, see table \ref{tab:generic}.
% It adds the following restrictions to any package
% that is part of this collection:
% \begin{description}
% \item[Generic formats:] The following formats must be supported:
%   \hologo{LaTeX}, \hologo{plainTeX} and even \hologo{iniTeX}.
%   This applies to the collection package \xpackage{hobsub-generic}
%   the same way.
% \item[Prefixed macros:]
%   Any package of this collection should only define macros with
%   prefixed names to avoid name clashes. Resources like registers
%   must not be allocated and other global states should not be
%   changed except for added macro definitions.
% \end{description}
% Thus the package \xpackage{hobsub-generic} can also preloaded
% in the format file with \hologo{iniTeX}.
%
% \subsection{Package \xpackage{hobsub-hyperref}}
%
% \begin{TableList}[tp]{hyperref}\TableHyperref\end{TableList}
%
% Collection `hyperref' contains additional packages that are
% used by package \xpackage{hyperref} and that does not fit
% in collection `generic', see table{tab:hyperref}.
% The latter collection package
% \xpackage{hobsub-generic} is loaded via a conventional
% \cs{RequirePackage}.
%
% Because most of the packages in the collection `hyperref'
% needs \hologo{LaTeX}, the collection requires \hologo{LaTeX}.
%
% \subsection{Package \xpackage{hobsub}}
%
% This section is for advanced users that might want to
% build a collection package. Then it can be done by
% using \xpackage{hobsub}:
% \begin{quote}
%   \dots\\
%   |\RequirePackage{hobsub}[2016/05/16]|\\
%   \dots\\
%   |\hobsub@StartPackage{foo1}|\\
%   \textit{\dots\unskip\ package code of \xfile{foo1.sty} \dots}\\
%   |\endinput|\\
%   |<hobsub>|\\
%   |\hobsub@StopPackage|\\
%   |\hobsub@StartPackage{bar2}|\\
%   \textit{\dots\unskip\ package code of \xfile{bar2.sty} \dots}\\
%   |\endinput|\\
%   |<hobsub>|\\
%   |\hobsub@StopPackage|
%   \dots
% \end{quote}
% But be aware of the caveats and limitations, see above.
% Only quite a small portion of the packages can be embedded
% like this.
%
% \clearpage
%
% \StopEventually{
% }
%
% \section{Implementation}
%
% \subsection{Reload check and package identification}
%
% \subsubsection{Package \xpackage{hobsub}}
%
%    \begin{macrocode}
%<*package>
%    \end{macrocode}
%    Reload check, especially if the package is not used with \LaTeX.
%    \begin{macrocode}
\begingroup\catcode61\catcode48\catcode32=10\relax%
  \catcode13=5 % ^^M
  \endlinechar=13 %
  \catcode35=6 % #
  \catcode39=12 % '
  \catcode44=12 % ,
  \catcode45=12 % -
  \catcode46=12 % .
  \catcode58=12 % :
  \catcode64=11 % @
  \catcode123=1 % {
  \catcode125=2 % }
  \expandafter\let\expandafter\x\csname ver@hobsub.sty\endcsname
  \ifx\x\relax % plain-TeX, first loading
  \else
    \def\empty{}%
    \ifx\x\empty % LaTeX, first loading,
      % variable is initialized, but \ProvidesPackage not yet seen
    \else
      \expandafter\ifx\csname PackageInfo\endcsname\relax
        \def\x#1#2{%
          \immediate\write-1{Package #1 Info: #2.}%
        }%
      \else
        \def\x#1#2{\PackageInfo{#1}{#2, stopped}}%
      \fi
      \x{hobsub}{The package is already loaded}%
      \aftergroup\endinput
    \fi
  \fi
\endgroup%
%    \end{macrocode}
%    Package identification:
%    \begin{macrocode}
\begingroup\catcode61\catcode48\catcode32=10\relax%
  \catcode13=5 % ^^M
  \endlinechar=13 %
  \catcode35=6 % #
  \catcode39=12 % '
  \catcode40=12 % (
  \catcode41=12 % )
  \catcode44=12 % ,
  \catcode45=12 % -
  \catcode46=12 % .
  \catcode47=12 % /
  \catcode58=12 % :
  \catcode64=11 % @
  \catcode91=12 % [
  \catcode93=12 % ]
  \catcode123=1 % {
  \catcode125=2 % }
  \expandafter\ifx\csname ProvidesPackage\endcsname\relax
    \def\x#1#2#3[#4]{\endgroup
      \immediate\write-1{Package: #3 #4}%
      \xdef#1{#4}%
    }%
  \else
    \def\x#1#2[#3]{\endgroup
      #2[{#3}]%
      \ifx#1\@undefined
        \xdef#1{#3}%
      \fi
      \ifx#1\relax
        \xdef#1{#3}%
      \fi
    }%
  \fi
\expandafter\x\csname ver@hobsub.sty\endcsname
\ProvidesPackage{hobsub}%
  [2016/05/16 v1.14 Construct package bundles (HO)]%
%    \end{macrocode}
%    \begin{macrocode}
%</package>
%    \end{macrocode}
%
% \subsubsection{Package \xpackage{hobsub-generic}}
%
%    \begin{macrocode}
%<*generic>
%    \end{macrocode}
%    Reload check, especially if the package is not used with \LaTeX.
%    \begin{macrocode}
\begingroup\catcode61\catcode48\catcode32=10\relax%
  \catcode13=5 % ^^M
  \endlinechar=13 %
  \catcode35=6 % #
  \catcode39=12 % '
  \catcode44=12 % ,
  \catcode45=12 % -
  \catcode46=12 % .
  \catcode58=12 % :
  \catcode64=11 % @
  \catcode123=1 % {
  \catcode125=2 % }
  \expandafter\let\expandafter\x\csname ver@hobsub-generic.sty\endcsname
  \ifx\x\relax % plain-TeX, first loading
  \else
    \def\empty{}%
    \ifx\x\empty % LaTeX, first loading,
      % variable is initialized, but \ProvidesPackage not yet seen
    \else
      \expandafter\ifx\csname PackageInfo\endcsname\relax
        \def\x#1#2{%
          \immediate\write-1{Package #1 Info: #2.}%
        }%
      \else
        \def\x#1#2{\PackageInfo{#1}{#2, stopped}}%
      \fi
      \x{hobsub-generic}{The package is already loaded}%
      \aftergroup\endinput
    \fi
  \fi
\endgroup%
%    \end{macrocode}
%    Package identification:
%    \begin{macrocode}
\begingroup\catcode61\catcode48\catcode32=10\relax%
  \catcode13=5 % ^^M
  \endlinechar=13 %
  \catcode35=6 % #
  \catcode39=12 % '
  \catcode40=12 % (
  \catcode41=12 % )
  \catcode44=12 % ,
  \catcode45=12 % -
  \catcode46=12 % .
  \catcode47=12 % /
  \catcode58=12 % :
  \catcode64=11 % @
  \catcode91=12 % [
  \catcode93=12 % ]
  \catcode123=1 % {
  \catcode125=2 % }
  \expandafter\ifx\csname ProvidesPackage\endcsname\relax
    \def\x#1#2#3[#4]{\endgroup
      \immediate\write-1{Package: #3 #4}%
      \xdef#1{#4}%
    }%
  \else
    \def\x#1#2[#3]{\endgroup
      #2[{#3}]%
      \ifx#1\@undefined
        \xdef#1{#3}%
      \fi
      \ifx#1\relax
        \xdef#1{#3}%
      \fi
    }%
  \fi
\expandafter\x\csname ver@hobsub-generic.sty\endcsname
\ProvidesPackage{hobsub-generic}%
  [2016/05/16 v1.14 Bundle oberdiek, subset generic (HO)]%
%    \end{macrocode}
%    \begin{macrocode}
%</generic>
%    \end{macrocode}
%
% \subsubsection{Package \xpackage{hobsub-hyperref}}
%
%    \begin{macrocode}
%<*hyperref>
%    \end{macrocode}
%    Reload check, especially if the package is not used with \LaTeX.
%    \begin{macrocode}
\begingroup\catcode61\catcode48\catcode32=10\relax%
  \catcode13=5 % ^^M
  \endlinechar=13 %
  \catcode35=6 % #
  \catcode39=12 % '
  \catcode44=12 % ,
  \catcode45=12 % -
  \catcode46=12 % .
  \catcode58=12 % :
  \catcode64=11 % @
  \catcode123=1 % {
  \catcode125=2 % }
  \expandafter\let\expandafter\x\csname ver@hobsub-hyperref.sty\endcsname
  \ifx\x\relax % plain-TeX, first loading
  \else
    \def\empty{}%
    \ifx\x\empty % LaTeX, first loading,
      % variable is initialized, but \ProvidesPackage not yet seen
    \else
      \expandafter\ifx\csname PackageInfo\endcsname\relax
        \def\x#1#2{%
          \immediate\write-1{Package #1 Info: #2.}%
        }%
      \else
        \def\x#1#2{\PackageInfo{#1}{#2, stopped}}%
      \fi
      \x{hobsub-hyperref}{The package is already loaded}%
      \aftergroup\endinput
    \fi
  \fi
\endgroup%
%    \end{macrocode}
%    Package identification:
%    \begin{macrocode}
\begingroup\catcode61\catcode48\catcode32=10\relax%
  \catcode13=5 % ^^M
  \endlinechar=13 %
  \catcode35=6 % #
  \catcode39=12 % '
  \catcode40=12 % (
  \catcode41=12 % )
  \catcode44=12 % ,
  \catcode45=12 % -
  \catcode46=12 % .
  \catcode47=12 % /
  \catcode58=12 % :
  \catcode64=11 % @
  \catcode91=12 % [
  \catcode93=12 % ]
  \catcode123=1 % {
  \catcode125=2 % }
  \expandafter\ifx\csname ProvidesPackage\endcsname\relax
    \def\x#1#2#3[#4]{\endgroup
      \immediate\write-1{Package: #3 #4}%
      \xdef#1{#4}%
    }%
  \else
    \def\x#1#2[#3]{\endgroup
      #2[{#3}]%
      \ifx#1\@undefined
        \xdef#1{#3}%
      \fi
      \ifx#1\relax
        \xdef#1{#3}%
      \fi
    }%
  \fi
\expandafter\x\csname ver@hobsub-hyperref.sty\endcsname
\ProvidesPackage{hobsub-hyperref}%
  [2016/05/16 v1.14 Bundle oberdiek, subset hyperref (HO)]%
%    \end{macrocode}
%    \begin{macrocode}
%</hyperref>
%    \end{macrocode}
%
% \subsection{Catcodes}
%
%    \begin{macrocode}
%<*package|generic|hyperref>
\begingroup\catcode61\catcode48\catcode32=10\relax%
  \catcode13=5 % ^^M
  \endlinechar=13 %
  \catcode123=1 % {
  \catcode125=2 % }
  \catcode64=11 % @
  \def\x{\endgroup
    \expandafter\edef\csname HOBsub@AtEnd\endcsname{%
      \endlinechar=\the\endlinechar\relax
      \catcode13=\the\catcode13\relax
      \catcode32=\the\catcode32\relax
      \catcode35=\the\catcode35\relax
      \catcode61=\the\catcode61\relax
      \catcode64=\the\catcode64\relax
      \catcode123=\the\catcode123\relax
      \catcode125=\the\catcode125\relax
    }%
  }%
\x\catcode61\catcode48\catcode32=10\relax%
\catcode13=5 % ^^M
\endlinechar=13 %
\catcode35=6 % #
\catcode64=11 % @
\catcode123=1 % {
\catcode125=2 % }
\def\TMP@EnsureCode#1#2{%
  \edef\HOBsub@AtEnd{%
    \HOBsub@AtEnd
    \catcode#1=\the\catcode#1\relax
  }%
  \catcode#1=#2\relax
}
\TMP@EnsureCode{39}{12}% '
\TMP@EnsureCode{40}{12}% (
\TMP@EnsureCode{41}{12}% )
\TMP@EnsureCode{45}{12}% -
\TMP@EnsureCode{46}{12}% .
\TMP@EnsureCode{47}{12}% /
\TMP@EnsureCode{58}{12}% :
\TMP@EnsureCode{60}{12}% <
\TMP@EnsureCode{62}{12}% >
\TMP@EnsureCode{96}{12}% `
\edef\HOBsub@AtEnd{\HOBsub@AtEnd\noexpand\endinput}
%    \end{macrocode}
%
%    \begin{macrocode}
%</package|generic|hyperref>
%    \end{macrocode}
%
% \subsection{Package \xpackage{hobsub-hyperref} loads \xpackage{hobsub-generic}}
%
%    \begin{macrocode}
%<*hyperref>
\NeedsTeXFormat{LaTeX2e}
\RequirePackage{hobsub-generic}[2016/05/16]
%</hyperref>
%    \end{macrocode}
%
% \subsection{Preamble}
%
%    \begin{macrocode}
%<*preamble>
\begingroup\expandafter\expandafter\expandafter\endgroup
\expandafter\ifx\csname ver@hobsub.sty\endcsname\relax
\else
  \expandafter\hobsub@GobbleRemainingPackage
\fi
%</preamble>
%    \end{macrocode}
%
%
%    \begin{macrocode}
%<*package|preamble>
%    \end{macrocode}
%
%    \begin{macro}{\HOBsub@OrgEndinput}
%    \begin{macrocode}
\begingroup\expandafter\expandafter\expandafter\endgroup
\expandafter\ifx\csname HOBsub@OrgEndinput\endcsname\relax
  \let\HOBsub@OrgEndinput\endinput
\fi
%    \end{macrocode}
%    \end{macro}
%    \begin{macro}{\hobsub@GobbleRemainingPackage}
%    \begin{macrocode}
\def\hobsub@GobbleRemainingPackage{%
  \begingroup
  \catcode92=14 % backslash: comment
  \catcode32=14 % space: comment
  \catcode35=14 % hash: comment
  \catcode123=14 % left brace: comment
  \catcode125=14 % right brace: comment
  \catcode60=3 % less: math
  \catcode62=4 % greater: align
  \endlinechar=-1 %
  \HOBsub@GobbleRemainingPackage
}
%    \end{macrocode}
%    \end{macro}
%    \begin{macro}{\HOBsub@GobbleRemainingPackage}
%    \begin{macrocode}
\catcode60=3 % less: dollar
\catcode62=4 % greater: align
\long\def\HOBsub@GobbleRemainingPackage#1<hobsub>{%
  \endgroup
}
\catcode60=12 % less: other
\catcode62=12 % greater: other
%    \end{macrocode}
%    \end{macro}
%
%    \begin{macro}{\hobsub@StartPackage}
%    \begin{macrocode}
\def\hobsub@StartPackage#1{%
  \begingroup\expandafter\expandafter\expandafter\endgroup
  \expandafter\ifx\csname ver@#1.sty\endcsname\relax
    \let\HOBsub@OrgCurrName\@currname
    \let\HOBsub@OrgCurrExt\@currext
    \csname @pushfilename\endcsname
    \def\@currname{#1}%
    \def\@currext{sty}%
    \expandafter\def\csname\@currname.\@currext-h@@k\endcsname{}%
    \let\endinput\hobsub@GobbleRemainingPackage
    \def\hobsub@StopPackage{%
      \let\hobsub@StopPackage\relax
      \HOBsub@StopPackage
%     \hobsub@Info{hobsub}{Package `#1' loaded}%
    }%
    \hobsub@AddToFileList{#1.sty}%
  \else
    \hobsub@Info{hobsub}{Skipping package `#1' (already loaded)}%
    \let\hobsub@StopPackage\relax
    \expandafter\hobsub@GobbleRemainingPackage
  \fi
}
%    \end{macrocode}
%    \end{macro}
%
%    \begin{macro}{\hobsub@StopPackage}
%    \begin{macrocode}
\let\hobsub@StopPackage\relax
%    \end{macrocode}
%    \end{macro}
%
%    \begin{macro}{\hobsub@Info}
%    \begin{macrocode}
\def\hobsub@Info#1#2{%
  \begingroup\expandafter\expandafter\expandafter\endgroup
  \expandafter\ifx\csname @PackageInfoNoLine\endcsname\relax
    \immediate\write-1{Package #1 Info: #2.}%
  \else
    \let\hobsub@Info\@PackageInfoNoLine
    \hobsub@Info{#1}{#2}%
  \fi
}
%    \end{macrocode}
%    \end{macro}
%
%    \begin{macro}{\HOBsub@StopPackage}
%    \begin{macrocode}
\def\HOBsub@StopPackage{%
  \csname\@currname.\@currext-h@@k\endcsname
  \let\endinput\HOBsub@OrgEndinput
  \csname @popfilename\endcsname
  \let\@currname\HOBsub@OrgCurrName
  \let\@currext\HOBsub@OrgCurrExt
}
%    \end{macrocode}
%    \end{macro}
%
%    \begin{macro}{\hobsub@AddToFileList}
%    \begin{macrocode}
\begingroup\expandafter\expandafter\expandafter\endgroup
\expandafter\ifx\csname @addtofilelist\endcsname\relax
  \def\hobsub@AddToFileList#1{}%
\else
  \def\hobsub@AddToFileList#1{%
    \@addtofilelist{#1}%
  }%
\fi
%    \end{macrocode}
%    \end{macro}
%
%    \begin{macrocode}
%</package|preamble>
%    \end{macrocode}
%
% \subsection{End of preamble for subset packages}
%
%    Add the end marker for possible skipping of previous
%    macros, if package \xpackage{hobsub} was already loaded.
%    \begin{macrocode}
%<*preamble>
\hobsub@GobbleRemainingPackage
<hobsub>
%    \end{macrocode}
%
%    All macros of package \xpackage{hobsub} are now available.
%    Now we officially load the package \xpackage{hobsub} inline
%    for proper package identification.
%    \begin{macrocode}
\hobsub@StartPackage{hobsub}%
\begingroup\expandafter\expandafter\expandafter\endgroup
\expandafter\ifx\csname ProvidesPackage\endcsname\relax
  \immediate\write-1{%
    Package: hobsub 2016/05/16 v1.14 Construct package bundles (HO)%
  }%
\else
  \ProvidesPackage{hobsub}%
  [2016/05/16 v1.14 Construct package bundles (HO)]%
\fi
%</preamble>
%    \end{macrocode}
%
%
%    \begin{macrocode}
%<*post|preamble>
 \endinput
<hobsub>
\hobsub@StopPackage
%</post|preamble>
%    \end{macrocode}
%
% \subsection{Package list}
%
%    \begin{macrocode}
%<infwarerr>\hobsub@StartPackage{infwarerr}
%<ltxcmds>\hobsub@StartPackage{ltxcmds}
%<ifluatex>\hobsub@StartPackage{ifluatex}
%<ifvtex>\hobsub@StartPackage{ifvtex}
%<intcalc>\hobsub@StartPackage{intcalc}
%<ifpdf>\hobsub@StartPackage{ifpdf}
%<etexcmds>\hobsub@StartPackage{etexcmds}
%<kvsetkeys>\hobsub@StartPackage{kvsetkeys}
%<kvdefinekeys>\hobsub@StartPackage{kvdefinekeys}
%<*luatex-loader>
\ifluatex
\else
  \expandafter\hobsub@GobbleRemainingPackage
\fi
\hobsub@StartPackage{luatex-loader}
%</luatex-loader>
%<pdftexcmds>\hobsub@StartPackage{pdftexcmds}
%<pdfescape>\hobsub@StartPackage{pdfescape}
%<bigintcalc>\hobsub@StartPackage{bigintcalc}
%<bitset>\hobsub@StartPackage{bitset}
%<uniquecounter>\hobsub@StartPackage{uniquecounter}
%<letltxmacro>\hobsub@StartPackage{letltxmacro}
%<hopatch>\hobsub@StartPackage{hopatch}
%<xcolor-patch>\hobsub@StartPackage{xcolor-patch}
%<atveryend>\hobsub@StartPackage{atveryend}
%<atbegshi>\hobsub@StartPackage{atbegshi}
%<refcount>\hobsub@StartPackage{refcount}
%<hycolor>\hobsub@StartPackage{hycolor}
%    \end{macrocode}
%
% \subsection{End of packages}
%
%    \begin{macrocode}
%<package|atend>\HOBsub@AtEnd%
%    \end{macrocode}
%
% \section{Test}
%
% \subsection{Catcode checks for loading}
%
%    \begin{macrocode}
%<*test1>
%    \end{macrocode}
%    \begin{macrocode}
\catcode`\{=1 %
\catcode`\}=2 %
\catcode`\#=6 %
\catcode`\@=11 %
\expandafter\ifx\csname count@\endcsname\relax
  \countdef\count@=255 %
\fi
\expandafter\ifx\csname @gobble\endcsname\relax
  \long\def\@gobble#1{}%
\fi
\expandafter\ifx\csname @firstofone\endcsname\relax
  \long\def\@firstofone#1{#1}%
\fi
\expandafter\ifx\csname loop\endcsname\relax
  \expandafter\@firstofone
\else
  \expandafter\@gobble
\fi
{%
  \def\loop#1\repeat{%
    \def\body{#1}%
    \iterate
  }%
  \def\iterate{%
    \body
      \let\next\iterate
    \else
      \let\next\relax
    \fi
    \next
  }%
  \let\repeat=\fi
}%
\def\RestoreCatcodes{}
\count@=0 %
\loop
  \edef\RestoreCatcodes{%
    \RestoreCatcodes
    \catcode\the\count@=\the\catcode\count@\relax
  }%
\ifnum\count@<255 %
  \advance\count@ 1 %
\repeat

\def\RangeCatcodeInvalid#1#2{%
  \count@=#1\relax
  \loop
    \catcode\count@=15 %
  \ifnum\count@<#2\relax
    \advance\count@ 1 %
  \repeat
}
\def\RangeCatcodeCheck#1#2#3{%
  \count@=#1\relax
  \loop
    \ifnum#3=\catcode\count@
    \else
      \errmessage{%
        Character \the\count@\space
        with wrong catcode \the\catcode\count@\space
        instead of \number#3%
      }%
    \fi
  \ifnum\count@<#2\relax
    \advance\count@ 1 %
  \repeat
}
\def\space{ }
\expandafter\ifx\csname LoadCommand\endcsname\relax
  \def\LoadCommand{\input hobsub.sty\relax}%
\fi
\def\Test{%
  \RangeCatcodeInvalid{0}{47}%
  \RangeCatcodeInvalid{58}{64}%
  \RangeCatcodeInvalid{91}{96}%
  \RangeCatcodeInvalid{123}{255}%
  \catcode`\@=12 %
  \catcode`\\=0 %
  \catcode`\%=14 %
  \LoadCommand
  \RangeCatcodeCheck{0}{36}{15}%
  \RangeCatcodeCheck{37}{37}{14}%
  \RangeCatcodeCheck{38}{47}{15}%
  \RangeCatcodeCheck{48}{57}{12}%
  \RangeCatcodeCheck{58}{63}{15}%
  \RangeCatcodeCheck{64}{64}{12}%
  \RangeCatcodeCheck{65}{90}{11}%
  \RangeCatcodeCheck{91}{91}{15}%
  \RangeCatcodeCheck{92}{92}{0}%
  \RangeCatcodeCheck{93}{96}{15}%
  \RangeCatcodeCheck{97}{122}{11}%
  \RangeCatcodeCheck{123}{255}{15}%
  \RestoreCatcodes
}
\Test
\csname @@end\endcsname
\end
%    \end{macrocode}
%    \begin{macrocode}
%</test1>
%    \end{macrocode}
%
% \section{Installation}
%
% \subsection{Download}
%
% \paragraph{Package.} This package is available on
% CTAN\footnote{\url{http://ctan.org/pkg/hobsub}}:
% \begin{description}
% \item[\CTAN{macros/latex/contrib/oberdiek/hobsub.dtx}] The source file.
% \item[\CTAN{macros/latex/contrib/oberdiek/hobsub.pdf}] Documentation.
% \end{description}
%
%
% \paragraph{Bundle.} All the packages of the bundle `oberdiek'
% are also available in a TDS compliant ZIP archive. There
% the packages are already unpacked and the documentation files
% are generated. The files and directories obey the TDS standard.
% \begin{description}
% \item[\CTAN{install/macros/latex/contrib/oberdiek.tds.zip}]
% \end{description}
% \emph{TDS} refers to the standard ``A Directory Structure
% for \TeX\ Files'' (\CTAN{tds/tds.pdf}). Directories
% with \xfile{texmf} in their name are usually organized this way.
%
% \subsection{Bundle installation}
%
% \paragraph{Unpacking.} Unpack the \xfile{oberdiek.tds.zip} in the
% TDS tree (also known as \xfile{texmf} tree) of your choice.
% Example (linux):
% \begin{quote}
%   |unzip oberdiek.tds.zip -d ~/texmf|
% \end{quote}
%
% \paragraph{Script installation.}
% Check the directory \xfile{TDS:scripts/oberdiek/} for
% scripts that need further installation steps.
% Package \xpackage{attachfile2} comes with the Perl script
% \xfile{pdfatfi.pl} that should be installed in such a way
% that it can be called as \texttt{pdfatfi}.
% Example (linux):
% \begin{quote}
%   |chmod +x scripts/oberdiek/pdfatfi.pl|\\
%   |cp scripts/oberdiek/pdfatfi.pl /usr/local/bin/|
% \end{quote}
%
% \subsection{Package installation}
%
% \paragraph{Unpacking.} The \xfile{.dtx} file is a self-extracting
% \docstrip\ archive. The files are extracted by running the
% \xfile{.dtx} through \plainTeX:
% \begin{quote}
%   \verb|tex hobsub.dtx|
% \end{quote}
%
% \paragraph{TDS.} Now the different files must be moved into
% the different directories in your installation TDS tree
% (also known as \xfile{texmf} tree):
% \begin{quote}
% \def\t{^^A
% \begin{tabular}{@{}>{\ttfamily}l@{ $\rightarrow$ }>{\ttfamily}l@{}}
%   hobsub.sty & tex/generic/oberdiek/hobsub.sty\\
%   hobsub-generic.sty & tex/generic/oberdiek/hobsub-generic.sty\\
%   hobsub-hyperref.sty & tex/generic/oberdiek/hobsub-hyperref.sty\\
%   hobsub.pdf & doc/latex/oberdiek/hobsub.pdf\\
%   test/hobsub-test1.tex & doc/latex/oberdiek/test/hobsub-test1.tex\\
%   hobsub.dtx & source/latex/oberdiek/hobsub.dtx\\
% \end{tabular}^^A
% }^^A
% \sbox0{\t}^^A
% \ifdim\wd0>\linewidth
%   \begingroup
%     \advance\linewidth by\leftmargin
%     \advance\linewidth by\rightmargin
%   \edef\x{\endgroup
%     \def\noexpand\lw{\the\linewidth}^^A
%   }\x
%   \def\lwbox{^^A
%     \leavevmode
%     \hbox to \linewidth{^^A
%       \kern-\leftmargin\relax
%       \hss
%       \usebox0
%       \hss
%       \kern-\rightmargin\relax
%     }^^A
%   }^^A
%   \ifdim\wd0>\lw
%     \sbox0{\small\t}^^A
%     \ifdim\wd0>\linewidth
%       \ifdim\wd0>\lw
%         \sbox0{\footnotesize\t}^^A
%         \ifdim\wd0>\linewidth
%           \ifdim\wd0>\lw
%             \sbox0{\scriptsize\t}^^A
%             \ifdim\wd0>\linewidth
%               \ifdim\wd0>\lw
%                 \sbox0{\tiny\t}^^A
%                 \ifdim\wd0>\linewidth
%                   \lwbox
%                 \else
%                   \usebox0
%                 \fi
%               \else
%                 \lwbox
%               \fi
%             \else
%               \usebox0
%             \fi
%           \else
%             \lwbox
%           \fi
%         \else
%           \usebox0
%         \fi
%       \else
%         \lwbox
%       \fi
%     \else
%       \usebox0
%     \fi
%   \else
%     \lwbox
%   \fi
% \else
%   \usebox0
% \fi
% \end{quote}
% If you have a \xfile{docstrip.cfg} that configures and enables \docstrip's
% TDS installing feature, then some files can already be in the right
% place, see the documentation of \docstrip.
%
% \subsection{Refresh file name databases}
%
% If your \TeX~distribution
% (\teTeX, \mikTeX, \dots) relies on file name databases, you must refresh
% these. For example, \teTeX\ users run \verb|texhash| or
% \verb|mktexlsr|.
%
% \subsection{Some details for the interested}
%
% \paragraph{Attached source.}
%
% The PDF documentation on CTAN also includes the
% \xfile{.dtx} source file. It can be extracted by
% AcrobatReader 6 or higher. Another option is \textsf{pdftk},
% e.g. unpack the file into the current directory:
% \begin{quote}
%   \verb|pdftk hobsub.pdf unpack_files output .|
% \end{quote}
%
% \paragraph{Unpacking with \LaTeX.}
% The \xfile{.dtx} chooses its action depending on the format:
% \begin{description}
% \item[\plainTeX:] Run \docstrip\ and extract the files.
% \item[\LaTeX:] Generate the documentation.
% \end{description}
% If you insist on using \LaTeX\ for \docstrip\ (really,
% \docstrip\ does not need \LaTeX), then inform the autodetect routine
% about your intention:
% \begin{quote}
%   \verb|latex \let\install=y\input{hobsub.dtx}|
% \end{quote}
% Do not forget to quote the argument according to the demands
% of your shell.
%
% \paragraph{Generating the documentation.}
% You can use both the \xfile{.dtx} or the \xfile{.drv} to generate
% the documentation. The process can be configured by the
% configuration file \xfile{ltxdoc.cfg}. For instance, put this
% line into this file, if you want to have A4 as paper format:
% \begin{quote}
%   \verb|\PassOptionsToClass{a4paper}{article}|
% \end{quote}
% An example follows how to generate the
% documentation with pdf\LaTeX:
% \begin{quote}
%\begin{verbatim}
%pdflatex hobsub.dtx
%makeindex -s gind.ist hobsub.idx
%pdflatex hobsub.dtx
%makeindex -s gind.ist hobsub.idx
%pdflatex hobsub.dtx
%\end{verbatim}
% \end{quote}
%
% \section{Catalogue}
%
% The following XML file can be used as source for the
% \href{http://mirror.ctan.org/help/Catalogue/catalogue.html}{\TeX\ Catalogue}.
% The elements \texttt{caption} and \texttt{description} are imported
% from the original XML file from the Catalogue.
% The name of the XML file in the Catalogue is \xfile{hobsub.xml}.
%    \begin{macrocode}
%<*catalogue>
<?xml version='1.0' encoding='us-ascii'?>
<!DOCTYPE entry SYSTEM 'catalogue.dtd'>
<entry datestamp='$Date$' modifier='$Author$' id='hobsub'>
  <name>hobsub</name>
  <caption>Construct package bundles</caption>
  <authorref id='auth:oberdiek'/>
  <copyright owner='Heiko Oberdiek' year='2011,2012'/>
  <license type='lppl1.3'/>
  <version number='1.14'/>
  <description>
    The package offers a means of loading a bundle of (other) packages
    &#x201C;at once&#x201D;, offering a means of reducing the amount of log output
    as the packages are loaded, and an aide-memoire for complicated
    requirements.  In some ways, loading a package via hobsub behaves
    much like loading the package in the normal way; for example,
    reloads are suppressed, as in LaTeX proper, and correct
    <tt>\listfiles</tt> output is produced.
    <p/>
    Examples provided are a &#x2018;generic&#x2019; bundle (comprising the author&#x2019;s
    packages that can be used with Plain TeX) and a &#x2018;hyperref&#x2019; bundle
    (comprising packages useful when generating hypertext output with
    LaTeX).
    <p/>
    The package is part of the <xref refid='oberdiek'>oberdiek</xref> bundle.
  </description>
  <documentation details='Package documentation'
      href='ctan:/macros/latex/contrib/oberdiek/hobsub.pdf'/>
  <ctan file='true' path='/macros/latex/contrib/oberdiek/hobsub.dtx'/>
  <miktex location='oberdiek'/>
  <texlive location='oberdiek'/>
  <install path='/macros/latex/contrib/oberdiek/oberdiek.tds.zip'/>
</entry>
%</catalogue>
%    \end{macrocode}
%
% \begin{thebibliography}{9}
% \raggedright
% \bibitem{lualatex:quiet}
% Will Robertson,
% \textit{[lltx] quietening lualatex console output};
% mailing list \textsf{lualatex-dev at tug.org},
% 2011-01-29;
% \url{http://tug.org/pipermail/lualatex-dev/2011-January/001031.html}.
%
% \end{thebibliography}
%
% \begin{History}
%   \begin{Version}{2011/01/30 v1.0}
%   \item
%     First public version.
%   \end{Version}
%   \begin{Version}{2011/04/16 v1.1}
%   \item
%     Package updates.
%   \end{Version}
%   \begin{Version}{2011/04/17 v1.2}
%   \item
%     White spaces at line begins are removed or reduced
%     in the generated collection packages.
%   \end{Version}
%   \begin{Version}{2011/04/18 v1.3}
%   \item
%     Package updates.
%   \end{Version}
%   \begin{Version}{2011/04/23 v1.4}
%   \item
%     Package updates.
%   \end{Version}
%   \begin{Version}{2011/06/24 v1.5}
%   \item
%     Package updates.
%   \end{Version}
%   \begin{Version}{2011/06/30 v1.6}
%   \item
%     Package updates.
%   \end{Version}
%   \begin{Version}{2011/07/01 v1.7}
%   \item
%     Package updates.
%   \end{Version}
%   \begin{Version}{2011/07/28 v1.8}
%   \item
%     Package updates.
%   \end{Version}
%   \begin{Version}{2011/08/22 v1.9}
%   \item
%     Package updates.
%   \end{Version}
%   \begin{Version}{2011/10/16 v1.10}
%   \item
%     Package updates.
%   \end{Version}
%   \begin{Version}{2011/11/29 v1.11}
%   \item
%     Package updates.
%   \end{Version}
%   \begin{Version}{2012/04/25 v1.12}
%   \item
%     Package updates.
%   \end{Version}
%   \begin{Version}{2012/05/28 v1.13}
%   \item
%     Package updates.
%   \end{Version}
%   \begin{Version}{2016/05/16 v1.14}
%   \item
%     Documentation updates.
%   \end{Version}
% \end{History}
%
% \PrintIndex
%
% \Finale
\endinput

%        (quote the arguments according to the demands of your shell)
%
% Documentation:
%    (a) If hobsub.drv is present:
%           latex hobsub.drv
%    (b) Without hobsub.drv:
%           latex hobsub.dtx; ...
%    The class ltxdoc loads the configuration file ltxdoc.cfg
%    if available. Here you can specify further options, e.g.
%    use A4 as paper format:
%       \PassOptionsToClass{a4paper}{article}
%
%    Programm calls to get the documentation (example):
%       pdflatex hobsub.dtx
%       makeindex -s gind.ist hobsub.idx
%       pdflatex hobsub.dtx
%       makeindex -s gind.ist hobsub.idx
%       pdflatex hobsub.dtx
%
% Installation:
%    TDS:tex/generic/oberdiek/hobsub.sty
%    TDS:tex/generic/oberdiek/hobsub-generic.sty
%    TDS:tex/generic/oberdiek/hobsub-hyperref.sty
%    TDS:doc/latex/oberdiek/hobsub.pdf
%    TDS:doc/latex/oberdiek/test/hobsub-test1.tex
%    TDS:source/latex/oberdiek/hobsub.dtx
%
%<*ignore>
\begingroup
  \catcode123=1 %
  \catcode125=2 %
  \def\x{LaTeX2e}%
\expandafter\endgroup
\ifcase 0\ifx\install y1\fi\expandafter
         \ifx\csname processbatchFile\endcsname\relax\else1\fi
         \ifx\fmtname\x\else 1\fi\relax
\else\csname fi\endcsname
%</ignore>
%<*install>
\input docstrip.tex
\Msg{************************************************************************}
\Msg{* Installation}
\Msg{* Package: hobsub 2016/05/16 v1.14 Construct package bundles (HO)}
\Msg{************************************************************************}

\keepsilent
\askforoverwritefalse

\let\MetaPrefix\relax
\preamble

This is a generated file.

Project: hobsub
Version: 2016/05/16 v1.14

Copyright (C) 2011, 2012 by
   Heiko Oberdiek <heiko.oberdiek at googlemail.com>

This work may be distributed and/or modified under the
conditions of the LaTeX Project Public License, either
version 1.3c of this license or (at your option) any later
version. This version of this license is in
   http://www.latex-project.org/lppl/lppl-1-3c.txt
and the latest version of this license is in
   http://www.latex-project.org/lppl.txt
and version 1.3 or later is part of all distributions of
LaTeX version 2005/12/01 or later.

This work has the LPPL maintenance status "maintained".

This Current Maintainer of this work is Heiko Oberdiek.

The Base Interpreter refers to any `TeX-Format',
because some files are installed in TDS:tex/generic//.

This work consists of the main source file hobsub.dtx
and the derived files
   hobsub.sty, hobsub.pdf, hobsub.ins, hobsub.drv, hobsub-generic.sty,
   hobsub-hyperref.sty, hobsub-test1.tex.

\endpreamble
\let\MetaPrefix\DoubleperCent

\generate{%
  \file{hobsub.ins}{\from{hobsub.dtx}{install}}%
  \file{hobsub.drv}{\from{hobsub.dtx}{driver}}%
  \usedir{tex/generic/oberdiek}%
  \file{hobsub.sty}{\from{hobsub.dtx}{package}}%
  \usedir{doc/latex/oberdiek/test}%
  \file{hobsub-test1.tex}{\from{hobsub.dtx}{test1}}%
  \nopreamble
  \nopostamble
  \usedir{source/latex/oberdiek/catalogue}%
  \file{hobsub.xml}{\from{hobsub.dtx}{catalogue}}%
}
\let\hobsubSavedNormalLine\normalLine
\def\hobsubGobble#1{}
\begingroup
  \lccode`\0=32 % space
  \lccode`\1=35 % hash
  \lccode`\2=37 % percent
  \lccode`\3=92 % backslash
  \lccode`\4=123 % {
  \lccode`\5=125 % }
\lowercase{\endgroup
  \let\hobsubTokenSpaceOther= 0%
  \let\hobsubTokenHash= 1%
  \let\hobsubTokenPercent= 2%
  \let\hobsubTokenBackslash= 3%
  \let\hobsubTokenBraceLeft= 4%
  \let\hobsubTokenBraceRight= 5%
  \def\hobsubNormalLineWithSpace{%
    \hobsubSavedNormalLine0%
  }%
}
\def\normalLine{%
  \futurelet\hobsubToken\hobsubFirst
}
\def\hobsubFirst{%
  \ifx\hobsubToken\hobsubTokenSpaceOther
    \expandafter\expandafter\expandafter\futurelet
    \expandafter\expandafter\expandafter\hobsubToken
    \expandafter\expandafter\expandafter\hobsubNext
    \expandafter\hobsubGobble
  \else
    \expandafter\hobsubSavedNormalLine
  \fi
}
\def\hobsubNext{%
  \ifx\hobsubToken\hobsubTokenSpaceOther
    \expandafter\expandafter\expandafter\futurelet
    \expandafter\expandafter\expandafter\hobsubToken
    \expandafter\expandafter\expandafter\hobsubNext 
    \expandafter\hobsubGobble
  \else
    \let\hobsubWithSpace=Y%
    \ifx\hobsubToken\hobsubTokenBackslash
      \let\hobsubWithSpace=N%
    \else
      \ifx\hobsubToken\hobsubTokenBraceRight
        \let\hobsubWithSpace=N%
      \else
        \ifx\hobsubToken\hobsubTokenBraceLeft
          \let\hobsubWithSpace=N%
        \else
          \ifx\hobsubToken\hobsubTokenPercent
            \let\hobsubWithSpace=N%
          \else
            \ifx\hobsubToken\hobsubTokenHash
              \let\hobsubWithSpace=N%
            \fi
          \fi
        \fi
      \fi
    \fi
    \ifx\hobsubWithSpace Y%
      \expandafter\expandafter\expandafter\hobsubNormalLineWithSpace
    \else
      \expandafter\expandafter\expandafter\hobsubSavedNormalLine
    \fi
  \fi
}
\generate{%
  \usedir{tex/generic/oberdiek}%
  \file{hobsub-generic.sty}{%
    \from{hobsub.dtx}{generic,preamble,infwarerr}%
    \from{infwarerr.dtx}{package}%
    \from{hobsub.dtx}{post,ltxcmds}%
    \from{ltxcmds.dtx}{package}%
    \from{hobsub.dtx}{post,ifluatex}%
    \from{ifluatex.dtx}{package}%
    \from{hobsub.dtx}{post,ifvtex}%
    \from{ifvtex.dtx}{package}%
    \from{hobsub.dtx}{post,intcalc}%
    \from{intcalc.dtx}{package}%
    \from{hobsub.dtx}{post,ifpdf}%
    \from{ifpdf.dtx}{package}%
    \from{hobsub.dtx}{post,etexcmds}%
    \from{etexcmds.dtx}{package}%
    \from{hobsub.dtx}{post,kvsetkeys}%
    \from{kvsetkeys.dtx}{package}%
    \from{hobsub.dtx}{post,kvdefinekeys}%
    \from{kvdefinekeys.dtx}{package}%
    \from{hobsub.dtx}{post,luatex-loader}%
    \from{luatex.dtx}{loader}%
    \from{hobsub.dtx}{post,pdftexcmds}%
    \from{pdftexcmds.dtx}{package}%
    \from{hobsub.dtx}{post,pdfescape}%
    \from{pdfescape.dtx}{package}%
    \from{hobsub.dtx}{post,bigintcalc}%
    \from{bigintcalc.dtx}{package}%
    \from{hobsub.dtx}{post,bitset}%
    \from{bitset.dtx}{package}%
    \from{hobsub.dtx}{post,uniquecounter}%
    \from{uniquecounter.dtx}{package}%
    \from{hobsub.dtx}{post,atend}%
  }%
}
\generate{%
  \usedir{tex/generic/oberdiek}%
  \file{hobsub-hyperref.sty}{%
    \from{hobsub.dtx}{hyperref,preamble,,letltxmacro}%
    \from{letltxmacro.dtx}{package}%
    \from{hobsub.dtx}{post,hopatch}%
    \from{hopatch.dtx}{package}%
    \from{hobsub.dtx}{post,xcolor-patch}%
    \from{hycolor.dtx}{xcolor}%
    \from{hobsub.dtx}{post,atveryend}%
    \from{atveryend.dtx}{package}%
    \from{hobsub.dtx}{post,atbegshi}%
    \from{atbegshi.dtx}{package}%
    \from{hobsub.dtx}{post,refcount}%
    \from{refcount.dtx}{package}%
    \from{hobsub.dtx}{post,hycolor}%
    \from{hycolor.dtx}{package}%
    \from{hobsub.dtx}{post}%
  }%
}
\let\normalLine\hobsubSavedNormalLine

\catcode32=13\relax% active space
\let =\space%
\Msg{************************************************************************}
\Msg{*}
\Msg{* To finish the installation you have to move the following}
\Msg{* files into a directory searched by TeX:}
\Msg{*}
\Msg{*     hobsub.sty, hobsub-generic.sty, hobsub-hyperref.sty}
\Msg{*}
\Msg{* To produce the documentation run the file `hobsub.drv'}
\Msg{* through LaTeX.}
\Msg{*}
\Msg{* Happy TeXing!}
\Msg{*}
\Msg{************************************************************************}

\endbatchfile
%</install>
%<*ignore>
\fi
%</ignore>
%<*driver>
\NeedsTeXFormat{LaTeX2e}
\ProvidesFile{hobsub.drv}%
  [2016/05/16 v1.14 Construct package bundles (HO)]%
\documentclass{ltxdoc}
\usepackage{holtxdoc}[2011/11/22]
\usepackage{array}
\usepackage{caption}
\renewcommand*{\bottomfraction}{.5}
\renewcommand*{\topfraction}{.5}
\begin{document}
  \DocInput{hobsub.dtx}%
\end{document}
%</driver>
% \fi
%
%
% \CharacterTable
%  {Upper-case    \A\B\C\D\E\F\G\H\I\J\K\L\M\N\O\P\Q\R\S\T\U\V\W\X\Y\Z
%   Lower-case    \a\b\c\d\e\f\g\h\i\j\k\l\m\n\o\p\q\r\s\t\u\v\w\x\y\z
%   Digits        \0\1\2\3\4\5\6\7\8\9
%   Exclamation   \!     Double quote  \"     Hash (number) \#
%   Dollar        \$     Percent       \%     Ampersand     \&
%   Acute accent  \'     Left paren    \(     Right paren   \)
%   Asterisk      \*     Plus          \+     Comma         \,
%   Minus         \-     Point         \.     Solidus       \/
%   Colon         \:     Semicolon     \;     Less than     \<
%   Equals        \=     Greater than  \>     Question mark \?
%   Commercial at \@     Left bracket  \[     Backslash     \\
%   Right bracket \]     Circumflex    \^     Underscore    \_
%   Grave accent  \`     Left brace    \{     Vertical bar  \|
%   Right brace   \}     Tilde         \~}
%
% \GetFileInfo{hobsub.drv}
%
% \title{The \xpackage{hobsub} package}
% \date{2016/05/16 v1.14}
% \author{Heiko Oberdiek\thanks
% {Please report any issues at https://github.com/ho-tex/oberdiek/issues}\\
% \xemail{heiko.oberdiek at googlemail.com}}
%
% \maketitle
%
% \begin{abstract}
% Package \xpackage{hobsub} implements the idea of loading
% several packages in one big collection package.
% \end{abstract}
%
% \tableofcontents
%
% \def\TableGeneric{^^A
%   \TableEntry{\xpackage{infwarerr}}{}{2010/04/08 v1.3}^^A
%   \TableEntry{\xpackage{ltxcmds}}{}{2011/11/09 v1.22}^^A
%   \TableEntry{\xpackage{ifluatex}}{}{2010/03/01 v1.3}^^A
%   \TableEntry{\xpackage{ifvtex}}{}{2010/03/01 v1.5}^^A
%   \TableEntry{\xpackage{intcalc}}{}{2007/09/27 v1.1}^^A
%   \TableEntry{\xpackage{ifpdf}}{\xpackage{ifluatex}}{2011/01/30 v2.3}^^A
%   \TableEntry{\xpackage{etexcmds}}{}{2011/02/16 v1.5}^^A
%   \TableEntry{\xpackage{kvsetkeys}}{}{2012/04/25 v1.16}^^A
%   \TableEntry{\xpackage{kvdefinekeys}}{}{2011/04/07 v1.3}^^A
%   \TableEntry{\xpackage{luatex-loader}}{}{2010/03/09 v0.4}^^A
%   \TableEntry{\xpackage{pdftexcmds}}{\xpackage{ifluatex}, \xpackage{ifpdf}, \xpackage{infwarerr}, \xpackage{ltxcmds}, \xpackage{luatex-loader}}{2011/11/29 v0.20}^^A
%   \TableEntry{\xpackage{pdfescape}}{\xpackage{ltxcmds}, \xpackage{pdftexcmds}}{2011/11/25 v1.13}^^A
%   \TableEntry{\xpackage{bigintcalc}}{\xpackage{pdftexcmds}}{2012/04/08 v1.3}^^A
%   \TableEntry{\xpackage{bitset}}{\xpackage{bigintcalc}, \xpackage{infwarerr}, \xpackage{intcalc}}{2011/01/30 v1.1}^^A
%   \TableEntry{\xpackage{uniquecounter}}{\xpackage{bigintcalc}, \xpackage{infwarerr}}{2011/01/30 v1.2}^^A
% }
% \def\TableHyperref{^^A
%   \TableEntry{\xpackage{letltxmacro}}{}{2010/09/02 v1.4}^^A
%   \TableEntry{\xpackage{hopatch}}{}{2016/05/16 v1.2}^^A
%   \TableEntry{\xpackage{xcolor-patch}}{}{2011/01/30 v1.7}^^A
%   \TableEntry{\xpackage{atveryend}}{}{2011/06/30 v1.8}^^A
%   \TableEntry{\xpackage{atbegshi}}{\xpackage{ifpdf}, \xpackage{infwarerr}, \xpackage{ltxcmds}}{2011/10/05 v1.16}^^A
%   \TableEntry{\xpackage{refcount}}{\xpackage{infwarerr}, \xpackage{ltxcmds}}{2011/10/16 v3.4}^^A
%   \TableEntry{\xpackage{hycolor}}{\xpackage{xcolor-patch}}{2011/01/30 v1.7}^^A
% }
%
% \newenvironment{TableList}[2][htbp]{^^A
%   \begin{table}[{#1}]^^A
%   \centering
%   \caption{Overview \xpackage{hobsub-#2}}^^A
%   \label{tab:#2}^^A
%   \begin{tabular}{ll>{\raggedright}p{.4\linewidth}}^^A
%   \bfseries Package & \bfseries Release &
%   \bfseries Dependencies\tabularnewline
%   \hline
% }{^^A
%   \end{tabular}^^A
%   \end{table}^^A
% }
% \newcommand*{\TableEntry}[3]{%
%   #1&#3&#2\tabularnewline
% }
%
% \renewcommand*{\arraystretch}{1.1}
% \setlength{\extrarowheight}{2pt}
%
% \section{Documentation}
%
% In January 2011 the mailing list lualatex-dev discussed
% ``quietening lualatex console output'' \cite{lualatex:quiet}.
% Inspired from this, I implemented this package \xpackage{hobsub}
% to load several packages in one file. The package is applied for
% two packages collections that are subsets of my package bundle
% \xpackage{oberdiek}.
% \begin{description}
% \item[\xpackage{hobsub-generic}] is a collection of packages
%   that can also be used with \hologo{plainTeX}.
% \item[\xpackage{hobsub-hyperref}] is a collection of packages
%   that is used by package \xpackage{hyperref}.
% \end{description}
%
% Package \xpackage{hobsub} provides the macros that
% are used in the collection packages and might be used
% in other projects as well. But there are many caveats,
% some of them are discussed below.
%
% The most serious problem is consistency. A package
% that is available as standalone package and is part
% of a collection might have different versions.
% Because my packages are organized in a bundle and should
% be installed and updated as bundle, the risk seems low
% to me. However, if a single package is updated manually, for
% example for testing, then it should be loaded before the
% collection that contains the package with an older version.
% In a collection, the package will not be loaded again if it
% is already loaded.
%
% \subsection{Usage}
%
% The collection packages \xpackage{hobsub-generic} and
% \xpackage{hobsub-hyperref} are loaded as usual:
% \begin{quote}
%  \begin{tabular}{@{}ll@{}}
%    |\usepackage{hobsub-generic}| & \hologo{LaTeX}\\
%    |\input hobsub-generic.sty| & \hologo{plainTeX}
%  \end{tabular}
% \end{quote}
% or
% \begin{quote}|\usepackage{hobsub-hyperref}|\end{quote}
% Of course, both \cs{usepackage} or \cs{RequirePackage}
% can be used.
%
% If you need requirements on the version date of a package
% inside a collection, then specify them afterwards, for
% example:
% \begin{quote}
%   |\usepackage{hobsub-generic}|\\
%   |\usepackage{ltxcmds}[2010/12/04]% |^^A
%   {\itshape |\ltx@ifblank is needed|}
% \end{quote}
%
% Also it is not garanteed that a package will be always
% part of a collection. Therefore it does not harm,
% to make the requirements explicite \emph{after}
% a collection is loaded, see the previous example.
% \hologo{LaTeX} knows that a
% package is already loaded and does not load it again.
%
% \subsection{Supported features}
%
% \begin{description}
% \item[No reloads:] If a package is already loaded then it
%  is not reloaded again if it is part of a collection package.
% \item[\cs{listfiles}:]
%   Each package that is loaded in a collection is also
%   added to the output of \cs{listfiles}.
% \item[\cs{@currname}, \cs{@currext}]
%   are updated, they are used by \hologo{LaTeX}'s
%  \cs{ProvidesPackage}.
% \item[\cs{AtEndOfPackage}:]
%   \hologo{LaTeX}'s hook mechanism that is used for
%   \cs{AtEndOfPackage} is supported. However none of the
%   packages in the collections are using \cs{AtEndOfPackage},
%   therefore this feature is completely untested.
% \end{description}
%
% \subsection{Common limitations}
%
% \begin{description}
% \item[\cs{endinput}:]
%   Often a package is using \cs{endinput}, either added by
%   \xpackage{docstrip}, or to stop package loading at an earlier
%   time. Therefore \cs{endinput} must be catched during package
%   loading to prevent that the collection package is closed
%   too early before reading all packages it contains.
%
%   This redefinition of \cs{endinput} fails if the package loads
%   other files that contain \cs{endinput}. To some degree it could
%   be catched by redefining macros that load files. But this will
%   not work for the primitive \cs{input}.
% \item[Options:]
%   \hologo{LaTeX}'s package options are not supported.
%   Therefore the packages that are put in a collection must
%   not support options.
% \item[Hook packages:]
%   Some packages and classes provides hooks for packages, see
%   packages \xpackage{scrlfile}, \xpackage{filehook}
%   or class \xclass{memoir}. These hooks are not supported
%   for embedded packages inside a collection package.
%   Because of the problem with \cs{endinput}, these hooks
%   are not supported. The hooks might contain code that
%   loads other files that execute \cs{endinput}.
% \end{description}
%
% \subsection{Package \xpackage{hobsub-generic}}
%
% \begin{TableList}[bp]{generic}\TableGeneric\end{TableList}
%
% Collection `generic' contains some of the \hologo{plainTeX} compatible
% packages of my bundle, see table \ref{tab:generic}.
% It adds the following restrictions to any package
% that is part of this collection:
% \begin{description}
% \item[Generic formats:] The following formats must be supported:
%   \hologo{LaTeX}, \hologo{plainTeX} and even \hologo{iniTeX}.
%   This applies to the collection package \xpackage{hobsub-generic}
%   the same way.
% \item[Prefixed macros:]
%   Any package of this collection should only define macros with
%   prefixed names to avoid name clashes. Resources like registers
%   must not be allocated and other global states should not be
%   changed except for added macro definitions.
% \end{description}
% Thus the package \xpackage{hobsub-generic} can also preloaded
% in the format file with \hologo{iniTeX}.
%
% \subsection{Package \xpackage{hobsub-hyperref}}
%
% \begin{TableList}[tp]{hyperref}\TableHyperref\end{TableList}
%
% Collection `hyperref' contains additional packages that are
% used by package \xpackage{hyperref} and that does not fit
% in collection `generic', see table{tab:hyperref}.
% The latter collection package
% \xpackage{hobsub-generic} is loaded via a conventional
% \cs{RequirePackage}.
%
% Because most of the packages in the collection `hyperref'
% needs \hologo{LaTeX}, the collection requires \hologo{LaTeX}.
%
% \subsection{Package \xpackage{hobsub}}
%
% This section is for advanced users that might want to
% build a collection package. Then it can be done by
% using \xpackage{hobsub}:
% \begin{quote}
%   \dots\\
%   |\RequirePackage{hobsub}[2016/05/16]|\\
%   \dots\\
%   |\hobsub@StartPackage{foo1}|\\
%   \textit{\dots\unskip\ package code of \xfile{foo1.sty} \dots}\\
%   |\endinput|\\
%   |<hobsub>|\\
%   |\hobsub@StopPackage|\\
%   |\hobsub@StartPackage{bar2}|\\
%   \textit{\dots\unskip\ package code of \xfile{bar2.sty} \dots}\\
%   |\endinput|\\
%   |<hobsub>|\\
%   |\hobsub@StopPackage|
%   \dots
% \end{quote}
% But be aware of the caveats and limitations, see above.
% Only quite a small portion of the packages can be embedded
% like this.
%
% \clearpage
%
% \StopEventually{
% }
%
% \section{Implementation}
%
% \subsection{Reload check and package identification}
%
% \subsubsection{Package \xpackage{hobsub}}
%
%    \begin{macrocode}
%<*package>
%    \end{macrocode}
%    Reload check, especially if the package is not used with \LaTeX.
%    \begin{macrocode}
\begingroup\catcode61\catcode48\catcode32=10\relax%
  \catcode13=5 % ^^M
  \endlinechar=13 %
  \catcode35=6 % #
  \catcode39=12 % '
  \catcode44=12 % ,
  \catcode45=12 % -
  \catcode46=12 % .
  \catcode58=12 % :
  \catcode64=11 % @
  \catcode123=1 % {
  \catcode125=2 % }
  \expandafter\let\expandafter\x\csname ver@hobsub.sty\endcsname
  \ifx\x\relax % plain-TeX, first loading
  \else
    \def\empty{}%
    \ifx\x\empty % LaTeX, first loading,
      % variable is initialized, but \ProvidesPackage not yet seen
    \else
      \expandafter\ifx\csname PackageInfo\endcsname\relax
        \def\x#1#2{%
          \immediate\write-1{Package #1 Info: #2.}%
        }%
      \else
        \def\x#1#2{\PackageInfo{#1}{#2, stopped}}%
      \fi
      \x{hobsub}{The package is already loaded}%
      \aftergroup\endinput
    \fi
  \fi
\endgroup%
%    \end{macrocode}
%    Package identification:
%    \begin{macrocode}
\begingroup\catcode61\catcode48\catcode32=10\relax%
  \catcode13=5 % ^^M
  \endlinechar=13 %
  \catcode35=6 % #
  \catcode39=12 % '
  \catcode40=12 % (
  \catcode41=12 % )
  \catcode44=12 % ,
  \catcode45=12 % -
  \catcode46=12 % .
  \catcode47=12 % /
  \catcode58=12 % :
  \catcode64=11 % @
  \catcode91=12 % [
  \catcode93=12 % ]
  \catcode123=1 % {
  \catcode125=2 % }
  \expandafter\ifx\csname ProvidesPackage\endcsname\relax
    \def\x#1#2#3[#4]{\endgroup
      \immediate\write-1{Package: #3 #4}%
      \xdef#1{#4}%
    }%
  \else
    \def\x#1#2[#3]{\endgroup
      #2[{#3}]%
      \ifx#1\@undefined
        \xdef#1{#3}%
      \fi
      \ifx#1\relax
        \xdef#1{#3}%
      \fi
    }%
  \fi
\expandafter\x\csname ver@hobsub.sty\endcsname
\ProvidesPackage{hobsub}%
  [2016/05/16 v1.14 Construct package bundles (HO)]%
%    \end{macrocode}
%    \begin{macrocode}
%</package>
%    \end{macrocode}
%
% \subsubsection{Package \xpackage{hobsub-generic}}
%
%    \begin{macrocode}
%<*generic>
%    \end{macrocode}
%    Reload check, especially if the package is not used with \LaTeX.
%    \begin{macrocode}
\begingroup\catcode61\catcode48\catcode32=10\relax%
  \catcode13=5 % ^^M
  \endlinechar=13 %
  \catcode35=6 % #
  \catcode39=12 % '
  \catcode44=12 % ,
  \catcode45=12 % -
  \catcode46=12 % .
  \catcode58=12 % :
  \catcode64=11 % @
  \catcode123=1 % {
  \catcode125=2 % }
  \expandafter\let\expandafter\x\csname ver@hobsub-generic.sty\endcsname
  \ifx\x\relax % plain-TeX, first loading
  \else
    \def\empty{}%
    \ifx\x\empty % LaTeX, first loading,
      % variable is initialized, but \ProvidesPackage not yet seen
    \else
      \expandafter\ifx\csname PackageInfo\endcsname\relax
        \def\x#1#2{%
          \immediate\write-1{Package #1 Info: #2.}%
        }%
      \else
        \def\x#1#2{\PackageInfo{#1}{#2, stopped}}%
      \fi
      \x{hobsub-generic}{The package is already loaded}%
      \aftergroup\endinput
    \fi
  \fi
\endgroup%
%    \end{macrocode}
%    Package identification:
%    \begin{macrocode}
\begingroup\catcode61\catcode48\catcode32=10\relax%
  \catcode13=5 % ^^M
  \endlinechar=13 %
  \catcode35=6 % #
  \catcode39=12 % '
  \catcode40=12 % (
  \catcode41=12 % )
  \catcode44=12 % ,
  \catcode45=12 % -
  \catcode46=12 % .
  \catcode47=12 % /
  \catcode58=12 % :
  \catcode64=11 % @
  \catcode91=12 % [
  \catcode93=12 % ]
  \catcode123=1 % {
  \catcode125=2 % }
  \expandafter\ifx\csname ProvidesPackage\endcsname\relax
    \def\x#1#2#3[#4]{\endgroup
      \immediate\write-1{Package: #3 #4}%
      \xdef#1{#4}%
    }%
  \else
    \def\x#1#2[#3]{\endgroup
      #2[{#3}]%
      \ifx#1\@undefined
        \xdef#1{#3}%
      \fi
      \ifx#1\relax
        \xdef#1{#3}%
      \fi
    }%
  \fi
\expandafter\x\csname ver@hobsub-generic.sty\endcsname
\ProvidesPackage{hobsub-generic}%
  [2016/05/16 v1.14 Bundle oberdiek, subset generic (HO)]%
%    \end{macrocode}
%    \begin{macrocode}
%</generic>
%    \end{macrocode}
%
% \subsubsection{Package \xpackage{hobsub-hyperref}}
%
%    \begin{macrocode}
%<*hyperref>
%    \end{macrocode}
%    Reload check, especially if the package is not used with \LaTeX.
%    \begin{macrocode}
\begingroup\catcode61\catcode48\catcode32=10\relax%
  \catcode13=5 % ^^M
  \endlinechar=13 %
  \catcode35=6 % #
  \catcode39=12 % '
  \catcode44=12 % ,
  \catcode45=12 % -
  \catcode46=12 % .
  \catcode58=12 % :
  \catcode64=11 % @
  \catcode123=1 % {
  \catcode125=2 % }
  \expandafter\let\expandafter\x\csname ver@hobsub-hyperref.sty\endcsname
  \ifx\x\relax % plain-TeX, first loading
  \else
    \def\empty{}%
    \ifx\x\empty % LaTeX, first loading,
      % variable is initialized, but \ProvidesPackage not yet seen
    \else
      \expandafter\ifx\csname PackageInfo\endcsname\relax
        \def\x#1#2{%
          \immediate\write-1{Package #1 Info: #2.}%
        }%
      \else
        \def\x#1#2{\PackageInfo{#1}{#2, stopped}}%
      \fi
      \x{hobsub-hyperref}{The package is already loaded}%
      \aftergroup\endinput
    \fi
  \fi
\endgroup%
%    \end{macrocode}
%    Package identification:
%    \begin{macrocode}
\begingroup\catcode61\catcode48\catcode32=10\relax%
  \catcode13=5 % ^^M
  \endlinechar=13 %
  \catcode35=6 % #
  \catcode39=12 % '
  \catcode40=12 % (
  \catcode41=12 % )
  \catcode44=12 % ,
  \catcode45=12 % -
  \catcode46=12 % .
  \catcode47=12 % /
  \catcode58=12 % :
  \catcode64=11 % @
  \catcode91=12 % [
  \catcode93=12 % ]
  \catcode123=1 % {
  \catcode125=2 % }
  \expandafter\ifx\csname ProvidesPackage\endcsname\relax
    \def\x#1#2#3[#4]{\endgroup
      \immediate\write-1{Package: #3 #4}%
      \xdef#1{#4}%
    }%
  \else
    \def\x#1#2[#3]{\endgroup
      #2[{#3}]%
      \ifx#1\@undefined
        \xdef#1{#3}%
      \fi
      \ifx#1\relax
        \xdef#1{#3}%
      \fi
    }%
  \fi
\expandafter\x\csname ver@hobsub-hyperref.sty\endcsname
\ProvidesPackage{hobsub-hyperref}%
  [2016/05/16 v1.14 Bundle oberdiek, subset hyperref (HO)]%
%    \end{macrocode}
%    \begin{macrocode}
%</hyperref>
%    \end{macrocode}
%
% \subsection{Catcodes}
%
%    \begin{macrocode}
%<*package|generic|hyperref>
\begingroup\catcode61\catcode48\catcode32=10\relax%
  \catcode13=5 % ^^M
  \endlinechar=13 %
  \catcode123=1 % {
  \catcode125=2 % }
  \catcode64=11 % @
  \def\x{\endgroup
    \expandafter\edef\csname HOBsub@AtEnd\endcsname{%
      \endlinechar=\the\endlinechar\relax
      \catcode13=\the\catcode13\relax
      \catcode32=\the\catcode32\relax
      \catcode35=\the\catcode35\relax
      \catcode61=\the\catcode61\relax
      \catcode64=\the\catcode64\relax
      \catcode123=\the\catcode123\relax
      \catcode125=\the\catcode125\relax
    }%
  }%
\x\catcode61\catcode48\catcode32=10\relax%
\catcode13=5 % ^^M
\endlinechar=13 %
\catcode35=6 % #
\catcode64=11 % @
\catcode123=1 % {
\catcode125=2 % }
\def\TMP@EnsureCode#1#2{%
  \edef\HOBsub@AtEnd{%
    \HOBsub@AtEnd
    \catcode#1=\the\catcode#1\relax
  }%
  \catcode#1=#2\relax
}
\TMP@EnsureCode{39}{12}% '
\TMP@EnsureCode{40}{12}% (
\TMP@EnsureCode{41}{12}% )
\TMP@EnsureCode{45}{12}% -
\TMP@EnsureCode{46}{12}% .
\TMP@EnsureCode{47}{12}% /
\TMP@EnsureCode{58}{12}% :
\TMP@EnsureCode{60}{12}% <
\TMP@EnsureCode{62}{12}% >
\TMP@EnsureCode{96}{12}% `
\edef\HOBsub@AtEnd{\HOBsub@AtEnd\noexpand\endinput}
%    \end{macrocode}
%
%    \begin{macrocode}
%</package|generic|hyperref>
%    \end{macrocode}
%
% \subsection{Package \xpackage{hobsub-hyperref} loads \xpackage{hobsub-generic}}
%
%    \begin{macrocode}
%<*hyperref>
\NeedsTeXFormat{LaTeX2e}
\RequirePackage{hobsub-generic}[2016/05/16]
%</hyperref>
%    \end{macrocode}
%
% \subsection{Preamble}
%
%    \begin{macrocode}
%<*preamble>
\begingroup\expandafter\expandafter\expandafter\endgroup
\expandafter\ifx\csname ver@hobsub.sty\endcsname\relax
\else
  \expandafter\hobsub@GobbleRemainingPackage
\fi
%</preamble>
%    \end{macrocode}
%
%
%    \begin{macrocode}
%<*package|preamble>
%    \end{macrocode}
%
%    \begin{macro}{\HOBsub@OrgEndinput}
%    \begin{macrocode}
\begingroup\expandafter\expandafter\expandafter\endgroup
\expandafter\ifx\csname HOBsub@OrgEndinput\endcsname\relax
  \let\HOBsub@OrgEndinput\endinput
\fi
%    \end{macrocode}
%    \end{macro}
%    \begin{macro}{\hobsub@GobbleRemainingPackage}
%    \begin{macrocode}
\def\hobsub@GobbleRemainingPackage{%
  \begingroup
  \catcode92=14 % backslash: comment
  \catcode32=14 % space: comment
  \catcode35=14 % hash: comment
  \catcode123=14 % left brace: comment
  \catcode125=14 % right brace: comment
  \catcode60=3 % less: math
  \catcode62=4 % greater: align
  \endlinechar=-1 %
  \HOBsub@GobbleRemainingPackage
}
%    \end{macrocode}
%    \end{macro}
%    \begin{macro}{\HOBsub@GobbleRemainingPackage}
%    \begin{macrocode}
\catcode60=3 % less: dollar
\catcode62=4 % greater: align
\long\def\HOBsub@GobbleRemainingPackage#1<hobsub>{%
  \endgroup
}
\catcode60=12 % less: other
\catcode62=12 % greater: other
%    \end{macrocode}
%    \end{macro}
%
%    \begin{macro}{\hobsub@StartPackage}
%    \begin{macrocode}
\def\hobsub@StartPackage#1{%
  \begingroup\expandafter\expandafter\expandafter\endgroup
  \expandafter\ifx\csname ver@#1.sty\endcsname\relax
    \let\HOBsub@OrgCurrName\@currname
    \let\HOBsub@OrgCurrExt\@currext
    \csname @pushfilename\endcsname
    \def\@currname{#1}%
    \def\@currext{sty}%
    \expandafter\def\csname\@currname.\@currext-h@@k\endcsname{}%
    \let\endinput\hobsub@GobbleRemainingPackage
    \def\hobsub@StopPackage{%
      \let\hobsub@StopPackage\relax
      \HOBsub@StopPackage
%     \hobsub@Info{hobsub}{Package `#1' loaded}%
    }%
    \hobsub@AddToFileList{#1.sty}%
  \else
    \hobsub@Info{hobsub}{Skipping package `#1' (already loaded)}%
    \let\hobsub@StopPackage\relax
    \expandafter\hobsub@GobbleRemainingPackage
  \fi
}
%    \end{macrocode}
%    \end{macro}
%
%    \begin{macro}{\hobsub@StopPackage}
%    \begin{macrocode}
\let\hobsub@StopPackage\relax
%    \end{macrocode}
%    \end{macro}
%
%    \begin{macro}{\hobsub@Info}
%    \begin{macrocode}
\def\hobsub@Info#1#2{%
  \begingroup\expandafter\expandafter\expandafter\endgroup
  \expandafter\ifx\csname @PackageInfoNoLine\endcsname\relax
    \immediate\write-1{Package #1 Info: #2.}%
  \else
    \let\hobsub@Info\@PackageInfoNoLine
    \hobsub@Info{#1}{#2}%
  \fi
}
%    \end{macrocode}
%    \end{macro}
%
%    \begin{macro}{\HOBsub@StopPackage}
%    \begin{macrocode}
\def\HOBsub@StopPackage{%
  \csname\@currname.\@currext-h@@k\endcsname
  \let\endinput\HOBsub@OrgEndinput
  \csname @popfilename\endcsname
  \let\@currname\HOBsub@OrgCurrName
  \let\@currext\HOBsub@OrgCurrExt
}
%    \end{macrocode}
%    \end{macro}
%
%    \begin{macro}{\hobsub@AddToFileList}
%    \begin{macrocode}
\begingroup\expandafter\expandafter\expandafter\endgroup
\expandafter\ifx\csname @addtofilelist\endcsname\relax
  \def\hobsub@AddToFileList#1{}%
\else
  \def\hobsub@AddToFileList#1{%
    \@addtofilelist{#1}%
  }%
\fi
%    \end{macrocode}
%    \end{macro}
%
%    \begin{macrocode}
%</package|preamble>
%    \end{macrocode}
%
% \subsection{End of preamble for subset packages}
%
%    Add the end marker for possible skipping of previous
%    macros, if package \xpackage{hobsub} was already loaded.
%    \begin{macrocode}
%<*preamble>
\hobsub@GobbleRemainingPackage
<hobsub>
%    \end{macrocode}
%
%    All macros of package \xpackage{hobsub} are now available.
%    Now we officially load the package \xpackage{hobsub} inline
%    for proper package identification.
%    \begin{macrocode}
\hobsub@StartPackage{hobsub}%
\begingroup\expandafter\expandafter\expandafter\endgroup
\expandafter\ifx\csname ProvidesPackage\endcsname\relax
  \immediate\write-1{%
    Package: hobsub 2016/05/16 v1.14 Construct package bundles (HO)%
  }%
\else
  \ProvidesPackage{hobsub}%
  [2016/05/16 v1.14 Construct package bundles (HO)]%
\fi
%</preamble>
%    \end{macrocode}
%
%
%    \begin{macrocode}
%<*post|preamble>
 \endinput
<hobsub>
\hobsub@StopPackage
%</post|preamble>
%    \end{macrocode}
%
% \subsection{Package list}
%
%    \begin{macrocode}
%<infwarerr>\hobsub@StartPackage{infwarerr}
%<ltxcmds>\hobsub@StartPackage{ltxcmds}
%<ifluatex>\hobsub@StartPackage{ifluatex}
%<ifvtex>\hobsub@StartPackage{ifvtex}
%<intcalc>\hobsub@StartPackage{intcalc}
%<ifpdf>\hobsub@StartPackage{ifpdf}
%<etexcmds>\hobsub@StartPackage{etexcmds}
%<kvsetkeys>\hobsub@StartPackage{kvsetkeys}
%<kvdefinekeys>\hobsub@StartPackage{kvdefinekeys}
%<*luatex-loader>
\ifluatex
\else
  \expandafter\hobsub@GobbleRemainingPackage
\fi
\hobsub@StartPackage{luatex-loader}
%</luatex-loader>
%<pdftexcmds>\hobsub@StartPackage{pdftexcmds}
%<pdfescape>\hobsub@StartPackage{pdfescape}
%<bigintcalc>\hobsub@StartPackage{bigintcalc}
%<bitset>\hobsub@StartPackage{bitset}
%<uniquecounter>\hobsub@StartPackage{uniquecounter}
%<letltxmacro>\hobsub@StartPackage{letltxmacro}
%<hopatch>\hobsub@StartPackage{hopatch}
%<xcolor-patch>\hobsub@StartPackage{xcolor-patch}
%<atveryend>\hobsub@StartPackage{atveryend}
%<atbegshi>\hobsub@StartPackage{atbegshi}
%<refcount>\hobsub@StartPackage{refcount}
%<hycolor>\hobsub@StartPackage{hycolor}
%    \end{macrocode}
%
% \subsection{End of packages}
%
%    \begin{macrocode}
%<package|atend>\HOBsub@AtEnd%
%    \end{macrocode}
%
% \section{Test}
%
% \subsection{Catcode checks for loading}
%
%    \begin{macrocode}
%<*test1>
%    \end{macrocode}
%    \begin{macrocode}
\catcode`\{=1 %
\catcode`\}=2 %
\catcode`\#=6 %
\catcode`\@=11 %
\expandafter\ifx\csname count@\endcsname\relax
  \countdef\count@=255 %
\fi
\expandafter\ifx\csname @gobble\endcsname\relax
  \long\def\@gobble#1{}%
\fi
\expandafter\ifx\csname @firstofone\endcsname\relax
  \long\def\@firstofone#1{#1}%
\fi
\expandafter\ifx\csname loop\endcsname\relax
  \expandafter\@firstofone
\else
  \expandafter\@gobble
\fi
{%
  \def\loop#1\repeat{%
    \def\body{#1}%
    \iterate
  }%
  \def\iterate{%
    \body
      \let\next\iterate
    \else
      \let\next\relax
    \fi
    \next
  }%
  \let\repeat=\fi
}%
\def\RestoreCatcodes{}
\count@=0 %
\loop
  \edef\RestoreCatcodes{%
    \RestoreCatcodes
    \catcode\the\count@=\the\catcode\count@\relax
  }%
\ifnum\count@<255 %
  \advance\count@ 1 %
\repeat

\def\RangeCatcodeInvalid#1#2{%
  \count@=#1\relax
  \loop
    \catcode\count@=15 %
  \ifnum\count@<#2\relax
    \advance\count@ 1 %
  \repeat
}
\def\RangeCatcodeCheck#1#2#3{%
  \count@=#1\relax
  \loop
    \ifnum#3=\catcode\count@
    \else
      \errmessage{%
        Character \the\count@\space
        with wrong catcode \the\catcode\count@\space
        instead of \number#3%
      }%
    \fi
  \ifnum\count@<#2\relax
    \advance\count@ 1 %
  \repeat
}
\def\space{ }
\expandafter\ifx\csname LoadCommand\endcsname\relax
  \def\LoadCommand{\input hobsub.sty\relax}%
\fi
\def\Test{%
  \RangeCatcodeInvalid{0}{47}%
  \RangeCatcodeInvalid{58}{64}%
  \RangeCatcodeInvalid{91}{96}%
  \RangeCatcodeInvalid{123}{255}%
  \catcode`\@=12 %
  \catcode`\\=0 %
  \catcode`\%=14 %
  \LoadCommand
  \RangeCatcodeCheck{0}{36}{15}%
  \RangeCatcodeCheck{37}{37}{14}%
  \RangeCatcodeCheck{38}{47}{15}%
  \RangeCatcodeCheck{48}{57}{12}%
  \RangeCatcodeCheck{58}{63}{15}%
  \RangeCatcodeCheck{64}{64}{12}%
  \RangeCatcodeCheck{65}{90}{11}%
  \RangeCatcodeCheck{91}{91}{15}%
  \RangeCatcodeCheck{92}{92}{0}%
  \RangeCatcodeCheck{93}{96}{15}%
  \RangeCatcodeCheck{97}{122}{11}%
  \RangeCatcodeCheck{123}{255}{15}%
  \RestoreCatcodes
}
\Test
\csname @@end\endcsname
\end
%    \end{macrocode}
%    \begin{macrocode}
%</test1>
%    \end{macrocode}
%
% \section{Installation}
%
% \subsection{Download}
%
% \paragraph{Package.} This package is available on
% CTAN\footnote{\url{http://ctan.org/pkg/hobsub}}:
% \begin{description}
% \item[\CTAN{macros/latex/contrib/oberdiek/hobsub.dtx}] The source file.
% \item[\CTAN{macros/latex/contrib/oberdiek/hobsub.pdf}] Documentation.
% \end{description}
%
%
% \paragraph{Bundle.} All the packages of the bundle `oberdiek'
% are also available in a TDS compliant ZIP archive. There
% the packages are already unpacked and the documentation files
% are generated. The files and directories obey the TDS standard.
% \begin{description}
% \item[\CTAN{install/macros/latex/contrib/oberdiek.tds.zip}]
% \end{description}
% \emph{TDS} refers to the standard ``A Directory Structure
% for \TeX\ Files'' (\CTAN{tds/tds.pdf}). Directories
% with \xfile{texmf} in their name are usually organized this way.
%
% \subsection{Bundle installation}
%
% \paragraph{Unpacking.} Unpack the \xfile{oberdiek.tds.zip} in the
% TDS tree (also known as \xfile{texmf} tree) of your choice.
% Example (linux):
% \begin{quote}
%   |unzip oberdiek.tds.zip -d ~/texmf|
% \end{quote}
%
% \paragraph{Script installation.}
% Check the directory \xfile{TDS:scripts/oberdiek/} for
% scripts that need further installation steps.
% Package \xpackage{attachfile2} comes with the Perl script
% \xfile{pdfatfi.pl} that should be installed in such a way
% that it can be called as \texttt{pdfatfi}.
% Example (linux):
% \begin{quote}
%   |chmod +x scripts/oberdiek/pdfatfi.pl|\\
%   |cp scripts/oberdiek/pdfatfi.pl /usr/local/bin/|
% \end{quote}
%
% \subsection{Package installation}
%
% \paragraph{Unpacking.} The \xfile{.dtx} file is a self-extracting
% \docstrip\ archive. The files are extracted by running the
% \xfile{.dtx} through \plainTeX:
% \begin{quote}
%   \verb|tex hobsub.dtx|
% \end{quote}
%
% \paragraph{TDS.} Now the different files must be moved into
% the different directories in your installation TDS tree
% (also known as \xfile{texmf} tree):
% \begin{quote}
% \def\t{^^A
% \begin{tabular}{@{}>{\ttfamily}l@{ $\rightarrow$ }>{\ttfamily}l@{}}
%   hobsub.sty & tex/generic/oberdiek/hobsub.sty\\
%   hobsub-generic.sty & tex/generic/oberdiek/hobsub-generic.sty\\
%   hobsub-hyperref.sty & tex/generic/oberdiek/hobsub-hyperref.sty\\
%   hobsub.pdf & doc/latex/oberdiek/hobsub.pdf\\
%   test/hobsub-test1.tex & doc/latex/oberdiek/test/hobsub-test1.tex\\
%   hobsub.dtx & source/latex/oberdiek/hobsub.dtx\\
% \end{tabular}^^A
% }^^A
% \sbox0{\t}^^A
% \ifdim\wd0>\linewidth
%   \begingroup
%     \advance\linewidth by\leftmargin
%     \advance\linewidth by\rightmargin
%   \edef\x{\endgroup
%     \def\noexpand\lw{\the\linewidth}^^A
%   }\x
%   \def\lwbox{^^A
%     \leavevmode
%     \hbox to \linewidth{^^A
%       \kern-\leftmargin\relax
%       \hss
%       \usebox0
%       \hss
%       \kern-\rightmargin\relax
%     }^^A
%   }^^A
%   \ifdim\wd0>\lw
%     \sbox0{\small\t}^^A
%     \ifdim\wd0>\linewidth
%       \ifdim\wd0>\lw
%         \sbox0{\footnotesize\t}^^A
%         \ifdim\wd0>\linewidth
%           \ifdim\wd0>\lw
%             \sbox0{\scriptsize\t}^^A
%             \ifdim\wd0>\linewidth
%               \ifdim\wd0>\lw
%                 \sbox0{\tiny\t}^^A
%                 \ifdim\wd0>\linewidth
%                   \lwbox
%                 \else
%                   \usebox0
%                 \fi
%               \else
%                 \lwbox
%               \fi
%             \else
%               \usebox0
%             \fi
%           \else
%             \lwbox
%           \fi
%         \else
%           \usebox0
%         \fi
%       \else
%         \lwbox
%       \fi
%     \else
%       \usebox0
%     \fi
%   \else
%     \lwbox
%   \fi
% \else
%   \usebox0
% \fi
% \end{quote}
% If you have a \xfile{docstrip.cfg} that configures and enables \docstrip's
% TDS installing feature, then some files can already be in the right
% place, see the documentation of \docstrip.
%
% \subsection{Refresh file name databases}
%
% If your \TeX~distribution
% (\teTeX, \mikTeX, \dots) relies on file name databases, you must refresh
% these. For example, \teTeX\ users run \verb|texhash| or
% \verb|mktexlsr|.
%
% \subsection{Some details for the interested}
%
% \paragraph{Attached source.}
%
% The PDF documentation on CTAN also includes the
% \xfile{.dtx} source file. It can be extracted by
% AcrobatReader 6 or higher. Another option is \textsf{pdftk},
% e.g. unpack the file into the current directory:
% \begin{quote}
%   \verb|pdftk hobsub.pdf unpack_files output .|
% \end{quote}
%
% \paragraph{Unpacking with \LaTeX.}
% The \xfile{.dtx} chooses its action depending on the format:
% \begin{description}
% \item[\plainTeX:] Run \docstrip\ and extract the files.
% \item[\LaTeX:] Generate the documentation.
% \end{description}
% If you insist on using \LaTeX\ for \docstrip\ (really,
% \docstrip\ does not need \LaTeX), then inform the autodetect routine
% about your intention:
% \begin{quote}
%   \verb|latex \let\install=y% \iffalse meta-comment
%
% File: hobsub.dtx
% Version: 2016/05/16 v1.14
% Info: Construct package bundles
%
% Copyright (C) 2011, 2012 by
%    Heiko Oberdiek <heiko.oberdiek at googlemail.com>
%    2016
%    https://github.com/ho-tex/oberdiek/issues
%
% This work may be distributed and/or modified under the
% conditions of the LaTeX Project Public License, either
% version 1.3c of this license or (at your option) any later
% version. This version of this license is in
%    http://www.latex-project.org/lppl/lppl-1-3c.txt
% and the latest version of this license is in
%    http://www.latex-project.org/lppl.txt
% and version 1.3 or later is part of all distributions of
% LaTeX version 2005/12/01 or later.
%
% This work has the LPPL maintenance status "maintained".
%
% This Current Maintainer of this work is Heiko Oberdiek.
%
% The Base Interpreter refers to any `TeX-Format',
% because some files are installed in TDS:tex/generic//.
%
% This work consists of the main source file hobsub.dtx
% and the derived files
%    hobsub.sty, hobsub.pdf, hobsub.ins, hobsub.drv, hobsub-generic.sty,
%    hobsub-hyperref.sty, hobsub-test1.tex.
%
% Distribution:
%    CTAN:macros/latex/contrib/oberdiek/hobsub.dtx
%    CTAN:macros/latex/contrib/oberdiek/hobsub.pdf
%
% Unpacking:
%    (a) If hobsub.ins is present:
%           tex hobsub.ins
%    (b) Without hobsub.ins:
%           tex hobsub.dtx
%    (c) If you insist on using LaTeX
%           latex \let\install=y\input{hobsub.dtx}
%        (quote the arguments according to the demands of your shell)
%
% Documentation:
%    (a) If hobsub.drv is present:
%           latex hobsub.drv
%    (b) Without hobsub.drv:
%           latex hobsub.dtx; ...
%    The class ltxdoc loads the configuration file ltxdoc.cfg
%    if available. Here you can specify further options, e.g.
%    use A4 as paper format:
%       \PassOptionsToClass{a4paper}{article}
%
%    Programm calls to get the documentation (example):
%       pdflatex hobsub.dtx
%       makeindex -s gind.ist hobsub.idx
%       pdflatex hobsub.dtx
%       makeindex -s gind.ist hobsub.idx
%       pdflatex hobsub.dtx
%
% Installation:
%    TDS:tex/generic/oberdiek/hobsub.sty
%    TDS:tex/generic/oberdiek/hobsub-generic.sty
%    TDS:tex/generic/oberdiek/hobsub-hyperref.sty
%    TDS:doc/latex/oberdiek/hobsub.pdf
%    TDS:doc/latex/oberdiek/test/hobsub-test1.tex
%    TDS:source/latex/oberdiek/hobsub.dtx
%
%<*ignore>
\begingroup
  \catcode123=1 %
  \catcode125=2 %
  \def\x{LaTeX2e}%
\expandafter\endgroup
\ifcase 0\ifx\install y1\fi\expandafter
         \ifx\csname processbatchFile\endcsname\relax\else1\fi
         \ifx\fmtname\x\else 1\fi\relax
\else\csname fi\endcsname
%</ignore>
%<*install>
\input docstrip.tex
\Msg{************************************************************************}
\Msg{* Installation}
\Msg{* Package: hobsub 2016/05/16 v1.14 Construct package bundles (HO)}
\Msg{************************************************************************}

\keepsilent
\askforoverwritefalse

\let\MetaPrefix\relax
\preamble

This is a generated file.

Project: hobsub
Version: 2016/05/16 v1.14

Copyright (C) 2011, 2012 by
   Heiko Oberdiek <heiko.oberdiek at googlemail.com>

This work may be distributed and/or modified under the
conditions of the LaTeX Project Public License, either
version 1.3c of this license or (at your option) any later
version. This version of this license is in
   http://www.latex-project.org/lppl/lppl-1-3c.txt
and the latest version of this license is in
   http://www.latex-project.org/lppl.txt
and version 1.3 or later is part of all distributions of
LaTeX version 2005/12/01 or later.

This work has the LPPL maintenance status "maintained".

This Current Maintainer of this work is Heiko Oberdiek.

The Base Interpreter refers to any `TeX-Format',
because some files are installed in TDS:tex/generic//.

This work consists of the main source file hobsub.dtx
and the derived files
   hobsub.sty, hobsub.pdf, hobsub.ins, hobsub.drv, hobsub-generic.sty,
   hobsub-hyperref.sty, hobsub-test1.tex.

\endpreamble
\let\MetaPrefix\DoubleperCent

\generate{%
  \file{hobsub.ins}{\from{hobsub.dtx}{install}}%
  \file{hobsub.drv}{\from{hobsub.dtx}{driver}}%
  \usedir{tex/generic/oberdiek}%
  \file{hobsub.sty}{\from{hobsub.dtx}{package}}%
  \usedir{doc/latex/oberdiek/test}%
  \file{hobsub-test1.tex}{\from{hobsub.dtx}{test1}}%
  \nopreamble
  \nopostamble
  \usedir{source/latex/oberdiek/catalogue}%
  \file{hobsub.xml}{\from{hobsub.dtx}{catalogue}}%
}
\let\hobsubSavedNormalLine\normalLine
\def\hobsubGobble#1{}
\begingroup
  \lccode`\0=32 % space
  \lccode`\1=35 % hash
  \lccode`\2=37 % percent
  \lccode`\3=92 % backslash
  \lccode`\4=123 % {
  \lccode`\5=125 % }
\lowercase{\endgroup
  \let\hobsubTokenSpaceOther= 0%
  \let\hobsubTokenHash= 1%
  \let\hobsubTokenPercent= 2%
  \let\hobsubTokenBackslash= 3%
  \let\hobsubTokenBraceLeft= 4%
  \let\hobsubTokenBraceRight= 5%
  \def\hobsubNormalLineWithSpace{%
    \hobsubSavedNormalLine0%
  }%
}
\def\normalLine{%
  \futurelet\hobsubToken\hobsubFirst
}
\def\hobsubFirst{%
  \ifx\hobsubToken\hobsubTokenSpaceOther
    \expandafter\expandafter\expandafter\futurelet
    \expandafter\expandafter\expandafter\hobsubToken
    \expandafter\expandafter\expandafter\hobsubNext
    \expandafter\hobsubGobble
  \else
    \expandafter\hobsubSavedNormalLine
  \fi
}
\def\hobsubNext{%
  \ifx\hobsubToken\hobsubTokenSpaceOther
    \expandafter\expandafter\expandafter\futurelet
    \expandafter\expandafter\expandafter\hobsubToken
    \expandafter\expandafter\expandafter\hobsubNext 
    \expandafter\hobsubGobble
  \else
    \let\hobsubWithSpace=Y%
    \ifx\hobsubToken\hobsubTokenBackslash
      \let\hobsubWithSpace=N%
    \else
      \ifx\hobsubToken\hobsubTokenBraceRight
        \let\hobsubWithSpace=N%
      \else
        \ifx\hobsubToken\hobsubTokenBraceLeft
          \let\hobsubWithSpace=N%
        \else
          \ifx\hobsubToken\hobsubTokenPercent
            \let\hobsubWithSpace=N%
          \else
            \ifx\hobsubToken\hobsubTokenHash
              \let\hobsubWithSpace=N%
            \fi
          \fi
        \fi
      \fi
    \fi
    \ifx\hobsubWithSpace Y%
      \expandafter\expandafter\expandafter\hobsubNormalLineWithSpace
    \else
      \expandafter\expandafter\expandafter\hobsubSavedNormalLine
    \fi
  \fi
}
\generate{%
  \usedir{tex/generic/oberdiek}%
  \file{hobsub-generic.sty}{%
    \from{hobsub.dtx}{generic,preamble,infwarerr}%
    \from{infwarerr.dtx}{package}%
    \from{hobsub.dtx}{post,ltxcmds}%
    \from{ltxcmds.dtx}{package}%
    \from{hobsub.dtx}{post,ifluatex}%
    \from{ifluatex.dtx}{package}%
    \from{hobsub.dtx}{post,ifvtex}%
    \from{ifvtex.dtx}{package}%
    \from{hobsub.dtx}{post,intcalc}%
    \from{intcalc.dtx}{package}%
    \from{hobsub.dtx}{post,ifpdf}%
    \from{ifpdf.dtx}{package}%
    \from{hobsub.dtx}{post,etexcmds}%
    \from{etexcmds.dtx}{package}%
    \from{hobsub.dtx}{post,kvsetkeys}%
    \from{kvsetkeys.dtx}{package}%
    \from{hobsub.dtx}{post,kvdefinekeys}%
    \from{kvdefinekeys.dtx}{package}%
    \from{hobsub.dtx}{post,luatex-loader}%
    \from{luatex.dtx}{loader}%
    \from{hobsub.dtx}{post,pdftexcmds}%
    \from{pdftexcmds.dtx}{package}%
    \from{hobsub.dtx}{post,pdfescape}%
    \from{pdfescape.dtx}{package}%
    \from{hobsub.dtx}{post,bigintcalc}%
    \from{bigintcalc.dtx}{package}%
    \from{hobsub.dtx}{post,bitset}%
    \from{bitset.dtx}{package}%
    \from{hobsub.dtx}{post,uniquecounter}%
    \from{uniquecounter.dtx}{package}%
    \from{hobsub.dtx}{post,atend}%
  }%
}
\generate{%
  \usedir{tex/generic/oberdiek}%
  \file{hobsub-hyperref.sty}{%
    \from{hobsub.dtx}{hyperref,preamble,,letltxmacro}%
    \from{letltxmacro.dtx}{package}%
    \from{hobsub.dtx}{post,hopatch}%
    \from{hopatch.dtx}{package}%
    \from{hobsub.dtx}{post,xcolor-patch}%
    \from{hycolor.dtx}{xcolor}%
    \from{hobsub.dtx}{post,atveryend}%
    \from{atveryend.dtx}{package}%
    \from{hobsub.dtx}{post,atbegshi}%
    \from{atbegshi.dtx}{package}%
    \from{hobsub.dtx}{post,refcount}%
    \from{refcount.dtx}{package}%
    \from{hobsub.dtx}{post,hycolor}%
    \from{hycolor.dtx}{package}%
    \from{hobsub.dtx}{post}%
  }%
}
\let\normalLine\hobsubSavedNormalLine

\catcode32=13\relax% active space
\let =\space%
\Msg{************************************************************************}
\Msg{*}
\Msg{* To finish the installation you have to move the following}
\Msg{* files into a directory searched by TeX:}
\Msg{*}
\Msg{*     hobsub.sty, hobsub-generic.sty, hobsub-hyperref.sty}
\Msg{*}
\Msg{* To produce the documentation run the file `hobsub.drv'}
\Msg{* through LaTeX.}
\Msg{*}
\Msg{* Happy TeXing!}
\Msg{*}
\Msg{************************************************************************}

\endbatchfile
%</install>
%<*ignore>
\fi
%</ignore>
%<*driver>
\NeedsTeXFormat{LaTeX2e}
\ProvidesFile{hobsub.drv}%
  [2016/05/16 v1.14 Construct package bundles (HO)]%
\documentclass{ltxdoc}
\usepackage{holtxdoc}[2011/11/22]
\usepackage{array}
\usepackage{caption}
\renewcommand*{\bottomfraction}{.5}
\renewcommand*{\topfraction}{.5}
\begin{document}
  \DocInput{hobsub.dtx}%
\end{document}
%</driver>
% \fi
%
%
% \CharacterTable
%  {Upper-case    \A\B\C\D\E\F\G\H\I\J\K\L\M\N\O\P\Q\R\S\T\U\V\W\X\Y\Z
%   Lower-case    \a\b\c\d\e\f\g\h\i\j\k\l\m\n\o\p\q\r\s\t\u\v\w\x\y\z
%   Digits        \0\1\2\3\4\5\6\7\8\9
%   Exclamation   \!     Double quote  \"     Hash (number) \#
%   Dollar        \$     Percent       \%     Ampersand     \&
%   Acute accent  \'     Left paren    \(     Right paren   \)
%   Asterisk      \*     Plus          \+     Comma         \,
%   Minus         \-     Point         \.     Solidus       \/
%   Colon         \:     Semicolon     \;     Less than     \<
%   Equals        \=     Greater than  \>     Question mark \?
%   Commercial at \@     Left bracket  \[     Backslash     \\
%   Right bracket \]     Circumflex    \^     Underscore    \_
%   Grave accent  \`     Left brace    \{     Vertical bar  \|
%   Right brace   \}     Tilde         \~}
%
% \GetFileInfo{hobsub.drv}
%
% \title{The \xpackage{hobsub} package}
% \date{2016/05/16 v1.14}
% \author{Heiko Oberdiek\thanks
% {Please report any issues at https://github.com/ho-tex/oberdiek/issues}\\
% \xemail{heiko.oberdiek at googlemail.com}}
%
% \maketitle
%
% \begin{abstract}
% Package \xpackage{hobsub} implements the idea of loading
% several packages in one big collection package.
% \end{abstract}
%
% \tableofcontents
%
% \def\TableGeneric{^^A
%   \TableEntry{\xpackage{infwarerr}}{}{2010/04/08 v1.3}^^A
%   \TableEntry{\xpackage{ltxcmds}}{}{2011/11/09 v1.22}^^A
%   \TableEntry{\xpackage{ifluatex}}{}{2010/03/01 v1.3}^^A
%   \TableEntry{\xpackage{ifvtex}}{}{2010/03/01 v1.5}^^A
%   \TableEntry{\xpackage{intcalc}}{}{2007/09/27 v1.1}^^A
%   \TableEntry{\xpackage{ifpdf}}{\xpackage{ifluatex}}{2011/01/30 v2.3}^^A
%   \TableEntry{\xpackage{etexcmds}}{}{2011/02/16 v1.5}^^A
%   \TableEntry{\xpackage{kvsetkeys}}{}{2012/04/25 v1.16}^^A
%   \TableEntry{\xpackage{kvdefinekeys}}{}{2011/04/07 v1.3}^^A
%   \TableEntry{\xpackage{luatex-loader}}{}{2010/03/09 v0.4}^^A
%   \TableEntry{\xpackage{pdftexcmds}}{\xpackage{ifluatex}, \xpackage{ifpdf}, \xpackage{infwarerr}, \xpackage{ltxcmds}, \xpackage{luatex-loader}}{2011/11/29 v0.20}^^A
%   \TableEntry{\xpackage{pdfescape}}{\xpackage{ltxcmds}, \xpackage{pdftexcmds}}{2011/11/25 v1.13}^^A
%   \TableEntry{\xpackage{bigintcalc}}{\xpackage{pdftexcmds}}{2012/04/08 v1.3}^^A
%   \TableEntry{\xpackage{bitset}}{\xpackage{bigintcalc}, \xpackage{infwarerr}, \xpackage{intcalc}}{2011/01/30 v1.1}^^A
%   \TableEntry{\xpackage{uniquecounter}}{\xpackage{bigintcalc}, \xpackage{infwarerr}}{2011/01/30 v1.2}^^A
% }
% \def\TableHyperref{^^A
%   \TableEntry{\xpackage{letltxmacro}}{}{2010/09/02 v1.4}^^A
%   \TableEntry{\xpackage{hopatch}}{}{2016/05/16 v1.2}^^A
%   \TableEntry{\xpackage{xcolor-patch}}{}{2011/01/30 v1.7}^^A
%   \TableEntry{\xpackage{atveryend}}{}{2011/06/30 v1.8}^^A
%   \TableEntry{\xpackage{atbegshi}}{\xpackage{ifpdf}, \xpackage{infwarerr}, \xpackage{ltxcmds}}{2011/10/05 v1.16}^^A
%   \TableEntry{\xpackage{refcount}}{\xpackage{infwarerr}, \xpackage{ltxcmds}}{2011/10/16 v3.4}^^A
%   \TableEntry{\xpackage{hycolor}}{\xpackage{xcolor-patch}}{2011/01/30 v1.7}^^A
% }
%
% \newenvironment{TableList}[2][htbp]{^^A
%   \begin{table}[{#1}]^^A
%   \centering
%   \caption{Overview \xpackage{hobsub-#2}}^^A
%   \label{tab:#2}^^A
%   \begin{tabular}{ll>{\raggedright}p{.4\linewidth}}^^A
%   \bfseries Package & \bfseries Release &
%   \bfseries Dependencies\tabularnewline
%   \hline
% }{^^A
%   \end{tabular}^^A
%   \end{table}^^A
% }
% \newcommand*{\TableEntry}[3]{%
%   #1&#3&#2\tabularnewline
% }
%
% \renewcommand*{\arraystretch}{1.1}
% \setlength{\extrarowheight}{2pt}
%
% \section{Documentation}
%
% In January 2011 the mailing list lualatex-dev discussed
% ``quietening lualatex console output'' \cite{lualatex:quiet}.
% Inspired from this, I implemented this package \xpackage{hobsub}
% to load several packages in one file. The package is applied for
% two packages collections that are subsets of my package bundle
% \xpackage{oberdiek}.
% \begin{description}
% \item[\xpackage{hobsub-generic}] is a collection of packages
%   that can also be used with \hologo{plainTeX}.
% \item[\xpackage{hobsub-hyperref}] is a collection of packages
%   that is used by package \xpackage{hyperref}.
% \end{description}
%
% Package \xpackage{hobsub} provides the macros that
% are used in the collection packages and might be used
% in other projects as well. But there are many caveats,
% some of them are discussed below.
%
% The most serious problem is consistency. A package
% that is available as standalone package and is part
% of a collection might have different versions.
% Because my packages are organized in a bundle and should
% be installed and updated as bundle, the risk seems low
% to me. However, if a single package is updated manually, for
% example for testing, then it should be loaded before the
% collection that contains the package with an older version.
% In a collection, the package will not be loaded again if it
% is already loaded.
%
% \subsection{Usage}
%
% The collection packages \xpackage{hobsub-generic} and
% \xpackage{hobsub-hyperref} are loaded as usual:
% \begin{quote}
%  \begin{tabular}{@{}ll@{}}
%    |\usepackage{hobsub-generic}| & \hologo{LaTeX}\\
%    |\input hobsub-generic.sty| & \hologo{plainTeX}
%  \end{tabular}
% \end{quote}
% or
% \begin{quote}|\usepackage{hobsub-hyperref}|\end{quote}
% Of course, both \cs{usepackage} or \cs{RequirePackage}
% can be used.
%
% If you need requirements on the version date of a package
% inside a collection, then specify them afterwards, for
% example:
% \begin{quote}
%   |\usepackage{hobsub-generic}|\\
%   |\usepackage{ltxcmds}[2010/12/04]% |^^A
%   {\itshape |\ltx@ifblank is needed|}
% \end{quote}
%
% Also it is not garanteed that a package will be always
% part of a collection. Therefore it does not harm,
% to make the requirements explicite \emph{after}
% a collection is loaded, see the previous example.
% \hologo{LaTeX} knows that a
% package is already loaded and does not load it again.
%
% \subsection{Supported features}
%
% \begin{description}
% \item[No reloads:] If a package is already loaded then it
%  is not reloaded again if it is part of a collection package.
% \item[\cs{listfiles}:]
%   Each package that is loaded in a collection is also
%   added to the output of \cs{listfiles}.
% \item[\cs{@currname}, \cs{@currext}]
%   are updated, they are used by \hologo{LaTeX}'s
%  \cs{ProvidesPackage}.
% \item[\cs{AtEndOfPackage}:]
%   \hologo{LaTeX}'s hook mechanism that is used for
%   \cs{AtEndOfPackage} is supported. However none of the
%   packages in the collections are using \cs{AtEndOfPackage},
%   therefore this feature is completely untested.
% \end{description}
%
% \subsection{Common limitations}
%
% \begin{description}
% \item[\cs{endinput}:]
%   Often a package is using \cs{endinput}, either added by
%   \xpackage{docstrip}, or to stop package loading at an earlier
%   time. Therefore \cs{endinput} must be catched during package
%   loading to prevent that the collection package is closed
%   too early before reading all packages it contains.
%
%   This redefinition of \cs{endinput} fails if the package loads
%   other files that contain \cs{endinput}. To some degree it could
%   be catched by redefining macros that load files. But this will
%   not work for the primitive \cs{input}.
% \item[Options:]
%   \hologo{LaTeX}'s package options are not supported.
%   Therefore the packages that are put in a collection must
%   not support options.
% \item[Hook packages:]
%   Some packages and classes provides hooks for packages, see
%   packages \xpackage{scrlfile}, \xpackage{filehook}
%   or class \xclass{memoir}. These hooks are not supported
%   for embedded packages inside a collection package.
%   Because of the problem with \cs{endinput}, these hooks
%   are not supported. The hooks might contain code that
%   loads other files that execute \cs{endinput}.
% \end{description}
%
% \subsection{Package \xpackage{hobsub-generic}}
%
% \begin{TableList}[bp]{generic}\TableGeneric\end{TableList}
%
% Collection `generic' contains some of the \hologo{plainTeX} compatible
% packages of my bundle, see table \ref{tab:generic}.
% It adds the following restrictions to any package
% that is part of this collection:
% \begin{description}
% \item[Generic formats:] The following formats must be supported:
%   \hologo{LaTeX}, \hologo{plainTeX} and even \hologo{iniTeX}.
%   This applies to the collection package \xpackage{hobsub-generic}
%   the same way.
% \item[Prefixed macros:]
%   Any package of this collection should only define macros with
%   prefixed names to avoid name clashes. Resources like registers
%   must not be allocated and other global states should not be
%   changed except for added macro definitions.
% \end{description}
% Thus the package \xpackage{hobsub-generic} can also preloaded
% in the format file with \hologo{iniTeX}.
%
% \subsection{Package \xpackage{hobsub-hyperref}}
%
% \begin{TableList}[tp]{hyperref}\TableHyperref\end{TableList}
%
% Collection `hyperref' contains additional packages that are
% used by package \xpackage{hyperref} and that does not fit
% in collection `generic', see table{tab:hyperref}.
% The latter collection package
% \xpackage{hobsub-generic} is loaded via a conventional
% \cs{RequirePackage}.
%
% Because most of the packages in the collection `hyperref'
% needs \hologo{LaTeX}, the collection requires \hologo{LaTeX}.
%
% \subsection{Package \xpackage{hobsub}}
%
% This section is for advanced users that might want to
% build a collection package. Then it can be done by
% using \xpackage{hobsub}:
% \begin{quote}
%   \dots\\
%   |\RequirePackage{hobsub}[2016/05/16]|\\
%   \dots\\
%   |\hobsub@StartPackage{foo1}|\\
%   \textit{\dots\unskip\ package code of \xfile{foo1.sty} \dots}\\
%   |\endinput|\\
%   |<hobsub>|\\
%   |\hobsub@StopPackage|\\
%   |\hobsub@StartPackage{bar2}|\\
%   \textit{\dots\unskip\ package code of \xfile{bar2.sty} \dots}\\
%   |\endinput|\\
%   |<hobsub>|\\
%   |\hobsub@StopPackage|
%   \dots
% \end{quote}
% But be aware of the caveats and limitations, see above.
% Only quite a small portion of the packages can be embedded
% like this.
%
% \clearpage
%
% \StopEventually{
% }
%
% \section{Implementation}
%
% \subsection{Reload check and package identification}
%
% \subsubsection{Package \xpackage{hobsub}}
%
%    \begin{macrocode}
%<*package>
%    \end{macrocode}
%    Reload check, especially if the package is not used with \LaTeX.
%    \begin{macrocode}
\begingroup\catcode61\catcode48\catcode32=10\relax%
  \catcode13=5 % ^^M
  \endlinechar=13 %
  \catcode35=6 % #
  \catcode39=12 % '
  \catcode44=12 % ,
  \catcode45=12 % -
  \catcode46=12 % .
  \catcode58=12 % :
  \catcode64=11 % @
  \catcode123=1 % {
  \catcode125=2 % }
  \expandafter\let\expandafter\x\csname ver@hobsub.sty\endcsname
  \ifx\x\relax % plain-TeX, first loading
  \else
    \def\empty{}%
    \ifx\x\empty % LaTeX, first loading,
      % variable is initialized, but \ProvidesPackage not yet seen
    \else
      \expandafter\ifx\csname PackageInfo\endcsname\relax
        \def\x#1#2{%
          \immediate\write-1{Package #1 Info: #2.}%
        }%
      \else
        \def\x#1#2{\PackageInfo{#1}{#2, stopped}}%
      \fi
      \x{hobsub}{The package is already loaded}%
      \aftergroup\endinput
    \fi
  \fi
\endgroup%
%    \end{macrocode}
%    Package identification:
%    \begin{macrocode}
\begingroup\catcode61\catcode48\catcode32=10\relax%
  \catcode13=5 % ^^M
  \endlinechar=13 %
  \catcode35=6 % #
  \catcode39=12 % '
  \catcode40=12 % (
  \catcode41=12 % )
  \catcode44=12 % ,
  \catcode45=12 % -
  \catcode46=12 % .
  \catcode47=12 % /
  \catcode58=12 % :
  \catcode64=11 % @
  \catcode91=12 % [
  \catcode93=12 % ]
  \catcode123=1 % {
  \catcode125=2 % }
  \expandafter\ifx\csname ProvidesPackage\endcsname\relax
    \def\x#1#2#3[#4]{\endgroup
      \immediate\write-1{Package: #3 #4}%
      \xdef#1{#4}%
    }%
  \else
    \def\x#1#2[#3]{\endgroup
      #2[{#3}]%
      \ifx#1\@undefined
        \xdef#1{#3}%
      \fi
      \ifx#1\relax
        \xdef#1{#3}%
      \fi
    }%
  \fi
\expandafter\x\csname ver@hobsub.sty\endcsname
\ProvidesPackage{hobsub}%
  [2016/05/16 v1.14 Construct package bundles (HO)]%
%    \end{macrocode}
%    \begin{macrocode}
%</package>
%    \end{macrocode}
%
% \subsubsection{Package \xpackage{hobsub-generic}}
%
%    \begin{macrocode}
%<*generic>
%    \end{macrocode}
%    Reload check, especially if the package is not used with \LaTeX.
%    \begin{macrocode}
\begingroup\catcode61\catcode48\catcode32=10\relax%
  \catcode13=5 % ^^M
  \endlinechar=13 %
  \catcode35=6 % #
  \catcode39=12 % '
  \catcode44=12 % ,
  \catcode45=12 % -
  \catcode46=12 % .
  \catcode58=12 % :
  \catcode64=11 % @
  \catcode123=1 % {
  \catcode125=2 % }
  \expandafter\let\expandafter\x\csname ver@hobsub-generic.sty\endcsname
  \ifx\x\relax % plain-TeX, first loading
  \else
    \def\empty{}%
    \ifx\x\empty % LaTeX, first loading,
      % variable is initialized, but \ProvidesPackage not yet seen
    \else
      \expandafter\ifx\csname PackageInfo\endcsname\relax
        \def\x#1#2{%
          \immediate\write-1{Package #1 Info: #2.}%
        }%
      \else
        \def\x#1#2{\PackageInfo{#1}{#2, stopped}}%
      \fi
      \x{hobsub-generic}{The package is already loaded}%
      \aftergroup\endinput
    \fi
  \fi
\endgroup%
%    \end{macrocode}
%    Package identification:
%    \begin{macrocode}
\begingroup\catcode61\catcode48\catcode32=10\relax%
  \catcode13=5 % ^^M
  \endlinechar=13 %
  \catcode35=6 % #
  \catcode39=12 % '
  \catcode40=12 % (
  \catcode41=12 % )
  \catcode44=12 % ,
  \catcode45=12 % -
  \catcode46=12 % .
  \catcode47=12 % /
  \catcode58=12 % :
  \catcode64=11 % @
  \catcode91=12 % [
  \catcode93=12 % ]
  \catcode123=1 % {
  \catcode125=2 % }
  \expandafter\ifx\csname ProvidesPackage\endcsname\relax
    \def\x#1#2#3[#4]{\endgroup
      \immediate\write-1{Package: #3 #4}%
      \xdef#1{#4}%
    }%
  \else
    \def\x#1#2[#3]{\endgroup
      #2[{#3}]%
      \ifx#1\@undefined
        \xdef#1{#3}%
      \fi
      \ifx#1\relax
        \xdef#1{#3}%
      \fi
    }%
  \fi
\expandafter\x\csname ver@hobsub-generic.sty\endcsname
\ProvidesPackage{hobsub-generic}%
  [2016/05/16 v1.14 Bundle oberdiek, subset generic (HO)]%
%    \end{macrocode}
%    \begin{macrocode}
%</generic>
%    \end{macrocode}
%
% \subsubsection{Package \xpackage{hobsub-hyperref}}
%
%    \begin{macrocode}
%<*hyperref>
%    \end{macrocode}
%    Reload check, especially if the package is not used with \LaTeX.
%    \begin{macrocode}
\begingroup\catcode61\catcode48\catcode32=10\relax%
  \catcode13=5 % ^^M
  \endlinechar=13 %
  \catcode35=6 % #
  \catcode39=12 % '
  \catcode44=12 % ,
  \catcode45=12 % -
  \catcode46=12 % .
  \catcode58=12 % :
  \catcode64=11 % @
  \catcode123=1 % {
  \catcode125=2 % }
  \expandafter\let\expandafter\x\csname ver@hobsub-hyperref.sty\endcsname
  \ifx\x\relax % plain-TeX, first loading
  \else
    \def\empty{}%
    \ifx\x\empty % LaTeX, first loading,
      % variable is initialized, but \ProvidesPackage not yet seen
    \else
      \expandafter\ifx\csname PackageInfo\endcsname\relax
        \def\x#1#2{%
          \immediate\write-1{Package #1 Info: #2.}%
        }%
      \else
        \def\x#1#2{\PackageInfo{#1}{#2, stopped}}%
      \fi
      \x{hobsub-hyperref}{The package is already loaded}%
      \aftergroup\endinput
    \fi
  \fi
\endgroup%
%    \end{macrocode}
%    Package identification:
%    \begin{macrocode}
\begingroup\catcode61\catcode48\catcode32=10\relax%
  \catcode13=5 % ^^M
  \endlinechar=13 %
  \catcode35=6 % #
  \catcode39=12 % '
  \catcode40=12 % (
  \catcode41=12 % )
  \catcode44=12 % ,
  \catcode45=12 % -
  \catcode46=12 % .
  \catcode47=12 % /
  \catcode58=12 % :
  \catcode64=11 % @
  \catcode91=12 % [
  \catcode93=12 % ]
  \catcode123=1 % {
  \catcode125=2 % }
  \expandafter\ifx\csname ProvidesPackage\endcsname\relax
    \def\x#1#2#3[#4]{\endgroup
      \immediate\write-1{Package: #3 #4}%
      \xdef#1{#4}%
    }%
  \else
    \def\x#1#2[#3]{\endgroup
      #2[{#3}]%
      \ifx#1\@undefined
        \xdef#1{#3}%
      \fi
      \ifx#1\relax
        \xdef#1{#3}%
      \fi
    }%
  \fi
\expandafter\x\csname ver@hobsub-hyperref.sty\endcsname
\ProvidesPackage{hobsub-hyperref}%
  [2016/05/16 v1.14 Bundle oberdiek, subset hyperref (HO)]%
%    \end{macrocode}
%    \begin{macrocode}
%</hyperref>
%    \end{macrocode}
%
% \subsection{Catcodes}
%
%    \begin{macrocode}
%<*package|generic|hyperref>
\begingroup\catcode61\catcode48\catcode32=10\relax%
  \catcode13=5 % ^^M
  \endlinechar=13 %
  \catcode123=1 % {
  \catcode125=2 % }
  \catcode64=11 % @
  \def\x{\endgroup
    \expandafter\edef\csname HOBsub@AtEnd\endcsname{%
      \endlinechar=\the\endlinechar\relax
      \catcode13=\the\catcode13\relax
      \catcode32=\the\catcode32\relax
      \catcode35=\the\catcode35\relax
      \catcode61=\the\catcode61\relax
      \catcode64=\the\catcode64\relax
      \catcode123=\the\catcode123\relax
      \catcode125=\the\catcode125\relax
    }%
  }%
\x\catcode61\catcode48\catcode32=10\relax%
\catcode13=5 % ^^M
\endlinechar=13 %
\catcode35=6 % #
\catcode64=11 % @
\catcode123=1 % {
\catcode125=2 % }
\def\TMP@EnsureCode#1#2{%
  \edef\HOBsub@AtEnd{%
    \HOBsub@AtEnd
    \catcode#1=\the\catcode#1\relax
  }%
  \catcode#1=#2\relax
}
\TMP@EnsureCode{39}{12}% '
\TMP@EnsureCode{40}{12}% (
\TMP@EnsureCode{41}{12}% )
\TMP@EnsureCode{45}{12}% -
\TMP@EnsureCode{46}{12}% .
\TMP@EnsureCode{47}{12}% /
\TMP@EnsureCode{58}{12}% :
\TMP@EnsureCode{60}{12}% <
\TMP@EnsureCode{62}{12}% >
\TMP@EnsureCode{96}{12}% `
\edef\HOBsub@AtEnd{\HOBsub@AtEnd\noexpand\endinput}
%    \end{macrocode}
%
%    \begin{macrocode}
%</package|generic|hyperref>
%    \end{macrocode}
%
% \subsection{Package \xpackage{hobsub-hyperref} loads \xpackage{hobsub-generic}}
%
%    \begin{macrocode}
%<*hyperref>
\NeedsTeXFormat{LaTeX2e}
\RequirePackage{hobsub-generic}[2016/05/16]
%</hyperref>
%    \end{macrocode}
%
% \subsection{Preamble}
%
%    \begin{macrocode}
%<*preamble>
\begingroup\expandafter\expandafter\expandafter\endgroup
\expandafter\ifx\csname ver@hobsub.sty\endcsname\relax
\else
  \expandafter\hobsub@GobbleRemainingPackage
\fi
%</preamble>
%    \end{macrocode}
%
%
%    \begin{macrocode}
%<*package|preamble>
%    \end{macrocode}
%
%    \begin{macro}{\HOBsub@OrgEndinput}
%    \begin{macrocode}
\begingroup\expandafter\expandafter\expandafter\endgroup
\expandafter\ifx\csname HOBsub@OrgEndinput\endcsname\relax
  \let\HOBsub@OrgEndinput\endinput
\fi
%    \end{macrocode}
%    \end{macro}
%    \begin{macro}{\hobsub@GobbleRemainingPackage}
%    \begin{macrocode}
\def\hobsub@GobbleRemainingPackage{%
  \begingroup
  \catcode92=14 % backslash: comment
  \catcode32=14 % space: comment
  \catcode35=14 % hash: comment
  \catcode123=14 % left brace: comment
  \catcode125=14 % right brace: comment
  \catcode60=3 % less: math
  \catcode62=4 % greater: align
  \endlinechar=-1 %
  \HOBsub@GobbleRemainingPackage
}
%    \end{macrocode}
%    \end{macro}
%    \begin{macro}{\HOBsub@GobbleRemainingPackage}
%    \begin{macrocode}
\catcode60=3 % less: dollar
\catcode62=4 % greater: align
\long\def\HOBsub@GobbleRemainingPackage#1<hobsub>{%
  \endgroup
}
\catcode60=12 % less: other
\catcode62=12 % greater: other
%    \end{macrocode}
%    \end{macro}
%
%    \begin{macro}{\hobsub@StartPackage}
%    \begin{macrocode}
\def\hobsub@StartPackage#1{%
  \begingroup\expandafter\expandafter\expandafter\endgroup
  \expandafter\ifx\csname ver@#1.sty\endcsname\relax
    \let\HOBsub@OrgCurrName\@currname
    \let\HOBsub@OrgCurrExt\@currext
    \csname @pushfilename\endcsname
    \def\@currname{#1}%
    \def\@currext{sty}%
    \expandafter\def\csname\@currname.\@currext-h@@k\endcsname{}%
    \let\endinput\hobsub@GobbleRemainingPackage
    \def\hobsub@StopPackage{%
      \let\hobsub@StopPackage\relax
      \HOBsub@StopPackage
%     \hobsub@Info{hobsub}{Package `#1' loaded}%
    }%
    \hobsub@AddToFileList{#1.sty}%
  \else
    \hobsub@Info{hobsub}{Skipping package `#1' (already loaded)}%
    \let\hobsub@StopPackage\relax
    \expandafter\hobsub@GobbleRemainingPackage
  \fi
}
%    \end{macrocode}
%    \end{macro}
%
%    \begin{macro}{\hobsub@StopPackage}
%    \begin{macrocode}
\let\hobsub@StopPackage\relax
%    \end{macrocode}
%    \end{macro}
%
%    \begin{macro}{\hobsub@Info}
%    \begin{macrocode}
\def\hobsub@Info#1#2{%
  \begingroup\expandafter\expandafter\expandafter\endgroup
  \expandafter\ifx\csname @PackageInfoNoLine\endcsname\relax
    \immediate\write-1{Package #1 Info: #2.}%
  \else
    \let\hobsub@Info\@PackageInfoNoLine
    \hobsub@Info{#1}{#2}%
  \fi
}
%    \end{macrocode}
%    \end{macro}
%
%    \begin{macro}{\HOBsub@StopPackage}
%    \begin{macrocode}
\def\HOBsub@StopPackage{%
  \csname\@currname.\@currext-h@@k\endcsname
  \let\endinput\HOBsub@OrgEndinput
  \csname @popfilename\endcsname
  \let\@currname\HOBsub@OrgCurrName
  \let\@currext\HOBsub@OrgCurrExt
}
%    \end{macrocode}
%    \end{macro}
%
%    \begin{macro}{\hobsub@AddToFileList}
%    \begin{macrocode}
\begingroup\expandafter\expandafter\expandafter\endgroup
\expandafter\ifx\csname @addtofilelist\endcsname\relax
  \def\hobsub@AddToFileList#1{}%
\else
  \def\hobsub@AddToFileList#1{%
    \@addtofilelist{#1}%
  }%
\fi
%    \end{macrocode}
%    \end{macro}
%
%    \begin{macrocode}
%</package|preamble>
%    \end{macrocode}
%
% \subsection{End of preamble for subset packages}
%
%    Add the end marker for possible skipping of previous
%    macros, if package \xpackage{hobsub} was already loaded.
%    \begin{macrocode}
%<*preamble>
\hobsub@GobbleRemainingPackage
<hobsub>
%    \end{macrocode}
%
%    All macros of package \xpackage{hobsub} are now available.
%    Now we officially load the package \xpackage{hobsub} inline
%    for proper package identification.
%    \begin{macrocode}
\hobsub@StartPackage{hobsub}%
\begingroup\expandafter\expandafter\expandafter\endgroup
\expandafter\ifx\csname ProvidesPackage\endcsname\relax
  \immediate\write-1{%
    Package: hobsub 2016/05/16 v1.14 Construct package bundles (HO)%
  }%
\else
  \ProvidesPackage{hobsub}%
  [2016/05/16 v1.14 Construct package bundles (HO)]%
\fi
%</preamble>
%    \end{macrocode}
%
%
%    \begin{macrocode}
%<*post|preamble>
 \endinput
<hobsub>
\hobsub@StopPackage
%</post|preamble>
%    \end{macrocode}
%
% \subsection{Package list}
%
%    \begin{macrocode}
%<infwarerr>\hobsub@StartPackage{infwarerr}
%<ltxcmds>\hobsub@StartPackage{ltxcmds}
%<ifluatex>\hobsub@StartPackage{ifluatex}
%<ifvtex>\hobsub@StartPackage{ifvtex}
%<intcalc>\hobsub@StartPackage{intcalc}
%<ifpdf>\hobsub@StartPackage{ifpdf}
%<etexcmds>\hobsub@StartPackage{etexcmds}
%<kvsetkeys>\hobsub@StartPackage{kvsetkeys}
%<kvdefinekeys>\hobsub@StartPackage{kvdefinekeys}
%<*luatex-loader>
\ifluatex
\else
  \expandafter\hobsub@GobbleRemainingPackage
\fi
\hobsub@StartPackage{luatex-loader}
%</luatex-loader>
%<pdftexcmds>\hobsub@StartPackage{pdftexcmds}
%<pdfescape>\hobsub@StartPackage{pdfescape}
%<bigintcalc>\hobsub@StartPackage{bigintcalc}
%<bitset>\hobsub@StartPackage{bitset}
%<uniquecounter>\hobsub@StartPackage{uniquecounter}
%<letltxmacro>\hobsub@StartPackage{letltxmacro}
%<hopatch>\hobsub@StartPackage{hopatch}
%<xcolor-patch>\hobsub@StartPackage{xcolor-patch}
%<atveryend>\hobsub@StartPackage{atveryend}
%<atbegshi>\hobsub@StartPackage{atbegshi}
%<refcount>\hobsub@StartPackage{refcount}
%<hycolor>\hobsub@StartPackage{hycolor}
%    \end{macrocode}
%
% \subsection{End of packages}
%
%    \begin{macrocode}
%<package|atend>\HOBsub@AtEnd%
%    \end{macrocode}
%
% \section{Test}
%
% \subsection{Catcode checks for loading}
%
%    \begin{macrocode}
%<*test1>
%    \end{macrocode}
%    \begin{macrocode}
\catcode`\{=1 %
\catcode`\}=2 %
\catcode`\#=6 %
\catcode`\@=11 %
\expandafter\ifx\csname count@\endcsname\relax
  \countdef\count@=255 %
\fi
\expandafter\ifx\csname @gobble\endcsname\relax
  \long\def\@gobble#1{}%
\fi
\expandafter\ifx\csname @firstofone\endcsname\relax
  \long\def\@firstofone#1{#1}%
\fi
\expandafter\ifx\csname loop\endcsname\relax
  \expandafter\@firstofone
\else
  \expandafter\@gobble
\fi
{%
  \def\loop#1\repeat{%
    \def\body{#1}%
    \iterate
  }%
  \def\iterate{%
    \body
      \let\next\iterate
    \else
      \let\next\relax
    \fi
    \next
  }%
  \let\repeat=\fi
}%
\def\RestoreCatcodes{}
\count@=0 %
\loop
  \edef\RestoreCatcodes{%
    \RestoreCatcodes
    \catcode\the\count@=\the\catcode\count@\relax
  }%
\ifnum\count@<255 %
  \advance\count@ 1 %
\repeat

\def\RangeCatcodeInvalid#1#2{%
  \count@=#1\relax
  \loop
    \catcode\count@=15 %
  \ifnum\count@<#2\relax
    \advance\count@ 1 %
  \repeat
}
\def\RangeCatcodeCheck#1#2#3{%
  \count@=#1\relax
  \loop
    \ifnum#3=\catcode\count@
    \else
      \errmessage{%
        Character \the\count@\space
        with wrong catcode \the\catcode\count@\space
        instead of \number#3%
      }%
    \fi
  \ifnum\count@<#2\relax
    \advance\count@ 1 %
  \repeat
}
\def\space{ }
\expandafter\ifx\csname LoadCommand\endcsname\relax
  \def\LoadCommand{\input hobsub.sty\relax}%
\fi
\def\Test{%
  \RangeCatcodeInvalid{0}{47}%
  \RangeCatcodeInvalid{58}{64}%
  \RangeCatcodeInvalid{91}{96}%
  \RangeCatcodeInvalid{123}{255}%
  \catcode`\@=12 %
  \catcode`\\=0 %
  \catcode`\%=14 %
  \LoadCommand
  \RangeCatcodeCheck{0}{36}{15}%
  \RangeCatcodeCheck{37}{37}{14}%
  \RangeCatcodeCheck{38}{47}{15}%
  \RangeCatcodeCheck{48}{57}{12}%
  \RangeCatcodeCheck{58}{63}{15}%
  \RangeCatcodeCheck{64}{64}{12}%
  \RangeCatcodeCheck{65}{90}{11}%
  \RangeCatcodeCheck{91}{91}{15}%
  \RangeCatcodeCheck{92}{92}{0}%
  \RangeCatcodeCheck{93}{96}{15}%
  \RangeCatcodeCheck{97}{122}{11}%
  \RangeCatcodeCheck{123}{255}{15}%
  \RestoreCatcodes
}
\Test
\csname @@end\endcsname
\end
%    \end{macrocode}
%    \begin{macrocode}
%</test1>
%    \end{macrocode}
%
% \section{Installation}
%
% \subsection{Download}
%
% \paragraph{Package.} This package is available on
% CTAN\footnote{\url{http://ctan.org/pkg/hobsub}}:
% \begin{description}
% \item[\CTAN{macros/latex/contrib/oberdiek/hobsub.dtx}] The source file.
% \item[\CTAN{macros/latex/contrib/oberdiek/hobsub.pdf}] Documentation.
% \end{description}
%
%
% \paragraph{Bundle.} All the packages of the bundle `oberdiek'
% are also available in a TDS compliant ZIP archive. There
% the packages are already unpacked and the documentation files
% are generated. The files and directories obey the TDS standard.
% \begin{description}
% \item[\CTAN{install/macros/latex/contrib/oberdiek.tds.zip}]
% \end{description}
% \emph{TDS} refers to the standard ``A Directory Structure
% for \TeX\ Files'' (\CTAN{tds/tds.pdf}). Directories
% with \xfile{texmf} in their name are usually organized this way.
%
% \subsection{Bundle installation}
%
% \paragraph{Unpacking.} Unpack the \xfile{oberdiek.tds.zip} in the
% TDS tree (also known as \xfile{texmf} tree) of your choice.
% Example (linux):
% \begin{quote}
%   |unzip oberdiek.tds.zip -d ~/texmf|
% \end{quote}
%
% \paragraph{Script installation.}
% Check the directory \xfile{TDS:scripts/oberdiek/} for
% scripts that need further installation steps.
% Package \xpackage{attachfile2} comes with the Perl script
% \xfile{pdfatfi.pl} that should be installed in such a way
% that it can be called as \texttt{pdfatfi}.
% Example (linux):
% \begin{quote}
%   |chmod +x scripts/oberdiek/pdfatfi.pl|\\
%   |cp scripts/oberdiek/pdfatfi.pl /usr/local/bin/|
% \end{quote}
%
% \subsection{Package installation}
%
% \paragraph{Unpacking.} The \xfile{.dtx} file is a self-extracting
% \docstrip\ archive. The files are extracted by running the
% \xfile{.dtx} through \plainTeX:
% \begin{quote}
%   \verb|tex hobsub.dtx|
% \end{quote}
%
% \paragraph{TDS.} Now the different files must be moved into
% the different directories in your installation TDS tree
% (also known as \xfile{texmf} tree):
% \begin{quote}
% \def\t{^^A
% \begin{tabular}{@{}>{\ttfamily}l@{ $\rightarrow$ }>{\ttfamily}l@{}}
%   hobsub.sty & tex/generic/oberdiek/hobsub.sty\\
%   hobsub-generic.sty & tex/generic/oberdiek/hobsub-generic.sty\\
%   hobsub-hyperref.sty & tex/generic/oberdiek/hobsub-hyperref.sty\\
%   hobsub.pdf & doc/latex/oberdiek/hobsub.pdf\\
%   test/hobsub-test1.tex & doc/latex/oberdiek/test/hobsub-test1.tex\\
%   hobsub.dtx & source/latex/oberdiek/hobsub.dtx\\
% \end{tabular}^^A
% }^^A
% \sbox0{\t}^^A
% \ifdim\wd0>\linewidth
%   \begingroup
%     \advance\linewidth by\leftmargin
%     \advance\linewidth by\rightmargin
%   \edef\x{\endgroup
%     \def\noexpand\lw{\the\linewidth}^^A
%   }\x
%   \def\lwbox{^^A
%     \leavevmode
%     \hbox to \linewidth{^^A
%       \kern-\leftmargin\relax
%       \hss
%       \usebox0
%       \hss
%       \kern-\rightmargin\relax
%     }^^A
%   }^^A
%   \ifdim\wd0>\lw
%     \sbox0{\small\t}^^A
%     \ifdim\wd0>\linewidth
%       \ifdim\wd0>\lw
%         \sbox0{\footnotesize\t}^^A
%         \ifdim\wd0>\linewidth
%           \ifdim\wd0>\lw
%             \sbox0{\scriptsize\t}^^A
%             \ifdim\wd0>\linewidth
%               \ifdim\wd0>\lw
%                 \sbox0{\tiny\t}^^A
%                 \ifdim\wd0>\linewidth
%                   \lwbox
%                 \else
%                   \usebox0
%                 \fi
%               \else
%                 \lwbox
%               \fi
%             \else
%               \usebox0
%             \fi
%           \else
%             \lwbox
%           \fi
%         \else
%           \usebox0
%         \fi
%       \else
%         \lwbox
%       \fi
%     \else
%       \usebox0
%     \fi
%   \else
%     \lwbox
%   \fi
% \else
%   \usebox0
% \fi
% \end{quote}
% If you have a \xfile{docstrip.cfg} that configures and enables \docstrip's
% TDS installing feature, then some files can already be in the right
% place, see the documentation of \docstrip.
%
% \subsection{Refresh file name databases}
%
% If your \TeX~distribution
% (\teTeX, \mikTeX, \dots) relies on file name databases, you must refresh
% these. For example, \teTeX\ users run \verb|texhash| or
% \verb|mktexlsr|.
%
% \subsection{Some details for the interested}
%
% \paragraph{Attached source.}
%
% The PDF documentation on CTAN also includes the
% \xfile{.dtx} source file. It can be extracted by
% AcrobatReader 6 or higher. Another option is \textsf{pdftk},
% e.g. unpack the file into the current directory:
% \begin{quote}
%   \verb|pdftk hobsub.pdf unpack_files output .|
% \end{quote}
%
% \paragraph{Unpacking with \LaTeX.}
% The \xfile{.dtx} chooses its action depending on the format:
% \begin{description}
% \item[\plainTeX:] Run \docstrip\ and extract the files.
% \item[\LaTeX:] Generate the documentation.
% \end{description}
% If you insist on using \LaTeX\ for \docstrip\ (really,
% \docstrip\ does not need \LaTeX), then inform the autodetect routine
% about your intention:
% \begin{quote}
%   \verb|latex \let\install=y\input{hobsub.dtx}|
% \end{quote}
% Do not forget to quote the argument according to the demands
% of your shell.
%
% \paragraph{Generating the documentation.}
% You can use both the \xfile{.dtx} or the \xfile{.drv} to generate
% the documentation. The process can be configured by the
% configuration file \xfile{ltxdoc.cfg}. For instance, put this
% line into this file, if you want to have A4 as paper format:
% \begin{quote}
%   \verb|\PassOptionsToClass{a4paper}{article}|
% \end{quote}
% An example follows how to generate the
% documentation with pdf\LaTeX:
% \begin{quote}
%\begin{verbatim}
%pdflatex hobsub.dtx
%makeindex -s gind.ist hobsub.idx
%pdflatex hobsub.dtx
%makeindex -s gind.ist hobsub.idx
%pdflatex hobsub.dtx
%\end{verbatim}
% \end{quote}
%
% \section{Catalogue}
%
% The following XML file can be used as source for the
% \href{http://mirror.ctan.org/help/Catalogue/catalogue.html}{\TeX\ Catalogue}.
% The elements \texttt{caption} and \texttt{description} are imported
% from the original XML file from the Catalogue.
% The name of the XML file in the Catalogue is \xfile{hobsub.xml}.
%    \begin{macrocode}
%<*catalogue>
<?xml version='1.0' encoding='us-ascii'?>
<!DOCTYPE entry SYSTEM 'catalogue.dtd'>
<entry datestamp='$Date$' modifier='$Author$' id='hobsub'>
  <name>hobsub</name>
  <caption>Construct package bundles</caption>
  <authorref id='auth:oberdiek'/>
  <copyright owner='Heiko Oberdiek' year='2011,2012'/>
  <license type='lppl1.3'/>
  <version number='1.14'/>
  <description>
    The package offers a means of loading a bundle of (other) packages
    &#x201C;at once&#x201D;, offering a means of reducing the amount of log output
    as the packages are loaded, and an aide-memoire for complicated
    requirements.  In some ways, loading a package via hobsub behaves
    much like loading the package in the normal way; for example,
    reloads are suppressed, as in LaTeX proper, and correct
    <tt>\listfiles</tt> output is produced.
    <p/>
    Examples provided are a &#x2018;generic&#x2019; bundle (comprising the author&#x2019;s
    packages that can be used with Plain TeX) and a &#x2018;hyperref&#x2019; bundle
    (comprising packages useful when generating hypertext output with
    LaTeX).
    <p/>
    The package is part of the <xref refid='oberdiek'>oberdiek</xref> bundle.
  </description>
  <documentation details='Package documentation'
      href='ctan:/macros/latex/contrib/oberdiek/hobsub.pdf'/>
  <ctan file='true' path='/macros/latex/contrib/oberdiek/hobsub.dtx'/>
  <miktex location='oberdiek'/>
  <texlive location='oberdiek'/>
  <install path='/macros/latex/contrib/oberdiek/oberdiek.tds.zip'/>
</entry>
%</catalogue>
%    \end{macrocode}
%
% \begin{thebibliography}{9}
% \raggedright
% \bibitem{lualatex:quiet}
% Will Robertson,
% \textit{[lltx] quietening lualatex console output};
% mailing list \textsf{lualatex-dev at tug.org},
% 2011-01-29;
% \url{http://tug.org/pipermail/lualatex-dev/2011-January/001031.html}.
%
% \end{thebibliography}
%
% \begin{History}
%   \begin{Version}{2011/01/30 v1.0}
%   \item
%     First public version.
%   \end{Version}
%   \begin{Version}{2011/04/16 v1.1}
%   \item
%     Package updates.
%   \end{Version}
%   \begin{Version}{2011/04/17 v1.2}
%   \item
%     White spaces at line begins are removed or reduced
%     in the generated collection packages.
%   \end{Version}
%   \begin{Version}{2011/04/18 v1.3}
%   \item
%     Package updates.
%   \end{Version}
%   \begin{Version}{2011/04/23 v1.4}
%   \item
%     Package updates.
%   \end{Version}
%   \begin{Version}{2011/06/24 v1.5}
%   \item
%     Package updates.
%   \end{Version}
%   \begin{Version}{2011/06/30 v1.6}
%   \item
%     Package updates.
%   \end{Version}
%   \begin{Version}{2011/07/01 v1.7}
%   \item
%     Package updates.
%   \end{Version}
%   \begin{Version}{2011/07/28 v1.8}
%   \item
%     Package updates.
%   \end{Version}
%   \begin{Version}{2011/08/22 v1.9}
%   \item
%     Package updates.
%   \end{Version}
%   \begin{Version}{2011/10/16 v1.10}
%   \item
%     Package updates.
%   \end{Version}
%   \begin{Version}{2011/11/29 v1.11}
%   \item
%     Package updates.
%   \end{Version}
%   \begin{Version}{2012/04/25 v1.12}
%   \item
%     Package updates.
%   \end{Version}
%   \begin{Version}{2012/05/28 v1.13}
%   \item
%     Package updates.
%   \end{Version}
%   \begin{Version}{2016/05/16 v1.14}
%   \item
%     Documentation updates.
%   \end{Version}
% \end{History}
%
% \PrintIndex
%
% \Finale
\endinput
|
% \end{quote}
% Do not forget to quote the argument according to the demands
% of your shell.
%
% \paragraph{Generating the documentation.}
% You can use both the \xfile{.dtx} or the \xfile{.drv} to generate
% the documentation. The process can be configured by the
% configuration file \xfile{ltxdoc.cfg}. For instance, put this
% line into this file, if you want to have A4 as paper format:
% \begin{quote}
%   \verb|\PassOptionsToClass{a4paper}{article}|
% \end{quote}
% An example follows how to generate the
% documentation with pdf\LaTeX:
% \begin{quote}
%\begin{verbatim}
%pdflatex hobsub.dtx
%makeindex -s gind.ist hobsub.idx
%pdflatex hobsub.dtx
%makeindex -s gind.ist hobsub.idx
%pdflatex hobsub.dtx
%\end{verbatim}
% \end{quote}
%
% \section{Catalogue}
%
% The following XML file can be used as source for the
% \href{http://mirror.ctan.org/help/Catalogue/catalogue.html}{\TeX\ Catalogue}.
% The elements \texttt{caption} and \texttt{description} are imported
% from the original XML file from the Catalogue.
% The name of the XML file in the Catalogue is \xfile{hobsub.xml}.
%    \begin{macrocode}
%<*catalogue>
<?xml version='1.0' encoding='us-ascii'?>
<!DOCTYPE entry SYSTEM 'catalogue.dtd'>
<entry datestamp='$Date$' modifier='$Author$' id='hobsub'>
  <name>hobsub</name>
  <caption>Construct package bundles</caption>
  <authorref id='auth:oberdiek'/>
  <copyright owner='Heiko Oberdiek' year='2011,2012'/>
  <license type='lppl1.3'/>
  <version number='1.14'/>
  <description>
    The package offers a means of loading a bundle of (other) packages
    &#x201C;at once&#x201D;, offering a means of reducing the amount of log output
    as the packages are loaded, and an aide-memoire for complicated
    requirements.  In some ways, loading a package via hobsub behaves
    much like loading the package in the normal way; for example,
    reloads are suppressed, as in LaTeX proper, and correct
    <tt>\listfiles</tt> output is produced.
    <p/>
    Examples provided are a &#x2018;generic&#x2019; bundle (comprising the author&#x2019;s
    packages that can be used with Plain TeX) and a &#x2018;hyperref&#x2019; bundle
    (comprising packages useful when generating hypertext output with
    LaTeX).
    <p/>
    The package is part of the <xref refid='oberdiek'>oberdiek</xref> bundle.
  </description>
  <documentation details='Package documentation'
      href='ctan:/macros/latex/contrib/oberdiek/hobsub.pdf'/>
  <ctan file='true' path='/macros/latex/contrib/oberdiek/hobsub.dtx'/>
  <miktex location='oberdiek'/>
  <texlive location='oberdiek'/>
  <install path='/macros/latex/contrib/oberdiek/oberdiek.tds.zip'/>
</entry>
%</catalogue>
%    \end{macrocode}
%
% \begin{thebibliography}{9}
% \raggedright
% \bibitem{lualatex:quiet}
% Will Robertson,
% \textit{[lltx] quietening lualatex console output};
% mailing list \textsf{lualatex-dev at tug.org},
% 2011-01-29;
% \url{http://tug.org/pipermail/lualatex-dev/2011-January/001031.html}.
%
% \end{thebibliography}
%
% \begin{History}
%   \begin{Version}{2011/01/30 v1.0}
%   \item
%     First public version.
%   \end{Version}
%   \begin{Version}{2011/04/16 v1.1}
%   \item
%     Package updates.
%   \end{Version}
%   \begin{Version}{2011/04/17 v1.2}
%   \item
%     White spaces at line begins are removed or reduced
%     in the generated collection packages.
%   \end{Version}
%   \begin{Version}{2011/04/18 v1.3}
%   \item
%     Package updates.
%   \end{Version}
%   \begin{Version}{2011/04/23 v1.4}
%   \item
%     Package updates.
%   \end{Version}
%   \begin{Version}{2011/06/24 v1.5}
%   \item
%     Package updates.
%   \end{Version}
%   \begin{Version}{2011/06/30 v1.6}
%   \item
%     Package updates.
%   \end{Version}
%   \begin{Version}{2011/07/01 v1.7}
%   \item
%     Package updates.
%   \end{Version}
%   \begin{Version}{2011/07/28 v1.8}
%   \item
%     Package updates.
%   \end{Version}
%   \begin{Version}{2011/08/22 v1.9}
%   \item
%     Package updates.
%   \end{Version}
%   \begin{Version}{2011/10/16 v1.10}
%   \item
%     Package updates.
%   \end{Version}
%   \begin{Version}{2011/11/29 v1.11}
%   \item
%     Package updates.
%   \end{Version}
%   \begin{Version}{2012/04/25 v1.12}
%   \item
%     Package updates.
%   \end{Version}
%   \begin{Version}{2012/05/28 v1.13}
%   \item
%     Package updates.
%   \end{Version}
%   \begin{Version}{2016/05/16 v1.14}
%   \item
%     Documentation updates.
%   \end{Version}
% \end{History}
%
% \PrintIndex
%
% \Finale
\endinput

%        (quote the arguments according to the demands of your shell)
%
% Documentation:
%    (a) If hobsub.drv is present:
%           latex hobsub.drv
%    (b) Without hobsub.drv:
%           latex hobsub.dtx; ...
%    The class ltxdoc loads the configuration file ltxdoc.cfg
%    if available. Here you can specify further options, e.g.
%    use A4 as paper format:
%       \PassOptionsToClass{a4paper}{article}
%
%    Programm calls to get the documentation (example):
%       pdflatex hobsub.dtx
%       makeindex -s gind.ist hobsub.idx
%       pdflatex hobsub.dtx
%       makeindex -s gind.ist hobsub.idx
%       pdflatex hobsub.dtx
%
% Installation:
%    TDS:tex/generic/oberdiek/hobsub.sty
%    TDS:tex/generic/oberdiek/hobsub-generic.sty
%    TDS:tex/generic/oberdiek/hobsub-hyperref.sty
%    TDS:doc/latex/oberdiek/hobsub.pdf
%    TDS:doc/latex/oberdiek/test/hobsub-test1.tex
%    TDS:source/latex/oberdiek/hobsub.dtx
%
%<*ignore>
\begingroup
  \catcode123=1 %
  \catcode125=2 %
  \def\x{LaTeX2e}%
\expandafter\endgroup
\ifcase 0\ifx\install y1\fi\expandafter
         \ifx\csname processbatchFile\endcsname\relax\else1\fi
         \ifx\fmtname\x\else 1\fi\relax
\else\csname fi\endcsname
%</ignore>
%<*install>
\input docstrip.tex
\Msg{************************************************************************}
\Msg{* Installation}
\Msg{* Package: hobsub 2016/05/16 v1.14 Construct package bundles (HO)}
\Msg{************************************************************************}

\keepsilent
\askforoverwritefalse

\let\MetaPrefix\relax
\preamble

This is a generated file.

Project: hobsub
Version: 2016/05/16 v1.14

Copyright (C) 2011, 2012 by
   Heiko Oberdiek <heiko.oberdiek at googlemail.com>

This work may be distributed and/or modified under the
conditions of the LaTeX Project Public License, either
version 1.3c of this license or (at your option) any later
version. This version of this license is in
   http://www.latex-project.org/lppl/lppl-1-3c.txt
and the latest version of this license is in
   http://www.latex-project.org/lppl.txt
and version 1.3 or later is part of all distributions of
LaTeX version 2005/12/01 or later.

This work has the LPPL maintenance status "maintained".

This Current Maintainer of this work is Heiko Oberdiek.

The Base Interpreter refers to any `TeX-Format',
because some files are installed in TDS:tex/generic//.

This work consists of the main source file hobsub.dtx
and the derived files
   hobsub.sty, hobsub.pdf, hobsub.ins, hobsub.drv, hobsub-generic.sty,
   hobsub-hyperref.sty, hobsub-test1.tex.

\endpreamble
\let\MetaPrefix\DoubleperCent

\generate{%
  \file{hobsub.ins}{\from{hobsub.dtx}{install}}%
  \file{hobsub.drv}{\from{hobsub.dtx}{driver}}%
  \usedir{tex/generic/oberdiek}%
  \file{hobsub.sty}{\from{hobsub.dtx}{package}}%
  \usedir{doc/latex/oberdiek/test}%
  \file{hobsub-test1.tex}{\from{hobsub.dtx}{test1}}%
  \nopreamble
  \nopostamble
  \usedir{source/latex/oberdiek/catalogue}%
  \file{hobsub.xml}{\from{hobsub.dtx}{catalogue}}%
}
\let\hobsubSavedNormalLine\normalLine
\def\hobsubGobble#1{}
\begingroup
  \lccode`\0=32 % space
  \lccode`\1=35 % hash
  \lccode`\2=37 % percent
  \lccode`\3=92 % backslash
  \lccode`\4=123 % {
  \lccode`\5=125 % }
\lowercase{\endgroup
  \let\hobsubTokenSpaceOther= 0%
  \let\hobsubTokenHash= 1%
  \let\hobsubTokenPercent= 2%
  \let\hobsubTokenBackslash= 3%
  \let\hobsubTokenBraceLeft= 4%
  \let\hobsubTokenBraceRight= 5%
  \def\hobsubNormalLineWithSpace{%
    \hobsubSavedNormalLine0%
  }%
}
\def\normalLine{%
  \futurelet\hobsubToken\hobsubFirst
}
\def\hobsubFirst{%
  \ifx\hobsubToken\hobsubTokenSpaceOther
    \expandafter\expandafter\expandafter\futurelet
    \expandafter\expandafter\expandafter\hobsubToken
    \expandafter\expandafter\expandafter\hobsubNext
    \expandafter\hobsubGobble
  \else
    \expandafter\hobsubSavedNormalLine
  \fi
}
\def\hobsubNext{%
  \ifx\hobsubToken\hobsubTokenSpaceOther
    \expandafter\expandafter\expandafter\futurelet
    \expandafter\expandafter\expandafter\hobsubToken
    \expandafter\expandafter\expandafter\hobsubNext 
    \expandafter\hobsubGobble
  \else
    \let\hobsubWithSpace=Y%
    \ifx\hobsubToken\hobsubTokenBackslash
      \let\hobsubWithSpace=N%
    \else
      \ifx\hobsubToken\hobsubTokenBraceRight
        \let\hobsubWithSpace=N%
      \else
        \ifx\hobsubToken\hobsubTokenBraceLeft
          \let\hobsubWithSpace=N%
        \else
          \ifx\hobsubToken\hobsubTokenPercent
            \let\hobsubWithSpace=N%
          \else
            \ifx\hobsubToken\hobsubTokenHash
              \let\hobsubWithSpace=N%
            \fi
          \fi
        \fi
      \fi
    \fi
    \ifx\hobsubWithSpace Y%
      \expandafter\expandafter\expandafter\hobsubNormalLineWithSpace
    \else
      \expandafter\expandafter\expandafter\hobsubSavedNormalLine
    \fi
  \fi
}
\generate{%
  \usedir{tex/generic/oberdiek}%
  \file{hobsub-generic.sty}{%
    \from{hobsub.dtx}{generic,preamble,infwarerr}%
    \from{infwarerr.dtx}{package}%
    \from{hobsub.dtx}{post,ltxcmds}%
    \from{ltxcmds.dtx}{package}%
    \from{hobsub.dtx}{post,ifluatex}%
    \from{ifluatex.dtx}{package}%
    \from{hobsub.dtx}{post,ifvtex}%
    \from{ifvtex.dtx}{package}%
    \from{hobsub.dtx}{post,intcalc}%
    \from{intcalc.dtx}{package}%
    \from{hobsub.dtx}{post,ifpdf}%
    \from{ifpdf.dtx}{package}%
    \from{hobsub.dtx}{post,etexcmds}%
    \from{etexcmds.dtx}{package}%
    \from{hobsub.dtx}{post,kvsetkeys}%
    \from{kvsetkeys.dtx}{package}%
    \from{hobsub.dtx}{post,kvdefinekeys}%
    \from{kvdefinekeys.dtx}{package}%
    \from{hobsub.dtx}{post,luatex-loader}%
    \from{luatex.dtx}{loader}%
    \from{hobsub.dtx}{post,pdftexcmds}%
    \from{pdftexcmds.dtx}{package}%
    \from{hobsub.dtx}{post,pdfescape}%
    \from{pdfescape.dtx}{package}%
    \from{hobsub.dtx}{post,bigintcalc}%
    \from{bigintcalc.dtx}{package}%
    \from{hobsub.dtx}{post,bitset}%
    \from{bitset.dtx}{package}%
    \from{hobsub.dtx}{post,uniquecounter}%
    \from{uniquecounter.dtx}{package}%
    \from{hobsub.dtx}{post,atend}%
  }%
}
\generate{%
  \usedir{tex/generic/oberdiek}%
  \file{hobsub-hyperref.sty}{%
    \from{hobsub.dtx}{hyperref,preamble,,letltxmacro}%
    \from{letltxmacro.dtx}{package}%
    \from{hobsub.dtx}{post,hopatch}%
    \from{hopatch.dtx}{package}%
    \from{hobsub.dtx}{post,xcolor-patch}%
    \from{hycolor.dtx}{xcolor}%
    \from{hobsub.dtx}{post,atveryend}%
    \from{atveryend.dtx}{package}%
    \from{hobsub.dtx}{post,atbegshi}%
    \from{atbegshi.dtx}{package}%
    \from{hobsub.dtx}{post,refcount}%
    \from{refcount.dtx}{package}%
    \from{hobsub.dtx}{post,hycolor}%
    \from{hycolor.dtx}{package}%
    \from{hobsub.dtx}{post}%
  }%
}
\let\normalLine\hobsubSavedNormalLine

\catcode32=13\relax% active space
\let =\space%
\Msg{************************************************************************}
\Msg{*}
\Msg{* To finish the installation you have to move the following}
\Msg{* files into a directory searched by TeX:}
\Msg{*}
\Msg{*     hobsub.sty, hobsub-generic.sty, hobsub-hyperref.sty}
\Msg{*}
\Msg{* To produce the documentation run the file `hobsub.drv'}
\Msg{* through LaTeX.}
\Msg{*}
\Msg{* Happy TeXing!}
\Msg{*}
\Msg{************************************************************************}

\endbatchfile
%</install>
%<*ignore>
\fi
%</ignore>
%<*driver>
\NeedsTeXFormat{LaTeX2e}
\ProvidesFile{hobsub.drv}%
  [2016/05/16 v1.14 Construct package bundles (HO)]%
\documentclass{ltxdoc}
\usepackage{holtxdoc}[2011/11/22]
\usepackage{array}
\usepackage{caption}
\renewcommand*{\bottomfraction}{.5}
\renewcommand*{\topfraction}{.5}
\begin{document}
  \DocInput{hobsub.dtx}%
\end{document}
%</driver>
% \fi
%
%
% \CharacterTable
%  {Upper-case    \A\B\C\D\E\F\G\H\I\J\K\L\M\N\O\P\Q\R\S\T\U\V\W\X\Y\Z
%   Lower-case    \a\b\c\d\e\f\g\h\i\j\k\l\m\n\o\p\q\r\s\t\u\v\w\x\y\z
%   Digits        \0\1\2\3\4\5\6\7\8\9
%   Exclamation   \!     Double quote  \"     Hash (number) \#
%   Dollar        \$     Percent       \%     Ampersand     \&
%   Acute accent  \'     Left paren    \(     Right paren   \)
%   Asterisk      \*     Plus          \+     Comma         \,
%   Minus         \-     Point         \.     Solidus       \/
%   Colon         \:     Semicolon     \;     Less than     \<
%   Equals        \=     Greater than  \>     Question mark \?
%   Commercial at \@     Left bracket  \[     Backslash     \\
%   Right bracket \]     Circumflex    \^     Underscore    \_
%   Grave accent  \`     Left brace    \{     Vertical bar  \|
%   Right brace   \}     Tilde         \~}
%
% \GetFileInfo{hobsub.drv}
%
% \title{The \xpackage{hobsub} package}
% \date{2016/05/16 v1.14}
% \author{Heiko Oberdiek\thanks
% {Please report any issues at https://github.com/ho-tex/oberdiek/issues}\\
% \xemail{heiko.oberdiek at googlemail.com}}
%
% \maketitle
%
% \begin{abstract}
% Package \xpackage{hobsub} implements the idea of loading
% several packages in one big collection package.
% \end{abstract}
%
% \tableofcontents
%
% \def\TableGeneric{^^A
%   \TableEntry{\xpackage{infwarerr}}{}{2010/04/08 v1.3}^^A
%   \TableEntry{\xpackage{ltxcmds}}{}{2011/11/09 v1.22}^^A
%   \TableEntry{\xpackage{ifluatex}}{}{2010/03/01 v1.3}^^A
%   \TableEntry{\xpackage{ifvtex}}{}{2010/03/01 v1.5}^^A
%   \TableEntry{\xpackage{intcalc}}{}{2007/09/27 v1.1}^^A
%   \TableEntry{\xpackage{ifpdf}}{\xpackage{ifluatex}}{2011/01/30 v2.3}^^A
%   \TableEntry{\xpackage{etexcmds}}{}{2011/02/16 v1.5}^^A
%   \TableEntry{\xpackage{kvsetkeys}}{}{2012/04/25 v1.16}^^A
%   \TableEntry{\xpackage{kvdefinekeys}}{}{2011/04/07 v1.3}^^A
%   \TableEntry{\xpackage{luatex-loader}}{}{2010/03/09 v0.4}^^A
%   \TableEntry{\xpackage{pdftexcmds}}{\xpackage{ifluatex}, \xpackage{ifpdf}, \xpackage{infwarerr}, \xpackage{ltxcmds}, \xpackage{luatex-loader}}{2011/11/29 v0.20}^^A
%   \TableEntry{\xpackage{pdfescape}}{\xpackage{ltxcmds}, \xpackage{pdftexcmds}}{2011/11/25 v1.13}^^A
%   \TableEntry{\xpackage{bigintcalc}}{\xpackage{pdftexcmds}}{2012/04/08 v1.3}^^A
%   \TableEntry{\xpackage{bitset}}{\xpackage{bigintcalc}, \xpackage{infwarerr}, \xpackage{intcalc}}{2011/01/30 v1.1}^^A
%   \TableEntry{\xpackage{uniquecounter}}{\xpackage{bigintcalc}, \xpackage{infwarerr}}{2011/01/30 v1.2}^^A
% }
% \def\TableHyperref{^^A
%   \TableEntry{\xpackage{letltxmacro}}{}{2010/09/02 v1.4}^^A
%   \TableEntry{\xpackage{hopatch}}{}{2016/05/16 v1.2}^^A
%   \TableEntry{\xpackage{xcolor-patch}}{}{2011/01/30 v1.7}^^A
%   \TableEntry{\xpackage{atveryend}}{}{2011/06/30 v1.8}^^A
%   \TableEntry{\xpackage{atbegshi}}{\xpackage{ifpdf}, \xpackage{infwarerr}, \xpackage{ltxcmds}}{2011/10/05 v1.16}^^A
%   \TableEntry{\xpackage{refcount}}{\xpackage{infwarerr}, \xpackage{ltxcmds}}{2011/10/16 v3.4}^^A
%   \TableEntry{\xpackage{hycolor}}{\xpackage{xcolor-patch}}{2011/01/30 v1.7}^^A
% }
%
% \newenvironment{TableList}[2][htbp]{^^A
%   \begin{table}[{#1}]^^A
%   \centering
%   \caption{Overview \xpackage{hobsub-#2}}^^A
%   \label{tab:#2}^^A
%   \begin{tabular}{ll>{\raggedright}p{.4\linewidth}}^^A
%   \bfseries Package & \bfseries Release &
%   \bfseries Dependencies\tabularnewline
%   \hline
% }{^^A
%   \end{tabular}^^A
%   \end{table}^^A
% }
% \newcommand*{\TableEntry}[3]{%
%   #1&#3&#2\tabularnewline
% }
%
% \renewcommand*{\arraystretch}{1.1}
% \setlength{\extrarowheight}{2pt}
%
% \section{Documentation}
%
% In January 2011 the mailing list lualatex-dev discussed
% ``quietening lualatex console output'' \cite{lualatex:quiet}.
% Inspired from this, I implemented this package \xpackage{hobsub}
% to load several packages in one file. The package is applied for
% two packages collections that are subsets of my package bundle
% \xpackage{oberdiek}.
% \begin{description}
% \item[\xpackage{hobsub-generic}] is a collection of packages
%   that can also be used with \hologo{plainTeX}.
% \item[\xpackage{hobsub-hyperref}] is a collection of packages
%   that is used by package \xpackage{hyperref}.
% \end{description}
%
% Package \xpackage{hobsub} provides the macros that
% are used in the collection packages and might be used
% in other projects as well. But there are many caveats,
% some of them are discussed below.
%
% The most serious problem is consistency. A package
% that is available as standalone package and is part
% of a collection might have different versions.
% Because my packages are organized in a bundle and should
% be installed and updated as bundle, the risk seems low
% to me. However, if a single package is updated manually, for
% example for testing, then it should be loaded before the
% collection that contains the package with an older version.
% In a collection, the package will not be loaded again if it
% is already loaded.
%
% \subsection{Usage}
%
% The collection packages \xpackage{hobsub-generic} and
% \xpackage{hobsub-hyperref} are loaded as usual:
% \begin{quote}
%  \begin{tabular}{@{}ll@{}}
%    |\usepackage{hobsub-generic}| & \hologo{LaTeX}\\
%    |\input hobsub-generic.sty| & \hologo{plainTeX}
%  \end{tabular}
% \end{quote}
% or
% \begin{quote}|\usepackage{hobsub-hyperref}|\end{quote}
% Of course, both \cs{usepackage} or \cs{RequirePackage}
% can be used.
%
% If you need requirements on the version date of a package
% inside a collection, then specify them afterwards, for
% example:
% \begin{quote}
%   |\usepackage{hobsub-generic}|\\
%   |\usepackage{ltxcmds}[2010/12/04]% |^^A
%   {\itshape |\ltx@ifblank is needed|}
% \end{quote}
%
% Also it is not garanteed that a package will be always
% part of a collection. Therefore it does not harm,
% to make the requirements explicite \emph{after}
% a collection is loaded, see the previous example.
% \hologo{LaTeX} knows that a
% package is already loaded and does not load it again.
%
% \subsection{Supported features}
%
% \begin{description}
% \item[No reloads:] If a package is already loaded then it
%  is not reloaded again if it is part of a collection package.
% \item[\cs{listfiles}:]
%   Each package that is loaded in a collection is also
%   added to the output of \cs{listfiles}.
% \item[\cs{@currname}, \cs{@currext}]
%   are updated, they are used by \hologo{LaTeX}'s
%  \cs{ProvidesPackage}.
% \item[\cs{AtEndOfPackage}:]
%   \hologo{LaTeX}'s hook mechanism that is used for
%   \cs{AtEndOfPackage} is supported. However none of the
%   packages in the collections are using \cs{AtEndOfPackage},
%   therefore this feature is completely untested.
% \end{description}
%
% \subsection{Common limitations}
%
% \begin{description}
% \item[\cs{endinput}:]
%   Often a package is using \cs{endinput}, either added by
%   \xpackage{docstrip}, or to stop package loading at an earlier
%   time. Therefore \cs{endinput} must be catched during package
%   loading to prevent that the collection package is closed
%   too early before reading all packages it contains.
%
%   This redefinition of \cs{endinput} fails if the package loads
%   other files that contain \cs{endinput}. To some degree it could
%   be catched by redefining macros that load files. But this will
%   not work for the primitive \cs{input}.
% \item[Options:]
%   \hologo{LaTeX}'s package options are not supported.
%   Therefore the packages that are put in a collection must
%   not support options.
% \item[Hook packages:]
%   Some packages and classes provides hooks for packages, see
%   packages \xpackage{scrlfile}, \xpackage{filehook}
%   or class \xclass{memoir}. These hooks are not supported
%   for embedded packages inside a collection package.
%   Because of the problem with \cs{endinput}, these hooks
%   are not supported. The hooks might contain code that
%   loads other files that execute \cs{endinput}.
% \end{description}
%
% \subsection{Package \xpackage{hobsub-generic}}
%
% \begin{TableList}[bp]{generic}\TableGeneric\end{TableList}
%
% Collection `generic' contains some of the \hologo{plainTeX} compatible
% packages of my bundle, see table \ref{tab:generic}.
% It adds the following restrictions to any package
% that is part of this collection:
% \begin{description}
% \item[Generic formats:] The following formats must be supported:
%   \hologo{LaTeX}, \hologo{plainTeX} and even \hologo{iniTeX}.
%   This applies to the collection package \xpackage{hobsub-generic}
%   the same way.
% \item[Prefixed macros:]
%   Any package of this collection should only define macros with
%   prefixed names to avoid name clashes. Resources like registers
%   must not be allocated and other global states should not be
%   changed except for added macro definitions.
% \end{description}
% Thus the package \xpackage{hobsub-generic} can also preloaded
% in the format file with \hologo{iniTeX}.
%
% \subsection{Package \xpackage{hobsub-hyperref}}
%
% \begin{TableList}[tp]{hyperref}\TableHyperref\end{TableList}
%
% Collection `hyperref' contains additional packages that are
% used by package \xpackage{hyperref} and that does not fit
% in collection `generic', see table{tab:hyperref}.
% The latter collection package
% \xpackage{hobsub-generic} is loaded via a conventional
% \cs{RequirePackage}.
%
% Because most of the packages in the collection `hyperref'
% needs \hologo{LaTeX}, the collection requires \hologo{LaTeX}.
%
% \subsection{Package \xpackage{hobsub}}
%
% This section is for advanced users that might want to
% build a collection package. Then it can be done by
% using \xpackage{hobsub}:
% \begin{quote}
%   \dots\\
%   |\RequirePackage{hobsub}[2016/05/16]|\\
%   \dots\\
%   |\hobsub@StartPackage{foo1}|\\
%   \textit{\dots\unskip\ package code of \xfile{foo1.sty} \dots}\\
%   |\endinput|\\
%   |<hobsub>|\\
%   |\hobsub@StopPackage|\\
%   |\hobsub@StartPackage{bar2}|\\
%   \textit{\dots\unskip\ package code of \xfile{bar2.sty} \dots}\\
%   |\endinput|\\
%   |<hobsub>|\\
%   |\hobsub@StopPackage|
%   \dots
% \end{quote}
% But be aware of the caveats and limitations, see above.
% Only quite a small portion of the packages can be embedded
% like this.
%
% \clearpage
%
% \StopEventually{
% }
%
% \section{Implementation}
%
% \subsection{Reload check and package identification}
%
% \subsubsection{Package \xpackage{hobsub}}
%
%    \begin{macrocode}
%<*package>
%    \end{macrocode}
%    Reload check, especially if the package is not used with \LaTeX.
%    \begin{macrocode}
\begingroup\catcode61\catcode48\catcode32=10\relax%
  \catcode13=5 % ^^M
  \endlinechar=13 %
  \catcode35=6 % #
  \catcode39=12 % '
  \catcode44=12 % ,
  \catcode45=12 % -
  \catcode46=12 % .
  \catcode58=12 % :
  \catcode64=11 % @
  \catcode123=1 % {
  \catcode125=2 % }
  \expandafter\let\expandafter\x\csname ver@hobsub.sty\endcsname
  \ifx\x\relax % plain-TeX, first loading
  \else
    \def\empty{}%
    \ifx\x\empty % LaTeX, first loading,
      % variable is initialized, but \ProvidesPackage not yet seen
    \else
      \expandafter\ifx\csname PackageInfo\endcsname\relax
        \def\x#1#2{%
          \immediate\write-1{Package #1 Info: #2.}%
        }%
      \else
        \def\x#1#2{\PackageInfo{#1}{#2, stopped}}%
      \fi
      \x{hobsub}{The package is already loaded}%
      \aftergroup\endinput
    \fi
  \fi
\endgroup%
%    \end{macrocode}
%    Package identification:
%    \begin{macrocode}
\begingroup\catcode61\catcode48\catcode32=10\relax%
  \catcode13=5 % ^^M
  \endlinechar=13 %
  \catcode35=6 % #
  \catcode39=12 % '
  \catcode40=12 % (
  \catcode41=12 % )
  \catcode44=12 % ,
  \catcode45=12 % -
  \catcode46=12 % .
  \catcode47=12 % /
  \catcode58=12 % :
  \catcode64=11 % @
  \catcode91=12 % [
  \catcode93=12 % ]
  \catcode123=1 % {
  \catcode125=2 % }
  \expandafter\ifx\csname ProvidesPackage\endcsname\relax
    \def\x#1#2#3[#4]{\endgroup
      \immediate\write-1{Package: #3 #4}%
      \xdef#1{#4}%
    }%
  \else
    \def\x#1#2[#3]{\endgroup
      #2[{#3}]%
      \ifx#1\@undefined
        \xdef#1{#3}%
      \fi
      \ifx#1\relax
        \xdef#1{#3}%
      \fi
    }%
  \fi
\expandafter\x\csname ver@hobsub.sty\endcsname
\ProvidesPackage{hobsub}%
  [2016/05/16 v1.14 Construct package bundles (HO)]%
%    \end{macrocode}
%    \begin{macrocode}
%</package>
%    \end{macrocode}
%
% \subsubsection{Package \xpackage{hobsub-generic}}
%
%    \begin{macrocode}
%<*generic>
%    \end{macrocode}
%    Reload check, especially if the package is not used with \LaTeX.
%    \begin{macrocode}
\begingroup\catcode61\catcode48\catcode32=10\relax%
  \catcode13=5 % ^^M
  \endlinechar=13 %
  \catcode35=6 % #
  \catcode39=12 % '
  \catcode44=12 % ,
  \catcode45=12 % -
  \catcode46=12 % .
  \catcode58=12 % :
  \catcode64=11 % @
  \catcode123=1 % {
  \catcode125=2 % }
  \expandafter\let\expandafter\x\csname ver@hobsub-generic.sty\endcsname
  \ifx\x\relax % plain-TeX, first loading
  \else
    \def\empty{}%
    \ifx\x\empty % LaTeX, first loading,
      % variable is initialized, but \ProvidesPackage not yet seen
    \else
      \expandafter\ifx\csname PackageInfo\endcsname\relax
        \def\x#1#2{%
          \immediate\write-1{Package #1 Info: #2.}%
        }%
      \else
        \def\x#1#2{\PackageInfo{#1}{#2, stopped}}%
      \fi
      \x{hobsub-generic}{The package is already loaded}%
      \aftergroup\endinput
    \fi
  \fi
\endgroup%
%    \end{macrocode}
%    Package identification:
%    \begin{macrocode}
\begingroup\catcode61\catcode48\catcode32=10\relax%
  \catcode13=5 % ^^M
  \endlinechar=13 %
  \catcode35=6 % #
  \catcode39=12 % '
  \catcode40=12 % (
  \catcode41=12 % )
  \catcode44=12 % ,
  \catcode45=12 % -
  \catcode46=12 % .
  \catcode47=12 % /
  \catcode58=12 % :
  \catcode64=11 % @
  \catcode91=12 % [
  \catcode93=12 % ]
  \catcode123=1 % {
  \catcode125=2 % }
  \expandafter\ifx\csname ProvidesPackage\endcsname\relax
    \def\x#1#2#3[#4]{\endgroup
      \immediate\write-1{Package: #3 #4}%
      \xdef#1{#4}%
    }%
  \else
    \def\x#1#2[#3]{\endgroup
      #2[{#3}]%
      \ifx#1\@undefined
        \xdef#1{#3}%
      \fi
      \ifx#1\relax
        \xdef#1{#3}%
      \fi
    }%
  \fi
\expandafter\x\csname ver@hobsub-generic.sty\endcsname
\ProvidesPackage{hobsub-generic}%
  [2016/05/16 v1.14 Bundle oberdiek, subset generic (HO)]%
%    \end{macrocode}
%    \begin{macrocode}
%</generic>
%    \end{macrocode}
%
% \subsubsection{Package \xpackage{hobsub-hyperref}}
%
%    \begin{macrocode}
%<*hyperref>
%    \end{macrocode}
%    Reload check, especially if the package is not used with \LaTeX.
%    \begin{macrocode}
\begingroup\catcode61\catcode48\catcode32=10\relax%
  \catcode13=5 % ^^M
  \endlinechar=13 %
  \catcode35=6 % #
  \catcode39=12 % '
  \catcode44=12 % ,
  \catcode45=12 % -
  \catcode46=12 % .
  \catcode58=12 % :
  \catcode64=11 % @
  \catcode123=1 % {
  \catcode125=2 % }
  \expandafter\let\expandafter\x\csname ver@hobsub-hyperref.sty\endcsname
  \ifx\x\relax % plain-TeX, first loading
  \else
    \def\empty{}%
    \ifx\x\empty % LaTeX, first loading,
      % variable is initialized, but \ProvidesPackage not yet seen
    \else
      \expandafter\ifx\csname PackageInfo\endcsname\relax
        \def\x#1#2{%
          \immediate\write-1{Package #1 Info: #2.}%
        }%
      \else
        \def\x#1#2{\PackageInfo{#1}{#2, stopped}}%
      \fi
      \x{hobsub-hyperref}{The package is already loaded}%
      \aftergroup\endinput
    \fi
  \fi
\endgroup%
%    \end{macrocode}
%    Package identification:
%    \begin{macrocode}
\begingroup\catcode61\catcode48\catcode32=10\relax%
  \catcode13=5 % ^^M
  \endlinechar=13 %
  \catcode35=6 % #
  \catcode39=12 % '
  \catcode40=12 % (
  \catcode41=12 % )
  \catcode44=12 % ,
  \catcode45=12 % -
  \catcode46=12 % .
  \catcode47=12 % /
  \catcode58=12 % :
  \catcode64=11 % @
  \catcode91=12 % [
  \catcode93=12 % ]
  \catcode123=1 % {
  \catcode125=2 % }
  \expandafter\ifx\csname ProvidesPackage\endcsname\relax
    \def\x#1#2#3[#4]{\endgroup
      \immediate\write-1{Package: #3 #4}%
      \xdef#1{#4}%
    }%
  \else
    \def\x#1#2[#3]{\endgroup
      #2[{#3}]%
      \ifx#1\@undefined
        \xdef#1{#3}%
      \fi
      \ifx#1\relax
        \xdef#1{#3}%
      \fi
    }%
  \fi
\expandafter\x\csname ver@hobsub-hyperref.sty\endcsname
\ProvidesPackage{hobsub-hyperref}%
  [2016/05/16 v1.14 Bundle oberdiek, subset hyperref (HO)]%
%    \end{macrocode}
%    \begin{macrocode}
%</hyperref>
%    \end{macrocode}
%
% \subsection{Catcodes}
%
%    \begin{macrocode}
%<*package|generic|hyperref>
\begingroup\catcode61\catcode48\catcode32=10\relax%
  \catcode13=5 % ^^M
  \endlinechar=13 %
  \catcode123=1 % {
  \catcode125=2 % }
  \catcode64=11 % @
  \def\x{\endgroup
    \expandafter\edef\csname HOBsub@AtEnd\endcsname{%
      \endlinechar=\the\endlinechar\relax
      \catcode13=\the\catcode13\relax
      \catcode32=\the\catcode32\relax
      \catcode35=\the\catcode35\relax
      \catcode61=\the\catcode61\relax
      \catcode64=\the\catcode64\relax
      \catcode123=\the\catcode123\relax
      \catcode125=\the\catcode125\relax
    }%
  }%
\x\catcode61\catcode48\catcode32=10\relax%
\catcode13=5 % ^^M
\endlinechar=13 %
\catcode35=6 % #
\catcode64=11 % @
\catcode123=1 % {
\catcode125=2 % }
\def\TMP@EnsureCode#1#2{%
  \edef\HOBsub@AtEnd{%
    \HOBsub@AtEnd
    \catcode#1=\the\catcode#1\relax
  }%
  \catcode#1=#2\relax
}
\TMP@EnsureCode{39}{12}% '
\TMP@EnsureCode{40}{12}% (
\TMP@EnsureCode{41}{12}% )
\TMP@EnsureCode{45}{12}% -
\TMP@EnsureCode{46}{12}% .
\TMP@EnsureCode{47}{12}% /
\TMP@EnsureCode{58}{12}% :
\TMP@EnsureCode{60}{12}% <
\TMP@EnsureCode{62}{12}% >
\TMP@EnsureCode{96}{12}% `
\edef\HOBsub@AtEnd{\HOBsub@AtEnd\noexpand\endinput}
%    \end{macrocode}
%
%    \begin{macrocode}
%</package|generic|hyperref>
%    \end{macrocode}
%
% \subsection{Package \xpackage{hobsub-hyperref} loads \xpackage{hobsub-generic}}
%
%    \begin{macrocode}
%<*hyperref>
\NeedsTeXFormat{LaTeX2e}
\RequirePackage{hobsub-generic}[2016/05/16]
%</hyperref>
%    \end{macrocode}
%
% \subsection{Preamble}
%
%    \begin{macrocode}
%<*preamble>
\begingroup\expandafter\expandafter\expandafter\endgroup
\expandafter\ifx\csname ver@hobsub.sty\endcsname\relax
\else
  \expandafter\hobsub@GobbleRemainingPackage
\fi
%</preamble>
%    \end{macrocode}
%
%
%    \begin{macrocode}
%<*package|preamble>
%    \end{macrocode}
%
%    \begin{macro}{\HOBsub@OrgEndinput}
%    \begin{macrocode}
\begingroup\expandafter\expandafter\expandafter\endgroup
\expandafter\ifx\csname HOBsub@OrgEndinput\endcsname\relax
  \let\HOBsub@OrgEndinput\endinput
\fi
%    \end{macrocode}
%    \end{macro}
%    \begin{macro}{\hobsub@GobbleRemainingPackage}
%    \begin{macrocode}
\def\hobsub@GobbleRemainingPackage{%
  \begingroup
  \catcode92=14 % backslash: comment
  \catcode32=14 % space: comment
  \catcode35=14 % hash: comment
  \catcode123=14 % left brace: comment
  \catcode125=14 % right brace: comment
  \catcode60=3 % less: math
  \catcode62=4 % greater: align
  \endlinechar=-1 %
  \HOBsub@GobbleRemainingPackage
}
%    \end{macrocode}
%    \end{macro}
%    \begin{macro}{\HOBsub@GobbleRemainingPackage}
%    \begin{macrocode}
\catcode60=3 % less: dollar
\catcode62=4 % greater: align
\long\def\HOBsub@GobbleRemainingPackage#1<hobsub>{%
  \endgroup
}
\catcode60=12 % less: other
\catcode62=12 % greater: other
%    \end{macrocode}
%    \end{macro}
%
%    \begin{macro}{\hobsub@StartPackage}
%    \begin{macrocode}
\def\hobsub@StartPackage#1{%
  \begingroup\expandafter\expandafter\expandafter\endgroup
  \expandafter\ifx\csname ver@#1.sty\endcsname\relax
    \let\HOBsub@OrgCurrName\@currname
    \let\HOBsub@OrgCurrExt\@currext
    \csname @pushfilename\endcsname
    \def\@currname{#1}%
    \def\@currext{sty}%
    \expandafter\def\csname\@currname.\@currext-h@@k\endcsname{}%
    \let\endinput\hobsub@GobbleRemainingPackage
    \def\hobsub@StopPackage{%
      \let\hobsub@StopPackage\relax
      \HOBsub@StopPackage
%     \hobsub@Info{hobsub}{Package `#1' loaded}%
    }%
    \hobsub@AddToFileList{#1.sty}%
  \else
    \hobsub@Info{hobsub}{Skipping package `#1' (already loaded)}%
    \let\hobsub@StopPackage\relax
    \expandafter\hobsub@GobbleRemainingPackage
  \fi
}
%    \end{macrocode}
%    \end{macro}
%
%    \begin{macro}{\hobsub@StopPackage}
%    \begin{macrocode}
\let\hobsub@StopPackage\relax
%    \end{macrocode}
%    \end{macro}
%
%    \begin{macro}{\hobsub@Info}
%    \begin{macrocode}
\def\hobsub@Info#1#2{%
  \begingroup\expandafter\expandafter\expandafter\endgroup
  \expandafter\ifx\csname @PackageInfoNoLine\endcsname\relax
    \immediate\write-1{Package #1 Info: #2.}%
  \else
    \let\hobsub@Info\@PackageInfoNoLine
    \hobsub@Info{#1}{#2}%
  \fi
}
%    \end{macrocode}
%    \end{macro}
%
%    \begin{macro}{\HOBsub@StopPackage}
%    \begin{macrocode}
\def\HOBsub@StopPackage{%
  \csname\@currname.\@currext-h@@k\endcsname
  \let\endinput\HOBsub@OrgEndinput
  \csname @popfilename\endcsname
  \let\@currname\HOBsub@OrgCurrName
  \let\@currext\HOBsub@OrgCurrExt
}
%    \end{macrocode}
%    \end{macro}
%
%    \begin{macro}{\hobsub@AddToFileList}
%    \begin{macrocode}
\begingroup\expandafter\expandafter\expandafter\endgroup
\expandafter\ifx\csname @addtofilelist\endcsname\relax
  \def\hobsub@AddToFileList#1{}%
\else
  \def\hobsub@AddToFileList#1{%
    \@addtofilelist{#1}%
  }%
\fi
%    \end{macrocode}
%    \end{macro}
%
%    \begin{macrocode}
%</package|preamble>
%    \end{macrocode}
%
% \subsection{End of preamble for subset packages}
%
%    Add the end marker for possible skipping of previous
%    macros, if package \xpackage{hobsub} was already loaded.
%    \begin{macrocode}
%<*preamble>
\hobsub@GobbleRemainingPackage
<hobsub>
%    \end{macrocode}
%
%    All macros of package \xpackage{hobsub} are now available.
%    Now we officially load the package \xpackage{hobsub} inline
%    for proper package identification.
%    \begin{macrocode}
\hobsub@StartPackage{hobsub}%
\begingroup\expandafter\expandafter\expandafter\endgroup
\expandafter\ifx\csname ProvidesPackage\endcsname\relax
  \immediate\write-1{%
    Package: hobsub 2016/05/16 v1.14 Construct package bundles (HO)%
  }%
\else
  \ProvidesPackage{hobsub}%
  [2016/05/16 v1.14 Construct package bundles (HO)]%
\fi
%</preamble>
%    \end{macrocode}
%
%
%    \begin{macrocode}
%<*post|preamble>
 \endinput
<hobsub>
\hobsub@StopPackage
%</post|preamble>
%    \end{macrocode}
%
% \subsection{Package list}
%
%    \begin{macrocode}
%<infwarerr>\hobsub@StartPackage{infwarerr}
%<ltxcmds>\hobsub@StartPackage{ltxcmds}
%<ifluatex>\hobsub@StartPackage{ifluatex}
%<ifvtex>\hobsub@StartPackage{ifvtex}
%<intcalc>\hobsub@StartPackage{intcalc}
%<ifpdf>\hobsub@StartPackage{ifpdf}
%<etexcmds>\hobsub@StartPackage{etexcmds}
%<kvsetkeys>\hobsub@StartPackage{kvsetkeys}
%<kvdefinekeys>\hobsub@StartPackage{kvdefinekeys}
%<*luatex-loader>
\ifluatex
\else
  \expandafter\hobsub@GobbleRemainingPackage
\fi
\hobsub@StartPackage{luatex-loader}
%</luatex-loader>
%<pdftexcmds>\hobsub@StartPackage{pdftexcmds}
%<pdfescape>\hobsub@StartPackage{pdfescape}
%<bigintcalc>\hobsub@StartPackage{bigintcalc}
%<bitset>\hobsub@StartPackage{bitset}
%<uniquecounter>\hobsub@StartPackage{uniquecounter}
%<letltxmacro>\hobsub@StartPackage{letltxmacro}
%<hopatch>\hobsub@StartPackage{hopatch}
%<xcolor-patch>\hobsub@StartPackage{xcolor-patch}
%<atveryend>\hobsub@StartPackage{atveryend}
%<atbegshi>\hobsub@StartPackage{atbegshi}
%<refcount>\hobsub@StartPackage{refcount}
%<hycolor>\hobsub@StartPackage{hycolor}
%    \end{macrocode}
%
% \subsection{End of packages}
%
%    \begin{macrocode}
%<package|atend>\HOBsub@AtEnd%
%    \end{macrocode}
%
% \section{Test}
%
% \subsection{Catcode checks for loading}
%
%    \begin{macrocode}
%<*test1>
%    \end{macrocode}
%    \begin{macrocode}
\catcode`\{=1 %
\catcode`\}=2 %
\catcode`\#=6 %
\catcode`\@=11 %
\expandafter\ifx\csname count@\endcsname\relax
  \countdef\count@=255 %
\fi
\expandafter\ifx\csname @gobble\endcsname\relax
  \long\def\@gobble#1{}%
\fi
\expandafter\ifx\csname @firstofone\endcsname\relax
  \long\def\@firstofone#1{#1}%
\fi
\expandafter\ifx\csname loop\endcsname\relax
  \expandafter\@firstofone
\else
  \expandafter\@gobble
\fi
{%
  \def\loop#1\repeat{%
    \def\body{#1}%
    \iterate
  }%
  \def\iterate{%
    \body
      \let\next\iterate
    \else
      \let\next\relax
    \fi
    \next
  }%
  \let\repeat=\fi
}%
\def\RestoreCatcodes{}
\count@=0 %
\loop
  \edef\RestoreCatcodes{%
    \RestoreCatcodes
    \catcode\the\count@=\the\catcode\count@\relax
  }%
\ifnum\count@<255 %
  \advance\count@ 1 %
\repeat

\def\RangeCatcodeInvalid#1#2{%
  \count@=#1\relax
  \loop
    \catcode\count@=15 %
  \ifnum\count@<#2\relax
    \advance\count@ 1 %
  \repeat
}
\def\RangeCatcodeCheck#1#2#3{%
  \count@=#1\relax
  \loop
    \ifnum#3=\catcode\count@
    \else
      \errmessage{%
        Character \the\count@\space
        with wrong catcode \the\catcode\count@\space
        instead of \number#3%
      }%
    \fi
  \ifnum\count@<#2\relax
    \advance\count@ 1 %
  \repeat
}
\def\space{ }
\expandafter\ifx\csname LoadCommand\endcsname\relax
  \def\LoadCommand{\input hobsub.sty\relax}%
\fi
\def\Test{%
  \RangeCatcodeInvalid{0}{47}%
  \RangeCatcodeInvalid{58}{64}%
  \RangeCatcodeInvalid{91}{96}%
  \RangeCatcodeInvalid{123}{255}%
  \catcode`\@=12 %
  \catcode`\\=0 %
  \catcode`\%=14 %
  \LoadCommand
  \RangeCatcodeCheck{0}{36}{15}%
  \RangeCatcodeCheck{37}{37}{14}%
  \RangeCatcodeCheck{38}{47}{15}%
  \RangeCatcodeCheck{48}{57}{12}%
  \RangeCatcodeCheck{58}{63}{15}%
  \RangeCatcodeCheck{64}{64}{12}%
  \RangeCatcodeCheck{65}{90}{11}%
  \RangeCatcodeCheck{91}{91}{15}%
  \RangeCatcodeCheck{92}{92}{0}%
  \RangeCatcodeCheck{93}{96}{15}%
  \RangeCatcodeCheck{97}{122}{11}%
  \RangeCatcodeCheck{123}{255}{15}%
  \RestoreCatcodes
}
\Test
\csname @@end\endcsname
\end
%    \end{macrocode}
%    \begin{macrocode}
%</test1>
%    \end{macrocode}
%
% \section{Installation}
%
% \subsection{Download}
%
% \paragraph{Package.} This package is available on
% CTAN\footnote{\url{http://ctan.org/pkg/hobsub}}:
% \begin{description}
% \item[\CTAN{macros/latex/contrib/oberdiek/hobsub.dtx}] The source file.
% \item[\CTAN{macros/latex/contrib/oberdiek/hobsub.pdf}] Documentation.
% \end{description}
%
%
% \paragraph{Bundle.} All the packages of the bundle `oberdiek'
% are also available in a TDS compliant ZIP archive. There
% the packages are already unpacked and the documentation files
% are generated. The files and directories obey the TDS standard.
% \begin{description}
% \item[\CTAN{install/macros/latex/contrib/oberdiek.tds.zip}]
% \end{description}
% \emph{TDS} refers to the standard ``A Directory Structure
% for \TeX\ Files'' (\CTAN{tds/tds.pdf}). Directories
% with \xfile{texmf} in their name are usually organized this way.
%
% \subsection{Bundle installation}
%
% \paragraph{Unpacking.} Unpack the \xfile{oberdiek.tds.zip} in the
% TDS tree (also known as \xfile{texmf} tree) of your choice.
% Example (linux):
% \begin{quote}
%   |unzip oberdiek.tds.zip -d ~/texmf|
% \end{quote}
%
% \paragraph{Script installation.}
% Check the directory \xfile{TDS:scripts/oberdiek/} for
% scripts that need further installation steps.
% Package \xpackage{attachfile2} comes with the Perl script
% \xfile{pdfatfi.pl} that should be installed in such a way
% that it can be called as \texttt{pdfatfi}.
% Example (linux):
% \begin{quote}
%   |chmod +x scripts/oberdiek/pdfatfi.pl|\\
%   |cp scripts/oberdiek/pdfatfi.pl /usr/local/bin/|
% \end{quote}
%
% \subsection{Package installation}
%
% \paragraph{Unpacking.} The \xfile{.dtx} file is a self-extracting
% \docstrip\ archive. The files are extracted by running the
% \xfile{.dtx} through \plainTeX:
% \begin{quote}
%   \verb|tex hobsub.dtx|
% \end{quote}
%
% \paragraph{TDS.} Now the different files must be moved into
% the different directories in your installation TDS tree
% (also known as \xfile{texmf} tree):
% \begin{quote}
% \def\t{^^A
% \begin{tabular}{@{}>{\ttfamily}l@{ $\rightarrow$ }>{\ttfamily}l@{}}
%   hobsub.sty & tex/generic/oberdiek/hobsub.sty\\
%   hobsub-generic.sty & tex/generic/oberdiek/hobsub-generic.sty\\
%   hobsub-hyperref.sty & tex/generic/oberdiek/hobsub-hyperref.sty\\
%   hobsub.pdf & doc/latex/oberdiek/hobsub.pdf\\
%   test/hobsub-test1.tex & doc/latex/oberdiek/test/hobsub-test1.tex\\
%   hobsub.dtx & source/latex/oberdiek/hobsub.dtx\\
% \end{tabular}^^A
% }^^A
% \sbox0{\t}^^A
% \ifdim\wd0>\linewidth
%   \begingroup
%     \advance\linewidth by\leftmargin
%     \advance\linewidth by\rightmargin
%   \edef\x{\endgroup
%     \def\noexpand\lw{\the\linewidth}^^A
%   }\x
%   \def\lwbox{^^A
%     \leavevmode
%     \hbox to \linewidth{^^A
%       \kern-\leftmargin\relax
%       \hss
%       \usebox0
%       \hss
%       \kern-\rightmargin\relax
%     }^^A
%   }^^A
%   \ifdim\wd0>\lw
%     \sbox0{\small\t}^^A
%     \ifdim\wd0>\linewidth
%       \ifdim\wd0>\lw
%         \sbox0{\footnotesize\t}^^A
%         \ifdim\wd0>\linewidth
%           \ifdim\wd0>\lw
%             \sbox0{\scriptsize\t}^^A
%             \ifdim\wd0>\linewidth
%               \ifdim\wd0>\lw
%                 \sbox0{\tiny\t}^^A
%                 \ifdim\wd0>\linewidth
%                   \lwbox
%                 \else
%                   \usebox0
%                 \fi
%               \else
%                 \lwbox
%               \fi
%             \else
%               \usebox0
%             \fi
%           \else
%             \lwbox
%           \fi
%         \else
%           \usebox0
%         \fi
%       \else
%         \lwbox
%       \fi
%     \else
%       \usebox0
%     \fi
%   \else
%     \lwbox
%   \fi
% \else
%   \usebox0
% \fi
% \end{quote}
% If you have a \xfile{docstrip.cfg} that configures and enables \docstrip's
% TDS installing feature, then some files can already be in the right
% place, see the documentation of \docstrip.
%
% \subsection{Refresh file name databases}
%
% If your \TeX~distribution
% (\teTeX, \mikTeX, \dots) relies on file name databases, you must refresh
% these. For example, \teTeX\ users run \verb|texhash| or
% \verb|mktexlsr|.
%
% \subsection{Some details for the interested}
%
% \paragraph{Attached source.}
%
% The PDF documentation on CTAN also includes the
% \xfile{.dtx} source file. It can be extracted by
% AcrobatReader 6 or higher. Another option is \textsf{pdftk},
% e.g. unpack the file into the current directory:
% \begin{quote}
%   \verb|pdftk hobsub.pdf unpack_files output .|
% \end{quote}
%
% \paragraph{Unpacking with \LaTeX.}
% The \xfile{.dtx} chooses its action depending on the format:
% \begin{description}
% \item[\plainTeX:] Run \docstrip\ and extract the files.
% \item[\LaTeX:] Generate the documentation.
% \end{description}
% If you insist on using \LaTeX\ for \docstrip\ (really,
% \docstrip\ does not need \LaTeX), then inform the autodetect routine
% about your intention:
% \begin{quote}
%   \verb|latex \let\install=y% \iffalse meta-comment
%
% File: hobsub.dtx
% Version: 2016/05/16 v1.14
% Info: Construct package bundles
%
% Copyright (C) 2011, 2012 by
%    Heiko Oberdiek <heiko.oberdiek at googlemail.com>
%    2016
%    https://github.com/ho-tex/oberdiek/issues
%
% This work may be distributed and/or modified under the
% conditions of the LaTeX Project Public License, either
% version 1.3c of this license or (at your option) any later
% version. This version of this license is in
%    http://www.latex-project.org/lppl/lppl-1-3c.txt
% and the latest version of this license is in
%    http://www.latex-project.org/lppl.txt
% and version 1.3 or later is part of all distributions of
% LaTeX version 2005/12/01 or later.
%
% This work has the LPPL maintenance status "maintained".
%
% This Current Maintainer of this work is Heiko Oberdiek.
%
% The Base Interpreter refers to any `TeX-Format',
% because some files are installed in TDS:tex/generic//.
%
% This work consists of the main source file hobsub.dtx
% and the derived files
%    hobsub.sty, hobsub.pdf, hobsub.ins, hobsub.drv, hobsub-generic.sty,
%    hobsub-hyperref.sty, hobsub-test1.tex.
%
% Distribution:
%    CTAN:macros/latex/contrib/oberdiek/hobsub.dtx
%    CTAN:macros/latex/contrib/oberdiek/hobsub.pdf
%
% Unpacking:
%    (a) If hobsub.ins is present:
%           tex hobsub.ins
%    (b) Without hobsub.ins:
%           tex hobsub.dtx
%    (c) If you insist on using LaTeX
%           latex \let\install=y% \iffalse meta-comment
%
% File: hobsub.dtx
% Version: 2016/05/16 v1.14
% Info: Construct package bundles
%
% Copyright (C) 2011, 2012 by
%    Heiko Oberdiek <heiko.oberdiek at googlemail.com>
%    2016
%    https://github.com/ho-tex/oberdiek/issues
%
% This work may be distributed and/or modified under the
% conditions of the LaTeX Project Public License, either
% version 1.3c of this license or (at your option) any later
% version. This version of this license is in
%    http://www.latex-project.org/lppl/lppl-1-3c.txt
% and the latest version of this license is in
%    http://www.latex-project.org/lppl.txt
% and version 1.3 or later is part of all distributions of
% LaTeX version 2005/12/01 or later.
%
% This work has the LPPL maintenance status "maintained".
%
% This Current Maintainer of this work is Heiko Oberdiek.
%
% The Base Interpreter refers to any `TeX-Format',
% because some files are installed in TDS:tex/generic//.
%
% This work consists of the main source file hobsub.dtx
% and the derived files
%    hobsub.sty, hobsub.pdf, hobsub.ins, hobsub.drv, hobsub-generic.sty,
%    hobsub-hyperref.sty, hobsub-test1.tex.
%
% Distribution:
%    CTAN:macros/latex/contrib/oberdiek/hobsub.dtx
%    CTAN:macros/latex/contrib/oberdiek/hobsub.pdf
%
% Unpacking:
%    (a) If hobsub.ins is present:
%           tex hobsub.ins
%    (b) Without hobsub.ins:
%           tex hobsub.dtx
%    (c) If you insist on using LaTeX
%           latex \let\install=y\input{hobsub.dtx}
%        (quote the arguments according to the demands of your shell)
%
% Documentation:
%    (a) If hobsub.drv is present:
%           latex hobsub.drv
%    (b) Without hobsub.drv:
%           latex hobsub.dtx; ...
%    The class ltxdoc loads the configuration file ltxdoc.cfg
%    if available. Here you can specify further options, e.g.
%    use A4 as paper format:
%       \PassOptionsToClass{a4paper}{article}
%
%    Programm calls to get the documentation (example):
%       pdflatex hobsub.dtx
%       makeindex -s gind.ist hobsub.idx
%       pdflatex hobsub.dtx
%       makeindex -s gind.ist hobsub.idx
%       pdflatex hobsub.dtx
%
% Installation:
%    TDS:tex/generic/oberdiek/hobsub.sty
%    TDS:tex/generic/oberdiek/hobsub-generic.sty
%    TDS:tex/generic/oberdiek/hobsub-hyperref.sty
%    TDS:doc/latex/oberdiek/hobsub.pdf
%    TDS:doc/latex/oberdiek/test/hobsub-test1.tex
%    TDS:source/latex/oberdiek/hobsub.dtx
%
%<*ignore>
\begingroup
  \catcode123=1 %
  \catcode125=2 %
  \def\x{LaTeX2e}%
\expandafter\endgroup
\ifcase 0\ifx\install y1\fi\expandafter
         \ifx\csname processbatchFile\endcsname\relax\else1\fi
         \ifx\fmtname\x\else 1\fi\relax
\else\csname fi\endcsname
%</ignore>
%<*install>
\input docstrip.tex
\Msg{************************************************************************}
\Msg{* Installation}
\Msg{* Package: hobsub 2016/05/16 v1.14 Construct package bundles (HO)}
\Msg{************************************************************************}

\keepsilent
\askforoverwritefalse

\let\MetaPrefix\relax
\preamble

This is a generated file.

Project: hobsub
Version: 2016/05/16 v1.14

Copyright (C) 2011, 2012 by
   Heiko Oberdiek <heiko.oberdiek at googlemail.com>

This work may be distributed and/or modified under the
conditions of the LaTeX Project Public License, either
version 1.3c of this license or (at your option) any later
version. This version of this license is in
   http://www.latex-project.org/lppl/lppl-1-3c.txt
and the latest version of this license is in
   http://www.latex-project.org/lppl.txt
and version 1.3 or later is part of all distributions of
LaTeX version 2005/12/01 or later.

This work has the LPPL maintenance status "maintained".

This Current Maintainer of this work is Heiko Oberdiek.

The Base Interpreter refers to any `TeX-Format',
because some files are installed in TDS:tex/generic//.

This work consists of the main source file hobsub.dtx
and the derived files
   hobsub.sty, hobsub.pdf, hobsub.ins, hobsub.drv, hobsub-generic.sty,
   hobsub-hyperref.sty, hobsub-test1.tex.

\endpreamble
\let\MetaPrefix\DoubleperCent

\generate{%
  \file{hobsub.ins}{\from{hobsub.dtx}{install}}%
  \file{hobsub.drv}{\from{hobsub.dtx}{driver}}%
  \usedir{tex/generic/oberdiek}%
  \file{hobsub.sty}{\from{hobsub.dtx}{package}}%
  \usedir{doc/latex/oberdiek/test}%
  \file{hobsub-test1.tex}{\from{hobsub.dtx}{test1}}%
  \nopreamble
  \nopostamble
  \usedir{source/latex/oberdiek/catalogue}%
  \file{hobsub.xml}{\from{hobsub.dtx}{catalogue}}%
}
\let\hobsubSavedNormalLine\normalLine
\def\hobsubGobble#1{}
\begingroup
  \lccode`\0=32 % space
  \lccode`\1=35 % hash
  \lccode`\2=37 % percent
  \lccode`\3=92 % backslash
  \lccode`\4=123 % {
  \lccode`\5=125 % }
\lowercase{\endgroup
  \let\hobsubTokenSpaceOther= 0%
  \let\hobsubTokenHash= 1%
  \let\hobsubTokenPercent= 2%
  \let\hobsubTokenBackslash= 3%
  \let\hobsubTokenBraceLeft= 4%
  \let\hobsubTokenBraceRight= 5%
  \def\hobsubNormalLineWithSpace{%
    \hobsubSavedNormalLine0%
  }%
}
\def\normalLine{%
  \futurelet\hobsubToken\hobsubFirst
}
\def\hobsubFirst{%
  \ifx\hobsubToken\hobsubTokenSpaceOther
    \expandafter\expandafter\expandafter\futurelet
    \expandafter\expandafter\expandafter\hobsubToken
    \expandafter\expandafter\expandafter\hobsubNext
    \expandafter\hobsubGobble
  \else
    \expandafter\hobsubSavedNormalLine
  \fi
}
\def\hobsubNext{%
  \ifx\hobsubToken\hobsubTokenSpaceOther
    \expandafter\expandafter\expandafter\futurelet
    \expandafter\expandafter\expandafter\hobsubToken
    \expandafter\expandafter\expandafter\hobsubNext 
    \expandafter\hobsubGobble
  \else
    \let\hobsubWithSpace=Y%
    \ifx\hobsubToken\hobsubTokenBackslash
      \let\hobsubWithSpace=N%
    \else
      \ifx\hobsubToken\hobsubTokenBraceRight
        \let\hobsubWithSpace=N%
      \else
        \ifx\hobsubToken\hobsubTokenBraceLeft
          \let\hobsubWithSpace=N%
        \else
          \ifx\hobsubToken\hobsubTokenPercent
            \let\hobsubWithSpace=N%
          \else
            \ifx\hobsubToken\hobsubTokenHash
              \let\hobsubWithSpace=N%
            \fi
          \fi
        \fi
      \fi
    \fi
    \ifx\hobsubWithSpace Y%
      \expandafter\expandafter\expandafter\hobsubNormalLineWithSpace
    \else
      \expandafter\expandafter\expandafter\hobsubSavedNormalLine
    \fi
  \fi
}
\generate{%
  \usedir{tex/generic/oberdiek}%
  \file{hobsub-generic.sty}{%
    \from{hobsub.dtx}{generic,preamble,infwarerr}%
    \from{infwarerr.dtx}{package}%
    \from{hobsub.dtx}{post,ltxcmds}%
    \from{ltxcmds.dtx}{package}%
    \from{hobsub.dtx}{post,ifluatex}%
    \from{ifluatex.dtx}{package}%
    \from{hobsub.dtx}{post,ifvtex}%
    \from{ifvtex.dtx}{package}%
    \from{hobsub.dtx}{post,intcalc}%
    \from{intcalc.dtx}{package}%
    \from{hobsub.dtx}{post,ifpdf}%
    \from{ifpdf.dtx}{package}%
    \from{hobsub.dtx}{post,etexcmds}%
    \from{etexcmds.dtx}{package}%
    \from{hobsub.dtx}{post,kvsetkeys}%
    \from{kvsetkeys.dtx}{package}%
    \from{hobsub.dtx}{post,kvdefinekeys}%
    \from{kvdefinekeys.dtx}{package}%
    \from{hobsub.dtx}{post,luatex-loader}%
    \from{luatex.dtx}{loader}%
    \from{hobsub.dtx}{post,pdftexcmds}%
    \from{pdftexcmds.dtx}{package}%
    \from{hobsub.dtx}{post,pdfescape}%
    \from{pdfescape.dtx}{package}%
    \from{hobsub.dtx}{post,bigintcalc}%
    \from{bigintcalc.dtx}{package}%
    \from{hobsub.dtx}{post,bitset}%
    \from{bitset.dtx}{package}%
    \from{hobsub.dtx}{post,uniquecounter}%
    \from{uniquecounter.dtx}{package}%
    \from{hobsub.dtx}{post,atend}%
  }%
}
\generate{%
  \usedir{tex/generic/oberdiek}%
  \file{hobsub-hyperref.sty}{%
    \from{hobsub.dtx}{hyperref,preamble,,letltxmacro}%
    \from{letltxmacro.dtx}{package}%
    \from{hobsub.dtx}{post,hopatch}%
    \from{hopatch.dtx}{package}%
    \from{hobsub.dtx}{post,xcolor-patch}%
    \from{hycolor.dtx}{xcolor}%
    \from{hobsub.dtx}{post,atveryend}%
    \from{atveryend.dtx}{package}%
    \from{hobsub.dtx}{post,atbegshi}%
    \from{atbegshi.dtx}{package}%
    \from{hobsub.dtx}{post,refcount}%
    \from{refcount.dtx}{package}%
    \from{hobsub.dtx}{post,hycolor}%
    \from{hycolor.dtx}{package}%
    \from{hobsub.dtx}{post}%
  }%
}
\let\normalLine\hobsubSavedNormalLine

\catcode32=13\relax% active space
\let =\space%
\Msg{************************************************************************}
\Msg{*}
\Msg{* To finish the installation you have to move the following}
\Msg{* files into a directory searched by TeX:}
\Msg{*}
\Msg{*     hobsub.sty, hobsub-generic.sty, hobsub-hyperref.sty}
\Msg{*}
\Msg{* To produce the documentation run the file `hobsub.drv'}
\Msg{* through LaTeX.}
\Msg{*}
\Msg{* Happy TeXing!}
\Msg{*}
\Msg{************************************************************************}

\endbatchfile
%</install>
%<*ignore>
\fi
%</ignore>
%<*driver>
\NeedsTeXFormat{LaTeX2e}
\ProvidesFile{hobsub.drv}%
  [2016/05/16 v1.14 Construct package bundles (HO)]%
\documentclass{ltxdoc}
\usepackage{holtxdoc}[2011/11/22]
\usepackage{array}
\usepackage{caption}
\renewcommand*{\bottomfraction}{.5}
\renewcommand*{\topfraction}{.5}
\begin{document}
  \DocInput{hobsub.dtx}%
\end{document}
%</driver>
% \fi
%
%
% \CharacterTable
%  {Upper-case    \A\B\C\D\E\F\G\H\I\J\K\L\M\N\O\P\Q\R\S\T\U\V\W\X\Y\Z
%   Lower-case    \a\b\c\d\e\f\g\h\i\j\k\l\m\n\o\p\q\r\s\t\u\v\w\x\y\z
%   Digits        \0\1\2\3\4\5\6\7\8\9
%   Exclamation   \!     Double quote  \"     Hash (number) \#
%   Dollar        \$     Percent       \%     Ampersand     \&
%   Acute accent  \'     Left paren    \(     Right paren   \)
%   Asterisk      \*     Plus          \+     Comma         \,
%   Minus         \-     Point         \.     Solidus       \/
%   Colon         \:     Semicolon     \;     Less than     \<
%   Equals        \=     Greater than  \>     Question mark \?
%   Commercial at \@     Left bracket  \[     Backslash     \\
%   Right bracket \]     Circumflex    \^     Underscore    \_
%   Grave accent  \`     Left brace    \{     Vertical bar  \|
%   Right brace   \}     Tilde         \~}
%
% \GetFileInfo{hobsub.drv}
%
% \title{The \xpackage{hobsub} package}
% \date{2016/05/16 v1.14}
% \author{Heiko Oberdiek\thanks
% {Please report any issues at https://github.com/ho-tex/oberdiek/issues}\\
% \xemail{heiko.oberdiek at googlemail.com}}
%
% \maketitle
%
% \begin{abstract}
% Package \xpackage{hobsub} implements the idea of loading
% several packages in one big collection package.
% \end{abstract}
%
% \tableofcontents
%
% \def\TableGeneric{^^A
%   \TableEntry{\xpackage{infwarerr}}{}{2010/04/08 v1.3}^^A
%   \TableEntry{\xpackage{ltxcmds}}{}{2011/11/09 v1.22}^^A
%   \TableEntry{\xpackage{ifluatex}}{}{2010/03/01 v1.3}^^A
%   \TableEntry{\xpackage{ifvtex}}{}{2010/03/01 v1.5}^^A
%   \TableEntry{\xpackage{intcalc}}{}{2007/09/27 v1.1}^^A
%   \TableEntry{\xpackage{ifpdf}}{\xpackage{ifluatex}}{2011/01/30 v2.3}^^A
%   \TableEntry{\xpackage{etexcmds}}{}{2011/02/16 v1.5}^^A
%   \TableEntry{\xpackage{kvsetkeys}}{}{2012/04/25 v1.16}^^A
%   \TableEntry{\xpackage{kvdefinekeys}}{}{2011/04/07 v1.3}^^A
%   \TableEntry{\xpackage{luatex-loader}}{}{2010/03/09 v0.4}^^A
%   \TableEntry{\xpackage{pdftexcmds}}{\xpackage{ifluatex}, \xpackage{ifpdf}, \xpackage{infwarerr}, \xpackage{ltxcmds}, \xpackage{luatex-loader}}{2011/11/29 v0.20}^^A
%   \TableEntry{\xpackage{pdfescape}}{\xpackage{ltxcmds}, \xpackage{pdftexcmds}}{2011/11/25 v1.13}^^A
%   \TableEntry{\xpackage{bigintcalc}}{\xpackage{pdftexcmds}}{2012/04/08 v1.3}^^A
%   \TableEntry{\xpackage{bitset}}{\xpackage{bigintcalc}, \xpackage{infwarerr}, \xpackage{intcalc}}{2011/01/30 v1.1}^^A
%   \TableEntry{\xpackage{uniquecounter}}{\xpackage{bigintcalc}, \xpackage{infwarerr}}{2011/01/30 v1.2}^^A
% }
% \def\TableHyperref{^^A
%   \TableEntry{\xpackage{letltxmacro}}{}{2010/09/02 v1.4}^^A
%   \TableEntry{\xpackage{hopatch}}{}{2016/05/16 v1.2}^^A
%   \TableEntry{\xpackage{xcolor-patch}}{}{2011/01/30 v1.7}^^A
%   \TableEntry{\xpackage{atveryend}}{}{2011/06/30 v1.8}^^A
%   \TableEntry{\xpackage{atbegshi}}{\xpackage{ifpdf}, \xpackage{infwarerr}, \xpackage{ltxcmds}}{2011/10/05 v1.16}^^A
%   \TableEntry{\xpackage{refcount}}{\xpackage{infwarerr}, \xpackage{ltxcmds}}{2011/10/16 v3.4}^^A
%   \TableEntry{\xpackage{hycolor}}{\xpackage{xcolor-patch}}{2011/01/30 v1.7}^^A
% }
%
% \newenvironment{TableList}[2][htbp]{^^A
%   \begin{table}[{#1}]^^A
%   \centering
%   \caption{Overview \xpackage{hobsub-#2}}^^A
%   \label{tab:#2}^^A
%   \begin{tabular}{ll>{\raggedright}p{.4\linewidth}}^^A
%   \bfseries Package & \bfseries Release &
%   \bfseries Dependencies\tabularnewline
%   \hline
% }{^^A
%   \end{tabular}^^A
%   \end{table}^^A
% }
% \newcommand*{\TableEntry}[3]{%
%   #1&#3&#2\tabularnewline
% }
%
% \renewcommand*{\arraystretch}{1.1}
% \setlength{\extrarowheight}{2pt}
%
% \section{Documentation}
%
% In January 2011 the mailing list lualatex-dev discussed
% ``quietening lualatex console output'' \cite{lualatex:quiet}.
% Inspired from this, I implemented this package \xpackage{hobsub}
% to load several packages in one file. The package is applied for
% two packages collections that are subsets of my package bundle
% \xpackage{oberdiek}.
% \begin{description}
% \item[\xpackage{hobsub-generic}] is a collection of packages
%   that can also be used with \hologo{plainTeX}.
% \item[\xpackage{hobsub-hyperref}] is a collection of packages
%   that is used by package \xpackage{hyperref}.
% \end{description}
%
% Package \xpackage{hobsub} provides the macros that
% are used in the collection packages and might be used
% in other projects as well. But there are many caveats,
% some of them are discussed below.
%
% The most serious problem is consistency. A package
% that is available as standalone package and is part
% of a collection might have different versions.
% Because my packages are organized in a bundle and should
% be installed and updated as bundle, the risk seems low
% to me. However, if a single package is updated manually, for
% example for testing, then it should be loaded before the
% collection that contains the package with an older version.
% In a collection, the package will not be loaded again if it
% is already loaded.
%
% \subsection{Usage}
%
% The collection packages \xpackage{hobsub-generic} and
% \xpackage{hobsub-hyperref} are loaded as usual:
% \begin{quote}
%  \begin{tabular}{@{}ll@{}}
%    |\usepackage{hobsub-generic}| & \hologo{LaTeX}\\
%    |\input hobsub-generic.sty| & \hologo{plainTeX}
%  \end{tabular}
% \end{quote}
% or
% \begin{quote}|\usepackage{hobsub-hyperref}|\end{quote}
% Of course, both \cs{usepackage} or \cs{RequirePackage}
% can be used.
%
% If you need requirements on the version date of a package
% inside a collection, then specify them afterwards, for
% example:
% \begin{quote}
%   |\usepackage{hobsub-generic}|\\
%   |\usepackage{ltxcmds}[2010/12/04]% |^^A
%   {\itshape |\ltx@ifblank is needed|}
% \end{quote}
%
% Also it is not garanteed that a package will be always
% part of a collection. Therefore it does not harm,
% to make the requirements explicite \emph{after}
% a collection is loaded, see the previous example.
% \hologo{LaTeX} knows that a
% package is already loaded and does not load it again.
%
% \subsection{Supported features}
%
% \begin{description}
% \item[No reloads:] If a package is already loaded then it
%  is not reloaded again if it is part of a collection package.
% \item[\cs{listfiles}:]
%   Each package that is loaded in a collection is also
%   added to the output of \cs{listfiles}.
% \item[\cs{@currname}, \cs{@currext}]
%   are updated, they are used by \hologo{LaTeX}'s
%  \cs{ProvidesPackage}.
% \item[\cs{AtEndOfPackage}:]
%   \hologo{LaTeX}'s hook mechanism that is used for
%   \cs{AtEndOfPackage} is supported. However none of the
%   packages in the collections are using \cs{AtEndOfPackage},
%   therefore this feature is completely untested.
% \end{description}
%
% \subsection{Common limitations}
%
% \begin{description}
% \item[\cs{endinput}:]
%   Often a package is using \cs{endinput}, either added by
%   \xpackage{docstrip}, or to stop package loading at an earlier
%   time. Therefore \cs{endinput} must be catched during package
%   loading to prevent that the collection package is closed
%   too early before reading all packages it contains.
%
%   This redefinition of \cs{endinput} fails if the package loads
%   other files that contain \cs{endinput}. To some degree it could
%   be catched by redefining macros that load files. But this will
%   not work for the primitive \cs{input}.
% \item[Options:]
%   \hologo{LaTeX}'s package options are not supported.
%   Therefore the packages that are put in a collection must
%   not support options.
% \item[Hook packages:]
%   Some packages and classes provides hooks for packages, see
%   packages \xpackage{scrlfile}, \xpackage{filehook}
%   or class \xclass{memoir}. These hooks are not supported
%   for embedded packages inside a collection package.
%   Because of the problem with \cs{endinput}, these hooks
%   are not supported. The hooks might contain code that
%   loads other files that execute \cs{endinput}.
% \end{description}
%
% \subsection{Package \xpackage{hobsub-generic}}
%
% \begin{TableList}[bp]{generic}\TableGeneric\end{TableList}
%
% Collection `generic' contains some of the \hologo{plainTeX} compatible
% packages of my bundle, see table \ref{tab:generic}.
% It adds the following restrictions to any package
% that is part of this collection:
% \begin{description}
% \item[Generic formats:] The following formats must be supported:
%   \hologo{LaTeX}, \hologo{plainTeX} and even \hologo{iniTeX}.
%   This applies to the collection package \xpackage{hobsub-generic}
%   the same way.
% \item[Prefixed macros:]
%   Any package of this collection should only define macros with
%   prefixed names to avoid name clashes. Resources like registers
%   must not be allocated and other global states should not be
%   changed except for added macro definitions.
% \end{description}
% Thus the package \xpackage{hobsub-generic} can also preloaded
% in the format file with \hologo{iniTeX}.
%
% \subsection{Package \xpackage{hobsub-hyperref}}
%
% \begin{TableList}[tp]{hyperref}\TableHyperref\end{TableList}
%
% Collection `hyperref' contains additional packages that are
% used by package \xpackage{hyperref} and that does not fit
% in collection `generic', see table{tab:hyperref}.
% The latter collection package
% \xpackage{hobsub-generic} is loaded via a conventional
% \cs{RequirePackage}.
%
% Because most of the packages in the collection `hyperref'
% needs \hologo{LaTeX}, the collection requires \hologo{LaTeX}.
%
% \subsection{Package \xpackage{hobsub}}
%
% This section is for advanced users that might want to
% build a collection package. Then it can be done by
% using \xpackage{hobsub}:
% \begin{quote}
%   \dots\\
%   |\RequirePackage{hobsub}[2016/05/16]|\\
%   \dots\\
%   |\hobsub@StartPackage{foo1}|\\
%   \textit{\dots\unskip\ package code of \xfile{foo1.sty} \dots}\\
%   |\endinput|\\
%   |<hobsub>|\\
%   |\hobsub@StopPackage|\\
%   |\hobsub@StartPackage{bar2}|\\
%   \textit{\dots\unskip\ package code of \xfile{bar2.sty} \dots}\\
%   |\endinput|\\
%   |<hobsub>|\\
%   |\hobsub@StopPackage|
%   \dots
% \end{quote}
% But be aware of the caveats and limitations, see above.
% Only quite a small portion of the packages can be embedded
% like this.
%
% \clearpage
%
% \StopEventually{
% }
%
% \section{Implementation}
%
% \subsection{Reload check and package identification}
%
% \subsubsection{Package \xpackage{hobsub}}
%
%    \begin{macrocode}
%<*package>
%    \end{macrocode}
%    Reload check, especially if the package is not used with \LaTeX.
%    \begin{macrocode}
\begingroup\catcode61\catcode48\catcode32=10\relax%
  \catcode13=5 % ^^M
  \endlinechar=13 %
  \catcode35=6 % #
  \catcode39=12 % '
  \catcode44=12 % ,
  \catcode45=12 % -
  \catcode46=12 % .
  \catcode58=12 % :
  \catcode64=11 % @
  \catcode123=1 % {
  \catcode125=2 % }
  \expandafter\let\expandafter\x\csname ver@hobsub.sty\endcsname
  \ifx\x\relax % plain-TeX, first loading
  \else
    \def\empty{}%
    \ifx\x\empty % LaTeX, first loading,
      % variable is initialized, but \ProvidesPackage not yet seen
    \else
      \expandafter\ifx\csname PackageInfo\endcsname\relax
        \def\x#1#2{%
          \immediate\write-1{Package #1 Info: #2.}%
        }%
      \else
        \def\x#1#2{\PackageInfo{#1}{#2, stopped}}%
      \fi
      \x{hobsub}{The package is already loaded}%
      \aftergroup\endinput
    \fi
  \fi
\endgroup%
%    \end{macrocode}
%    Package identification:
%    \begin{macrocode}
\begingroup\catcode61\catcode48\catcode32=10\relax%
  \catcode13=5 % ^^M
  \endlinechar=13 %
  \catcode35=6 % #
  \catcode39=12 % '
  \catcode40=12 % (
  \catcode41=12 % )
  \catcode44=12 % ,
  \catcode45=12 % -
  \catcode46=12 % .
  \catcode47=12 % /
  \catcode58=12 % :
  \catcode64=11 % @
  \catcode91=12 % [
  \catcode93=12 % ]
  \catcode123=1 % {
  \catcode125=2 % }
  \expandafter\ifx\csname ProvidesPackage\endcsname\relax
    \def\x#1#2#3[#4]{\endgroup
      \immediate\write-1{Package: #3 #4}%
      \xdef#1{#4}%
    }%
  \else
    \def\x#1#2[#3]{\endgroup
      #2[{#3}]%
      \ifx#1\@undefined
        \xdef#1{#3}%
      \fi
      \ifx#1\relax
        \xdef#1{#3}%
      \fi
    }%
  \fi
\expandafter\x\csname ver@hobsub.sty\endcsname
\ProvidesPackage{hobsub}%
  [2016/05/16 v1.14 Construct package bundles (HO)]%
%    \end{macrocode}
%    \begin{macrocode}
%</package>
%    \end{macrocode}
%
% \subsubsection{Package \xpackage{hobsub-generic}}
%
%    \begin{macrocode}
%<*generic>
%    \end{macrocode}
%    Reload check, especially if the package is not used with \LaTeX.
%    \begin{macrocode}
\begingroup\catcode61\catcode48\catcode32=10\relax%
  \catcode13=5 % ^^M
  \endlinechar=13 %
  \catcode35=6 % #
  \catcode39=12 % '
  \catcode44=12 % ,
  \catcode45=12 % -
  \catcode46=12 % .
  \catcode58=12 % :
  \catcode64=11 % @
  \catcode123=1 % {
  \catcode125=2 % }
  \expandafter\let\expandafter\x\csname ver@hobsub-generic.sty\endcsname
  \ifx\x\relax % plain-TeX, first loading
  \else
    \def\empty{}%
    \ifx\x\empty % LaTeX, first loading,
      % variable is initialized, but \ProvidesPackage not yet seen
    \else
      \expandafter\ifx\csname PackageInfo\endcsname\relax
        \def\x#1#2{%
          \immediate\write-1{Package #1 Info: #2.}%
        }%
      \else
        \def\x#1#2{\PackageInfo{#1}{#2, stopped}}%
      \fi
      \x{hobsub-generic}{The package is already loaded}%
      \aftergroup\endinput
    \fi
  \fi
\endgroup%
%    \end{macrocode}
%    Package identification:
%    \begin{macrocode}
\begingroup\catcode61\catcode48\catcode32=10\relax%
  \catcode13=5 % ^^M
  \endlinechar=13 %
  \catcode35=6 % #
  \catcode39=12 % '
  \catcode40=12 % (
  \catcode41=12 % )
  \catcode44=12 % ,
  \catcode45=12 % -
  \catcode46=12 % .
  \catcode47=12 % /
  \catcode58=12 % :
  \catcode64=11 % @
  \catcode91=12 % [
  \catcode93=12 % ]
  \catcode123=1 % {
  \catcode125=2 % }
  \expandafter\ifx\csname ProvidesPackage\endcsname\relax
    \def\x#1#2#3[#4]{\endgroup
      \immediate\write-1{Package: #3 #4}%
      \xdef#1{#4}%
    }%
  \else
    \def\x#1#2[#3]{\endgroup
      #2[{#3}]%
      \ifx#1\@undefined
        \xdef#1{#3}%
      \fi
      \ifx#1\relax
        \xdef#1{#3}%
      \fi
    }%
  \fi
\expandafter\x\csname ver@hobsub-generic.sty\endcsname
\ProvidesPackage{hobsub-generic}%
  [2016/05/16 v1.14 Bundle oberdiek, subset generic (HO)]%
%    \end{macrocode}
%    \begin{macrocode}
%</generic>
%    \end{macrocode}
%
% \subsubsection{Package \xpackage{hobsub-hyperref}}
%
%    \begin{macrocode}
%<*hyperref>
%    \end{macrocode}
%    Reload check, especially if the package is not used with \LaTeX.
%    \begin{macrocode}
\begingroup\catcode61\catcode48\catcode32=10\relax%
  \catcode13=5 % ^^M
  \endlinechar=13 %
  \catcode35=6 % #
  \catcode39=12 % '
  \catcode44=12 % ,
  \catcode45=12 % -
  \catcode46=12 % .
  \catcode58=12 % :
  \catcode64=11 % @
  \catcode123=1 % {
  \catcode125=2 % }
  \expandafter\let\expandafter\x\csname ver@hobsub-hyperref.sty\endcsname
  \ifx\x\relax % plain-TeX, first loading
  \else
    \def\empty{}%
    \ifx\x\empty % LaTeX, first loading,
      % variable is initialized, but \ProvidesPackage not yet seen
    \else
      \expandafter\ifx\csname PackageInfo\endcsname\relax
        \def\x#1#2{%
          \immediate\write-1{Package #1 Info: #2.}%
        }%
      \else
        \def\x#1#2{\PackageInfo{#1}{#2, stopped}}%
      \fi
      \x{hobsub-hyperref}{The package is already loaded}%
      \aftergroup\endinput
    \fi
  \fi
\endgroup%
%    \end{macrocode}
%    Package identification:
%    \begin{macrocode}
\begingroup\catcode61\catcode48\catcode32=10\relax%
  \catcode13=5 % ^^M
  \endlinechar=13 %
  \catcode35=6 % #
  \catcode39=12 % '
  \catcode40=12 % (
  \catcode41=12 % )
  \catcode44=12 % ,
  \catcode45=12 % -
  \catcode46=12 % .
  \catcode47=12 % /
  \catcode58=12 % :
  \catcode64=11 % @
  \catcode91=12 % [
  \catcode93=12 % ]
  \catcode123=1 % {
  \catcode125=2 % }
  \expandafter\ifx\csname ProvidesPackage\endcsname\relax
    \def\x#1#2#3[#4]{\endgroup
      \immediate\write-1{Package: #3 #4}%
      \xdef#1{#4}%
    }%
  \else
    \def\x#1#2[#3]{\endgroup
      #2[{#3}]%
      \ifx#1\@undefined
        \xdef#1{#3}%
      \fi
      \ifx#1\relax
        \xdef#1{#3}%
      \fi
    }%
  \fi
\expandafter\x\csname ver@hobsub-hyperref.sty\endcsname
\ProvidesPackage{hobsub-hyperref}%
  [2016/05/16 v1.14 Bundle oberdiek, subset hyperref (HO)]%
%    \end{macrocode}
%    \begin{macrocode}
%</hyperref>
%    \end{macrocode}
%
% \subsection{Catcodes}
%
%    \begin{macrocode}
%<*package|generic|hyperref>
\begingroup\catcode61\catcode48\catcode32=10\relax%
  \catcode13=5 % ^^M
  \endlinechar=13 %
  \catcode123=1 % {
  \catcode125=2 % }
  \catcode64=11 % @
  \def\x{\endgroup
    \expandafter\edef\csname HOBsub@AtEnd\endcsname{%
      \endlinechar=\the\endlinechar\relax
      \catcode13=\the\catcode13\relax
      \catcode32=\the\catcode32\relax
      \catcode35=\the\catcode35\relax
      \catcode61=\the\catcode61\relax
      \catcode64=\the\catcode64\relax
      \catcode123=\the\catcode123\relax
      \catcode125=\the\catcode125\relax
    }%
  }%
\x\catcode61\catcode48\catcode32=10\relax%
\catcode13=5 % ^^M
\endlinechar=13 %
\catcode35=6 % #
\catcode64=11 % @
\catcode123=1 % {
\catcode125=2 % }
\def\TMP@EnsureCode#1#2{%
  \edef\HOBsub@AtEnd{%
    \HOBsub@AtEnd
    \catcode#1=\the\catcode#1\relax
  }%
  \catcode#1=#2\relax
}
\TMP@EnsureCode{39}{12}% '
\TMP@EnsureCode{40}{12}% (
\TMP@EnsureCode{41}{12}% )
\TMP@EnsureCode{45}{12}% -
\TMP@EnsureCode{46}{12}% .
\TMP@EnsureCode{47}{12}% /
\TMP@EnsureCode{58}{12}% :
\TMP@EnsureCode{60}{12}% <
\TMP@EnsureCode{62}{12}% >
\TMP@EnsureCode{96}{12}% `
\edef\HOBsub@AtEnd{\HOBsub@AtEnd\noexpand\endinput}
%    \end{macrocode}
%
%    \begin{macrocode}
%</package|generic|hyperref>
%    \end{macrocode}
%
% \subsection{Package \xpackage{hobsub-hyperref} loads \xpackage{hobsub-generic}}
%
%    \begin{macrocode}
%<*hyperref>
\NeedsTeXFormat{LaTeX2e}
\RequirePackage{hobsub-generic}[2016/05/16]
%</hyperref>
%    \end{macrocode}
%
% \subsection{Preamble}
%
%    \begin{macrocode}
%<*preamble>
\begingroup\expandafter\expandafter\expandafter\endgroup
\expandafter\ifx\csname ver@hobsub.sty\endcsname\relax
\else
  \expandafter\hobsub@GobbleRemainingPackage
\fi
%</preamble>
%    \end{macrocode}
%
%
%    \begin{macrocode}
%<*package|preamble>
%    \end{macrocode}
%
%    \begin{macro}{\HOBsub@OrgEndinput}
%    \begin{macrocode}
\begingroup\expandafter\expandafter\expandafter\endgroup
\expandafter\ifx\csname HOBsub@OrgEndinput\endcsname\relax
  \let\HOBsub@OrgEndinput\endinput
\fi
%    \end{macrocode}
%    \end{macro}
%    \begin{macro}{\hobsub@GobbleRemainingPackage}
%    \begin{macrocode}
\def\hobsub@GobbleRemainingPackage{%
  \begingroup
  \catcode92=14 % backslash: comment
  \catcode32=14 % space: comment
  \catcode35=14 % hash: comment
  \catcode123=14 % left brace: comment
  \catcode125=14 % right brace: comment
  \catcode60=3 % less: math
  \catcode62=4 % greater: align
  \endlinechar=-1 %
  \HOBsub@GobbleRemainingPackage
}
%    \end{macrocode}
%    \end{macro}
%    \begin{macro}{\HOBsub@GobbleRemainingPackage}
%    \begin{macrocode}
\catcode60=3 % less: dollar
\catcode62=4 % greater: align
\long\def\HOBsub@GobbleRemainingPackage#1<hobsub>{%
  \endgroup
}
\catcode60=12 % less: other
\catcode62=12 % greater: other
%    \end{macrocode}
%    \end{macro}
%
%    \begin{macro}{\hobsub@StartPackage}
%    \begin{macrocode}
\def\hobsub@StartPackage#1{%
  \begingroup\expandafter\expandafter\expandafter\endgroup
  \expandafter\ifx\csname ver@#1.sty\endcsname\relax
    \let\HOBsub@OrgCurrName\@currname
    \let\HOBsub@OrgCurrExt\@currext
    \csname @pushfilename\endcsname
    \def\@currname{#1}%
    \def\@currext{sty}%
    \expandafter\def\csname\@currname.\@currext-h@@k\endcsname{}%
    \let\endinput\hobsub@GobbleRemainingPackage
    \def\hobsub@StopPackage{%
      \let\hobsub@StopPackage\relax
      \HOBsub@StopPackage
%     \hobsub@Info{hobsub}{Package `#1' loaded}%
    }%
    \hobsub@AddToFileList{#1.sty}%
  \else
    \hobsub@Info{hobsub}{Skipping package `#1' (already loaded)}%
    \let\hobsub@StopPackage\relax
    \expandafter\hobsub@GobbleRemainingPackage
  \fi
}
%    \end{macrocode}
%    \end{macro}
%
%    \begin{macro}{\hobsub@StopPackage}
%    \begin{macrocode}
\let\hobsub@StopPackage\relax
%    \end{macrocode}
%    \end{macro}
%
%    \begin{macro}{\hobsub@Info}
%    \begin{macrocode}
\def\hobsub@Info#1#2{%
  \begingroup\expandafter\expandafter\expandafter\endgroup
  \expandafter\ifx\csname @PackageInfoNoLine\endcsname\relax
    \immediate\write-1{Package #1 Info: #2.}%
  \else
    \let\hobsub@Info\@PackageInfoNoLine
    \hobsub@Info{#1}{#2}%
  \fi
}
%    \end{macrocode}
%    \end{macro}
%
%    \begin{macro}{\HOBsub@StopPackage}
%    \begin{macrocode}
\def\HOBsub@StopPackage{%
  \csname\@currname.\@currext-h@@k\endcsname
  \let\endinput\HOBsub@OrgEndinput
  \csname @popfilename\endcsname
  \let\@currname\HOBsub@OrgCurrName
  \let\@currext\HOBsub@OrgCurrExt
}
%    \end{macrocode}
%    \end{macro}
%
%    \begin{macro}{\hobsub@AddToFileList}
%    \begin{macrocode}
\begingroup\expandafter\expandafter\expandafter\endgroup
\expandafter\ifx\csname @addtofilelist\endcsname\relax
  \def\hobsub@AddToFileList#1{}%
\else
  \def\hobsub@AddToFileList#1{%
    \@addtofilelist{#1}%
  }%
\fi
%    \end{macrocode}
%    \end{macro}
%
%    \begin{macrocode}
%</package|preamble>
%    \end{macrocode}
%
% \subsection{End of preamble for subset packages}
%
%    Add the end marker for possible skipping of previous
%    macros, if package \xpackage{hobsub} was already loaded.
%    \begin{macrocode}
%<*preamble>
\hobsub@GobbleRemainingPackage
<hobsub>
%    \end{macrocode}
%
%    All macros of package \xpackage{hobsub} are now available.
%    Now we officially load the package \xpackage{hobsub} inline
%    for proper package identification.
%    \begin{macrocode}
\hobsub@StartPackage{hobsub}%
\begingroup\expandafter\expandafter\expandafter\endgroup
\expandafter\ifx\csname ProvidesPackage\endcsname\relax
  \immediate\write-1{%
    Package: hobsub 2016/05/16 v1.14 Construct package bundles (HO)%
  }%
\else
  \ProvidesPackage{hobsub}%
  [2016/05/16 v1.14 Construct package bundles (HO)]%
\fi
%</preamble>
%    \end{macrocode}
%
%
%    \begin{macrocode}
%<*post|preamble>
 \endinput
<hobsub>
\hobsub@StopPackage
%</post|preamble>
%    \end{macrocode}
%
% \subsection{Package list}
%
%    \begin{macrocode}
%<infwarerr>\hobsub@StartPackage{infwarerr}
%<ltxcmds>\hobsub@StartPackage{ltxcmds}
%<ifluatex>\hobsub@StartPackage{ifluatex}
%<ifvtex>\hobsub@StartPackage{ifvtex}
%<intcalc>\hobsub@StartPackage{intcalc}
%<ifpdf>\hobsub@StartPackage{ifpdf}
%<etexcmds>\hobsub@StartPackage{etexcmds}
%<kvsetkeys>\hobsub@StartPackage{kvsetkeys}
%<kvdefinekeys>\hobsub@StartPackage{kvdefinekeys}
%<*luatex-loader>
\ifluatex
\else
  \expandafter\hobsub@GobbleRemainingPackage
\fi
\hobsub@StartPackage{luatex-loader}
%</luatex-loader>
%<pdftexcmds>\hobsub@StartPackage{pdftexcmds}
%<pdfescape>\hobsub@StartPackage{pdfescape}
%<bigintcalc>\hobsub@StartPackage{bigintcalc}
%<bitset>\hobsub@StartPackage{bitset}
%<uniquecounter>\hobsub@StartPackage{uniquecounter}
%<letltxmacro>\hobsub@StartPackage{letltxmacro}
%<hopatch>\hobsub@StartPackage{hopatch}
%<xcolor-patch>\hobsub@StartPackage{xcolor-patch}
%<atveryend>\hobsub@StartPackage{atveryend}
%<atbegshi>\hobsub@StartPackage{atbegshi}
%<refcount>\hobsub@StartPackage{refcount}
%<hycolor>\hobsub@StartPackage{hycolor}
%    \end{macrocode}
%
% \subsection{End of packages}
%
%    \begin{macrocode}
%<package|atend>\HOBsub@AtEnd%
%    \end{macrocode}
%
% \section{Test}
%
% \subsection{Catcode checks for loading}
%
%    \begin{macrocode}
%<*test1>
%    \end{macrocode}
%    \begin{macrocode}
\catcode`\{=1 %
\catcode`\}=2 %
\catcode`\#=6 %
\catcode`\@=11 %
\expandafter\ifx\csname count@\endcsname\relax
  \countdef\count@=255 %
\fi
\expandafter\ifx\csname @gobble\endcsname\relax
  \long\def\@gobble#1{}%
\fi
\expandafter\ifx\csname @firstofone\endcsname\relax
  \long\def\@firstofone#1{#1}%
\fi
\expandafter\ifx\csname loop\endcsname\relax
  \expandafter\@firstofone
\else
  \expandafter\@gobble
\fi
{%
  \def\loop#1\repeat{%
    \def\body{#1}%
    \iterate
  }%
  \def\iterate{%
    \body
      \let\next\iterate
    \else
      \let\next\relax
    \fi
    \next
  }%
  \let\repeat=\fi
}%
\def\RestoreCatcodes{}
\count@=0 %
\loop
  \edef\RestoreCatcodes{%
    \RestoreCatcodes
    \catcode\the\count@=\the\catcode\count@\relax
  }%
\ifnum\count@<255 %
  \advance\count@ 1 %
\repeat

\def\RangeCatcodeInvalid#1#2{%
  \count@=#1\relax
  \loop
    \catcode\count@=15 %
  \ifnum\count@<#2\relax
    \advance\count@ 1 %
  \repeat
}
\def\RangeCatcodeCheck#1#2#3{%
  \count@=#1\relax
  \loop
    \ifnum#3=\catcode\count@
    \else
      \errmessage{%
        Character \the\count@\space
        with wrong catcode \the\catcode\count@\space
        instead of \number#3%
      }%
    \fi
  \ifnum\count@<#2\relax
    \advance\count@ 1 %
  \repeat
}
\def\space{ }
\expandafter\ifx\csname LoadCommand\endcsname\relax
  \def\LoadCommand{\input hobsub.sty\relax}%
\fi
\def\Test{%
  \RangeCatcodeInvalid{0}{47}%
  \RangeCatcodeInvalid{58}{64}%
  \RangeCatcodeInvalid{91}{96}%
  \RangeCatcodeInvalid{123}{255}%
  \catcode`\@=12 %
  \catcode`\\=0 %
  \catcode`\%=14 %
  \LoadCommand
  \RangeCatcodeCheck{0}{36}{15}%
  \RangeCatcodeCheck{37}{37}{14}%
  \RangeCatcodeCheck{38}{47}{15}%
  \RangeCatcodeCheck{48}{57}{12}%
  \RangeCatcodeCheck{58}{63}{15}%
  \RangeCatcodeCheck{64}{64}{12}%
  \RangeCatcodeCheck{65}{90}{11}%
  \RangeCatcodeCheck{91}{91}{15}%
  \RangeCatcodeCheck{92}{92}{0}%
  \RangeCatcodeCheck{93}{96}{15}%
  \RangeCatcodeCheck{97}{122}{11}%
  \RangeCatcodeCheck{123}{255}{15}%
  \RestoreCatcodes
}
\Test
\csname @@end\endcsname
\end
%    \end{macrocode}
%    \begin{macrocode}
%</test1>
%    \end{macrocode}
%
% \section{Installation}
%
% \subsection{Download}
%
% \paragraph{Package.} This package is available on
% CTAN\footnote{\url{http://ctan.org/pkg/hobsub}}:
% \begin{description}
% \item[\CTAN{macros/latex/contrib/oberdiek/hobsub.dtx}] The source file.
% \item[\CTAN{macros/latex/contrib/oberdiek/hobsub.pdf}] Documentation.
% \end{description}
%
%
% \paragraph{Bundle.} All the packages of the bundle `oberdiek'
% are also available in a TDS compliant ZIP archive. There
% the packages are already unpacked and the documentation files
% are generated. The files and directories obey the TDS standard.
% \begin{description}
% \item[\CTAN{install/macros/latex/contrib/oberdiek.tds.zip}]
% \end{description}
% \emph{TDS} refers to the standard ``A Directory Structure
% for \TeX\ Files'' (\CTAN{tds/tds.pdf}). Directories
% with \xfile{texmf} in their name are usually organized this way.
%
% \subsection{Bundle installation}
%
% \paragraph{Unpacking.} Unpack the \xfile{oberdiek.tds.zip} in the
% TDS tree (also known as \xfile{texmf} tree) of your choice.
% Example (linux):
% \begin{quote}
%   |unzip oberdiek.tds.zip -d ~/texmf|
% \end{quote}
%
% \paragraph{Script installation.}
% Check the directory \xfile{TDS:scripts/oberdiek/} for
% scripts that need further installation steps.
% Package \xpackage{attachfile2} comes with the Perl script
% \xfile{pdfatfi.pl} that should be installed in such a way
% that it can be called as \texttt{pdfatfi}.
% Example (linux):
% \begin{quote}
%   |chmod +x scripts/oberdiek/pdfatfi.pl|\\
%   |cp scripts/oberdiek/pdfatfi.pl /usr/local/bin/|
% \end{quote}
%
% \subsection{Package installation}
%
% \paragraph{Unpacking.} The \xfile{.dtx} file is a self-extracting
% \docstrip\ archive. The files are extracted by running the
% \xfile{.dtx} through \plainTeX:
% \begin{quote}
%   \verb|tex hobsub.dtx|
% \end{quote}
%
% \paragraph{TDS.} Now the different files must be moved into
% the different directories in your installation TDS tree
% (also known as \xfile{texmf} tree):
% \begin{quote}
% \def\t{^^A
% \begin{tabular}{@{}>{\ttfamily}l@{ $\rightarrow$ }>{\ttfamily}l@{}}
%   hobsub.sty & tex/generic/oberdiek/hobsub.sty\\
%   hobsub-generic.sty & tex/generic/oberdiek/hobsub-generic.sty\\
%   hobsub-hyperref.sty & tex/generic/oberdiek/hobsub-hyperref.sty\\
%   hobsub.pdf & doc/latex/oberdiek/hobsub.pdf\\
%   test/hobsub-test1.tex & doc/latex/oberdiek/test/hobsub-test1.tex\\
%   hobsub.dtx & source/latex/oberdiek/hobsub.dtx\\
% \end{tabular}^^A
% }^^A
% \sbox0{\t}^^A
% \ifdim\wd0>\linewidth
%   \begingroup
%     \advance\linewidth by\leftmargin
%     \advance\linewidth by\rightmargin
%   \edef\x{\endgroup
%     \def\noexpand\lw{\the\linewidth}^^A
%   }\x
%   \def\lwbox{^^A
%     \leavevmode
%     \hbox to \linewidth{^^A
%       \kern-\leftmargin\relax
%       \hss
%       \usebox0
%       \hss
%       \kern-\rightmargin\relax
%     }^^A
%   }^^A
%   \ifdim\wd0>\lw
%     \sbox0{\small\t}^^A
%     \ifdim\wd0>\linewidth
%       \ifdim\wd0>\lw
%         \sbox0{\footnotesize\t}^^A
%         \ifdim\wd0>\linewidth
%           \ifdim\wd0>\lw
%             \sbox0{\scriptsize\t}^^A
%             \ifdim\wd0>\linewidth
%               \ifdim\wd0>\lw
%                 \sbox0{\tiny\t}^^A
%                 \ifdim\wd0>\linewidth
%                   \lwbox
%                 \else
%                   \usebox0
%                 \fi
%               \else
%                 \lwbox
%               \fi
%             \else
%               \usebox0
%             \fi
%           \else
%             \lwbox
%           \fi
%         \else
%           \usebox0
%         \fi
%       \else
%         \lwbox
%       \fi
%     \else
%       \usebox0
%     \fi
%   \else
%     \lwbox
%   \fi
% \else
%   \usebox0
% \fi
% \end{quote}
% If you have a \xfile{docstrip.cfg} that configures and enables \docstrip's
% TDS installing feature, then some files can already be in the right
% place, see the documentation of \docstrip.
%
% \subsection{Refresh file name databases}
%
% If your \TeX~distribution
% (\teTeX, \mikTeX, \dots) relies on file name databases, you must refresh
% these. For example, \teTeX\ users run \verb|texhash| or
% \verb|mktexlsr|.
%
% \subsection{Some details for the interested}
%
% \paragraph{Attached source.}
%
% The PDF documentation on CTAN also includes the
% \xfile{.dtx} source file. It can be extracted by
% AcrobatReader 6 or higher. Another option is \textsf{pdftk},
% e.g. unpack the file into the current directory:
% \begin{quote}
%   \verb|pdftk hobsub.pdf unpack_files output .|
% \end{quote}
%
% \paragraph{Unpacking with \LaTeX.}
% The \xfile{.dtx} chooses its action depending on the format:
% \begin{description}
% \item[\plainTeX:] Run \docstrip\ and extract the files.
% \item[\LaTeX:] Generate the documentation.
% \end{description}
% If you insist on using \LaTeX\ for \docstrip\ (really,
% \docstrip\ does not need \LaTeX), then inform the autodetect routine
% about your intention:
% \begin{quote}
%   \verb|latex \let\install=y\input{hobsub.dtx}|
% \end{quote}
% Do not forget to quote the argument according to the demands
% of your shell.
%
% \paragraph{Generating the documentation.}
% You can use both the \xfile{.dtx} or the \xfile{.drv} to generate
% the documentation. The process can be configured by the
% configuration file \xfile{ltxdoc.cfg}. For instance, put this
% line into this file, if you want to have A4 as paper format:
% \begin{quote}
%   \verb|\PassOptionsToClass{a4paper}{article}|
% \end{quote}
% An example follows how to generate the
% documentation with pdf\LaTeX:
% \begin{quote}
%\begin{verbatim}
%pdflatex hobsub.dtx
%makeindex -s gind.ist hobsub.idx
%pdflatex hobsub.dtx
%makeindex -s gind.ist hobsub.idx
%pdflatex hobsub.dtx
%\end{verbatim}
% \end{quote}
%
% \section{Catalogue}
%
% The following XML file can be used as source for the
% \href{http://mirror.ctan.org/help/Catalogue/catalogue.html}{\TeX\ Catalogue}.
% The elements \texttt{caption} and \texttt{description} are imported
% from the original XML file from the Catalogue.
% The name of the XML file in the Catalogue is \xfile{hobsub.xml}.
%    \begin{macrocode}
%<*catalogue>
<?xml version='1.0' encoding='us-ascii'?>
<!DOCTYPE entry SYSTEM 'catalogue.dtd'>
<entry datestamp='$Date$' modifier='$Author$' id='hobsub'>
  <name>hobsub</name>
  <caption>Construct package bundles</caption>
  <authorref id='auth:oberdiek'/>
  <copyright owner='Heiko Oberdiek' year='2011,2012'/>
  <license type='lppl1.3'/>
  <version number='1.14'/>
  <description>
    The package offers a means of loading a bundle of (other) packages
    &#x201C;at once&#x201D;, offering a means of reducing the amount of log output
    as the packages are loaded, and an aide-memoire for complicated
    requirements.  In some ways, loading a package via hobsub behaves
    much like loading the package in the normal way; for example,
    reloads are suppressed, as in LaTeX proper, and correct
    <tt>\listfiles</tt> output is produced.
    <p/>
    Examples provided are a &#x2018;generic&#x2019; bundle (comprising the author&#x2019;s
    packages that can be used with Plain TeX) and a &#x2018;hyperref&#x2019; bundle
    (comprising packages useful when generating hypertext output with
    LaTeX).
    <p/>
    The package is part of the <xref refid='oberdiek'>oberdiek</xref> bundle.
  </description>
  <documentation details='Package documentation'
      href='ctan:/macros/latex/contrib/oberdiek/hobsub.pdf'/>
  <ctan file='true' path='/macros/latex/contrib/oberdiek/hobsub.dtx'/>
  <miktex location='oberdiek'/>
  <texlive location='oberdiek'/>
  <install path='/macros/latex/contrib/oberdiek/oberdiek.tds.zip'/>
</entry>
%</catalogue>
%    \end{macrocode}
%
% \begin{thebibliography}{9}
% \raggedright
% \bibitem{lualatex:quiet}
% Will Robertson,
% \textit{[lltx] quietening lualatex console output};
% mailing list \textsf{lualatex-dev at tug.org},
% 2011-01-29;
% \url{http://tug.org/pipermail/lualatex-dev/2011-January/001031.html}.
%
% \end{thebibliography}
%
% \begin{History}
%   \begin{Version}{2011/01/30 v1.0}
%   \item
%     First public version.
%   \end{Version}
%   \begin{Version}{2011/04/16 v1.1}
%   \item
%     Package updates.
%   \end{Version}
%   \begin{Version}{2011/04/17 v1.2}
%   \item
%     White spaces at line begins are removed or reduced
%     in the generated collection packages.
%   \end{Version}
%   \begin{Version}{2011/04/18 v1.3}
%   \item
%     Package updates.
%   \end{Version}
%   \begin{Version}{2011/04/23 v1.4}
%   \item
%     Package updates.
%   \end{Version}
%   \begin{Version}{2011/06/24 v1.5}
%   \item
%     Package updates.
%   \end{Version}
%   \begin{Version}{2011/06/30 v1.6}
%   \item
%     Package updates.
%   \end{Version}
%   \begin{Version}{2011/07/01 v1.7}
%   \item
%     Package updates.
%   \end{Version}
%   \begin{Version}{2011/07/28 v1.8}
%   \item
%     Package updates.
%   \end{Version}
%   \begin{Version}{2011/08/22 v1.9}
%   \item
%     Package updates.
%   \end{Version}
%   \begin{Version}{2011/10/16 v1.10}
%   \item
%     Package updates.
%   \end{Version}
%   \begin{Version}{2011/11/29 v1.11}
%   \item
%     Package updates.
%   \end{Version}
%   \begin{Version}{2012/04/25 v1.12}
%   \item
%     Package updates.
%   \end{Version}
%   \begin{Version}{2012/05/28 v1.13}
%   \item
%     Package updates.
%   \end{Version}
%   \begin{Version}{2016/05/16 v1.14}
%   \item
%     Documentation updates.
%   \end{Version}
% \end{History}
%
% \PrintIndex
%
% \Finale
\endinput

%        (quote the arguments according to the demands of your shell)
%
% Documentation:
%    (a) If hobsub.drv is present:
%           latex hobsub.drv
%    (b) Without hobsub.drv:
%           latex hobsub.dtx; ...
%    The class ltxdoc loads the configuration file ltxdoc.cfg
%    if available. Here you can specify further options, e.g.
%    use A4 as paper format:
%       \PassOptionsToClass{a4paper}{article}
%
%    Programm calls to get the documentation (example):
%       pdflatex hobsub.dtx
%       makeindex -s gind.ist hobsub.idx
%       pdflatex hobsub.dtx
%       makeindex -s gind.ist hobsub.idx
%       pdflatex hobsub.dtx
%
% Installation:
%    TDS:tex/generic/oberdiek/hobsub.sty
%    TDS:tex/generic/oberdiek/hobsub-generic.sty
%    TDS:tex/generic/oberdiek/hobsub-hyperref.sty
%    TDS:doc/latex/oberdiek/hobsub.pdf
%    TDS:doc/latex/oberdiek/test/hobsub-test1.tex
%    TDS:source/latex/oberdiek/hobsub.dtx
%
%<*ignore>
\begingroup
  \catcode123=1 %
  \catcode125=2 %
  \def\x{LaTeX2e}%
\expandafter\endgroup
\ifcase 0\ifx\install y1\fi\expandafter
         \ifx\csname processbatchFile\endcsname\relax\else1\fi
         \ifx\fmtname\x\else 1\fi\relax
\else\csname fi\endcsname
%</ignore>
%<*install>
\input docstrip.tex
\Msg{************************************************************************}
\Msg{* Installation}
\Msg{* Package: hobsub 2016/05/16 v1.14 Construct package bundles (HO)}
\Msg{************************************************************************}

\keepsilent
\askforoverwritefalse

\let\MetaPrefix\relax
\preamble

This is a generated file.

Project: hobsub
Version: 2016/05/16 v1.14

Copyright (C) 2011, 2012 by
   Heiko Oberdiek <heiko.oberdiek at googlemail.com>

This work may be distributed and/or modified under the
conditions of the LaTeX Project Public License, either
version 1.3c of this license or (at your option) any later
version. This version of this license is in
   http://www.latex-project.org/lppl/lppl-1-3c.txt
and the latest version of this license is in
   http://www.latex-project.org/lppl.txt
and version 1.3 or later is part of all distributions of
LaTeX version 2005/12/01 or later.

This work has the LPPL maintenance status "maintained".

This Current Maintainer of this work is Heiko Oberdiek.

The Base Interpreter refers to any `TeX-Format',
because some files are installed in TDS:tex/generic//.

This work consists of the main source file hobsub.dtx
and the derived files
   hobsub.sty, hobsub.pdf, hobsub.ins, hobsub.drv, hobsub-generic.sty,
   hobsub-hyperref.sty, hobsub-test1.tex.

\endpreamble
\let\MetaPrefix\DoubleperCent

\generate{%
  \file{hobsub.ins}{\from{hobsub.dtx}{install}}%
  \file{hobsub.drv}{\from{hobsub.dtx}{driver}}%
  \usedir{tex/generic/oberdiek}%
  \file{hobsub.sty}{\from{hobsub.dtx}{package}}%
  \usedir{doc/latex/oberdiek/test}%
  \file{hobsub-test1.tex}{\from{hobsub.dtx}{test1}}%
  \nopreamble
  \nopostamble
  \usedir{source/latex/oberdiek/catalogue}%
  \file{hobsub.xml}{\from{hobsub.dtx}{catalogue}}%
}
\let\hobsubSavedNormalLine\normalLine
\def\hobsubGobble#1{}
\begingroup
  \lccode`\0=32 % space
  \lccode`\1=35 % hash
  \lccode`\2=37 % percent
  \lccode`\3=92 % backslash
  \lccode`\4=123 % {
  \lccode`\5=125 % }
\lowercase{\endgroup
  \let\hobsubTokenSpaceOther= 0%
  \let\hobsubTokenHash= 1%
  \let\hobsubTokenPercent= 2%
  \let\hobsubTokenBackslash= 3%
  \let\hobsubTokenBraceLeft= 4%
  \let\hobsubTokenBraceRight= 5%
  \def\hobsubNormalLineWithSpace{%
    \hobsubSavedNormalLine0%
  }%
}
\def\normalLine{%
  \futurelet\hobsubToken\hobsubFirst
}
\def\hobsubFirst{%
  \ifx\hobsubToken\hobsubTokenSpaceOther
    \expandafter\expandafter\expandafter\futurelet
    \expandafter\expandafter\expandafter\hobsubToken
    \expandafter\expandafter\expandafter\hobsubNext
    \expandafter\hobsubGobble
  \else
    \expandafter\hobsubSavedNormalLine
  \fi
}
\def\hobsubNext{%
  \ifx\hobsubToken\hobsubTokenSpaceOther
    \expandafter\expandafter\expandafter\futurelet
    \expandafter\expandafter\expandafter\hobsubToken
    \expandafter\expandafter\expandafter\hobsubNext 
    \expandafter\hobsubGobble
  \else
    \let\hobsubWithSpace=Y%
    \ifx\hobsubToken\hobsubTokenBackslash
      \let\hobsubWithSpace=N%
    \else
      \ifx\hobsubToken\hobsubTokenBraceRight
        \let\hobsubWithSpace=N%
      \else
        \ifx\hobsubToken\hobsubTokenBraceLeft
          \let\hobsubWithSpace=N%
        \else
          \ifx\hobsubToken\hobsubTokenPercent
            \let\hobsubWithSpace=N%
          \else
            \ifx\hobsubToken\hobsubTokenHash
              \let\hobsubWithSpace=N%
            \fi
          \fi
        \fi
      \fi
    \fi
    \ifx\hobsubWithSpace Y%
      \expandafter\expandafter\expandafter\hobsubNormalLineWithSpace
    \else
      \expandafter\expandafter\expandafter\hobsubSavedNormalLine
    \fi
  \fi
}
\generate{%
  \usedir{tex/generic/oberdiek}%
  \file{hobsub-generic.sty}{%
    \from{hobsub.dtx}{generic,preamble,infwarerr}%
    \from{infwarerr.dtx}{package}%
    \from{hobsub.dtx}{post,ltxcmds}%
    \from{ltxcmds.dtx}{package}%
    \from{hobsub.dtx}{post,ifluatex}%
    \from{ifluatex.dtx}{package}%
    \from{hobsub.dtx}{post,ifvtex}%
    \from{ifvtex.dtx}{package}%
    \from{hobsub.dtx}{post,intcalc}%
    \from{intcalc.dtx}{package}%
    \from{hobsub.dtx}{post,ifpdf}%
    \from{ifpdf.dtx}{package}%
    \from{hobsub.dtx}{post,etexcmds}%
    \from{etexcmds.dtx}{package}%
    \from{hobsub.dtx}{post,kvsetkeys}%
    \from{kvsetkeys.dtx}{package}%
    \from{hobsub.dtx}{post,kvdefinekeys}%
    \from{kvdefinekeys.dtx}{package}%
    \from{hobsub.dtx}{post,luatex-loader}%
    \from{luatex.dtx}{loader}%
    \from{hobsub.dtx}{post,pdftexcmds}%
    \from{pdftexcmds.dtx}{package}%
    \from{hobsub.dtx}{post,pdfescape}%
    \from{pdfescape.dtx}{package}%
    \from{hobsub.dtx}{post,bigintcalc}%
    \from{bigintcalc.dtx}{package}%
    \from{hobsub.dtx}{post,bitset}%
    \from{bitset.dtx}{package}%
    \from{hobsub.dtx}{post,uniquecounter}%
    \from{uniquecounter.dtx}{package}%
    \from{hobsub.dtx}{post,atend}%
  }%
}
\generate{%
  \usedir{tex/generic/oberdiek}%
  \file{hobsub-hyperref.sty}{%
    \from{hobsub.dtx}{hyperref,preamble,,letltxmacro}%
    \from{letltxmacro.dtx}{package}%
    \from{hobsub.dtx}{post,hopatch}%
    \from{hopatch.dtx}{package}%
    \from{hobsub.dtx}{post,xcolor-patch}%
    \from{hycolor.dtx}{xcolor}%
    \from{hobsub.dtx}{post,atveryend}%
    \from{atveryend.dtx}{package}%
    \from{hobsub.dtx}{post,atbegshi}%
    \from{atbegshi.dtx}{package}%
    \from{hobsub.dtx}{post,refcount}%
    \from{refcount.dtx}{package}%
    \from{hobsub.dtx}{post,hycolor}%
    \from{hycolor.dtx}{package}%
    \from{hobsub.dtx}{post}%
  }%
}
\let\normalLine\hobsubSavedNormalLine

\catcode32=13\relax% active space
\let =\space%
\Msg{************************************************************************}
\Msg{*}
\Msg{* To finish the installation you have to move the following}
\Msg{* files into a directory searched by TeX:}
\Msg{*}
\Msg{*     hobsub.sty, hobsub-generic.sty, hobsub-hyperref.sty}
\Msg{*}
\Msg{* To produce the documentation run the file `hobsub.drv'}
\Msg{* through LaTeX.}
\Msg{*}
\Msg{* Happy TeXing!}
\Msg{*}
\Msg{************************************************************************}

\endbatchfile
%</install>
%<*ignore>
\fi
%</ignore>
%<*driver>
\NeedsTeXFormat{LaTeX2e}
\ProvidesFile{hobsub.drv}%
  [2016/05/16 v1.14 Construct package bundles (HO)]%
\documentclass{ltxdoc}
\usepackage{holtxdoc}[2011/11/22]
\usepackage{array}
\usepackage{caption}
\renewcommand*{\bottomfraction}{.5}
\renewcommand*{\topfraction}{.5}
\begin{document}
  \DocInput{hobsub.dtx}%
\end{document}
%</driver>
% \fi
%
%
% \CharacterTable
%  {Upper-case    \A\B\C\D\E\F\G\H\I\J\K\L\M\N\O\P\Q\R\S\T\U\V\W\X\Y\Z
%   Lower-case    \a\b\c\d\e\f\g\h\i\j\k\l\m\n\o\p\q\r\s\t\u\v\w\x\y\z
%   Digits        \0\1\2\3\4\5\6\7\8\9
%   Exclamation   \!     Double quote  \"     Hash (number) \#
%   Dollar        \$     Percent       \%     Ampersand     \&
%   Acute accent  \'     Left paren    \(     Right paren   \)
%   Asterisk      \*     Plus          \+     Comma         \,
%   Minus         \-     Point         \.     Solidus       \/
%   Colon         \:     Semicolon     \;     Less than     \<
%   Equals        \=     Greater than  \>     Question mark \?
%   Commercial at \@     Left bracket  \[     Backslash     \\
%   Right bracket \]     Circumflex    \^     Underscore    \_
%   Grave accent  \`     Left brace    \{     Vertical bar  \|
%   Right brace   \}     Tilde         \~}
%
% \GetFileInfo{hobsub.drv}
%
% \title{The \xpackage{hobsub} package}
% \date{2016/05/16 v1.14}
% \author{Heiko Oberdiek\thanks
% {Please report any issues at https://github.com/ho-tex/oberdiek/issues}\\
% \xemail{heiko.oberdiek at googlemail.com}}
%
% \maketitle
%
% \begin{abstract}
% Package \xpackage{hobsub} implements the idea of loading
% several packages in one big collection package.
% \end{abstract}
%
% \tableofcontents
%
% \def\TableGeneric{^^A
%   \TableEntry{\xpackage{infwarerr}}{}{2010/04/08 v1.3}^^A
%   \TableEntry{\xpackage{ltxcmds}}{}{2011/11/09 v1.22}^^A
%   \TableEntry{\xpackage{ifluatex}}{}{2010/03/01 v1.3}^^A
%   \TableEntry{\xpackage{ifvtex}}{}{2010/03/01 v1.5}^^A
%   \TableEntry{\xpackage{intcalc}}{}{2007/09/27 v1.1}^^A
%   \TableEntry{\xpackage{ifpdf}}{\xpackage{ifluatex}}{2011/01/30 v2.3}^^A
%   \TableEntry{\xpackage{etexcmds}}{}{2011/02/16 v1.5}^^A
%   \TableEntry{\xpackage{kvsetkeys}}{}{2012/04/25 v1.16}^^A
%   \TableEntry{\xpackage{kvdefinekeys}}{}{2011/04/07 v1.3}^^A
%   \TableEntry{\xpackage{luatex-loader}}{}{2010/03/09 v0.4}^^A
%   \TableEntry{\xpackage{pdftexcmds}}{\xpackage{ifluatex}, \xpackage{ifpdf}, \xpackage{infwarerr}, \xpackage{ltxcmds}, \xpackage{luatex-loader}}{2011/11/29 v0.20}^^A
%   \TableEntry{\xpackage{pdfescape}}{\xpackage{ltxcmds}, \xpackage{pdftexcmds}}{2011/11/25 v1.13}^^A
%   \TableEntry{\xpackage{bigintcalc}}{\xpackage{pdftexcmds}}{2012/04/08 v1.3}^^A
%   \TableEntry{\xpackage{bitset}}{\xpackage{bigintcalc}, \xpackage{infwarerr}, \xpackage{intcalc}}{2011/01/30 v1.1}^^A
%   \TableEntry{\xpackage{uniquecounter}}{\xpackage{bigintcalc}, \xpackage{infwarerr}}{2011/01/30 v1.2}^^A
% }
% \def\TableHyperref{^^A
%   \TableEntry{\xpackage{letltxmacro}}{}{2010/09/02 v1.4}^^A
%   \TableEntry{\xpackage{hopatch}}{}{2016/05/16 v1.2}^^A
%   \TableEntry{\xpackage{xcolor-patch}}{}{2011/01/30 v1.7}^^A
%   \TableEntry{\xpackage{atveryend}}{}{2011/06/30 v1.8}^^A
%   \TableEntry{\xpackage{atbegshi}}{\xpackage{ifpdf}, \xpackage{infwarerr}, \xpackage{ltxcmds}}{2011/10/05 v1.16}^^A
%   \TableEntry{\xpackage{refcount}}{\xpackage{infwarerr}, \xpackage{ltxcmds}}{2011/10/16 v3.4}^^A
%   \TableEntry{\xpackage{hycolor}}{\xpackage{xcolor-patch}}{2011/01/30 v1.7}^^A
% }
%
% \newenvironment{TableList}[2][htbp]{^^A
%   \begin{table}[{#1}]^^A
%   \centering
%   \caption{Overview \xpackage{hobsub-#2}}^^A
%   \label{tab:#2}^^A
%   \begin{tabular}{ll>{\raggedright}p{.4\linewidth}}^^A
%   \bfseries Package & \bfseries Release &
%   \bfseries Dependencies\tabularnewline
%   \hline
% }{^^A
%   \end{tabular}^^A
%   \end{table}^^A
% }
% \newcommand*{\TableEntry}[3]{%
%   #1&#3&#2\tabularnewline
% }
%
% \renewcommand*{\arraystretch}{1.1}
% \setlength{\extrarowheight}{2pt}
%
% \section{Documentation}
%
% In January 2011 the mailing list lualatex-dev discussed
% ``quietening lualatex console output'' \cite{lualatex:quiet}.
% Inspired from this, I implemented this package \xpackage{hobsub}
% to load several packages in one file. The package is applied for
% two packages collections that are subsets of my package bundle
% \xpackage{oberdiek}.
% \begin{description}
% \item[\xpackage{hobsub-generic}] is a collection of packages
%   that can also be used with \hologo{plainTeX}.
% \item[\xpackage{hobsub-hyperref}] is a collection of packages
%   that is used by package \xpackage{hyperref}.
% \end{description}
%
% Package \xpackage{hobsub} provides the macros that
% are used in the collection packages and might be used
% in other projects as well. But there are many caveats,
% some of them are discussed below.
%
% The most serious problem is consistency. A package
% that is available as standalone package and is part
% of a collection might have different versions.
% Because my packages are organized in a bundle and should
% be installed and updated as bundle, the risk seems low
% to me. However, if a single package is updated manually, for
% example for testing, then it should be loaded before the
% collection that contains the package with an older version.
% In a collection, the package will not be loaded again if it
% is already loaded.
%
% \subsection{Usage}
%
% The collection packages \xpackage{hobsub-generic} and
% \xpackage{hobsub-hyperref} are loaded as usual:
% \begin{quote}
%  \begin{tabular}{@{}ll@{}}
%    |\usepackage{hobsub-generic}| & \hologo{LaTeX}\\
%    |\input hobsub-generic.sty| & \hologo{plainTeX}
%  \end{tabular}
% \end{quote}
% or
% \begin{quote}|\usepackage{hobsub-hyperref}|\end{quote}
% Of course, both \cs{usepackage} or \cs{RequirePackage}
% can be used.
%
% If you need requirements on the version date of a package
% inside a collection, then specify them afterwards, for
% example:
% \begin{quote}
%   |\usepackage{hobsub-generic}|\\
%   |\usepackage{ltxcmds}[2010/12/04]% |^^A
%   {\itshape |\ltx@ifblank is needed|}
% \end{quote}
%
% Also it is not garanteed that a package will be always
% part of a collection. Therefore it does not harm,
% to make the requirements explicite \emph{after}
% a collection is loaded, see the previous example.
% \hologo{LaTeX} knows that a
% package is already loaded and does not load it again.
%
% \subsection{Supported features}
%
% \begin{description}
% \item[No reloads:] If a package is already loaded then it
%  is not reloaded again if it is part of a collection package.
% \item[\cs{listfiles}:]
%   Each package that is loaded in a collection is also
%   added to the output of \cs{listfiles}.
% \item[\cs{@currname}, \cs{@currext}]
%   are updated, they are used by \hologo{LaTeX}'s
%  \cs{ProvidesPackage}.
% \item[\cs{AtEndOfPackage}:]
%   \hologo{LaTeX}'s hook mechanism that is used for
%   \cs{AtEndOfPackage} is supported. However none of the
%   packages in the collections are using \cs{AtEndOfPackage},
%   therefore this feature is completely untested.
% \end{description}
%
% \subsection{Common limitations}
%
% \begin{description}
% \item[\cs{endinput}:]
%   Often a package is using \cs{endinput}, either added by
%   \xpackage{docstrip}, or to stop package loading at an earlier
%   time. Therefore \cs{endinput} must be catched during package
%   loading to prevent that the collection package is closed
%   too early before reading all packages it contains.
%
%   This redefinition of \cs{endinput} fails if the package loads
%   other files that contain \cs{endinput}. To some degree it could
%   be catched by redefining macros that load files. But this will
%   not work for the primitive \cs{input}.
% \item[Options:]
%   \hologo{LaTeX}'s package options are not supported.
%   Therefore the packages that are put in a collection must
%   not support options.
% \item[Hook packages:]
%   Some packages and classes provides hooks for packages, see
%   packages \xpackage{scrlfile}, \xpackage{filehook}
%   or class \xclass{memoir}. These hooks are not supported
%   for embedded packages inside a collection package.
%   Because of the problem with \cs{endinput}, these hooks
%   are not supported. The hooks might contain code that
%   loads other files that execute \cs{endinput}.
% \end{description}
%
% \subsection{Package \xpackage{hobsub-generic}}
%
% \begin{TableList}[bp]{generic}\TableGeneric\end{TableList}
%
% Collection `generic' contains some of the \hologo{plainTeX} compatible
% packages of my bundle, see table \ref{tab:generic}.
% It adds the following restrictions to any package
% that is part of this collection:
% \begin{description}
% \item[Generic formats:] The following formats must be supported:
%   \hologo{LaTeX}, \hologo{plainTeX} and even \hologo{iniTeX}.
%   This applies to the collection package \xpackage{hobsub-generic}
%   the same way.
% \item[Prefixed macros:]
%   Any package of this collection should only define macros with
%   prefixed names to avoid name clashes. Resources like registers
%   must not be allocated and other global states should not be
%   changed except for added macro definitions.
% \end{description}
% Thus the package \xpackage{hobsub-generic} can also preloaded
% in the format file with \hologo{iniTeX}.
%
% \subsection{Package \xpackage{hobsub-hyperref}}
%
% \begin{TableList}[tp]{hyperref}\TableHyperref\end{TableList}
%
% Collection `hyperref' contains additional packages that are
% used by package \xpackage{hyperref} and that does not fit
% in collection `generic', see table{tab:hyperref}.
% The latter collection package
% \xpackage{hobsub-generic} is loaded via a conventional
% \cs{RequirePackage}.
%
% Because most of the packages in the collection `hyperref'
% needs \hologo{LaTeX}, the collection requires \hologo{LaTeX}.
%
% \subsection{Package \xpackage{hobsub}}
%
% This section is for advanced users that might want to
% build a collection package. Then it can be done by
% using \xpackage{hobsub}:
% \begin{quote}
%   \dots\\
%   |\RequirePackage{hobsub}[2016/05/16]|\\
%   \dots\\
%   |\hobsub@StartPackage{foo1}|\\
%   \textit{\dots\unskip\ package code of \xfile{foo1.sty} \dots}\\
%   |\endinput|\\
%   |<hobsub>|\\
%   |\hobsub@StopPackage|\\
%   |\hobsub@StartPackage{bar2}|\\
%   \textit{\dots\unskip\ package code of \xfile{bar2.sty} \dots}\\
%   |\endinput|\\
%   |<hobsub>|\\
%   |\hobsub@StopPackage|
%   \dots
% \end{quote}
% But be aware of the caveats and limitations, see above.
% Only quite a small portion of the packages can be embedded
% like this.
%
% \clearpage
%
% \StopEventually{
% }
%
% \section{Implementation}
%
% \subsection{Reload check and package identification}
%
% \subsubsection{Package \xpackage{hobsub}}
%
%    \begin{macrocode}
%<*package>
%    \end{macrocode}
%    Reload check, especially if the package is not used with \LaTeX.
%    \begin{macrocode}
\begingroup\catcode61\catcode48\catcode32=10\relax%
  \catcode13=5 % ^^M
  \endlinechar=13 %
  \catcode35=6 % #
  \catcode39=12 % '
  \catcode44=12 % ,
  \catcode45=12 % -
  \catcode46=12 % .
  \catcode58=12 % :
  \catcode64=11 % @
  \catcode123=1 % {
  \catcode125=2 % }
  \expandafter\let\expandafter\x\csname ver@hobsub.sty\endcsname
  \ifx\x\relax % plain-TeX, first loading
  \else
    \def\empty{}%
    \ifx\x\empty % LaTeX, first loading,
      % variable is initialized, but \ProvidesPackage not yet seen
    \else
      \expandafter\ifx\csname PackageInfo\endcsname\relax
        \def\x#1#2{%
          \immediate\write-1{Package #1 Info: #2.}%
        }%
      \else
        \def\x#1#2{\PackageInfo{#1}{#2, stopped}}%
      \fi
      \x{hobsub}{The package is already loaded}%
      \aftergroup\endinput
    \fi
  \fi
\endgroup%
%    \end{macrocode}
%    Package identification:
%    \begin{macrocode}
\begingroup\catcode61\catcode48\catcode32=10\relax%
  \catcode13=5 % ^^M
  \endlinechar=13 %
  \catcode35=6 % #
  \catcode39=12 % '
  \catcode40=12 % (
  \catcode41=12 % )
  \catcode44=12 % ,
  \catcode45=12 % -
  \catcode46=12 % .
  \catcode47=12 % /
  \catcode58=12 % :
  \catcode64=11 % @
  \catcode91=12 % [
  \catcode93=12 % ]
  \catcode123=1 % {
  \catcode125=2 % }
  \expandafter\ifx\csname ProvidesPackage\endcsname\relax
    \def\x#1#2#3[#4]{\endgroup
      \immediate\write-1{Package: #3 #4}%
      \xdef#1{#4}%
    }%
  \else
    \def\x#1#2[#3]{\endgroup
      #2[{#3}]%
      \ifx#1\@undefined
        \xdef#1{#3}%
      \fi
      \ifx#1\relax
        \xdef#1{#3}%
      \fi
    }%
  \fi
\expandafter\x\csname ver@hobsub.sty\endcsname
\ProvidesPackage{hobsub}%
  [2016/05/16 v1.14 Construct package bundles (HO)]%
%    \end{macrocode}
%    \begin{macrocode}
%</package>
%    \end{macrocode}
%
% \subsubsection{Package \xpackage{hobsub-generic}}
%
%    \begin{macrocode}
%<*generic>
%    \end{macrocode}
%    Reload check, especially if the package is not used with \LaTeX.
%    \begin{macrocode}
\begingroup\catcode61\catcode48\catcode32=10\relax%
  \catcode13=5 % ^^M
  \endlinechar=13 %
  \catcode35=6 % #
  \catcode39=12 % '
  \catcode44=12 % ,
  \catcode45=12 % -
  \catcode46=12 % .
  \catcode58=12 % :
  \catcode64=11 % @
  \catcode123=1 % {
  \catcode125=2 % }
  \expandafter\let\expandafter\x\csname ver@hobsub-generic.sty\endcsname
  \ifx\x\relax % plain-TeX, first loading
  \else
    \def\empty{}%
    \ifx\x\empty % LaTeX, first loading,
      % variable is initialized, but \ProvidesPackage not yet seen
    \else
      \expandafter\ifx\csname PackageInfo\endcsname\relax
        \def\x#1#2{%
          \immediate\write-1{Package #1 Info: #2.}%
        }%
      \else
        \def\x#1#2{\PackageInfo{#1}{#2, stopped}}%
      \fi
      \x{hobsub-generic}{The package is already loaded}%
      \aftergroup\endinput
    \fi
  \fi
\endgroup%
%    \end{macrocode}
%    Package identification:
%    \begin{macrocode}
\begingroup\catcode61\catcode48\catcode32=10\relax%
  \catcode13=5 % ^^M
  \endlinechar=13 %
  \catcode35=6 % #
  \catcode39=12 % '
  \catcode40=12 % (
  \catcode41=12 % )
  \catcode44=12 % ,
  \catcode45=12 % -
  \catcode46=12 % .
  \catcode47=12 % /
  \catcode58=12 % :
  \catcode64=11 % @
  \catcode91=12 % [
  \catcode93=12 % ]
  \catcode123=1 % {
  \catcode125=2 % }
  \expandafter\ifx\csname ProvidesPackage\endcsname\relax
    \def\x#1#2#3[#4]{\endgroup
      \immediate\write-1{Package: #3 #4}%
      \xdef#1{#4}%
    }%
  \else
    \def\x#1#2[#3]{\endgroup
      #2[{#3}]%
      \ifx#1\@undefined
        \xdef#1{#3}%
      \fi
      \ifx#1\relax
        \xdef#1{#3}%
      \fi
    }%
  \fi
\expandafter\x\csname ver@hobsub-generic.sty\endcsname
\ProvidesPackage{hobsub-generic}%
  [2016/05/16 v1.14 Bundle oberdiek, subset generic (HO)]%
%    \end{macrocode}
%    \begin{macrocode}
%</generic>
%    \end{macrocode}
%
% \subsubsection{Package \xpackage{hobsub-hyperref}}
%
%    \begin{macrocode}
%<*hyperref>
%    \end{macrocode}
%    Reload check, especially if the package is not used with \LaTeX.
%    \begin{macrocode}
\begingroup\catcode61\catcode48\catcode32=10\relax%
  \catcode13=5 % ^^M
  \endlinechar=13 %
  \catcode35=6 % #
  \catcode39=12 % '
  \catcode44=12 % ,
  \catcode45=12 % -
  \catcode46=12 % .
  \catcode58=12 % :
  \catcode64=11 % @
  \catcode123=1 % {
  \catcode125=2 % }
  \expandafter\let\expandafter\x\csname ver@hobsub-hyperref.sty\endcsname
  \ifx\x\relax % plain-TeX, first loading
  \else
    \def\empty{}%
    \ifx\x\empty % LaTeX, first loading,
      % variable is initialized, but \ProvidesPackage not yet seen
    \else
      \expandafter\ifx\csname PackageInfo\endcsname\relax
        \def\x#1#2{%
          \immediate\write-1{Package #1 Info: #2.}%
        }%
      \else
        \def\x#1#2{\PackageInfo{#1}{#2, stopped}}%
      \fi
      \x{hobsub-hyperref}{The package is already loaded}%
      \aftergroup\endinput
    \fi
  \fi
\endgroup%
%    \end{macrocode}
%    Package identification:
%    \begin{macrocode}
\begingroup\catcode61\catcode48\catcode32=10\relax%
  \catcode13=5 % ^^M
  \endlinechar=13 %
  \catcode35=6 % #
  \catcode39=12 % '
  \catcode40=12 % (
  \catcode41=12 % )
  \catcode44=12 % ,
  \catcode45=12 % -
  \catcode46=12 % .
  \catcode47=12 % /
  \catcode58=12 % :
  \catcode64=11 % @
  \catcode91=12 % [
  \catcode93=12 % ]
  \catcode123=1 % {
  \catcode125=2 % }
  \expandafter\ifx\csname ProvidesPackage\endcsname\relax
    \def\x#1#2#3[#4]{\endgroup
      \immediate\write-1{Package: #3 #4}%
      \xdef#1{#4}%
    }%
  \else
    \def\x#1#2[#3]{\endgroup
      #2[{#3}]%
      \ifx#1\@undefined
        \xdef#1{#3}%
      \fi
      \ifx#1\relax
        \xdef#1{#3}%
      \fi
    }%
  \fi
\expandafter\x\csname ver@hobsub-hyperref.sty\endcsname
\ProvidesPackage{hobsub-hyperref}%
  [2016/05/16 v1.14 Bundle oberdiek, subset hyperref (HO)]%
%    \end{macrocode}
%    \begin{macrocode}
%</hyperref>
%    \end{macrocode}
%
% \subsection{Catcodes}
%
%    \begin{macrocode}
%<*package|generic|hyperref>
\begingroup\catcode61\catcode48\catcode32=10\relax%
  \catcode13=5 % ^^M
  \endlinechar=13 %
  \catcode123=1 % {
  \catcode125=2 % }
  \catcode64=11 % @
  \def\x{\endgroup
    \expandafter\edef\csname HOBsub@AtEnd\endcsname{%
      \endlinechar=\the\endlinechar\relax
      \catcode13=\the\catcode13\relax
      \catcode32=\the\catcode32\relax
      \catcode35=\the\catcode35\relax
      \catcode61=\the\catcode61\relax
      \catcode64=\the\catcode64\relax
      \catcode123=\the\catcode123\relax
      \catcode125=\the\catcode125\relax
    }%
  }%
\x\catcode61\catcode48\catcode32=10\relax%
\catcode13=5 % ^^M
\endlinechar=13 %
\catcode35=6 % #
\catcode64=11 % @
\catcode123=1 % {
\catcode125=2 % }
\def\TMP@EnsureCode#1#2{%
  \edef\HOBsub@AtEnd{%
    \HOBsub@AtEnd
    \catcode#1=\the\catcode#1\relax
  }%
  \catcode#1=#2\relax
}
\TMP@EnsureCode{39}{12}% '
\TMP@EnsureCode{40}{12}% (
\TMP@EnsureCode{41}{12}% )
\TMP@EnsureCode{45}{12}% -
\TMP@EnsureCode{46}{12}% .
\TMP@EnsureCode{47}{12}% /
\TMP@EnsureCode{58}{12}% :
\TMP@EnsureCode{60}{12}% <
\TMP@EnsureCode{62}{12}% >
\TMP@EnsureCode{96}{12}% `
\edef\HOBsub@AtEnd{\HOBsub@AtEnd\noexpand\endinput}
%    \end{macrocode}
%
%    \begin{macrocode}
%</package|generic|hyperref>
%    \end{macrocode}
%
% \subsection{Package \xpackage{hobsub-hyperref} loads \xpackage{hobsub-generic}}
%
%    \begin{macrocode}
%<*hyperref>
\NeedsTeXFormat{LaTeX2e}
\RequirePackage{hobsub-generic}[2016/05/16]
%</hyperref>
%    \end{macrocode}
%
% \subsection{Preamble}
%
%    \begin{macrocode}
%<*preamble>
\begingroup\expandafter\expandafter\expandafter\endgroup
\expandafter\ifx\csname ver@hobsub.sty\endcsname\relax
\else
  \expandafter\hobsub@GobbleRemainingPackage
\fi
%</preamble>
%    \end{macrocode}
%
%
%    \begin{macrocode}
%<*package|preamble>
%    \end{macrocode}
%
%    \begin{macro}{\HOBsub@OrgEndinput}
%    \begin{macrocode}
\begingroup\expandafter\expandafter\expandafter\endgroup
\expandafter\ifx\csname HOBsub@OrgEndinput\endcsname\relax
  \let\HOBsub@OrgEndinput\endinput
\fi
%    \end{macrocode}
%    \end{macro}
%    \begin{macro}{\hobsub@GobbleRemainingPackage}
%    \begin{macrocode}
\def\hobsub@GobbleRemainingPackage{%
  \begingroup
  \catcode92=14 % backslash: comment
  \catcode32=14 % space: comment
  \catcode35=14 % hash: comment
  \catcode123=14 % left brace: comment
  \catcode125=14 % right brace: comment
  \catcode60=3 % less: math
  \catcode62=4 % greater: align
  \endlinechar=-1 %
  \HOBsub@GobbleRemainingPackage
}
%    \end{macrocode}
%    \end{macro}
%    \begin{macro}{\HOBsub@GobbleRemainingPackage}
%    \begin{macrocode}
\catcode60=3 % less: dollar
\catcode62=4 % greater: align
\long\def\HOBsub@GobbleRemainingPackage#1<hobsub>{%
  \endgroup
}
\catcode60=12 % less: other
\catcode62=12 % greater: other
%    \end{macrocode}
%    \end{macro}
%
%    \begin{macro}{\hobsub@StartPackage}
%    \begin{macrocode}
\def\hobsub@StartPackage#1{%
  \begingroup\expandafter\expandafter\expandafter\endgroup
  \expandafter\ifx\csname ver@#1.sty\endcsname\relax
    \let\HOBsub@OrgCurrName\@currname
    \let\HOBsub@OrgCurrExt\@currext
    \csname @pushfilename\endcsname
    \def\@currname{#1}%
    \def\@currext{sty}%
    \expandafter\def\csname\@currname.\@currext-h@@k\endcsname{}%
    \let\endinput\hobsub@GobbleRemainingPackage
    \def\hobsub@StopPackage{%
      \let\hobsub@StopPackage\relax
      \HOBsub@StopPackage
%     \hobsub@Info{hobsub}{Package `#1' loaded}%
    }%
    \hobsub@AddToFileList{#1.sty}%
  \else
    \hobsub@Info{hobsub}{Skipping package `#1' (already loaded)}%
    \let\hobsub@StopPackage\relax
    \expandafter\hobsub@GobbleRemainingPackage
  \fi
}
%    \end{macrocode}
%    \end{macro}
%
%    \begin{macro}{\hobsub@StopPackage}
%    \begin{macrocode}
\let\hobsub@StopPackage\relax
%    \end{macrocode}
%    \end{macro}
%
%    \begin{macro}{\hobsub@Info}
%    \begin{macrocode}
\def\hobsub@Info#1#2{%
  \begingroup\expandafter\expandafter\expandafter\endgroup
  \expandafter\ifx\csname @PackageInfoNoLine\endcsname\relax
    \immediate\write-1{Package #1 Info: #2.}%
  \else
    \let\hobsub@Info\@PackageInfoNoLine
    \hobsub@Info{#1}{#2}%
  \fi
}
%    \end{macrocode}
%    \end{macro}
%
%    \begin{macro}{\HOBsub@StopPackage}
%    \begin{macrocode}
\def\HOBsub@StopPackage{%
  \csname\@currname.\@currext-h@@k\endcsname
  \let\endinput\HOBsub@OrgEndinput
  \csname @popfilename\endcsname
  \let\@currname\HOBsub@OrgCurrName
  \let\@currext\HOBsub@OrgCurrExt
}
%    \end{macrocode}
%    \end{macro}
%
%    \begin{macro}{\hobsub@AddToFileList}
%    \begin{macrocode}
\begingroup\expandafter\expandafter\expandafter\endgroup
\expandafter\ifx\csname @addtofilelist\endcsname\relax
  \def\hobsub@AddToFileList#1{}%
\else
  \def\hobsub@AddToFileList#1{%
    \@addtofilelist{#1}%
  }%
\fi
%    \end{macrocode}
%    \end{macro}
%
%    \begin{macrocode}
%</package|preamble>
%    \end{macrocode}
%
% \subsection{End of preamble for subset packages}
%
%    Add the end marker for possible skipping of previous
%    macros, if package \xpackage{hobsub} was already loaded.
%    \begin{macrocode}
%<*preamble>
\hobsub@GobbleRemainingPackage
<hobsub>
%    \end{macrocode}
%
%    All macros of package \xpackage{hobsub} are now available.
%    Now we officially load the package \xpackage{hobsub} inline
%    for proper package identification.
%    \begin{macrocode}
\hobsub@StartPackage{hobsub}%
\begingroup\expandafter\expandafter\expandafter\endgroup
\expandafter\ifx\csname ProvidesPackage\endcsname\relax
  \immediate\write-1{%
    Package: hobsub 2016/05/16 v1.14 Construct package bundles (HO)%
  }%
\else
  \ProvidesPackage{hobsub}%
  [2016/05/16 v1.14 Construct package bundles (HO)]%
\fi
%</preamble>
%    \end{macrocode}
%
%
%    \begin{macrocode}
%<*post|preamble>
 \endinput
<hobsub>
\hobsub@StopPackage
%</post|preamble>
%    \end{macrocode}
%
% \subsection{Package list}
%
%    \begin{macrocode}
%<infwarerr>\hobsub@StartPackage{infwarerr}
%<ltxcmds>\hobsub@StartPackage{ltxcmds}
%<ifluatex>\hobsub@StartPackage{ifluatex}
%<ifvtex>\hobsub@StartPackage{ifvtex}
%<intcalc>\hobsub@StartPackage{intcalc}
%<ifpdf>\hobsub@StartPackage{ifpdf}
%<etexcmds>\hobsub@StartPackage{etexcmds}
%<kvsetkeys>\hobsub@StartPackage{kvsetkeys}
%<kvdefinekeys>\hobsub@StartPackage{kvdefinekeys}
%<*luatex-loader>
\ifluatex
\else
  \expandafter\hobsub@GobbleRemainingPackage
\fi
\hobsub@StartPackage{luatex-loader}
%</luatex-loader>
%<pdftexcmds>\hobsub@StartPackage{pdftexcmds}
%<pdfescape>\hobsub@StartPackage{pdfescape}
%<bigintcalc>\hobsub@StartPackage{bigintcalc}
%<bitset>\hobsub@StartPackage{bitset}
%<uniquecounter>\hobsub@StartPackage{uniquecounter}
%<letltxmacro>\hobsub@StartPackage{letltxmacro}
%<hopatch>\hobsub@StartPackage{hopatch}
%<xcolor-patch>\hobsub@StartPackage{xcolor-patch}
%<atveryend>\hobsub@StartPackage{atveryend}
%<atbegshi>\hobsub@StartPackage{atbegshi}
%<refcount>\hobsub@StartPackage{refcount}
%<hycolor>\hobsub@StartPackage{hycolor}
%    \end{macrocode}
%
% \subsection{End of packages}
%
%    \begin{macrocode}
%<package|atend>\HOBsub@AtEnd%
%    \end{macrocode}
%
% \section{Test}
%
% \subsection{Catcode checks for loading}
%
%    \begin{macrocode}
%<*test1>
%    \end{macrocode}
%    \begin{macrocode}
\catcode`\{=1 %
\catcode`\}=2 %
\catcode`\#=6 %
\catcode`\@=11 %
\expandafter\ifx\csname count@\endcsname\relax
  \countdef\count@=255 %
\fi
\expandafter\ifx\csname @gobble\endcsname\relax
  \long\def\@gobble#1{}%
\fi
\expandafter\ifx\csname @firstofone\endcsname\relax
  \long\def\@firstofone#1{#1}%
\fi
\expandafter\ifx\csname loop\endcsname\relax
  \expandafter\@firstofone
\else
  \expandafter\@gobble
\fi
{%
  \def\loop#1\repeat{%
    \def\body{#1}%
    \iterate
  }%
  \def\iterate{%
    \body
      \let\next\iterate
    \else
      \let\next\relax
    \fi
    \next
  }%
  \let\repeat=\fi
}%
\def\RestoreCatcodes{}
\count@=0 %
\loop
  \edef\RestoreCatcodes{%
    \RestoreCatcodes
    \catcode\the\count@=\the\catcode\count@\relax
  }%
\ifnum\count@<255 %
  \advance\count@ 1 %
\repeat

\def\RangeCatcodeInvalid#1#2{%
  \count@=#1\relax
  \loop
    \catcode\count@=15 %
  \ifnum\count@<#2\relax
    \advance\count@ 1 %
  \repeat
}
\def\RangeCatcodeCheck#1#2#3{%
  \count@=#1\relax
  \loop
    \ifnum#3=\catcode\count@
    \else
      \errmessage{%
        Character \the\count@\space
        with wrong catcode \the\catcode\count@\space
        instead of \number#3%
      }%
    \fi
  \ifnum\count@<#2\relax
    \advance\count@ 1 %
  \repeat
}
\def\space{ }
\expandafter\ifx\csname LoadCommand\endcsname\relax
  \def\LoadCommand{\input hobsub.sty\relax}%
\fi
\def\Test{%
  \RangeCatcodeInvalid{0}{47}%
  \RangeCatcodeInvalid{58}{64}%
  \RangeCatcodeInvalid{91}{96}%
  \RangeCatcodeInvalid{123}{255}%
  \catcode`\@=12 %
  \catcode`\\=0 %
  \catcode`\%=14 %
  \LoadCommand
  \RangeCatcodeCheck{0}{36}{15}%
  \RangeCatcodeCheck{37}{37}{14}%
  \RangeCatcodeCheck{38}{47}{15}%
  \RangeCatcodeCheck{48}{57}{12}%
  \RangeCatcodeCheck{58}{63}{15}%
  \RangeCatcodeCheck{64}{64}{12}%
  \RangeCatcodeCheck{65}{90}{11}%
  \RangeCatcodeCheck{91}{91}{15}%
  \RangeCatcodeCheck{92}{92}{0}%
  \RangeCatcodeCheck{93}{96}{15}%
  \RangeCatcodeCheck{97}{122}{11}%
  \RangeCatcodeCheck{123}{255}{15}%
  \RestoreCatcodes
}
\Test
\csname @@end\endcsname
\end
%    \end{macrocode}
%    \begin{macrocode}
%</test1>
%    \end{macrocode}
%
% \section{Installation}
%
% \subsection{Download}
%
% \paragraph{Package.} This package is available on
% CTAN\footnote{\url{http://ctan.org/pkg/hobsub}}:
% \begin{description}
% \item[\CTAN{macros/latex/contrib/oberdiek/hobsub.dtx}] The source file.
% \item[\CTAN{macros/latex/contrib/oberdiek/hobsub.pdf}] Documentation.
% \end{description}
%
%
% \paragraph{Bundle.} All the packages of the bundle `oberdiek'
% are also available in a TDS compliant ZIP archive. There
% the packages are already unpacked and the documentation files
% are generated. The files and directories obey the TDS standard.
% \begin{description}
% \item[\CTAN{install/macros/latex/contrib/oberdiek.tds.zip}]
% \end{description}
% \emph{TDS} refers to the standard ``A Directory Structure
% for \TeX\ Files'' (\CTAN{tds/tds.pdf}). Directories
% with \xfile{texmf} in their name are usually organized this way.
%
% \subsection{Bundle installation}
%
% \paragraph{Unpacking.} Unpack the \xfile{oberdiek.tds.zip} in the
% TDS tree (also known as \xfile{texmf} tree) of your choice.
% Example (linux):
% \begin{quote}
%   |unzip oberdiek.tds.zip -d ~/texmf|
% \end{quote}
%
% \paragraph{Script installation.}
% Check the directory \xfile{TDS:scripts/oberdiek/} for
% scripts that need further installation steps.
% Package \xpackage{attachfile2} comes with the Perl script
% \xfile{pdfatfi.pl} that should be installed in such a way
% that it can be called as \texttt{pdfatfi}.
% Example (linux):
% \begin{quote}
%   |chmod +x scripts/oberdiek/pdfatfi.pl|\\
%   |cp scripts/oberdiek/pdfatfi.pl /usr/local/bin/|
% \end{quote}
%
% \subsection{Package installation}
%
% \paragraph{Unpacking.} The \xfile{.dtx} file is a self-extracting
% \docstrip\ archive. The files are extracted by running the
% \xfile{.dtx} through \plainTeX:
% \begin{quote}
%   \verb|tex hobsub.dtx|
% \end{quote}
%
% \paragraph{TDS.} Now the different files must be moved into
% the different directories in your installation TDS tree
% (also known as \xfile{texmf} tree):
% \begin{quote}
% \def\t{^^A
% \begin{tabular}{@{}>{\ttfamily}l@{ $\rightarrow$ }>{\ttfamily}l@{}}
%   hobsub.sty & tex/generic/oberdiek/hobsub.sty\\
%   hobsub-generic.sty & tex/generic/oberdiek/hobsub-generic.sty\\
%   hobsub-hyperref.sty & tex/generic/oberdiek/hobsub-hyperref.sty\\
%   hobsub.pdf & doc/latex/oberdiek/hobsub.pdf\\
%   test/hobsub-test1.tex & doc/latex/oberdiek/test/hobsub-test1.tex\\
%   hobsub.dtx & source/latex/oberdiek/hobsub.dtx\\
% \end{tabular}^^A
% }^^A
% \sbox0{\t}^^A
% \ifdim\wd0>\linewidth
%   \begingroup
%     \advance\linewidth by\leftmargin
%     \advance\linewidth by\rightmargin
%   \edef\x{\endgroup
%     \def\noexpand\lw{\the\linewidth}^^A
%   }\x
%   \def\lwbox{^^A
%     \leavevmode
%     \hbox to \linewidth{^^A
%       \kern-\leftmargin\relax
%       \hss
%       \usebox0
%       \hss
%       \kern-\rightmargin\relax
%     }^^A
%   }^^A
%   \ifdim\wd0>\lw
%     \sbox0{\small\t}^^A
%     \ifdim\wd0>\linewidth
%       \ifdim\wd0>\lw
%         \sbox0{\footnotesize\t}^^A
%         \ifdim\wd0>\linewidth
%           \ifdim\wd0>\lw
%             \sbox0{\scriptsize\t}^^A
%             \ifdim\wd0>\linewidth
%               \ifdim\wd0>\lw
%                 \sbox0{\tiny\t}^^A
%                 \ifdim\wd0>\linewidth
%                   \lwbox
%                 \else
%                   \usebox0
%                 \fi
%               \else
%                 \lwbox
%               \fi
%             \else
%               \usebox0
%             \fi
%           \else
%             \lwbox
%           \fi
%         \else
%           \usebox0
%         \fi
%       \else
%         \lwbox
%       \fi
%     \else
%       \usebox0
%     \fi
%   \else
%     \lwbox
%   \fi
% \else
%   \usebox0
% \fi
% \end{quote}
% If you have a \xfile{docstrip.cfg} that configures and enables \docstrip's
% TDS installing feature, then some files can already be in the right
% place, see the documentation of \docstrip.
%
% \subsection{Refresh file name databases}
%
% If your \TeX~distribution
% (\teTeX, \mikTeX, \dots) relies on file name databases, you must refresh
% these. For example, \teTeX\ users run \verb|texhash| or
% \verb|mktexlsr|.
%
% \subsection{Some details for the interested}
%
% \paragraph{Attached source.}
%
% The PDF documentation on CTAN also includes the
% \xfile{.dtx} source file. It can be extracted by
% AcrobatReader 6 or higher. Another option is \textsf{pdftk},
% e.g. unpack the file into the current directory:
% \begin{quote}
%   \verb|pdftk hobsub.pdf unpack_files output .|
% \end{quote}
%
% \paragraph{Unpacking with \LaTeX.}
% The \xfile{.dtx} chooses its action depending on the format:
% \begin{description}
% \item[\plainTeX:] Run \docstrip\ and extract the files.
% \item[\LaTeX:] Generate the documentation.
% \end{description}
% If you insist on using \LaTeX\ for \docstrip\ (really,
% \docstrip\ does not need \LaTeX), then inform the autodetect routine
% about your intention:
% \begin{quote}
%   \verb|latex \let\install=y% \iffalse meta-comment
%
% File: hobsub.dtx
% Version: 2016/05/16 v1.14
% Info: Construct package bundles
%
% Copyright (C) 2011, 2012 by
%    Heiko Oberdiek <heiko.oberdiek at googlemail.com>
%    2016
%    https://github.com/ho-tex/oberdiek/issues
%
% This work may be distributed and/or modified under the
% conditions of the LaTeX Project Public License, either
% version 1.3c of this license or (at your option) any later
% version. This version of this license is in
%    http://www.latex-project.org/lppl/lppl-1-3c.txt
% and the latest version of this license is in
%    http://www.latex-project.org/lppl.txt
% and version 1.3 or later is part of all distributions of
% LaTeX version 2005/12/01 or later.
%
% This work has the LPPL maintenance status "maintained".
%
% This Current Maintainer of this work is Heiko Oberdiek.
%
% The Base Interpreter refers to any `TeX-Format',
% because some files are installed in TDS:tex/generic//.
%
% This work consists of the main source file hobsub.dtx
% and the derived files
%    hobsub.sty, hobsub.pdf, hobsub.ins, hobsub.drv, hobsub-generic.sty,
%    hobsub-hyperref.sty, hobsub-test1.tex.
%
% Distribution:
%    CTAN:macros/latex/contrib/oberdiek/hobsub.dtx
%    CTAN:macros/latex/contrib/oberdiek/hobsub.pdf
%
% Unpacking:
%    (a) If hobsub.ins is present:
%           tex hobsub.ins
%    (b) Without hobsub.ins:
%           tex hobsub.dtx
%    (c) If you insist on using LaTeX
%           latex \let\install=y\input{hobsub.dtx}
%        (quote the arguments according to the demands of your shell)
%
% Documentation:
%    (a) If hobsub.drv is present:
%           latex hobsub.drv
%    (b) Without hobsub.drv:
%           latex hobsub.dtx; ...
%    The class ltxdoc loads the configuration file ltxdoc.cfg
%    if available. Here you can specify further options, e.g.
%    use A4 as paper format:
%       \PassOptionsToClass{a4paper}{article}
%
%    Programm calls to get the documentation (example):
%       pdflatex hobsub.dtx
%       makeindex -s gind.ist hobsub.idx
%       pdflatex hobsub.dtx
%       makeindex -s gind.ist hobsub.idx
%       pdflatex hobsub.dtx
%
% Installation:
%    TDS:tex/generic/oberdiek/hobsub.sty
%    TDS:tex/generic/oberdiek/hobsub-generic.sty
%    TDS:tex/generic/oberdiek/hobsub-hyperref.sty
%    TDS:doc/latex/oberdiek/hobsub.pdf
%    TDS:doc/latex/oberdiek/test/hobsub-test1.tex
%    TDS:source/latex/oberdiek/hobsub.dtx
%
%<*ignore>
\begingroup
  \catcode123=1 %
  \catcode125=2 %
  \def\x{LaTeX2e}%
\expandafter\endgroup
\ifcase 0\ifx\install y1\fi\expandafter
         \ifx\csname processbatchFile\endcsname\relax\else1\fi
         \ifx\fmtname\x\else 1\fi\relax
\else\csname fi\endcsname
%</ignore>
%<*install>
\input docstrip.tex
\Msg{************************************************************************}
\Msg{* Installation}
\Msg{* Package: hobsub 2016/05/16 v1.14 Construct package bundles (HO)}
\Msg{************************************************************************}

\keepsilent
\askforoverwritefalse

\let\MetaPrefix\relax
\preamble

This is a generated file.

Project: hobsub
Version: 2016/05/16 v1.14

Copyright (C) 2011, 2012 by
   Heiko Oberdiek <heiko.oberdiek at googlemail.com>

This work may be distributed and/or modified under the
conditions of the LaTeX Project Public License, either
version 1.3c of this license or (at your option) any later
version. This version of this license is in
   http://www.latex-project.org/lppl/lppl-1-3c.txt
and the latest version of this license is in
   http://www.latex-project.org/lppl.txt
and version 1.3 or later is part of all distributions of
LaTeX version 2005/12/01 or later.

This work has the LPPL maintenance status "maintained".

This Current Maintainer of this work is Heiko Oberdiek.

The Base Interpreter refers to any `TeX-Format',
because some files are installed in TDS:tex/generic//.

This work consists of the main source file hobsub.dtx
and the derived files
   hobsub.sty, hobsub.pdf, hobsub.ins, hobsub.drv, hobsub-generic.sty,
   hobsub-hyperref.sty, hobsub-test1.tex.

\endpreamble
\let\MetaPrefix\DoubleperCent

\generate{%
  \file{hobsub.ins}{\from{hobsub.dtx}{install}}%
  \file{hobsub.drv}{\from{hobsub.dtx}{driver}}%
  \usedir{tex/generic/oberdiek}%
  \file{hobsub.sty}{\from{hobsub.dtx}{package}}%
  \usedir{doc/latex/oberdiek/test}%
  \file{hobsub-test1.tex}{\from{hobsub.dtx}{test1}}%
  \nopreamble
  \nopostamble
  \usedir{source/latex/oberdiek/catalogue}%
  \file{hobsub.xml}{\from{hobsub.dtx}{catalogue}}%
}
\let\hobsubSavedNormalLine\normalLine
\def\hobsubGobble#1{}
\begingroup
  \lccode`\0=32 % space
  \lccode`\1=35 % hash
  \lccode`\2=37 % percent
  \lccode`\3=92 % backslash
  \lccode`\4=123 % {
  \lccode`\5=125 % }
\lowercase{\endgroup
  \let\hobsubTokenSpaceOther= 0%
  \let\hobsubTokenHash= 1%
  \let\hobsubTokenPercent= 2%
  \let\hobsubTokenBackslash= 3%
  \let\hobsubTokenBraceLeft= 4%
  \let\hobsubTokenBraceRight= 5%
  \def\hobsubNormalLineWithSpace{%
    \hobsubSavedNormalLine0%
  }%
}
\def\normalLine{%
  \futurelet\hobsubToken\hobsubFirst
}
\def\hobsubFirst{%
  \ifx\hobsubToken\hobsubTokenSpaceOther
    \expandafter\expandafter\expandafter\futurelet
    \expandafter\expandafter\expandafter\hobsubToken
    \expandafter\expandafter\expandafter\hobsubNext
    \expandafter\hobsubGobble
  \else
    \expandafter\hobsubSavedNormalLine
  \fi
}
\def\hobsubNext{%
  \ifx\hobsubToken\hobsubTokenSpaceOther
    \expandafter\expandafter\expandafter\futurelet
    \expandafter\expandafter\expandafter\hobsubToken
    \expandafter\expandafter\expandafter\hobsubNext 
    \expandafter\hobsubGobble
  \else
    \let\hobsubWithSpace=Y%
    \ifx\hobsubToken\hobsubTokenBackslash
      \let\hobsubWithSpace=N%
    \else
      \ifx\hobsubToken\hobsubTokenBraceRight
        \let\hobsubWithSpace=N%
      \else
        \ifx\hobsubToken\hobsubTokenBraceLeft
          \let\hobsubWithSpace=N%
        \else
          \ifx\hobsubToken\hobsubTokenPercent
            \let\hobsubWithSpace=N%
          \else
            \ifx\hobsubToken\hobsubTokenHash
              \let\hobsubWithSpace=N%
            \fi
          \fi
        \fi
      \fi
    \fi
    \ifx\hobsubWithSpace Y%
      \expandafter\expandafter\expandafter\hobsubNormalLineWithSpace
    \else
      \expandafter\expandafter\expandafter\hobsubSavedNormalLine
    \fi
  \fi
}
\generate{%
  \usedir{tex/generic/oberdiek}%
  \file{hobsub-generic.sty}{%
    \from{hobsub.dtx}{generic,preamble,infwarerr}%
    \from{infwarerr.dtx}{package}%
    \from{hobsub.dtx}{post,ltxcmds}%
    \from{ltxcmds.dtx}{package}%
    \from{hobsub.dtx}{post,ifluatex}%
    \from{ifluatex.dtx}{package}%
    \from{hobsub.dtx}{post,ifvtex}%
    \from{ifvtex.dtx}{package}%
    \from{hobsub.dtx}{post,intcalc}%
    \from{intcalc.dtx}{package}%
    \from{hobsub.dtx}{post,ifpdf}%
    \from{ifpdf.dtx}{package}%
    \from{hobsub.dtx}{post,etexcmds}%
    \from{etexcmds.dtx}{package}%
    \from{hobsub.dtx}{post,kvsetkeys}%
    \from{kvsetkeys.dtx}{package}%
    \from{hobsub.dtx}{post,kvdefinekeys}%
    \from{kvdefinekeys.dtx}{package}%
    \from{hobsub.dtx}{post,luatex-loader}%
    \from{luatex.dtx}{loader}%
    \from{hobsub.dtx}{post,pdftexcmds}%
    \from{pdftexcmds.dtx}{package}%
    \from{hobsub.dtx}{post,pdfescape}%
    \from{pdfescape.dtx}{package}%
    \from{hobsub.dtx}{post,bigintcalc}%
    \from{bigintcalc.dtx}{package}%
    \from{hobsub.dtx}{post,bitset}%
    \from{bitset.dtx}{package}%
    \from{hobsub.dtx}{post,uniquecounter}%
    \from{uniquecounter.dtx}{package}%
    \from{hobsub.dtx}{post,atend}%
  }%
}
\generate{%
  \usedir{tex/generic/oberdiek}%
  \file{hobsub-hyperref.sty}{%
    \from{hobsub.dtx}{hyperref,preamble,,letltxmacro}%
    \from{letltxmacro.dtx}{package}%
    \from{hobsub.dtx}{post,hopatch}%
    \from{hopatch.dtx}{package}%
    \from{hobsub.dtx}{post,xcolor-patch}%
    \from{hycolor.dtx}{xcolor}%
    \from{hobsub.dtx}{post,atveryend}%
    \from{atveryend.dtx}{package}%
    \from{hobsub.dtx}{post,atbegshi}%
    \from{atbegshi.dtx}{package}%
    \from{hobsub.dtx}{post,refcount}%
    \from{refcount.dtx}{package}%
    \from{hobsub.dtx}{post,hycolor}%
    \from{hycolor.dtx}{package}%
    \from{hobsub.dtx}{post}%
  }%
}
\let\normalLine\hobsubSavedNormalLine

\catcode32=13\relax% active space
\let =\space%
\Msg{************************************************************************}
\Msg{*}
\Msg{* To finish the installation you have to move the following}
\Msg{* files into a directory searched by TeX:}
\Msg{*}
\Msg{*     hobsub.sty, hobsub-generic.sty, hobsub-hyperref.sty}
\Msg{*}
\Msg{* To produce the documentation run the file `hobsub.drv'}
\Msg{* through LaTeX.}
\Msg{*}
\Msg{* Happy TeXing!}
\Msg{*}
\Msg{************************************************************************}

\endbatchfile
%</install>
%<*ignore>
\fi
%</ignore>
%<*driver>
\NeedsTeXFormat{LaTeX2e}
\ProvidesFile{hobsub.drv}%
  [2016/05/16 v1.14 Construct package bundles (HO)]%
\documentclass{ltxdoc}
\usepackage{holtxdoc}[2011/11/22]
\usepackage{array}
\usepackage{caption}
\renewcommand*{\bottomfraction}{.5}
\renewcommand*{\topfraction}{.5}
\begin{document}
  \DocInput{hobsub.dtx}%
\end{document}
%</driver>
% \fi
%
%
% \CharacterTable
%  {Upper-case    \A\B\C\D\E\F\G\H\I\J\K\L\M\N\O\P\Q\R\S\T\U\V\W\X\Y\Z
%   Lower-case    \a\b\c\d\e\f\g\h\i\j\k\l\m\n\o\p\q\r\s\t\u\v\w\x\y\z
%   Digits        \0\1\2\3\4\5\6\7\8\9
%   Exclamation   \!     Double quote  \"     Hash (number) \#
%   Dollar        \$     Percent       \%     Ampersand     \&
%   Acute accent  \'     Left paren    \(     Right paren   \)
%   Asterisk      \*     Plus          \+     Comma         \,
%   Minus         \-     Point         \.     Solidus       \/
%   Colon         \:     Semicolon     \;     Less than     \<
%   Equals        \=     Greater than  \>     Question mark \?
%   Commercial at \@     Left bracket  \[     Backslash     \\
%   Right bracket \]     Circumflex    \^     Underscore    \_
%   Grave accent  \`     Left brace    \{     Vertical bar  \|
%   Right brace   \}     Tilde         \~}
%
% \GetFileInfo{hobsub.drv}
%
% \title{The \xpackage{hobsub} package}
% \date{2016/05/16 v1.14}
% \author{Heiko Oberdiek\thanks
% {Please report any issues at https://github.com/ho-tex/oberdiek/issues}\\
% \xemail{heiko.oberdiek at googlemail.com}}
%
% \maketitle
%
% \begin{abstract}
% Package \xpackage{hobsub} implements the idea of loading
% several packages in one big collection package.
% \end{abstract}
%
% \tableofcontents
%
% \def\TableGeneric{^^A
%   \TableEntry{\xpackage{infwarerr}}{}{2010/04/08 v1.3}^^A
%   \TableEntry{\xpackage{ltxcmds}}{}{2011/11/09 v1.22}^^A
%   \TableEntry{\xpackage{ifluatex}}{}{2010/03/01 v1.3}^^A
%   \TableEntry{\xpackage{ifvtex}}{}{2010/03/01 v1.5}^^A
%   \TableEntry{\xpackage{intcalc}}{}{2007/09/27 v1.1}^^A
%   \TableEntry{\xpackage{ifpdf}}{\xpackage{ifluatex}}{2011/01/30 v2.3}^^A
%   \TableEntry{\xpackage{etexcmds}}{}{2011/02/16 v1.5}^^A
%   \TableEntry{\xpackage{kvsetkeys}}{}{2012/04/25 v1.16}^^A
%   \TableEntry{\xpackage{kvdefinekeys}}{}{2011/04/07 v1.3}^^A
%   \TableEntry{\xpackage{luatex-loader}}{}{2010/03/09 v0.4}^^A
%   \TableEntry{\xpackage{pdftexcmds}}{\xpackage{ifluatex}, \xpackage{ifpdf}, \xpackage{infwarerr}, \xpackage{ltxcmds}, \xpackage{luatex-loader}}{2011/11/29 v0.20}^^A
%   \TableEntry{\xpackage{pdfescape}}{\xpackage{ltxcmds}, \xpackage{pdftexcmds}}{2011/11/25 v1.13}^^A
%   \TableEntry{\xpackage{bigintcalc}}{\xpackage{pdftexcmds}}{2012/04/08 v1.3}^^A
%   \TableEntry{\xpackage{bitset}}{\xpackage{bigintcalc}, \xpackage{infwarerr}, \xpackage{intcalc}}{2011/01/30 v1.1}^^A
%   \TableEntry{\xpackage{uniquecounter}}{\xpackage{bigintcalc}, \xpackage{infwarerr}}{2011/01/30 v1.2}^^A
% }
% \def\TableHyperref{^^A
%   \TableEntry{\xpackage{letltxmacro}}{}{2010/09/02 v1.4}^^A
%   \TableEntry{\xpackage{hopatch}}{}{2016/05/16 v1.2}^^A
%   \TableEntry{\xpackage{xcolor-patch}}{}{2011/01/30 v1.7}^^A
%   \TableEntry{\xpackage{atveryend}}{}{2011/06/30 v1.8}^^A
%   \TableEntry{\xpackage{atbegshi}}{\xpackage{ifpdf}, \xpackage{infwarerr}, \xpackage{ltxcmds}}{2011/10/05 v1.16}^^A
%   \TableEntry{\xpackage{refcount}}{\xpackage{infwarerr}, \xpackage{ltxcmds}}{2011/10/16 v3.4}^^A
%   \TableEntry{\xpackage{hycolor}}{\xpackage{xcolor-patch}}{2011/01/30 v1.7}^^A
% }
%
% \newenvironment{TableList}[2][htbp]{^^A
%   \begin{table}[{#1}]^^A
%   \centering
%   \caption{Overview \xpackage{hobsub-#2}}^^A
%   \label{tab:#2}^^A
%   \begin{tabular}{ll>{\raggedright}p{.4\linewidth}}^^A
%   \bfseries Package & \bfseries Release &
%   \bfseries Dependencies\tabularnewline
%   \hline
% }{^^A
%   \end{tabular}^^A
%   \end{table}^^A
% }
% \newcommand*{\TableEntry}[3]{%
%   #1&#3&#2\tabularnewline
% }
%
% \renewcommand*{\arraystretch}{1.1}
% \setlength{\extrarowheight}{2pt}
%
% \section{Documentation}
%
% In January 2011 the mailing list lualatex-dev discussed
% ``quietening lualatex console output'' \cite{lualatex:quiet}.
% Inspired from this, I implemented this package \xpackage{hobsub}
% to load several packages in one file. The package is applied for
% two packages collections that are subsets of my package bundle
% \xpackage{oberdiek}.
% \begin{description}
% \item[\xpackage{hobsub-generic}] is a collection of packages
%   that can also be used with \hologo{plainTeX}.
% \item[\xpackage{hobsub-hyperref}] is a collection of packages
%   that is used by package \xpackage{hyperref}.
% \end{description}
%
% Package \xpackage{hobsub} provides the macros that
% are used in the collection packages and might be used
% in other projects as well. But there are many caveats,
% some of them are discussed below.
%
% The most serious problem is consistency. A package
% that is available as standalone package and is part
% of a collection might have different versions.
% Because my packages are organized in a bundle and should
% be installed and updated as bundle, the risk seems low
% to me. However, if a single package is updated manually, for
% example for testing, then it should be loaded before the
% collection that contains the package with an older version.
% In a collection, the package will not be loaded again if it
% is already loaded.
%
% \subsection{Usage}
%
% The collection packages \xpackage{hobsub-generic} and
% \xpackage{hobsub-hyperref} are loaded as usual:
% \begin{quote}
%  \begin{tabular}{@{}ll@{}}
%    |\usepackage{hobsub-generic}| & \hologo{LaTeX}\\
%    |\input hobsub-generic.sty| & \hologo{plainTeX}
%  \end{tabular}
% \end{quote}
% or
% \begin{quote}|\usepackage{hobsub-hyperref}|\end{quote}
% Of course, both \cs{usepackage} or \cs{RequirePackage}
% can be used.
%
% If you need requirements on the version date of a package
% inside a collection, then specify them afterwards, for
% example:
% \begin{quote}
%   |\usepackage{hobsub-generic}|\\
%   |\usepackage{ltxcmds}[2010/12/04]% |^^A
%   {\itshape |\ltx@ifblank is needed|}
% \end{quote}
%
% Also it is not garanteed that a package will be always
% part of a collection. Therefore it does not harm,
% to make the requirements explicite \emph{after}
% a collection is loaded, see the previous example.
% \hologo{LaTeX} knows that a
% package is already loaded and does not load it again.
%
% \subsection{Supported features}
%
% \begin{description}
% \item[No reloads:] If a package is already loaded then it
%  is not reloaded again if it is part of a collection package.
% \item[\cs{listfiles}:]
%   Each package that is loaded in a collection is also
%   added to the output of \cs{listfiles}.
% \item[\cs{@currname}, \cs{@currext}]
%   are updated, they are used by \hologo{LaTeX}'s
%  \cs{ProvidesPackage}.
% \item[\cs{AtEndOfPackage}:]
%   \hologo{LaTeX}'s hook mechanism that is used for
%   \cs{AtEndOfPackage} is supported. However none of the
%   packages in the collections are using \cs{AtEndOfPackage},
%   therefore this feature is completely untested.
% \end{description}
%
% \subsection{Common limitations}
%
% \begin{description}
% \item[\cs{endinput}:]
%   Often a package is using \cs{endinput}, either added by
%   \xpackage{docstrip}, or to stop package loading at an earlier
%   time. Therefore \cs{endinput} must be catched during package
%   loading to prevent that the collection package is closed
%   too early before reading all packages it contains.
%
%   This redefinition of \cs{endinput} fails if the package loads
%   other files that contain \cs{endinput}. To some degree it could
%   be catched by redefining macros that load files. But this will
%   not work for the primitive \cs{input}.
% \item[Options:]
%   \hologo{LaTeX}'s package options are not supported.
%   Therefore the packages that are put in a collection must
%   not support options.
% \item[Hook packages:]
%   Some packages and classes provides hooks for packages, see
%   packages \xpackage{scrlfile}, \xpackage{filehook}
%   or class \xclass{memoir}. These hooks are not supported
%   for embedded packages inside a collection package.
%   Because of the problem with \cs{endinput}, these hooks
%   are not supported. The hooks might contain code that
%   loads other files that execute \cs{endinput}.
% \end{description}
%
% \subsection{Package \xpackage{hobsub-generic}}
%
% \begin{TableList}[bp]{generic}\TableGeneric\end{TableList}
%
% Collection `generic' contains some of the \hologo{plainTeX} compatible
% packages of my bundle, see table \ref{tab:generic}.
% It adds the following restrictions to any package
% that is part of this collection:
% \begin{description}
% \item[Generic formats:] The following formats must be supported:
%   \hologo{LaTeX}, \hologo{plainTeX} and even \hologo{iniTeX}.
%   This applies to the collection package \xpackage{hobsub-generic}
%   the same way.
% \item[Prefixed macros:]
%   Any package of this collection should only define macros with
%   prefixed names to avoid name clashes. Resources like registers
%   must not be allocated and other global states should not be
%   changed except for added macro definitions.
% \end{description}
% Thus the package \xpackage{hobsub-generic} can also preloaded
% in the format file with \hologo{iniTeX}.
%
% \subsection{Package \xpackage{hobsub-hyperref}}
%
% \begin{TableList}[tp]{hyperref}\TableHyperref\end{TableList}
%
% Collection `hyperref' contains additional packages that are
% used by package \xpackage{hyperref} and that does not fit
% in collection `generic', see table{tab:hyperref}.
% The latter collection package
% \xpackage{hobsub-generic} is loaded via a conventional
% \cs{RequirePackage}.
%
% Because most of the packages in the collection `hyperref'
% needs \hologo{LaTeX}, the collection requires \hologo{LaTeX}.
%
% \subsection{Package \xpackage{hobsub}}
%
% This section is for advanced users that might want to
% build a collection package. Then it can be done by
% using \xpackage{hobsub}:
% \begin{quote}
%   \dots\\
%   |\RequirePackage{hobsub}[2016/05/16]|\\
%   \dots\\
%   |\hobsub@StartPackage{foo1}|\\
%   \textit{\dots\unskip\ package code of \xfile{foo1.sty} \dots}\\
%   |\endinput|\\
%   |<hobsub>|\\
%   |\hobsub@StopPackage|\\
%   |\hobsub@StartPackage{bar2}|\\
%   \textit{\dots\unskip\ package code of \xfile{bar2.sty} \dots}\\
%   |\endinput|\\
%   |<hobsub>|\\
%   |\hobsub@StopPackage|
%   \dots
% \end{quote}
% But be aware of the caveats and limitations, see above.
% Only quite a small portion of the packages can be embedded
% like this.
%
% \clearpage
%
% \StopEventually{
% }
%
% \section{Implementation}
%
% \subsection{Reload check and package identification}
%
% \subsubsection{Package \xpackage{hobsub}}
%
%    \begin{macrocode}
%<*package>
%    \end{macrocode}
%    Reload check, especially if the package is not used with \LaTeX.
%    \begin{macrocode}
\begingroup\catcode61\catcode48\catcode32=10\relax%
  \catcode13=5 % ^^M
  \endlinechar=13 %
  \catcode35=6 % #
  \catcode39=12 % '
  \catcode44=12 % ,
  \catcode45=12 % -
  \catcode46=12 % .
  \catcode58=12 % :
  \catcode64=11 % @
  \catcode123=1 % {
  \catcode125=2 % }
  \expandafter\let\expandafter\x\csname ver@hobsub.sty\endcsname
  \ifx\x\relax % plain-TeX, first loading
  \else
    \def\empty{}%
    \ifx\x\empty % LaTeX, first loading,
      % variable is initialized, but \ProvidesPackage not yet seen
    \else
      \expandafter\ifx\csname PackageInfo\endcsname\relax
        \def\x#1#2{%
          \immediate\write-1{Package #1 Info: #2.}%
        }%
      \else
        \def\x#1#2{\PackageInfo{#1}{#2, stopped}}%
      \fi
      \x{hobsub}{The package is already loaded}%
      \aftergroup\endinput
    \fi
  \fi
\endgroup%
%    \end{macrocode}
%    Package identification:
%    \begin{macrocode}
\begingroup\catcode61\catcode48\catcode32=10\relax%
  \catcode13=5 % ^^M
  \endlinechar=13 %
  \catcode35=6 % #
  \catcode39=12 % '
  \catcode40=12 % (
  \catcode41=12 % )
  \catcode44=12 % ,
  \catcode45=12 % -
  \catcode46=12 % .
  \catcode47=12 % /
  \catcode58=12 % :
  \catcode64=11 % @
  \catcode91=12 % [
  \catcode93=12 % ]
  \catcode123=1 % {
  \catcode125=2 % }
  \expandafter\ifx\csname ProvidesPackage\endcsname\relax
    \def\x#1#2#3[#4]{\endgroup
      \immediate\write-1{Package: #3 #4}%
      \xdef#1{#4}%
    }%
  \else
    \def\x#1#2[#3]{\endgroup
      #2[{#3}]%
      \ifx#1\@undefined
        \xdef#1{#3}%
      \fi
      \ifx#1\relax
        \xdef#1{#3}%
      \fi
    }%
  \fi
\expandafter\x\csname ver@hobsub.sty\endcsname
\ProvidesPackage{hobsub}%
  [2016/05/16 v1.14 Construct package bundles (HO)]%
%    \end{macrocode}
%    \begin{macrocode}
%</package>
%    \end{macrocode}
%
% \subsubsection{Package \xpackage{hobsub-generic}}
%
%    \begin{macrocode}
%<*generic>
%    \end{macrocode}
%    Reload check, especially if the package is not used with \LaTeX.
%    \begin{macrocode}
\begingroup\catcode61\catcode48\catcode32=10\relax%
  \catcode13=5 % ^^M
  \endlinechar=13 %
  \catcode35=6 % #
  \catcode39=12 % '
  \catcode44=12 % ,
  \catcode45=12 % -
  \catcode46=12 % .
  \catcode58=12 % :
  \catcode64=11 % @
  \catcode123=1 % {
  \catcode125=2 % }
  \expandafter\let\expandafter\x\csname ver@hobsub-generic.sty\endcsname
  \ifx\x\relax % plain-TeX, first loading
  \else
    \def\empty{}%
    \ifx\x\empty % LaTeX, first loading,
      % variable is initialized, but \ProvidesPackage not yet seen
    \else
      \expandafter\ifx\csname PackageInfo\endcsname\relax
        \def\x#1#2{%
          \immediate\write-1{Package #1 Info: #2.}%
        }%
      \else
        \def\x#1#2{\PackageInfo{#1}{#2, stopped}}%
      \fi
      \x{hobsub-generic}{The package is already loaded}%
      \aftergroup\endinput
    \fi
  \fi
\endgroup%
%    \end{macrocode}
%    Package identification:
%    \begin{macrocode}
\begingroup\catcode61\catcode48\catcode32=10\relax%
  \catcode13=5 % ^^M
  \endlinechar=13 %
  \catcode35=6 % #
  \catcode39=12 % '
  \catcode40=12 % (
  \catcode41=12 % )
  \catcode44=12 % ,
  \catcode45=12 % -
  \catcode46=12 % .
  \catcode47=12 % /
  \catcode58=12 % :
  \catcode64=11 % @
  \catcode91=12 % [
  \catcode93=12 % ]
  \catcode123=1 % {
  \catcode125=2 % }
  \expandafter\ifx\csname ProvidesPackage\endcsname\relax
    \def\x#1#2#3[#4]{\endgroup
      \immediate\write-1{Package: #3 #4}%
      \xdef#1{#4}%
    }%
  \else
    \def\x#1#2[#3]{\endgroup
      #2[{#3}]%
      \ifx#1\@undefined
        \xdef#1{#3}%
      \fi
      \ifx#1\relax
        \xdef#1{#3}%
      \fi
    }%
  \fi
\expandafter\x\csname ver@hobsub-generic.sty\endcsname
\ProvidesPackage{hobsub-generic}%
  [2016/05/16 v1.14 Bundle oberdiek, subset generic (HO)]%
%    \end{macrocode}
%    \begin{macrocode}
%</generic>
%    \end{macrocode}
%
% \subsubsection{Package \xpackage{hobsub-hyperref}}
%
%    \begin{macrocode}
%<*hyperref>
%    \end{macrocode}
%    Reload check, especially if the package is not used with \LaTeX.
%    \begin{macrocode}
\begingroup\catcode61\catcode48\catcode32=10\relax%
  \catcode13=5 % ^^M
  \endlinechar=13 %
  \catcode35=6 % #
  \catcode39=12 % '
  \catcode44=12 % ,
  \catcode45=12 % -
  \catcode46=12 % .
  \catcode58=12 % :
  \catcode64=11 % @
  \catcode123=1 % {
  \catcode125=2 % }
  \expandafter\let\expandafter\x\csname ver@hobsub-hyperref.sty\endcsname
  \ifx\x\relax % plain-TeX, first loading
  \else
    \def\empty{}%
    \ifx\x\empty % LaTeX, first loading,
      % variable is initialized, but \ProvidesPackage not yet seen
    \else
      \expandafter\ifx\csname PackageInfo\endcsname\relax
        \def\x#1#2{%
          \immediate\write-1{Package #1 Info: #2.}%
        }%
      \else
        \def\x#1#2{\PackageInfo{#1}{#2, stopped}}%
      \fi
      \x{hobsub-hyperref}{The package is already loaded}%
      \aftergroup\endinput
    \fi
  \fi
\endgroup%
%    \end{macrocode}
%    Package identification:
%    \begin{macrocode}
\begingroup\catcode61\catcode48\catcode32=10\relax%
  \catcode13=5 % ^^M
  \endlinechar=13 %
  \catcode35=6 % #
  \catcode39=12 % '
  \catcode40=12 % (
  \catcode41=12 % )
  \catcode44=12 % ,
  \catcode45=12 % -
  \catcode46=12 % .
  \catcode47=12 % /
  \catcode58=12 % :
  \catcode64=11 % @
  \catcode91=12 % [
  \catcode93=12 % ]
  \catcode123=1 % {
  \catcode125=2 % }
  \expandafter\ifx\csname ProvidesPackage\endcsname\relax
    \def\x#1#2#3[#4]{\endgroup
      \immediate\write-1{Package: #3 #4}%
      \xdef#1{#4}%
    }%
  \else
    \def\x#1#2[#3]{\endgroup
      #2[{#3}]%
      \ifx#1\@undefined
        \xdef#1{#3}%
      \fi
      \ifx#1\relax
        \xdef#1{#3}%
      \fi
    }%
  \fi
\expandafter\x\csname ver@hobsub-hyperref.sty\endcsname
\ProvidesPackage{hobsub-hyperref}%
  [2016/05/16 v1.14 Bundle oberdiek, subset hyperref (HO)]%
%    \end{macrocode}
%    \begin{macrocode}
%</hyperref>
%    \end{macrocode}
%
% \subsection{Catcodes}
%
%    \begin{macrocode}
%<*package|generic|hyperref>
\begingroup\catcode61\catcode48\catcode32=10\relax%
  \catcode13=5 % ^^M
  \endlinechar=13 %
  \catcode123=1 % {
  \catcode125=2 % }
  \catcode64=11 % @
  \def\x{\endgroup
    \expandafter\edef\csname HOBsub@AtEnd\endcsname{%
      \endlinechar=\the\endlinechar\relax
      \catcode13=\the\catcode13\relax
      \catcode32=\the\catcode32\relax
      \catcode35=\the\catcode35\relax
      \catcode61=\the\catcode61\relax
      \catcode64=\the\catcode64\relax
      \catcode123=\the\catcode123\relax
      \catcode125=\the\catcode125\relax
    }%
  }%
\x\catcode61\catcode48\catcode32=10\relax%
\catcode13=5 % ^^M
\endlinechar=13 %
\catcode35=6 % #
\catcode64=11 % @
\catcode123=1 % {
\catcode125=2 % }
\def\TMP@EnsureCode#1#2{%
  \edef\HOBsub@AtEnd{%
    \HOBsub@AtEnd
    \catcode#1=\the\catcode#1\relax
  }%
  \catcode#1=#2\relax
}
\TMP@EnsureCode{39}{12}% '
\TMP@EnsureCode{40}{12}% (
\TMP@EnsureCode{41}{12}% )
\TMP@EnsureCode{45}{12}% -
\TMP@EnsureCode{46}{12}% .
\TMP@EnsureCode{47}{12}% /
\TMP@EnsureCode{58}{12}% :
\TMP@EnsureCode{60}{12}% <
\TMP@EnsureCode{62}{12}% >
\TMP@EnsureCode{96}{12}% `
\edef\HOBsub@AtEnd{\HOBsub@AtEnd\noexpand\endinput}
%    \end{macrocode}
%
%    \begin{macrocode}
%</package|generic|hyperref>
%    \end{macrocode}
%
% \subsection{Package \xpackage{hobsub-hyperref} loads \xpackage{hobsub-generic}}
%
%    \begin{macrocode}
%<*hyperref>
\NeedsTeXFormat{LaTeX2e}
\RequirePackage{hobsub-generic}[2016/05/16]
%</hyperref>
%    \end{macrocode}
%
% \subsection{Preamble}
%
%    \begin{macrocode}
%<*preamble>
\begingroup\expandafter\expandafter\expandafter\endgroup
\expandafter\ifx\csname ver@hobsub.sty\endcsname\relax
\else
  \expandafter\hobsub@GobbleRemainingPackage
\fi
%</preamble>
%    \end{macrocode}
%
%
%    \begin{macrocode}
%<*package|preamble>
%    \end{macrocode}
%
%    \begin{macro}{\HOBsub@OrgEndinput}
%    \begin{macrocode}
\begingroup\expandafter\expandafter\expandafter\endgroup
\expandafter\ifx\csname HOBsub@OrgEndinput\endcsname\relax
  \let\HOBsub@OrgEndinput\endinput
\fi
%    \end{macrocode}
%    \end{macro}
%    \begin{macro}{\hobsub@GobbleRemainingPackage}
%    \begin{macrocode}
\def\hobsub@GobbleRemainingPackage{%
  \begingroup
  \catcode92=14 % backslash: comment
  \catcode32=14 % space: comment
  \catcode35=14 % hash: comment
  \catcode123=14 % left brace: comment
  \catcode125=14 % right brace: comment
  \catcode60=3 % less: math
  \catcode62=4 % greater: align
  \endlinechar=-1 %
  \HOBsub@GobbleRemainingPackage
}
%    \end{macrocode}
%    \end{macro}
%    \begin{macro}{\HOBsub@GobbleRemainingPackage}
%    \begin{macrocode}
\catcode60=3 % less: dollar
\catcode62=4 % greater: align
\long\def\HOBsub@GobbleRemainingPackage#1<hobsub>{%
  \endgroup
}
\catcode60=12 % less: other
\catcode62=12 % greater: other
%    \end{macrocode}
%    \end{macro}
%
%    \begin{macro}{\hobsub@StartPackage}
%    \begin{macrocode}
\def\hobsub@StartPackage#1{%
  \begingroup\expandafter\expandafter\expandafter\endgroup
  \expandafter\ifx\csname ver@#1.sty\endcsname\relax
    \let\HOBsub@OrgCurrName\@currname
    \let\HOBsub@OrgCurrExt\@currext
    \csname @pushfilename\endcsname
    \def\@currname{#1}%
    \def\@currext{sty}%
    \expandafter\def\csname\@currname.\@currext-h@@k\endcsname{}%
    \let\endinput\hobsub@GobbleRemainingPackage
    \def\hobsub@StopPackage{%
      \let\hobsub@StopPackage\relax
      \HOBsub@StopPackage
%     \hobsub@Info{hobsub}{Package `#1' loaded}%
    }%
    \hobsub@AddToFileList{#1.sty}%
  \else
    \hobsub@Info{hobsub}{Skipping package `#1' (already loaded)}%
    \let\hobsub@StopPackage\relax
    \expandafter\hobsub@GobbleRemainingPackage
  \fi
}
%    \end{macrocode}
%    \end{macro}
%
%    \begin{macro}{\hobsub@StopPackage}
%    \begin{macrocode}
\let\hobsub@StopPackage\relax
%    \end{macrocode}
%    \end{macro}
%
%    \begin{macro}{\hobsub@Info}
%    \begin{macrocode}
\def\hobsub@Info#1#2{%
  \begingroup\expandafter\expandafter\expandafter\endgroup
  \expandafter\ifx\csname @PackageInfoNoLine\endcsname\relax
    \immediate\write-1{Package #1 Info: #2.}%
  \else
    \let\hobsub@Info\@PackageInfoNoLine
    \hobsub@Info{#1}{#2}%
  \fi
}
%    \end{macrocode}
%    \end{macro}
%
%    \begin{macro}{\HOBsub@StopPackage}
%    \begin{macrocode}
\def\HOBsub@StopPackage{%
  \csname\@currname.\@currext-h@@k\endcsname
  \let\endinput\HOBsub@OrgEndinput
  \csname @popfilename\endcsname
  \let\@currname\HOBsub@OrgCurrName
  \let\@currext\HOBsub@OrgCurrExt
}
%    \end{macrocode}
%    \end{macro}
%
%    \begin{macro}{\hobsub@AddToFileList}
%    \begin{macrocode}
\begingroup\expandafter\expandafter\expandafter\endgroup
\expandafter\ifx\csname @addtofilelist\endcsname\relax
  \def\hobsub@AddToFileList#1{}%
\else
  \def\hobsub@AddToFileList#1{%
    \@addtofilelist{#1}%
  }%
\fi
%    \end{macrocode}
%    \end{macro}
%
%    \begin{macrocode}
%</package|preamble>
%    \end{macrocode}
%
% \subsection{End of preamble for subset packages}
%
%    Add the end marker for possible skipping of previous
%    macros, if package \xpackage{hobsub} was already loaded.
%    \begin{macrocode}
%<*preamble>
\hobsub@GobbleRemainingPackage
<hobsub>
%    \end{macrocode}
%
%    All macros of package \xpackage{hobsub} are now available.
%    Now we officially load the package \xpackage{hobsub} inline
%    for proper package identification.
%    \begin{macrocode}
\hobsub@StartPackage{hobsub}%
\begingroup\expandafter\expandafter\expandafter\endgroup
\expandafter\ifx\csname ProvidesPackage\endcsname\relax
  \immediate\write-1{%
    Package: hobsub 2016/05/16 v1.14 Construct package bundles (HO)%
  }%
\else
  \ProvidesPackage{hobsub}%
  [2016/05/16 v1.14 Construct package bundles (HO)]%
\fi
%</preamble>
%    \end{macrocode}
%
%
%    \begin{macrocode}
%<*post|preamble>
 \endinput
<hobsub>
\hobsub@StopPackage
%</post|preamble>
%    \end{macrocode}
%
% \subsection{Package list}
%
%    \begin{macrocode}
%<infwarerr>\hobsub@StartPackage{infwarerr}
%<ltxcmds>\hobsub@StartPackage{ltxcmds}
%<ifluatex>\hobsub@StartPackage{ifluatex}
%<ifvtex>\hobsub@StartPackage{ifvtex}
%<intcalc>\hobsub@StartPackage{intcalc}
%<ifpdf>\hobsub@StartPackage{ifpdf}
%<etexcmds>\hobsub@StartPackage{etexcmds}
%<kvsetkeys>\hobsub@StartPackage{kvsetkeys}
%<kvdefinekeys>\hobsub@StartPackage{kvdefinekeys}
%<*luatex-loader>
\ifluatex
\else
  \expandafter\hobsub@GobbleRemainingPackage
\fi
\hobsub@StartPackage{luatex-loader}
%</luatex-loader>
%<pdftexcmds>\hobsub@StartPackage{pdftexcmds}
%<pdfescape>\hobsub@StartPackage{pdfescape}
%<bigintcalc>\hobsub@StartPackage{bigintcalc}
%<bitset>\hobsub@StartPackage{bitset}
%<uniquecounter>\hobsub@StartPackage{uniquecounter}
%<letltxmacro>\hobsub@StartPackage{letltxmacro}
%<hopatch>\hobsub@StartPackage{hopatch}
%<xcolor-patch>\hobsub@StartPackage{xcolor-patch}
%<atveryend>\hobsub@StartPackage{atveryend}
%<atbegshi>\hobsub@StartPackage{atbegshi}
%<refcount>\hobsub@StartPackage{refcount}
%<hycolor>\hobsub@StartPackage{hycolor}
%    \end{macrocode}
%
% \subsection{End of packages}
%
%    \begin{macrocode}
%<package|atend>\HOBsub@AtEnd%
%    \end{macrocode}
%
% \section{Test}
%
% \subsection{Catcode checks for loading}
%
%    \begin{macrocode}
%<*test1>
%    \end{macrocode}
%    \begin{macrocode}
\catcode`\{=1 %
\catcode`\}=2 %
\catcode`\#=6 %
\catcode`\@=11 %
\expandafter\ifx\csname count@\endcsname\relax
  \countdef\count@=255 %
\fi
\expandafter\ifx\csname @gobble\endcsname\relax
  \long\def\@gobble#1{}%
\fi
\expandafter\ifx\csname @firstofone\endcsname\relax
  \long\def\@firstofone#1{#1}%
\fi
\expandafter\ifx\csname loop\endcsname\relax
  \expandafter\@firstofone
\else
  \expandafter\@gobble
\fi
{%
  \def\loop#1\repeat{%
    \def\body{#1}%
    \iterate
  }%
  \def\iterate{%
    \body
      \let\next\iterate
    \else
      \let\next\relax
    \fi
    \next
  }%
  \let\repeat=\fi
}%
\def\RestoreCatcodes{}
\count@=0 %
\loop
  \edef\RestoreCatcodes{%
    \RestoreCatcodes
    \catcode\the\count@=\the\catcode\count@\relax
  }%
\ifnum\count@<255 %
  \advance\count@ 1 %
\repeat

\def\RangeCatcodeInvalid#1#2{%
  \count@=#1\relax
  \loop
    \catcode\count@=15 %
  \ifnum\count@<#2\relax
    \advance\count@ 1 %
  \repeat
}
\def\RangeCatcodeCheck#1#2#3{%
  \count@=#1\relax
  \loop
    \ifnum#3=\catcode\count@
    \else
      \errmessage{%
        Character \the\count@\space
        with wrong catcode \the\catcode\count@\space
        instead of \number#3%
      }%
    \fi
  \ifnum\count@<#2\relax
    \advance\count@ 1 %
  \repeat
}
\def\space{ }
\expandafter\ifx\csname LoadCommand\endcsname\relax
  \def\LoadCommand{\input hobsub.sty\relax}%
\fi
\def\Test{%
  \RangeCatcodeInvalid{0}{47}%
  \RangeCatcodeInvalid{58}{64}%
  \RangeCatcodeInvalid{91}{96}%
  \RangeCatcodeInvalid{123}{255}%
  \catcode`\@=12 %
  \catcode`\\=0 %
  \catcode`\%=14 %
  \LoadCommand
  \RangeCatcodeCheck{0}{36}{15}%
  \RangeCatcodeCheck{37}{37}{14}%
  \RangeCatcodeCheck{38}{47}{15}%
  \RangeCatcodeCheck{48}{57}{12}%
  \RangeCatcodeCheck{58}{63}{15}%
  \RangeCatcodeCheck{64}{64}{12}%
  \RangeCatcodeCheck{65}{90}{11}%
  \RangeCatcodeCheck{91}{91}{15}%
  \RangeCatcodeCheck{92}{92}{0}%
  \RangeCatcodeCheck{93}{96}{15}%
  \RangeCatcodeCheck{97}{122}{11}%
  \RangeCatcodeCheck{123}{255}{15}%
  \RestoreCatcodes
}
\Test
\csname @@end\endcsname
\end
%    \end{macrocode}
%    \begin{macrocode}
%</test1>
%    \end{macrocode}
%
% \section{Installation}
%
% \subsection{Download}
%
% \paragraph{Package.} This package is available on
% CTAN\footnote{\url{http://ctan.org/pkg/hobsub}}:
% \begin{description}
% \item[\CTAN{macros/latex/contrib/oberdiek/hobsub.dtx}] The source file.
% \item[\CTAN{macros/latex/contrib/oberdiek/hobsub.pdf}] Documentation.
% \end{description}
%
%
% \paragraph{Bundle.} All the packages of the bundle `oberdiek'
% are also available in a TDS compliant ZIP archive. There
% the packages are already unpacked and the documentation files
% are generated. The files and directories obey the TDS standard.
% \begin{description}
% \item[\CTAN{install/macros/latex/contrib/oberdiek.tds.zip}]
% \end{description}
% \emph{TDS} refers to the standard ``A Directory Structure
% for \TeX\ Files'' (\CTAN{tds/tds.pdf}). Directories
% with \xfile{texmf} in their name are usually organized this way.
%
% \subsection{Bundle installation}
%
% \paragraph{Unpacking.} Unpack the \xfile{oberdiek.tds.zip} in the
% TDS tree (also known as \xfile{texmf} tree) of your choice.
% Example (linux):
% \begin{quote}
%   |unzip oberdiek.tds.zip -d ~/texmf|
% \end{quote}
%
% \paragraph{Script installation.}
% Check the directory \xfile{TDS:scripts/oberdiek/} for
% scripts that need further installation steps.
% Package \xpackage{attachfile2} comes with the Perl script
% \xfile{pdfatfi.pl} that should be installed in such a way
% that it can be called as \texttt{pdfatfi}.
% Example (linux):
% \begin{quote}
%   |chmod +x scripts/oberdiek/pdfatfi.pl|\\
%   |cp scripts/oberdiek/pdfatfi.pl /usr/local/bin/|
% \end{quote}
%
% \subsection{Package installation}
%
% \paragraph{Unpacking.} The \xfile{.dtx} file is a self-extracting
% \docstrip\ archive. The files are extracted by running the
% \xfile{.dtx} through \plainTeX:
% \begin{quote}
%   \verb|tex hobsub.dtx|
% \end{quote}
%
% \paragraph{TDS.} Now the different files must be moved into
% the different directories in your installation TDS tree
% (also known as \xfile{texmf} tree):
% \begin{quote}
% \def\t{^^A
% \begin{tabular}{@{}>{\ttfamily}l@{ $\rightarrow$ }>{\ttfamily}l@{}}
%   hobsub.sty & tex/generic/oberdiek/hobsub.sty\\
%   hobsub-generic.sty & tex/generic/oberdiek/hobsub-generic.sty\\
%   hobsub-hyperref.sty & tex/generic/oberdiek/hobsub-hyperref.sty\\
%   hobsub.pdf & doc/latex/oberdiek/hobsub.pdf\\
%   test/hobsub-test1.tex & doc/latex/oberdiek/test/hobsub-test1.tex\\
%   hobsub.dtx & source/latex/oberdiek/hobsub.dtx\\
% \end{tabular}^^A
% }^^A
% \sbox0{\t}^^A
% \ifdim\wd0>\linewidth
%   \begingroup
%     \advance\linewidth by\leftmargin
%     \advance\linewidth by\rightmargin
%   \edef\x{\endgroup
%     \def\noexpand\lw{\the\linewidth}^^A
%   }\x
%   \def\lwbox{^^A
%     \leavevmode
%     \hbox to \linewidth{^^A
%       \kern-\leftmargin\relax
%       \hss
%       \usebox0
%       \hss
%       \kern-\rightmargin\relax
%     }^^A
%   }^^A
%   \ifdim\wd0>\lw
%     \sbox0{\small\t}^^A
%     \ifdim\wd0>\linewidth
%       \ifdim\wd0>\lw
%         \sbox0{\footnotesize\t}^^A
%         \ifdim\wd0>\linewidth
%           \ifdim\wd0>\lw
%             \sbox0{\scriptsize\t}^^A
%             \ifdim\wd0>\linewidth
%               \ifdim\wd0>\lw
%                 \sbox0{\tiny\t}^^A
%                 \ifdim\wd0>\linewidth
%                   \lwbox
%                 \else
%                   \usebox0
%                 \fi
%               \else
%                 \lwbox
%               \fi
%             \else
%               \usebox0
%             \fi
%           \else
%             \lwbox
%           \fi
%         \else
%           \usebox0
%         \fi
%       \else
%         \lwbox
%       \fi
%     \else
%       \usebox0
%     \fi
%   \else
%     \lwbox
%   \fi
% \else
%   \usebox0
% \fi
% \end{quote}
% If you have a \xfile{docstrip.cfg} that configures and enables \docstrip's
% TDS installing feature, then some files can already be in the right
% place, see the documentation of \docstrip.
%
% \subsection{Refresh file name databases}
%
% If your \TeX~distribution
% (\teTeX, \mikTeX, \dots) relies on file name databases, you must refresh
% these. For example, \teTeX\ users run \verb|texhash| or
% \verb|mktexlsr|.
%
% \subsection{Some details for the interested}
%
% \paragraph{Attached source.}
%
% The PDF documentation on CTAN also includes the
% \xfile{.dtx} source file. It can be extracted by
% AcrobatReader 6 or higher. Another option is \textsf{pdftk},
% e.g. unpack the file into the current directory:
% \begin{quote}
%   \verb|pdftk hobsub.pdf unpack_files output .|
% \end{quote}
%
% \paragraph{Unpacking with \LaTeX.}
% The \xfile{.dtx} chooses its action depending on the format:
% \begin{description}
% \item[\plainTeX:] Run \docstrip\ and extract the files.
% \item[\LaTeX:] Generate the documentation.
% \end{description}
% If you insist on using \LaTeX\ for \docstrip\ (really,
% \docstrip\ does not need \LaTeX), then inform the autodetect routine
% about your intention:
% \begin{quote}
%   \verb|latex \let\install=y\input{hobsub.dtx}|
% \end{quote}
% Do not forget to quote the argument according to the demands
% of your shell.
%
% \paragraph{Generating the documentation.}
% You can use both the \xfile{.dtx} or the \xfile{.drv} to generate
% the documentation. The process can be configured by the
% configuration file \xfile{ltxdoc.cfg}. For instance, put this
% line into this file, if you want to have A4 as paper format:
% \begin{quote}
%   \verb|\PassOptionsToClass{a4paper}{article}|
% \end{quote}
% An example follows how to generate the
% documentation with pdf\LaTeX:
% \begin{quote}
%\begin{verbatim}
%pdflatex hobsub.dtx
%makeindex -s gind.ist hobsub.idx
%pdflatex hobsub.dtx
%makeindex -s gind.ist hobsub.idx
%pdflatex hobsub.dtx
%\end{verbatim}
% \end{quote}
%
% \section{Catalogue}
%
% The following XML file can be used as source for the
% \href{http://mirror.ctan.org/help/Catalogue/catalogue.html}{\TeX\ Catalogue}.
% The elements \texttt{caption} and \texttt{description} are imported
% from the original XML file from the Catalogue.
% The name of the XML file in the Catalogue is \xfile{hobsub.xml}.
%    \begin{macrocode}
%<*catalogue>
<?xml version='1.0' encoding='us-ascii'?>
<!DOCTYPE entry SYSTEM 'catalogue.dtd'>
<entry datestamp='$Date$' modifier='$Author$' id='hobsub'>
  <name>hobsub</name>
  <caption>Construct package bundles</caption>
  <authorref id='auth:oberdiek'/>
  <copyright owner='Heiko Oberdiek' year='2011,2012'/>
  <license type='lppl1.3'/>
  <version number='1.14'/>
  <description>
    The package offers a means of loading a bundle of (other) packages
    &#x201C;at once&#x201D;, offering a means of reducing the amount of log output
    as the packages are loaded, and an aide-memoire for complicated
    requirements.  In some ways, loading a package via hobsub behaves
    much like loading the package in the normal way; for example,
    reloads are suppressed, as in LaTeX proper, and correct
    <tt>\listfiles</tt> output is produced.
    <p/>
    Examples provided are a &#x2018;generic&#x2019; bundle (comprising the author&#x2019;s
    packages that can be used with Plain TeX) and a &#x2018;hyperref&#x2019; bundle
    (comprising packages useful when generating hypertext output with
    LaTeX).
    <p/>
    The package is part of the <xref refid='oberdiek'>oberdiek</xref> bundle.
  </description>
  <documentation details='Package documentation'
      href='ctan:/macros/latex/contrib/oberdiek/hobsub.pdf'/>
  <ctan file='true' path='/macros/latex/contrib/oberdiek/hobsub.dtx'/>
  <miktex location='oberdiek'/>
  <texlive location='oberdiek'/>
  <install path='/macros/latex/contrib/oberdiek/oberdiek.tds.zip'/>
</entry>
%</catalogue>
%    \end{macrocode}
%
% \begin{thebibliography}{9}
% \raggedright
% \bibitem{lualatex:quiet}
% Will Robertson,
% \textit{[lltx] quietening lualatex console output};
% mailing list \textsf{lualatex-dev at tug.org},
% 2011-01-29;
% \url{http://tug.org/pipermail/lualatex-dev/2011-January/001031.html}.
%
% \end{thebibliography}
%
% \begin{History}
%   \begin{Version}{2011/01/30 v1.0}
%   \item
%     First public version.
%   \end{Version}
%   \begin{Version}{2011/04/16 v1.1}
%   \item
%     Package updates.
%   \end{Version}
%   \begin{Version}{2011/04/17 v1.2}
%   \item
%     White spaces at line begins are removed or reduced
%     in the generated collection packages.
%   \end{Version}
%   \begin{Version}{2011/04/18 v1.3}
%   \item
%     Package updates.
%   \end{Version}
%   \begin{Version}{2011/04/23 v1.4}
%   \item
%     Package updates.
%   \end{Version}
%   \begin{Version}{2011/06/24 v1.5}
%   \item
%     Package updates.
%   \end{Version}
%   \begin{Version}{2011/06/30 v1.6}
%   \item
%     Package updates.
%   \end{Version}
%   \begin{Version}{2011/07/01 v1.7}
%   \item
%     Package updates.
%   \end{Version}
%   \begin{Version}{2011/07/28 v1.8}
%   \item
%     Package updates.
%   \end{Version}
%   \begin{Version}{2011/08/22 v1.9}
%   \item
%     Package updates.
%   \end{Version}
%   \begin{Version}{2011/10/16 v1.10}
%   \item
%     Package updates.
%   \end{Version}
%   \begin{Version}{2011/11/29 v1.11}
%   \item
%     Package updates.
%   \end{Version}
%   \begin{Version}{2012/04/25 v1.12}
%   \item
%     Package updates.
%   \end{Version}
%   \begin{Version}{2012/05/28 v1.13}
%   \item
%     Package updates.
%   \end{Version}
%   \begin{Version}{2016/05/16 v1.14}
%   \item
%     Documentation updates.
%   \end{Version}
% \end{History}
%
% \PrintIndex
%
% \Finale
\endinput
|
% \end{quote}
% Do not forget to quote the argument according to the demands
% of your shell.
%
% \paragraph{Generating the documentation.}
% You can use both the \xfile{.dtx} or the \xfile{.drv} to generate
% the documentation. The process can be configured by the
% configuration file \xfile{ltxdoc.cfg}. For instance, put this
% line into this file, if you want to have A4 as paper format:
% \begin{quote}
%   \verb|\PassOptionsToClass{a4paper}{article}|
% \end{quote}
% An example follows how to generate the
% documentation with pdf\LaTeX:
% \begin{quote}
%\begin{verbatim}
%pdflatex hobsub.dtx
%makeindex -s gind.ist hobsub.idx
%pdflatex hobsub.dtx
%makeindex -s gind.ist hobsub.idx
%pdflatex hobsub.dtx
%\end{verbatim}
% \end{quote}
%
% \section{Catalogue}
%
% The following XML file can be used as source for the
% \href{http://mirror.ctan.org/help/Catalogue/catalogue.html}{\TeX\ Catalogue}.
% The elements \texttt{caption} and \texttt{description} are imported
% from the original XML file from the Catalogue.
% The name of the XML file in the Catalogue is \xfile{hobsub.xml}.
%    \begin{macrocode}
%<*catalogue>
<?xml version='1.0' encoding='us-ascii'?>
<!DOCTYPE entry SYSTEM 'catalogue.dtd'>
<entry datestamp='$Date$' modifier='$Author$' id='hobsub'>
  <name>hobsub</name>
  <caption>Construct package bundles</caption>
  <authorref id='auth:oberdiek'/>
  <copyright owner='Heiko Oberdiek' year='2011,2012'/>
  <license type='lppl1.3'/>
  <version number='1.14'/>
  <description>
    The package offers a means of loading a bundle of (other) packages
    &#x201C;at once&#x201D;, offering a means of reducing the amount of log output
    as the packages are loaded, and an aide-memoire for complicated
    requirements.  In some ways, loading a package via hobsub behaves
    much like loading the package in the normal way; for example,
    reloads are suppressed, as in LaTeX proper, and correct
    <tt>\listfiles</tt> output is produced.
    <p/>
    Examples provided are a &#x2018;generic&#x2019; bundle (comprising the author&#x2019;s
    packages that can be used with Plain TeX) and a &#x2018;hyperref&#x2019; bundle
    (comprising packages useful when generating hypertext output with
    LaTeX).
    <p/>
    The package is part of the <xref refid='oberdiek'>oberdiek</xref> bundle.
  </description>
  <documentation details='Package documentation'
      href='ctan:/macros/latex/contrib/oberdiek/hobsub.pdf'/>
  <ctan file='true' path='/macros/latex/contrib/oberdiek/hobsub.dtx'/>
  <miktex location='oberdiek'/>
  <texlive location='oberdiek'/>
  <install path='/macros/latex/contrib/oberdiek/oberdiek.tds.zip'/>
</entry>
%</catalogue>
%    \end{macrocode}
%
% \begin{thebibliography}{9}
% \raggedright
% \bibitem{lualatex:quiet}
% Will Robertson,
% \textit{[lltx] quietening lualatex console output};
% mailing list \textsf{lualatex-dev at tug.org},
% 2011-01-29;
% \url{http://tug.org/pipermail/lualatex-dev/2011-January/001031.html}.
%
% \end{thebibliography}
%
% \begin{History}
%   \begin{Version}{2011/01/30 v1.0}
%   \item
%     First public version.
%   \end{Version}
%   \begin{Version}{2011/04/16 v1.1}
%   \item
%     Package updates.
%   \end{Version}
%   \begin{Version}{2011/04/17 v1.2}
%   \item
%     White spaces at line begins are removed or reduced
%     in the generated collection packages.
%   \end{Version}
%   \begin{Version}{2011/04/18 v1.3}
%   \item
%     Package updates.
%   \end{Version}
%   \begin{Version}{2011/04/23 v1.4}
%   \item
%     Package updates.
%   \end{Version}
%   \begin{Version}{2011/06/24 v1.5}
%   \item
%     Package updates.
%   \end{Version}
%   \begin{Version}{2011/06/30 v1.6}
%   \item
%     Package updates.
%   \end{Version}
%   \begin{Version}{2011/07/01 v1.7}
%   \item
%     Package updates.
%   \end{Version}
%   \begin{Version}{2011/07/28 v1.8}
%   \item
%     Package updates.
%   \end{Version}
%   \begin{Version}{2011/08/22 v1.9}
%   \item
%     Package updates.
%   \end{Version}
%   \begin{Version}{2011/10/16 v1.10}
%   \item
%     Package updates.
%   \end{Version}
%   \begin{Version}{2011/11/29 v1.11}
%   \item
%     Package updates.
%   \end{Version}
%   \begin{Version}{2012/04/25 v1.12}
%   \item
%     Package updates.
%   \end{Version}
%   \begin{Version}{2012/05/28 v1.13}
%   \item
%     Package updates.
%   \end{Version}
%   \begin{Version}{2016/05/16 v1.14}
%   \item
%     Documentation updates.
%   \end{Version}
% \end{History}
%
% \PrintIndex
%
% \Finale
\endinput
|
% \end{quote}
% Do not forget to quote the argument according to the demands
% of your shell.
%
% \paragraph{Generating the documentation.}
% You can use both the \xfile{.dtx} or the \xfile{.drv} to generate
% the documentation. The process can be configured by the
% configuration file \xfile{ltxdoc.cfg}. For instance, put this
% line into this file, if you want to have A4 as paper format:
% \begin{quote}
%   \verb|\PassOptionsToClass{a4paper}{article}|
% \end{quote}
% An example follows how to generate the
% documentation with pdf\LaTeX:
% \begin{quote}
%\begin{verbatim}
%pdflatex hobsub.dtx
%makeindex -s gind.ist hobsub.idx
%pdflatex hobsub.dtx
%makeindex -s gind.ist hobsub.idx
%pdflatex hobsub.dtx
%\end{verbatim}
% \end{quote}
%
% \section{Catalogue}
%
% The following XML file can be used as source for the
% \href{http://mirror.ctan.org/help/Catalogue/catalogue.html}{\TeX\ Catalogue}.
% The elements \texttt{caption} and \texttt{description} are imported
% from the original XML file from the Catalogue.
% The name of the XML file in the Catalogue is \xfile{hobsub.xml}.
%    \begin{macrocode}
%<*catalogue>
<?xml version='1.0' encoding='us-ascii'?>
<!DOCTYPE entry SYSTEM 'catalogue.dtd'>
<entry datestamp='$Date$' modifier='$Author$' id='hobsub'>
  <name>hobsub</name>
  <caption>Construct package bundles</caption>
  <authorref id='auth:oberdiek'/>
  <copyright owner='Heiko Oberdiek' year='2011,2012'/>
  <license type='lppl1.3'/>
  <version number='1.14'/>
  <description>
    The package offers a means of loading a bundle of (other) packages
    &#x201C;at once&#x201D;, offering a means of reducing the amount of log output
    as the packages are loaded, and an aide-memoire for complicated
    requirements.  In some ways, loading a package via hobsub behaves
    much like loading the package in the normal way; for example,
    reloads are suppressed, as in LaTeX proper, and correct
    <tt>\listfiles</tt> output is produced.
    <p/>
    Examples provided are a &#x2018;generic&#x2019; bundle (comprising the author&#x2019;s
    packages that can be used with Plain TeX) and a &#x2018;hyperref&#x2019; bundle
    (comprising packages useful when generating hypertext output with
    LaTeX).
    <p/>
    The package is part of the <xref refid='oberdiek'>oberdiek</xref> bundle.
  </description>
  <documentation details='Package documentation'
      href='ctan:/macros/latex/contrib/oberdiek/hobsub.pdf'/>
  <ctan file='true' path='/macros/latex/contrib/oberdiek/hobsub.dtx'/>
  <miktex location='oberdiek'/>
  <texlive location='oberdiek'/>
  <install path='/macros/latex/contrib/oberdiek/oberdiek.tds.zip'/>
</entry>
%</catalogue>
%    \end{macrocode}
%
% \begin{thebibliography}{9}
% \raggedright
% \bibitem{lualatex:quiet}
% Will Robertson,
% \textit{[lltx] quietening lualatex console output};
% mailing list \textsf{lualatex-dev at tug.org},
% 2011-01-29;
% \url{http://tug.org/pipermail/lualatex-dev/2011-January/001031.html}.
%
% \end{thebibliography}
%
% \begin{History}
%   \begin{Version}{2011/01/30 v1.0}
%   \item
%     First public version.
%   \end{Version}
%   \begin{Version}{2011/04/16 v1.1}
%   \item
%     Package updates.
%   \end{Version}
%   \begin{Version}{2011/04/17 v1.2}
%   \item
%     White spaces at line begins are removed or reduced
%     in the generated collection packages.
%   \end{Version}
%   \begin{Version}{2011/04/18 v1.3}
%   \item
%     Package updates.
%   \end{Version}
%   \begin{Version}{2011/04/23 v1.4}
%   \item
%     Package updates.
%   \end{Version}
%   \begin{Version}{2011/06/24 v1.5}
%   \item
%     Package updates.
%   \end{Version}
%   \begin{Version}{2011/06/30 v1.6}
%   \item
%     Package updates.
%   \end{Version}
%   \begin{Version}{2011/07/01 v1.7}
%   \item
%     Package updates.
%   \end{Version}
%   \begin{Version}{2011/07/28 v1.8}
%   \item
%     Package updates.
%   \end{Version}
%   \begin{Version}{2011/08/22 v1.9}
%   \item
%     Package updates.
%   \end{Version}
%   \begin{Version}{2011/10/16 v1.10}
%   \item
%     Package updates.
%   \end{Version}
%   \begin{Version}{2011/11/29 v1.11}
%   \item
%     Package updates.
%   \end{Version}
%   \begin{Version}{2012/04/25 v1.12}
%   \item
%     Package updates.
%   \end{Version}
%   \begin{Version}{2012/05/28 v1.13}
%   \item
%     Package updates.
%   \end{Version}
%   \begin{Version}{2016/05/16 v1.14}
%   \item
%     Documentation updates.
%   \end{Version}
% \end{History}
%
% \PrintIndex
%
% \Finale
\endinput
|
% \end{quote}
% Do not forget to quote the argument according to the demands
% of your shell.
%
% \paragraph{Generating the documentation.}
% You can use both the \xfile{.dtx} or the \xfile{.drv} to generate
% the documentation. The process can be configured by the
% configuration file \xfile{ltxdoc.cfg}. For instance, put this
% line into this file, if you want to have A4 as paper format:
% \begin{quote}
%   \verb|\PassOptionsToClass{a4paper}{article}|
% \end{quote}
% An example follows how to generate the
% documentation with pdf\LaTeX:
% \begin{quote}
%\begin{verbatim}
%pdflatex hobsub.dtx
%makeindex -s gind.ist hobsub.idx
%pdflatex hobsub.dtx
%makeindex -s gind.ist hobsub.idx
%pdflatex hobsub.dtx
%\end{verbatim}
% \end{quote}
%
% \section{Catalogue}
%
% The following XML file can be used as source for the
% \href{http://mirror.ctan.org/help/Catalogue/catalogue.html}{\TeX\ Catalogue}.
% The elements \texttt{caption} and \texttt{description} are imported
% from the original XML file from the Catalogue.
% The name of the XML file in the Catalogue is \xfile{hobsub.xml}.
%    \begin{macrocode}
%<*catalogue>
<?xml version='1.0' encoding='us-ascii'?>
<!DOCTYPE entry SYSTEM 'catalogue.dtd'>
<entry datestamp='$Date$' modifier='$Author$' id='hobsub'>
  <name>hobsub</name>
  <caption>Construct package bundles</caption>
  <authorref id='auth:oberdiek'/>
  <copyright owner='Heiko Oberdiek' year='2011,2012'/>
  <license type='lppl1.3'/>
  <version number='1.14'/>
  <description>
    The package offers a means of loading a bundle of (other) packages
    &#x201C;at once&#x201D;, offering a means of reducing the amount of log output
    as the packages are loaded, and an aide-memoire for complicated
    requirements.  In some ways, loading a package via hobsub behaves
    much like loading the package in the normal way; for example,
    reloads are suppressed, as in LaTeX proper, and correct
    <tt>\listfiles</tt> output is produced.
    <p/>
    Examples provided are a &#x2018;generic&#x2019; bundle (comprising the author&#x2019;s
    packages that can be used with Plain TeX) and a &#x2018;hyperref&#x2019; bundle
    (comprising packages useful when generating hypertext output with
    LaTeX).
    <p/>
    The package is part of the <xref refid='oberdiek'>oberdiek</xref> bundle.
  </description>
  <documentation details='Package documentation'
      href='ctan:/macros/latex/contrib/oberdiek/hobsub.pdf'/>
  <ctan file='true' path='/macros/latex/contrib/oberdiek/hobsub.dtx'/>
  <miktex location='oberdiek'/>
  <texlive location='oberdiek'/>
  <install path='/macros/latex/contrib/oberdiek/oberdiek.tds.zip'/>
</entry>
%</catalogue>
%    \end{macrocode}
%
% \begin{thebibliography}{9}
% \raggedright
% \bibitem{lualatex:quiet}
% Will Robertson,
% \textit{[lltx] quietening lualatex console output};
% mailing list \textsf{lualatex-dev at tug.org},
% 2011-01-29;
% \url{http://tug.org/pipermail/lualatex-dev/2011-January/001031.html}.
%
% \end{thebibliography}
%
% \begin{History}
%   \begin{Version}{2011/01/30 v1.0}
%   \item
%     First public version.
%   \end{Version}
%   \begin{Version}{2011/04/16 v1.1}
%   \item
%     Package updates.
%   \end{Version}
%   \begin{Version}{2011/04/17 v1.2}
%   \item
%     White spaces at line begins are removed or reduced
%     in the generated collection packages.
%   \end{Version}
%   \begin{Version}{2011/04/18 v1.3}
%   \item
%     Package updates.
%   \end{Version}
%   \begin{Version}{2011/04/23 v1.4}
%   \item
%     Package updates.
%   \end{Version}
%   \begin{Version}{2011/06/24 v1.5}
%   \item
%     Package updates.
%   \end{Version}
%   \begin{Version}{2011/06/30 v1.6}
%   \item
%     Package updates.
%   \end{Version}
%   \begin{Version}{2011/07/01 v1.7}
%   \item
%     Package updates.
%   \end{Version}
%   \begin{Version}{2011/07/28 v1.8}
%   \item
%     Package updates.
%   \end{Version}
%   \begin{Version}{2011/08/22 v1.9}
%   \item
%     Package updates.
%   \end{Version}
%   \begin{Version}{2011/10/16 v1.10}
%   \item
%     Package updates.
%   \end{Version}
%   \begin{Version}{2011/11/29 v1.11}
%   \item
%     Package updates.
%   \end{Version}
%   \begin{Version}{2012/04/25 v1.12}
%   \item
%     Package updates.
%   \end{Version}
%   \begin{Version}{2012/05/28 v1.13}
%   \item
%     Package updates.
%   \end{Version}
%   \begin{Version}{2016/05/16 v1.14}
%   \item
%     Documentation updates.
%   \end{Version}
% \end{History}
%
% \PrintIndex
%
% \Finale
\endinput
