% \iffalse meta-comment
%  GNU GPL license 
%
% \fi
%
% \iffalse
%<*driver>
\ProvidesFile{hrlatex.dtx}
%</driver>

%<*driver>
\documentclass{ltxdoc}
\RequirePackage{hrlatex}
\RequirePackage{enumerate}


\EnableCrossrefs         
\CodelineIndex
\RecordChanges
\begin{document}
%    \tableofcontents
  \DocInput{hrlatex.dtx}
\end{document}
%</driver>
%<*package>
\NeedsTeXFormat{LaTeX2e}[1999/12/01]
\ProvidesPackage{hrlatex}[2010/04/05 v0.23 LaTeX Macros for HRLaTeX project]
% 
% 
	\usepackage{xkeyval}
% 
	\DeclareOption{slovene}{\PassOptionsToPackage{slovene}{babel}}
% 
% ^^A	\newif\ifenc@valundef
% ^^A	\enc@valundeftrue
	\DeclareOptionX{enc}[utf8]{%
% ^^A		\enc@valundeffalse
		\PassOptionsToPackage{#1}{inputenc}
	}

	\DeclareOption{last}{

	}

	\DeclareOption*{}
	\DeclareOptionX*{}

 
% ^^A	\ifenc@valundef
	\PassOptionsToPackage{utf8}{inputenc} %% Default
% ^^A	\fi
	\PassOptionsToPackage{croatian}{babel}

	\ProcessOptions* 
	\ProcessOptionsX

	\RequirePackage{inputenc}
	\RequirePackage[T1]{fontenc}
% 	\RequirePackage{tracefnt}
	\RequirePackage[croatian]{babel}
	\RequirePackage{amsopn}

%</package>
% \fi
% 
%
% \CheckSum{21}
%
% \CharacterTable
%  {Upper-case    \A\B\C\D\E\F\G\H\I\J\K\L\M\N\O\P\Q\R\S\T\U\V\W\X\Y\Z
%   Lower-case    \a\b\c\d\e\f\g\h\i\j\k\l\m\n\o\p\q\r\s\t\u\v\w\x\y\z
%   Digits        \0\1\2\3\4\5\6\7\8\9
%   Exclamation   \!     Double quote  \"     Hash (number) \#
%   Dollar        \$     Percent       \%     Ampersand     \&
%   Acute accent  \'     Left paren    \(     Right paren   \)
%   Asterisk      \*     Plus          \+     Comma         \,
%   Minus         \-     Point         \.     Solidus       \/
%   Colon         \:     Semicolon     \;     Less than     \<
%   Equals        \=     Greater than  \>     Question mark \?
%   Commercial at \@     Left bracket  \[     Backslash     \\
%   Right bracket \]     Circumflex    \^     Underscore    \_
%   Grave accent  \`     Left brace    \{     Vertical bar  \|
%   Right brace   \}     Tilde         \~}
%
%
% \changes{v0.1}{2006/02/11}{Prva verzija}
% \changes{v0.15}{2006/05/20}{.dtx verzija}
% \changes{v0.2}{2006/06/20}{CTAN upload}
% \changes{v0.21}{2006/06/25}{utf8 kao default input encoding}
% \changes{v0.22}{2006/09/20}{MikTeX fix}
% \changes{v0.23}{2010/04/25}{Ciscenje}
%
% \GetFileInfo{hrlatex.sty}
%
% \DoNotIndex{\newcommand,\newenvironment}
% 
%
% \title{{\Large \textsf{HRLaTeX}} paket\thanks{Ovaj dokument
%   odnosi se na \textsf{hrlatex}~\fileversion, \filedate.}}
% \author{Marcel Mareti\'c \\ \texttt{marcelix at gmail dot com} \\ \texttt{http://hrlatex.wordpress.com/}}
%
% \maketitle
%
% \section{Uvod}
%
% \textsf{HRLaTeX} paket (hrlatex.sty) je \LaTeX2e prilagodba prosje\v{c}nom hrvatskom \LaTeX{} korisniku.
% \textsf{hrlatex} uklju\v{c}uje \textsf{babel}, \textsf{inputenc}, \textsf{fontenc} pakete s uobi\v{c}ajenim opcijama
% i definira uobi\v{c}ajene matemati\v{c}ke operatore, npr.: $\tg$, $\ch$ i dr.
%
% \subsubsection*{Za\v{s}to?}
% % %   ^^A Iz iskustva znam da ve\'cina hrvatskih \LaTeX{} hakera ima vlastite pomo\'cne datoteke u kojima sa sobom vu\v{c}e ono \v{s}to bi svaki dobar \TeX{} dokument trebao imati. Ili jednostavno svaki novi dokument po\v{c}inje tako da se pi\v{s}e preko zadnjeg.
% ^^A Za neke znam. :-)
% 
% \begin{itemize}
% \item[--] \v{C}ini se da ve\'cina hrvatskih \LaTeX{} hakera ima vlastite dokumente u kojima dr\v{z}i ono \v{s}to treba svakom dobrom \LaTeX{} dokumentu. Novi dokumenti se u pravilu rade tako da se pi\v{s}e "preko" zadnjeg.
% \item[--] hrlatex.sty je podloga za nekoliko predlo\v{z}aka (\textit{templates}): diplomski rad, ispit, ud\v{z}benik, itd. 
% \end{itemize}
% \section{Uporaba}
%
% \DescribeMacro{\usepackage\{hrlatex\}} \DescribeMacro{\RequirePackage\{hrlatex\}}
% U preambuli treba uklju\v{c}iti \textsf{hrlatex} paket.
% 
%
%
% \StopEventually{\PrintChanges}
% \section{Kodne stranice}
% \DescribeMacro{enc=utf8} \DescribeMacro{enc=cp1250} \DescribeMacro{enc=latin2}
% \textsf{hrlatex} u\v{c}itava paket \textsf{inputenc} kojim omogu\'cuje direktan unos hrvatskih znakova.
% Ukoliko koristite kodnu stranicu koja nije ASCII ili UTF-8 u opcijama prilikom u\v{c}itavanja paketa 
% morate to\v{c}no navesti koju kodnu stranicu koristite. Npr., zasad je na hrvatskim windows instalacijama (default) kodna stranica u kojoj se 
% unosi hrvatski tekst windows-CP1250. Tako npr.~ako \v{z}elite u WinEdt-u unositi hrvatske znakove paket hrlatex u\v{c}itajte s
% $$\verb+\usepackage[enc=cp1250]{hrlatex}+ $$
% Analogno se navode ostale kodne stranice, uz napomenu da opciju \verb+utf8+ \textit{default}  ne treba se specijalno navoditi.
% Za o\v{c}ekivati je da \'ce unicode UTF-8 kodiranje istisnuti starija 8-bitna kodiranja, pa ovu opciju ne\'ce trebati navoditi.
% 
% 
% ^^A ===============================================
% 
% \section{Napomene}
% 
%    Kodna stranica je (sad ipak) po defaultu stavljena na |utf8| -- u nadi da \'ce to uskoro postati standard na hrvatskim MS Windows instalacijama. Na linuxu to nije problem. Npr.~kile se lako mo\v{z}e namjestiti tako da mu je cp1250 \textit{default encoding} za editor. Po\v{s}to je cp1250 kodna stranica pro\v{s}irenje ascii-ja sa starijim \LaTeX datotekama nema problema.
% ^^A |\th|
% 
% $$\th 1, \tg \frac{\pi}2, \arctg{1}, \sh 1, \ch 0$$
%
% \section{Implementacija}
%
%    \begin{macrocode}
\DeclareMathOperator{\tg}{tg}
\DeclareMathOperator{\arctg}{arc\,tg}
\DeclareMathOperator{\ctg}{ctg}
\DeclareMathOperator{\arcctg}{arc\,ctg}
\DeclareMathOperator{\sh}{sh}
\DeclareMathOperator{\ch}{ch}
\DeclareMathOperator{\cth}{cth}
\DeclareMathOperator{\tgh}{th}
\let\th\tgh
% 
%    \end{macrocode}
% 
% 
% \section{Promjene}
% \begin{enumerate}[\qquad --]
% \item[v0.10] Inicijalna verzija 
% \item[v0.15] .dtx verzija
% \item[v0.20] prva verzija na na CTAN-u
% \item[v0.21] Popravio da je \textit{default} ulazna kodna stranica UTF-8 pomo\'cu \textsf{xkeyval} paketa (hvala Mojci Miklavec)
% \item[v0.22] fix za MikTeX
% \item[v0.23] ...
% \end{enumerate}
% 
% \section{Primjedbe}
%    Molim Vas da mi javite svoje prijedbe i komentare.
% \Finale
\endinput
