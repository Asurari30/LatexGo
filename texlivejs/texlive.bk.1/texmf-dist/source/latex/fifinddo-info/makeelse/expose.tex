\def\exposedate{2011/09/04}
\ProvidesFile{expose.tex}[\exposedate]
\documentclass[a4paper,11pt]{article}
\usepackage{german}
\newcommand*{\DQ}[1]{"`#1"'}
\usepackage{parskip}
\parskip=.84\parskip
\usepackage[atari]{umlaute}
version https://git-lfs.github.com/spec/v1
oid sha256:ded97341479a4e2387b08b2fca0a903a98527805994be66f525785a0cdabece0
size 1168

\providecommand{\pkg}{\pkgnamefmt}
\providecommand{\code}{\texttt}
\title{Paketdokumentation und Webseitenpflege\\ 
       mit (\pkg{niceverb.sty} und) \pkg{fifinddo.sty}}
\author{\acro{DANTE}-Herbsttagung 2011\\Uwe L�ck}
\date{Entwurf \exposedate}
\begin{document}
\maketitle
\thispagestyle{empty}

Das 'nicetext'-B�ndel\urlpkgfoot{nicetext} 
versucht sich an
\wikideref{Syntaktischer Zucker}{\strong{%
           \DQ{syntaktischem Zucker}}:} 
Schriftsatz in \TeX-Qualit�t 
bei m�glichst wenig \TeX-artiger Auszeichnung. 
% Grunds�tzlich sehe ich daf�r zwei Methoden: 
Das kann durch 
(MS)~Zeichenkettenersetzung %%%, 
oder (auch) durch
(MA)~aktive Zeichen 
(die einige Fallunterscheidungen ausf�hren) %%%.
% Sie schlie�en sich keineswegs gegenseitig aus.
% Die praktische Bedeutung des B�ndels liegt f�r mich zur Zeit 
% ganz in der Paketdokumentation. 
erreicht werden.

Die \DQ{herrschende Lehre} f�r die \strong{Dokumentation} von \LaTeX-Paketen 
ist die \DQ{\file{.dtx}}-Methode, die z.\,B. eine spezielle 
Markierung des Paketcodes erfordert. 
'makedoc.sty' aus dem 'nicetext'-B�ndel dagegen 
setzt die gute alte Idee um, dass man Kommentarzeilen schon 
% dadurch dadurch von Codezeilen unterscheiden kann, 
% dass erstere mit einem Kommentarzeichen (\lq\code{\%}\rq) beginnen! 
am Kommentarzeichen (\lq`%'\rq) erkennt! 

Mit derselben Absicht verwendete Stephan B�ttcher die 
\strong{Skriptsprache} \meta{\Wikideref{awk}} 
% f�r die Dokumentation von 
zur Kommentierung von 'lineno.sty'.\urlpkgfoot{lineno}
'makedoc.sty' bietet % dem gegen�ber 
stattdessen eine Skriptsprache, % deren Interpretationsengine 
die wie bei 'docstrip.tex'\urlpkgfoot{docstrip} 
durch die \TeX-Engine selbst verarbeitet wird, 
zusammen mit besonderen Makros. 
'makedoc' wandelt die Paketdatei in eine Dokumentationsdatei um, 
indem Kommentarzeichen entfernt und Codezeilen in Listing-Umgebungen 
eingebettet werden. Bei dieser Gelegenheit kann auch gem�� Methode (MS) 
\TeX-Code f�r typografische Feinheiten eingef�gt und die 
\Wikideref{MediaWiki}-\wikideref{Hilfe:Inhaltsverzeichnis}{Gliederung} 
umgesetzt werden.

'niceverb.sty' (ebenfalls in 'nicetext') % bietet weitere Feinheiten 
erm�glicht weiter
nach der Methode (MA) 
% f�r 
eine geradezu \acro{WYS\-I\-WYG}-artige \strong{Syntaxbeschreibung} 
f�r Paketmakros. \linebreak
% Sogar Pakete anderer Autoren (z.\,B. Donald Arseneau) 
% k�nnen in \TeX-Qualit�t aus der reinen ASCII-Dokumentation 
% in der Paketdatei gesetzt werden, ohne �nderungen an dieser Datei 
% vorzunehmen. 
% Bei eigenen Paketen erfordert die Kommentierung 
% wenig mehr Aufwand als Kommentarzeichen, und der Kommentarcode 
% ist so leserlich wie �berhaupt nur m�glich.
Makros k�n\-nen \DQ{im Handumdrehen} kommentiert werden. 
% (oder man schreibt zuerst den Kommentar als Plan)~\dots
Vorhandene \linebreak
\acro{ASCII}-Dokumentation kann \DQ{automatisch} 
in \TeX-Qualit�t gesetzt werden.

% \enlargethispage{1\baselineskip}

'makedoc.sty' erweitert die \TeX-Skriptsprache aus 'fifinddo.sty'
zum Verarbeiten von Text-Dateien zu Text-Dateien. 
'blog.sty' aus dem \ctanpkgref{morehype}-B�ndel 
setzt ebenfalls auf 'fifinddo' auf. 
Im Gegensatz zu 'makedoc' werden dabei die Makros der Quelldateien 
\emph{expandiert} -- in \strong{\acro{HTML}}-Tags und -Zeichen.

Der Vortrag vergleicht diese Ans�tze mit denen 
\httpref{www.webdesign-bu.de/uwe\string_lueck/heyctan.htm\#mine-related}
        {\strong{anderer}} Autoren.

\end{document}
