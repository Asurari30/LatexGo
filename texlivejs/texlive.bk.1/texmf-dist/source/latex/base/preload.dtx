% \iffalse meta-comment
%
% Copyright 1993-2016
% The LaTeX3 Project and any individual authors listed elsewhere
% in this file.
%
% This file is part of the LaTeX base system.
% -------------------------------------------
%
% It may be distributed and/or modified under the
% conditions of the LaTeX Project Public License, either version 1.3c
% of this license or (at your option) any later version.
% The latest version of this license is in
%    https://www.latex-project.org/lppl.txt
% and version 1.3c or later is part of all distributions of LaTeX
% version 2005/12/01 or later.
%
% This file has the LPPL maintenance status "maintained".
%
% The list of all files belonging to the LaTeX base distribution is
% given in the file `manifest.txt'. See also `legal.txt' for additional
% information.
%
% The list of derived (unpacked) files belonging to the distribution
% and covered by LPPL is defined by the unpacking scripts (with
% extension .ins) which are part of the distribution.
%
% \fi
%
% \iffalse
%%% From File: preload.dtx
%<*dtx>
           \ProvidesFile{preload.dtx}
%</dtx>
%<*preload>
%<*!tex>
%<+cm>  \ProvidesFile{cmpreloa.%
%<+dc>  \ProvidesFile{dcpreloa.%
%<+xpt>                         xpt}
%<+xipt>                        xip}
%<+xiipt>                       xii}
%<+min> \ProvidesFile{preload.min}
%<+ori> \ProvidesFile{preload.ori}
%</!tex>
%<+tex> \ProvidesFile{preload.ltx}
% \fi
%       \ProvidesFile{preload.dtx}
         [2014/09/29 v2.1g LaTeX Kernel (Font Preloading)]
%
%
%
%\iffalse       This is a META comment
%
% File `preload.dtx'.
% Copyright (C) 1989-1994 Frank Mittelbach and Rainer Sch\"opf,
% all rights reserved.
%
% \fi
%
% \GetFileInfo{preload.dtx}
% \title{The \texttt{preload.dtx} file\thanks {This file has version
%    number \fileversion, dated \filedate}\\ for use with \LaTeXe}
% \date{\filedate}
% \author{Frank Mittelbach \and Rainer Sch\"opf}
%
% \changes{v2.0b}{1993/03/08}{Added 12pt preloads}
% \changes{v2.1e}{1994/11/07}{(DPC) Updated to use \cs{ProvidesFile}}
% \changes{v2.1g}{1998/08/17}{(RmS) Minor documentation fixes.}
%
% \def\dst{\expandafter{\normalfont\scshape docstrip}}
%
% \setcounter{StandardModuleDepth}{1}
%
% \MaintainedByLaTeXTeam{latex}
% \maketitle
%
% \section{Overview}
%
%   This file contains an number of possible settings for preloading
%   fonts during installation of NFSS2 (which is used by \LaTeXe).  It
%   will be used to generate the following files:
%   \begin{center}
%   \begin{tabular}{ll}
%   preload.min   &  minimal subset of fonts necessary to run NFSS2 \\
%   preload.ori   &  preload of CM fonts similar to the old
%                        \texttt{lfonts.tex}                       \\
%   preload.ltx    &  The standard selection of preloads \\
%   cmpreloa.xpt   &  preload of CM fonts for 10pt document size\\
%   cmpreloa.xip   &  preload of CM fonts for 11pt document size\\
%   cmpreloa.xii   &  preload of CM fonts for 12pt document size\\
%   dcpreloa.xpt   &  preload of DC fonts for 10pt size \\
%   dcpreloa.xip   &  preload of DC fonts for 11pt size \\
%   dcpreloa.xii   &  preload of DC fonts for 12pt size \\
%   \end{tabular}
%   \end{center}
%
%    These files are for installations that make use of Computer
%    Modern fonts either old encoding (OT1) or Cork encoding (T1). The
%    Computer Modern fonts with Cork encoding are known as DC-fonts.
%
%    Most important is \texttt{preload.ltx} which is used during
%    format generation. You are \emph{not} allowed to change this file.
%
% \section{Customization}
%
%    You can customize the preloaded fonts in your \LaTeXe{} system by
%    installing a file with the name \texttt{preload.cfg}. If this
%    file exists it will be used in place of the system file
%    \texttt{preload.ltx}.  You can, for example, copy one of the
%    files mentioned above (that can be generated from this source) to
%    \texttt{preload.cfg}.
%
%    Or you can define completely other preloads. In that case start
%    from \texttt{preload.min} since that contains the fonts that have
%    to be preloaded by *all* \LaTeXe{} systems.
%
%    Avoid using \texttt{preload.ori}, it will load so many fonts that
%    on most installations it is nearly impossible to load other font
%    families afterwards. This file is only generated to show what
%    fonts have been preloaded by \LaTeX~2.09.
%
%    If you normally use other fonts than Computer Modern
%    \texttt{preload.min} might be best.
%
%    \begin{quote} \textbf{Warning:} If you preload fonts with
%    encodings other than the normally supported encodings you have to
%    declare that encoding in a \texttt{fontdef.cfg} configuration
%    file (see the documentation in the file \texttt{fontdef.dtx}).
%    Adding an extra encoding to the format might produce non-portable
%    documents, thus this should be avoided if possible.
%    \end{quote}
%
%
% \StopEventually{}
%
% \section{Module switches for the \dst{} program}
%
%  The \dst{} will generate the above file from this source using the
%  following module directives:
% \begin{center}
% \begin{tabular}{ll}
%   driver & produce a documentation driver file \\
%   preload& produce a preload\ldots file \\[2pt]
%   cm     & for OT1 encoded Computer Modern \\
%   dc     & for T1 encoded Computer Modern \\[2pt]
%   min    & produce minimal subset \\
%   xpt    & produce 10pt preloads \\
%   xipt   & produce 11pt preloads \\
%   xiipt  & produce 12pt preloads \\
%   ori    & produce preloads similar to old \texttt{lfonts.tex}\\
%   tex    & produce preload.ltx\\
% \end{tabular}
% \end{center}
% A typical \dst{} command file would then have entries like:
% \begin{verbatim}
%\generateFile{preload.min}{t}{\from{preload.dtx}{preload,min}}
%\end{verbatim}
% for generating preload files.
%
% \section{A driver for this document}
%
%    The next bit of code contains the documentation driver file for
%    \TeX{}, i.e., the file that will produce the documentation you
%    are currently reading. It will be extracted from this file by the
%    \dst{} program.
%    \begin{macrocode}
%<*driver>
\documentclass{ltxdoc}
%\OnlyDescription  % comment out for implementation details
\begin{document}
   \DocInput{preload.dtx}
\end{document}
%</driver>
%    \end{macrocode}
%
%
% \section{The code}
%
%    We begin by loading the math extension font (cmex10)
%    and the \LaTeX{} line and circle fonts.
%    It is necessary to do this explicitly since these are
%    used by \texttt{lplain.tex} and \texttt{latex.tex}.
%    Since the internal font name contains |/| characters
%    and digits we construct the name via |\csname|.
%    These are the only fonts (!) that must be loaded in this file.
%
%    All |\DeclarePreloadSizes| can be removed or others can be added,
%    they only influence the processing speed.
% \changes{v2.0c}{1993/08/13}{Added \cs{relax} at end of font names.}
%    \begin{macrocode}
\expandafter\font\csname OMX/cmex/m/n/10\endcsname=cmex10\relax
\font\tenln  =line10   \font\tenlnw  =linew10\relax
\font\tencirc=lcircle10 \font\tencircw=lcirclew10\relax
%    \end{macrocode}
%    The above fonts should not be touched but anything below this
%    point here in the preload suggestions can be modified without any
%    problems.
%    \begin{macrocode}
%<-tex>%*******************************************
%<-tex>% Start any modification below this point **
%<-tex>%*******************************************
%<-tex>
%%
%% Computer Modern Roman:
%%-----------------------
%<*ori>
\DeclarePreloadSizes{OT1}{cmr}{m}{n}
        {5,6,7,8,9,10,10.95,12,14.4,17.28,20.74,24.88}
\DeclarePreloadSizes{OT1}{cmr}{bx}{n}{9,10,10.95,12,14.4,17.28}
\DeclarePreloadSizes{OT1}{cmr}{m}{sl}{10,10.95,12}
\DeclarePreloadSizes{OT1}{cmr}{m}{it}{7,8,9,10,10.95,12}
%</ori>
%<+xpt&cm> \DeclarePreloadSizes{OT1}{cmr}{m}{n}{5,7,10}
%<+xpt&dc> \DeclarePreloadSizes{T1}{cmr}{m}{n}{5,7,10}
%<+xipt&cm> \DeclarePreloadSizes{OT1}{cmr}{m}{n}{6,8,10.95}
%<+xipt&dc> \DeclarePreloadSizes{T1}{cmr}{m}{n}{6,8,10.95}
%<+xiipt&cm> \DeclarePreloadSizes{OT1}{cmr}{m}{n}{6,8,12}
%<+xiipt&dc> \DeclarePreloadSizes{T1}{cmr}{m}{n}{6,8,12}
%%
%% Computer Modern Sans:
%%----------------------
%<+ori> \DeclarePreloadSizes{OT1}{cmss}{m}{n}{10,10.95,12}
%%
%% Computer Modern Typewriter:
%%----------------------------
%<+ori> \DeclarePreloadSizes{OT1}{cmtt}{m}{n}{9,10,10.95,12}
%%
%% Computer Modern Math:
%%----------------------
%<*ori>
\DeclarePreloadSizes{OML}{cmm}{m}{it}
         {5,6,7,8,9,10,10.95,12,14.4,17.28,20.74}
\DeclarePreloadSizes{OMS}{cmsy}{m}{n}
         {5,6,7,8,9,10,10.95,12,14.4,17.28,20.74}
%</ori>
%    \end{macrocode}
%
%    The math fonts are the same for both DC and CM fonts. So far
%    there isn't an agreed on standard.
% \changes{v2.4e}{1995/12/04}
%      {Ulrik Vieth. added 12pt OMS and OML preloads  /1989}
%    \begin{macrocode}
%<*xpt>
\DeclarePreloadSizes{OML}{cmm}{m}{it}{5,7,10}
\DeclarePreloadSizes{OMS}{cmsy}{m}{n}{5,7,10}
%</xpt>
%<*xipt>
\DeclarePreloadSizes{OML}{cmm}{m}{it}{6,8,10.95}
\DeclarePreloadSizes{OMS}{cmsy}{m}{n}{6,8,10.95}
%</xipt>
%<*xiipt>
\DeclarePreloadSizes{OML}{cmm}{m}{it}{6,8,12}
\DeclarePreloadSizes{OMS}{cmsy}{m}{n}{6,8,12}
%</xiipt>
%%
%% LaTeX symbol fonts:
%%--------------------
%<*ori>
\DeclarePreloadSizes{U}{lasy}{m}{n}
         {5,6,7,8,9,10,10.95,12,14.4,17.28,20.74}
%</ori>
%</preload>
%    \end{macrocode}
%
%
%
% \Finale
%
\endinput
