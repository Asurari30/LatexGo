\newif\ifenigmaisrunningplain
\ifcsname ver@enigma.sty\endcsname\else
  \enigmaisrunningplaintrue
  \ifx\BeginCatcodeRegime\undefined\else\expandafter\endinput\fi

\ifx
  \ProvidesPackage\undefined\begingroup\def\ProvidesPackage
  #1#2[#3]{\endgroup\immediate\write-1{Package: #1 #3}}
\fi
\ProvidesPackage{luatexbase}
[2015/10/04 v1.3
  luatexbase interface to LuaTeX
]
\edef\emuatcatcode{\the\catcode`\@}
\catcode`\@=11
\ifx\@setrangecatcode\@undefined
  \ifx\RequirePackage\@undefined
    version https://git-lfs.github.com/spec/v1
oid sha256:2993431c0138c4c9b715e121ffea8bab0b8edd9eea7344c7d4a1665f0901643f
size 1510
%
  \else
    \RequirePackage{ctablestack}
  \fi
\fi
\def\RequireLuaModule#1{\directlua{require("#1")}\@gobbleoptarg}
\ifdefined\@ifnextchar
\def\@gobbleoptarg{\@ifnextchar[\@gobble@optarg{}}%
\else
\long\def\@gobbleoptarg#1{\ifx[#1\expandafter\@gobble@optarg\fi#1}%
\fi
\def\@gobble@optarg[#1]{}
\let\CatcodeTableIniTeX\catcodetable@initex
\let\CatcodeTableString\catcodetable@string
\let\CatcodeTableLaTeX\catcodetable@latex
\let\CatcodeTableLaTeXAtLetter\catcodetable@atletter
\newcatcodetable\CatcodeTableOther
\@setcatcodetable\CatcodeTableOther{%
  \catcodetable\CatcodeTableString
  \catcode32 12 }
\newcatcodetable\CatcodeTableExpl
\@setcatcodetable\CatcodeTableExpl{%
  \catcodetable\CatcodeTableLaTeX
  \catcode126 10 % tilde is a space char
  \catcode32  9  % space is ignored
  \catcode9   9  % tab also ignored
  \catcode95  11 % underscore letter
  \catcode58  11 % colon letter
}
\def\BeginCatcodeRegime#1{%
  \@pushcatcodetable
  \catcodetable#1\relax}
\def\EndCatcodeRegime{%
  \@popcatcodetable}
\let\PushCatcodeTableNumStack\@pushcatcodetable
\let\PopCatcodeTableNumStack\@popcatcodetable
\let\SetCatcodeRange\@setrangecatcode
\let\setcatcodetable\@setcatcodetable
\directlua{
function luatexbase.reset_callback(name,make_false)
  for _,v in pairs(luatexbase.callback_descriptions(name))
  do
    luatexbase.remove_from_callback(name,v)
  end
  if make_false == true then
    luatexbase.disable_callback(name)
  end
end
luatexbase.base_add_to_callback=luatexbase.add_to_callback
function luatexbase.add_to_callback(name,fun,description,priority)
  local priority= priority
  if priority==nil then
   priority=\string#luatexbase.callback_descriptions(name)+1
  end
  if(luatexbase.callbacktypes[name] == 3 and
     priority == 1 and
     \string#luatexbase.callback_descriptions(name)==1) then
    luatexbase.module_warning("luatexbase",
                              "resetting exclusive callback: " .. name)
    luatexbase.reset_callback(name)
  end
  local saved_callback={},ff,dd
  for k,v in pairs(luatexbase.callback_descriptions(name)) do
    if k >= priority then
      ff,dd= luatexbase.remove_from_callback(name, v)
      saved_callback[k]={ff,dd}
    end
  end
  luatexbase.base_add_to_callback(name,fun,description)
  for k,v in pairs(saved_callback) do
    luatexbase.base_add_to_callback(name,v[1],v[2])
  end
  return
end
luatexbase.catcodetables=setmetatable(
 {['latex-package'] = \number\CatcodeTableLaTeXAtLetter,
  ini    = \number\CatcodeTableIniTeX,
  string = \number\CatcodeTableString,
  other  = \number\CatcodeTableOther,
  latex  = \number\CatcodeTableLaTeX,
  expl   = \number\CatcodeTableExpl,
  expl3  = \number\CatcodeTableExpl},
 { __index = function(t,key)
    return luatexbase.registernumber(key) or nil
  end}
)}
\ifnum\luatexversion<80 %
\def\newcatcodetable#1{%
  \e@alloc\catcodetable\chardef
    \e@alloc@ccodetable@count\m@ne{"8000}#1%
  \initcatcodetable\allocationnumber
  {\escapechar=\m@ne
  \directlua{luatexbase.catcodetables['\string#1']=%
    \the\allocationnumber}}%
}
\fi
\directlua{
function luatexbase.priority_in_callback (name,description)
  for i,v in ipairs(luatexbase.callback_descriptions(name))
  do
    if v == description then
      return i
    end
  end
  return false
end
luatexbase.is_active_callback = luatexbase.in_callback
luatexbase.base_provides_module=luatexbase.provides_module
function luatexbase.errwarinf(name)
    return
    function(s,...) return luatexbase.module_error(name, s:format(...)) end,
    function(s,...) return luatexbase.module_warning(name, s:format(...)) end,
    function(s,...) return luatexbase.module_info(name, s:format(...)) end,
    function(s,...) return luatexbase.module_info(name, s:format(...)) end
end
function luatexbase.provides_module(info)
  luatexbase.base_provides_module(info)
  return luatexbase.errwarinf(info.name)
end
}
\ifnum\luatexversion<80 %
\def\newattribute#1{%
  \e@alloc\attribute\attributedef
    \e@alloc@attribute@count\m@ne\e@alloc@top#1%
  {\escapechar=\m@ne
  \directlua{luatexbase.attributes['\string#1']=%
    \the\allocationnumber}}%
}
\fi
\ifx\@percentchar\@undefined
  {\catcode`\%=12 \gdef\@percentchar{%}}
\fi
\directlua{%
local copynode          = node.copy
local newnode           = node.new
local nodesubtype       = node.subtype
local nodetype          = node.id
local stringformat      = string.format
local tableunpack       = unpack or table.unpack
local texiowrite_nl     = texio.write_nl
local texiowrite        = texio.write
local whatsit_t         = nodetype"whatsit"
local user_defined_t    = nodesubtype"user_defined"
local unassociated      = "__unassociated"
local user_whatsits       = {  __unassociated = { } }
local whatsit_ids         = { }
local anonymous_whatsits  = 0
local anonymous_prefix    = "anon"
local new_user_whatsit_id = function (name, package)
    if name then
        if not package then
            package = unassociated
        end
    else % anonymous
        anonymous_whatsits = anonymous_whatsits + 1
        warning("defining anonymous user whatsit no. \@percentchar
                  d", anonymous_whatsits)
        package = unassociated
        name    = anonymous_prefix .. tostring(anonymous_whatsits)
    end

    local whatsitdata = user_whatsits[package]
    if not whatsitdata then
        whatsitdata             = { }
        user_whatsits[package]  = whatsitdata
    end

    local id = whatsitdata[name]
    if id then %- warning
        warning("replacing whatsit \@percentchar s:\@percentchar
                  s (\@percentchar d)", package, name, id)
    else %- new id
        id=luatexbase.new_whatsit(name)
        whatsitdata[name]   = id
        whatsit_ids[id]     = { name, package }
    end
    return id
end
luatexbase.new_user_whatsit_id = new_user_whatsit_id
local new_user_whatsit = function (req, package)
    local id, whatsit
    if type(req) == "string" then
        id              = new_user_whatsit_id(req, package)
        whatsit         = newnode(whatsit_t, user_defined_t)
        whatsit.user_id = id
    elseif req.id == whatsit_t and req.subtype == user_defined_t then
        id      = req.user_id
        whatsit = copynode(req)
        if not whatsit_ids[id] then
            warning("whatsit id \@percentchar d unregistered; "
                    .. "inconsistencies may arise", id)
        end
    end
    return function () return copynode(whatsit) end, id
end
luatexbase.new_user_whatsit         = new_user_whatsit
local get_user_whatsit_id = function (name, package)
    if not package then
        package = unassociated
    end
    return user_whatsits[package][name]
end
luatexbase.get_user_whatsit_id = get_user_whatsit_id
local get_user_whatsit_name = function (asked)
    local id
    if type(asked) == "number" then
        id = asked
    elseif type(asked) == "function" then
        %- node generator
        local n = asked()
        id = n.user_id
    else %- node
        id = asked.user_id
    end
    local metadata = whatsit_ids[id]
    if not metadata then % unknown
        warning("whatsit id \@percentchar d unregistered;
                   inconsistencies may arise", id)
        return "", ""
    end
    return tableunpack(metadata)
end
luatexbase.get_user_whatsit_name = get_user_whatsit_name
local dump_registered_whatsits = function (asked_package)
    local whatsit_list = { }
    if asked_package then
        local whatsitdata = user_whatsits[asked_package]
        if not whatsitdata then
            error("(no user whatsits registered for package
                      \@percentchar s)", asked_package)
            return
        end
        texiowrite_nl("(user whatsit allocation stats for " ..
                          asked_package)
        for name, id in next, whatsitdata do
            whatsit_list[\string#whatsit_list+1] =
                stringformat("(\@percentchar s:\@percentchar
                     s \@percentchar d)", asked_package, name, id)
        end
    else
        texiowrite_nl("(user whatsit allocation stats")
        texiowrite_nl(stringformat(" ((total \@percentchar d)\string\n
                         (anonymous \@percentchar d))",
            current_whatsit, anonymous_whatsits))
        for package, whatsitdata in next, user_whatsits do
            for name, id in next, whatsitdata do
                whatsit_list[\string#whatsit_list+1] =
                    stringformat("(\@percentchar s:\@percentchar
                        s \@percentchar d)", package, name, id)
            end
        end
    end
    texiowrite_nl" ("
    local first = true
    for i=1, \string#whatsit_list do
        if first then
            first = false
        else % indent
            texiowrite_nl"  "
        end
        texiowrite(whatsit_list[i])
    end
    texiowrite"))\string\n"
end
luatexbase.dump_registered_whatsits = dump_registered_whatsits
luatexbase.newattribute            = new_attribute
luatexbase.newuserwhatsit          = new_user_whatsit
luatexbase.newuserwhatsitid        = new_user_whatsit_id
luatexbase.getuserwhatsitid        = get_user_whatsit_id
luatexbase.getuserwhatsitname      = get_user_whatsit_name
luatexbase.dumpregisteredwhatsits  = dump_registered_whatsits
}
\let\newluatexattribute\newattribute
\let\setluatexattribute\setattribute
\let\unsetluatexattribute\unsetattribute
\let\newluatexcatcodetable\newcatcodetable
\let\setluatexcatcodetable\setcatcodetable
\let\luatexbase@directlua\directlua
\let\luatexbase@ensure@primitive\@gobble
\let\luatexattribute\attribute
\let\luatexattributedef\attributedef
\let\luatexcatcodetable\catcodetable
\let\luatexluaescapestring\luaescapestring
\let\luatexlatelua\latelua
\let\luatexoutputbox\outputbox
\let\luatexscantextokens\scantextokens
\catcode`\@=\emuatcatcode\relax

  \catcode`\@=11
% \else latex
\fi
\catcode`\_=11 % There’s no reason why this shouldn’t be the case.
\catcode`\!=11
%D Nice tool from luat-ini.mkiv. This really helps with those annoying
%D string separators of Lua’s that clutter the source.
% this permits \typefile{self} otherwise nested b/e sep problems
\def\luastringsep{===}
\edef\!!bs{[\luastringsep[}
\edef\!!es{]\luastringsep]}
%D \startdocsection[title=Prerequisites]
%D \startparagraph
%D Package loading and the namespacing issue are commented on in
%D \identifier{enigma.lua}.
%D \stopparagraph
\directlua{
  packagedata = packagedata or { }
  dofile(kpse.find_file\!!bs enigma.lua\!!es)
}

%D \startparagraph
%D First, create somthing like \CONTEXT’s asciimode. We found
%D \texmacro{newluatexcatcodetable} in \identifier{luacode.sty} and it
%D seems to get the job done.
%D \stopparagraph
\newluatexcatcodetable \enigmasetupcatcodes
\bgroup
  \def\escapecatcode      {0}
  \def\begingroupcatcode  {1}
  \def\endgroupcatcode    {2}
  \def\spacecatcode      {10}
  \def\lettercatcode     {11}
  \setluatexcatcodetable\enigmasetupcatcodes {
      \catcode`\^^I = \spacecatcode % tab
      \catcode`\    = \spacecatcode
      \catcode`\{   = \begingroupcatcode
      \catcode`\}   = \endgroupcatcode
      \catcode`\^^L = \lettercatcode    % form feed
      \catcode`\^^M = \lettercatcode    % eol
  }
\egroup
%D \stopdocsection

%D \startdocsection[title=Setups]
%D \startparagraph
%D Once the proper catcodes are in place, the setup macro
%D \texmacro{do_setup_enigma} doesn’t to anything besides passing stuff
%D through to Lua.
%D \stopparagraph
\def\do_setup_enigma#1{%
    \directlua{
      local enigma = packagedata.enigma
      local current_args = enigma.parse_args(\!!bs\detokenize{#1}\!!es)
      enigma.save_raw_args(current_args, \!!bs\current_enigma_id\!!es)
      enigma.new_callback(
        enigma.new_machine(\!!bs\current_enigma_id\!!es),
        \!!bs\current_enigma_id\!!es)
    }%
  \egroup%
}

%D The module setup \texmacro{setupenigma} expects key=value, notation.
%D All the logic is at the Lua end, not much to see here …
\def\setupenigma#1{%
  \bgroup
    \edef\current_enigma_id{#1}
    \luatexcatcodetable \enigmasetupcatcodes
    \do_setup_enigma%
}
%D \stopdocsection

%D \startdocsection[title=Encoding Macros]
%D \startparagraph
%D The environment of \texmacro{begin<enigmaid>} and
%D \texmacro{end<enigmaid>} toggles Enigma encoding.
%D (Regarding environment delimiters we adhere to Knuth’s
%D practice of prefixing with \type{begin}/\type{end}.)
%D \stopparagraph

\def\e!start{begin} %{start}
\def \e!stop{end}   %{stop}
\edef\c!pre_linebreak_filter{pre_linebreak_filter}
\def\do_define_enigma#1{%
  \@EA\gdef\csname \e!start\current_enigma_id\endcsname{%
    \endgraf
    \bgroup%
    \directlua{%
      if packagedata.enigma                         and
         packagedata.enigma.machines[ \!!bs#1\!!es] then
        luatexbase.add_to_callback(
          \!!bs\c!pre_linebreak_filter\!!es,
          packagedata.enigma.callbacks[ \!!bs#1\!!es],
          \!!bs#1\!!es)
      else
        print\!!bs ENIGMA: No machine of that name: #1!\!!es
        os.exit()
      end
    }%
  }%
  \@EA\gdef\csname \e!stop\current_enigma_id\endcsname{%
    \endgraf
    \directlua{
      luatexbase.remove_from_callback(
        \!!bs\c!pre_linebreak_filter\!!es,
        \!!bs#1\!!es)
      packagedata.enigma.machines[ \!!bs#1\!!es]:processed_chars()
    }%
    \egroup%
  }%
}

\def\defineenigma#1{%
  \begingroup
  \let\@EA\expandafter
  \edef\current_enigma_id{#1}%
  \@EA\do_define_enigma\@EA{\current_enigma_id}%
  \endgroup%
}

%D \stopdocsection

\catcode`\_=8  % \popcatcodes
\catcode`\!=12 % reserved according to source2e
\ifenigmaisrunningplain\catcode`\@=12\fi
% vim:ft=plaintex:sw=2:ts=2:expandtab:tw=71
