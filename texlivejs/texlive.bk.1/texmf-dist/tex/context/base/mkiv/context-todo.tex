\setuplayout
  [width=middle,
   height=middle,
   topspace=2cm,
   header=0pt,
   footer=1cm]

\setupbodyfont
  [bookman]

\usemodule
  [punk,abr-02]

\setuphead
  [section]
  [color=ColorThree,
   style=\bfb]

\setupitemgroup
  [itemize] [each]
  [packed] [color=ColorThree,symcolor=ColorThree]

\setupfooter
  [color=ColorThree,
   style=bold]

\setupfootertexts
  [pagenumber]

\setupwhitespace
  [big]

\definefont[PunkFont][demo@punk at 20pt]

% \def\aterm{\sym{?}}
% \def\rterm{\sym{--}}
% \def\dterm{\sym{+}}
% \def\pterm{\sym{!}}
%
% \startitemize[packed]
%     \aterm on the agenda (update, extension, rewrite)
%     \rterm no longer on the agenda, rejected
%     \dterm no longer on the agenda, done
%     \pterm work in progress (so keep an eye on the betas)
% \stopitemize

\definecolor[ColorOne]  [c=0.5,m=0.2,y=0.5,k=0.2]
\definecolor[ColorTwo]  [c=0.5,m=0.5,y=0.1,k=0.1]
\definecolor[ColorThree][c=0.1,m=1.0,y=1.0,k=0.2]

\starttext

\startMPpage
    StartPage ;
        numeric n ; n := 200 ;
        numeric o ; o :=  25 ;

        pair p[] ;
        for i=1 upto n :
            p[i] = (o + uniformdeviate (PaperWidth-2*o), o + uniformdeviate (PaperHeight-2*o)) ;
        endfor ;

        picture d ; d := image (      for i=1 upto n :             draw p[i] ; endfor ) ;
        picture l ; l := image ( draw for i=1 upto n : if i > 1 : -- fi p[i]   endfor ) ;
        picture t ; t := textext("\framed[frame=off,align={middle,lohi},foregroundcolor=ColorThree,foregroundstyle=\PunkFont]{\ConTeXt\endgraf MkIV\endgraf\kern-\strutdepth RoadMap}") ;

        fill Page enlarged 10 withcolor "ColorOne" ;

        draw d withcolor white      withpen pencircle scaled  o      ;
        draw d withcolor "ColorTwo" withpen pencircle scaled (o - 5) ;
        draw l withcolor white      withpen pencircle scaled (o / 5) ;
        draw l withcolor "ColorTwo" withpen pencircle scaled (o /10) ;
        draw thelabel.ulft(t xsized .5PaperWidth,lrcorner Page shifted - (PaperWidth/20,-PaperWidth/40)) ;
    StopPage ;
\stopMPpage


\startsubject[title={Introduction}]

There is not really a long term roadmap for development. One reason is that there is already
a lot available. When we started with \LUATEX, the \CONTEXT\ code was mostly rewritten,
and that process is more of less finished. Of course there is always work left.

This file is not a complete overview of our plans but users can at least get an
idea of what we're working on and what is coming. Feel free to submit
suggestions.

\startlines
Hans Hagen
Hasselt NL
\currentdate
\stoplines

\stopsubject

\startsubject[title={On the agenda for \LUATEX}]

\startitemize
    \startitem
        cleanup passive nodes
    \stopitem
    \startitem
        optimize some callback resolution (more direct)
    \stopitem
    \startitem
        add \type {--output-filename} for \PDF\ filename
    \stopitem
    \startitem
        more consistent \type {lang_variables} and \type {tex_language} in \type
        {texlang.w} and also store the \type {*mins}
    \stopitem
    \startitem
        remove local par in head of line when done with linebreak
    \stopitem
    \startitem
        check why leftskip doesn't always inherit attributes (maybe dir notes don't have them)
        (also check redundant \type {delete_attribute_ref} after \type {new_glue})
    \stopitem
    \startitem
        only return nil when we expect multiple calls in in one line
    \stopitem
    \startitem
        pdf injection in virtual characters (currently qQ interferes with font switch
        flushing) so a pdf page hack is needed
    \stopitem
\stopitemize

\stopsubject

\startsubject[title={On the agenda for \CONTEXT\ \MKIV}]

\startitemize
    \startitem
        play with par callback and properties
    \stopitem
    \startitem
        optimize positions for columnareas and parpos (sequential)
    \stopitem
    \startitem
        add flag to font for math engine
    \stopitem
    \startitem
        get rid of components
    \stopitem
    \startitem
        play with box attributes
    \stopitem
    \startitem
        check consistency between footnotes and running text (main color,
        styles, properties)
    \stopitem
    \startitem
        redo some of the spacing (adapt to improvements in engine)
    \stopitem
    \startitem
        reorganize position data (more subtables)
    \stopitem
    \startitem
        use \type {\matheqnogapstep}, \type {\Ustack}, \type {\mathscriptsmode},
        \type {\mathdisplayskipmode} and other new math primitives
    \stopitem
\stopitemize

\stopsubject

\stoptext
