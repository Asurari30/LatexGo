%% @texfile{
%%   filename  = "amsppt1.tex",
%%   version   = "2.2",
%%   date      = "2001/08/07",
%%   time      = "13:17:37 EDT",
%%   checksum  = "11263 141 693 5811",
%%   filetype  = "AMS-TeX: option",
%%   author    = "American Mathematical Society",
%%   copyright = "Copyright 1991, 2001 American Mathematical Society,
%%                all rights reserved.  Copying of this file is
%%                authorized only if either:
%%                (1) you make absolutely no changes to your copy
%%                    including name; OR
%%                (2) if you do make changes, you first rename it
%%                    to some other name.",
%%   address   = "American Mathematical Society,
%%                Technical Support,
%%                Publications Technical Group,
%%                P. O. Box 6248,
%%                Providence, RI 02940,
%%                USA",
%%   telephone = "401-455-4080 or (in the USA and Canada)
%%                800-321-4AMS (321-4267)",
%%   FAX       = "401-331-3842",
%%   email     = "tech-support@ams.org (Internet)",
%%   codetable = "ISO/ASCII",
%%   keywords  = "amstex, ams-tex, tex",
%%   abstract  = "This is a a conversion file that makes documents
%%                prepared for AMSPPT version 1 compatible with AMSPPT
%%                version 2.0+.  The statement
%%                  \input amsppt1
%%                needs to be added after the \documentstyle line in each
%%                document to be converted.",
%%   docstring = "The checksum field above contains a CRC-16 checksum
%%                as the first value, followed by the equivalent of
%%                the standard UNIX wc (word count) utility output of
%%                lines, words, and characters.  This is produced by
%%                Robert Solovay's checksum utility.",
%% }
%%%%%%%%%%%%%%%%%%%%%%%%%%%%%%%%%%%%%%%%%%%%%%%%%%%%%%%%%%%%%%%%%%%%%%%%
%
%      We start by testing the control sequence \amsppt1.tex so that we
%      can prevent accidentally loading this file twice.
\expandafter\ifx\csname amsppt1.tex\endcsname\relax
 \else\endinput\fi
%      Now we define \amsppt1.tex to handle catcoding of the @ sign.
\expandafter\edef\csname amsppt1.tex\endcsname{%
  \catcode`\noexpand\@=\the\catcode`\@\space}
\catcode`\@=11
%

\let\ad@\address \def\address#1{\ad@#1\endaddress}
\let\da@\date \def\date#1{\da@#1\enddate}
\let\tk@\thanks \def\thanks#1{\tk@#1\endthanks}
\let\kw@\keywords \def\keywords#1{\kw@#1\endkeywords}
\let\su@\subjclass \def\subjclass#1{\su@#1\endsubjclass}
\let\ab@\abstract \long\def\abstract#1{\ab@#1\endabstract}
%
%      If we assume that documents that use amsppt1.tex will not be
%      intended for submission to the AMS, then it's better to revert to
%      the old page dimensions:
%
\parindent10\p@ \hsize26pc \vsize42pc
\captionwidth@\hsize \advance\captionwidth@-1.5in
%
\def\nologo{\let\logo@\empty}
%
%      Backward compatibility for \subheading and \heading are already
%      provided in AMSPPT 2.0 because (unlike the topmatter items) there
%      is no name conflict.
%
%      In AMSPPT version 2.0+ \endRefs is expected at the end of the
%      References section.  We install a test in \enddocument
%      to let us know if \endRefs needs to be added.
%      Just in case someone has amsppt1.tex but not amsppt.sty vers. 2.2
%      we first make the change in the older version of \enddocument
%      and then use a test to prevent the newer version from being
%      read.
\outer\def\enddocument{%
 \runaway@{proclaim}%
%      To test whether we are inside a \Refs section, we check
%      the \sfcode of the period. Just in case someone has
%      used \frenchspacing, we also check the \sfcode of the
%      comma as well.
  \ifnum\sfcode`\.=\@m % if sfcode of period = 1000, we must be in \Refs ...
  \ifnum\sfcode`\,=\@m\else % unless comma also has sfcode 1000, which
                            % means \frenchspacing was used.
   \message{Note: For \styname\space version 2 compatibility, add
       \string\endRefs\space at end of
       \expandafter\eat@\string\\Refs section}%
   \endRefs
 \fi\fi
\ifmonograph@ % do nothing
\else
 \nobreak
 \thetranslator@
 \count@\z@ \loop\ifnum\count@<\addresscount@\advance\count@\@ne
 \csname address\number\count@\endcsname
 \csname email\number\count@\endcsname
 \repeat
\fi
 \vfill\supereject\end}

%      If \add@missing is undefined, let's quit here.
\expandafter\ifx\csname add@missing\endcsname\relax
%      Restore the old catcode of the @ character:
  \csname amsppt1.tex\endcsname
  \endinput\fi

\outer\def\enddocument{\par % \par will do a runaway check for \endref
  \expandafter\ifx\envir@end\endRefs
    \message{Note: For \styname\space version 2 compatibility, add
       \string\endRefs\space at end of
       \expandafter\eat@\string\\Refs section}%
   \endRefs
 \fi
%      Repeat these in case an article (or book chapter!) doesn't
%      have a references section:
  \add@missing\endroster \add@missing\endproclaim \add@missing\enddefinition
  \add@missing\enddemo \add@missing\endremark \add@missing\endexample
%      In a monograph we expect the translator name and author addresses
%      to be handled in the front matter rather than at the end of the
%      individual chapters:
 \ifmonograph@ % do nothing
 \else
%      No break between the References and the final matter.
 \nobreak
 \thetranslator@
%      Print all the \address's, including e-mail addresses if present.
%      If any of the \email's are undefined the \csname will just evaluate
%      to \relax.
 \count@\z@ \loop\ifnum\count@<\addresscount@\advance\count@\@ne
 \csname address\number\count@\endcsname
 \csname email\number\count@\endcsname
 \repeat
\fi
 \vfill\supereject\end}
%
%      Restore the old catcode of the @ character:
\csname amsppt1.tex\endcsname
\endinput
