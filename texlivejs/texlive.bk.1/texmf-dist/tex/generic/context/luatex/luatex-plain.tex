%D \module
%D   [       file=luatex-plain,
%D        version=2009.12.01,
%D          title=\LUATEX\ Macros,
%D       subtitle=Plain Format,
%D         author=Hans Hagen,
%D           date=\currentdate,
%D      copyright={PRAGMA ADE \& \CONTEXT\ Development Team}]

\input plain

\directlua {tex.enableprimitives('', tex.extraprimitives())}

% We assume that pdf is used.

\ifdefined\pdfextension
    \input luatex-pdf \relax
\fi

\outputmode 1

% \outputmode 0 \magnification\magstep5

% We set the page dimensions because otherwise the backend does weird things
% when we have for instance this on a line of its own:
%
%   \hbox to 100cm {\hss wide indeed\hss}
%
% The page dimension calculation is a fuzzy one as there are some compensations
% for the \hoffset and \voffset and such. I remember long discussions and much
% trial and error in figuring this out during pdftex development times. Where
% a dvi driver will project on a papersize (and thereby clip) the pdf backend
% has to deal with the lack of a page concept on tex by some guessing. Normally
% a macro package will set the dimensions to something reasonable anyway.

\pagewidth   8.5truein
\pageheight 11.0truein

% We load some code at runtime:

\everyjob \expandafter {%
    \the\everyjob
    version https://git-lfs.github.com/spec/v1
oid sha256:c67330041f15e40ed25a2e25a95bd39b325ba9e051ce356ddbad006972d762c2
size 2842
%
    version https://git-lfs.github.com/spec/v1
oid sha256:b9c572b36b88c03ddf56b518ba4d85577ce7c4e215ecd58cfcf00ee6575e5065
size 5885
%
    version https://git-lfs.github.com/spec/v1
oid sha256:41ad39e69c8ff6c4dd9c0e7d93b3446445fe86863fa7345c7c96a45112317db4
size 100829
%
    version https://git-lfs.github.com/spec/v1
oid sha256:1f8397ff4f77f8194f5362f7f7aadf511c376af4a5acb7ca98aedd65712207dc
size 480
%
    %D \module
%D   [       file=luatex-mplib,
%D        version=2009.12.01,
%D          title=\LUATEX\ Support Macros,
%D       subtitle=\METAPOST\ to \PDF\ conversion,
%D         author=Taco Hoekwater \& Hans Hagen,
%D      copyright={PRAGMA ADE \& \CONTEXT\ Development Team}]

%D This is the companion to the \LUA\ module \type {supp-mpl.lua}. Further
%D embedding is up to others. A simple example of usage in plain \TEX\ is:
%D
%D \starttyping
%D \pdfoutput=1
%D
%D \input luatex-mplib.tex
%D
%D \setmplibformat{plain}
%D
%D \mplibcode
%D   beginfig(1);
%D     draw fullcircle
%D       scaled 10cm
%D       withcolor red
%D       withpen pencircle xscaled 4mm yscaled 2mm rotated 30 ;
%D   endfig;
%D \endmplibcode
%D
%D \end
%D \stoptyping

\def\setmplibformat#1{\def\mplibformat{#1}}
\def\setmplibmode  #1{\def\mplibmode  {#1}}

\def\setupmplibcatcodes
  {\catcode`\{=12 % could be optional .. not really needed
   \catcode`\}=12 % could be optional .. not really needed
   \catcode`\#=12
   \catcode`\^=12
   \catcode`\~=12
   \catcode`\_=12
   \catcode`\%=12
   \catcode`\&=12
   \catcode`\$=12
   \obeylines}

\def\mplibcode
  {\bgroup
   \setupmplibcatcodes
   \domplibcode}

\long\def\domplibcode#1\endmplibcode
  {\egroup
   \directlua{metapost.process('\mplibformat',[[#1]],'\mplibmode')}}

%D We default to \type {plain} \METAPOST:

\def\mplibformat{plain}
\def\mplibmode  {scaled}

%D We use a dedicated scratchbox:

\ifx\mplibscratchbox\undefined \newbox\mplibscratchbox \fi

%D Now load the needed \LUA\ code.

\directlua{dofile(kpse.find_file('luatex-mplib.lua'))}
% \directlua{dofile(resolvers.findfile('luatex-mplib.lua'))}

%D The following code takes care of encapsulating the literals:

\def\startMPLIBtoPDF#1#2#3#4%
  {\hbox\bgroup
   \xdef\MPllx{#1}\xdef\MPlly{#2}%
   \xdef\MPurx{#3}\xdef\MPury{#4}%
   \xdef\MPwidth{\the\dimexpr#3bp-#1bp\relax}%
   \xdef\MPheight{\the\dimexpr#4bp-#2bp\relax}%
   \parskip0pt%
   \leftskip0pt%
   \parindent0pt%
   \everypar{}%
   \setbox\mplibscratchbox\vbox\bgroup
   \noindent}

\def\stopMPLIBtoPDF
  {\egroup
   \setbox\mplibscratchbox\hbox
     {\hskip-\MPllx bp%
      \raise-\MPlly bp%
      \box\mplibscratchbox}%
   \setbox\mplibscratchbox\vbox to \MPheight
     {\vfill
      \hsize\MPwidth
      \wd\mplibscratchbox0pt%
      \ht\mplibscratchbox0pt%
      \dp\mplibscratchbox0pt%
      \box\mplibscratchbox}%
   \wd\mplibscratchbox\MPwidth
   \ht\mplibscratchbox\MPheight
   \box\mplibscratchbox
   \egroup}

%D The body of picture, except for text items, is taken care of by:

\ifnum\pdfoutput>0
    \let\MPLIBtoPDF\pdfliteral
\else
    \def\MPLIBtoPDF#1{\special{pdf:literal direct #1}} % not ok yet
\fi

%D Text items have a special handler:

\def\MPLIBtextext#1#2#3#4%
  {\begingroup
   \setbox\mplibscratchbox\hbox
     {\font\temp=#1 at #2bp%
      \temp
      #3}%
   \setbox\mplibscratchbox\hbox
     {\raise#4sp%
      \box\mplibscratchbox}%
   \wd\mplibscratchbox0pt%
   \ht\mplibscratchbox0pt%
   \dp\mplibscratchbox0pt%
   \box\mplibscratchbox
   \endgroup}

\def\MPLIBpdftext#1#2%
  {\ifcsname mplib::#1\endcsname
     % already done, forgotten outside convert group
     \message{<reusing mplib: #1>}%
   \else
     \message{<embedding mplib: #1>}%
     \immediate\pdfximage{#1}% we cannot remove the file as it is included last
     \expandafter\edef\csname mplib::#1\endcsname{\the\pdflastximage}%
   \fi
   \setbox\mplibscratchbox\hbox
     {\raise#2sp\hbox{\pdfrefximage\csname mplib::#1\endcsname}}%
   \wd\mplibscratchbox0pt%
   \ht\mplibscratchbox0pt%
   \dp\mplibscratchbox0pt%
   \box\mplibscratchbox}

\endinput
%
    %D \module
%D   [       file=luatex-gadgets,
%D        version=2015.05.12,
%D          title=\LUATEX\ Support Macros,
%D       subtitle=Useful stuff from articles,
%D         author=Hans Hagen,
%D           date=\currentdate,
%D      copyright={PRAGMA ADE \& \CONTEXT\ Development Team}]

\directlua{dofile(resolvers.findfile('luatex-gadgets.lua'))}

% optional removal of marked content
%
% before\marksomething{gone}{\em HERE}\unsomething{gone}after
% before\marksomething{kept}{\em HERE}\unsomething{gone}after
% \marksomething{gone}{\em HERE}\unsomething{gone}last
% \marksomething{kept}{\em HERE}\unsomething{gone}last

\def\setmarksignal  #1{\directlua{gadgets.marking.setsignal(\number#1)}}
\def\marksomething#1#2{{\directlua{gadgets.marking.mark("#1")}{#2}}}
\def\unsomething    #1{\directlua{gadgets.marking.remove("#1")}}

\newattribute\gadgetmarkattribute \setmarksignal\gadgetmarkattribute

\endinput
%
}

% We also patch the version number:

\edef\fmtversion{\fmtversion+luatex}

\dump
