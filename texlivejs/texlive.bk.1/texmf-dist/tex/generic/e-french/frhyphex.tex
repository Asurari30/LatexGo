% frhyphex.tex % French hyphenation exceptions provided with accent macros.
%                                                        last mods: 2005/03/21 
% After Bernard Gaulle's decease this work is now maintained as
% the e-French project by a group of enthusiast users 
% under LPPL copyright as declared in http://www.efrench.org/
%
% Usually \hyphenation doesn't allow accent macros in its argument, just
% only characters, but french.sty and hyconfig.tex allows this. Thus this
% file can be reloaded each time you change your input encoding. The keyboard
% package do it via kbconfig.tex (and \kbencoding macro).
%
\ifx\undefined\@uclclist% is it initex? yes probably
\expandafter\message\else\expandafter\wlog\fi{frhyphex.tex french exceptions (%
% Do NOT make any alterations to this french hyphenation exceptions list! --bg
% This is my official one.    
                                         V3.3) %   <========= VERSION
                                              }%
% But you can have your private one via language.dat and \frhyphex
% look at the documentation...
%
%%%%%%%%%%%%%%%%%%%%%%%%%%%%%%%%%%%%%%%%%%%%%%%%%%%%%%%%%%%%%%%%%%%%%%%%%%%%
% Comme Babel n'accepte pas les lignes blanches et que Braams n'a pas repondu 
% a mon message (1999/12) j'elimine le chargement des exceptions si babel
\ifx\adddialect\undefined% a ete charge dans le format.

\hyphenation{% La liste est volontairement reduite. Les exceptions qui ont
             % ete reconnues sont mises sous forme de motif dans frhyph.tex

% noms propres de renomee mondiale (!) :
% ============
%

GUTenberg % Groupe francophone des Utilisateurs de TeX, association loi 1901.

%
% noms communs dont la division par TeX est << actuellement >> impropre :
% ============ (les noms suivants on fait l'objet d'une etude GUTenberg et
%               n'ont pas ete retenus  dans les fichier des motifs de cesure)
%

extra-conju-gal % proposition de Etienne  Riga 1999/12/23
extra-conju-gale extra-conju-gales %      Riga 1999/12/23
extra-conju-gaux %                        Riga 1999/12/23
extra-cor-po-rel extra-cor-po-rels %      Riga 1999/12/23
extra-cor-po-relle extra-cor-po-relles %  Riga 1999/12/23
fichier fichiers %proposition de Michel Bovani 1999/08/09
manu-scrit manu-scrits %          Denis Roegel 2002/12/06 
% Je n'ai pas reussi a ajouter "Grand' " pour Grand'place par exemple,
% alors que \hyphenation{Grand'} fonctionne bien dans le corps d'un 
% document avec FrenchPro.                B.G. 2005/03/21

%
%%%
% (Les noms suivants feront l'objet - ulterieurement - d'une etude GUTenberg et
%  - eventuellement - d'une modification du fichier des motifs de cesure)
%
% noms techniques d'usage assez courant
% ===============
%

            } % end of exceptions

% Second list of exceptions:
\ifx\undefined\@uclclist\else% The following not used at initex:
\hyphenation{%

            } % end of exceptions 
\fi

\else%
\ifx\undefined\@uclclist% is it initex? yes probably
\expandafter\message\else\expandafter\wlog\fi%
{NOT LOADED when BABEL is active.}%
\fi%
%\endinput%%%%%%%%%%%%%%%%%%%%%%
