% This file loads NTX math fonts by plainTeX macros
%%%%%%%%%%%%%%%%%%%%%%%%%%%%%%%%%%%%%%%%%%%%%%%%%%%
% Petr Olsak, 2012, 2014 (derived from tx-math.tex)

\message{FONT: NTX math fonts - \string\mathchardef's prepared, 14 math families preloaded.}
\let\mathpreloaded=X

% After \input ntx-math
%
% you can use hundreds characters from NTX math fonts
% (see TX Fonts manual or \mathchardefs below).
% By default: - the fonts are loaded at 10/7/5 sizes.
%             - variables are typeset by current text italic, 
%             - digits and \sin, \cos, etc. are typeset by current text rm
%
% You can use the following alphabets:
% \mit ...... mathematical variables
% \rm, \it .. text roman font, text italic 
% \bf, \bi .. bold sans fonts (may be different than text fonts)        
% \cal    ... normal calligraphics
% \script ... script
% \frak   ... fraktur
% \bbchar ... double stroked letters
%
% You can reload all families of math fonts in two shapes:
% \normalmath ... normal shape
% \boldmath ... bold shape at implicit sizes or sizes set by 
% Before reloading the fonts by previous comand you can set the sizes:
% \setmathsizes[text/script/scriptscript]
% Example \setmathsizes[12/8.4/6]\normalmath ... load fonts at given sizes
%
% You can set typesetting of math variables from NTX font, not from current
% text font, by the command: \mitvariables. The \itvariables reverts to the
% default.

\def\normalmath{%
  \loadmathfamily  0 txr      % TX Roman
  \loadmathfamilyx 1 ntxmi ntxmi7 ntxmi5   % TX Math Italic
  \loadmathfamilyx 2 ntxsy ntxsy7 ntxsy5   % TX Standard symbols
  \loadmathfamily  3 ntxex    % TX extra symbols   
  \loadmathfamily  4 ntxsya   % TX symbols from AMSTeX
  \loadmathfamily  5 ntxsyb   % TX symbols from AMSTeX
  \loadmathfamily  6 ntxsyc   % symbols from TX fonts
  \loadmathfamily  7 ntxexa   % TX new extra symbols
  \loadmathfamily  8 ntxmia   % fraktur, upright greek
  \loadmathfamily  9 rsfs10  % script
  \loadmathfamily 10 phvb8z  % sans serif bold
  \loadmathfamily 11 phvbo8z % sans serif bold slanted (for vectors)
  \setmathfamily  12 \tenrm
  \setmathfamily  13 \tenit
  \setmathdimens
}
\def\boldmath{%
  \loadmathfamily  0 txb      % TX Roman
  \loadmathfamilyx 1 ntxbmi ntxbmi7 ntxbmi5  % TX Math Italic
  \loadmathfamilyx 2 ntxbsy ntxbsy7 ntxbsy5  % TX Standard symbols
  \loadmathfamily  3 ntxbex   % TX extra symbols   
  \loadmathfamily  4 ntxbsya  % TX symbols from AMSTeX
  \loadmathfamily  5 ntxbsyb  % TX symbols from AMSTeX
  \loadmathfamily  6 ntxbsyc  % symbols from TX fonts
  \loadmathfamily  7 ntxbexa  % TX new extra symbols
  \loadmathfamily  8 ntxbmia  % fraktur, upright greek
  \loadmathfamily  9 rsfs10  % \bf script is unavailable
  \loadmathfamily 10 phvb8z  % sans serif bold
  \loadmathfamily 11 phvbo8z % sans serif bold slanted (for vectors)
  \setmathfamily  12 \tenbf 
  \setmathfamily  13 \tenbi
  \setmathdimens
}
\count18=13       % families declared by \newfam are 14, 15 only

\let\normalTXmath=\normalmath  \let\boldTXmath=\boldmath

\def\bi{\tenbi \fam\bifam} % in csplain is done \def\bi{\tenbi} only
\def\bbchar{\fam5 }        % double stroked letters
\def\frak{\fam8 }          % fraktur
\def\script{\fam9 }        % more extensive script than \cal
\chardef\bffam 10          % sans serif bold
\chardef\bifam 11          % sans serif bold slanted
\chardef\rmfam 12          % for \rm (can differ from CM Roman)
\chardef\itfam 13          % normal italic
\let\slfam=\itfam \let\ttfam=\rmfam % for raw similarity with plainTeX

\def\corrmsizes{\ptmunit=1\ptunit\relax
     \ifnum\tmp=10 \ptmunit=.83\ptmunit \fi   % wee need to correct
     \ifnum\tmp=11 \ptmunit=.83\ptmunit \fi}  % the sizes o phvb(o)8t

% User can use \corrmsizes if he/she loads new family. The following example
% loads ZapfChancery as \fam 15 with fonts scaled by 1.32 as compared with
% others fonts in math formula:

% \def\zapf {\fam 15 }
% \addto\corrmsizes {\ifnum\tmp=15 \ptmunit=1.32\ptmunit \fi}
% \addto\normalmath {\loadmathfamily 15 pzcmi8z } \normalmath
% \addto\boldmath   {\loadmathfamily 15 pzcmi8z }

% macros:

\ifx\rfontskipat\undefined \input csfontsm \fi

\def\loadmathfamily #1 #2 {\chardef\tmp#1\corrmsizes
  \let\dgsize=\sizemtext    \font\mF=\whichtfm{#2} at\dgsize \textfont#1=\mF
  \let\dgsize=\sizemscript  \font\mF=\whichtfm{#2} at\dgsize \scriptfont#1=\mF
  \let\dgsize=\sizemsscript \font\mF=\whichtfm{#2} at\dgsize \scriptscriptfont#1=\mF
  \let\dgsize=\undefined
}
\def\loadmathfamilyx #1 #2 #3 #4 {\chardef\tmp#1\corrmsizes
  \let\dgsize=\sizemtext    \font\mF=\whichtfm{#2} at\dgsize \textfont#1=\mF
  \let\dgsize=\sizemscript  \font\mF=\whichtfm{#3} at\dgsize \scriptfont#1=\mF
  \let\dgsize=\sizemsscript \font\mF=\whichtfm{#4} at\dgsize \scriptscriptfont#1=\mF
  \let\dgsize=\undefined
}
\def\setmathfamily #1 #2{\let\mF=#2\chardef\tmp#1\corrmsizes
  \let\dgsize=\sizemtext    \letfont#2=#2 at\dgsize \textfont#1=#2%
  \let\dgsize=\sizemscript  \letfont#2=#2 at\dgsize \scriptfont#1=#2%
  \let\dgsize=\sizemsscript \letfont#2=#2 at\dgsize \scriptscriptfont#1=#2%
  \let\dgsize=\undefined \let#2=\mF
}
\def\itvariables{\def\rm{\fam\rmfam \tenrm}%
  \mathcodechanges C:0-9\mathcodechanges D:A-Z\mathcodechanges D:a-z}
\def\mitvariables{\def\rm{\fam0\tenrm}%
  \mathcodechanges 0:0-9\mathcodechanges 1:A-Z\mathcodechanges 1:a-z}

\def\mathcodechanges#1:#2-#3{\edef\tmp{\count0=\the\count0 \count1=\the\count1 }%
   \count0=`#2  \count1=\count0  \advance\count1 by"7#100
   \loop \mathcode\count0=\count1
         \ifnum\count0<`#3 \advance\count0 by1 \advance\count1 by1 \repeat
   \tmp\relax
}
\ifx\whichtfm\undefined \def\whichtfm#1{#1}\fi

\def\setmathdimens{% PlainTeX sets these dimens for 10pt size only:
  \delimitershortfall=0.5\fontdimen6\textfont3 
  \nulldelimiterspace=0.12\fontdimen6\textfont3
  \scriptspace=0.05\fontdimen6\textfont3
  \skewchar\textfont1=127 \skewchar\scriptfont1=127
  \skewchar\scriptscriptfont1=127
  \skewchar\textfont2=48  \skewchar\scriptfont2=48
  \skewchar\scriptscriptfont2=48
  \fontdimen8\scriptfont3 = \fontdimen8\textfont3  
  \fontdimen8\scriptscriptfont3 = \fontdimen8\textfont3
}
\def\setmathsizes[#1/#2/#3]{%
   \def\sizemtext{#1\ptmunit}\def\sizemscript{#2\ptmunit}% 
   \def\sizemsscript{#3\ptmunit}%
}
\ifx\ptuint\undefined  \def\ptunit{pt}\fi
\ifx\ptmunit\undefined \csname newdimen\endcsname\ptmunit\fi \ptmunit=1\ptunit
\ifx\sizemtext\undefined \setmathsizes[10/7/5]\fi

\ifx\tenbi\undefined \font\tenbi=ptmbi8z \relax \fi
\ifx\normalmathloading\relax\else \normalmath \fi  % load families, normal shape
\itvariables  % \rm in math and variables in math by current text font

%% \mathchardef declarations

\def\amsafam{4} \def\amsbfam{5} \def\txsycfam{6} 
\def\txexafam{7} \def\txmiafam{8}

%% AMSA

\mathchardef \boxdot   "2\amsafam 00
\mathchardef \boxplus   "2\amsafam 01
\mathchardef \boxtimes   "2\amsafam 02
\mathchardef \square   "0\amsafam 03
\mathchardef \blacksquare   "0\amsafam 04
\mathchardef \centerdot   "2\amsafam 05
\mathchardef \lozenge   "0\amsafam 06
\mathchardef \blacklozenge   "0\amsafam 07
\mathchardef \circlearrowright   "3\amsafam 08
\mathchardef \circlearrowleft   "3\amsafam 09
\mathchardef \rightleftharpoons   "3\amsafam 0A
\mathchardef \leftrightharpoons   "3\amsafam 0B
\mathchardef \boxminus   "2\amsafam 0C
\mathchardef \Vdash   "3\amsafam 0D
\mathchardef \Vvdash   "3\amsafam 0E
\mathchardef \vDash   "3\amsafam 0F
\mathchardef \twoheadrightarrow   "3\amsafam 10
\mathchardef \twoheadleftarrow   "3\amsafam 11
\mathchardef \leftleftarrows   "3\amsafam 12
\mathchardef \rightrightarrows   "3\amsafam 13
\mathchardef \upuparrows   "3\amsafam 14
\mathchardef \downdownarrows   "3\amsafam 15
\mathchardef \upharpoonright   "3\amsafam 16
\mathchardef \downharpoonright   "3\amsafam 17
\mathchardef \upharpoonleft   "3\amsafam 18
\mathchardef \downharpoonleft   "3\amsafam 19
\mathchardef \rightarrowtail   "3\amsafam 1A
\mathchardef \leftarrowtail   "3\amsafam 1B
\mathchardef \leftrightarrows   "3\amsafam 1C
\mathchardef \rightleftarrows   "3\amsafam 1D
\mathchardef \Lsh   "3\amsafam 1E
\mathchardef \Rsh   "3\amsafam 1F
\mathchardef \rightsquigarrow   "3\amsafam 20
\mathchardef \leftrightsquigarrow   "3\amsafam 21
\mathchardef \looparrowleft   "3\amsafam 22
\mathchardef \looparrowright   "3\amsafam 23
\mathchardef \circeq   "3\amsafam 24
\mathchardef \succsim   "3\amsafam 25
\mathchardef \gtrsim   "3\amsafam 26
\mathchardef \gtrapprox   "3\amsafam 27
\mathchardef \multimap   "3\amsafam 28
\mathchardef \therefore   "3\amsafam 29
\mathchardef \because   "3\amsafam 2A
\mathchardef \doteqdot   "3\amsafam 2B
\mathchardef \triangleq   "3\amsafam 2C
\mathchardef \precsim   "3\amsafam 2D
\mathchardef \lesssim   "3\amsafam 2E
\mathchardef \lessapprox   "3\amsafam 2F
\mathchardef \eqslantless   "3\amsafam 30
\mathchardef \eqslantgtr   "3\amsafam 31
\mathchardef \curlyeqprec   "3\amsafam 32
\mathchardef \curlyeqsucc   "3\amsafam 33
\mathchardef \preccurlyeq   "3\amsafam 34
\mathchardef \leqq   "3\amsafam 35
\mathchardef \leqslant   "3\amsafam 36
\mathchardef \lessgtr   "3\amsafam 37
\mathchardef \backprime   "0\amsafam 38
\mathchardef \risingdotseq   "3\amsafam 3A
\mathchardef \fallingdotseq   "3\amsafam 3B
\mathchardef \succcurlyeq   "3\amsafam 3C
\mathchardef \geqq   "3\amsafam 3D
\mathchardef \geqslant   "3\amsafam 3E
\mathchardef \gtrless   "3\amsafam 3F
\mathchardef \sqsubset   "3\amsafam 40
\mathchardef \sqsupset   "3\amsafam 41
\mathchardef \vartriangleright   "3\amsafam 42
\mathchardef \vartriangleleft   "3\amsafam 43
\mathchardef \trianglerighteq   "3\amsafam 44
\mathchardef \trianglelefteq   "3\amsafam 45
\mathchardef \bigstar   "0\amsafam 46
\mathchardef \between   "3\amsafam 47
\mathchardef \blacktriangledown   "0\amsafam 48
\mathchardef \blacktriangleright   "3\amsafam 49
\mathchardef \blacktriangleleft   "3\amsafam 4A
\mathchardef \vartriangle   "3\amsafam 4D
\mathchardef \blacktriangle   "0\amsafam 4E
\mathchardef \triangledown   "0\amsafam 4F
\mathchardef \eqcirc   "3\amsafam 50
\mathchardef \lesseqgtr   "3\amsafam 51
\mathchardef \gtreqless   "3\amsafam 52
\mathchardef \lesseqqgtr   "3\amsafam 53
\mathchardef \gtreqqless   "3\amsafam 54
\mathchardef \Rrightarrow   "3\amsafam 56
\mathchardef \Lleftarrow   "3\amsafam 57
\mathchardef \veebar   "2\amsafam 59
\mathchardef \barwedge   "2\amsafam 5A
\mathchardef \doublebarwedge   "2\amsafam 5B
\mathchardef \angle   "0\amsafam 5C
\mathchardef \measuredangle   "0\amsafam 5D
\mathchardef \sphericalangle   "0\amsafam 5E
\mathchardef \varpropto   "3\amsafam 5F
\mathchardef \smallsmile   "3\amsafam 60
\mathchardef \smallfrown   "3\amsafam 61
\mathchardef \Subset   "3\amsafam 62
\mathchardef \Supset   "3\amsafam 63
\mathchardef \Cup   "2\amsafam 64
\mathchardef \Cap   "2\amsafam 65
\mathchardef \curlywedge   "2\amsafam 66
\mathchardef \curlyvee   "2\amsafam 67
\mathchardef \leftthreetimes   "2\amsafam 68
\mathchardef \rightthreetimes   "2\amsafam 69
\mathchardef \subseteqq   "3\amsafam 6A
\mathchardef \supseteqq   "3\amsafam 6B
\mathchardef \bumpeq   "3\amsafam 6C
\mathchardef \Bumpeq   "3\amsafam 6D
\mathchardef \lll   "3\amsafam 6E
\mathchardef \ggg   "3\amsafam 6F
\def \ulcorner {\delimiter"4\amsafam 70\amsafam 70 }
\def \urcorner {\delimiter"5\amsafam 71\amsafam 71 }
\mathchardef \circledS   "0\amsafam 73
\mathchardef \pitchfork   "3\amsafam 74
\mathchardef \dotplus   "2\amsafam 75
\mathchardef \backsim   "3\amsafam 76
\mathchardef \backsimeq   "3\amsafam 77
\def \llcorner {\delimiter"4\amsafam 78\amsafam 78 }
\def \lrcorner {\delimiter"5\amsafam 79\amsafam 79 }
\mathchardef \complement   "0\amsafam 7B
\mathchardef \intercal   "2\amsafam 7C
\mathchardef \circledcirc   "2\amsafam 7D
\mathchardef \circledast   "2\amsafam 7E
\mathchardef \circleddash   "2\amsafam 7F
\mathchardef \rhd   "2\amsafam 42
\mathchardef \lhd   "2\amsafam 43
\mathchardef \unrhd   "2\amsafam 44
\mathchardef \unlhd   "2\amsafam 45

   \let\restriction\upharpoonright
   \let\Doteq\doteqdot
   \let\doublecup\Cup
   \let\doublecap\Cap
   \let\llless\lll
   \let\gggtr\ggg
   \let\Box=\square % LaTeX symbol
   \let\Box=\square % LaTeX symbol

%% AMSB

\mathchardef \lvertneqq   "3\amsbfam 00
\mathchardef \gvertneqq   "3\amsbfam 01
\mathchardef \nleq   "3\amsbfam 02
\mathchardef \ngeq   "3\amsbfam 03
\mathchardef \nless   "3\amsbfam 04
\mathchardef \ngtr   "3\amsbfam 05
\mathchardef \nprec   "3\amsbfam 06
\mathchardef \nsucc   "3\amsbfam 07
\mathchardef \lneqq   "3\amsbfam 08
\mathchardef \gneqq   "3\amsbfam 09
\mathchardef \nleqslant   "3\amsbfam 0A
\mathchardef \ngeqslant   "3\amsbfam 0B
\mathchardef \lneq   "3\amsbfam 0C
\mathchardef \gneq   "3\amsbfam 0D
\mathchardef \npreceq   "3\amsbfam 0E
\mathchardef \nsucceq   "3\amsbfam 0F
\mathchardef \precnsim   "3\amsbfam 10
\mathchardef \succnsim   "3\amsbfam 11
\mathchardef \lnsim   "3\amsbfam 12
\mathchardef \gnsim   "3\amsbfam 13
\mathchardef \nleqq   "3\amsbfam 14
\mathchardef \ngeqq   "3\amsbfam 15
\mathchardef \precneqq   "3\amsbfam 16
\mathchardef \succneqq   "3\amsbfam 17
\mathchardef \precnapprox   "3\amsbfam 18
\mathchardef \succnapprox   "3\amsbfam 19
\mathchardef \lnapprox   "3\amsbfam 1A
\mathchardef \gnapprox   "3\amsbfam 1B
\mathchardef \nsim   "3\amsbfam 1C
\mathchardef \ncong   "3\amsbfam 1D
\mathchardef \diagup   "0\amsbfam 1E
\mathchardef \diagdown   "0\amsbfam 1F
\mathchardef \varsubsetneq   "3\amsbfam 20
\mathchardef \varsupsetneq   "3\amsbfam 21
\mathchardef \nsubseteqq   "3\amsbfam 22
\mathchardef \nsupseteqq   "3\amsbfam 23
\mathchardef \subsetneqq   "3\amsbfam 24
\mathchardef \supsetneqq   "3\amsbfam 25
\mathchardef \varsubsetneqq   "3\amsbfam 26
\mathchardef \varsupsetneqq   "3\amsbfam 27
\mathchardef \subsetneq   "3\amsbfam 28
\mathchardef \supsetneq   "3\amsbfam 29
\mathchardef \nsubseteq   "3\amsbfam 2A
\mathchardef \nsupseteq   "3\amsbfam 2B
\mathchardef \nparallel   "3\amsbfam 2C
\mathchardef \nmid   "3\amsbfam 2D
\mathchardef \nshortmid   "3\amsbfam 2E
\mathchardef \nshortparallel   "3\amsbfam 2F
\mathchardef \nvdash   "3\amsbfam 30
\mathchardef \nVdash   "3\amsbfam 31
\mathchardef \nvDash   "3\amsbfam 32
\mathchardef \nVDash   "3\amsbfam 33
\mathchardef \ntrianglerighteq   "3\amsbfam 34
\mathchardef \ntrianglelefteq   "3\amsbfam 35
\mathchardef \ntriangleleft   "3\amsbfam 36
\mathchardef \ntriangleright   "3\amsbfam 37
\mathchardef \nleftarrow   "3\amsbfam 38
\mathchardef \nrightarrow   "3\amsbfam 39
\mathchardef \nLeftarrow   "3\amsbfam 3A
\mathchardef \nRightarrow   "3\amsbfam 3B
\mathchardef \nLeftrightarrow   "3\amsbfam 3C
\mathchardef \nleftrightarrow   "3\amsbfam 3D
\mathchardef \divideontimes   "2\amsbfam 3E
\mathchardef \varnothing   "0\amsbfam 3F
\mathchardef \nexists   "0\amsbfam 40
\mathchardef \Finv   "0\amsbfam 60
\mathchardef \Game   "0\amsbfam 61
\mathchardef \mho   "0\amsbfam 66
\mathchardef \eth   "0\amsbfam 67
\mathchardef \eqsim   "3\amsbfam 68
\mathchardef \beth   "0\amsbfam 69
\mathchardef \gimel   "0\amsbfam 6A
\mathchardef \daleth   "0\amsbfam 6B
\mathchardef \lessdot   "2\amsbfam 6C
\mathchardef \gtrdot   "2\amsbfam 6D
\mathchardef \ltimes   "2\amsbfam 6E
\mathchardef \rtimes   "2\amsbfam 6F
\mathchardef \shortmid   "3\amsbfam 70
\mathchardef \shortparallel   "3\amsbfam 71
\mathchardef \smallsetminus   "2\amsbfam 72
\mathchardef \thicksim   "3\amsbfam 73
\mathchardef \thickapprox   "3\amsbfam 74
\mathchardef \approxeq   "3\amsbfam 75
\mathchardef \precapprox   "3\amsbfam 76
\mathchardef \succapprox   "3\amsbfam 77
\mathchardef \curvearrowleft   "3\amsbfam 78
\mathchardef \curvearrowright   "3\amsbfam 79
\mathchardef \digamma   "0\amsbfam 7A
\mathchardef \varkappa   "0\amsbfam 7B
\mathchardef \Bbbk   "0\amsbfam 7C
\mathchardef \hslash   "0\amsbfam 7D
\mathchardef \hbar   "0\amsbfam 7E
\mathchardef \backepsilon   "3\amsbfam 7F

%% TXC

\mathchardef \mappedfromchar   "3\txsycfam 00
\mathchardef \Mapstochar   "3\txsycfam 01
\mathchardef \Mappedfromchar   "3\txsycfam 02
\mathchardef \mmapstochar   "3\txsycfam 03
\mathchardef \mmappedfromchar   "3\txsycfam 04
\mathchardef \Mmapstochar   "3\txsycfam 05
\mathchardef \Mmappedfromchar   "3\txsycfam 06
\mathchardef \medcirc   "2\txsycfam 07
\mathchardef \medbullet   "2\txsycfam 08
\mathchardef \varparallel   "3\txsycfam 09
\mathchardef \varparallelinv   "3\txsycfam 0A
\mathchardef \nvarparallel   "3\txsycfam 0B
\mathchardef \nvarparallelinv   "3\txsycfam 0C
\mathchardef \colonapprox   "3\txsycfam 0D
\mathchardef \colonsim   "3\txsycfam 0E
\mathchardef \Colonapprox   "3\txsycfam 0F
\mathchardef \Colonsim   "3\txsycfam 10
\mathchardef \doteq   "3\txsycfam 11
\mathchardef \multimapinv   "3\txsycfam 12
\mathchardef \multimapboth   "3\txsycfam 13
\mathchardef \multimapdot   "3\txsycfam 14
\mathchardef \multimapdotinv   "3\txsycfam 15
\mathchardef \multimapdotboth   "3\txsycfam 16
\mathchardef \multimapdotbothA   "3\txsycfam 17
\mathchardef \multimapdotbothB   "3\txsycfam 18
\mathchardef \VDash   "3\txsycfam 19
\mathchardef \VvDash   "3\txsycfam 1A
\mathchardef \cong   "3\txsycfam 1B
\mathchardef \preceqq   "3\txsycfam 1C
\mathchardef \succeqq   "3\txsycfam 1D
\mathchardef \nprecsim   "3\txsycfam 1E
\mathchardef \nsuccsim   "3\txsycfam 1F
\mathchardef \nlesssim   "3\txsycfam 20
\mathchardef \ngtrsim   "3\txsycfam 21
\mathchardef \nlessapprox   "3\txsycfam 22
\mathchardef \ngtrapprox   "3\txsycfam 23
\mathchardef \npreccurlyeq   "3\txsycfam 24
\mathchardef \nsucccurlyeq   "3\txsycfam 25
\mathchardef \ngtrless   "3\txsycfam 26
\mathchardef \nlessgtr   "3\txsycfam 27
\mathchardef \nbumpeq   "3\txsycfam 28
\mathchardef \nBumpeq   "3\txsycfam 29
\mathchardef \nbacksim   "3\txsycfam 2A
\mathchardef \nbacksimeq   "3\txsycfam 2B
\mathchardef \neq   "3\txsycfam 2C
\mathchardef \nasymp   "3\txsycfam 2D
\mathchardef \nequiv   "3\txsycfam 2E
\mathchardef \nsim   "3\txsycfam 2F
\mathchardef \napprox   "3\txsycfam 30
\mathchardef \nsubset   "3\txsycfam 31
\mathchardef \nsupset   "3\txsycfam 32
\mathchardef \nll   "3\txsycfam 33
\mathchardef \ngg   "3\txsycfam 34
\mathchardef \nthickapprox   "3\txsycfam 35
\mathchardef \napproxeq   "3\txsycfam 36
\mathchardef \nprecapprox   "3\txsycfam 37
\mathchardef \nsuccapprox   "3\txsycfam 38
\mathchardef \npreceqq   "3\txsycfam 39
\mathchardef \nsucceqq   "3\txsycfam 3A
\mathchardef \nsimeq   "3\txsycfam 3B
\mathchardef \notin   "3\txsycfam 3C
\mathchardef \notni   "3\txsycfam 3D
\mathchardef \nSubset   "3\txsycfam 3E
\mathchardef \nSupset   "3\txsycfam 3F
\mathchardef \nsqsubseteq   "3\txsycfam 40
\mathchardef \nsqsupseteq   "3\txsycfam 41
\mathchardef \coloneqq   "3\txsycfam 42
\mathchardef \eqqcolon   "3\txsycfam 43
\mathchardef \coloneq   "3\txsycfam 44
\mathchardef \eqcolon   "3\txsycfam 45
\mathchardef \Coloneqq   "3\txsycfam 46
\mathchardef \Eqqcolon   "3\txsycfam 47
\mathchardef \Coloneq   "3\txsycfam 48
\mathchardef \Eqcolon   "3\txsycfam 49
\mathchardef \strictif   "3\txsycfam 4A
\mathchardef \strictfi   "3\txsycfam 4B
\mathchardef \strictiff   "3\txsycfam 4C
\mathchardef \invamp   "2\txsycfam 4D
\def \lbag {\delimiter"4\txsycfam 4E\txexafam 30 }
\def \rbag {\delimiter"5\txsycfam 4F\txexafam 31 }
\mathchardef \Lbag   "4\txsycfam 50
\mathchardef \Rbag   "5\txsycfam 51
\mathchardef \circledless   "3\txsycfam 52
\mathchardef \circledgtr   "3\txsycfam 53
\mathchardef \circledwedge   "2\txsycfam 54
\mathchardef \circledvee   "2\txsycfam 55
\mathchardef \circledbar   "2\txsycfam 56
\mathchardef \circledbslash   "2\txsycfam 57
\mathchardef \lJoin   "3\txsycfam 58
\mathchardef \rJoin   "3\txsycfam 59
\mathchardef \Join   "3\txsycfam 5A
\mathchardef \openJoin   "3\txsycfam 5B
\mathchardef \lrtimes   "3\txsycfam 5C
\mathchardef \opentimes   "3\txsycfam 5D
\mathchardef \Diamond   "0\txsycfam 5E
\mathchardef \Diamondblack   "0\txsycfam 5F
\mathchardef \nplus   "2\txsycfam 60
\mathchardef \nsqsubset   "3\txsycfam 61
\mathchardef \nsqsupset   "3\txsycfam 62
\mathchardef \dashleftarrow   "3\txsycfam 63
\mathchardef \dashrightarrow   "3\txsycfam 64
\mathchardef \dashleftrightarrow   "3\txsycfam 65
\mathchardef \leftsquigarrow   "3\txsycfam 66
\mathchardef \ntwoheadrightarrow   "3\txsycfam 67
\mathchardef \ntwoheadleftarrow   "3\txsycfam 68
\mathchardef \boxast   "2\txsycfam 69
\mathchardef \boxbslash   "2\txsycfam 6A
\mathchardef \boxbar   "2\txsycfam 6B
\mathchardef \boxslash   "2\txsycfam 6C
\mathchardef \Wr   "2\txsycfam 6D
\mathchardef \lambdaslash   "0\txsycfam 6E
\mathchardef \lambdabar   "0\txsycfam 6F
\mathchardef \varclubsuit   "0\txsycfam 70
\mathchardef \vardiamondsuit   "0\txsycfam 71
\mathchardef \varheartsuit   "0\txsycfam 72
\mathchardef \varspadesuit   "0\txsycfam 73
\mathchardef \Nearrow   "3\txsycfam 74
\mathchardef \Searrow   "3\txsycfam 75
\mathchardef \Nwarrow   "3\txsycfam 76
\mathchardef \Swarrow   "3\txsycfam 77
\mathchardef \Top   "0\txsycfam 78
\mathchardef \Bot   "0\txsycfam 79
\mathchardef \Perp   "3\txsycfam 79
\mathchardef \leadstoext   "3\txsycfam 7A
\mathchardef \leadsto   "3\txsycfam 7B
\mathchardef \sqcupplus   "2\txsycfam 7C
\mathchardef \sqcapplus   "2\txsycfam 7D
\def \llbracket {\delimiter"4\txsycfam 7E\txexafam 12 }
\def \rrbracket {\delimiter"5\txsycfam 7F\txexafam 13 }
\mathchardef \boxright   "3\txsycfam 80
\mathchardef \boxleft   "3\txsycfam 81
\mathchardef \boxdotright   "3\txsycfam 82
\mathchardef \boxdotleft   "3\txsycfam 83
\mathchardef \Diamondright   "3\txsycfam 84
\mathchardef \Diamondleft   "3\txsycfam 85
\mathchardef \Diamonddotright   "3\txsycfam 86
\mathchardef \Diamonddotleft   "3\txsycfam 87
\mathchardef \boxRight   "3\txsycfam 88
\mathchardef \boxLeft   "3\txsycfam 89
\mathchardef \boxdotRight   "3\txsycfam 8A
\mathchardef \boxdotLeft   "3\txsycfam 8B
\mathchardef \DiamondRight   "3\txsycfam 8C
\mathchardef \DiamondLeft   "3\txsycfam 8D
\mathchardef \DiamonddotRight   "3\txsycfam 8E
\mathchardef \DiamonddotLeft   "3\txsycfam 8F
\mathchardef \Diamonddot   "0\txsycfam 90
\mathchardef \circleright   "3\txsycfam 91
\mathchardef \circleleft   "3\txsycfam 92
\mathchardef \circleddotright   "3\txsycfam 93
\mathchardef \circleddotleft   "3\txsycfam 94
\mathchardef \multimapbothvert   "3\txsycfam 95
\mathchardef \multimapdotbothvert   "3\txsycfam 96
\mathchardef \multimapdotbothBvert   "3\txsycfam 97
\mathchardef \multimapdotbothAvert   "3\txsycfam 98

   \def\mappedfrom{\leftarrow\mappedfromchar}
   \def\longmappedfrom{\longleftarrow\mappedfromchar}
   \def\Mapsto{\Mapstochar\Rightarrow}
   \def\Longmapsto{\Mapstochar\Longrightarrow}
   \def\Mappedfrom{\Leftarrow\Mappedfromchar}
   \def\Longmappedfrom{\Longleftarrow\Mappedfromchar}
   \def\mmapsto{\mmapstochar\rightarrow}
   \def\longmmapsto{\mmapstochar\longrightarrow}
   \def\mmappedfrom{\leftarrow\mmappedfromchar}
   \def\longmmappedfrom{\longleftarrow\mmappedfromchar}
   \def\Mmapsto{\Mmapstochar\Rightarrow}
   \def\Longmmapsto{\Mmapstochar\Longrightarrow}
   \def\Mmappedfrom{\Leftarrow\Mmappedfromchar}
   \def\Longmmappedfrom{\Longleftarrow\Mmappedfromchar}
   \let\ne=\neq
   \let\notowns=\notni
   \let\lrJoin=\Join
   % \let\bowtie\lrtimes
   \let\dasharrow\dashrightarrow
   \let\circledotright\circleddotright
   \let\circledotleft\circleddotleft


%% TXexa large symbols

\mathchardef \bignplus   "1\txexafam 00
\mathchardef \bigsqcupplus   "1\txexafam 02
\mathchardef \bigsqcapplus   "1\txexafam 04
\mathchardef \bigsqcap   "1\txexafam 06
\mathchardef \oiintop   "1\txexafam 08
\mathchardef \ointctrclockwiseop   "1\txexafam 0A
\mathchardef \ointclockwiseop   "1\txexafam 0C
\mathchardef \sqintop   "1\txexafam 0E
\mathchardef \varprod   "1\txexafam 10
\mathchardef \braacext   "0\txexafam 20
\mathchardef \iintop   "1\txexafam 21
\mathchardef \iiintop   "1\txexafam 23
\mathchardef \iiiintop   "1\txexafam 25
\mathchardef \idotsintop   "1\txexafam 27
\mathchardef \oiiintop   "1\txexafam 29
\mathchardef \varointctrclockwiseop   "1\txexafam 2B
\mathchardef \varointclockwiseop   "1\txexafam 2D
\mathchardef \fintop   "1\txexafam 3E
\mathchardef \oiintctrclockwiseop   "1\txexafam 40
\mathchardef \varoiintclockwiseop   "1\txexafam 42
\mathchardef \oiintclockwiseop   "1\txexafam 48
\mathchardef \varoiintctrclockwiseop   "1\txexafam 4A
\mathchardef \oiiintctrclockwiseop   "1\txexafam 44
\mathchardef \varoiiintclockwiseop   "1\txexafam 46
\mathchardef \oiiintclockwiseop   "1\txexafam 4C
\mathchardef \varoiiintctrclockwiseop   "1\txexafam 4E
\mathchardef \sqiintop   "1\txexafam 50
\mathchardef \sqiiintop   "1\txexafam 52

   \def\oiint{\oiintop\nolimits}
   \def\ointctrclockwise{\ointctrclockwiseop\nolimits}
   \def\ointclockwise{\ointclockwiseop\nolimits}
   \def\iint{\iintop\nolimits}
   \def\iiint{\iiintop\nolimits}
   \def\sqint{\sqintop\nolimits}
   \def\iiiint{\iiiintop\nolimits}
   \def\oiiint{\oiiintop\nolimits}
   \def\idotsint{\idotsintop\nolimits}
   \def\varointctrclockwise{\varointctrclockwiseop\nolimits}
   \def\varointclockwise{\varointclockwiseop\nolimits}
   \def\fint{\fintop\nolimits}
   \def\oiintctrclockwise{\oiintctrclockwiseop\nolimits}
   \def\varoiintclockwise{\varoiintclockwiseop\nolimits}
   \def\oiintclockwise{\oiintclockwiseop\nolimits}
   \def\varoiintctrclockwise{\varoiintctrclockwiseop\nolimits}
   \def\oiiintctrclockwise{\oiiintctrclockwiseop\nolimits}
   \def\varoiiintclockwise{\varoiiintclockwiseop\nolimits}
   \def\oiiintclockwise{\oiiintclockwiseop\nolimits}
   \def\varoiiintctrclockwise{\varoiiintctrclockwiseop\nolimits}
   \def\sqiint{\sqiintop\nolimits}
   \def\sqiiint{\sqiiintop\nolimits}

%% TXMIA

\mathchardef \upalpha   "0\txmiafam 0B
\mathchardef \upbeta   "0\txmiafam 0C
\mathchardef \upgamma   "0\txmiafam 0D
\mathchardef \updelta   "0\txmiafam 0E
\mathchardef \upepsilon   "0\txmiafam 0F
\mathchardef \upzeta   "0\txmiafam 10
\mathchardef \upeta   "0\txmiafam 11
\mathchardef \uptheta   "0\txmiafam 12
\mathchardef \upiota   "0\txmiafam 13
\mathchardef \upkappa   "0\txmiafam 14
\mathchardef \uplambda   "0\txmiafam 15
\mathchardef \upmu   "0\txmiafam 16
\mathchardef \upnu   "0\txmiafam 17
\mathchardef \upxi   "0\txmiafam 18
\mathchardef \uppi   "0\txmiafam 19
\mathchardef \uprho   "0\txmiafam 1A
\mathchardef \upsigma   "0\txmiafam 1B
\mathchardef \uptau   "0\txmiafam 1C
\mathchardef \upupsilon   "0\txmiafam 1D
\mathchardef \upphi   "0\txmiafam 1E
\mathchardef \upchi   "0\txmiafam 1F
\mathchardef \uppsi   "0\txmiafam 20
\mathchardef \upomega   "0\txmiafam 21
\mathchardef \upvarepsilon   "0\txmiafam 22
\mathchardef \upvartheta   "0\txmiafam 23
\mathchardef \upvarpi   "0\txmiafam 24
\mathchardef \upvarrho   "0\txmiafam 25
\mathchardef \upvarsigma   "0\txmiafam 26
\mathchardef \upvarphi   "0\txmiafam 27

%%%  macros

\def\joinrel{\mathrel{\mkern-2.5mu}}  %-3mu in plain TeX

\let\circledplus\oplus
\let\circledminus\ominus
\let\circledtimes\otimes
\let\circledslash\oslash
\let\circleddot\odot

%%% For \underbrace and \overbrace:
%%% use brace extenstion bar (in "20 of txexa) instead of vrule

\def\downbracefill{$\mathsurround0pt
   \braceld\mkern-1mu
   \cleaders\hbox{$\mkern-.5mu\braacext\mkern-.5mu$}\hfill
   \mkern-1mu\braceru\bracelu\mkern-1mu
   \cleaders\hbox{$\mkern-.5mu\braacext\mkern-.5mu$}\hfill
   \mkern-1mu\bracerd$}

\def\upbracefill{$\mathsurround0pt
   \bracelu\mkern-1mu
   \cleaders\hbox{$\mkern-.5mu\braacext\mkern-.5mu$}\hfill
   \mkern-1mu\bracerd\braceld\mkern-1mu
   \cleaders\hbox{$\mkern-.5mu\braacext\mkern-.5mu$}\hfill
   \mkern-1mu\braceru$}

%%% \big, \bigg, etc.

\def\scalebig#1#2{{\left#1\vbox to#2\fontdimen6\textfont3{}%
                   \kern-\nulldelimiterspace\right.}}
\def\big#1{\scalebig{#1}{.85}}  
\def\Big#1{\scalebig{#1}{1.15}} 
\def\bigg#1{\scalebig{#1}{1.45}}
\def\Bigg#1{\scalebig{#1}{1.75}}

%%% \not redefined:
%%%    \not= becomes \ne
%%%    \not< becomes \nless
%%%    \not> becomes \ngtr
%%%    if \notXXX is defined, \not\XXX becomes \notXXX;
%%%    if \nXXX is defined, \not\XXX becomes \nXXX;
%%%    otherwise, \not\XXX is done in the usual way.

\mathchardef \notchar  "3236

\def\not#1{%
  \ifx\TeX\relax \noexpand\not \else % \let\TeX=\relax in \output routine
  \ifx #1=\ne \else
  \ifx #1<\nless \else
  \ifx #1>\ngtr \else
  \begingroup\escapechar=-1\xdef\tmpn{\string#1}\endgroup
  \expandafter\ifx \csname not\tmpn\endcsname \relax
     \expandafter\ifx \csname n\tmpn\endcsname \relax
         \mathrel{\mathord{\notchar}\mathord{#1}}%
     \else \csname n\tmpn\endcsname \fi
  \else \csname not\tmpn\endcsname \fi
  \fi\fi\fi\fi}

\endinput

% end of ntx-math.tex file

